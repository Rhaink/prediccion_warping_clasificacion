\documentclass[12pt]{article}

\usepackage[T1]{fontenc}
\usepackage[utf8]{inputenc}
\usepackage[spanish]{babel}
\usepackage{graphicx}
\usepackage{float}
\usepackage{caption}
\usepackage{subcaption}
\usepackage{booktabs}
\usepackage{amsmath}
\usepackage[margin=1in]{geometry}
\usepackage{hyperref}

\setcounter{secnumdepth}{-1}

\begin{document}

\section{Resumen Clasificación}

\textbf{Clasificación de Radiografías con Eigenfaces + Fisher + KNN}

Este documento contiene el resumen completo.

\section{Visión General}

La metodología implementada es la siguiente:

\begin{verbatim}
Imagen Warped (224x224)
      |
   APLANAR --> Vector de 50,176 dimensiones
      |
   PCA (9300 imágenes de entrenamiento) --> 50 ponderantes (características)
      |
   Z-SCORE (media/sigma del entrenamiento) (media=0, std=1)
      --> Ponderantes estandarizados
      |
   CRITERIO DE FISHER  (J grande = separa bien las clases)
      --> Calcular razón de Fisher (J) por característica
      |
   AMPLIFICAR (característica x J) (ponderar por importancia)
      --> Características amplificadas
      |
   KNN (K vecinos más cercanos) --> Predicción: Enfermo/Normal (2 clases)
\end{verbatim}

\textbf{Nota:} Los parámetros (PCA, media/sigma, Fisher) se calculan SOLO con datos de entrenamiento y se aplican igual a validación y prueba.

\section{Resultados Principales}

\begin{itemize}
  \item \textbf{2 clases (2C-12K)}: - 12,402 imágenes (entrenamiento=9,300, validación=1,857, prueba=1,245)
  \item \textbf{3 clases(3C-6K)}: - 6,725 imágenes (entrenamiento=5,040, validación=1,005, prueba=680)
\end{itemize}

\subsection{El Warping Mejora la Accuracy (Exactitud)}

\begin{table}[H]
  \centering
  \begin{tabular}{lccc}
    \toprule
    Experimento & Dataset Original & Dataset Warped & Mejora \\
    \midrule
    \textbf{2 clases} & 77.75\% & \textbf{81.69\%} & \textbf{+3.94\%} \\
    3 clases & 77.06\% & \textbf{80.44\%} & \textbf{+3.38\%} \\
    \bottomrule
  \end{tabular}
\end{table}

\subsection{Comparación Visual}

\begin{figure}[H]
  \centering
  \includegraphics[width=\linewidth]{../results/figures/phase7_comparison/comparacion_simple.png}
\caption{Comparación de accuracy y mejora por warping en ambos experimentos.}
\end{figure}

\section{Concentración de Varianza}

\subsection{Varianza Explicada por PC1 (Principal Component 1)}

\begin{table}[H]
  \centering
  \begin{tabular}{lcc}
    \toprule
    Dataset & PC1 Varianza & Top 10 Varianza \\
    \midrule
    Warped & \textbf{46.4\%} & 82.0\% \\
    Original & 27.1\% & 72.3\% \\
    \bottomrule
  \end{tabular}
\end{table}

\subsection{Comparación de Varianza: 2 Datasets}

\begin{figure}[H]
  \centering
  \includegraphics[width=\linewidth]{../results/figures/phase3_pca/comparisons/fig_05_comparison_variance.png}
  \caption{Curvas de varianza explicada acumulada. El warping concentra significativamente más varianza en los primeros componentes principales.}
\end{figure}

\subsection{Interpretación}

El warping concentra {\boldmath\textbf{$\sim 70\%$}} más varianza en PC1:

\begin{itemize}
  \item Warped: 46.4\% vs Original: 27.1\%
  \item Ratio: 46.4/27.1 = 1.71\(\times\)
\end{itemize}

\textbf{Significado:} La información está más ``organizada'' después de alinear. Las variaciones importantes se concentran en menos componentes.

\subsection{Reorganización de Información Discriminativa}

\textbf{Experimento reportado:} 2C-12K (12,402 imágenes, entrenamiento=9,300).

\begin{table}[H]
  \centering
  \begin{tabular}{lcc}
    \toprule
    PC & Razón de Fisher (Warped) & Razón de Fisher (Original) \\
    \midrule
    1 & 0.234971 & 0.001269 \\
    2 & 0.071129 & 0.295204 \\
    3 & 0.233157 & 0.007725 \\
    4 & 0.048323 & 0.342791 \\
    5 & 0.001470 & 0.007417 \\
    6 & 0.027052 & 0.008971 \\
    7 & 0.052372 & 0.011364 \\
    8 & 0.000219 & 0.011638 \\
    9 & 0.000362 & 0.008320 \\
    10 & 0.008179 & 0.000762 \\
    \bottomrule
  \end{tabular}
\end{table}

\begin{figure}[H]
  \centering
  \includegraphics[width=\linewidth]{../results/figures/phase5_fisher/comparisons/fisher_2c_12k_50pcs.png}
  \caption{Comparación de Razón de Fisher (PC1-PC50) entre Warped y Original en 2C-12K.}
\end{figure}

\begin{itemize}
  \item En \textbf{WARPED} domina PC1 (muy cercano a PC3)
  \item En \textbf{ORIGINAL} domina PC4, seguido de PC2
\end{itemize}

\textbf{Significado:} En originales, la información discriminativa aparece en componentes más tardíos; el warping la mueve hacia los componentes principales.

\section{Optimización de K}

\subsection{K Óptimo por Experimento (2C-12K y 3C-6K)}

\begin{table}[H]
  \centering
  \begin{tabular}{lcccc}
    \toprule
    Experimento & K Óptimo & Val Accuracy & Prueba Accuracy & Macro F1 \\
    \midrule
    2C-12K Warped & 7 & 83.58\% & 81.69\% & 0.8052 \\
    2C-12K Original & 5 & 78.89\% & 77.75\% & 0.7670 \\
    3C-6K Warped & 21 & 80.00\% & 80.44\% & 0.8106 \\
    3C-6K Original & 21 & 77.71\% & 77.06\% & 0.7809 \\
    \bottomrule
  \end{tabular}
\end{table}

\subsection{Observaciones}

\begin{itemize}
  \item \textbf{3 clases requiere K más alto:} K=21 vs K=5-7 para 2C-12K
  \item \textbf{Val vs Test:} Diferencia pequeña, buena generalización
  \item \textbf{Warped consistentemente mejor en ambos escenarios}
\end{itemize}

\subsection{Gráficos de Optimización de K}

\begin{figure}[H]
  \centering
  \includegraphics[width=\linewidth]{../results/figures/phase6_classification/full_warped/k_optimization.png}
  \caption{Curva de optimización de K para 2C-12K (Warped). K óptimo = 7.}
\end{figure}

\begin{figure}[H]
  \centering
  \includegraphics[width=\linewidth]{../results/figures/phase7_comparison/3class_warped/k_optimization.png}
  \caption{Curva de optimización de K para 3C-6K (Warped). K óptimo}
\end{figure}

\section{Matrices de Confusión}

\begin{figure}[H]
  \centering
  \includegraphics[width=\linewidth]{../results/figures/phase7_comparison/confusion_matrices_2c12k_3c6k.png}
  \caption{Matrices de confusión para 2C-12K y 3C-6K.}
\end{figure}

\section{Desempeño por Clase}

\subsection{2C-12K (Warped, K=7)}

\subsection{Métricas por clase}

\begin{table}[H]
  \centering
  \begin{tabular}{lccc}
    \toprule
    Clase & Precisión & Recall & F1-Score \\
    \midrule
    Enfermo & 0.805 & 0.715 & 0.757 \\
    Normal & 0.823 & 0.885 & 0.853 \\
    \textbf{Macro} & - & - & \textbf{0.805} \\
    \bottomrule
  \end{tabular}
\end{table}

\subsection{3C-6K (Warped, K=21)}

\subsection{Confusiones más frecuentes}

\begin{table}[H]
  \centering
  \begin{tabular}{lc}
    \toprule
    Confusión & Cantidad \\
    \midrule
    Normal $\rightarrow$ COVID & 50 \\
    COVID $\rightarrow$ Normal & 40 \\
    Normal $\rightarrow$ Viral\_Pneumonia & 19 \\
    Viral\_Pneumonia $\rightarrow$ Normal & 14 \\
    COVID $\rightarrow$ Viral\_Pneumonia & 5 \\
    Viral\_Pneumonia $\rightarrow$ COVID & 5 \\
    \bottomrule
  \end{tabular}
\end{table}

\subsection{Métricas por clase}

\begin{table}[H]
  \centering
  \begin{tabular}{lccc}
    \toprule
    Clase & Precisión & Recall & F1-Score \\
    \midrule
    COVID & 0.805 & 0.835 & 0.819 \\
    Normal & 0.790 & 0.746 & 0.767 \\
    Viral\_Pneumonia & 0.830 & 0.860 & 0.845 \\
    \textbf{Macro} & - & - & \textbf{0.811} \\
    \bottomrule
  \end{tabular}
\end{table}

\begin{itemize}
  \item La mayor confusión es entre COVID y Normal (90 casos en total).
\end{itemize}

\section{Conclusión}

\begin{quote}
\textbf{El warping geométrico mejora la clasificación de radiografías al normalizar las variaciones de pose, permitiendo que PCA capture variaciones de patología en lugar de variaciones de posición.}
\end{quote}

\end{document}
