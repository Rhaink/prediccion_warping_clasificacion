\documentclass[12pt]{article}

\usepackage[T1]{fontenc}
\usepackage[utf8]{inputenc}
\usepackage[spanish]{babel}
\usepackage{graphicx}
\usepackage{float}
\usepackage{caption}
\usepackage{subcaption}
\usepackage{booktabs}
\usepackage{amsmath}
\usepackage[margin=1in]{geometry}
\usepackage{hyperref}
\usepackage{tikz}
\usetikzlibrary{arrows.meta, positioning}

\tikzset{
  block/.style={draw, rounded corners, align=center, text width=9cm, minimum height=0.9cm, fill=gray!10},
  line/.style={-Latex, thick},
}

\setcounter{secnumdepth}{-1}

\begin{document}

\section{Metodología Completa: Preprocesamiento + Eigenfaces + Fisher + KNN}

Este documento contiene la metodología completa incluyendo la etapa de preprocesamiento (predicción + warping) que es crucial para entender el proyecto.

\subsection{ETAPA 0: PREPROCESAMIENTO (PREDICCIÓN + WARPING)}

\subsubsection{Ejemplos de Imagenes del Dataset}

\begin{figure}[H]
  \centering
  \includegraphics[width=\linewidth]{../results/figures/phase2_samples/panel_full_01.png}
  \caption{Ejemplos del dataset mostrando imágenes originales (arriba) y warped (abajo).}
\end{figure}

\begin{figure}[H]
  \centering
  \includegraphics[width=\linewidth]{../results/figures/phase2_samples/panel_full_02.png}
  \caption{Más ejemplos del dataset mostrando la transformación warping.}
\end{figure}

\subsection{Paso 1. Predicción de Landmarks}

\textbf{Objetivo:} localizar 15 landmarks por radiografía para el warping.

\textbf{Qué se hace:}
\begin{itemize}
  \item Se normaliza la imagen (CLAHE + resize 224x224).
  \item El modelo predice 15 puntos (30 coords) en [0,1].
\end{itemize}

\textbf{Datos y configuración:}
\begin{itemize}
  \item 957 radiografías con 15 landmarks (dividido en 75\% train/15\% val/10\% test).
  \item ResNet-18 preentrenada + cabeza de regresión.
  \item Loss: Wing Loss en coordenadas normalizadas.
  \item Salida usada para warping: landmarks en 224x224.
\end{itemize}

\subsubsection{Métrica clave (224x224)}

\begin{table}[H]
  \centering
  \begin{tabular}{lccc}
    \toprule
    Configuración & Error medio & Mediana & Std \\
    \midrule
    Ensemble (4) + TTA (Test-Time Augmentation) & \textbf{3.71 px} & 3.17 px & 2.42 px \\
    \bottomrule
  \end{tabular}
\end{table}

\subsubsection{Diagrama de bloques}

\begin{figure}[H]
  \centering
  \begin{tikzpicture}[node distance=0.5cm, every node/.style={font=\small}]
    \node[block] (a) {Radiografía (299x299)};
    \node[block, below=of a] (b) {CLAHE + resize 224x224};
    \node[block, below=of b] (c) {ResNet-18 + head (30 coords)};
    \node[block, below=of c] (d) {TTA (flip + unflip)};
    \node[block, below=of d] (e) {Ensemble (promedio modelos)};
    \node[block, below=of e] (f) {Landmarks (15x2) en 224x224};
    \draw[line] (a) -- (b);
    \draw[line] (b) -- (c);
    \draw[line] (c) -- (d);
    \draw[line] (d) -- (e);
    \draw[line] (e) -- (f);
  \end{tikzpicture}
\end{figure}

\subsubsection{Evidencia visual}

\begin{figure}[H]
  \centering
  \includegraphics[width=\linewidth]{../results/figures/warping_explained/landmarks_overlay_example.png}
  \caption{Ejemplo de landmarks predichos sobre una radiografía real.}
\end{figure}

\begin{figure}[H]
  \centering
  \includegraphics[width=\linewidth]{../../../outputs/thesis_figures/prediction_examples.png}
  \caption{Comparación GT vs predicción (muestra de test).}
\end{figure}

\subsection{Paso 2. Forma Canónica (GPA)}

\textbf{Objetivo:} construir una forma de referencia única para alinear todas las radiografías.

\textbf{Qué se hace:}
\begin{itemize}
  \item Se usan todas las formas (957 radiografías, 15 landmarks) del dataset.
  \item GPA (General Procrustes Analysis) elimina traslación, escala y rotación para alinear las formas.
  \item Se calcula la forma canónica y se escala a 224x224.
\end{itemize}

\textbf{Resultado esperado:}
\begin{itemize}
  \item Forma canónica en 224x224 con 15 landmarks.
\end{itemize}

\subsubsection{Diagrama de bloques}

\begin{figure}[H]
  \centering
  \begin{tikzpicture}[node distance=0.5cm, every node/.style={font=\small}]
    \node[block] (a) {Landmarks (N muestras, 15x2)};
    \node[block, below=of a] (b) {Centrado + escala};
    \node[block, below=of b] (c) {GPA iterativo (alineación)};
    \node[block, below=of c] (d) {Forma canónica (normalizada)};
    \node[block, below=of d] (e) {Escalar a 224x224 (padding)};
    \node[block, below=of e] (f) {Forma canónica final};
    \draw[line] (a) -- (b);
    \draw[line] (b) -- (c);
    \draw[line] (c) -- (d);
    \draw[line] (d) -- (e);
    \draw[line] (e) -- (f);
  \end{tikzpicture}
\end{figure}

\subsubsection{Evidencia visual}

\begin{figure}[H]
  \centering
  \includegraphics[width=\linewidth]{../../../outputs/shape_analysis/figures/methodology/04_efecto_gpa_antes_despues.png}
  \caption{Comparación antes/después; tras centrar y escalar, el cambio principal es la orientación.}
\end{figure}

\subsection{Paso 3. Warping Geométrico (Delaunay + Piecewise Affine)}

\textbf{Objetivo:} alinear cada radiografía a la forma canónica para normalizar la pose.

\textbf{Qué se hace:}
\begin{itemize}
  \item Se toman los landmarks predichos (Paso 1) y la forma canónica (Paso 2).
  \item Se usa la triangulación de Delaunay calculada sobre la forma canónica.
  \item Se aplica una transformación afín por triángulo y se reconstruye la imagen warpeada (224x224).
\end{itemize}

\textbf{Resultado esperado:}
\begin{itemize}
  \item Imágenes warped.
\end{itemize}

\subsubsection{Diagrama de bloques}

\begin{figure}[H]
  \centering
  \begin{tikzpicture}[node distance=0.5cm, every node/.style={font=\small}]
    \node[block] (a) {Landmarks por radiografía + forma canónica};
    \node[block, below=of a] (b) {Delaunay fijo};
    \node[block, below=of b] (c) {Afín por triángulo (piecewise)};
    \node[block, below=of c] (d) {Warping triángulo a triángulo};
    \node[block, below=of d] (e) {Imagen warped 224x224};
    \draw[line] (a) -- (b);
    \draw[line] (b) -- (c);
    \draw[line] (c) -- (d);
    \draw[line] (d) -- (e);
  \end{tikzpicture}
\end{figure}

\subsubsection{Evidencia visual}

\begin{figure}[H]
  \centering
  \includegraphics[width=\linewidth]{../results/figures/warping_explained/warping_step_by_step.png}
  \caption{Proceso completo desde la radiografía original hasta la imagen warpeada.}
\end{figure}

\subsection{Paso 4. PCA / Eigenfaces (Reducción de Dimensionalidad)}

\textbf{Objetivo:} compactar cada radiografía en un conjunto pequeño de componentes principales.

\textbf{Qué se hace:}
\begin{itemize}
  \item Se vectorizan las imágenes warped 224x224 (50,176 píxeles).
  \item Se centra la información y se calcula PCA.
  \item Se obtienen eigenfaces y varianza explicada.
  \item Se proyecta a K componentes (K=50 \(\approx\) 95\% varianza).
\end{itemize}

\textbf{Resultado esperado:}
\begin{itemize}
  \item Representación compacta para cada imagen.
  \item Eigenfaces que capturan la variabilidad dominante.
\end{itemize}

\subsubsection{Diagrama de bloques}

\begin{figure}[H]
  \centering
  \begin{tikzpicture}[node distance=0.5cm, every node/.style={font=\small}]
    \node[block] (a) {Imágenes warped 224x224};
    \node[block, below=of a] (b) {Flatten (50,176)};
    \node[block, below=of b] (c) {Centrar (restar media)};
    \node[block, below=of c] (d) {PCA (covarianza pequeña)};
    \node[block, below=of d] (e) {Eigenfaces + varianza explicada};
    \node[block, below=of e] (f) {Proyección a K componentes (K=50)};
    \draw[line] (a) -- (b);
    \draw[line] (b) -- (c);
    \draw[line] (c) -- (d);
    \draw[line] (d) -- (e);
    \draw[line] (e) -- (f);
  \end{tikzpicture}
\end{figure}

\subsubsection{Evidencia visual}

\begin{figure}[H]
  \centering
  \includegraphics[width=\linewidth]{../results/figures/phase3_pca/full_warped/fig_01_mean_face.png}
  \caption{Radiografía promedio (mean face) del dataset warped.}
\end{figure}

\begin{figure}[H]
  \centering
  \includegraphics[width=\linewidth]{../results/figures/phase3_pca/comparisons/fig_04_comparison_eigenfaces.png}
  \caption{Comparación de eigenfaces entre warped y original.}
\end{figure}

\subsection{Paso 5. Estandarización Z-Score (Características PCA)}

\textbf{Objetivo:} poner todas las componentes en la misma escala antes de Fisher y KNN.

\textbf{Qué se hace:}
\begin{itemize}
  \item Se calculan media y desviación por componente usando solo datos de entrenamiento.
  \item Se estandarizan train/val/test con esos mismos parámetros.
  \item Cada componente queda con media \(\sim\) 0 y std \(\sim\) 1.
\end{itemize}

\textbf{Resultado esperado:}
\begin{itemize}
  \item Características comparables.
\end{itemize}

\subsubsection{Diagrama de bloques}

\begin{figure}[H]
  \centering
  \begin{tikzpicture}[node distance=0.5cm, every node/.style={font=\small}]
    \node[block] (a) {Pesos PCA (K=50)};
    \node[block, below=of a] (b) {Calcular media/sigma (train)};
    \node[block, below=of b] (c) {Z-Score};
    \node[block, below=of c] (d) {Características estandarizadas};
    \draw[line] (a) -- (b);
    \draw[line] (b) -- (c);
    \draw[line] (c) -- (d);
  \end{tikzpicture}
\end{figure}

\subsubsection{Evidencia visual}

\begin{figure}[H]
  \centering
  \includegraphics[width=\linewidth]{../results/figures/phase4_features/full_warped/distribution.png}
  \caption{Distribución de las primeras características tras Z-Score (media \(\sim\) 0, std \(\sim\) 1).}
\end{figure}

\subsection{Paso 6. Criterio de Fisher (Selección Discriminativa)}

\textbf{Objetivo:} medir qué componentes PCA separan mejor las clases.

\textbf{Qué se hace:}
\begin{itemize}
  \item Se calcula la razón de Fisher por componente usando solo entrenamiento.
  \item Se ordenan las componentes por poder discriminativo.
  \item Se amplifican las características multiplicando por la razón de Fisher.
\end{itemize}

\textbf{Resultado esperado:}
\begin{itemize}
  \item Características amplificadas para clasificación.
\end{itemize}

\subsubsection{Diagrama de bloques}

\begin{figure}[H]
  \centering
  \begin{tikzpicture}[node distance=0.5cm, every node/.style={font=\small}]
    \node[block] (a) {Características estandarizadas};
    \node[block, below=of a] (b) {Razón de Fisher por componente};
    \node[block, below=of b] (c) {Ranking de componentes principales};
    \node[block, below=of c] (d) {Amplificación (x J)};
    \node[block, below=of d] (e) {Características amplificadas};
    \draw[line] (a) -- (b);
    \draw[line] (b) -- (c);
    \draw[line] (c) -- (d);
    \draw[line] (d) -- (e);
  \end{tikzpicture}
\end{figure}

\subsubsection{Evidencia visual}

\begin{figure}[H]
  \centering
  \includegraphics[width=\linewidth]{../results/figures/phase5_fisher/comparisons/fisher_2c_12k_50pcs.png}
  \caption{Razón de Fisher por componente (Warped vs Original, 2C-12K).}
\end{figure}

\subsection{Paso 7. Clasificación KNN}

\textbf{Objetivo:} clasificar las radiografías usando las características amplificadas por Fisher.

\textbf{Qué se hace:}
\begin{itemize}
  \item Se usan las características amplificadas (Paso 6).
  \item Se selecciona K óptimo con validación.
  \item Se evalúa en test con accuracy y métricas por clase.
\end{itemize}

\textbf{Resultado esperado:}
\begin{itemize}
  \item Desempeño final.
\end{itemize}

\subsubsection{Diagrama de bloques}

\begin{figure}[H]
  \centering
  \begin{tikzpicture}[node distance=0.5cm, every node/.style={font=\small}]
    \node[block] (a) {Características amplificadas (Fisher)};
    \node[block, below=of a] (b) {Elegir K (validación)};
    \node[block, below=of b] (c) {KNN};
    \node[block, below=of c] (d) {Predicción de clase};
    \draw[line] (a) -- (b);
    \draw[line] (b) -- (c);
    \draw[line] (c) -- (d);
  \end{tikzpicture}
\end{figure}

\subsubsection{Evidencia visual}

\begin{figure}[H]
  \centering
\includegraphics[width=0.9\linewidth]{../results/figures/phase7_comparison/confusion_matrices_2c12k_3c6k.png}
  \caption{Matrices de confusión para 2C-12K y 3C-6K (Warped vs Original).}
\end{figure}

\end{document}
