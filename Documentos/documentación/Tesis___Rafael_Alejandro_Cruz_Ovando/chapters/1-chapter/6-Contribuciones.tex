\section{Contribuciones Principales}
\label{sec:contribuciones}

Las contribuciones fundamentales de esta tesis se centran en el desarrollo y la evaluación de una nueva metodología para el análisis de radiografías de tórax, con un énfasis particular en la normalización robusta de la forma pulmonar como prerrequisito para una mejor detección de patologías:

\begin{enumerate}
\item \textbf{Desarrollo y validación del enfoque MaShDL-CNN Hybrid para la alineación y normalización de la forma pulmonar:} La principal contribución consiste en un novedoso método híbrido que integra Modelos Estadísticos de Forma (SSM) con Redes Neuronales Convolucionales (CNNs) de manera sinérgica. Este sistema utiliza CNNs, entrenadas para analizar la información contenida en parches 2D de la imagen, con el fin de predecir los coeficientes de deformación de un SSM pulmonar. Esta predicción permite una adaptación precisa y automática de la forma pulmonar a la variabilidad geométrica presente en las radiografías de tórax. El desarrollo incluye el diseño y la optimización de la arquitectura CNN específica para la extracción de características relevantes de los parches y la arquitectura de la Red Neuronal Densa (DNN) para la regresión de los coeficientes de forma.
\item \textbf{Implementación y evaluación de un pipeline completo y reproducible:} Se ha implementado un pipeline completo que abarca desde la estimación de pose inicial de la región pulmonar (utilizando ESL), la extracción de parches 2D basada en la pose, el entrenamiento y la inferencia del modelo MaShDL-CNN, la desdiscretización de los parámetros de forma predichos, hasta la reconstrucción final de la forma pulmonar segmentada y su evaluación cuantitativa rigurosa mediante el coeficiente de Dice. Un componente crucial de este pipeline es la generación de máscaras Ground Truth consistentes con la definición del SSM, asegurando una evaluación justa y precisa del método de segmentación.
\item \textbf{Avance en la estimación de parámetros de forma mediante aprendizaje profundo en imágenes médicas:} Se demuestra empíricamente que el uso de CNNs operando sobre parches 2D mejora significativamente la precisión en la estimación de los parámetros de forma del SSM en comparación con enfoques previos que se basaban en características de perfiles de intensidad 1D. Esta mejora aborda las limitaciones de precisión previamente identificadas y permite una modelización más fiel de las variaciones sutiles de la forma pulmonar.
\item \textbf{Cuantificación del impacto de la normalización de forma avanzada en la detección de patologías pulmonares:} Se investigará y demostrará el beneficio de la alineación y normalización de forma obtenida con el método MaShDL-CNN Hybrid en la tarea subsecuente de clasificación de neumonía y COVID-19. El rendimiento se comparará sistemáticamente con sistemas que carecen de esta etapa de normalización explícita o que utilizan métodos de alineación más simples o menos precisos, estableciendo la importancia de una correcta estandarización geométrica.
% \item \textbf{Normalización de contraste (si se mantiene como contribución distintiva):** Una técnica de expansión estadística del histograma para la normalización del contraste de las imágenes de tórax, mejorando la consistencia visual y facilitando el análisis posterior (si esta técnica es una parte significativa y novedosa del preprocesamiento, mantenerla; si no, podría ser solo un paso de preprocesamiento mencionado).
\end{enumerate}

Estas contribuciones, en conjunto, buscan ofrecer un sistema más preciso, robusto y clínicamente relevante para la interpretación de radiografías de tórax, con el potencial de mejorar significativamente el diagnóstico asistido por computadora de enfermedades pulmonares críticas y de alta prevalencia.