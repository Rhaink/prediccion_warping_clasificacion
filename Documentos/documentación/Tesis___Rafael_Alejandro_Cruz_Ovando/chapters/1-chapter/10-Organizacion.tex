\section{Organización de la Tesis}
\label{sec:organizacion_tesis}

La presente tesis se estructura en los siguientes capítulos, diseñados para guiar al lector desde los fundamentos hasta los resultados y conclusiones de la investigación:

El Capítulo 1 introduce el problema de la detección de enfermedades pulmonares en radiografías de tórax, la motivación subyacente a la investigación, el planteamiento del problema específico que se aborda, la justificación de la metodología propuesta, los objetivos generales y específicos que se persiguen y la organización general del documento.

El Capítulo 2 presenta el marco teórico y una revisión de los antecedentes relevantes en el campo. Se exploran los métodos existentes para la detección de patologías pulmonares, las técnicas de segmentación y normalización de la región pulmonar con un énfasis particular en los Modelos Estadísticos de Forma (SSM), la estimación de pose, y los principios del aprendizaje de variedades. Además, se discuten diversos enfoques para la extracción de características, incluyendo el rol emergente de las Redes Neuronales Convolucionales (CNNs), los tipos de clasificadores supervisados comúnmente empleados en el diagnóstico asistido por computadora y las métricas de evaluación estándar utilizadas para valorar el rendimiento de estos sistemas.

El Capítulo 3 detalla exhaustivamente la metodología propuesta en esta tesis: el enfoque híbrido MaShDL-CNN. Se describe en profundidad la construcción del Modelo Estadístico de Forma pulmonar, el proceso de estimación de pose inicial mediante Efficient Subspace Learning (ESL), la estrategia para la generación de datos de parches 2D, la arquitectura específica y el proceso de entrenamiento del modelo MaShDL-CNN diseñado para la predicción de los coeficientes de forma del SSM. Finalmente, se explica cómo se realiza la reconstrucción de la forma pulmonar y los métodos empleados para la evaluación de la precisión de la segmentación resultante.

% El Capítulo 4 describe el diseño experimental y la configuración utilizada para desarrollar, optimizar y validar la metodología propuesta. Se especifican los conjuntos de datos de radiografías de tórax utilizados, detallando sus fuentes y características. Se describe el entorno computacional (hardware y software) y se presentan los diferentes experimentos llevados a cabo para ajustar los hiperparámetros y comparar distintas configuraciones del modelo de alineación MaShDL-CNN. Adicionalmente, se expone el protocolo experimental diseñado para la evaluación de la tarea final de clasificación de neumonía y COVID-19, incluyendo los escenarios de comparación (sin alineación y con métodos de alineación previos).

El Capítulo 4 presenta y analiza críticamente los resultados obtenidos a lo largo de la investigación. En una primera sección, se evalúa el rendimiento de la etapa de alineación y normalización, enfocándose en la precisión de la predicción de los coeficientes de forma y en los resultados cuantitativos de la segmentación pulmonar (e.g., coeficiente de Dice). Se comparan diferentes configuraciones del modelo MaShDL-CNN. En una segunda sección, se presentan los resultados de la detección de enfermedades (neumonía y COVID-19), comparando el rendimiento de los clasificadores entrenados con características extraídas de las regiones pulmonares normalizadas por el método propuesto, contra los escenarios de control.

% Finalmente, el Capítulo 6 extrae las conclusiones principales del trabajo, reflexionando sobre el cumplimiento de los objetivos planteados y la validación de la hipótesis inicial. Se resumen las contribuciones más significativas de la tesis al campo de estudio, se discuten las limitaciones inherentes al estudio realizado y se proponen posibles líneas de investigación futura que podrían derivarse de este trabajo.