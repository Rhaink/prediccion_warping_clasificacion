\section{Justificación}
\label{sec:justificacion}

\subsection{Contexto Global y Necesidad Clínica}

Las radiografías de tórax constituyen el estudio de imagenología más frecuentemente realizado a nivel mundial, representando la primera línea de evaluación para enfermedades pulmonares y cardíacas en servicios de urgencias, unidades de cuidados intensivos y consulta ambulatoria. La localización precisa de estructuras anatómicas clave mediante la identificación de puntos de referencia (\textit{landmarks}) es esencial para el análisis cuantitativo, la toma de decisiones clínicas y el seguimiento longitudinal de pacientes \cite{Tang2019, Sogancioglu2021}.

El proceso tradicional de anotación manual de \textit{landmarks} anatómicos enfrenta limitaciones críticas en el contexto clínico contemporáneo: (1) el tiempo requerido resulta prohibitivo en escenarios de alta demanda; (2) la variabilidad inter e intra-observador afecta la reproducibilidad de mediciones cuantitativas \cite{Payer2016}; (3) la fatiga del observador incrementa errores en sesiones prolongadas de anotación; y (4) el crecimiento exponencial en volumen de estudios radiológicos supera la disponibilidad de especialistas capacitados, particularmente en regiones con recursos limitados. La pandemia de COVID-19 ha evidenciado dramáticamente esta brecha, con incrementos sostenidos en demanda de interpretación de radiografías torácicas que exceden la capacidad de respuesta del personal médico disponible \cite{Jacobi2020}.

\subsection{Estado del Arte y Limitaciones de Enfoques Existentes}

El aprendizaje profundo (\textit{deep learning}) ha transformado el análisis de imágenes médicas en la última década, alcanzando desempeño comparable o superior a especialistas humanos en diversas tareas de clasificación y segmentación \cite{Litjens2017}. Trabajos seminales han demostrado este potencial en dermatología \cite{Esteva2017}, oftalmología \cite{Gulshan2016} y radiología torácica \cite{Wang2020COVID}. Sin embargo, la detección precisa de \textit{landmarks} anatómicos mediante redes neuronales convolucionales presenta desafíos específicos que no se resuelven simplemente mediante el escalamiento de arquitecturas o el incremento de datos de entrenamiento.

Los métodos existentes para detección de \textit{landmarks} en radiografías de tórax exhiben limitaciones significativas: (1) enfoques basados en regresión de mapas de calor (\textit{heatmap regression}) requieren resolución espacial elevada y memoria computacional sustancial, dificultando su despliegue en equipo físico (\textit{hardware}) de consumo \cite{Newell2016}; (2) métodos de regresión coordinada (\textit{coordinate regression}) con funciones de pérdida estándar como el Error Cuadrático Medio (Mean Squared Error, MSE) no alcanzan los niveles de precisión requeridos para excelencia clínica (error < 8.5 píxeles) \cite{Zhang2014}; (3) sistemas que ignoran restricciones geométricas anatómicas producen predicciones anatómicamente implausibles (asimetrías artificiales, violación de relaciones espaciales fundamentales); y (4) la mayoría de trabajos reportan validación en conjuntos de datos (\textit{datasets}) con más de 10,000 imágenes, dejando sin resolver el problema de entrenamiento efectivo con \textit{datasets} médicos de tamaño limitado (típicamente cientos o pocos miles de imágenes) \cite{Ker2018}.

Revisiones sistemáticas de sistemas de inteligencia artificial para COVID-19 han identificado riesgo de sesgo elevado, falta de validación externa y reporte inadecuado de metodología en la mayoría de publicaciones \cite{Roberts2021, Wynants2020}. Estas limitaciones metodológicas subrayan la necesidad de investigación rigurosa que establezca estándares reproducibles para desarrollo y validación de sistemas de análisis automatizado de imágenes médicas.

\subsection{Contribución Científica y Técnica del Trabajo}

Esta tesis aborda las limitaciones identificadas mediante las siguientes contribuciones:

\textbf{1. Función de pérdida geométrica con conocimiento anatómico.} Se desarrolla una función de pérdida multi-componente que integra: (a) Wing Loss \cite{Feng2018} para mejora de precisión mediante amplificación de gradientes en el régimen de errores pequeños (refinando predicciones cercanas al objetivo), (b) Symmetry Loss para imponer simetría bilateral anatómica \cite{Donner2013}, y (c) Distance Preservation Loss para preservar relaciones espaciales críticas entre estructuras \cite{Thaler2021}. Esta integración de conocimiento del dominio médico mediante restricciones geométricas constituye una contribución metodológica que supera el enfoque tradicional de incrementar complejidad arquitectural.

\textbf{2. Estrategia de entrenamiento progresivo en cuatro fases.} Se propone una metodología sistemática que progresa desde congelamiento de columna vertebral de la red (\textit{backbone}) hasta optimización completa con restricciones geométricas incrementales. Esta estrategia permite convergencia estable y mejora progresiva del desempeño, demostrando ser superior al entrenamiento extremo a extremo (\textit{end-to-end}) directo en datasets médicos de tamaño limitado \cite{Yosinski2014, Raghu2019}.

\textbf{3. Validación empírica rigurosa.} El sistema se valida sobre 956 radiografías que incluyen casos de COVID-19, neumonía viral y pacientes normales, con evaluación multi-dimensional mediante métricas estándar (error radial medio) y métricas geométricas especializadas (consistencia bilateral, validez anatómica). El diseño experimental incluye análisis de ablación sistemático que cuantifica la contribución individual de cada componente metodológico, proporcionando evidencia empírica del valor de restricciones geométricas sobre complejidad arquitectural.

\textbf{4. Eficiencia computacional y reproducibilidad.} El sistema alcanza precisión de excelencia clínica con inferencia en menos de 1 segundo por imagen en \textit{hardware} de consumo (GPU de gama media con 8GB VRAM), demostrando viabilidad para despliegue en entornos con recursos limitados. Todo el código, configuraciones experimentales y resultados se documentan exhaustivamente para facilitar reproducción y validación independiente, abordando las deficiencias metodológicas identificadas en revisiones sistemáticas \cite{Roberts2021}.

\subsection{Impacto y Aplicaciones Potenciales}

Los \textit{landmarks} anatómicos detectados automáticamente por el sistema propuesto constituyen la base para una secuencia de procesamiento (en adelante referida como \textit{pipeline}) de análisis completa que permitirá, como trabajo futuro, desarrollar:

\textbf{1. Segmentación automática precisa.} Los 15 \textit{landmarks} pueden inicializar Modelos Activos de Forma (Active Shape Models, ASM) para delineación automatizada de contornos pulmonares con modelado de forma anatómicamente plausible \cite{Cootes1995, Heimann2009}.

\textbf{2. Normalización geométrica robusta.} Las coordenadas de \textit{landmarks} permiten calcular transformaciones geométricas que estandaricen pose, escala y orientación, eliminando variaciones extrínsecas y facilitando análisis cuantitativo reproducible.

\textbf{3. Extracción de ROI normalizadas.} Regiones de interés (Regions of Interest, ROI) estandarizadas geométricamente reducen variabilidad inter-sujeto no relacionada con patología, mejorando la sensibilidad y especificidad de análisis posteriores.

\textbf{4. Sistemas de clasificación de patologías.} Representaciones normalizadas pueden alimentar clasificadores de aprendizaje profundo para detección automática de neumonía, COVID-19 y otras patologías torácicas \cite{Wang2020COVID, Apostolopoulos2020}.

La metodología desarrollada no se limita a radiografías de tórax; es generalizable a otros problemas de localización anatómica en imágenes médicas donde existen restricciones geométricas inherentes (simetría en imágenes cerebrales, proporciones anatómicas en radiografías pediátricas, etc.). Esta generalización amplifica el impacto potencial del trabajo más allá del dominio específico de aplicación.

\subsection{Relevancia en el Contexto de COVID-19 y Salud Pública}

La pandemia de COVID-19 ha incrementado exponencialmente la demanda de herramientas de diagnóstico asistido por computadora para triaje rápido y seguimiento de pacientes \cite{Jacobi2020}. Las manifestaciones radiológicas de COVID-19 (opacidades en vidrio esmerilado, consolidaciones, distribución periférica) requieren evaluación cuantitativa de extensión y distribución que se beneficiaría significativamente de \textit{landmarks} anatómicos localizados automáticamente. El sistema propuesto demuestra robustez ante variabilidad patológica, manteniendo precisión clínicamente útil en casos de COVID-19 y neumonía viral, validando su aplicabilidad en escenarios clínicos reales.

Más allá de COVID-19, las enfermedades respiratorias crónicas (EPOC, fibrosis pulmonar, asma severa) y agudas (neumonía bacteriana, tuberculosis) requieren seguimiento longitudinal mediante radiografías seriadas. Sistemas automatizados de análisis cuantitativo basados en \textit{landmarks} precisos permitirían monitoreo objetivo de progresión de enfermedad y respuesta a tratamiento, mejorando la calidad de atención médica especialmente en entornos con acceso limitado a especialistas.

\subsection{Justificación Metodológica}

La elección de ResNet-18 como arquitectura base se justifica por su balance óptimo entre capacidad de representación (11.7M parámetros) y eficiencia computacional, permitiendo entrenamiento efectivo con conjuntos de datos (\textit{datasets}) de tamaño limitado mediante aprendizaje por transferencia (\textit{transfer learning}) desde ImageNet \cite{He2016, Raghu2019}. Estudios sistemáticos han demostrado que \textit{transfer learning} desde ImageNet beneficia tareas médicas especialmente con menos de 10,000 imágenes, siendo las capas iniciales altamente transferibles entre dominios \cite{Tajbakhsh2016}.

La formulación del problema como regresión coordinada directa (\textit{coordinate regression}) en lugar de regresión de mapas de calor (\textit{heatmap regression}) se justifica por: (1) eficiencia computacional (30 salidas vs. 15 mapas de calor de alta resolución), (2) predicciones con valores de coordenadas continuos (ej. 120.37px) sin necesidad de post-procesamiento de mapas de calor, y (3) facilidad de integración con restricciones geométricas en la función de pérdida \cite{Zhang2014}.

En conclusión, esta investigación se justifica por su contribución metodológica (funciones de pérdida geométricas, entrenamiento progresivo), validación rigurosa (análisis multi-dimensional con datasets multi-categoría), eficiencia computacional (despliegue viable en hardware de consumo), y potencial de impacto en salud pública (base para sistemas de diagnóstico asistido accesibles globalmente). Los resultados establecen una metodología reproducible para integración de conocimiento del dominio en sistemas de aprendizaje profundo médico, principio generalizable más allá del problema específico abordado \cite{Litjens2017}.
