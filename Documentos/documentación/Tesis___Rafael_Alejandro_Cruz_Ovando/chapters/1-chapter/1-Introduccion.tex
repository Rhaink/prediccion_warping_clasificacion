\chapter{Introducción}

El análisis preciso de radiografías de tórax es fundamental en el diagnóstico médico, siendo estas imágenes una herramienta ampliamente utilizada debido a su disponibilidad y bajo costo. Sin embargo, la interpretación manual es un proceso subjetivo, susceptible a errores y variabilidad inter-observador, especialmente en contextos de alta demanda \cite{mansoor2015segmentation}. Para mejorar la objetividad y eficiencia, la investigación en diagnóstico médico asistido por computadora se ha enfocado en desarrollar herramientas automáticas que asistan en el análisis de estas imágenes. Un paso crucial en este proceso es la correcta identificación y delimitación de estructuras anatómicas relevantes, como los campos pulmonares y puntos de referencia clave.

Esta tesis aborda el desarrollo de un sistema automatizado para la detección de puntos de referencia (en adelante referidos como \textit{landmarks}) anatómicos  en radiografías de tórax, componente fundamental para el análisis cuantitativo de imágenes médicas. Se propone una metodología basada en aprendizaje profundo (\textit{deep learning}) que utiliza redes neuronales convolucionales (Convolutional Neural Networks, CNNs) que incorporan conocimiento anatómico del dominio médico mediante restricciones geométricas \cite{Litjens2017, Shen2017}. El sistema predice de manera directa las coordenadas de 15 puntos de referencia anatómicos clave, aprovechando aprendizaje por transferencia (\textit{transfer learning}) desde dominios de imágenes naturales \cite{Raghu2019}. Los resultados experimentales demuestran que el enfoque propuesto alcanza niveles de precisión que cumplen con los estándares internacionales de excelencia clínica establecidos para tareas de localización anatómica \cite{Payer2016}, validado sobre un conjunto de datos que incluye casos de COVID-19, neumonía viral y pacientes saludables.

La detección precisa de \textit{landmarks} anatómicos constituye un componente fundamental para el desarrollo futuro de sistemas completos de diagnóstico asistido por computadora. Los puntos de referencia detectados automáticamente proporcionan una base para posteriores etapas de análisis, incluyendo la segmentación automática de regiones anatómicas, la normalización geométrica de imágenes y la clasificación de patologías torácicas. Esta investigación se enfoca específicamente en la primera etapa: la localización robusta y precisa de \textit{landmarks} anatómicos mediante técnicas de aprendizaje profundo. Las líneas de investigación futuras derivadas de este trabajo incluyen el desarrollo de modelos de segmentación pulmonar, sistemas de normalización espacial y clasificadores de patologías que aprovechen los \textit{landmarks} detectados automáticamente.

Esta tesis se organiza de la siguiente manera: el Capítulo 2 presenta el marco teórico y el estado del arte en detección de \textit{landmarks} anatómicos y aprendizaje profundo aplicado a imágenes médicas; el Capítulo 3 detalla la metodología propuesta, incluyendo la arquitectura de red neuronal, las funciones de pérdida especializadas y las estrategias de entrenamiento; el Capítulo 4 describe el conjunto de datos utilizado, las métricas de evaluación y el protocolo experimental; el Capítulo 5 presenta los resultados obtenidos y su análisis comparativo; finalmente, el Capítulo 6 discute las conclusiones, limitaciones y líneas futuras de investigación.

% \begin{figure}
%     \centering
%     \includegraphics[width=1\linewidth]{0.png}
%     \caption{Diagrama de bloques del sistema propuesto para localización de landmarks y segmentación pulmonar.} % Modificado para reflejar el alcance real
%     \label{fig:enter-label}
% \end{figure}
