\section{Síntesis del Marco Teórico}

El presente capítulo ha desarrollado un marco teórico comprehensivo que fundamenta la detección automática de \textit{landmarks} anatómicos en radiografías de tórax mediante aprendizaje profundo. La estructura del capítulo integra progresivamente: (1) los principios físicos de formación de imágenes radiográficas y la definición anatómica de 15 \textit{landmarks} específicos con propiedades de simetría bilateral (Sección~2.1), (2) los fundamentos matemáticos de redes neuronales convolucionales incluyendo la operación de convolución, retropropagación de gradientes, y algoritmos de optimización (Sección~2.2), (3) las arquitecturas residuales profundas que permiten el entrenamiento efectivo de redes con decenas de capas mediante conexiones de atajo y normalización por lotes (Sección~2.3), (4) el paradigma de aprendizaje por transferencia que aprovecha representaciones pre-aprendidas en ImageNet para mejorar el desempeño en el dominio médico con datos limitados (Sección~2.4), (5) las funciones de pérdida especializadas que incorporan amplificación de gradientes para errores pequeños y restricciones geométricas anatómicas (Sección~2.5), (6) el análisis comparativo de enfoques de regresión de coordenadas versus mapas de calor con justificación técnica para la selección del primero (Sección~2.6), y (7) la revisión exhaustiva del estado del arte con tabla comparativa de 21 trabajos publicados entre 2016 y 2024, identificando gaps específicos en la combinación de funciones de pérdida y restricciones geométricas (Sección~2.7).

La integración conceptual de estos elementos constituye la base metodológica del presente trabajo. Las arquitecturas ResNet-18 presentadas en la Sección~2.3, con 11.7 millones de parámetros distribuidos en bloques residuales básicos con conexiones de atajo, proporcionan un balance óptimo entre capacidad expresiva y eficiencia computacional apropiado para conjuntos de datos médicos de tamaño moderado. El aprendizaje por transferencia (Sección~2.4) permite inicializar estos modelos con pesos pre-entrenados en ImageNet, aprovechando características de bajo y medio nivel (bordes, texturas, estructuras geométricas) que son transferibles al dominio de radiografías de tórax a pesar de la brecha substancial entre imágenes naturales RGB e imágenes médicas de canal único. El ajuste fino (\textit{fine-tuning}) con tasas de aprendizaje diferenciales adapta las capas profundas al dominio médico mientras preserva las representaciones genéricas en capas tempranas. La arquitectura de salida mediante regresión directa de coordenadas (Sección~2.6) predice un vector compacto de 30 valores continuos ($2 \times 15$ \textit{landmarks}) mediante capas completamente conectadas aplicadas sobre \textit{global average pooling}, proporcionando eficiencia de memoria, precisión sub-píxel inherente, y compatibilidad directa con arquitecturas ResNet pre-entrenadas. La función de pérdida compuesta propuesta en la Sección~2.5 integra tres componentes complementarios: \textit{Wing Loss} para amplificación de gradientes en el régimen de errores pequeños clínicamente relevantes ($|x| < 10$ píxeles), restricciones de simetría bilateral que penalizan desviaciones entre los 7 pares de \textit{landmarks} simétricos identificados en la Tabla~2.1.1, y preservación de distancias anatómicas que regulariza las distancias inter-\textit{landmark} hacia valores de referencia consistentes con la anatomía torácica normal. Esta combinación explota directamente el conocimiento anatómico específico establecido en la Sección~2.1 sin incrementar la complejidad arquitectónica, proporcionando supervisión adicional que guía el aprendizaje hacia configuraciones de \textit{landmarks} anatómicamente plausibles.

El análisis del estado del arte presentado en la Sección~2.7 evidencia que, si bien métodos basados en aprendizaje profundo han alcanzado precisión sub-milimétrica en aplicaciones cefalométricas controladas (Oh 2020: 1.18 mm, Ma \& Luo 2021: 1.29 mm) y 3-5 píxeles en radiografías de tórax con mayor variabilidad (Li 2023: 4.22 píxeles, Cheng 2023: 3.78 píxeles), existen gaps significativos en la literatura: ningún trabajo reportado combina simultáneamente \textit{Wing Loss}, restricciones de simetría bilateral, y preservación de distancias anatómicas para detección de \textit{landmarks} en radiografías de tórax. La mayoría de métodos utiliza MSE estándar o una única restricción geométrica (Song 2020: solo simetría, Payer 2019: solo configuración espacial), y los estudios de ablación cuantitativos que descomponen las contribuciones individuales de cada componente de función de pérdida son limitados (solo 11 de 21 trabajos revisados). Adicionalmente, las restricciones de simetría bilateral y preservación de distancias anatómicas son infrautilizadas en radiografías de tórax a pesar de ser propiedades anatómicas fundamentales que pueden explotarse sin requerir anotaciones adicionales. Estos gaps motivan la investigación presentada en este trabajo: la integración de conocimiento anatómico específico del tórax mediante una función de pérdida multi-componente con una arquitectura eficiente de regresión de coordenadas.

Las contribuciones del marco teórico desarrollado en este capítulo son múltiples. Primero, se ha proporcionado una fundamentación matemática rigurosa de cada componente metodológico, incluyendo derivaciones completas de la retropropagación de gradientes (Ecuaciones~2.8-2.12), el optimizador Adam (Ecuaciones~2.16-2.18), los bloques residuales (Ecuaciones~2.19-2.22), la función \textit{Wing Loss} con análisis de gradientes (Ecuaciones~2.24-2.26), las restricciones de simetría y preservación de distancias (Ecuaciones~2.27-2.30), y las formulaciones de regresión de coordenadas versus mapas de calor (Ecuaciones~2.32-2.38). Esta fundamentación establece precisamente qué propiedades matemáticas de cada componente son relevantes para la tarea de detección de \textit{landmarks} y cómo interactúan durante el proceso de optimización. Segundo, se ha presentado un análisis exhaustivo del estado del arte mediante una tabla comparativa de 21 trabajos representativos publicados en \textit{venues} de alto impacto (IEEE Transactions on Medical Imaging, Medical Image Analysis, CVPR/ICCV, MICCAI) entre 2016 y 2024, categorizados según enfoque metodológico, con identificación explícita de tendencias temporales, divergencias metodológicas entre dominios, y gaps específicos en la literatura. Tercero, se ha justificado técnicamente cada decisión de diseño mediante análisis de ventajas, limitaciones, y trade-offs: la selección de ResNet-18 sobre arquitecturas más profundas se justifica por eficiencia de parámetros y menor propensión al sobreajuste en conjuntos de datos moderados; la preferencia por regresión de coordenadas sobre mapas de calor se fundamenta en eficiencia computacional, precisión sub-píxel inherente, y restricciones de hardware; la combinación específica de componentes de función de pérdida se motiva por los gaps identificados en el estado del arte y las propiedades anatómicas específicas del tórax.

El marco teórico establecido proporciona todos los fundamentos conceptuales, matemáticos y contextuales necesarios para proceder a la descripción de la metodología experimental. El Capítulo 3 presenta la implementación concreta de los conceptos teóricos desarrollados en este capítulo: la descripción del conjunto de datos específico utilizado incluyendo procedimientos de adquisición, anotación, y preprocesamiento; la arquitectura de red neuronal implementada con detalles de todas las capas, dimensiones de tensores, y funciones de activación; el protocolo experimental completo incluyendo partición de datos, hiperparámetros de entrenamiento, y estrategias de aumentación de datos; las métricas de evaluación cuantitativas para medir el desempeño de localización; y crucialmente, estudios de ablación sistemáticos que cuantifican el impacto individual de cada componente de la función de pérdida compuesta (\textit{Wing Loss} aislado, \textit{Wing Loss} + simetría, \textit{Wing Loss} + distancias, y la combinación completa) para validar empíricamente las hipótesis establecidas en el marco teórico. El Capítulo 4 presentará los resultados experimentales completos, comparaciones cuantitativas con el estado del arte revisado en la Sección~2.7, visualizaciones de predicciones con análisis de casos exitosos y errores, y discusión de las implicaciones de los hallazgos experimentales en el contexto del marco teórico desarrollado.

En síntesis, el Capítulo 2 ha construido una fundamentación teórica sólida, matemáticamente rigurosa, y contextualizada en el estado del arte contemporáneo, que establece las bases para la metodología experimental que se presenta a continuación. Los conceptos de redes neuronales convolucionales profundas, arquitecturas residuales, aprendizaje por transferencia, funciones de pérdida especializadas con restricciones geométricas, y regresión de coordenadas eficiente han sido desarrollados sistemáticamente con nivel de detalle apropiado para una tesis de maestría en ingeniería electrónica, incluyendo derivaciones matemáticas completas, análisis de propiedades relevantes, y conexiones explícitas entre componentes. El marco teórico está preparado para guiar la implementación metodológica y la interpretación de resultados experimentales que constituyen los capítulos subsecuentes.
