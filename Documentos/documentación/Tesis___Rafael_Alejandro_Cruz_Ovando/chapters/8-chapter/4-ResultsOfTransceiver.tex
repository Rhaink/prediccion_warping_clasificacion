\section{Resultados del modelo comportamental del transceptor}

En la figura \ref{fig:003:007:000} se muestra el diagrama de bloques de un sistema de comunicación digital compuesto por un transmisor y un receptor hecho en CppSim.

\begin{figure}[h!]
    \centering
    \includegraphics[width=16cm]{chapters/3-chapter/figuras/transceptor_propuesto.png}
    \caption{Transceptor completo.}
    \label{fig:003:007:000}
\end{figure}

Transmisor:

\begin{enumerate}
    \item Generador de onda triangular (sub-portadora): Se encarga de generar una señal triangular que actúa como sub-portadora para la modulación.
    \item Oscilador Controlado por Voltaje (VCO): Modula la frecuencia de la sub-portadora en función de la señal de datos digitales.
    \item Amplificador de Potencia (PA): Aumenta la potencia de la señal modulada para su transmisión eficiente a través de la antena.
\end{enumerate}

Receptor:

\begin{enumerate}
    \item Amplificador de Bajo Ruido (LNA): Amplifica la señal recibida desde la antena, minimizando el ruido añadido.
    \item Filtro pasa bajas (primer orden): Elimina el ruido de alta frecuencia para mejorar la calidad de la señal.
    \item Detector de envoltura: Extrae la envolvente de la señal para su posterior procesamiento.
    \item Filtro pasa banda (segundo orden Butterworth): Selecciona la banda de frecuencia deseada, eliminando componentes fuera de la banda.
    \item Comparador: Convierte la señal detectada en datos digitales.
Este sistema utiliza técnicas de modulación y filtrado para transmitir y recibir datos digitales, maximizando la eficiencia y minimizando el ruido en el proceso.
\end{enumerate}

En la \ref{tab:modelado_matematico} se muestran las ecuaciones que describen los bloques del propuesto transceptor, estas ecuaciones son codificadas en lenguaje C para simular su comportamiento.

\begin{table}[h!]
    \centering
    \begin{tabular}{|c|c|}
    \hline
    \textbf{Bloque}                                                                     & \textbf{Modelo matemático} \\ \hline
    \textit{Sub-carrier}                                                      &         $v_{out}(t) = \frac{4A}{T} \left| t - \frac{T}{2} \left( 2 \left\lfloor \frac{t}{T} + \frac{1}{2} \right\rfloor + 1 \right) \right| - A
    \label{eq:004:001}$           \\ \hline
    VCO                                                                       &         $v_{out}(t) = Acos(w(t))$           \\ \hline
    PA                                                                        &         $v_{out}(t)=Av_{in}(t)$           \\ \hline
    LNA                                                                       &                    \\ \hline
    \begin{tabular}[c]{@{}c@{}}Filtro pasa bajas\\ (Primer orden)\end{tabular}   &         $V_{out}(s)=G\left[\frac{1}{s+W_0}\right]V_{in}(s)$           \\ \hline
    \begin{tabular}[c]{@{}c@{}}Filtro pasa bajas\\ (Segundo orden)\end{tabular}  &         $V_{out}(s)=G\left[\frac{W_0^2}{s^2+Q+W_0^2}\right]V_{in}(s)$           \\ \hline
    \begin{tabular}[c]{@{}c@{}}Filtro pasa banda\\ (Segundo orden)\end{tabular} &         $V_{out}(s)=G\left[\frac{Qs}{s^2+Q+W_0^2}\right]V_{in}(s)$           \\ \hline
    Rectificador                                                                 &         $v_{out}(t)=\left|v_{in}(t)\right|$           \\ \hline
    Restador                                                               &         $v_{out}(t)=v_{2}(t)-v_{1}(t)$           \\ \hline
    Comparador                                                                &         $v_{out}=\begin{cases} 
      0 & v_{in}(t)< 0 \\
      1 & v_{in}\geq 0 
   \end{cases}$           \\ \hline
    \end{tabular}
    \caption{Ecuaciones de los bloques del modelo comportamental}
    \label{tab:modelado_matematico}
\end{table}

En la figura \ref{fig:003:007:001}, se tiene dos gráficos que representan una señal digital transmitida y la correspondiente señal recibida en un sistema de comunicación digital.

El primer gráfico muestra la señal digital transmitida. La señal oscila entre dos niveles de voltaje, \SI{0}{\volt} y \SI{1}{\volt}, típicos de un sistema digital binario (0 y 1).

En el segundo gráfico representa la señal recibida por el receptor después de la transmisión. Se observa que la señal recibida tiene el mismo patrón de forma que la señal transmitida en el gráfico superior, lo cual indica que el receptor ha captado la secuencia de bits enviada por el emisor correctamente.
Aunque la forma general y la secuencia de los pulsos se corresponden entre ambos gráficos, es posible que haya ligeras diferencias en el tiempo de las transiciones debido a pequeñas demoras o atenuaciones propias del sistema de transmisión. El retardo del sistema es \SI{2.4291}{\micro\second}.

\begin{figure}[h!]
    \centering
    \includegraphics[width=16cm]{chapters/3-chapter/figuras/senal_transceptor.jpg}
    \caption{Señal transmitida y recibida del transceptor.}
    \label{fig:003:007:001}
\end{figure}

La coherencia entre estos dos gráficos indica que el sistema de comunicación digital está funcionando de manera efectiva. La señal recibida por el receptor refleja la misma información que fue enviada por el emisor.
Esto sugiere que no ha habido una distorsión significativa ni pérdida de información durante la transmisión, lo cual es fundamental en los sistemas digitales para garantizar la integridad de los datos.
La transmisión es consistente, ya que los cambios en la señal de entrada (superior) se ven reflejados de manera similar en la señal de salida (inferior), lo cual confirma que el receptor interpreta correctamente la información enviada.

\section{Conclusiones}
El bloque de transmisión FM-UWB tiene las siguientes ventajas:
\begin{enumerate}
    \item Simplicidad del Diseño del Circuito: Los transmisores FM-UWB suelen tener una arquitectura de circuito relativamente sencilla en comparación con otras tecnologías de transmisión de banda ancha.
    \item Facilidad de Integración con Sistemas Digitales: La tecnología FM-UWB es compatible con sistemas digitales modernos, lo que permite su integración directa en plataformas de comunicación existentes.
\end{enumerate}

En cuanto al receptor, a partir del análisis del proceso de filtrado, extracción de envolvente, resta y comparación de señales, se pueden concluir varios puntos clave. Primero, el uso de filtros pasa banda segmenta la señal en bandas de frecuencia específicas, permitiendo un análisis independiente y evitando interferencias entre componentes. La extracción de la envolvente facilita la captura de variaciones de amplitud en el tiempo, destacando información relevante. La resta entre bandas elimina componentes comunes como el ruido, resaltando diferencias significativas en las señales. El comparador convierte la señal en una representación digital, útil para sistemas que requieren activaciones binarias o detección de eventos. En conjunto, esta estrategia es eficaz para reducir ruido, aislar información, y simplificar el procesamiento de señales en aplicaciones de comunicación y análisis de datos en entornos complejos.

Finalmente se muestra un sistema de comunicación digital robusto, donde la señal de entrada se transmite y se recibe con coherencia.
Las pequeñas diferencias temporales, si existen, no afectan la integridad de los datos, lo que indica que el sistema está diseñado para tolerar ligeras variaciones sin perder la sincronización.
La coherencia en los resultados respalda la correcta operación del sistema, permitiendo una transmisión de datos efectiva y precisa.
En resumen, la consistencia en la forma y secuencia de la señal transmitida y recibida demuestra que el sistema de comunicación digital es fiable y capaz de preservar la integridad de los datos.