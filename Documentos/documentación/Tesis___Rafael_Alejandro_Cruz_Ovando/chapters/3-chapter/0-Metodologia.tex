\section{Metodología segmentación automática}
\label{sec:metodologia_simplified}

Esta sección describe la metodología para la segmentación automática de la región pulmonar en imágenes radiográficas de tórax, combinando Modelos Estadísticos de Forma (SSM) y Modelos Activos de Forma (ASM) con Redes Neuronales Convolucionales (CNN). Se detalla el proceso, comenzando por la representación inicial de forma mediante landmarks etiquetados manualmente, seguido de la densificación con interpolación spline para obtener una forma con detalles finos. Usando esta forma densificada se construye un SSM mediante Alineamiento de Procrustes Generalizado (GPA) y Análisis de Componentes Principales (PCA) para modelar la variabilidad de la forma de los pulmones y obtener una forma estándar, se extraen también perfiles de intensidad para construir un Modelo Estadístico de Apariencia (SAM) que captura la apariencia local, para tener una forma estándar de la apariencia alrededor de los contornos pulmonares. Continuamos con la Estimación de Pose Inicial (ESL) que utiliza clasificadores para predecir la transformación global que nos permitirá conocer el área que delimita a los pulmones en base a nuestro modelo estándar de forma y apariencia. El paso siguiente es extraer parches de imagen alrededor de los landmarks para entrenar una CNN, que aprenderá a predecir parámetros de forma discretizados. Estos parámetros se desdiscretizan y se refinan mediante un ajuste iterativo ASM para obtener un modelo robusto. Finalmente, se generan máscaras de segmentación Ground truth (GT) y se evalúa el rendimiento utilizando el Coeficiente de Dice (DSC).

\begin{figure}[htbp] 
    \centering
    \includegraphics[width=1\linewidth]{Figures/diagrama_bloques_1.png}
    \caption{Diagrama de bloques de la metodología de segmentación automática}
    \label{fig:diagrama_bloques_1}
\end{figure}

\section{Visión general de la metodología}
\label{sec:vision_metodologia_simplified}

\begin{enumerate}
    \item \textbf{Representando la Geometría Inicial (Sección~\ref{subsec:AdquisicionDatos}).}
        Todo comienza con una forma geométrica simplificada de los pulmones, un polígono. En lugar de tratar con cada píxel de la imagen, se identifican unos puntos clave (o "landmarks") que marcan lugares anatómicos alrededor de los contornos pulmonares. Se puede pensar en esto como unir de una manera muy básica los puntos de la forma del pulmón.
        Así se obtiene una representación numérica sencilla de la forma inicial de los pulmones para cada imagen.

    \item \textbf{Añadiendo Detalle a los Contornos (Sección~\ref{sec:densificacion_forma_simplificada}).}
        Los quince puntos iniciales no capturan las curvas suaves y los detalles finos de los pulmones. Para mejorar esto, se usa una técnica matemática (interpolación con splines) que ``dibuja'' curvas suaves a través de nuestros puntos iniciales. Luego, se toman muchas más muestras a lo largo de estas curvas suaves.
        De esta manera se consigue una representación mucho más detallada y suave de los contornos pulmonares, con muchos más puntos.

    \item \textbf{Aprendiendo la ``Esencia'' de la Forma Pulmonar (Sección~\ref{sec:ssm_simplified}).}
        Los pulmones varían mucho entre personas. se desea entender cuál es la ``forma pulmonar típica'' y cuáles son las maneras más comunes en que esta forma puede variar (ej., más ancha, más alargada). Para ello, se analizan muchas formas de pulmones de diferentes pacientes.
        \begin{itemize}
            \item \textit{Alineamiento (GPA, Sección~\ref{sec:gpa_simplified}):} Primero, se ``alinean'' todas estas formas para que estén en la misma posición, orientación y tamaño promedio. De esta manera todas las fotos de pulmones ahora tienen características similares antes de estudiarlas.
            \item \textit{Análisis de Variación (PCA, Sección~\ref{sec:pca_ssm_simplified}):} Luego, se identifican las ``direcciones'' o ``modos'' principales en los que las formas alineadas tienden a diferir. Esto da un ``modelo estadístico de form'' (SSM) que puede describir cualquier forma pulmonar plausible como una combinación de la forma media y estas variaciones.
        \end{itemize}
        Así se adquiere un modelo matemático compacto que conoce cómo es un pulmón ``promedio'' y cómo puede deformarse. Este modelo es clave para asegurar que las segmentaciones finales parezcan pulmones de verdad.

    \item \textbf{Comprendiendo la Apariencia de los Bordes (Sección~\ref{sec:sam_simplified}).}
        Para que el modelo se ajuste bien a una nueva imagen, necesita saber qué buscar. Para cada punto del modelo de forma, se aprende cómo se ve típicamente la imagen (los niveles de gris) en la zona perpendicular al borde del pulmón en ese punto. Esto crea ``modelos de perfil de intensidad''.
        De esta forma se extrae el conocimiento de la apariencia local esperada en los bordes pulmonares, que guiará el ajuste del modelo.

    \item \textbf{Encontrando una Ubicación Global Aproximada (Sección~\ref{sec:esl_simplified}).}
        Antes de ajustar los detalles finos, es necesario una idea general de dónde están los pulmones en una nueva imagen y cuáles son sus orientaciones y tamaños aproximados. Se usan unos ``detectores rápidos'' entrenados para encontrar una ``caja'' que enmarque los pulmones.
        Así se obtiene una estimación inicial de la posición, escala y rotación de los pulmones en la imagen.

    \item \textbf{Predicción Inteligente de la Forma con Redes Neuronales (Secciones~\ref{sec:extraccion_parches_cnn_simplified}, \ref{sec:entrenamiento_cnn_simplified}, y \ref{sec:prediccion_desdiscretizacion_b_simplified}).}
        Con la ubicación global estimada, ahora se desea una predicción más precisa de la forma específica de los pulmones en esa imagen. Se utilizan Redes Neuronales Convolucionales (CNNs).
        \begin{itemize}
            \item \textit{Entrada a la CNN:} Se extraen pequeños ``parches'' (trozos) de la imagen alrededor de donde se espera que estén los puntos del contorno pulmonar (basándose en la forma media y la estimación inicial de posición, escala y rotación.).
            \item \textit{Entrenamiento:} Se entrenan a las CNNs para que, mirando estos parches, aprendan a predecir los parámetros del modelo de forma (SSM) que mejor describen los pulmones en la imagen. Es decir, la CNN aprende qué tan ``estirada'' o ``encogida'' debe estar la forma media en cada una de sus direcciones de variación aprendidas.
            \item \textit{Predicción:} Dada una nueva imagen, la CNN predice estos parámetros de forma.
        \end{itemize}
        De esta manera se consigue una estimación inicial muy buena de la forma específica de los pulmones en la nueva imagen, guiada por el aprendizaje profundo a partir de los datos visuales.

    \item \textbf{Ajuste Fino y Refinamiento del Contorno (Sección~\ref{sec:ajuste_asm_simplified}).}
        La predicción de la CNN da un excelente punto de partida, pero es posible refinarlo aún más. Se usa un método iterativo (Active Shape Model - ASM) que toma la forma predicha y la "mueve" sutilmente. En cada paso:
        \begin{itemize}
            \item Para cada punto del contorno, busca en la imagen cercana (usando los ``modelos de perfil de intensidad'' del Paso 4) la posición que mejor parece un borde pulmonar.
            \item Ajusta la forma global para que se parezca a estos nuevos puntos encontrados, pero asegurándose de que la forma siga siendo ``realista'' según nuestro modelo SSM (Paso 3).
            \item Repite hasta que la forma ya no cambie significativamente.
        \end{itemize}
        Así se adquiere una forma final que se ajusta con precisión a los bordes pulmonares en la imagen, respetando al mismo tiempo las variaciones anatómicas aprendidas.

    \item \textbf{De Puntos a Regiones (Sección~\ref{sec:generacion_mascaras_simplified}).}
        Una vez que se tienen los puntos del contorno final, se conectan para formar polígonos y se ``rellena'' el interior. Esto crea una ``máscara'' binaria donde los píxeles dentro de los pulmones tienen un valor y los de fuera otro.
        De esta forma se extrae la segmentación final: una imagen donde la región pulmonar está claramente delimitada.

    \item \textbf{Midiendo el Éxito (Sección~\ref{sec:evaluacion_simplified}).}
        Para saber qué tan bien funciona el sistema, se comparan las máscaras predichas con máscaras ``perfectas'' (ground truth). Se usan métricas numéricas (como el Coeficiente de Dice) para cuantificar la similitud.
        Así se obtiene una evaluación objetiva del rendimiento de la predicción.
\end{enumerate}

Esta es la visualización general de la metodología. Cada una de estas etapas involucra conceptos y formulaciones matemáticas específicas que serán exploradas en las siguientes secciones. Se espera que esta visión general haya proporcionado un entendimiento claro del flujo de trabajo y la lógica detrás de cada componente de la metodología de segmentación automática de la región pulmonar.

\begin{figure}[htbp] 
    \centering
    \includegraphics[width=1\linewidth]{Figures/diagrama_bloques_2_2.drawio.png}
    \caption{Diagrama de bloques visual de la metodología}
    \label{fig:diagrama_bloques_visual_1}
\end{figure}
