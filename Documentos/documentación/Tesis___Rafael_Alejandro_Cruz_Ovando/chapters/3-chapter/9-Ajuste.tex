
% \section{Ajuste Iterativo del SSM (Active Shape Model - ASM)}
% \label{sec:ajuste_asm}

% Una vez obtenida una estimación inicial de los parámetros de forma $\hat{\vect{b}}$ (Sección~\ref{sec:prediccion_desdiscretizacion_b}) y una pose inicial (e.g., escala $s_{ESL}$, orientación $\theta_{ESL}$, y traslación $\vect{t}_{ESL}$) a partir de la etapa ESL (Sección~\ref{sec:esl_matematica}), se procede a refinar estos parámetros mediante un proceso de ajuste iterativo conocido como Active Shape Model (ASM) \cite{cootes1995active}. El ASM busca la instancia del modelo de forma y la pose que mejor se adaptan a la evidencia presente en la imagen de entrada.

% \begin{figure}[htbp]
%     \centering
%     \includegraphics[width=0.5\textwidth]{Figures/step0_initial_projection_newIdx20_cat1_id103_1.png}
%     \caption{Paso 0: Proyección Inicial del SSM sobre la imagen. Se muestra la forma inicial (generada a partir de $\hat{\vect{b}}$ y la pose ESL) superpuesta en la imagen.}
%     \label{fig:step0_initial_projection}
% \end{figure}

% El proceso de ajuste es iterativo y consta de los siguientes pasos principales en cada iteración $t$:

% \begin{enumerate}
%     \item \textbf{Generación de la Instancia de Forma Actual:}
%     A partir de los parámetros de forma actuales $\vect{b}^{(t)}$, se genera una instancia de forma vectorizada $\vect{s}_{\text{SSM}}(\vect{b}^{(t)})$ en el espacio canónico del SSM utilizando la Ecuación~\eqref{eq:ssm_reconstruction}:
%     \begin{equation}
%     \vect{s}_{\text{SSM}}(\vect{b}^{(t)}) = \overline{\vect{s}} + \mat{P} \vect{b}^{(t)}.
%     \label{eq:asm_shape_instance_vec}
%     \end{equation}
%     Esta forma vectorizada se remodela a una matriz $\mat{S}'_{\text{SSM}}(\vect{b}^{(t)}) \in \R^{\Ktotal \times \dval}$ (donde $\Ktotal$ es el número de landmarks y $\dval$ la dimensionalidad, e.g., 2).
    
%     \begin{figure}[htbp]
%     \centering
%     \begin{subfigure}[b]{0.4\textwidth}
%         \centering
%         \includegraphics[width=\textwidth]{Figures/step1_canonical_shape_newIdx20_cat1_id103_1.png}
%         \caption{Iteración 1.}
%         \label{fig:step1_canonical_iter1}
%     \end{subfigure}
%     \hfill % Espacio entre subfigures
%     \begin{subfigure}[b]{0.4\textwidth}
%         \centering
%         \includegraphics[width=\textwidth]{Figures/step1_canonical_shape_newIdx20_cat1_id103.png}
%         \caption{Iteración 10.}
%         \label{fig:step1_canonical_iter10}
%     \end{subfigure}
%     \caption{Paso 1: Generación de la Instancia de Forma Canónica $\mat{S}'_{\text{SSM}}(\vect{b}^{(t)})$.}
%     \label{fig:step1_canonical_shape_comparison}
%     \end{figure}

%     \item \textbf{Proyección al Espacio de la Imagen:}
%     La instancia de forma canónica $\mat{S}'_{\text{SSM}}(\vect{b}^{(t)})$ se proyecta al espacio de la imagen aplicando la transformación de similitud definida por los parámetros de pose actuales $(s^{(t)}, \theta^{(t)}, \vect{t}^{(t)})$ a cada uno de sus $\Ktotal$ puntos:
%     \begin{equation}
%     (\mat{S}'_{\text{img}})_{i,\cdot} = s^{(t)} (\mat{S}'_{\text{SSM}}(\vect{b}^{(t)}))_{i,\cdot} \mat{R}(\theta^{(t)})^\transpose + (\vect{t}^{(t)})^\transpose, \quad \text{para } i=1,\dots,\Ktotal,
%     \label{eq:asm_projection_to_image}
%     \end{equation}
%     donde $(\mat{X})_{i,\cdot}$ denota la $i$-ésima fila (landmark) de la matriz $\mat{X}$, y $\mat{R}(\theta^{(t)})$ es la matriz de rotación. El resultado es $\mat{S}'_{\text{img}}(\vect{b}^{(t)}, s^{(t)}, \theta^{(t)}, \vect{t}^{(t)})$.

%     \begin{figure}[htbp]
%     \centering
%     \begin{subfigure}[b]{0.48\textwidth}
%         \centering
%         \includegraphics[width=\textwidth]{Figures/step2_projected_shape_newIdx20_cat1_id103_1.png}
%         \caption{Iteración 1.}
%         \label{fig:step2_projected_iter1}
%     \end{subfigure}
%     \hfill
%     \begin{subfigure}[b]{0.48\textwidth}
%         \centering
%         \includegraphics[width=\textwidth]{Figures/step2_projected_shape_newIdx20_cat1_id103.png}
%         \caption{Iteración 10.}
%         \label{fig:step2_projected_iter10}
%     \end{subfigure}
%     \caption{Paso 2: Proyección de la Forma Canónica al Espacio de la Imagen, $\mat{S}'_{\text{img}}(\vect{b}^{(t)}, s^{(t)}, \theta^{(t)}, \vect{t}^{(t)})$.}
%     \label{fig:step2_projected_shape_comparison}
%     \end{figure}

%     \item \textbf{Búsqueda Local de Puntos Óptimos:}
%     Para cada uno de los $\Ktotal$ landmarks $\vect{p}'_{i,\text{img}}$ (la $i$-ésima fila) de la forma proyectada $\mat{S}'_{\text{img}}$, se busca una nueva posición candidata $\vect{p}^*_{i,\text{img}}$ en la imagen que mejor se corresponda con la apariencia local esperada. Esta búsqueda se realiza a lo largo de la normal $\vect{n}_i$ al contorno en el landmark $i$ (calculada a partir de la forma media $\overline{\mat{S}}$ para estabilidad).
%     Se muestrean perfiles de intensidad $\vect{g}_{\text{obs}}$ en puntos a lo largo de $\vect{n}_i$ (en un rango de búsqueda, ej. $\pm L_s$ píxeles) alrededor de $\vect{p}'_{i,\text{img}}$. Cada perfil observado se compara con el modelo de perfil estadístico $(\overline{\vect{g}}_i, \mat{\Sigma}_i)$ del landmark $i$ usando la distancia de Mahalanobis (Ecuación~\eqref{eq:mahalanobis_distance}). El punto candidato que minimiza esta distancia se selecciona como $\vect{p}^*_{i,\text{img}}$. El conjunto de estos $\Ktotal$ puntos óptimos forma la "forma objetivo" $\mat{S}^*_{\text{target,img}}$.

%     \begin{figure}[htbp]
%     \centering
%     \begin{subfigure}[b]{0.48\textwidth}
%         \centering
%         \includegraphics[width=\textwidth]{Figures/step3_local_search_newIdx20_cat1_id103_1.png}
%         \caption{Iteración 1.}
%         \label{fig:step3_localsearch_iter1}
%     \end{subfigure}
%     \hfill
%     \begin{subfigure}[b]{0.48\textwidth}
%         \centering
%         \includegraphics[width=\textwidth]{Figures/step3_local_search_newIdx20_cat1_id103.png}
%         \caption{Iteración 10.}
%         \label{fig:step3_localsearch_iter10}
%     \end{subfigure}
%     \caption{Paso 3: Búsqueda Local de Puntos Óptimos en la imagen, generando la forma objetivo $\mat{S}^*_{\text{target,img}}$.}
%     \label{fig:step3_local_search_comparison}
%     \end{figure}

%     \item \textbf{Actualización de Parámetros de Pose:}
%     Se calcula la transformación de similitud $(s_{\text{new}}, \theta_{\text{new}}, \vect{t}_{\text{new}})$ que mejor alinea la instancia de forma canónica actual $\mat{S}'_{\text{SSM}}(\vect{b}^{(t)})$ con la forma objetivo $\mat{S}^*_{\text{target,img}}$. Esto se realiza resolviendo el problema de Procrustes.
%     Los parámetros de pose se actualizan con un factor de amortiguamiento $\alpha_p \in (0, 1]$:
%     \begin{align}
%     \theta^{(t+1)} &= \theta^{(t)} + \alpha_p (\theta_{\text{new}} - \theta^{(t)}), \\
%     \vect{t}^{(t+1)} &= \vect{t}^{(t)} + \alpha_p (\vect{t}_{\text{new}} - \vect{t}^{(t)}).
%     \end{align}
%     La escala $s$ puede actualizarse de forma similar ($s^{(t+1)} = s^{(t)} + \alpha_s (s_{\text{new}} - s^{(t)})$) o mantenerse fija ($s^{(t+1)} = s^{(t)}$), como se menciona para la variante P2.

%     \begin{figure}[htbp]
%     \centering
%     \begin{subfigure}[b]{0.48\textwidth}
%         \centering
%         \includegraphics[width=\textwidth]{Figures/step4_pose_update_newIdx20_cat1_id103_1.png}
%         \caption{Iteración 1.}
%         \label{fig:step4_poseupdate_iter1}
%     \end{subfigure}
%     \hfill
%     \begin{subfigure}[b]{0.48\textwidth}
%         \centering
%         \includegraphics[width=\textwidth]{Figures/step4_pose_update_newIdx20_cat1_id103.png}
%         \caption{Iteración 10.}
%         \label{fig:step4_poseupdate_iter10}
%     \end{subfigure}
%     \caption{Paso 4: Actualización de Parámetros de Pose $(s^{(t+1)}, \theta^{(t+1)}, \vect{t}^{(t+1)})$ alineando $\mat{S}'_{\text{SSM}}(\vect{b}^{(t)})$ con $\mat{S}^*_{\text{target,img}}$.}
%     \label{fig:step4_pose_update_comparison}
%     \end{figure}

%     \item \textbf{Actualización de Parámetros de Forma:}
%     La forma objetivo $\mat{S}^*_{\text{target,img}}$ se transforma de vuelta al espacio canónico del SSM utilizando la inversa de la transformación de pose \textit{actualizada} $(s^{(t+1)}, \theta^{(t+1)}, \vect{t}^{(t+1)})$. Para cada punto $\vect{p}^*_{i,\text{img}}$ de $\mat{S}^*_{\text{target,img}}$:
%     \begin{equation}
%     (\mat{S}'^*_{\text{target,SSM}})_{i,\cdot} = (s^{(t+1)})^{-1} ((\vect{p}^*_{i,\text{img}})^\transpose - (\vect{t}^{(t+1)})^\transpose) \mat{R}(-\theta^{(t+1)})^\transpose.
%     \label{eq:asm_target_to_ssm_space}
%     \end{equation}
%     Esta forma $\mat{S}'^*_{\text{target,SSM}}$ se vectoriza a $\vect{s}'^*_{\text{target,SSM}}$ y se proyecta sobre la base del SSM para obtener un nuevo vector de parámetros de forma $\vect{b}_{\text{new}}$:
%     \begin{equation}
%     \vect{b}_{\text{new}} = \mat{P}^\transpose (\vect{s}'^*_{\text{target,SSM}} - \overline{\vect{s}}).
%     \label{eq:asm_project_to_b}
%     \end{equation}
%     Los nuevos parámetros de forma se limitan (clamping) a un rango plausible, típicamente $\pm n_{\text{std}}\sigma_k$ para cada componente $b_{k,\text{new}}$ (donde $\sigma_k=\sqrt{\lambda_k}$):
%     \begin{equation}
%     b_k^{(t+1)} = \text{clip}(b_{k,\text{new}}, -n_{\text{std}}\sqrt{\lambda_k}, n_{\text{std}}\sqrt{\lambda_k}).
%     \label{eq:asm_b_clamping}
%     \end{equation}

%     \begin{figure}[htbp]
%     \centering
%     \begin{subfigure}[b]{0.48\textwidth}
%         \centering
%         \includegraphics[width=\textwidth]{Figures/step5_shape_update_newIdx20_cat1_id103_1.png}
%         \caption{Iteración 1.}
%         \label{fig:step5_shapeupdate_iter1}
%     \end{subfigure}
%     \hfill
%     \begin{subfigure}[b]{0.48\textwidth}
%         \centering
%         \includegraphics[width=\textwidth]{Figures/step5_shape_update_newIdx20_cat1_id103.png}
%         \caption{Iteración 10.}
%         \label{fig:step5_shapeupdate_iter10}
%     \end{subfigure}
%     \caption{Paso 5: Actualización de Parámetros de Forma $\vect{b}^{(t+1)}$ proyectando $\mat{S}'^*_{\text{target,SSM}}$ (vectorizada) sobre la base del SSM.}
%     \label{fig:step5_shape_update_comparison}
%     \end{figure}

%     \item \textbf{Comprobación de Convergencia:}
%     El proceso iterativo se detiene si se alcanza un número máximo de iteraciones ($max\_iters$) o si el cambio en la posición de los landmarks (o en los parámetros $\vect{b}$ y de pose) entre iteraciones sucesivas cae por debajo de una tolerancia predefinida $\epsilon$. Una medida común es la norma de la diferencia entre las formas proyectadas: $\|\mat{S}'_{\text{img}} - \mat{S}'_{\text{img,prev}}\|_F < \epsilon$.

%     \begin{figure}[htbp]
%     \centering
%     \begin{subfigure}[b]{0.48\textwidth}
%         \centering
%         \includegraphics[width=\textwidth]{Figures/step6_end_of_iteration_newIdx20_cat1_id103_1.png}
%         \caption{Iteración 1.}
%         \label{fig:step6_enditeration_iter1}
%     \end{subfigure}
%     \hfill
%     \begin{subfigure}[b]{0.48\textwidth}
%         \centering
%         \includegraphics[width=\textwidth]{Figures/step6_end_of_iteration_newIdx20_cat1_id103.png}
%         \caption{Iteración 10.}
%         \label{fig:step6_enditeration_iter10}
%     \end{subfigure}
%     \caption{Paso 6: Estado Final de la Iteración, mostrando la forma $\mat{S}'_{\text{img}}$ ajustada en la imagen después de la actualización de forma y pose.}
%     \label{fig:step6_end_of_iteration_comparison}
%     \end{figure}
% \end{enumerate}

% \begin{algorithm}[htbp]
% \caption{Ajuste Iterativo del SSM (ASM)}
% \label{alg:asm_fitting}
% \begin{algorithmic}[1]
% \State Inicializar parámetros de forma $\vect{b}^{(0)}$ y pose $(s^{(0)}, \theta^{(0)}, \vect{t}^{(0)})$.
% \State $\mat{S}'_{\text{img,prev}} \leftarrow \text{Proyectar}(\text{reshape}(\overline{\vect{s}} + \mat{P}\vect{b}^{(0)}, \Ktotal, \dval), s^{(0)}, \theta^{(0)}, \vect{t}^{(0)})$ \Comment{Proyección inicial}
% \For{$t = 0$ \textbf{to} $max\_iters - 1$}
%     \State $\vect{s}_{\text{SSM}} \leftarrow \overline{\vect{s}} + \mat{P} \vect{b}^{(t)}$
%     \State $\mat{S}'_{\text{SSM}} \leftarrow \text{reshape}(\vect{s}_{\text{SSM}}, \Ktotal, \dval)$ \Comment{Paso 1}
%     \State $\mat{S}'_{\text{img}} \leftarrow \text{Proyectar}(\mat{S}'_{\text{SSM}}, s^{(t)}, \theta^{(t)}, \vect{t}^{(t)})$ \Comment{Paso 2, ver Ecuación~\eqref{eq:asm_projection_to_image}}
%     \State \textbf{Búsqueda Local (Paso 3):}
%     \State Inicializar $\mat{S}^*_{\text{target,img}}$ (matriz $\Ktotal \times \dval$)
%     \For{cada landmark $i = 1, \dots, \Ktotal$}
%         \State $\vect{p}'_{i,\text{img}} \leftarrow i\text{-ésima fila de } \mat{S}'_{\text{img}}$
%         \State Calcular normal estable $\vect{n}_i$ en $\vect{p}'_{i,\text{img}}$ (basada en $\overline{\mat{S}}$).
%         \State Buscar a lo largo de $\vect{n}_i$ para encontrar $\vect{p}^*_{i,\text{img}}$ que minimice $D_M^2(\vect{g}_{\text{obs}}, (\overline{\vect{g}}_i, \mat{\Sigma}_i))$.
%         \State Asignar $\vect{p}^*_{i,\text{img}}$ a la $i$-ésima fila de $\mat{S}^*_{\text{target,img}}$.
%     \EndFor
%     \State \textbf{Actualización de Pose (Paso 4):}
%     \State Calcular $(s_{\text{new}}, \theta_{\text{new}}, \vect{t}_{\text{new}})$ alineando $\mat{S}'_{\text{SSM}}$ con $\mat{S}^*_{\text{target,img}}$ (Procrustes).
%     \State $\theta^{(t+1)} \leftarrow \theta^{(t)} + \alpha_p (\theta_{\text{new}} - \theta^{(t)})$
%     \State $\vect{t}^{(t+1)} \leftarrow \vect{t}^{(t)} + \alpha_p (\vect{t}_{\text{new}} - \vect{t}^{(t)})$
%     \State $s^{(t+1)} \leftarrow s^{(t)}$ \Comment{Escala fija en este ejemplo}
%     \State \textbf{Actualización de Forma (Paso 5):}
%     \State $\mat{S}'^*_{\text{target,SSM}} \leftarrow \text{InversaProyectar}(\mat{S}^*_{\text{target,img}}, s^{(t+1)}, \theta^{(t+1)}, \vect{t}^{(t+1)})$ \Comment{Ver Ecuación~\eqref{eq:asm_target_to_ssm_space}}
%     \State $\vect{s}'^*_{\text{target,SSM}} \leftarrow \text{vec}(\mat{S}'^*_{\text{target,SSM}})$
%     \State $\vect{b}_{\text{new}} \leftarrow \mat{P}^\transpose (\vect{s}'^*_{\text{target,SSM}} - \overline{\vect{s}})$
%     \State $\vect{b}^{(t+1)} \leftarrow \text{clip}(\vect{b}_{\text{new}}, -n_{\text{std}}\vect{\sigma}_b, n_{\text{std}}\vect{\sigma}_b)$ \Comment{Donde $\vect{\sigma}_b = (\sqrt{\lambda_0}, \dots, \sqrt{\lambda_{m-1}})^\transpose$}
%     \State \textbf{Convergencia (Paso 6):}
%     \State Calcular $\mat{S}'_{\text{img,current}} \leftarrow \text{Proyectar}(\text{reshape}(\overline{\vect{s}} + \mat{P}\vect{b}^{(t+1)}, \Ktotal, \dval), s^{(t+1)}, \theta^{(t+1)}, \vect{t}^{(t+1)})$
%     \If{$\|\mat{S}'_{\text{img,current}} - \mat{S}'_{\text{img,prev}}\|_F < \epsilon$} \textbf{break} \EndIf
%     \State $\mat{S}'_{\text{img,prev}} \leftarrow \mat{S}'_{\text{img,current}}$
% \EndFor
% \State \Return $\mat{S}'_{\text{img,current}}, \vect{b}^{(t+1)}, (s^{(t+1)}, \theta^{(t+1)}, \vect{t}^{(t+1)})$
% \end{algorithmic}
% \end{algorithm}

% El resultado de este proceso de ajuste es un conjunto final de parámetros de forma $\hat{\vect{b}}_{\text{final}}$ y parámetros de pose $(s_{\text{final}}, \theta_{\text{final}}, \vect{t}_{\text{final}})$ que representan la mejor adaptación del modelo a la imagen. La forma final en el espacio de la imagen, $\mat{S}'_{\text{final,img}}$, se utiliza para la generación de la máscara de segmentación.

\section{Ajuste Iterativo del SSM (Active Shape Model - ASM)}
\label{sec:ajuste_asm_simplified}

Con una estimación inicial de los parámetros de forma $\hat{\vect{b}}$ (Sección~\ref{sec:prediccion_desdiscretizacion_b_simplified}) y una pose inicial $(s_{ESL}, \theta_{ESL}, \vect{t}_{ESL})$ de la etapa ESL (Sección~\ref{sec:esl_simplified}), se refinan estos parámetros mediante el Active Shape Model (ASM) \cite{cootes1995active}. El ASM ajusta iterativamente el modelo de forma a la evidencia en la imagen.

\begin{figure}[htbp]
    \centering
    \includegraphics[width=0.9\textwidth]{Figures/step0_initial_projection_newIdx20_cat1_id103_1.png}
    \caption{Proyección Inicial del SSM sobre la imagen usando $\hat{\vect{b}}$ y la pose de ESL.}
    \label{fig:step0_initial_projection_simplified}
\end{figure}

El proceso iterativo consta de los siguientes pasos principales en cada iteración $t$:

\begin{enumerate}
    \item \textbf{Generación de la Forma Actual en Espacio Canónico:}
    Se genera la instancia de forma $\vect{s}_{\text{SSM}}(\vect{b}^{(t)})$ usando los parámetros de forma actuales $\vect{b}^{(t)}$ y la ecuación de reconstrucción del SSM (referencia a Ecuación~\eqref{eq:ssm_reconstruction_simplified} de una sección anterior):
    \begin{equation}
    \vect{s}_{\text{SSM}}(\vect{b}^{(t)}) = \mean{\vect{s}} + \mat{P} \vect{b}^{(t)}.
    \label{eq:asm_shape_instance_vec_simplified}
    \end{equation}
    Esta forma se remodela a una matriz de landmarks $\mat{S}'_{\text{SSM}}(\vect{b}^{(t)})$.

    \item \textbf{Proyección al Espacio de la Imagen:}
    La forma canónica $\mat{S}'_{\text{SSM}}(\vect{b}^{(t)})$ se proyecta a la imagen $\mat{S}'_{\text{img}}$ aplicando la transformación de similitud actual (escala $s^{(t)}$, rotación $\mat{R}(\theta^{(t)})$, traslación $\vect{t}^{(t)}$).


    \item \textbf{Búsqueda Local de Puntos Óptimos:}
    Para cada landmark de $\mat{S}'_{\text{img}}$, se busca una nueva posición $\vect{p}^*_{i,\text{img}}$ en la imagen. Esta búsqueda se realiza a lo largo de la normal al contorno, muestreando perfiles de intensidad $\vect{g}_{\text{obs}}$. El punto que minimiza la distancia de Mahalanobis $D_M^2(\vect{g}_{\text{obs}}, (\mean{\vect{g}}_i, \matSigma_i))$ (referencia a Ecuación~\eqref{eq:mahalanobis_distance_simplified} de la sección SAM) se selecciona. El conjunto de estos puntos forma la "forma objetivo" $\mat{S}^*_{\text{target,img}}$.

    \item \textbf{Actualización de Parámetros de Pose:}
    Se calcula una nueva transformación de similitud $(s_{\text{new}}, \theta_{\text{new}}, \vect{t}_{\text{new}})$ que alinea la forma canónica actual $\mat{S}'_{\text{SSM}}(\vect{b}^{(t)})$ con la forma objetivo $\mat{S}^*_{\text{target,img}}$ (usando Procrustes). Los parámetros de pose se actualizan, a menudo con amortiguamiento $\alpha_p$:
    \begin{align}
    \theta^{(t+1)} &= \theta^{(t)} + \alpha_p (\theta_{\text{new}} - \theta^{(t)}), \\
    \vect{t}^{(t+1)} &= \vect{t}^{(t)} + \alpha_p (\vect{t}_{\text{new}} - \vect{t}^{(t)}).
    \end{align}

    \item \textbf{Actualización de Parámetros de Forma:}
    La forma objetivo $\mat{S}^*_{\text{target,img}}$ se transforma de vuelta al espacio canónico del SSM usando la inversa de la pose actualizada $(s^{(t+1)}, \theta^{(t+1)}, \vect{t}^{(t+1)})$, resultando en $\vect{s}'^*_{\text{target,SSM}}$. Esta se proyecta sobre la base del SSM para obtener nuevos parámetros $\vect{b}_{\text{new}}$:
    \begin{equation}
    \vect{b}_{\text{new}} = \transpose{\mat{P}} (\vect{s}'^*_{\text{target,SSM}} - \mean{\vect{s}}).
    \label{eq:asm_project_to_b_simplified}
    \end{equation}
    Los nuevos parámetros de forma se limitan a un rango plausible (e.g., $\pm n_{\text{std}}\sigma_k$ para cada $b_k$), donde $\sigma_k=\sqrt{\lambda_k}$:
    \begin{equation}
    b_k^{(t+1)} = \text{clip}(b_{k,\text{new}}, -n_{\text{std}}\sqrt{\lambda_k}, n_{\text{std}}\sqrt{\lambda_k}).
    \label{eq:asm_b_clamping_simplified}
    \end{equation}

    \item \textbf{Comprobación de Convergencia:}
    El proceso se detiene si se alcanza un número máximo de iteraciones o si el cambio en la forma o los parámetros es menor que una tolerancia $\epsilon$.
\end{enumerate}

\begin{figure}[htbp]
    \centering
    \includegraphics[width=0.9\textwidth]{Figures/step6_end_of_iteration_newIdx20_cat1_id103.png}
    \caption{Ejemplo de la forma SSM ajustada a la imagen después de varias iteraciones.}
    \label{fig:final_asm_fit_simplified}
\end{figure}

El resultado es un conjunto final de parámetros de forma $\hat{\vect{b}}_{\text{final}}$ y de pose $(s_{\text{final}}, \theta_{\text{final}}, \vect{t}_{\text{final}})$ que mejor adaptan el modelo a la imagen.