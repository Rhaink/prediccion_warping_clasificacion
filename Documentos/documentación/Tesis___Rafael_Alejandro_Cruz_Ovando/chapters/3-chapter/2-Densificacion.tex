% \section{Densificación de la Representación de Forma mediante Interpolación Spline}
% \label{sec:densificacion_forma}

% La representación inicial de la forma pulmonar, constituida por los $k=15$ \textit{landmarks} originales, si bien captura la estructura global, resulta insuficiente para modelar con precisión los detalles finos y la curvatura continua de los contornos pulmonares. Para superar esta limitación, se emplea un proceso de densificación basado en la teoría de aproximación de curvas, específicamente mediante la interpolación con splines cúbicos. Este enfoque permite generar una representación de forma más rica y detallada, con un mayor número de puntos de referencia.

% Dada una secuencia ordenada de $k_s$ \textit{landmarks} de un segmento del contorno (por ejemplo, un lóbulo pulmonar):
% \begin{equation}
% \label{eq:landmark_segment_sequence} 
% \hspace*{\fill}
% \{\mathbf{p}_j\}_{j=1}^{k_s},
% \hspace*{\fill}
% \end{equation}
% se construye una curva paramétrica continua, $\mathbf{c}(t)$, que interpola estos puntos:
% \begin{equation}
% \label{eq:parametric_curve_definition} 
% \hspace*{\fill}
% \mathbf{c}(t) = (x(t), y(t)).
% \hspace*{\fill}
% \end{equation}
% Los splines cúbicos (o B-splines de grado 3) son seleccionados por sus propiedades de suavidad, garantizando continuidad hasta la segunda derivada ($C^2$) en los puntos de unión de los segmentos de la curva. Esto es crucial para evitar artefactos angulares en la representación de la forma.

% El proceso de interpolación y densificación se aplica de manera independiente a cada uno de los dos contornos principales (correspondientes a los lóbulos pulmonares) definidos a partir de los \textit{landmarks} iniciales. Para una secuencia $\{\mathbf{p}_j\}_{j=1}^{k_s}$ de un contorno (Ecuación~\eqref{eq:landmark_segment_sequence}), el primer paso es establecer una parametrización. Se asigna un valor paramétrico $u_j$ a cada \textit{landmark} $\mathbf{p}_j$, comúnmente utilizando la longitud de cuerda acumulada para aproximar una parametrización por longitud de arco:
% \begin{equation}
% \label{eq:param_u_initial} 
% \hspace*{\fill}
% u_1 = 0,
% \hspace*{\fill}
% \end{equation}
% \begin{equation}
% \label{eq:param_ui_cumulative} % Etiqueta renombrada para claridad
% \hspace*{\fill}
% u_j = u_{j-1} + ||\mathbf{p}_j - \mathbf{p}_{j-1}||_2, \quad \text{para } j = 2, \dots, k_s.
% \hspace*{\fill}
% \end{equation}

% \begin{figure}[htbp] 
% \centering 
% \includegraphics[width=1\linewidth]{Figures/figura_parametrizacion_cuerda_v7.png} 
% \caption{Ilustración del método de parametrización por longitud de cuerda acumulada para una secuencia de \textit{landmarks} $\mathbf{p}_{j-2}, \mathbf{p}_{j-1}, \mathbf{p}_j$. Los valores paramétricos $u_j$ (ilustrados como $u_1, u_2, u_3$ para los primeros puntos) se calculan según las Ecuaciones~\eqref{eq:param_u_initial} y~\eqref{eq:param_ui_cumulative}, donde cada $u_j$ representa la suma de las distancias euclidianas entre \textit{landmarks} consecutivos a partir del punto inicial.}
% \label{fig:parametrizacion_cuerda}
% \end{figure}

% Posteriormente, se ajusta una curva spline cúbica $\mathbf{c}(u)$ que interpola estos \textit{landmarks} en sus correspondientes valores paramétricos, es decir, se cumple la condición:
% \begin{equation}
% \label{eq:spline_interpolation_condition}
% \hspace*{\fill}
% \mathbf{c}(u_j) = \mathbf{p}_j, \quad \text{para } j = 1, \dots, k_s.
% \hspace*{\fill}
% \end{equation}

% Una vez obtenida la representación continua $\mathbf{c}(u)$, cuyo dominio paramétrico para este contorno es $[u_1, u_{k_s}]$, la curva se remuestrea uniformemente para generar un conjunto más denso de $k_d = 72$ puntos por contorno (donde el subíndice 'd' indica "densificado"). Los nuevos valores paramétricos $v_l$ para estos puntos densificados se calculan como:
% \begin{equation}
% \label{eq:resampling_param}
% \hspace*{\fill}
% v_l = u_1 + (l-1) \frac{u_{k_s} - u_1}{k_d - 1}, \quad \text{para } l = 1, \dots, k_d.
% \hspace*{\fill}
% \end{equation}
% Los nuevos \textit{landmarks} densificados para este contorno, denotados como $\mathbf{q}_l$, se obtienen evaluando la curva spline en estos nuevos parámetros:
% \begin{equation}
% \label{eq:resampled_points}
% \hspace*{\fill}
% \mathbf{q}_l = \mathbf{c}(v_l), \quad \text{para } l = 1, \dots, k_d.
% \hspace*{\fill}
% \end{equation}

% Este procedimiento se aplica a los dos contornos principales. Si denotamos los puntos densificados del primer contorno como $\{\mathbf{q}^{(1)}_l\}_{l=1}^{k_d}$ y los del segundo contorno como $\{\mathbf{q}^{(2)}_l\}_{l=1}^{k_d}$, la matriz de forma densificada completa, $\mathbf{S}' \in \mathbb{R}^{K_{total} \times 2}$ con $K_{total}=2k_d=144$ puntos, se construye concatenando estos dos conjuntos:
% \begin{equation}
% \label{eq:shape_matrix_densified}
% \hspace*{\fill}
% \mathbf{S}' = \begin{bmatrix} \mathbf{q}^{(1)}_1 \\ \vdots \\ \mathbf{q}^{(1)}_{k_d} \\ \mathbf{q}^{(2)}_1 \\ \vdots \\ \mathbf{q}^{(2)}_{k_d} \end{bmatrix}.
% \hspace*{\fill}
% \end{equation}
% Esta matriz $\mathbf{S}'$ proporciona una descripción significativamente más detallada del contorno pulmonar.

% Matemáticamente, la curva spline $\mathbf{c}(u)$ que satisface la Ecuación~\eqref{eq:spline_interpolation_condition} se expresa como una combinación lineal de funciones base B-spline $N_{i,p}(u)$: 
% \begin{equation}
% \label{eq:spline_curve_definition}
% \hspace*{\fill}
% \mathbf{c}(u) = \sum_{i=0}^{n} \mathbf{d}_i N_{i,p}(u), % Cambiado j por i en d_i y N_i,p
% \hspace*{\fill}
% \end{equation}
% donde $\mathbf{d}_i \in \mathbb{R}^2$ son los coeficientes del spline (puntos de control de De Boor), $p$ es el grado del spline (en este caso, $p=3$ para splines cúbicos), y $n+1$ es el número de puntos de control. Las funciones base B-spline $N_{i,p}(u)$ se definen recursivamente sobre un vector de nodos $\mathbf{t} = (t_0, t_1, \dots, t_m)$, donde $m = n+p+1$. La recursión de Cox-de Boor es:
% \begin{equation}
% \label{eq:bspline_basis_0}
% \hspace*{\fill}
% N_{i,0}(u) = \begin{cases} 1 & \text{si } t_i \le u < t_{i+1} \\ 0 & \text{en otro caso} \end{cases} % Cambiado j por i
% \hspace*{\fill}
% \end{equation}
% \begin{equation}
% \label{eq:bspline_basis_p}
% \hspace*{\fill}
% N_{i,p}(u) = \frac{u - t_i}{t_{i+p} - t_i} N_{i,p-1}(u) + \frac{t_{i+p+1} - u}{t_{i+p+1} - t_{i+1}} N_{i+1,p-1}(u). % Cambiado j por i
% \hspace*{\fill}
% \end{equation}
% El proceso de ajuste del spline (Ecuación~\eqref{eq:spline_interpolation_condition}) implica determinar los coeficientes $\mathbf{d}_i$ y el vector de nodos $\mathbf{t}$ de tal manera que la curva resultante interpole los \textit{landmarks} $\mathbf{p}_j$ (de la Ecuación~\eqref{eq:landmark_segment_sequence}) en los valores paramétricos $u_j$.

% \begin{figure}[htbp]
% \centering
% \includegraphics[width=0.8\linewidth]{Figures/spline_densification_visualization.png} 
% \caption{Proceso de densificación de forma mediante interpolación spline. Se muestran los \textit{landmarks} iniciales para un contorno (puntos rojos), parametrizados según Ecuaciones~\eqref{eq:param_u_initial} y~\eqref{eq:param_ui_cumulative}. Una curva spline cúbica (línea colorida) se ajusta para interpolar estos \textit{landmarks} (Ecuación~\eqref{eq:spline_interpolation_condition}), definida por las Ecuaciones~\eqref{eq:spline_curve_definition}-\eqref{eq:bspline_basis_p}. Finalmente, se obtienen puntos densificados (puntos azules, $k_d=72$ por contorno) mediante remuestreo uniforme (Ecuaciones~\eqref{eq:resampling_param} y~\eqref{eq:resampled_points}). La matriz final $\mathbf{S}'$ (Ecuación~\eqref{eq:shape_matrix_densified}) combina los puntos de ambos lóbulos.}
% \label{fig:spline_interpolation_detailed}
% \end{figure}

% La elección de $K_{total}=144$ puntos densificados representa un equilibrio entre la fidelidad de la representación de la forma y la complejidad computacional del modelo estadístico de forma subsecuente. Esta densificación es un paso crucial para capturar la variabilidad morfológica de los pulmones con mayor granularidad.

\section{Densificación de la Representación de Forma mediante Interpolación Spline}
\label{sec:densificacion_forma_simplificada}

La representación inicial de la forma, compuesta por un conjunto de \landmarks{}, puede no ser suficiente para capturar detalles finos. Para mejorar esto, se utiliza la interpolación con splines cúbicos, generando una representación más detallada con más puntos.

Dado un conjunto ordenado de $k_s$ \landmarks{} iniciales de un contorno:
\begin{equation}
\label{eq:landmark_sequence_simplified} 
\hspace*{\fill}
\{\mathbf{p}_j\}_{j=1}^{k_s},
\hspace*{\fill}
\end{equation}
donde cada $\mathbf{p}_j = (x_j, y_j)$ es un punto en 2D.
Se busca construir una curva paramétrica continua, $\mathbf{c}(u)$, que pase a través de estos puntos:
\begin{equation}
\label{eq:parametric_curve_definition_simplified} 
\hspace*{\fill}
\mathbf{c}(u) = (x(u), y(u)).
\hspace*{\fill}
\end{equation}
Los splines cúbicos se eligen por su suavidad, asegurando una curvatura continua.

A cada \landmark{} $\mathbf{p}_j$ se le asigna un valor paramétrico $u_j$. Luego, se ajusta una curva spline cúbica $\mathbf{c}(u)$ tal que interpola estos \landmarks{} en sus correspondientes valores paramétricos:
\begin{equation}
\label{eq:spline_interpolation_condition_simplified}
\hspace*{\fill}
\mathbf{c}(u_j) = \mathbf{p}_j, \quad \text{para } j = 1, \dots, k_s.
\hspace*{\fill}
\end{equation}

Una vez obtenida la curva continua $\mathbf{c}(u)$ (definida en un rango paramétrico, por ejemplo, $[u_{min}, u_{max}]$), esta se remuestrea para generar un conjunto más denso de $k_d$ puntos. Los nuevos valores paramétricos $v_l$ para estos puntos densificados se pueden calcular, por ejemplo, mediante un espaciado uniforme:
\begin{equation}
\label{eq:resampling_param_simplified}
\hspace*{\fill}
v_l = u_{min} + (l-1) \frac{u_{max} - u_{min}}{k_d - 1}, \quad \text{para } l = 1, \dots, k_d.
\hspace*{\fill}
\end{equation}
Los nuevos \landmarks{} densificados, $\mathbf{q}_l$, se obtienen evaluando la curva spline en estos nuevos parámetros:
\begin{equation}
\label{eq:resampled_points_simplified}
\hspace*{\fill}
\mathbf{q}_l = \mathbf{c}(v_l), \quad \text{para } l = 1, \dots, k_d.
\hspace*{\fill}
\end{equation}

Si se procesan múltiples contornos (por ejemplo, dos lóbulos pulmonares), los puntos densificados de cada uno, $\{\mathbf{q}^{(1)}_l\}_{l=1}^{k_d}$ y $\{\mathbf{q}^{(2)}_l\}_{l=1}^{k_d}$, se pueden agrupar para formar una matriz de forma densificada completa, $\mathbf{S}' \in \mathbb{R}^{(N \cdot k_d) \times 2}$ (donde $N$ es el número de contornos):
\begin{equation}
\label{eq:shape_matrix_densified_simplified}
\hspace*{\fill}
\mathbf{S}' = \begin{bmatrix} \mathbf{q}^{(1)}_1 \\ \vdots \\ \mathbf{q}^{(1)}_{k_d} \\ \mathbf{q}^{(2)}_1 \\ \vdots \\ \mathbf{q}^{(2)}_{k_d} \\ \vdots \end{bmatrix}.
\hspace*{\fill}
\end{equation}

Matemáticamente, la curva spline $\mathbf{c}(u)$ (Ecuación~\eqref{eq:spline_interpolation_condition_simplified}) se puede expresar como una combinación lineal de funciones base B-spline $N_{i,p}(u)$:
\begin{equation}
\label{eq:spline_curve_definition_simplified}
\hspace*{\fill}
\mathbf{c}(u) = \sum_{i=0}^{n} \mathbf{d}_i N_{i,p}(u),
\hspace*{\fill}
\end{equation}
donde $\mathbf{d}_i$ son los puntos de control que definen la forma de la curva, $p$ es el grado del spline (usualmente $p=3$ para splines cúbicos), y $N_{i,p}(u)$ son las funciones base B-spline que ponderan la influencia de estos puntos de control a lo largo del parámetro $u$. El ajuste del spline implica determinar los coeficientes $\mathbf{d}_i$ y otros parámetros necesarios para que la curva interpole los \landmarks{} $\mathbf{p}_j$.

\begin{figure}[htbp]
\centering
% Se asume que la imagen sigue siendo relevante para el proceso general.
% Actualiza el path si es necesario.
\includegraphics[width=0.9\linewidth]{Figures/spline_densification_visualization.png} 
\caption{Proceso de densificación de forma. (Puntos rojos) \textit{Landmarks} iniciales $\mathbf{p}_j$. (Línea verde y magenta) Curva spline cúbica $\mathbf{c}(u)$ que interpola los puntos (Ecuación~\eqref{eq:spline_interpolation_condition_simplified}), definida como en Ecuación~\eqref{eq:spline_curve_definition_simplified}. (Puntos azules) Puntos densificados $\mathbf{q}_l$ obtenidos mediante remuestreo (Ecuaciones~\eqref{eq:resampling_param_simplified} y~\eqref{eq:resampled_points_simplified}). La matriz final $\mathbf{S}'$ (Ecuación~\eqref{eq:shape_matrix_densified_simplified}) agrupa estos puntos.}
\label{fig:spline_interpolation_simplified}
\end{figure}

Esta densificación es un paso importante para capturar la variabilidad morfológica con mayor detalle.