%\usepackage[
%	citestyle=ieee, 
%    bibstyle=ieee,
%    style=numeric-comp,
%    sorting=none,
%    doi=false, 
%    url=true,
%    maxbibnames=99, % Make sure we are printing all authors in the appendix
%    ]{biblatex}


% Makes the last name first in the bibliography.
% \DeclareNameAlias{author}{last-first}
\usepackage[
    citestyle=ieee, 
    bibstyle=ieee,
    style=numeric-comp,
    sorting=none,
    doi=false, 
    url=true,
    maxbibnames=99, % Mostrar todos los autores en la bibliografía
]{biblatex}


\addbibresource{reference.bib}  % Nombre del archivo .bib

\DeclareNameAlias{author}{family-given}
\AtEveryBibitem{% Limpiar algunas entradas en la bibliografía
  \clearlist{language}
  \clearfield{note}%
  \ifentrytype{online}{}{\clearfield{url}\clearfield{urlyear}\clearfield{urlmonth}\clearfield{urlday}}%
}

% Modify the label for 'online' entry type
\DeclareLabeldate{%
  \field{date}\field{year}\field{eventdate}\field{origdate}%
  \literal{nodate}%
}

\usepackage[spanish]{babel} % Para la correcta separación silábica en español

\usepackage{amsmath}    % Para entornos de ecuación y comandos matemáticos avanzados
\usepackage{amssymb}    % Para símbolos matemáticos como \mathbb
\usepackage{bm}         % Para mejor "bold math" (opcional, \mathbf es a menudo suficiente)
\usepackage{enumitem}   % Para personalizar listas
% Specify the margins. This is 6.25inches in text with which 
% can be used to size figures to the correct size.
\usepackage{comment}
\usepackage[a4paper, margin=2.5625cm, centering]{geometry}
\usepackage{eso-pic}					% Packages for layout and graphics 
\usepackage{graphicx}
\usepackage{svg}                        % for svg include
\usepackage{tikz}
\usetikzlibrary{fadings, chains, shapes.symbols, positioning}
\usepackage{setspace}
% \usepackage{tocloft}		 		    % Fixing a bug with page style changes for toc	
% \tocloftpagestyle{plain}
\usepackage{bm}                         % Para negritas en símbolos matemáticos
\usepackage{algorithm}                  % Paquete para algoritmos
\usepackage[noend]{algpseudocode}
\usepackage{etoc} 						% Separate tocs for appendix and the rest    
\usepackage{chngcntr}					% Count figures within chapters
\usepackage{booktabs}					% Table formatting
\usepackage{fancyhdr}					% Setting the style for header and footer.
\usepackage{tabularx}
\usepackage{multirow}                   % For better tables 
\usepackage[hidelinks]{hyperref}		% Clickable links
\usepackage{nameref}					% References with names
\usepackage[parfill]{parskip}			% New line instead of indent for sections
\usepackage{tcolorbox}					% Create boxes around content
\tcbset{colback=white,arc=0mm}
\usepackage{pgfgantt}                   % For gantt chart 
\usepackage{amsmath}                    % For math
% \usepackage{mathdots}
% \usepackage{yhmath}
\usepackage{siunitx}
\usepackage{pdfpages}                  % for pdf insert
% \usepackage{rotating}                  % sideway figures
% \usepackage{array}
% \usepackage{gensymb}
\usepackage{amssymb}
\usepackage{mathtools}              % Add text to math arrows.
\usepackage{float}
% \usepackage{cancel}
\usepackage{tocloft}                
\usepackage{color}
\usepackage{multirow}
\usepackage{textcomp}               % Fixing warning for gensyb \perthousand
\usepackage{svg}                    % including svg files
\usepackage{caption}                % For subfigures
\usepackage{subcaption}
\usepackage{fontspec}               % For Fonts specification
\usepackage{fontawesome}
\usepackage{datetime}
\usepackage{titlesec}               % For title
\usepackage{longtable}              % For long table
\usepackage{makecell}
\usepackage{array, multirow, multicol}
% \usepackage{smartdiagram}
\usepackage{overpic}                % overlay graphics
\usepackage[absolute,overlay]{textpos} % a flexible way to position tex
\usepackage{lmodern}
%\usepackage{glossaries}             % For glossarries
\usepackage{xeCJK}                  % For Chinese
% \usepackage{xpatch}                 % provides commands for patching existing commands or environments.
% \usepackage{transparent}
\usepackage{pgfplots}
\pgfplotsset{compat=newest}
% \usepackage{listings}              % To include the code
% \usepackage{pythonhighlight}        % beautiful python code apperance listing
\usepackage{etoolbox}               % for the groups in nomenclature
% \usepackage[printonlyused]{acronym} %[nohyperlinks]
\usepackage{nomencl}               % for nomenclature
\makenomenclature
\renewcommand\nomgroup[1]{%
  \item[\bfseries
  \ifstrequal{#1}{A}{Abreviaturas}{
    \ifstrequal{#1}{P}{Constantes Físicas}{%
      \ifstrequal{#1}{N}{Símbolos matemáticos}{%
        \ifstrequal{#1}{C}{Constantes matemáticas}{}}}}%
]}

\counterwithin{figure}{section} 
\counterwithin{table}{section}

% Specifying fonts
\setmainfont{TeX Gyre Termes} 
% \setmainfont{Georgia} 
\setsansfont{Arial}

\newfontfamily\footerfont{Georgia}
\newfontfamily\colvillebold{ColvilleBold}[
    Path = {setup/font/}, % Change this path if you uploaded the font file to a different location
    Extension = .otf % Change the extension if your font file has a different extension
]
\newfontfamily\Opticopperplate{OPTICopperplate}[
    Path = {setup/font/}, % Change this path if you uploaded the font file to a different location
    Extension = .otf % Change the extension if your font file has a different extension
]
\setCJKmainfont{Noto Serif CJK SC} % Use Noto Serif CJK SC as the CJK font
\setCJKsansfont{Noto Sans CJK SC} % Use Noto Sans CJK SC as the CJK sans-serif font
\setCJKmonofont{Noto Sans Mono CJK SC} % Use Noto Sans Mono CJK SC as the CJK monospace font

%\tableofcontents
%\newcommand{\chapfnt}
% \chapterfont{\sffamily\fontsize{17}{17}}
% \sectionfont{\sffamily\fontsize{14}{15}}
% \subsectionfont{\sffamily\fontsize{13}{15}}
% \subsubsectionfont{\sffamily\fontsize{12}{15}}
\titleformat{\chapter}[display]
  {\normalfont\sffamily\huge\bfseries}
  {Capítulo \thechapter}{17pt}{\Huge}

% \titleformat{\section}
%   {\normalfont\sffamily\Large\bfseries}
%   {\thesection}{1em}{}

% Define \faBilibili command
\newcommand{\faBilibili}{%
  \includesvg[width=1em]{setup/img/bilibili.svg}% Adjust the width as needed
}

% Remove the title and make sure that the text is adjusted
% \usepackage{abstract}
% \setlength{\absleftindent}{0mm}
% \renewcommand{\abstractname}{\vspace{-\baselineskip}}
% \renewcommand{\abstractnamefont}{\sffamily\fontsize{14}{15}}
% \renewcommand{\abstracttextfont}{\normalfont\fontsize{12}{13}}

% Renaming and setting style of table of contents

\renewcommand*\contentsname{Tabla de Contenido}
\renewcommand*\cfttoctitlefont{\fontsize{16}{0}\bf\sffamily}
\renewcommand\cftchapfont{\fontsize{15}{0}\bf}
\renewcommand\cftchappagefont{\fontsize{14}{0}\bf}
\renewcommand\cftsecfont{\fontsize{12}{0}}
\renewcommand\cftsecpagefont{\fontsize{12}{0}}
\renewcommand\cftsubsecfont{\fontsize{12}{0}}
\renewcommand\cftsubsecpagefont{\fontsize{12}{0}}
\captionsetup{font=small}

% Styling the header and footer
\fancypagestyle{myfancy}{%
    \fancyhf{}
    \fancyhead{}
    \fancyfoot{}
%     Codigo original-INICIO, Para la tesis debe estar esta configuración
%     Cambiar para encabezados
    \fancyhead[OL]{\selectfont\leftmark}
    \fancyhead[OR]{\selectfont\thepage}
    \fancyhead[ER]{\selectfont\rightmark}
    \fancyhead[EL]{\selectfont\thepage}
%     Codigo original-FIN

%     Codigo modificado-INICIO, Para el protocolo
%    \fancyhead[OL]{\selectfont\rightmark}% Encabezado izquierdo en páginas impares
%    \fancyhead[OR]{\selectfont\thepage}% Encabezado derecho en páginas impares
%    \fancyhead[EL]{\selectfont\rightmark} % Encabezado izquierdo en páginas pares
%    \fancyhead[ER]{\selectfont\thepage}   % Encabezado derecho en páginas pares
    %\fancyhead[ER]{\selectfont\rightmark}
    %\fancyhead[EL]{\selectfont\thepage}
%     Codigo modificado-FIN

    \setlength{\headheight}{15.5pt}
    \renewcommand{\headrulewidth}{0.4pt} % Header rule
    %\renewcommand{\footrulewidth}{0.4pt} % Footer rule
}

\fancypagestyle{myempty}{%
    \fancyhf{} % Clear all header and footer fields
    \setlength{\headheight}{0em}
    % \cfoot[C]{\selectfont\thepage}
    % \fancyhead[L]{\selectfont\leftmark}
    \setlength{\headheight}{15.5pt}
    \fancyfoot[C]{\thepage} % Centered page number in the footer
    \renewcommand{\headrulewidth}{0pt} % Remove header rule
    \renewcommand{\footrulewidth}{0pt} % Remove footer rule
}
% \fancyfoot[R]{\footerfont\thepage}


\fancypagestyle{plain}{
    \fancyhf{}
    % \fancyheadoffset{1cm}
    \setlength{\headheight}{0em}
    
    % \fancyhead{}
    % \fancyfoot{}
    \renewcommand{\headrulewidth}{0pt}
    \fancyfoot[C]{\selectfont\thepage}
}

\titlespacing*{\chapter}{0pt}{0pt}{10pt}
\AtBeginDocument{\addtocontents{toc}{\protect\thispagestyle{plain}}}

% Making the command for placing text in random locations
\newcommand\PlaceText[3]{%
\begin{tikzpicture}[remember picture,overlay]
\node[outer sep=0pt,inner sep=0pt,anchor=south west] 
  at ([xshift=#1,yshift=-#2]current page.north west) {#3};
\end{tikzpicture}%
}

% Disable hyphenation
\pretolerance=10000
\tolerance=2000 
\emergencystretch=50pt

\renewcommand{\figurename}{Figura }
\addto\captionsspanish{\renewcommand{\tablename}{Tabla}}

\numberwithin{equation}{section} % Numerar ecuaciones según las secciones

\newcommand{\code}[1]{\texttt{\detokenize{#1}}}


%Declaración de nuevas unidades

\DeclareSIUnit\kbps{kbps}
\DeclareSIUnit\mbps{Mbps}
\DeclareSIUnit\kbps{kbps}
\DeclareSIUnit\gbps{Gbps}

\usepackage{pgfplots}
\pgfplotsset{compat=newest}