% ==============================================================================
% DOCUMENTO 02: ARQUITECTURA DEL MODELO DE PREDICCIÓN DE LANDMARKS
% Sesiones cubiertas: 2-3
% Proyecto: Detección de COVID-19 mediante Landmarks Anatómicos
% ==============================================================================

\documentclass[12pt,a4paper]{article}
% ==============================================================================
% PREÁMBULO LATEX PARA DOCUMENTACIÓN CIENTÍFICA DE NIVEL DOCTORAL
% Proyecto: Detección de COVID-19 mediante Landmarks Anatómicos y Warping Geométrico
% ==============================================================================

% --- Codificación y idioma ---
\usepackage[utf8]{inputenc}
\usepackage[T1]{fontenc}
\usepackage[spanish,es-tabla]{babel}

% --- Matemáticas ---
\usepackage{amsmath,amssymb,amsfonts,amsthm}
\usepackage{mathtools}
\usepackage{bm}  % Negritas en matemáticas

% --- Tablas profesionales ---
\usepackage{booktabs}
\usepackage{multirow}
\usepackage{array}
\usepackage{longtable}
\usepackage{tabularx}

% --- Figuras y gráficos ---
\usepackage{graphicx}
\usepackage{subfig}
\usepackage{float}
\usepackage{caption}

% --- Algoritmos y pseudocódigo ---
\usepackage{algorithm}
\usepackage{algorithmic}

% --- Colores (debe cargarse antes de listings) ---
\usepackage{xcolor}
\definecolor{covidred}{RGB}{220,53,69}
\definecolor{normalgreen}{RGB}{40,167,69}
\definecolor{viralblue}{RGB}{0,123,255}
\definecolor{codeblue}{RGB}{0,0,180}
\definecolor{codegray}{RGB}{128,128,128}

% --- Código fuente ---
\usepackage{listings}
\lstset{
    language=Python,
    basicstyle=\ttfamily\small,
    keywordstyle=\color{blue}\bfseries,
    commentstyle=\color{gray}\itshape,
    stringstyle=\color{red},
    numbers=left,
    numberstyle=\tiny\color{gray},
    stepnumber=1,
    numbersep=5pt,
    backgroundcolor=\color{white},
    frame=single,
    rulecolor=\color{black},
    tabsize=4,
    captionpos=b,
    breaklines=true,
    breakatwhitespace=false,
    showspaces=false,
    showstringspaces=false,
    showtabs=false,
    literate={á}{{\'a}}1 {é}{{\'e}}1 {í}{{\'i}}1 {ó}{{\'o}}1 {ú}{{\'u}}1
             {Á}{{\'A}}1 {É}{{\'E}}1 {Í}{{\'I}}1 {Ó}{{\'O}}1 {Ú}{{\'U}}1
             {ñ}{{\~n}}1 {Ñ}{{\~N}}1
             {ü}{{\"u}}1 {Ü}{{\"U}}1
             {¿}{{?`}}1 {¡}{{!`}}1
}

% --- Referencias y enlaces ---
\usepackage{hyperref}
\hypersetup{
    colorlinks=true,
    linkcolor=blue,
    filecolor=magenta,
    urlcolor=cyan,
    citecolor=blue,
    pdftitle={Documentación del Proyecto de Tesis},
    pdfauthor={},
}
\usepackage{cleveref}

% --- Bibliografía ---
\usepackage{natbib}
\bibliographystyle{plainnat}

% --- Geometría de página ---
\usepackage[a4paper,margin=2.5cm]{geometry}

% --- Espaciado ---
\usepackage{setspace}
\onehalfspacing

% --- Encabezados y pies ---
\usepackage{fancyhdr}
\pagestyle{fancy}
\fancyhf{}
\rhead{\rightmark}
\lhead{\leftmark}
\cfoot{\thepage}

% --- Entornos personalizados ---
\theoremstyle{definition}
\newtheorem{definicion}{Definición}[section]
\newtheorem{proposicion}{Proposición}[section]
\newtheorem{teorema}{Teorema}[section]
\newtheorem{lema}{Lema}[section]
\newtheorem{corolario}{Corolario}[section]

\theoremstyle{remark}
\newtheorem{observacion}{Observación}[section]
\newtheorem{nota}{Nota}[section]
\newtheorem{hipotesis}{Hipótesis}[section]

% --- Comandos personalizados ---
% Vectores y matrices
\newcommand{\vect}[1]{\mathbf{#1}}
\newcommand{\mat}[1]{\mathbf{#1}}

% Operadores
\DeclareMathOperator*{\argmin}{arg\,min}
\DeclareMathOperator*{\argmax}{arg\,max}
\DeclareMathOperator{\sgn}{sgn}
\DeclareMathOperator{\diag}{diag}
\DeclareMathOperator{\trace}{tr}

% Normas y productos
\newcommand{\norm}[1]{\left\|#1\right\|}
\newcommand{\abs}[1]{\left|#1\right|}
\newcommand{\inner}[2]{\langle #1, #2 \rangle}

% Conjuntos
\newcommand{\R}{\mathbb{R}}
\newcommand{\N}{\mathbb{N}}
\newcommand{\Z}{\mathbb{Z}}

% Espacios de imágenes y landmarks
\newcommand{\imgspace}{\mathcal{I}}
\newcommand{\landmarkspace}{\mathcal{L}}
\newcommand{\classspace}{\mathcal{Y}}

% Landmarks específicos
\newcommand{\landmark}[1]{L_{#1}}
\newcommand{\landmarkpair}[2]{(L_{#1}, L_{#2})}

% Funciones de pérdida
\newcommand{\loss}{\mathcal{L}}
\newcommand{\wingloss}{\mathcal{L}_{\text{wing}}}
\newcommand{\mseloss}{\mathcal{L}_{\text{MSE}}}

% Métricas
\newcommand{\accuracy}{\text{Acc}}
\newcommand{\precision}{\text{Prec}}
\newcommand{\recall}{\text{Rec}}
\newcommand{\fscore}{F_1}

% Abreviaciones
\newcommand{\ie}{\textit{i.e.}}
\newcommand{\eg}{\textit{e.g.}}
\newcommand{\etal}{\textit{et al.}}
\newcommand{\etc}{\textit{etc.}}

% Referencias a archivos del proyecto
\newcommand{\archivo}[1]{\texttt{#1}}
\newcommand{\funcion}[1]{\texttt{#1()}}
\newcommand{\clase}[1]{\texttt{#1}}
\newcommand{\parametro}[1]{\texttt{#1}}

% Unidades
\newcommand{\px}{\,\text{px}}
\newcommand{\epoch}{\,\text{época}}
\newcommand{\epochs}{\,\text{épocas}}

% --- Información del documento ---
\newcommand{\proyectotitulo}{Detección de COVID-19 en Radiografías de Tórax mediante Landmarks Anatómicos y Normalización Geométrica}
\newcommand{\proyectosubtitulo}{Documentación Científica del Proceso de Desarrollo}

% --- Formato de secciones ---
\usepackage{titlesec}
\titleformat{\section}
  {\normalfont\Large\bfseries}{\thesection}{1em}{}
\titleformat{\subsection}
  {\normalfont\large\bfseries}{\thesubsection}{1em}{}
\titleformat{\subsubsection}
  {\normalfont\normalsize\bfseries}{\thesubsubsection}{1em}{}

% --- Notas al margen para figuras sugeridas ---
\usepackage{marginnote}
\newcommand{\figurasugerida}[1]{\marginnote{\scriptsize\textcolor{blue}{[Figura: #1]}}}

% --- Cajas para resultados importantes ---
\usepackage{tcolorbox}
\newtcolorbox{resultadoimportante}[1][]{
    colback=green!5!white,
    colframe=green!75!black,
    fonttitle=\bfseries,
    title=Resultado Importante,
    #1
}

\newtcolorbox{hallazgo}[1][]{
    colback=blue!5!white,
    colframe=blue!75!black,
    fonttitle=\bfseries,
    title=Hallazgo,
    #1
}

\newtcolorbox{metodologia}[1][]{
    colback=yellow!5!white,
    colframe=yellow!75!black,
    fonttitle=\bfseries,
    title=Metodología,
    #1
}

\newtcolorbox{figuradescripcion}[1][]{
    colback=gray!5!white,
    colframe=gray!75!black,
    fonttitle=\bfseries,
    title=Descripción de Figura,
    #1
}

% ==============================================================================
% FIN DEL PREÁMBULO
% ==============================================================================


\title{\textbf{Arquitectura del Modelo de Predicción de Landmarks}\\[0.5em]
\large Documentación del Proceso de Desarrollo - Sesiones 2-3}
\author{Proyecto de Tesis Doctoral\\Detección de COVID-19 mediante Landmarks Anatómicos}
\date{Noviembre 2024}

\begin{document}
\maketitle

\begin{abstract}
Este documento presenta la arquitectura del modelo de predicción de landmarks anatómicos basado en ResNet-18 con modificaciones específicas para regresión de coordenadas. Se describe en detalle el backbone convolucional, el módulo de Coordinate Attention (CVPR 2021), la cabeza de regresión profunda con GroupNorm, y la estrategia de normalización de salidas. El modelo resultante tiene 11.3M parámetros y produce 30 valores de salida correspondientes a las coordenadas $(x, y)$ de 15 landmarks.
\end{abstract}

\tableofcontents
\newpage

% ==============================================================================
\section{Introducción}
% ==============================================================================

\subsection{Contexto del Problema}

La predicción de landmarks anatómicos en imágenes médicas constituye un problema de regresión donde el objetivo es estimar las coordenadas espaciales de puntos de interés anatómico. A diferencia de la clasificación, donde la salida es discreta, la regresión de landmarks requiere predicciones continuas precisas.

\begin{definicion}[Problema de Regresión de Landmarks]
Dado un espacio de imágenes $\imgspace \subset \R^{H \times W \times C}$ y un espacio de landmarks $\landmarkspace = \R^{K \times 2}$ donde $K$ es el número de puntos, el objetivo es aprender una función:
\begin{equation}
f_\theta: \imgspace \rightarrow \landmarkspace
\end{equation}
que minimice el error de localización respecto a las anotaciones ground truth.
\end{definicion}

\subsection{Requisitos de Diseño}

La arquitectura fue diseñada considerando los siguientes requisitos:

\begin{enumerate}
    \item \textbf{Capacidad de extracción de características}: Representaciones jerárquicas multi-escala
    \item \textbf{Transfer learning}: Aprovechamiento de pesos pre-entrenados en ImageNet
    \item \textbf{Atención espacial}: Focalización en regiones anatómicamente relevantes
    \item \textbf{Regularización}: Control de overfitting en dataset pequeño (957 muestras)
    \item \textbf{Salida normalizada}: Coordenadas en rango $[0, 1]$ para invarianza a escala
\end{enumerate}

% ==============================================================================
\section{Arquitectura General}
% ==============================================================================

\subsection{Visión de Alto Nivel}

La arquitectura se compone de tres módulos principales conectados secuencialmente:

\begin{equation}
f_\theta = h_{\text{head}} \circ \phi_{\text{attn}} \circ g_{\text{backbone}}
\end{equation}

donde:
\begin{itemize}
    \item $g_{\text{backbone}}: \R^{3 \times 224 \times 224} \rightarrow \R^{512 \times 7 \times 7}$ es el backbone ResNet-18
    \item $\phi_{\text{attn}}: \R^{512 \times 7 \times 7} \rightarrow \R^{512 \times 7 \times 7}$ es el módulo de atención
    \item $h_{\text{head}}: \R^{512} \rightarrow \R^{30}$ es la cabeza de regresión
\end{itemize}

\begin{figure}[H]
\centering
% [Figura sugerida: Diagrama de bloques de la arquitectura completa]
\caption{Arquitectura del modelo ResNet18Landmarks. La imagen de entrada pasa por el backbone ResNet-18, seguido del módulo Coordinate Attention, Global Average Pooling, y finalmente la cabeza de regresión profunda que produce 30 coordenadas normalizadas.}
\label{fig:arquitectura_general}
\end{figure}

\subsection{Parámetros del Modelo}

\begin{table}[H]
\centering
\caption{Resumen de parámetros del modelo}
\label{tab:parametros_modelo}
\begin{tabular}{lcc}
\toprule
\textbf{Componente} & \textbf{Parámetros} & \textbf{Porcentaje} \\
\midrule
Backbone ResNet-18 & 11,176,512 & 98.56\% \\
Coordinate Attention & 24,608 & 0.22\% \\
Cabeza de Regresión & 139,038 & 1.23\% \\
\midrule
\textbf{Total} & \textbf{11,340,158} & \textbf{100\%} \\
\bottomrule
\end{tabular}
\end{table}

% ==============================================================================
\section{Backbone: ResNet-18}
% ==============================================================================

\subsection{Fundamentos de Redes Residuales}

Las redes residuales (ResNet) introducidas por He \etal\ (2016) resuelven el problema de degradación en redes profundas mediante conexiones de identidad (skip connections).

\begin{definicion}[Bloque Residual]
Un bloque residual implementa la transformación:
\begin{equation}
\vect{y} = \mathcal{F}(\vect{x}, \{W_i\}) + \vect{x}
\label{eq:bloque_residual}
\end{equation}
donde $\mathcal{F}(\vect{x}, \{W_i\})$ representa las capas de peso y $\vect{x}$ es la conexión de identidad.
\end{definicion}

\subsection{Estructura de ResNet-18}

ResNet-18 consiste en 18 capas con pesos entrenables, organizadas en 4 etapas (stages):

\begin{table}[H]
\centering
\caption{Arquitectura de ResNet-18}
\label{tab:resnet18}
\begin{tabular}{lccc}
\toprule
\textbf{Etapa} & \textbf{Salida} & \textbf{Bloques} & \textbf{Filtros} \\
\midrule
Conv1 + MaxPool & $56 \times 56$ & 1 & 64 \\
Layer1 (Stage 1) & $56 \times 56$ & 2 & 64 \\
Layer2 (Stage 2) & $28 \times 28$ & 2 & 128 \\
Layer3 (Stage 3) & $14 \times 14$ & 2 & 256 \\
Layer4 (Stage 4) & $7 \times 7$ & 2 & 512 \\
\bottomrule
\end{tabular}
\end{table}

\subsection{Bloque Básico de ResNet-18}

Cada bloque básico de ResNet-18 contiene dos capas convolucionales $3 \times 3$:

\begin{equation}
\mathcal{F}(\vect{x}) = W_2 \cdot \sigma(BN(W_1 \cdot \vect{x}))
\end{equation}

donde $\sigma$ es ReLU, $BN$ es Batch Normalization, y $W_1$, $W_2$ son matrices de convolución.

\begin{algorithm}[H]
\caption{Bloque Básico ResNet}
\label{alg:basic_block}
\begin{algorithmic}[1]
\REQUIRE Feature map $\vect{x} \in \R^{C \times H \times W}$
\ENSURE Feature map $\vect{y} \in \R^{C' \times H' \times W'}$
\STATE $\vect{z}_1 \leftarrow \text{Conv}_{3\times3}(\vect{x})$
\STATE $\vect{z}_1 \leftarrow \text{BatchNorm}(\vect{z}_1)$
\STATE $\vect{z}_1 \leftarrow \text{ReLU}(\vect{z}_1)$
\STATE $\vect{z}_2 \leftarrow \text{Conv}_{3\times3}(\vect{z}_1)$
\STATE $\vect{z}_2 \leftarrow \text{BatchNorm}(\vect{z}_2)$
\IF{dimensiones cambian}
    \STATE $\vect{x} \leftarrow \text{Conv}_{1\times1}(\vect{x})$ \COMMENT{Proyección}
\ENDIF
\STATE $\vect{y} \leftarrow \text{ReLU}(\vect{z}_2 + \vect{x})$
\RETURN $\vect{y}$
\end{algorithmic}
\end{algorithm}

\subsection{Modificaciones al Backbone Original}

Para adaptar ResNet-18 a regresión de landmarks, se realizaron las siguientes modificaciones:

\begin{enumerate}
    \item \textbf{Eliminación de la capa FC}: La capa fully-connected original (clasificación 1000 clases) fue removida
    \item \textbf{Global Average Pooling}: Aplicado después de layer4 para obtener vector de 512 dimensiones
    \item \textbf{Pre-entrenamiento ImageNet}: Se utilizaron pesos pre-entrenados como inicialización
\end{enumerate}

\begin{lstlisting}[caption={Modificación del backbone ResNet-18}]
import torchvision.models as models

# Cargar ResNet-18 pretrained
backbone = models.resnet18(pretrained=True)

# Remover capa FC original
backbone = nn.Sequential(*list(backbone.children())[:-2])

# Output: (batch, 512, 7, 7) para input (batch, 3, 224, 224)
\end{lstlisting}

% ==============================================================================
\section{Módulo Coordinate Attention}
% ==============================================================================

\subsection{Motivación}

El módulo Coordinate Attention fue propuesto por Hou \etal\ (CVPR 2021) para capturar dependencias espaciales de largo alcance mientras preserva información posicional precisa. A diferencia de SE-Net o CBAM, Coordinate Attention codifica información de canal y posición simultáneamente.

\begin{definicion}[Coordinate Attention]
Dado un feature map $\vect{X} \in \R^{C \times H \times W}$, Coordinate Attention produce pesos de atención $\vect{A} \in \R^{C \times H \times W}$ que reponderan el feature map original:
\begin{equation}
\vect{Y} = \vect{A} \odot \vect{X}
\end{equation}
donde $\odot$ denota multiplicación elemento a elemento.
\end{definicion}

\subsection{Arquitectura del Módulo}

El módulo consta de tres etapas:

\subsubsection{Etapa 1: Pooling Direccional}

Se aplican dos operaciones de average pooling en direcciones ortogonales:

\begin{equation}
z_c^h(h) = \frac{1}{W} \sum_{0 \leq i < W} x_c(h, i)
\label{eq:pool_h}
\end{equation}

\begin{equation}
z_c^w(w) = \frac{1}{H} \sum_{0 \leq j < H} x_c(j, w)
\label{eq:pool_w}
\end{equation}

donde $z_c^h \in \R^{C \times H \times 1}$ y $z_c^w \in \R^{C \times 1 \times W}$ capturan información espacial a lo largo de dimensiones ortogonales.

\subsubsection{Etapa 2: Transformación Compartida}

Los vectores pooled se concatenan y transforman:

\begin{equation}
\vect{f} = \delta\left(F_1\left([\vect{z}^h, \vect{z}^w]\right)\right)
\label{eq:transform}
\end{equation}

donde $[\cdot, \cdot]$ denota concatenación, $F_1$ es una convolución $1 \times 1$ que reduce canales por factor $r=32$, y $\delta$ es la función BatchNorm + ReLU.

\subsubsection{Etapa 3: Generación de Pesos de Atención}

El vector transformado se divide y se aplican convoluciones separadas:

\begin{equation}
\vect{g}^h = \sigma(F_h(\vect{f}^h)), \quad \vect{g}^w = \sigma(F_w(\vect{f}^w))
\label{eq:attention_weights}
\end{equation}

donde $\sigma$ es la función sigmoide y $F_h$, $F_w$ son convoluciones $1 \times 1$ que restauran los canales originales.

La salida final se calcula como:

\begin{equation}
y_c(h, w) = x_c(h, w) \cdot g_c^h(h) \cdot g_c^w(w)
\label{eq:coord_attn_output}
\end{equation}

\begin{algorithm}[H]
\caption{Coordinate Attention}
\label{alg:coord_attn}
\begin{algorithmic}[1]
\REQUIRE Feature map $\vect{X} \in \R^{C \times H \times W}$, factor de reducción $r$
\ENSURE Feature map atentido $\vect{Y} \in \R^{C \times H \times W}$
\STATE \COMMENT{Pooling direccional}
\STATE $\vect{z}^h \leftarrow \text{AvgPool}_W(\vect{X})$ \COMMENT{$(C, H, 1)$}
\STATE $\vect{z}^w \leftarrow \text{AvgPool}_H(\vect{X})$ \COMMENT{$(C, 1, W)$}
\STATE \COMMENT{Concatenar y transformar}
\STATE $\vect{z}^w_{\text{perm}} \leftarrow \text{permute}(\vect{z}^w, [0, 2, 1])$ \COMMENT{$(C, W, 1)$}
\STATE $\vect{z} \leftarrow \text{concat}([\vect{z}^h, \vect{z}^w_{\text{perm}}], \text{dim}=2)$ \COMMENT{$(C, H+W, 1)$}
\STATE $\vect{f} \leftarrow \text{ReLU}(\text{BN}(\text{Conv}_{1\times1}(\vect{z})))$ \COMMENT{$(C/r, H+W, 1)$}
\STATE \COMMENT{Dividir y generar atención}
\STATE $\vect{f}^h, \vect{f}^w \leftarrow \text{split}(\vect{f}, [H, W], \text{dim}=2)$
\STATE $\vect{g}^h \leftarrow \sigma(\text{Conv}_{1\times1}(\vect{f}^h))$ \COMMENT{$(C, H, 1)$}
\STATE $\vect{g}^w \leftarrow \sigma(\text{Conv}_{1\times1}(\text{permute}(\vect{f}^w)))$ \COMMENT{$(C, 1, W)$}
\STATE \COMMENT{Aplicar atención}
\STATE $\vect{Y} \leftarrow \vect{X} \odot \vect{g}^h \odot \vect{g}^w$
\RETURN $\vect{Y}$
\end{algorithmic}
\end{algorithm}

\subsection{Configuración Utilizada}

\begin{table}[H]
\centering
\caption{Parámetros del módulo Coordinate Attention}
\label{tab:coord_attn_params}
\begin{tabular}{lcc}
\toprule
\textbf{Parámetro} & \textbf{Valor} & \textbf{Justificación} \\
\midrule
Factor de reducción $r$ & 32 & Balance complejidad/capacidad \\
Canales de entrada & 512 & Salida de layer4 \\
Canales intermedios & 16 & $512/32 = 16$ \\
Posición en red & Después de layer4 & Máxima receptive field \\
\bottomrule
\end{tabular}
\end{table}

\begin{hallazgo}
La inserción de Coordinate Attention después de layer4 permite que el módulo opere sobre features de alto nivel semántico con máximo receptive field ($224 \times 224$ píxeles para entrada de $224 \times 224$), facilitando la captura de relaciones espaciales entre landmarks distantes.
\end{hallazgo}

% ==============================================================================
\section{Cabeza de Regresión Profunda}
% ==============================================================================

\subsection{Diseño de la Cabeza}

La cabeza de regresión transforma el vector de características del backbone en coordenadas de landmarks:

\begin{equation}
h_{\text{head}}: \R^{512} \rightarrow \R^{30}
\end{equation}

\subsection{Arquitectura Detallada}

La cabeza profunda consiste en tres bloques secuenciales:

\begin{equation}
h_{\text{head}} = \sigma \circ \text{FC}_3 \circ \text{Block}_2 \circ \text{Block}_1
\end{equation}

donde cada bloque tiene la estructura:

\begin{equation}
\text{Block}_i = \text{Dropout} \circ \text{ReLU} \circ \text{GroupNorm} \circ \text{FC}_i
\end{equation}

\begin{table}[H]
\centering
\caption{Arquitectura de la cabeza de regresión}
\label{tab:head_arquitectura}
\begin{tabular}{lccc}
\toprule
\textbf{Capa} & \textbf{Entrada} & \textbf{Salida} & \textbf{Activación} \\
\midrule
FC1 & 512 & 512 & GroupNorm + ReLU + Dropout \\
FC2 & 512 & 768 & GroupNorm + ReLU + Dropout \\
FC3 & 768 & 30 & Sigmoid \\
\bottomrule
\end{tabular}
\end{table}

\subsection{Group Normalization vs Batch Normalization}

Se optó por Group Normalization en lugar de Batch Normalization debido a:

\begin{definicion}[Group Normalization]
Para un feature $\vect{x} \in \R^{N \times C \times H \times W}$, Group Normalization divide los canales en $G$ grupos y normaliza dentro de cada grupo:
\begin{equation}
\hat{x}_{n,c} = \frac{x_{n,c} - \mu_{n,g(c)}}{\sqrt{\sigma_{n,g(c)}^2 + \epsilon}}
\end{equation}
donde $g(c) = \lfloor c \cdot G / C \rfloor$ mapea el canal $c$ a su grupo.
\end{definicion}

\begin{observacion}
Group Normalization es preferible cuando:
\begin{itemize}
    \item El batch size es pequeño (< 16)
    \item Las estadísticas de batch son inestables
    \item Se requiere consistencia entre entrenamiento e inferencia
\end{itemize}
En este proyecto, con batch size de 16, Group Normalization con $G=32$ grupos proporcionó entrenamiento más estable.
\end{observacion}

\subsection{Regularización con Dropout}

Se aplica Dropout después de cada capa de normalización:

\begin{equation}
\tilde{\vect{h}} = \vect{m} \odot \vect{h}, \quad m_i \sim \text{Bernoulli}(1-p)
\end{equation}

\begin{table}[H]
\centering
\caption{Comparación de tasas de dropout}
\label{tab:dropout_comparison}
\begin{tabular}{lcc}
\toprule
\textbf{Dropout Rate} & \textbf{Error Validación (px)} & \textbf{Observación} \\
\midrule
$p = 0.5$ & 8.42 & Regularización excesiva \\
$p = 0.4$ & 7.89 & Bueno \\
$p = 0.3$ & \textbf{7.21} & \textbf{Óptimo} \\
$p = 0.2$ & 7.68 & Underfitting \\
\bottomrule
\end{tabular}
\end{table}

% ==============================================================================
\section{Normalización de Salidas}
% ==============================================================================

\subsection{Motivación}

Las coordenadas de landmarks se normalizan al rango $[0, 1]$ para:

\begin{enumerate}
    \item Hacer el modelo invariante a la resolución de entrada
    \item Facilitar el uso de funciones de pérdida normalizadas
    \item Mejorar la estabilidad numérica del entrenamiento
\end{enumerate}

\subsection{Función Sigmoide de Salida}

La última capa aplica la función sigmoide:

\begin{equation}
\sigma(z) = \frac{1}{1 + e^{-z}}
\end{equation}

que mapea cualquier valor real al intervalo $(0, 1)$.

\begin{definicion}[Coordenadas Normalizadas]
Las coordenadas normalizadas $(\hat{x}, \hat{y}) \in [0, 1]^2$ se relacionan con las coordenadas en píxeles mediante:
\begin{equation}
x_{\text{px}} = \hat{x} \cdot W, \quad y_{\text{px}} = \hat{y} \cdot H
\end{equation}
donde $W = H = 224$ es la resolución de entrada.
\end{definicion}

\subsection{Estructura de la Salida}

El modelo produce un vector de 30 valores:

\begin{equation}
\hat{\vect{L}} = [\hat{x}_1, \hat{y}_1, \hat{x}_2, \hat{y}_2, \ldots, \hat{x}_{15}, \hat{y}_{15}]^T \in [0, 1]^{30}
\end{equation}

Para conversión a formato de landmarks:

\begin{equation}
\vect{L} = \text{reshape}(\hat{\vect{L}}, (15, 2)) \cdot S
\end{equation}

donde $S = 224$ es el factor de escala.

% ==============================================================================
\section{Implementación}
% ==============================================================================

\subsection{Clase Principal: ResNet18Landmarks}

\begin{lstlisting}[caption={Implementación de la arquitectura}]
class ResNet18Landmarks(nn.Module):
    def __init__(self,
                 num_landmarks=15,
                 pretrained=True,
                 freeze_backbone=True,
                 dropout_rate=0.3,
                 hidden_dim=768,
                 use_coord_attention=True,
                 deep_head=True):
        super().__init__()

        # Cargar backbone ResNet-18
        resnet = models.resnet18(pretrained=pretrained)
        self.backbone = nn.Sequential(*list(resnet.children())[:-2])

        # Coordinate Attention (opcional)
        if use_coord_attention:
            self.coord_attn = CoordinateAttention(512, 512, reduction=32)
        else:
            self.coord_attn = nn.Identity()

        # Global Average Pooling
        self.gap = nn.AdaptiveAvgPool2d(1)

        # Cabeza de regresion profunda
        if deep_head:
            self.head = nn.Sequential(
                nn.Linear(512, 512),
                nn.GroupNorm(32, 512),
                nn.ReLU(inplace=True),
                nn.Dropout(dropout_rate),
                nn.Linear(512, hidden_dim),
                nn.GroupNorm(32, hidden_dim),
                nn.ReLU(inplace=True),
                nn.Dropout(dropout_rate),
                nn.Linear(hidden_dim, num_landmarks * 2),
                nn.Sigmoid()  # Salida normalizada [0, 1]
            )

        # Congelar backbone si se especifica
        if freeze_backbone:
            for param in self.backbone.parameters():
                param.requires_grad = False

    def forward(self, x):
        # Backbone: (B, 3, 224, 224) -> (B, 512, 7, 7)
        features = self.backbone(x)

        # Coordinate Attention
        features = self.coord_attn(features)

        # GAP: (B, 512, 7, 7) -> (B, 512)
        features = self.gap(features).flatten(1)

        # Head: (B, 512) -> (B, 30)
        landmarks = self.head(features)

        return landmarks
\end{lstlisting}

\subsection{Clase CoordinateAttention}

\begin{lstlisting}[caption={Implementación de Coordinate Attention}]
class CoordinateAttention(nn.Module):
    def __init__(self, in_channels, out_channels, reduction=32):
        super().__init__()
        mid_channels = max(8, in_channels // reduction)

        self.pool_h = nn.AdaptiveAvgPool2d((None, 1))
        self.pool_w = nn.AdaptiveAvgPool2d((1, None))

        self.conv1 = nn.Conv2d(in_channels, mid_channels, 1, bias=False)
        self.bn1 = nn.BatchNorm2d(mid_channels)
        self.act = nn.ReLU(inplace=True)

        self.conv_h = nn.Conv2d(mid_channels, out_channels, 1, bias=False)
        self.conv_w = nn.Conv2d(mid_channels, out_channels, 1, bias=False)

    def forward(self, x):
        identity = x
        n, c, h, w = x.size()

        # Pooling direccional
        x_h = self.pool_h(x)  # (n, c, h, 1)
        x_w = self.pool_w(x).permute(0, 1, 3, 2)  # (n, c, w, 1)

        # Concatenar y transformar
        y = torch.cat([x_h, x_w], dim=2)  # (n, c, h+w, 1)
        y = self.act(self.bn1(self.conv1(y)))

        # Dividir
        x_h, x_w = torch.split(y, [h, w], dim=2)
        x_w = x_w.permute(0, 1, 3, 2)

        # Generar pesos de atencion
        a_h = self.conv_h(x_h).sigmoid()
        a_w = self.conv_w(x_w).sigmoid()

        # Aplicar atencion
        out = identity * a_h * a_w

        return out
\end{lstlisting}

% ==============================================================================
\section{Figuras Sugeridas}
% ==============================================================================

\subsection{Figura 2.1: Arquitectura Completa del Modelo}

\textbf{Descripción}: Diagrama de bloques mostrando el flujo de datos desde la entrada hasta la salida.

\textbf{Elementos}:
\begin{itemize}
    \item Entrada: Imagen $3 \times 224 \times 224$
    \item ResNet-18 backbone con 4 stages
    \item Módulo Coordinate Attention
    \item Global Average Pooling
    \item Cabeza FC con dimensiones
    \item Salida: 30 valores $\in [0, 1]$
\end{itemize}

\subsection{Figura 2.2: Detalle del Módulo Coordinate Attention}

\textbf{Descripción}: Diagrama del flujo interno de Coordinate Attention.

\textbf{Elementos}:
\begin{itemize}
    \item Dos ramas de pooling (horizontal y vertical)
    \item Operación de concatenación
    \item Convolución $1 \times 1$ compartida
    \item División y convoluciones separadas
    \item Multiplicación elemento a elemento
\end{itemize}

\subsection{Figura 2.3: Estructura de la Cabeza de Regresión}

\textbf{Descripción}: Detalle de las capas de la cabeza profunda.

\textbf{Elementos}:
\begin{itemize}
    \item Secuencia de capas FC con dimensiones
    \item Bloques GroupNorm + ReLU + Dropout
    \item Activación Sigmoid final
\end{itemize}

\subsection{Figura 2.4: Bloque Residual de ResNet-18}

\textbf{Descripción}: Diagrama del bloque básico con skip connection.

\textbf{Elementos}:
\begin{itemize}
    \item Dos convoluciones $3 \times 3$
    \item Batch Normalization
    \item ReLU
    \item Conexión de identidad
    \item Suma elemento a elemento
\end{itemize}

% ==============================================================================
\section{Conclusiones}
% ==============================================================================

La arquitectura diseñada integra componentes de última generación para la tarea de regresión de landmarks:

\begin{enumerate}
    \item \textbf{Transfer Learning Efectivo}: ResNet-18 pre-entrenado proporciona características visuales robustas

    \item \textbf{Atención Espacial Precisa}: Coordinate Attention captura dependencias espaciales preservando información posicional

    \item \textbf{Regularización Apropiada}: GroupNorm y Dropout controlan overfitting en dataset pequeño

    \item \textbf{Salida Normalizada}: Facilita entrenamiento estable y uso de Wing Loss
\end{enumerate}

\begin{resultadoimportante}
La arquitectura final con \parametro{hidden\_dim=768}, \parametro{dropout=0.3}, y Coordinate Attention habilitado logró un error de \textbf{7.21 px} en validación, superando el objetivo de < 8 px.
\end{resultadoimportante}

% ==============================================================================
\section{Archivos de Referencia}
% ==============================================================================

\begin{table}[H]
\centering
\caption{Archivos fuente relacionados con este documento}
\begin{tabular}{ll}
\toprule
\textbf{Archivo} & \textbf{Descripción} \\
\midrule
\archivo{src\_v2/models/resnet\_landmark.py} & Implementación de ResNet18Landmarks \\
\archivo{src\_v2/models/\_\_init\_\_.py} & Exports del módulo models \\
\archivo{scripts/test\_forward\_pass.py} & Tests del forward pass \\
\archivo{SESSION\_LOG.md} & Registro de sesiones 2-3 \\
\bottomrule
\end{tabular}
\end{table}

% ==============================================================================
\bibliographystyle{ieeetr}
% \bibliography{referencias}
% ==============================================================================

\end{document}
