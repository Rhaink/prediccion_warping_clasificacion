% ==============================================================================
% DOCUMENTACIÓN CIENTÍFICA - ARQUITECTURA JERÁRQUICA (EXPLORACIÓN)
% Proyecto: Detección de COVID-19 mediante Landmarks Anatómicos
% Sesiones cubiertas: 13-14
% ==============================================================================

\documentclass[12pt,a4paper]{article}
% ==============================================================================
% PREÁMBULO LATEX PARA DOCUMENTACIÓN CIENTÍFICA DE NIVEL DOCTORAL
% Proyecto: Detección de COVID-19 mediante Landmarks Anatómicos y Warping Geométrico
% ==============================================================================

% --- Codificación y idioma ---
\usepackage[utf8]{inputenc}
\usepackage[T1]{fontenc}
\usepackage[spanish,es-tabla]{babel}

% --- Matemáticas ---
\usepackage{amsmath,amssymb,amsfonts,amsthm}
\usepackage{mathtools}
\usepackage{bm}  % Negritas en matemáticas

% --- Tablas profesionales ---
\usepackage{booktabs}
\usepackage{multirow}
\usepackage{array}
\usepackage{longtable}
\usepackage{tabularx}

% --- Figuras y gráficos ---
\usepackage{graphicx}
\usepackage{subfig}
\usepackage{float}
\usepackage{caption}

% --- Algoritmos y pseudocódigo ---
\usepackage{algorithm}
\usepackage{algorithmic}

% --- Colores (debe cargarse antes de listings) ---
\usepackage{xcolor}
\definecolor{covidred}{RGB}{220,53,69}
\definecolor{normalgreen}{RGB}{40,167,69}
\definecolor{viralblue}{RGB}{0,123,255}
\definecolor{codeblue}{RGB}{0,0,180}
\definecolor{codegray}{RGB}{128,128,128}

% --- Código fuente ---
\usepackage{listings}
\lstset{
    language=Python,
    basicstyle=\ttfamily\small,
    keywordstyle=\color{blue}\bfseries,
    commentstyle=\color{gray}\itshape,
    stringstyle=\color{red},
    numbers=left,
    numberstyle=\tiny\color{gray},
    stepnumber=1,
    numbersep=5pt,
    backgroundcolor=\color{white},
    frame=single,
    rulecolor=\color{black},
    tabsize=4,
    captionpos=b,
    breaklines=true,
    breakatwhitespace=false,
    showspaces=false,
    showstringspaces=false,
    showtabs=false,
    literate={á}{{\'a}}1 {é}{{\'e}}1 {í}{{\'i}}1 {ó}{{\'o}}1 {ú}{{\'u}}1
             {Á}{{\'A}}1 {É}{{\'E}}1 {Í}{{\'I}}1 {Ó}{{\'O}}1 {Ú}{{\'U}}1
             {ñ}{{\~n}}1 {Ñ}{{\~N}}1
             {ü}{{\"u}}1 {Ü}{{\"U}}1
             {¿}{{?`}}1 {¡}{{!`}}1
}

% --- Referencias y enlaces ---
\usepackage{hyperref}
\hypersetup{
    colorlinks=true,
    linkcolor=blue,
    filecolor=magenta,
    urlcolor=cyan,
    citecolor=blue,
    pdftitle={Documentación del Proyecto de Tesis},
    pdfauthor={},
}
\usepackage{cleveref}

% --- Bibliografía ---
\usepackage{natbib}
\bibliographystyle{plainnat}

% --- Geometría de página ---
\usepackage[a4paper,margin=2.5cm]{geometry}

% --- Espaciado ---
\usepackage{setspace}
\onehalfspacing

% --- Encabezados y pies ---
\usepackage{fancyhdr}
\pagestyle{fancy}
\fancyhf{}
\rhead{\rightmark}
\lhead{\leftmark}
\cfoot{\thepage}

% --- Entornos personalizados ---
\theoremstyle{definition}
\newtheorem{definicion}{Definición}[section]
\newtheorem{proposicion}{Proposición}[section]
\newtheorem{teorema}{Teorema}[section]
\newtheorem{lema}{Lema}[section]
\newtheorem{corolario}{Corolario}[section]

\theoremstyle{remark}
\newtheorem{observacion}{Observación}[section]
\newtheorem{nota}{Nota}[section]
\newtheorem{hipotesis}{Hipótesis}[section]

% --- Comandos personalizados ---
% Vectores y matrices
\newcommand{\vect}[1]{\mathbf{#1}}
\newcommand{\mat}[1]{\mathbf{#1}}

% Operadores
\DeclareMathOperator*{\argmin}{arg\,min}
\DeclareMathOperator*{\argmax}{arg\,max}
\DeclareMathOperator{\sgn}{sgn}
\DeclareMathOperator{\diag}{diag}
\DeclareMathOperator{\trace}{tr}

% Normas y productos
\newcommand{\norm}[1]{\left\|#1\right\|}
\newcommand{\abs}[1]{\left|#1\right|}
\newcommand{\inner}[2]{\langle #1, #2 \rangle}

% Conjuntos
\newcommand{\R}{\mathbb{R}}
\newcommand{\N}{\mathbb{N}}
\newcommand{\Z}{\mathbb{Z}}

% Espacios de imágenes y landmarks
\newcommand{\imgspace}{\mathcal{I}}
\newcommand{\landmarkspace}{\mathcal{L}}
\newcommand{\classspace}{\mathcal{Y}}

% Landmarks específicos
\newcommand{\landmark}[1]{L_{#1}}
\newcommand{\landmarkpair}[2]{(L_{#1}, L_{#2})}

% Funciones de pérdida
\newcommand{\loss}{\mathcal{L}}
\newcommand{\wingloss}{\mathcal{L}_{\text{wing}}}
\newcommand{\mseloss}{\mathcal{L}_{\text{MSE}}}

% Métricas
\newcommand{\accuracy}{\text{Acc}}
\newcommand{\precision}{\text{Prec}}
\newcommand{\recall}{\text{Rec}}
\newcommand{\fscore}{F_1}

% Abreviaciones
\newcommand{\ie}{\textit{i.e.}}
\newcommand{\eg}{\textit{e.g.}}
\newcommand{\etal}{\textit{et al.}}
\newcommand{\etc}{\textit{etc.}}

% Referencias a archivos del proyecto
\newcommand{\archivo}[1]{\texttt{#1}}
\newcommand{\funcion}[1]{\texttt{#1()}}
\newcommand{\clase}[1]{\texttt{#1}}
\newcommand{\parametro}[1]{\texttt{#1}}

% Unidades
\newcommand{\px}{\,\text{px}}
\newcommand{\epoch}{\,\text{época}}
\newcommand{\epochs}{\,\text{épocas}}

% --- Información del documento ---
\newcommand{\proyectotitulo}{Detección de COVID-19 en Radiografías de Tórax mediante Landmarks Anatómicos y Normalización Geométrica}
\newcommand{\proyectosubtitulo}{Documentación Científica del Proceso de Desarrollo}

% --- Formato de secciones ---
\usepackage{titlesec}
\titleformat{\section}
  {\normalfont\Large\bfseries}{\thesection}{1em}{}
\titleformat{\subsection}
  {\normalfont\large\bfseries}{\thesubsection}{1em}{}
\titleformat{\subsubsection}
  {\normalfont\normalsize\bfseries}{\thesubsubsection}{1em}{}

% --- Notas al margen para figuras sugeridas ---
\usepackage{marginnote}
\newcommand{\figurasugerida}[1]{\marginnote{\scriptsize\textcolor{blue}{[Figura: #1]}}}

% --- Cajas para resultados importantes ---
\usepackage{tcolorbox}
\newtcolorbox{resultadoimportante}[1][]{
    colback=green!5!white,
    colframe=green!75!black,
    fonttitle=\bfseries,
    title=Resultado Importante,
    #1
}

\newtcolorbox{hallazgo}[1][]{
    colback=blue!5!white,
    colframe=blue!75!black,
    fonttitle=\bfseries,
    title=Hallazgo,
    #1
}

\newtcolorbox{metodologia}[1][]{
    colback=yellow!5!white,
    colframe=yellow!75!black,
    fonttitle=\bfseries,
    title=Metodología,
    #1
}

\newtcolorbox{figuradescripcion}[1][]{
    colback=gray!5!white,
    colframe=gray!75!black,
    fonttitle=\bfseries,
    title=Descripción de Figura,
    #1
}

% ==============================================================================
% FIN DEL PREÁMBULO
% ==============================================================================


\title{Arquitectura Jerárquica para Predicción de Landmarks:\\Exploración y Análisis de Resultados Negativos}
\author{Documentación del Proceso de Desarrollo}
\date{Sesiones: 13-14}

\begin{document}
\maketitle

\begin{abstract}
Este documento presenta la exploración de una arquitectura jerárquica alternativa
para la predicción de landmarks anatómicos. Basada en los descubrimientos
geométricos del proceso de etiquetado (eje central L1-L2 y posiciones relativas),
la arquitectura propuesta predice primero el eje y luego los parámetros relativos
de cada landmark. Aunque la idea era prometedora, el modelo jerárquico corregido
logró solo 6.83 píxeles vs 3.71 del modelo directo. Se documentan los tres bugs
críticos encontrados (posiciones $t$ incorrectas, rango insuficiente, signos
invertidos), su corrección, y el análisis que explica por qué las restricciones
geométricas explícitas no superan al modelo directo que las aprende implícitamente.
Este es un resultado negativo científicamente valioso que valida el enfoque
de regresión directa.
\end{abstract}

\tableofcontents
\newpage

% ==============================================================================
\section{Introducción y Motivación}
% ==============================================================================

\subsection{Hipótesis de Partida}

Los descubrimientos geométricos de la Sesión 19 revelaron una estructura
altamente regular en el proceso de etiquetado manual:

\begin{itemize}
    \item El eje L1-L2 define una línea casi perfectamente vertical
    \item L9, L10, L11 dividen el eje en \textbf{exactamente 4 partes} ($t = 0.25, 0.50, 0.75$)
    \item Los landmarks bilaterales se ubican a distancias perpendiculares del eje
    \item El error de alineación de los centrales al eje es solo 1.3 px
\end{itemize}

\begin{quote}
\textit{Hipótesis}: Si el proceso de etiquetado sigue esta estructura paramétrica,
un modelo que la explote explícitamente debería superar al modelo de regresión
directa que debe aprender estas relaciones implícitamente.
\end{quote}

\subsection{Objetivo de la Arquitectura Jerárquica}

Diseñar un modelo que prediga landmarks en dos etapas:
\begin{enumerate}
    \item \textbf{Etapa 1}: Predecir el eje central (L1, L2) $\to$ 4 valores
    \item \textbf{Etapa 2}: Predecir parámetros relativos al eje $\to$ 18 valores
\end{enumerate}

Los landmarks finales se reconstruyen matemáticamente desde estos parámetros,
garantizando que las restricciones geométricas se satisfagan por construcción.

% ==============================================================================
\section{Diseño de la Arquitectura}
% ==============================================================================

\subsection{Parametrización de Landmarks}

La parametrización de landmarks relativa al eje se define como:

\begin{definicion}[Coordenadas del Eje]
El eje central se define por dos puntos:
\begin{equation}
\text{Eje} = \{L_1, L_2\} \in \R^4
\label{eq:axis_coords}
\end{equation}
donde $L_1 = (x_1, y_1)$ es el ápice superior y $L_2 = (x_2, y_2)$ es el ápice
inferior de la columna vertebral torácica.
\end{definicion}

\begin{definicion}[Parámetro $t$ (Posición en el Eje)]
Para un punto $P$ sobre el eje, su posición se parametriza como:
\begin{equation}
P(t) = L_1 + t \cdot (L_2 - L_1), \quad t \in [0, 1]
\label{eq:t_parameter}
\end{equation}
donde $t=0$ corresponde a $L_1$ y $t=1$ corresponde a $L_2$.
\end{definicion}

\begin{definicion}[Distancia Perpendicular]
Para landmarks no centrales, se define la distancia perpendicular al eje:
\begin{equation}
\vect{d} = (P - P_{base}) \cdot \vect{u}^\perp
\label{eq:perp_distance}
\end{equation}
donde $\vect{u}^\perp$ es el vector unitario perpendicular al eje.
\end{definicion}

\subsection{Salidas del Modelo}

\begin{table}[htbp]
\centering
\caption{Parametrización de la arquitectura jerárquica}
\label{tab:hierarchical_params}
\begin{tabular}{llcc}
\toprule
\textbf{Tipo} & \textbf{Landmarks} & \textbf{Parámetros} & \textbf{Cantidad} \\
\midrule
Eje & L1, L2 & $x_1, y_1, x_2, y_2$ & 4 \\
Centrales & L9, L10, L11 & $\Delta t_9, \Delta t_{10}, \Delta t_{11}$ & 3 \\
Bilaterales & 5 pares & $t_i, d_{left}, d_{right}$ & 15 \\
\midrule
& & \textbf{Total} & \textbf{22} \\
\bottomrule
\end{tabular}
\end{table}

\subsection{Arquitectura de Red}

\begin{lstlisting}[language=Python, caption={Estructura de HierarchicalLandmarkModel}]
class HierarchicalLandmarkModel(nn.Module):
    def __init__(self, hidden_dim=512, dropout=0.3):
        # Backbone compartido
        self.backbone = ResNet18_layers()  # Sin FC
        self.avgpool = nn.AdaptiveAvgPool2d((1, 1))

        # Cabeza para eje (L1, L2) -> 4 valores
        self.axis_head = nn.Sequential(
            nn.Linear(512, hidden_dim),
            nn.GroupNorm(32, hidden_dim),
            nn.ReLU(),
            nn.Dropout(dropout),
            nn.Linear(hidden_dim, 4),
            nn.Sigmoid()
        )

        # Cabeza para parametros relativos -> 18 valores
        # Input: features (512) + axis (4) = 516
        self.relative_head = nn.Sequential(
            nn.Linear(516, hidden_dim),
            nn.GroupNorm(32, hidden_dim),
            nn.ReLU(),
            nn.Dropout(dropout),
            nn.Linear(hidden_dim, hidden_dim // 2),
            nn.GroupNorm(16, hidden_dim // 2),
            nn.ReLU(),
            nn.Dropout(dropout),
            nn.Linear(hidden_dim // 2, 18)
        )
\end{lstlisting}

\subsection{Reconstrucción de Landmarks}

El proceso de reconstrucción convierte los 22 parámetros a 30 coordenadas:

\begin{algorithm}[H]
\caption{Reconstrucción de Landmarks desde Parámetros}
\label{alg:reconstruct}
\begin{algorithmic}[1]
\REQUIRE $L_1, L_2 \in \R^2$, params $\in \R^{18}$
\ENSURE landmarks $\in \R^{15 \times 2}$
\STATE $\vect{v}_{axis} \leftarrow L_2 - L_1$
\STATE $\ell \leftarrow \|\vect{v}_{axis}\|$
\STATE $\hat{\vect{u}} \leftarrow \vect{v}_{axis} / \ell$
\STATE $\hat{\vect{u}}^\perp \leftarrow (-\hat{u}_y, \hat{u}_x)$ \COMMENT{Rotación 90°}
\STATE landmarks[0] $\leftarrow L_1$
\STATE landmarks[1] $\leftarrow L_2$
\STATE \COMMENT{Landmarks centrales}
\FOR{$i \in \{8, 9, 10\}$}
    \STATE $t \leftarrow t_{base}[i] + \tanh(\Delta t_i) \cdot 0.1$
    \STATE landmarks[$i$] $\leftarrow L_1 + t \cdot \vect{v}_{axis}$
\ENDFOR
\STATE \COMMENT{Landmarks bilaterales}
\FORALL{par $(idx_L, idx_R)$ con parámetros $(t, d_L, d_R)$}
    \STATE $P_{base} \leftarrow L_1 + t \cdot \vect{v}_{axis}$
    \STATE landmarks[$idx_L$] $\leftarrow P_{base} + \sigma(d_L) \cdot 0.7 \cdot \hat{\vect{u}}^\perp \cdot \ell$
    \STATE landmarks[$idx_R$] $\leftarrow P_{base} - \sigma(d_R) \cdot 0.7 \cdot \hat{\vect{u}}^\perp \cdot \ell$
\ENDFOR
\RETURN landmarks
\end{algorithmic}
\end{algorithm}

% ==============================================================================
\section{Sesión 13: Implementación Inicial}
% ==============================================================================

\subsection{Entrenamiento}

El modelo jerárquico se entrenó con los mismos hiperparámetros que el modelo
directo:

\begin{table}[htbp]
\centering
\caption{Configuración de entrenamiento del modelo jerárquico}
\label{tab:hier_training}
\begin{tabular}{ll}
\toprule
\textbf{Parámetro} & \textbf{Valor} \\
\midrule
Épocas & 100 \\
Backbone LR & $2 \times 10^{-5}$ \\
Head LR & $2 \times 10^{-4}$ \\
Loss & Wing Loss \\
CLAHE & clip=2.0, tile=4 \\
Dropout & 0.3 \\
\bottomrule
\end{tabular}
\end{table}

\subsection{Resultado Inicial: Fallo Catastrófico}

\begin{table}[htbp]
\centering
\caption{Comparación inicial: jerárquico vs directo}
\label{tab:initial_failure}
\begin{tabular}{lcc}
\toprule
\textbf{Modelo} & \textbf{Error Test} & \textbf{Status} \\
\midrule
Regresión directa (ensemble) & 3.71 px & Óptimo \\
Jerárquico (inicial) & \textbf{46.6 px} & \textcolor{red}{Fallo} \\
\bottomrule
\end{tabular}
\end{table}

\begin{observacion}
El error de 46.6 px (12× peor que el modelo directo) indicaba la presencia
de bugs críticos en la implementación, no un fallo del concepto.
\end{observacion}

% ==============================================================================
\section{Sesión 14: Debugging y Corrección}
% ==============================================================================

\subsection{Bug 1: Posiciones $t$ Incorrectas para L12, L13}

\textbf{Problema}: Los valores $t_{base}$ para los pares bilaterales estaban
hardcodeados incorrectamente.

\begin{table}[htbp]
\centering
\caption{Corrección de posiciones $t$ para pares bilaterales}
\label{tab:t_correction}
\begin{tabular}{lccl}
\toprule
\textbf{Par} & \textbf{$t$ Incorrecto} & \textbf{$t$ Correcto} & \textbf{Ubicación Real} \\
\midrule
L3, L4 & 0.25 & 0.25 & OK \\
L5, L6 & 0.50 & 0.50 & OK \\
L7, L8 & 0.75 & 0.75 & OK \\
L12, L13 & \textcolor{red}{0.10} & \textcolor{green!50!black}{0.00} & En L1 \\
L14, L15 & \textcolor{red}{0.90} & \textcolor{green!50!black}{1.00} & En L2 \\
\bottomrule
\end{tabular}
\end{table}

\begin{lstlisting}[language=Python, caption={Corrección de bilateral\_t\_base}]
# ANTES (incorrecto)
bilateral_t_base = [0.25, 0.50, 0.75, 0.10, 0.90]

# DESPUES (correcto)
bilateral_t_base = [0.25, 0.50, 0.75, 0.00, 1.00]
\end{lstlisting}

\subsection{Bug 2: Rango de Distancias Perpendiculares Insuficiente}

\textbf{Problema}: Las distancias perpendiculares predichas estaban limitadas
a $[0, 0.5]$, pero las distancias reales en el ground truth alcanzan hasta 0.65.

\begin{equation}
d_{max}^{predicho} = \sigma(\cdot) \times 0.5 = 0.5 \quad < \quad d_{max}^{GT} = 0.65
\label{eq:range_bug}
\end{equation}

\begin{lstlisting}[language=Python, caption={Corrección del rango de distancias}]
# ANTES (insuficiente)
d_left = torch.sigmoid(params[...]) * 0.5

# DESPUES (correcto)
d_left = torch.sigmoid(params[...]) * 0.7
\end{lstlisting}

\subsection{Bug 3: Vector Perpendicular Invertido}

\textbf{Problema}: Los signos del vector perpendicular estaban invertidos,
causando que los landmarks izquierdos aparecieran a la derecha y viceversa.

\begin{equation}
\begin{aligned}
\text{Incorrecto: } & P_{left} = P_{base} - d \cdot \hat{\vect{u}}^\perp \\
\text{Correcto: } & P_{left} = P_{base} + d \cdot \hat{\vect{u}}^\perp
\end{aligned}
\label{eq:sign_bug}
\end{equation}

\begin{lstlisting}[language=Python, caption={Corrección de signos}]
# ANTES (invertido)
landmarks[:, left_idx] = base_point - d_left * perp_unit * axis_len
landmarks[:, right_idx] = base_point + d_right * perp_unit * axis_len

# DESPUES (correcto)
landmarks[:, left_idx] = base_point + d_left * perp_unit * axis_len
landmarks[:, right_idx] = base_point - d_right * perp_unit * axis_len
\end{lstlisting}

\subsection{Scripts de Debugging}

Se crearon tres scripts para diagnosticar los problemas:

\begin{table}[htbp]
\centering
\caption{Scripts de debugging creados}
\label{tab:debug_scripts}
\begin{tabular}{ll}
\toprule
\textbf{Script} & \textbf{Propósito} \\
\midrule
\archivo{debug\_hierarchical.py} & Analizar parámetros geométricos del GT \\
\archivo{test\_reconstruct.py} & Verificar reconstrucción con parámetros conocidos \\
\archivo{test\_hierarchical\_forward.py} & Test de forward pass del modelo \\
\bottomrule
\end{tabular}
\end{table}

% ==============================================================================
\section{Resultados Post-Corrección}
% ==============================================================================

\subsection{Mejora Dramática}

\begin{table}[htbp]
\centering
\caption{Resultado del modelo jerárquico corregido}
\label{tab:corrected_results}
\begin{tabular}{lccc}
\toprule
\textbf{Métrica} & \textbf{Antes (bug)} & \textbf{Después} & \textbf{Mejora} \\
\midrule
Error Test & 46.6 px & \textbf{6.83 px} & \textbf{7× mejor} \\
Normal & - & 5.98 px & - \\
COVID & - & 7.44 px & - \\
Viral & - & 8.01 px & - \\
\bottomrule
\end{tabular}
\end{table}

\subsection{Comparación Final}

\begin{table}[htbp]
\centering
\caption{Comparación final: jerárquico vs directo}
\label{tab:final_comparison}
\begin{tabular}{lccc}
\toprule
\textbf{Modelo} & \textbf{Error Test} & \textbf{vs Baseline} & \textbf{Resultado} \\
\midrule
\textbf{Regresión directa (ensemble 4)} & \textbf{3.71 px} & -59\% & \checkmark\checkmark \\
Jerárquico corregido & 6.83 px & -25\% & Funciona \\
\bottomrule
\end{tabular}
\end{table}

\begin{resultadoimportante}[title={El modelo directo supera al jerárquico}]
A pesar de que la arquitectura jerárquica explota la estructura geométrica
conocida, el modelo de regresión directa logra un error 45\% menor
(3.71 vs 6.83 px). Este es un \textbf{resultado negativo científicamente
valioso}.
\end{resultadoimportante}

% ==============================================================================
\section{Análisis: ¿Por qué el Modelo Directo es Mejor?}
% ==============================================================================

\subsection{Hipótesis 1: Restricciones Redundantes}

El modelo de regresión directa ya aprende implícitamente las restricciones
geométricas durante el entrenamiento, haciendo que las restricciones explícitas
del modelo jerárquico sean redundantes.

\begin{observacion}
El análisis de las predicciones del modelo directo muestra que:
\begin{itemize}
    \item L9, L10, L11 predichos están a solo 0.0025 de distancia del eje
    (mejor que el GT con 0.0044)
    \item La simetría se preserva dentro del margen esperado
    \item Las proporciones $t$ se mantienen aproximadamente
\end{itemize}
\end{observacion}

\subsection{Hipótesis 2: Propagación de Error}

En la arquitectura jerárquica, los errores se propagan:

\begin{equation}
\epsilon_{landmark} = f(\epsilon_{axis}, \epsilon_{params})
\label{eq:error_propagation}
\end{equation}

Un error en la predicción del eje (L1, L2) afecta a \textbf{todos} los demás
landmarks, amplificando el error total.

\subsection{Hipótesis 3: Capacidad de Representación}

El modelo directo tiene 30 grados de libertad independientes, mientras que
el jerárquico tiene solo 22 parámetros con restricciones. Esta limitación
puede impedir ajustar pequeñas desviaciones del ground truth que no siguen
exactamente la estructura paramétrica idealizada.

\subsection{Conclusión del Análisis}

\begin{hallazgo}[title={Valor del resultado negativo}]
Este experimento demuestra que:
\begin{enumerate}
    \item Las restricciones geométricas del etiquetado son aprendidas
    implícitamente por el modelo de regresión directa
    \item Imponer restricciones explícitas no mejora el rendimiento
    \item El modelo directo es más robusto a variaciones en el GT
    \item La simplicidad arquitectónica es preferible cuando funciona
\end{enumerate}
\end{hallazgo}

% ==============================================================================
\section{Lecciones Aprendidas}
% ==============================================================================

\subsection{Sobre Debugging de Modelos Geométricos}

\begin{enumerate}
    \item \textbf{Verificar parámetros contra GT}: Antes de entrenar,
    verificar que los parámetros hardcodeados corresponden al análisis real.

    \item \textbf{Probar reconstrucción con GT}: Usar parámetros extraídos
    del ground truth para verificar que la reconstrucción funciona.

    \item \textbf{Visualizar predicciones}: Los errores geométricos son
    evidentes visualmente (landmarks en posiciones incorrectas).
\end{enumerate}

\subsection{Sobre Diseño de Arquitecturas}

\begin{enumerate}
    \item \textbf{No asumir que restricciones mejoran}: Las restricciones
    explícitas pueden limitar la capacidad del modelo.

    \item \textbf{Preferir simplicidad}: Un modelo que funciona bien no
    necesita complicaciones adicionales.

    \item \textbf{Medir antes de optimizar}: Verificar que el problema
    existe antes de implementar soluciones complejas.
\end{enumerate}

% ==============================================================================
\section{Figuras Sugeridas}
% ==============================================================================

\subsection{Figura 8.1: Arquitectura Jerárquica}
\textit{Descripción}: Diagrama mostrando:
\begin{itemize}
    \item Backbone compartido
    \item Bifurcación a axis\_head y relative\_head
    \item Bloque de reconstrucción
    \item Salida de 30 coordenadas
\end{itemize}

\subsection{Figura 8.2: Parametrización Geométrica}
\textit{Descripción}: Diagrama anatómico mostrando:
\begin{itemize}
    \item Eje L1-L2 con parámetro $t$ marcado
    \item Landmarks centrales en $t = 0.25, 0.50, 0.75$
    \item Distancias perpendiculares $d_{left}$, $d_{right}$
    \item Vector perpendicular $\hat{u}^\perp$
\end{itemize}

\subsection{Figura 8.3: Comparación de Errores}
\textit{Descripción}: Gráfico de barras comparando:
\begin{itemize}
    \item Jerárquico con bugs: 46.6 px
    \item Jerárquico corregido: 6.83 px
    \item Regresión directa: 3.71 px
\end{itemize}

\subsection{Figura 8.4: Visualización de Bugs}
\textit{Descripción}: Grid de imágenes mostrando predicciones:
\begin{itemize}
    \item Con bug de $t$: L12, L13 desplazados
    \item Con bug de rango: landmarks ``aplastados''
    \item Con bug de signos: izquierda/derecha invertidos
    \item Corregido: landmarks correctos
\end{itemize}

% ==============================================================================
\section{Archivos Fuente}
% ==============================================================================

\begin{table}[htbp]
\centering
\caption{Archivos de implementación de arquitectura jerárquica}
\label{tab:source_files}
\begin{tabular}{ll}
\toprule
\textbf{Archivo} & \textbf{Contenido} \\
\midrule
\archivo{src\_v2/models/hierarchical.py} & HierarchicalLandmarkModel \\
\archivo{scripts/train\_hierarchical.py} & Script de entrenamiento \\
\archivo{scripts/debug\_hierarchical.py} & Análisis de parámetros GT \\
\archivo{scripts/test\_reconstruct.py} & Verificación de reconstrucción \\
\archivo{scripts/test\_hierarchical\_forward.py} & Test de forward pass \\
\bottomrule
\end{tabular}
\end{table}

% ==============================================================================
\section{Conclusiones}
% ==============================================================================

\begin{enumerate}
    \item \textbf{Arquitectura jerárquica funciona}: Tras corregir bugs,
    logra 6.83 px (25\% mejor que baseline).

    \item \textbf{Modelo directo es superior}: Con 3.71 px, supera al
    jerárquico por 45\%.

    \item \textbf{Restricciones geométricas son redundantes}: El modelo
    directo las aprende implícitamente.

    \item \textbf{Debugging requiere herramientas específicas}: Scripts
    para analizar parámetros y verificar reconstrucción son esenciales.

    \item \textbf{Resultado negativo es valioso}: Documenta que la
    complejidad adicional no garantiza mejora.

    \item \textbf{Bugs sutiles causan fallos catastróficos}: Errores de
    0.1 en $t$ o 0.2 en rango causaron 12× peor rendimiento.

    \item \textbf{Validación del enfoque simple}: La regresión directa
    con ensemble es la solución óptima para este problema.
\end{enumerate}

\end{document}
