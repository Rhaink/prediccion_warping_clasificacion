% ==============================================================================
% DOCUMENTACIÓN CIENTÍFICA - GENERACIÓN DEL DATASET WARPEADO
% Proyecto: Detección de COVID-19 mediante Landmarks Anatómicos
% Sesiones cubiertas: 21, 25
% Nivel: Doctoral/Científico - Completo y Detallado
% ==============================================================================

\documentclass[12pt,a4paper]{article}
% ==============================================================================
% PREÁMBULO LATEX PARA DOCUMENTACIÓN CIENTÍFICA DE NIVEL DOCTORAL
% Proyecto: Detección de COVID-19 mediante Landmarks Anatómicos y Warping Geométrico
% ==============================================================================

% --- Codificación y idioma ---
\usepackage[utf8]{inputenc}
\usepackage[T1]{fontenc}
\usepackage[spanish,es-tabla]{babel}

% --- Matemáticas ---
\usepackage{amsmath,amssymb,amsfonts,amsthm}
\usepackage{mathtools}
\usepackage{bm}  % Negritas en matemáticas

% --- Tablas profesionales ---
\usepackage{booktabs}
\usepackage{multirow}
\usepackage{array}
\usepackage{longtable}
\usepackage{tabularx}

% --- Figuras y gráficos ---
\usepackage{graphicx}
\usepackage{subfig}
\usepackage{float}
\usepackage{caption}

% --- Algoritmos y pseudocódigo ---
\usepackage{algorithm}
\usepackage{algorithmic}

% --- Colores (debe cargarse antes de listings) ---
\usepackage{xcolor}
\definecolor{covidred}{RGB}{220,53,69}
\definecolor{normalgreen}{RGB}{40,167,69}
\definecolor{viralblue}{RGB}{0,123,255}
\definecolor{codeblue}{RGB}{0,0,180}
\definecolor{codegray}{RGB}{128,128,128}

% --- Código fuente ---
\usepackage{listings}
\lstset{
    language=Python,
    basicstyle=\ttfamily\small,
    keywordstyle=\color{blue}\bfseries,
    commentstyle=\color{gray}\itshape,
    stringstyle=\color{red},
    numbers=left,
    numberstyle=\tiny\color{gray},
    stepnumber=1,
    numbersep=5pt,
    backgroundcolor=\color{white},
    frame=single,
    rulecolor=\color{black},
    tabsize=4,
    captionpos=b,
    breaklines=true,
    breakatwhitespace=false,
    showspaces=false,
    showstringspaces=false,
    showtabs=false,
    literate={á}{{\'a}}1 {é}{{\'e}}1 {í}{{\'i}}1 {ó}{{\'o}}1 {ú}{{\'u}}1
             {Á}{{\'A}}1 {É}{{\'E}}1 {Í}{{\'I}}1 {Ó}{{\'O}}1 {Ú}{{\'U}}1
             {ñ}{{\~n}}1 {Ñ}{{\~N}}1
             {ü}{{\"u}}1 {Ü}{{\"U}}1
             {¿}{{?`}}1 {¡}{{!`}}1
}

% --- Referencias y enlaces ---
\usepackage{hyperref}
\hypersetup{
    colorlinks=true,
    linkcolor=blue,
    filecolor=magenta,
    urlcolor=cyan,
    citecolor=blue,
    pdftitle={Documentación del Proyecto de Tesis},
    pdfauthor={},
}
\usepackage{cleveref}

% --- Bibliografía ---
\usepackage{natbib}
\bibliographystyle{plainnat}

% --- Geometría de página ---
\usepackage[a4paper,margin=2.5cm]{geometry}

% --- Espaciado ---
\usepackage{setspace}
\onehalfspacing

% --- Encabezados y pies ---
\usepackage{fancyhdr}
\pagestyle{fancy}
\fancyhf{}
\rhead{\rightmark}
\lhead{\leftmark}
\cfoot{\thepage}

% --- Entornos personalizados ---
\theoremstyle{definition}
\newtheorem{definicion}{Definición}[section]
\newtheorem{proposicion}{Proposición}[section]
\newtheorem{teorema}{Teorema}[section]
\newtheorem{lema}{Lema}[section]
\newtheorem{corolario}{Corolario}[section]

\theoremstyle{remark}
\newtheorem{observacion}{Observación}[section]
\newtheorem{nota}{Nota}[section]
\newtheorem{hipotesis}{Hipótesis}[section]

% --- Comandos personalizados ---
% Vectores y matrices
\newcommand{\vect}[1]{\mathbf{#1}}
\newcommand{\mat}[1]{\mathbf{#1}}

% Operadores
\DeclareMathOperator*{\argmin}{arg\,min}
\DeclareMathOperator*{\argmax}{arg\,max}
\DeclareMathOperator{\sgn}{sgn}
\DeclareMathOperator{\diag}{diag}
\DeclareMathOperator{\trace}{tr}

% Normas y productos
\newcommand{\norm}[1]{\left\|#1\right\|}
\newcommand{\abs}[1]{\left|#1\right|}
\newcommand{\inner}[2]{\langle #1, #2 \rangle}

% Conjuntos
\newcommand{\R}{\mathbb{R}}
\newcommand{\N}{\mathbb{N}}
\newcommand{\Z}{\mathbb{Z}}

% Espacios de imágenes y landmarks
\newcommand{\imgspace}{\mathcal{I}}
\newcommand{\landmarkspace}{\mathcal{L}}
\newcommand{\classspace}{\mathcal{Y}}

% Landmarks específicos
\newcommand{\landmark}[1]{L_{#1}}
\newcommand{\landmarkpair}[2]{(L_{#1}, L_{#2})}

% Funciones de pérdida
\newcommand{\loss}{\mathcal{L}}
\newcommand{\wingloss}{\mathcal{L}_{\text{wing}}}
\newcommand{\mseloss}{\mathcal{L}_{\text{MSE}}}

% Métricas
\newcommand{\accuracy}{\text{Acc}}
\newcommand{\precision}{\text{Prec}}
\newcommand{\recall}{\text{Rec}}
\newcommand{\fscore}{F_1}

% Abreviaciones
\newcommand{\ie}{\textit{i.e.}}
\newcommand{\eg}{\textit{e.g.}}
\newcommand{\etal}{\textit{et al.}}
\newcommand{\etc}{\textit{etc.}}

% Referencias a archivos del proyecto
\newcommand{\archivo}[1]{\texttt{#1}}
\newcommand{\funcion}[1]{\texttt{#1()}}
\newcommand{\clase}[1]{\texttt{#1}}
\newcommand{\parametro}[1]{\texttt{#1}}

% Unidades
\newcommand{\px}{\,\text{px}}
\newcommand{\epoch}{\,\text{época}}
\newcommand{\epochs}{\,\text{épocas}}

% --- Información del documento ---
\newcommand{\proyectotitulo}{Detección de COVID-19 en Radiografías de Tórax mediante Landmarks Anatómicos y Normalización Geométrica}
\newcommand{\proyectosubtitulo}{Documentación Científica del Proceso de Desarrollo}

% --- Formato de secciones ---
\usepackage{titlesec}
\titleformat{\section}
  {\normalfont\Large\bfseries}{\thesection}{1em}{}
\titleformat{\subsection}
  {\normalfont\large\bfseries}{\thesubsection}{1em}{}
\titleformat{\subsubsection}
  {\normalfont\normalsize\bfseries}{\thesubsubsection}{1em}{}

% --- Notas al margen para figuras sugeridas ---
\usepackage{marginnote}
\newcommand{\figurasugerida}[1]{\marginnote{\scriptsize\textcolor{blue}{[Figura: #1]}}}

% --- Cajas para resultados importantes ---
\usepackage{tcolorbox}
\newtcolorbox{resultadoimportante}[1][]{
    colback=green!5!white,
    colframe=green!75!black,
    fonttitle=\bfseries,
    title=Resultado Importante,
    #1
}

\newtcolorbox{hallazgo}[1][]{
    colback=blue!5!white,
    colframe=blue!75!black,
    fonttitle=\bfseries,
    title=Hallazgo,
    #1
}

\newtcolorbox{metodologia}[1][]{
    colback=yellow!5!white,
    colframe=yellow!75!black,
    fonttitle=\bfseries,
    title=Metodología,
    #1
}

\newtcolorbox{figuradescripcion}[1][]{
    colback=gray!5!white,
    colframe=gray!75!black,
    fonttitle=\bfseries,
    title=Descripción de Figura,
    #1
}

% ==============================================================================
% FIN DEL PREÁMBULO
% ==============================================================================


\title{Generación del Dataset Geométricamente Normalizado:\\
Pipeline de Transformación y Organización de Datos}
\author{Documentación del Proceso de Desarrollo}
\date{Sesiones: 21, 25}

\begin{document}
\maketitle

\begin{abstract}
Este documento describe el proceso de generación del dataset de radiografías
de tórax geométricamente normalizadas mediante warping piecewise affine.
Se detalla el pipeline de transformación que convierte las 957 imágenes
originales del dataset COVID-19 Radiography en imágenes normalizadas a una
forma canónica común. Se discuten dos enfoques: el warping de solo el área
pulmonar (18 triángulos de Delaunay, fill rate 47.1\%) y el warping con
cobertura completa (23 puntos incluyendo bordes, fill rate 96.1\%).
El documento presenta las decisiones de diseño para el uso de landmarks
ground truth en train/val versus landmarks predichos en test, la estructura
de organización compatible con frameworks de deep learning, y las métricas
de calidad del proceso de transformación. El dataset resultante preserva
la estratificación original (train: 717, val: 144, test: 96 imágenes) y
está preparado para entrenamiento y evaluación de clasificadores de
neumonía COVID-19 vs. viral vs. normal.
\end{abstract}

\tableofcontents
\newpage

% ==============================================================================
\section{Introducción y Objetivos}
% ==============================================================================

\subsection{Motivación del Dataset Warpeado}

El dataset warpeado tiene dos propósitos fundamentales en el proyecto:

\begin{enumerate}
    \item \textbf{Normalización geométrica}: Eliminar variabilidad anatómica
    inter-paciente para que el clasificador se enfoque en patrones patológicos
    (consolidaciones, opacidades, infiltrados) en lugar de diferencias de forma

    \item \textbf{Evaluación de hipótesis}: Permitir comparación empírica
    entre clasificadores entrenados en imágenes originales versus normalizadas,
    para cuantificar el beneficio de la normalización geométrica
\end{enumerate}

\subsection{Fuentes de Landmarks para Warping}

Un aspecto crítico del diseño es la fuente de landmarks para realizar el
warping:

\begin{table}[htbp]
\centering
\caption{Fuente de landmarks por split del dataset}
\label{tab:landmark_sources}
\begin{tabular}{lll}
\toprule
\textbf{Split} & \textbf{Fuente de Landmarks} & \textbf{Justificación} \\
\midrule
Train (717) & Ground Truth & Disponibles, máxima precisión \\
Val (144) & Ground Truth & Disponibles, consistencia con train \\
Test (96) & Predicciones del modelo & Simula escenario de inferencia \\
\bottomrule
\end{tabular}
\end{table}

\begin{hallazgo}[title={Decisión de diseño: GT para train/val, predicciones para test}]
Para train/val se usan landmarks ground truth porque:
\begin{enumerate}
    \item Están disponibles y son más precisos
    \item Eliminan ruido del modelo de landmarks durante entrenamiento
    \item El clasificador aprende de imágenes warpeadas óptimamente
\end{enumerate}
Para test se usan predicciones porque:
\begin{enumerate}
    \item Simula el escenario real de inferencia (sin GT disponible)
    \item Evalúa el pipeline completo end-to-end
    \item Los errores del modelo de landmarks (~3.79 px) se propagan al warping
\end{enumerate}
\end{hallazgo}

% ==============================================================================
\section{Estructura del Dataset Original}
% ==============================================================================

\subsection{Composición del Dataset}

El dataset COVID-19 Radiography Database contiene radiografías de tórax
clasificadas en tres categorías diagnósticas:

\begin{table}[htbp]
\centering
\caption{Distribución del dataset por categoría y split}
\label{tab:dataset_distribution}
\begin{tabular}{lcccc}
\toprule
\textbf{Categoría} & \textbf{Train} & \textbf{Val} & \textbf{Test} & \textbf{Total} \\
\midrule
Normal & 351 & 70 & 47 & 468 \\
COVID-19 & 229 & 46 & 31 & 306 \\
Viral Pneumonia & 137 & 28 & 18 & 183 \\
\midrule
\textbf{Total} & \textbf{717} & \textbf{144} & \textbf{96} & \textbf{957} \\
\textbf{Proporción} & 74.9\% & 15.0\% & 10.0\% & 100\% \\
\bottomrule
\end{tabular}
\end{table}

\begin{observacion}[Desbalance de Clases]
El dataset presenta desbalance significativo:
\begin{itemize}
    \item Normal: 48.8\% (clase mayoritaria)
    \item COVID-19: 32.0\% (clase objetivo principal)
    \item Viral Pneumonia: 19.2\% (clase minoritaria)
\end{itemize}
Este desbalance debe considerarse durante el entrenamiento del clasificador
mediante técnicas como class weights o oversampling.
\end{observacion}

\subsection{Especificaciones de Imagen}

\begin{table}[htbp]
\centering
\caption{Especificaciones técnicas de las imágenes}
\label{tab:image_specs}
\begin{tabular}{ll}
\toprule
\textbf{Parámetro} & \textbf{Valor} \\
\midrule
Tamaño original & 299 $\times$ 299 píxeles \\
Tamaño procesado & 224 $\times$ 224 píxeles \\
Profundidad de color & 8 bits (escala de grises) \\
Formato & PNG \\
Canales & 1 (grayscale) \\
Rango de valores & [0, 255] \\
\bottomrule
\end{tabular}
\end{table}

\subsection{Landmarks Ground Truth}

Cada imagen tiene asociados 15 landmarks anatómicos anotados manualmente:

\begin{table}[htbp]
\centering
\caption{Resumen de landmarks anatómicos}
\label{tab:landmarks_summary}
\begin{tabular}{lllc}
\toprule
\textbf{ID} & \textbf{Nombre} & \textbf{Región} & \textbf{Variabilidad ($\sigma$)} \\
\midrule
L1 & Superior & Mediastino sup. & Baja \\
L2 & Inferior & Vértebra inf. & Baja \\
L3, L4 & Ápices & Pulmonar sup. & Media \\
L5, L6 & Hilios & Pulmonar medio & Media \\
L7, L8 & Bases & Pulmonar inf. & Media \\
L9, L10, L11 & Centrales & Eje medial & Muy baja \\
L12, L13 & Bordes sup. & Clavículas & Alta \\
L14, L15 & Costofrénicos & Ángulos inf. & Muy alta \\
\bottomrule
\end{tabular}
\end{table}

% ==============================================================================
\section{Pipeline de Generación del Dataset Warpeado}
% ==============================================================================

\subsection{Diagrama de Flujo}

El proceso de generación del dataset warpeado sigue el siguiente pipeline:

\begin{enumerate}
    \item \textbf{Carga de datos}:
    \begin{itemize}
        \item Forma canónica desde \texttt{canonical\_shape\_gpa.json}
        \item Triángulos de Delaunay desde \texttt{canonical\_delaunay\_triangles.json}
        \item Landmarks GT desde \texttt{all\_landmarks.npz}
        \item Predicciones (para test) desde \texttt{test\_predictions.npz}
    \end{itemize}

    \item \textbf{Iteración por split} (train, val, test):
    \begin{itemize}
        \item Seleccionar fuente de landmarks (GT o predicciones)
        \item Procesar cada imagen del split
    \end{itemize}

    \item \textbf{Por cada imagen}:
    \begin{itemize}
        \item Cargar imagen original (grayscale)
        \item Redimensionar a 224 $\times$ 224 si necesario
        \item Aplicar warping piecewise affine
        \item Calcular métricas de calidad (fill rate)
        \item Guardar imagen warpeada
    \end{itemize}

    \item \textbf{Post-procesamiento}:
    \begin{itemize}
        \item Generar estadísticas globales
        \item Crear visualizaciones de verificación
        \item Guardar configuración del proceso
    \end{itemize}
\end{enumerate}

\subsection{Algoritmo de Generación}

\begin{algorithm}[H]
\caption{Generación del Dataset Warpeado}
\label{alg:generate_warped_dataset}
\begin{algorithmic}[1]
\REQUIRE $\mathcal{D} = \{(I_i, L_i, c_i, s_i)\}_{i=1}^{957}$: dataset original, $L_{\text{can}}$: forma canónica, $\mathcal{T}$: triángulos de Delaunay
\ENSURE $\mathcal{D}_{\text{warped}} = \{(I'_i, c_i, s_i)\}_{i=1}^{957}$: dataset warpeado
\STATE \COMMENT{Crear estructura de directorios}
\FORALL{$s \in \{\text{train}, \text{val}, \text{test}\}$}
    \FORALL{$c \in \{\text{Normal}, \text{COVID}, \text{Viral}\}$}
        \STATE CrearDirectorio($\text{output}/s/c$)
    \ENDFOR
\ENDFOR
\STATE \COMMENT{Procesar cada split}
\FORALL{split $s$ en $\{\text{train}, \text{val}, \text{test}\}$}
    \STATE \COMMENT{Seleccionar fuente de landmarks}
    \IF{$s \in \{\text{train}, \text{val}\}$}
        \STATE $L_{\text{source}} \leftarrow L_{\text{GT}}$
    \ELSE
        \STATE $L_{\text{source}} \leftarrow L_{\text{pred}}$
    \ENDIF
    \STATE \COMMENT{Procesar imágenes del split}
    \FORALL{$(I_i, L_i, c_i)$ en split $s$}
        \STATE $I_i \leftarrow$ Cargar($I_i$)
        \IF{$\text{size}(I_i) \neq 224$}
            \STATE $I_i \leftarrow$ Redimensionar($I_i$, $224 \times 224$)
        \ENDIF
        \STATE \COMMENT{Aplicar warping}
        \STATE $I'_i \leftarrow$ PiecewiseAffineWarp($I_i$, $L_i$, $L_{\text{can}}$, $\mathcal{T}$)
        \STATE \COMMENT{Calcular métricas}
        \STATE $\text{fill\_rate}_i \leftarrow 1 - \frac{\#\{p : I'_i(p) = 0\}}{224^2}$
        \STATE \COMMENT{Guardar}
        \STATE Guardar($I'_i$, $\text{output}/s/c_i/\text{nombre}_i$\_warped.png)
    \ENDFOR
\ENDFOR
\RETURN $\mathcal{D}_{\text{warped}}$, estadísticas
\end{algorithmic}
\end{algorithm}

\subsection{Dos Modalidades de Warping}

\subsubsection{Modalidad 1: Solo Área Pulmonar}

En esta modalidad, se usa únicamente la triangulación de Delaunay sobre
los 15 landmarks anatómicos (18 triángulos), sin puntos de borde adicionales.

\begin{table}[htbp]
\centering
\caption{Características del warping de solo área pulmonar}
\label{tab:lung_only_warp}
\begin{tabular}{ll}
\toprule
\textbf{Parámetro} & \textbf{Valor} \\
\midrule
Puntos de control & 15 (landmarks anatómicos) \\
Número de triángulos & 18 \\
Fill rate promedio & 47.1\% \\
Área warpeada & Región pulmonar central \\
Área no warpeada & Bordes (negro) \\
use\_full\_coverage & False \\
\bottomrule
\end{tabular}
\end{table}

\textbf{Ventajas}:
\begin{itemize}
    \item Focaliza la atención del modelo en el área clínicamente relevante
    \item Elimina información irrelevante (fondo, clavículas periféricas)
    \item Menor tamaño de datos efectivo para procesar
\end{itemize}

\textbf{Desventajas}:
\begin{itemize}
    \item Pérdida de información anatómica periférica
    \item Posible pérdida de ángulos costofrénicos completos
    \item Fondo negro puede introducir artefactos en el aprendizaje
\end{itemize}

\subsubsection{Modalidad 2: Cobertura Completa}

En esta modalidad, se añaden 8 puntos de borde (4 esquinas + 4 puntos medios)
para garantizar cobertura casi total de la imagen.

\begin{table}[htbp]
\centering
\caption{Características del warping con cobertura completa}
\label{tab:full_coverage_warp}
\begin{tabular}{ll}
\toprule
\textbf{Parámetro} & \textbf{Valor} \\
\midrule
Puntos de control & 23 (15 landmarks + 8 borde) \\
Número de triángulos & $\sim$34 (variable) \\
Fill rate promedio & 96.1\% \\
Área warpeada & Imagen casi completa \\
Área no warpeada & Mínima ($\sim$4\%) \\
use\_full\_coverage & True \\
\bottomrule
\end{tabular}
\end{table}

\textbf{Ventajas}:
\begin{itemize}
    \item Preserva toda la información anatómica
    \item Evita artefactos de bordes negros
    \item Compatible con CNNs que esperan imágenes completas
\end{itemize}

\textbf{Desventajas}:
\begin{itemize}
    \item Los bordes se preservan sin deformación (puntos fijos)
    \item Mayor carga computacional
\end{itemize}

% ==============================================================================
\section{Métricas de Calidad del Warping}
% ==============================================================================

\subsection{Fill Rate}

\begin{definicion}[Fill Rate]
El fill rate cuantifica la proporción de la imagen destino que recibe
contenido válido del warping:
\begin{equation}
\text{Fill Rate} = 1 - \frac{N_{\text{negro}}}{N_{\text{total}}}
= 1 - \frac{\sum_{i,j} \mathbb{1}[I'(i,j) = 0]}{224 \times 224}
\label{eq:fill_rate_def}
\end{equation}
donde $N_{\text{negro}}$ es el número de píxeles con valor 0 y
$N_{\text{total}} = 224^2 = 50,176$ es el total de píxeles.
\end{definicion}

\subsection{Resultados de Fill Rate por Modalidad}

\begin{table}[htbp]
\centering
\caption{Estadísticas de fill rate por modalidad de warping}
\label{tab:fill_rate_comparison}
\begin{tabular}{lcccc}
\toprule
\textbf{Modalidad} & \textbf{Media (\%)} & \textbf{Std (\%)} & \textbf{Mín (\%)} & \textbf{Máx (\%)} \\
\midrule
Solo área pulmonar & 47.1 & 3.2 & 38.9 & 54.7 \\
Cobertura completa & 96.1 & 1.4 & 93.2 & 98.7 \\
\bottomrule
\end{tabular}
\end{table}

\subsection{Fill Rate por Categoría (Modalidad Solo Pulmonar)}

\begin{table}[htbp]
\centering
\caption{Fill rate por categoría diagnóstica}
\label{tab:fill_rate_by_category}
\begin{tabular}{lccc}
\toprule
\textbf{Categoría} & \textbf{$n$} & \textbf{Fill Rate (\%)} & \textbf{Std (\%)} \\
\midrule
Normal & 468 & 47.3 & 3.0 \\
COVID-19 & 306 & 46.8 & 3.4 \\
Viral Pneumonia & 183 & 47.0 & 3.3 \\
\midrule
\textbf{Total} & \textbf{957} & \textbf{47.1} & \textbf{3.2} \\
\bottomrule
\end{tabular}
\end{table}

\begin{observacion}[Uniformidad del Fill Rate]
El fill rate es estadísticamente indistinguible entre categorías,
confirmando que:
\begin{enumerate}
    \item La forma anatómica no difiere significativamente entre categorías
    (consistente con análisis GPA)
    \item El warping no introduce sesgo por categoría
    \item Las patologías no afectan la geometría detectable
\end{enumerate}
\end{observacion}

% ==============================================================================
\section{Estructura del Dataset Resultante}
% ==============================================================================

\subsection{Organización de Directorios}

El dataset warpeado se organiza siguiendo la convención de ImageFolder
de PyTorch/torchvision:

\begin{verbatim}
outputs/warped_dataset/
├── train/
│   ├── Normal/
│   │   ├── image_001_warped.png
│   │   ├── image_002_warped.png
│   │   └── ... (351 imágenes)
│   ├── COVID/
│   │   └── ... (229 imágenes)
│   └── Viral_Pneumonia/
│       └── ... (137 imágenes)
├── val/
│   ├── Normal/
│   │   └── ... (70 imágenes)
│   ├── COVID/
│   │   └── ... (46 imágenes)
│   └── Viral_Pneumonia/
│       └── ... (28 imágenes)
└── test/
    ├── Normal/
    │   └── ... (47 imágenes)
    ├── COVID/
    │   └── ... (31 imágenes)
    └── Viral_Pneumonia/
        └── ... (18 imágenes)
\end{verbatim}

\subsection{Convención de Nomenclatura}

Las imágenes warpeadas siguen la convención:
\begin{equation}
\texttt{<nombre\_original>\_warped.png}
\end{equation}

Por ejemplo: \texttt{Normal-1234\_warped.png}

\subsection{Metadatos del Proceso}

Se genera un archivo de configuración JSON con metadatos del proceso:

\begin{verbatim}
{
    "canonical_shape_file": "canonical_shape_gpa.json",
    "triangles_file": "canonical_delaunay_triangles.json",
    "num_triangles": 18,
    "num_landmarks": 15,
    "image_size": 224,
    "interpolation": "bilinear",
    "border_mode": "reflect_101",
    "fill_rate_mean": 0.471,
    "fill_rate_std": 0.032,
    "images_processed": 957,
    "source_gt_splits": ["train", "val"],
    "source_pred_splits": ["test"],
    "date": "2024-11-28",
    "session": 21
}
\end{verbatim}

% ==============================================================================
\section{Verificación y Validación}
% ==============================================================================

\subsection{Verificación Visual}

Se generan imágenes de verificación para confirmar el correcto funcionamiento
del warping:

\begin{enumerate}
    \item \textbf{Grilla de comparación}: Matriz mostrando original vs.
    warped para muestras representativas de cada categoría

    \item \textbf{Superposición de landmarks}: Imágenes warpeadas con
    landmarks canónicos superpuestos para verificar alineación

    \item \textbf{Verificación de máscara}: Visualización del área cubierta
    por los triángulos de Delaunay
\end{enumerate}

\subsection{Bug Corregido: Escalado de Landmarks}

Durante el desarrollo se identificó y corrigió un bug importante:

\begin{problema}[Bug de Escalado]
Inicialmente, el script aplicaba un factor de escalado $224/299$ a los
landmarks, asumiendo que estaban en escala 299 (tamaño de imagen original).
Sin embargo, los landmarks en \texttt{all\_landmarks.npz} ya estaban
normalizados a escala 224 durante la extracción.

\textbf{Síntoma}: Landmarks desplazados hacia el centro de la imagen.

\textbf{Solución}: Eliminar el escalado innecesario, usar landmarks
directamente.

\textbf{Verificación}: Imágenes \texttt{verification.png} y
\texttt{mask\_verification.png} confirman warping correcto post-corrección.
\end{problema}

\subsection{Validación Estadística}

\begin{enumerate}
    \item \textbf{Conteo de imágenes}: Verificar que el número de imágenes
    generadas coincide con el esperado (957 total)

    \item \textbf{Distribución de categorías}: Confirmar que se preserva
    la estratificación original

    \item \textbf{Integridad de archivos}: Verificar que todas las imágenes
    se guardan correctamente y son legibles

    \item \textbf{Rango de valores}: Confirmar que los píxeles están en
    rango [0, 255]
\end{enumerate}

% ==============================================================================
\section{Consideraciones para Entrenamiento}
% ==============================================================================

\subsection{Class Weights para Desbalance}

Dado el desbalance de clases, se recomienda usar class weights durante
el entrenamiento:

\begin{equation}
w_c = \frac{N_{\text{total}}}{k \cdot N_c}
\end{equation}

donde $N_c$ es el número de muestras de la clase $c$ y $k$ es el número
de clases.

\begin{table}[htbp]
\centering
\caption{Class weights recomendados}
\label{tab:class_weights}
\begin{tabular}{lcccc}
\toprule
\textbf{Clase} & \textbf{$N_c$ (train)} & \textbf{Proporción} & \textbf{Weight} \\
\midrule
Normal & 350 & 48.8\% & 0.68 \\
COVID & 229 & 31.9\% & 1.04 \\
Viral & 138 & 19.2\% & 1.73 \\
\bottomrule
\end{tabular}
\end{table}

\subsection{Data Augmentation}

Las siguientes transformaciones son seguras para imágenes warpeadas:

\begin{itemize}
    \item \textbf{Flip horizontal}: Aplicable con cuidado (la forma canónica
    ya está normalizada)
    \item \textbf{Rotación pequeña}: $\pm 5°$ para variabilidad
    \item \textbf{Jitter de brillo/contraste}: Preserva estructura geométrica
    \item \textbf{Normalización}: Necesaria para transfer learning
\end{itemize}

\textbf{Precaución}: Evitar transformaciones geométricas agresivas que
contrarresten la normalización aplicada.

\subsection{Compatibilidad con Frameworks}

El dataset es directamente compatible con:

\begin{itemize}
    \item \textbf{PyTorch}: \texttt{torchvision.datasets.ImageFolder}
    \item \textbf{TensorFlow/Keras}: \texttt{tf.keras.preprocessing.image\_dataset\_from\_directory}
    \item \textbf{Cualquier framework}: Estructura estándar de directorios
    por clase
\end{itemize}

% ==============================================================================
\section{Figuras Sugeridas}
% ==============================================================================

\subsection{Figura 12.1: Pipeline de Generación}

\begin{figuradescripcion}
\textbf{Título}: Pipeline de generación del dataset warpeado

\textbf{Contenido}: Diagrama de flujo mostrando las etapas del proceso:
\begin{enumerate}
    \item Dataset original (957 imágenes)
    \item Carga de forma canónica y triángulos
    \item Selección de fuente de landmarks (GT vs. predicciones)
    \item Warping piecewise affine
    \item Dataset warpeado organizado por splits y categorías
\end{enumerate}

\textbf{Elementos visuales}:
\begin{itemize}
    \item Cajas con iconos representando datos
    \item Flechas indicando flujo
    \item Bifurcación para train/val (GT) vs. test (predicciones)
\end{itemize}
\end{figuradescripcion}

\subsection{Figura 12.2: Comparación de Modalidades de Warping}

\begin{figuradescripcion}
\textbf{Título}: Comparación: solo área pulmonar vs. cobertura completa

\textbf{Contenido}: Matriz de 2 filas $\times$ 4 columnas:
\begin{itemize}
    \item Fila 1: Warping solo área pulmonar (fill rate 47\%)
    \item Fila 2: Warping cobertura completa (fill rate 96\%)
    \item Columnas: Original, Máscara, Warped, Diferencia
\end{itemize}

\textbf{Elementos visuales}:
\begin{itemize}
    \item Áreas negras claramente visibles en fila 1
    \item Cobertura casi completa en fila 2
    \item Anotaciones con fill rate
\end{itemize}
\end{figuradescripcion}

\subsection{Figura 12.3: Grilla de Ejemplos Warpeados}

\begin{figuradescripcion}
\textbf{Título}: Ejemplos de imágenes warpeadas por categoría

\textbf{Contenido}: Grilla de 3 filas (categorías) $\times$ 6 columnas (ejemplos):
\begin{itemize}
    \item Fila 1: Normal (6 ejemplos)
    \item Fila 2: COVID (6 ejemplos)
    \item Fila 3: Viral Pneumonia (6 ejemplos)
\end{itemize}

\textbf{Observación clave}: Todas las imágenes tienen la misma forma geométrica
(forma canónica) pero diferente textura/patología.
\end{figuradescripcion}

\subsection{Figura 12.4: Distribución de Fill Rates}

\begin{figuradescripcion}
\textbf{Título}: Distribución de fill rates en el dataset

\textbf{Contenido}: Histograma de fill rates para las 957 imágenes.

\textbf{Elementos visuales}:
\begin{itemize}
    \item Barras del histograma (20 bins)
    \item Línea vertical en la media
    \item Histogramas superpuestos por categoría (diferentes colores)
\end{itemize}

\textbf{Anotaciones}: Media, std, rango
\end{figuradescripcion}

\subsection{Figura 12.5: Estructura de Directorios}

\begin{figuradescripcion}
\textbf{Título}: Organización del dataset warpeado

\textbf{Contenido}: Diagrama de árbol de directorios con conteo de imágenes.

\textbf{Formato}: Representación visual del filesystem mostrando:
\begin{itemize}
    \item Directorio raíz \texttt{warped\_dataset/}
    \item Subdirectorios por split
    \item Subdirectorios por categoría
    \item Conteo de archivos en cada hoja
\end{itemize}
\end{figuradescripcion}

% ==============================================================================
\section{Archivos Fuente y Reproducibilidad}
% ==============================================================================

\begin{table}[htbp]
\centering
\caption{Archivos de implementación del pipeline de generación}
\label{tab:source_files_generation}
\begin{tabular}{p{5.5cm}p{7.5cm}}
\toprule
\textbf{Archivo} & \textbf{Contenido} \\
\midrule
\archivo{scripts/generate\_warped\_dataset.py} & Script principal de generación:\\
& \quad - \texttt{load\_all\_landmarks()}: Carga GT completo\\
& \quad - \texttt{process\_split()}: Procesa un split\\
& \quad - \texttt{generate\_summary\_statistics()}: Métricas\\
\midrule
\archivo{outputs/predictions/\\all\_landmarks.npz} & Landmarks GT de 957 imágenes:\\
& \quad - train\_landmarks: (717, 15, 2)\\
& \quad - val\_landmarks: (144, 15, 2)\\
& \quad - test\_landmarks: (96, 15, 2)\\
\midrule
\archivo{outputs/predictions/\\test\_predictions.npz} & Predicciones del ensemble (96 test)\\
\midrule
\archivo{outputs/warped\_dataset/} & Dataset warpeado resultante:\\
& \quad - train/, val/, test/\\
& \quad - warp\_config.json: Configuración\\
& \quad - statistics.json: Métricas globales\\
\bottomrule
\end{tabular}
\end{table}

% ==============================================================================
\section{Discusión}
% ==============================================================================

\subsection{Elección de Modalidad}

Para el proyecto se optó por la modalidad de \textbf{solo área pulmonar}
porque:

\begin{enumerate}
    \item \textbf{Focalización clínica}: Las patologías de interés (COVID,
    neumonía viral) se manifiestan principalmente en el parénquima pulmonar

    \item \textbf{Reducción de ruido}: Elimina información irrelevante que
    podría confundir al clasificador

    \item \textbf{Evaluación de hipótesis}: El fondo negro permite evaluar
    si el clasificador aprende de características genuinas vs. artefactos
\end{enumerate}

\subsection{Impacto del Error de Landmarks en Test}

Las predicciones del modelo de landmarks tienen un error de $\sim$3.79 px.
Este error se propaga al warping de las imágenes de test:

\begin{itemize}
    \item La forma canónica no se alcanza perfectamente
    \item Existe variabilidad residual en las imágenes warpeadas de test
    \item Sin embargo, esta variabilidad simula el escenario real de inferencia
\end{itemize}

\begin{hallazgo}[title={El pipeline completo es evaluable en test}]
Usar predicciones para test (en lugar de GT) permite evaluar el rendimiento
del sistema completo: modelo de landmarks + warping + clasificador. Esta
es la métrica relevante para aplicación práctica.
\end{hallazgo}

\subsection{Limitaciones}

\begin{enumerate}
    \item \textbf{Dependencia de GT para train/val}: El clasificador se
    entrena con warping ``perfecto'' pero se evalúa con warping ``imperfecto''

    \item \textbf{Fill rate variable}: El 47.1\% de fill rate introduce
    áreas negras que podrían afectar features aprendidas

    \item \textbf{Información perdida}: Los ángulos costofrénicos no están
    completamente representados en algunos casos
\end{enumerate}

% ==============================================================================
\section{Conclusiones}
% ==============================================================================

\begin{enumerate}
    \item \textbf{Dataset warpeado generado exitosamente}: Las 957 imágenes
    fueron procesadas y organizadas en estructura compatible con frameworks
    de deep learning.

    \item \textbf{Estratificación preservada}: La distribución train/val/test
    y la proporción de categorías se mantienen idénticas al dataset original.

    \item \textbf{Decisión de diseño justificada}: Usar GT para train/val y
    predicciones para test permite entrenar con warping óptimo mientras se
    evalúa el pipeline completo.

    \item \textbf{Fill rate consistente}: El fill rate de 47.1\% es uniforme
    entre categorías, confirmando que el warping no introduce sesgo.

    \item \textbf{Bug de escalado corregido}: El descubrimiento y corrección
    del bug de escalado fue crítico para la correcta generación del dataset.

    \item \textbf{Dataset listo para clasificación}: La estructura de
    directorios permite uso inmediato con \texttt{ImageFolder} y frameworks
    estándar de deep learning.
\end{enumerate}

\end{document}
