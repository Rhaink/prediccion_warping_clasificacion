% =============================================================================
% PREAMBLE.TEX - Configuracion de Paquetes para Tesis de Maestria
% =============================================================================
% Prediccion de Landmarks Anatomicos en Radiografias de Torax mediante Deep Learning
% Autor: [Nombre del Autor]
% Universidad: [Nombre de la Universidad]
% Maestria en Ingenieria Electronica
% =============================================================================

% -----------------------------------------------------------------------------
% CODIFICACION Y LENGUAJE
% -----------------------------------------------------------------------------
\usepackage[utf8]{inputenc}
\usepackage[T1]{fontenc}
\usepackage[spanish,es-tabla]{babel}

% -----------------------------------------------------------------------------
% GEOMETRIA Y MARGENES
% -----------------------------------------------------------------------------
\usepackage[
    a4paper,
    top=2.5cm,
    bottom=2.5cm,
    left=3cm,
    right=2.5cm,
    headheight=14pt
]{geometry}

% -----------------------------------------------------------------------------
% TIPOGRAFIA
% -----------------------------------------------------------------------------
\usepackage{lmodern}           % Fuente Latin Modern
\usepackage{microtype}         % Mejoras tipograficas
\usepackage{setspace}          % Espaciado entre lineas
\onehalfspacing                % Espaciado 1.5

% -----------------------------------------------------------------------------
% MATEMATICAS
% -----------------------------------------------------------------------------
\usepackage{amsmath}
\usepackage{amsfonts}
\usepackage{amssymb}
\usepackage{amsthm}
\usepackage{mathtools}
\usepackage{bm}                % Negritas en matematicas

% Definiciones de teoremas
\theoremstyle{definition}
\newtheorem{definicion}{Definicion}[chapter]
\newtheorem{ejemplo}{Ejemplo}[chapter]

\theoremstyle{plain}
\newtheorem{teorema}{Teorema}[chapter]
\newtheorem{lema}{Lema}[chapter]
\newtheorem{proposicion}{Proposicion}[chapter]
\newtheorem{corolario}{Corolario}[chapter]

% -----------------------------------------------------------------------------
% GRAFICOS E IMAGENES
% -----------------------------------------------------------------------------
\usepackage{graphicx}
\graphicspath{{figures/}{../outputs/diagrams/}{../outputs/thesis_figures/}}
\usepackage{float}
\usepackage{subcaption}        % Subfiguras
\usepackage{wrapfig}           % Figuras flotantes

% Configuracion de captions
\usepackage[
    font=small,
    labelfont=bf,
    textfont=it,
    format=hang,
    margin=1cm
]{caption}

% -----------------------------------------------------------------------------
% TABLAS
% -----------------------------------------------------------------------------
\usepackage{booktabs}          % Tablas profesionales
\usepackage{multirow}          % Celdas multirow
\usepackage{array}             % Extensiones de tablas
\usepackage{longtable}         % Tablas largas
\usepackage{tabularx}          % Tablas con ancho ajustable
\usepackage{colortbl}          % Color en tablas

% Nuevo tipo de columna centrada con ancho fijo
\newcolumntype{C}[1]{>{\centering\arraybackslash}p{#1}}
\newcolumntype{L}[1]{>{\raggedright\arraybackslash}p{#1}}
\newcolumntype{R}[1]{>{\raggedleft\arraybackslash}p{#1}}

% -----------------------------------------------------------------------------
% COLORES
% -----------------------------------------------------------------------------
\usepackage[dvipsnames,table]{xcolor}

% Colores personalizados
\definecolor{primaryblue}{RGB}{0, 84, 159}
\definecolor{secondaryblue}{RGB}{64, 127, 183}
\definecolor{accentorange}{RGB}{246, 168, 0}
\definecolor{darkgray}{RGB}{64, 64, 64}
\definecolor{lightgray}{RGB}{240, 240, 240}
\definecolor{codegreen}{RGB}{28, 172, 120}
\definecolor{codepurple}{RGB}{170, 55, 241}
\definecolor{codeblue}{RGB}{0, 119, 170}

% -----------------------------------------------------------------------------
% CODIGO FUENTE
% -----------------------------------------------------------------------------
\usepackage{listings}

\lstdefinestyle{pythonstyle}{
    language=Python,
    basicstyle=\ttfamily\small,
    keywordstyle=\color{codeblue}\bfseries,
    stringstyle=\color{codegreen},
    commentstyle=\color{gray}\itshape,
    numberstyle=\tiny\color{gray},
    numbers=left,
    numbersep=8pt,
    frame=single,
    framerule=0.5pt,
    rulecolor=\color{lightgray},
    backgroundcolor=\color{lightgray!30},
    breaklines=true,
    breakatwhitespace=true,
    tabsize=4,
    showstringspaces=false,
    captionpos=b,
    morekeywords={self, True, False, None, torch, nn, Sequential, Linear, ReLU, Sigmoid, Dropout},
    emph={__init__, forward, train, eval},
    emphstyle=\color{codepurple}
}

\lstset{style=pythonstyle}

% Entorno para codigo con caption
\lstnewenvironment{codigo}[1][]
{\lstset{#1}}
{}

% -----------------------------------------------------------------------------
% ALGORITMOS
% -----------------------------------------------------------------------------
\usepackage{algorithm}
\usepackage{algpseudocode}

% Traduccion de algoritmos
\floatname{algorithm}{Algoritmo}
\renewcommand{\algorithmicrequire}{\textbf{Entrada:}}
\renewcommand{\algorithmicensure}{\textbf{Salida:}}

% -----------------------------------------------------------------------------
% REFERENCIAS Y ENLACES
% -----------------------------------------------------------------------------
\usepackage[
    colorlinks=true,
    linkcolor=primaryblue,
    citecolor=secondaryblue,
    urlcolor=accentorange,
    bookmarks=true,
    bookmarksnumbered=true,
    pdfstartview=FitH
]{hyperref}

\usepackage{url}
\usepackage[nameinlink]{cleveref}

% Configuracion de cleveref en espanol
\crefname{figure}{Figura}{Figuras}
\crefname{table}{Tabla}{Tablas}
\crefname{equation}{Ecuacion}{Ecuaciones}
\crefname{chapter}{Capitulo}{Capitulos}
\crefname{section}{Seccion}{Secciones}
\crefname{algorithm}{Algoritmo}{Algoritmos}
\crefname{listing}{Codigo}{Codigos}
\crefname{definicion}{Definicion}{Definiciones}
\crefname{teorema}{Teorema}{Teoremas}

% -----------------------------------------------------------------------------
% BIBLIOGRAFIA
% -----------------------------------------------------------------------------
\usepackage[
    backend=biber,
    style=ieee,
    sorting=none,
    maxbibnames=99,
    minbibnames=3,
    maxcitenames=2,
    mincitenames=1
]{biblatex}

\addbibresource{bibliography.bib}

% -----------------------------------------------------------------------------
% ENCABEZADOS Y PIES DE PAGINA
% -----------------------------------------------------------------------------
\usepackage{fancyhdr}

\pagestyle{fancy}
\fancyhf{}
\fancyhead[LE]{\slshape\nouppercase{\leftmark}}
\fancyhead[RO]{\slshape\nouppercase{\rightmark}}
\fancyfoot[C]{\thepage}
\renewcommand{\headrulewidth}{0.4pt}
\renewcommand{\footrulewidth}{0pt}

% Estilo para paginas de inicio de capitulo
\fancypagestyle{plain}{
    \fancyhf{}
    \fancyfoot[C]{\thepage}
    \renewcommand{\headrulewidth}{0pt}
}

% -----------------------------------------------------------------------------
% APENDICES
% -----------------------------------------------------------------------------
\usepackage[toc,page]{appendix}
\renewcommand{\appendixname}{Apendice}
\renewcommand{\appendixtocname}{Apendices}
\renewcommand{\appendixpagename}{Apendices}

% -----------------------------------------------------------------------------
% UNIDADES SI
% -----------------------------------------------------------------------------
\usepackage{siunitx}
\sisetup{
    output-decimal-marker = {,},
    group-separator = {\,},
    group-minimum-digits = 4
}

% Unidad personalizada para pixeles
\DeclareSIUnit{\px}{px}
\DeclareSIUnit{\epoch}{epoch}

% -----------------------------------------------------------------------------
% NOTAS AL MARGEN Y TO-DO
% -----------------------------------------------------------------------------
\usepackage{todonotes}
\setuptodonotes{inline}

% -----------------------------------------------------------------------------
% OTROS PAQUETES UTILES
% -----------------------------------------------------------------------------
\usepackage{enumitem}          % Listas personalizadas
\usepackage{pdfpages}          % Incluir PDFs
\usepackage{lipsum}            % Texto de relleno (eliminar en version final)

% Configuracion de listas
\setlist{nosep, leftmargin=*}

% -----------------------------------------------------------------------------
% COMANDOS PERSONALIZADOS
% -----------------------------------------------------------------------------

% Abreviaciones comunes
\newcommand{\eg}{e.g.,\xspace}
\newcommand{\ie}{i.e.,\xspace}
\newcommand{\etc}{etc.\xspace}
\newcommand{\etal}{\textit{et al.}\xspace}

% Terminos tecnicos
\newcommand{\CNN}{\textsc{CNN}\xspace}
\newcommand{\ResNet}{\textsc{ResNet}\xspace}
\newcommand{\TTA}{\textsc{TTA}\xspace}
\newcommand{\CLAHE}{\textsc{CLAHE}\xspace}
\newcommand{\WingLoss}{\textit{Wing Loss}\xspace}
\newcommand{\CoordAttn}{\textit{Coordinate Attention}\xspace}
\newcommand{\GroupNorm}{\textit{GroupNorm}\xspace}
\newcommand{\DeepHead}{\textit{Deep Head}\xspace}

% Vectores y matrices
\renewcommand{\vec}[1]{\mathbf{#1}}
\newcommand{\mat}[1]{\mathbf{#1}}

% Operadores matematicos
\DeclareMathOperator*{\argmin}{arg\,min}
\DeclareMathOperator*{\argmax}{arg\,max}
\DeclareMathOperator{\softmax}{softmax}
\DeclareMathOperator{\sigmoid}{sigmoid}

% Notacion de conjuntos
\newcommand{\R}{\mathbb{R}}
\newcommand{\N}{\mathbb{N}}
\newcommand{\Z}{\mathbb{Z}}

% Landmarks
\newcommand{\landmark}[1]{\texttt{L#1}}

% Metricas
\newcommand{\errorpx}[1]{\SI{#1}{\px}}
\newcommand{\mejora}[1]{\textcolor{codegreen}{#1\%}}
\newcommand{\empeora}[1]{\textcolor{red}{+#1\%}}

% Caja para resultados importantes
\usepackage{tcolorbox}
\newtcolorbox{resultadoclave}{
    colback=lightgray!20,
    colframe=primaryblue,
    arc=3mm,
    boxrule=1pt,
    left=5mm,
    right=5mm,
    top=3mm,
    bottom=3mm,
    title=Resultado Clave,
    fonttitle=\bfseries
}

% Caja para notas
\newtcolorbox{nota}{
    colback=accentorange!10,
    colframe=accentorange,
    arc=2mm,
    boxrule=0.5pt,
    left=3mm,
    right=3mm
}

% -----------------------------------------------------------------------------
% CONFIGURACION DEL DOCUMENTO
% -----------------------------------------------------------------------------

% Profundidad del indice
\setcounter{tocdepth}{3}
\setcounter{secnumdepth}{3}

% Evitar viudas y huerfanas
\widowpenalty=10000
\clubpenalty=10000

% Espacio entre parrafos
\setlength{\parskip}{0.5em}
\setlength{\parindent}{1.5em}

% Ajuste de flotantes
\renewcommand{\floatpagefraction}{0.8}
\renewcommand{\topfraction}{0.9}
\renewcommand{\bottomfraction}{0.8}
\renewcommand{\textfraction}{0.1}

% =============================================================================
% FIN DEL PREAMBLE
% =============================================================================
