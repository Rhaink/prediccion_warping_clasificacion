% =============================================================================
% TESIS DE MAESTRIA EN INGENIERIA ELECTRONICA
% =============================================================================
% Titulo: Prediccion de Landmarks Anatomicos en Radiografias de Torax
%         mediante Deep Learning
% Autor: [Nombre del Autor]
% Director: [Nombre del Director]
% Universidad: [Nombre de la Universidad]
% Fecha: Noviembre 2024
% =============================================================================

\documentclass[12pt,twoside,openright]{book}

% Cargar configuracion de paquetes
% =============================================================================
% PREAMBLE.TEX - Configuracion de Paquetes para Tesis de Maestria
% =============================================================================
% Prediccion de Landmarks Anatomicos en Radiografias de Torax mediante Deep Learning
% Autor: [Nombre del Autor]
% Universidad: [Nombre de la Universidad]
% Maestria en Ingenieria Electronica
% =============================================================================

% -----------------------------------------------------------------------------
% CODIFICACION Y LENGUAJE
% -----------------------------------------------------------------------------
\usepackage[utf8]{inputenc}
\usepackage[T1]{fontenc}
\usepackage[spanish,es-tabla]{babel}

% -----------------------------------------------------------------------------
% GEOMETRIA Y MARGENES
% -----------------------------------------------------------------------------
\usepackage[
    a4paper,
    top=2.5cm,
    bottom=2.5cm,
    left=3cm,
    right=2.5cm,
    headheight=14pt
]{geometry}

% -----------------------------------------------------------------------------
% TIPOGRAFIA
% -----------------------------------------------------------------------------
\usepackage{lmodern}           % Fuente Latin Modern
\usepackage{microtype}         % Mejoras tipograficas
\usepackage{setspace}          % Espaciado entre lineas
\onehalfspacing                % Espaciado 1.5

% -----------------------------------------------------------------------------
% MATEMATICAS
% -----------------------------------------------------------------------------
\usepackage{amsmath}
\usepackage{amsfonts}
\usepackage{amssymb}
\usepackage{amsthm}
\usepackage{mathtools}
\usepackage{bm}                % Negritas en matematicas

% Definiciones de teoremas
\theoremstyle{definition}
\newtheorem{definicion}{Definicion}[chapter]
\newtheorem{ejemplo}{Ejemplo}[chapter]

\theoremstyle{plain}
\newtheorem{teorema}{Teorema}[chapter]
\newtheorem{lema}{Lema}[chapter]
\newtheorem{proposicion}{Proposicion}[chapter]
\newtheorem{corolario}{Corolario}[chapter]

% -----------------------------------------------------------------------------
% GRAFICOS E IMAGENES
% -----------------------------------------------------------------------------
\usepackage{graphicx}
\graphicspath{{figures/}{../outputs/diagrams/}{../outputs/thesis_figures/}}
\usepackage{float}
\usepackage{subcaption}        % Subfiguras
\usepackage{wrapfig}           % Figuras flotantes

% Configuracion de captions
\usepackage[
    font=small,
    labelfont=bf,
    textfont=it,
    format=hang,
    margin=1cm
]{caption}

% -----------------------------------------------------------------------------
% TABLAS
% -----------------------------------------------------------------------------
\usepackage{booktabs}          % Tablas profesionales
\usepackage{multirow}          % Celdas multirow
\usepackage{array}             % Extensiones de tablas
\usepackage{longtable}         % Tablas largas
\usepackage{tabularx}          % Tablas con ancho ajustable
\usepackage{colortbl}          % Color en tablas

% Nuevo tipo de columna centrada con ancho fijo
\newcolumntype{C}[1]{>{\centering\arraybackslash}p{#1}}
\newcolumntype{L}[1]{>{\raggedright\arraybackslash}p{#1}}
\newcolumntype{R}[1]{>{\raggedleft\arraybackslash}p{#1}}

% -----------------------------------------------------------------------------
% COLORES
% -----------------------------------------------------------------------------
\usepackage[dvipsnames,table]{xcolor}

% Colores personalizados
\definecolor{primaryblue}{RGB}{0, 84, 159}
\definecolor{secondaryblue}{RGB}{64, 127, 183}
\definecolor{accentorange}{RGB}{246, 168, 0}
\definecolor{darkgray}{RGB}{64, 64, 64}
\definecolor{lightgray}{RGB}{240, 240, 240}
\definecolor{codegreen}{RGB}{28, 172, 120}
\definecolor{codepurple}{RGB}{170, 55, 241}
\definecolor{codeblue}{RGB}{0, 119, 170}

% -----------------------------------------------------------------------------
% CODIGO FUENTE
% -----------------------------------------------------------------------------
\usepackage{listings}

\lstdefinestyle{pythonstyle}{
    language=Python,
    basicstyle=\ttfamily\small,
    keywordstyle=\color{codeblue}\bfseries,
    stringstyle=\color{codegreen},
    commentstyle=\color{gray}\itshape,
    numberstyle=\tiny\color{gray},
    numbers=left,
    numbersep=8pt,
    frame=single,
    framerule=0.5pt,
    rulecolor=\color{lightgray},
    backgroundcolor=\color{lightgray!30},
    breaklines=true,
    breakatwhitespace=true,
    tabsize=4,
    showstringspaces=false,
    captionpos=b,
    morekeywords={self, True, False, None, torch, nn, Sequential, Linear, ReLU, Sigmoid, Dropout},
    emph={__init__, forward, train, eval},
    emphstyle=\color{codepurple}
}

\lstset{style=pythonstyle}

% Entorno para codigo con caption
\lstnewenvironment{codigo}[1][]
{\lstset{#1}}
{}

% -----------------------------------------------------------------------------
% ALGORITMOS
% -----------------------------------------------------------------------------
\usepackage{algorithm}
\usepackage{algpseudocode}

% Traduccion de algoritmos
\floatname{algorithm}{Algoritmo}
\renewcommand{\algorithmicrequire}{\textbf{Entrada:}}
\renewcommand{\algorithmicensure}{\textbf{Salida:}}

% -----------------------------------------------------------------------------
% REFERENCIAS Y ENLACES
% -----------------------------------------------------------------------------
\usepackage[
    colorlinks=true,
    linkcolor=primaryblue,
    citecolor=secondaryblue,
    urlcolor=accentorange,
    bookmarks=true,
    bookmarksnumbered=true,
    pdfstartview=FitH
]{hyperref}

\usepackage{url}
\usepackage[nameinlink]{cleveref}

% Configuracion de cleveref en espanol
\crefname{figure}{Figura}{Figuras}
\crefname{table}{Tabla}{Tablas}
\crefname{equation}{Ecuacion}{Ecuaciones}
\crefname{chapter}{Capitulo}{Capitulos}
\crefname{section}{Seccion}{Secciones}
\crefname{algorithm}{Algoritmo}{Algoritmos}
\crefname{listing}{Codigo}{Codigos}
\crefname{definicion}{Definicion}{Definiciones}
\crefname{teorema}{Teorema}{Teoremas}

% -----------------------------------------------------------------------------
% BIBLIOGRAFIA
% -----------------------------------------------------------------------------
\usepackage[
    backend=biber,
    style=ieee,
    sorting=none,
    maxbibnames=99,
    minbibnames=3,
    maxcitenames=2,
    mincitenames=1
]{biblatex}

\addbibresource{bibliography.bib}

% -----------------------------------------------------------------------------
% ENCABEZADOS Y PIES DE PAGINA
% -----------------------------------------------------------------------------
\usepackage{fancyhdr}

\pagestyle{fancy}
\fancyhf{}
\fancyhead[LE]{\slshape\nouppercase{\leftmark}}
\fancyhead[RO]{\slshape\nouppercase{\rightmark}}
\fancyfoot[C]{\thepage}
\renewcommand{\headrulewidth}{0.4pt}
\renewcommand{\footrulewidth}{0pt}

% Estilo para paginas de inicio de capitulo
\fancypagestyle{plain}{
    \fancyhf{}
    \fancyfoot[C]{\thepage}
    \renewcommand{\headrulewidth}{0pt}
}

% -----------------------------------------------------------------------------
% APENDICES
% -----------------------------------------------------------------------------
\usepackage[toc,page]{appendix}
\renewcommand{\appendixname}{Apendice}
\renewcommand{\appendixtocname}{Apendices}
\renewcommand{\appendixpagename}{Apendices}

% -----------------------------------------------------------------------------
% UNIDADES SI
% -----------------------------------------------------------------------------
\usepackage{siunitx}
\sisetup{
    output-decimal-marker = {,},
    group-separator = {\,},
    group-minimum-digits = 4
}

% Unidad personalizada para pixeles
\DeclareSIUnit{\px}{px}
\DeclareSIUnit{\epoch}{epoch}

% -----------------------------------------------------------------------------
% NOTAS AL MARGEN Y TO-DO
% -----------------------------------------------------------------------------
\usepackage{todonotes}
\setuptodonotes{inline}

% -----------------------------------------------------------------------------
% OTROS PAQUETES UTILES
% -----------------------------------------------------------------------------
\usepackage{enumitem}          % Listas personalizadas
\usepackage{pdfpages}          % Incluir PDFs
\usepackage{lipsum}            % Texto de relleno (eliminar en version final)

% Configuracion de listas
\setlist{nosep, leftmargin=*}

% -----------------------------------------------------------------------------
% COMANDOS PERSONALIZADOS
% -----------------------------------------------------------------------------

% Abreviaciones comunes
\newcommand{\eg}{e.g.,\xspace}
\newcommand{\ie}{i.e.,\xspace}
\newcommand{\etc}{etc.\xspace}
\newcommand{\etal}{\textit{et al.}\xspace}

% Terminos tecnicos
\newcommand{\CNN}{\textsc{CNN}\xspace}
\newcommand{\ResNet}{\textsc{ResNet}\xspace}
\newcommand{\TTA}{\textsc{TTA}\xspace}
\newcommand{\CLAHE}{\textsc{CLAHE}\xspace}
\newcommand{\WingLoss}{\textit{Wing Loss}\xspace}
\newcommand{\CoordAttn}{\textit{Coordinate Attention}\xspace}
\newcommand{\GroupNorm}{\textit{GroupNorm}\xspace}
\newcommand{\DeepHead}{\textit{Deep Head}\xspace}

% Vectores y matrices
\renewcommand{\vec}[1]{\mathbf{#1}}
\newcommand{\mat}[1]{\mathbf{#1}}

% Operadores matematicos
\DeclareMathOperator*{\argmin}{arg\,min}
\DeclareMathOperator*{\argmax}{arg\,max}
\DeclareMathOperator{\softmax}{softmax}
\DeclareMathOperator{\sigmoid}{sigmoid}

% Notacion de conjuntos
\newcommand{\R}{\mathbb{R}}
\newcommand{\N}{\mathbb{N}}
\newcommand{\Z}{\mathbb{Z}}

% Landmarks
\newcommand{\landmark}[1]{\texttt{L#1}}

% Metricas
\newcommand{\errorpx}[1]{\SI{#1}{\px}}
\newcommand{\mejora}[1]{\textcolor{codegreen}{#1\%}}
\newcommand{\empeora}[1]{\textcolor{red}{+#1\%}}

% Caja para resultados importantes
\usepackage{tcolorbox}
\newtcolorbox{resultadoclave}{
    colback=lightgray!20,
    colframe=primaryblue,
    arc=3mm,
    boxrule=1pt,
    left=5mm,
    right=5mm,
    top=3mm,
    bottom=3mm,
    title=Resultado Clave,
    fonttitle=\bfseries
}

% Caja para notas
\newtcolorbox{nota}{
    colback=accentorange!10,
    colframe=accentorange,
    arc=2mm,
    boxrule=0.5pt,
    left=3mm,
    right=3mm
}

% -----------------------------------------------------------------------------
% CONFIGURACION DEL DOCUMENTO
% -----------------------------------------------------------------------------

% Profundidad del indice
\setcounter{tocdepth}{3}
\setcounter{secnumdepth}{3}

% Evitar viudas y huerfanas
\widowpenalty=10000
\clubpenalty=10000

% Espacio entre parrafos
\setlength{\parskip}{0.5em}
\setlength{\parindent}{1.5em}

% Ajuste de flotantes
\renewcommand{\floatpagefraction}{0.8}
\renewcommand{\topfraction}{0.9}
\renewcommand{\bottomfraction}{0.8}
\renewcommand{\textfraction}{0.1}

% =============================================================================
% FIN DEL PREAMBLE
% =============================================================================


% =============================================================================
% METADATOS DEL DOCUMENTO
% =============================================================================
\hypersetup{
    pdftitle={Prediccion de Landmarks Anatomicos en Radiografias de Torax mediante Deep Learning},
    pdfauthor={[Nombre del Autor]},
    pdfsubject={Tesis de Maestria en Ingenieria Electronica},
    pdfkeywords={Deep Learning, Landmarks, Radiografias, ResNet, Computer Vision, Medical Imaging}
}

% =============================================================================
% INICIO DEL DOCUMENTO
% =============================================================================
\begin{document}

% -----------------------------------------------------------------------------
% PAGINAS PRELIMINARES (numeracion romana)
% -----------------------------------------------------------------------------
\frontmatter
\pagestyle{plain}

% Portada
\begin{titlepage}
    \centering

    % Logo de la universidad (descomentar cuando se tenga)
    % \includegraphics[width=0.3\textwidth]{figures/logo_universidad.png}

    \vspace{1cm}

    {\Large\bfseries [NOMBRE DE LA UNIVERSIDAD]}

    \vspace{0.5cm}

    {\large Facultad de Ingenieria}

    \vspace{0.3cm}

    {\large Maestria en Ingenieria Electronica}

    \vspace{2cm}

    \rule{\textwidth}{1.5pt}

    \vspace{0.5cm}

    {\LARGE\bfseries Prediccion de Landmarks Anatomicos en Radiografias de Torax mediante Deep Learning}

    \vspace{0.5cm}

    \rule{\textwidth}{1.5pt}

    \vspace{2cm}

    {\large Tesis presentada para obtener el grado de}

    \vspace{0.3cm}

    {\Large\bfseries Maestro en Ingenieria Electronica}

    \vspace{2cm}

    {\large Presenta:}

    \vspace{0.3cm}

    {\Large [Nombre del Autor]}

    \vspace{1.5cm}

    {\large Director de Tesis:}

    \vspace{0.3cm}

    {\Large Dr. [Nombre del Director]}

    \vfill

    {\large [Ciudad], [Pais]}

    \vspace{0.3cm}

    {\large Noviembre 2024}

\end{titlepage}

% Pagina en blanco despues de portada
\cleardoublepage

% -----------------------------------------------------------------------------
% DEDICATORIA (opcional)
% -----------------------------------------------------------------------------
\chapter*{Dedicatoria}
\thispagestyle{empty}
\vspace*{5cm}
\begin{flushright}
    \textit{A mi familia, por su apoyo incondicional.}
\end{flushright}
\cleardoublepage

% -----------------------------------------------------------------------------
% AGRADECIMIENTOS
% -----------------------------------------------------------------------------
\chapter*{Agradecimientos}
\addcontentsline{toc}{chapter}{Agradecimientos}

Agradezco profundamente a mi director de tesis, Dr. [Nombre], por su guia experta y su constante apoyo durante el desarrollo de este trabajo.

A la [Universidad/Institucion], por proporcionar los recursos computacionales necesarios para el entrenamiento de los modelos de deep learning.

A mis companeros del laboratorio de [Nombre del Laboratorio], por las valiosas discusiones y retroalimentacion que enriquecieron esta investigacion.

A los revisores y desarrolladores de las bibliotecas de codigo abierto PyTorch, OpenCV y las demas herramientas utilizadas, sin las cuales este trabajo no habria sido posible.

Finalmente, a mi familia, por su paciencia y apoyo incondicional durante todo este proceso.

\cleardoublepage

% -----------------------------------------------------------------------------
% RESUMEN EN ESPANOL
% -----------------------------------------------------------------------------
\chapter*{Resumen}
\addcontentsline{toc}{chapter}{Resumen}

\textbf{Palabras clave:} Deep Learning, Landmarks Anatomicos, Radiografias de Torax, ResNet, Coordinate Attention, Wing Loss, Ensemble Learning.

\vspace{1cm}

La deteccion automatica de landmarks anatomicos en imagenes medicas es fundamental para el diagnostico asistido por computadora y la planificacion de tratamientos. En este trabajo se presenta un sistema de deep learning para la prediccion de 15 landmarks anatomicos en radiografias de torax, abordando los desafios particulares que presentan las imagenes de pacientes con COVID-19 y neumonia viral.

Se desarrollo una arquitectura basada en ResNet-18 con un modulo de Coordinate Attention y una cabeza de regresion profunda optimizada. El sistema incorpora un preprocesamiento con CLAHE (Contrast Limited Adaptive Histogram Equalization) que mejora significativamente la deteccion en casos con consolidaciones pulmonares. La funcion de perdida Wing Loss demostro ser superior a MSE para este problema de regresion de coordenadas.

El modelo final consiste en un ensemble de 4 redes entrenadas con diferentes inicializaciones, combinado con Test-Time Augmentation (TTA). Sobre un dataset de 957 radiografias divididas en entrenamiento (717), validacion (144) y prueba (96), se logro un error promedio de \textbf{3.71 pixeles}, lo que representa una mejora del \textbf{59\%} respecto al baseline inicial de 9.08 pixeles. Este resultado supera ampliamente el objetivo original de $<$8 pixeles y se acerca al limite teorico impuesto por el ruido inherente del etiquetado manual ($\sim$1.5-2.0 pixeles).

El analisis detallado revela que los landmarks centrales (L9, L10) son los mas precisos (2.64-2.83 px), mientras que los senos costofrenicos (L14, L15) representan el mayor desafio (4.46-4.82 px). CLAHE resulto particularmente efectivo para casos de COVID-19, reduciendo el error en esa categoria de 11.74 a 3.80 pixeles (-68\%).

Los resultados demuestran que las tecnicas modernas de deep learning pueden alcanzar precision clinicamente util en la deteccion de landmarks anatomicos, con potencial aplicacion en sistemas de diagnostico asistido.

\cleardoublepage

% -----------------------------------------------------------------------------
% ABSTRACT EN INGLES
% -----------------------------------------------------------------------------
\chapter*{Abstract}
\addcontentsline{toc}{chapter}{Abstract}

\textbf{Keywords:} Deep Learning, Anatomical Landmarks, Chest X-rays, ResNet, Coordinate Attention, Wing Loss, Ensemble Learning.

\vspace{1cm}

Automatic detection of anatomical landmarks in medical images is fundamental for computer-aided diagnosis and treatment planning. This work presents a deep learning system for predicting 15 anatomical landmarks in chest X-rays, addressing the particular challenges posed by images from patients with COVID-19 and viral pneumonia.

An architecture based on ResNet-18 was developed, incorporating a Coordinate Attention module and an optimized deep regression head. The system includes CLAHE (Contrast Limited Adaptive Histogram Equalization) preprocessing, which significantly improves detection in cases with pulmonary consolidations. Wing Loss proved superior to MSE for this coordinate regression problem.

The final model consists of an ensemble of 4 networks trained with different initializations, combined with Test-Time Augmentation (TTA). On a dataset of 957 radiographs divided into training (717), validation (144), and test (96) sets, an average error of \textbf{3.71 pixels} was achieved, representing a \textbf{59\%} improvement over the initial baseline of 9.08 pixels. This result far exceeds the original goal of $<$8 pixels and approaches the theoretical limit imposed by inherent manual labeling noise ($\sim$1.5-2.0 pixels).

Detailed analysis reveals that central landmarks (L9, L10) are the most accurate (2.64-2.83 px), while costophrenic angles (L14, L15) represent the greatest challenge (4.46-4.82 px). CLAHE proved particularly effective for COVID-19 cases, reducing error in that category from 11.74 to 3.80 pixels (-68\%).

Results demonstrate that modern deep learning techniques can achieve clinically useful precision in anatomical landmark detection, with potential application in assisted diagnostic systems.

\cleardoublepage

% -----------------------------------------------------------------------------
% INDICES
% -----------------------------------------------------------------------------

% Indice general
\tableofcontents
\cleardoublepage

% Lista de figuras
\listoffigures
\addcontentsline{toc}{chapter}{Lista de Figuras}
\cleardoublepage

% Lista de tablas
\listoftables
\addcontentsline{toc}{chapter}{Lista de Tablas}
\cleardoublepage

% Lista de algoritmos
\listofalgorithms
\addcontentsline{toc}{chapter}{Lista de Algoritmos}
\cleardoublepage

% Lista de codigos
\lstlistoflistings
\addcontentsline{toc}{chapter}{Lista de Codigos}
\cleardoublepage

% -----------------------------------------------------------------------------
% LISTA DE SIMBOLOS Y ABREVIATURAS
% -----------------------------------------------------------------------------
\chapter*{Lista de Simbolos y Abreviaturas}
\addcontentsline{toc}{chapter}{Lista de Simbolos y Abreviaturas}

\section*{Simbolos}

\begin{tabular}{@{}ll@{}}
    $\vec{x}$ & Vector de entrada \\
    $\vec{y}$ & Vector de salida (coordenadas de landmarks) \\
    $\hat{\vec{y}}$ & Vector de predicciones \\
    $\mathcal{L}$ & Funcion de perdida \\
    $\omega$ & Parametro de umbral en Wing Loss \\
    $\epsilon$ & Parametro de curvatura en Wing Loss \\
    $\theta$ & Parametros del modelo \\
    $\eta$ & Tasa de aprendizaje (learning rate) \\
    $\sigma$ & Desviacion estandar \\
    $t$ & Parametro de posicion sobre el eje central \\
\end{tabular}

\section*{Abreviaturas}

\begin{tabular}{@{}ll@{}}
    \CNN & Convolutional Neural Network (Red Neuronal Convolucional) \\
    \CLAHE & Contrast Limited Adaptive Histogram Equalization \\
    \TTA & Test-Time Augmentation \\
    FC & Fully Connected (Completamente Conectada) \\
    GAP & Global Average Pooling \\
    GPU & Graphics Processing Unit \\
    MSE & Mean Squared Error \\
    \ResNet & Residual Network \\
    ReLU & Rectified Linear Unit \\
    SGD & Stochastic Gradient Descent \\
    Adam & Adaptive Moment Estimation \\
    ROC & Receiver Operating Characteristic \\
    px & Pixeles \\
\end{tabular}

\cleardoublepage

% -----------------------------------------------------------------------------
% CUERPO PRINCIPAL (numeracion arabiga)
% -----------------------------------------------------------------------------
\mainmatter
\pagestyle{fancy}

% Capitulos
% =============================================================================
% CAPITULO 1: INTRODUCCION
% =============================================================================

\chapter{Introduccion}
\label{ch:introduccion}

% Cita inicial del capitulo
\begin{flushright}
    \textit{``El verdadero viaje de descubrimiento no consiste en buscar \\
    nuevos paisajes, sino en tener nuevos ojos.''}\\
    --- Marcel Proust
\end{flushright}

\vspace{1cm}

% -----------------------------------------------------------------------------
\section{Contexto y Motivacion}
\label{sec:contexto}
% -----------------------------------------------------------------------------

La radiografia de torax constituye una de las modalidades de imagenes medicas mas utilizadas a nivel mundial, con mas de 2 mil millones de examenes realizados anualmente \cite{raoof2012interpretation}. Su bajo costo, rapida adquisicion, baja dosis de radiacion y amplia disponibilidad la convierten en la primera linea de diagnostico para una amplia variedad de condiciones pulmonares y cardiacas \cite{speets2006chest}.

La interpretacion de radiografias de torax requiere la identificacion precisa de estructuras anatomicas de referencia, conocidas como \textit{landmarks} o puntos de referencia anatomicos. Estos landmarks son fundamentales para:

\begin{itemize}
    \item \textbf{Mediciones clinicas}: Calculo del indice cardiotoracico, evaluacion del tamano cardiaco y medicion de la silueta mediastinica.
    \item \textbf{Deteccion de patologias}: Identificacion de cardiomegalia, derrames pleurales, neumotorax y masas pulmonares.
    \item \textbf{Seguimiento longitudinal}: Comparacion objetiva entre estudios sucesivos para evaluar progresion o regresion de enfermedades.
    \item \textbf{Planificacion de tratamientos}: Definicion de campos de radioterapia y guia para procedimientos intervencionistas.
\end{itemize}

La pandemia de COVID-19 ha intensificado dramaticamente la demanda de interpretacion de radiografias toracicas. Durante los picos de la pandemia, los departamentos de radiologia experimentaron aumentos de hasta el 300\% en su carga de trabajo \cite{jacobi2020portable}, lo que genero una necesidad urgente de herramientas automatizadas que asistan en el triaje y diagnostico.

Sin embargo, la deteccion manual de landmarks anatomicos presenta limitaciones significativas:

\begin{enumerate}
    \item \textbf{Variabilidad inter-observador}: Diferentes radiologos pueden identificar los mismos puntos anatomicos con discrepancias de 5 a 15 pixeles \cite{van2017automatic}.
    \item \textbf{Fatiga y errores humanos}: La carga de trabajo elevada incrementa la probabilidad de errores diagnosticos.
    \item \textbf{Tiempo de procesamiento}: El etiquetado manual de multiples landmarks por imagen consume recursos valiosos del especialista.
    \item \textbf{Escalabilidad limitada}: Es impractico mantener consistencia en estudios poblacionales a gran escala.
\end{enumerate}

Estas limitaciones motivan el desarrollo de sistemas automaticos de deteccion de landmarks basados en tecnicas de aprendizaje profundo (deep learning), capaces de proporcionar mediciones rapidas, reproducibles y objetivas.

% -----------------------------------------------------------------------------
\section{Problematica}
\label{sec:problematica}
% -----------------------------------------------------------------------------

La deteccion automatica de landmarks anatomicos en radiografias de torax presenta desafios tecnicos especificos que la distinguen de otros problemas de vision por computadora:

\subsection{Variabilidad Anatomica}

La anatomia toracica exhibe considerable variabilidad entre individuos debido a factores como edad, sexo, constitucion fisica y posicion durante la adquisicion de la imagen. Los senos costofrenicos, en particular, presentan una alta variabilidad posicional (desviacion estandar de hasta 35 pixeles en nuestro dataset), lo que dificulta su localizacion precisa.

\subsection{Presencia de Patologia}

Las condiciones patologicas alteran significativamente la apariencia de las estructuras anatomicas:

\begin{itemize}
    \item \textbf{COVID-19}: Produce opacidades en vidrio esmerilado y consolidaciones que oscurecen los bordes pulmonares, dificultando la identificacion de landmarks en las bases pulmonares.
    \item \textbf{Neumonia viral}: Genera infiltrados difusos que reducen el contraste local.
    \item \textbf{Derrames pleurales}: Alteran la posicion aparente de los senos costofrenicos.
    \item \textbf{Cardiomegalia}: Modifica las relaciones espaciales entre estructuras mediastinicas.
\end{itemize}

\subsection{Limitaciones del Etiquetado}

El ground truth disponible para entrenar modelos supervisados proviene de etiquetado manual, el cual conlleva:

\begin{itemize}
    \item \textbf{Ruido inherente}: Error de etiquetado estimado en 1.5 a 2.0 pixeles.
    \item \textbf{Asimetrias naturales}: Los pares de landmarks bilaterales presentan asimetrias de 5.5 a 7.9 pixeles en el ground truth, reflejando tanto la anatomia real como la variabilidad del etiquetado.
    \item \textbf{Ambiguedad en bordes difusos}: En zonas de bajo contraste, la definicion exacta del landmark es subjetiva.
\end{itemize}

\subsection{Desbalance de Datos}

Los datasets disponibles para radiografias de torax frecuentemente presentan desbalances entre categorias diagnosticas. En el dataset utilizado en este trabajo, la distribucion es:
\begin{itemize}
    \item Normal: 48.9\%
    \item COVID-19: 32.0\%
    \item Neumonia viral: 19.1\%
\end{itemize}

Este desbalance puede sesgar el modelo hacia las categorias mayoritarias, degradando el rendimiento en casos patologicos que son precisamente los de mayor interes clinico.

% -----------------------------------------------------------------------------
\section{Objetivos}
\label{sec:objetivos}
% -----------------------------------------------------------------------------

\subsection{Objetivo General}

Desarrollar un sistema de deep learning para la deteccion automatica de 15 landmarks anatomicos en radiografias de torax, alcanzando un error de localizacion menor a 8 pixeles y demostrando robustez ante condiciones patologicas como COVID-19 y neumonia viral.

\subsection{Objetivos Especificos}

\begin{enumerate}
    \item \textbf{Implementar una arquitectura de red neuronal optimizada}: Disenar y entrenar un modelo basado en ResNet-18 con mecanismos de atencion espacial que maximice la precision en la localizacion de landmarks.

    \item \textbf{Desarrollar un pipeline de preprocesamiento robusto}: Implementar tecnicas de realce de contraste (CLAHE) y normalizacion que mejoren la visibilidad de estructuras anatomicas, especialmente en casos patologicos.

    \item \textbf{Evaluar funciones de perdida especializadas}: Comparar el rendimiento de diferentes funciones de perdida (MSE, Wing Loss, perdidas geometricas) para el problema de regresion de coordenadas.

    \item \textbf{Implementar estrategias de ensemble}: Desarrollar un sistema de combinacion de multiples modelos que reduzca la varianza de las predicciones y mejore la precision final.

    \item \textbf{Analizar el rendimiento por landmark y categoria}: Caracterizar el comportamiento del modelo en diferentes landmarks anatomicos y categorias diagnosticas, identificando fortalezas y limitaciones.

    \item \textbf{Documentar el proceso de desarrollo}: Registrar sistematicamente las decisiones de diseno, experimentos realizados y lecciones aprendidas para facilitar la reproducibilidad y futura extension del trabajo.
\end{enumerate}

% -----------------------------------------------------------------------------
\section{Contribuciones de la Tesis}
\label{sec:contribuciones}
% -----------------------------------------------------------------------------

Las principales contribuciones de este trabajo son:

\begin{resultadoclave}
\begin{enumerate}
    \item \textbf{Sistema de alta precision}: Se logro un error promedio de \errorpx{3.71}, superando ampliamente el objetivo inicial de $<$8 pixeles y acercandose al limite teorico impuesto por el ruido de etiquetado.

    \item \textbf{Mejora del 59\% sobre baseline}: Reduccion del error de 9.08 px a 3.71 px mediante optimizacion sistematica de arquitectura, preprocesamiento y entrenamiento.

    \item \textbf{Demostracion de eficacia de CLAHE para COVID-19}: El preprocesamiento con CLAHE (tile\_size=4) redujo el error en casos de COVID-19 de 11.74 a 3.80 pixeles, una mejora del 68\%.

    \item \textbf{Estrategia de ensemble selectivo}: Se demostro que excluir modelos de bajo rendimiento del ensemble es mas efectivo que usar esquemas de ponderacion.
\end{enumerate}
\end{resultadoclave}

Adicionalmente, se realizaron contribuciones metodologicas:

\begin{itemize}
    \item \textbf{Analisis geometrico del etiquetado}: Identificacion de la estructura parametrica del proceso de etiquetado manual, revelando que los landmarks centrales dividen el eje mediastinico en proporciones exactas (t = 0.25, 0.50, 0.75).

    \item \textbf{Documentacion exhaustiva}: Registro detallado de 15 sesiones de desarrollo, incluyendo experimentos fallidos y lecciones aprendidas, facilitando la reproducibilidad.

    \item \textbf{Codigo abierto}: Implementacion completa disponible para la comunidad cientifica, incluyendo scripts de entrenamiento, evaluacion e inferencia.
\end{itemize}

% -----------------------------------------------------------------------------
\section{Alcances y Limitaciones}
\label{sec:alcances}
% -----------------------------------------------------------------------------

\subsection{Alcances}

\begin{itemize}
    \item El sistema esta disenado para radiografias de torax posteroanterior (PA) en formato digital.
    \item Se consideran 15 landmarks anatomicos especificos, seleccionados por su relevancia clinica.
    \item El modelo ha sido entrenado y evaluado con imagenes de tres categorias: normal, COVID-19 y neumonia viral.
    \item La implementacion es compatible con hardware de consumo (GPU AMD RX 6600 utilizada en este trabajo).
\end{itemize}

\subsection{Limitaciones}

\begin{itemize}
    \item \textbf{Tamano del dataset}: Con 957 imagenes totales y 96 en el conjunto de prueba, la generalizacion a poblaciones mas amplias requiere validacion adicional.

    \item \textbf{Tipos de patologia}: El modelo no ha sido evaluado en condiciones como tuberculosis, cancer pulmonar, fibrosis o cardiopatias congenitas.

    \item \textbf{Variabilidad de adquisicion}: Las imagenes provienen de fuentes limitadas; el rendimiento en imagenes de diferentes equipos o protocolos de adquisicion puede variar.

    \item \textbf{Landmarks costofrenicos}: Los senos costofrenicos (L14, L15) presentan errores mayores (~4.8 px) debido a su alta variabilidad anatomica.

    \item \textbf{Limite teorico}: El error minimo alcanzable esta limitado por el ruido inherente del etiquetado manual ($\sim$1.5-2.0 px).
\end{itemize}

% -----------------------------------------------------------------------------
\section{Metodologia de Investigacion}
\label{sec:metodologia_inv}
% -----------------------------------------------------------------------------

Este trabajo siguio una metodologia de desarrollo iterativo e incremental, estructurada en las siguientes fases:

\begin{enumerate}
    \item \textbf{Analisis exploratorio}: Caracterizacion estadistica del dataset, identificacion de patrones geometricos en el etiquetado y definicion de metricas de evaluacion.

    \item \textbf{Implementacion del baseline}: Desarrollo de la arquitectura base (ResNet-18 con cabeza de regresion) y establecimiento de la linea base de rendimiento.

    \item \textbf{Optimizacion iterativa}: Ciclos de experimentacion sistematica para mejorar cada componente del sistema (preprocesamiento, arquitectura, entrenamiento, inferencia).

    \item \textbf{Validacion rigurosa}: Verificacion de ausencia de data leakage, evaluacion en conjunto de prueba independiente y analisis de casos fallidos.

    \item \textbf{Documentacion continua}: Registro de cada sesion de trabajo, decisiones de diseno y resultados experimentales.
\end{enumerate}

El proceso de desarrollo se extendio durante 15 sesiones de trabajo, cada una con objetivos especificos y entregables definidos. Este enfoque permitio un progreso medible y la identificacion temprana de problemas.

% -----------------------------------------------------------------------------
\section{Organizacion del Documento}
\label{sec:organizacion}
% -----------------------------------------------------------------------------

El resto de este documento esta organizado de la siguiente manera:

\begin{description}
    \item[Capitulo 2: Marco Teorico] Presenta los fundamentos teoricos de redes neuronales convolucionales, arquitecturas de deep learning, mecanismos de atencion, funciones de perdida especializadas y tecnicas de preprocesamiento de imagenes medicas.

    \item[Capitulo 3: Estado del Arte] Revisa la literatura relacionada con deteccion de landmarks en imagenes medicas, comparando diferentes enfoques (heatmaps vs. regresion directa) e identificando gaps en la investigacion actual.

    \item[Capitulo 4: Metodologia] Describe en detalle la arquitectura propuesta, el pipeline de preprocesamiento, la estrategia de entrenamiento y las metricas de evaluacion utilizadas.

    \item[Capitulo 5: Experimentacion y Resultados] Presenta los experimentos realizados, incluyendo estudios de ablacion, analisis por landmark y categoria, y comparacion de configuraciones.

    \item[Capitulo 6: Discusion] Interpreta los resultados obtenidos, compara con el estado del arte, analiza casos exitosos y fallidos, y discute las implicaciones practicas.

    \item[Capitulo 7: Conclusiones] Resume las contribuciones principales, evalua el cumplimiento de objetivos y propone direcciones de trabajo futuro.

    \item[Apendice A: Codigo Fuente] Incluye fragmentos de codigo relevantes para la implementacion del modelo y funciones de perdida.

    \item[Apendice B: Hiperparametros] Documenta la configuracion completa del sistema final.

    \item[Apendice C: Visualizaciones Adicionales] Presenta figuras y diagramas complementarios.
\end{description}

% -----------------------------------------------------------------------------
% FIN DEL CAPITULO
% -----------------------------------------------------------------------------

% =============================================================================
% CAPITULO 2: MARCO TEORICO
% =============================================================================

\chapter{Marco Teorico}
\label{ch:marco_teorico}

Este capitulo presenta los fundamentos teoricos necesarios para comprender la arquitectura y metodologia desarrollada en este trabajo. Se abordan los conceptos de redes neuronales convolucionales, arquitecturas modernas, mecanismos de atencion, funciones de perdida especializadas y tecnicas de preprocesamiento de imagenes medicas.

% -----------------------------------------------------------------------------
\section{Redes Neuronales Artificiales}
\label{sec:redes_neuronales}
% -----------------------------------------------------------------------------

\subsection{Perceptron y Redes Multicapa}

Las redes neuronales artificiales estan inspiradas en el funcionamiento del cerebro biologico. La unidad basica es el \textit{perceptron}, introducido por Rosenblatt en 1958 \cite{rosenblatt1958perceptron}, que computa una suma ponderada de sus entradas y aplica una funcion de activacion:

\begin{equation}
    y = \phi\left(\sum_{i=1}^{n} w_i x_i + b\right) = \phi(\vec{w}^T \vec{x} + b)
    \label{eq:perceptron}
\end{equation}

donde $\vec{x} \in \R^n$ es el vector de entrada, $\vec{w} \in \R^n$ son los pesos, $b$ es el sesgo (bias) y $\phi$ es la funcion de activacion.

Un \textit{perceptron multicapa} (MLP) apila multiples capas de neuronas, formando una arquitectura de capas ocultas:

\begin{equation}
    \vec{h}^{(l)} = \phi\left(\mat{W}^{(l)} \vec{h}^{(l-1)} + \vec{b}^{(l)}\right)
    \label{eq:mlp_layer}
\end{equation}

donde $\vec{h}^{(l)}$ representa las activaciones de la capa $l$, con $\vec{h}^{(0)} = \vec{x}$ siendo la entrada.

\subsection{Funciones de Activacion}

Las funciones de activacion introducen no linealidad en la red, permitiendole aproximar funciones complejas:

\begin{definicion}[ReLU]
La funcion Rectified Linear Unit (ReLU) \cite{nair2010rectified} se define como:
\begin{equation}
    \text{ReLU}(x) = \max(0, x)
    \label{eq:relu}
\end{equation}
\end{definicion}

ReLU es la funcion de activacion mas utilizada en redes profundas debido a su simplicidad computacional y la mitigacion del problema de gradientes desvanecientes.

\begin{definicion}[Sigmoid]
La funcion sigmoide mapea valores al intervalo $(0, 1)$:
\begin{equation}
    \sigma(x) = \frac{1}{1 + e^{-x}}
    \label{eq:sigmoid}
\end{equation}
\end{definicion}

En este trabajo, la sigmoide se utiliza en la capa de salida para normalizar las coordenadas predichas al rango $[0, 1]$.

\subsection{Aprendizaje por Retropropagacion}

El entrenamiento de redes neuronales se realiza mediante el algoritmo de \textit{backpropagation} \cite{rumelhart1986learning}, que computa el gradiente de la funcion de perdida respecto a cada parametro usando la regla de la cadena:

\begin{equation}
    \frac{\partial \mathcal{L}}{\partial w_{ij}^{(l)}} = \frac{\partial \mathcal{L}}{\partial h_j^{(l)}} \cdot \frac{\partial h_j^{(l)}}{\partial w_{ij}^{(l)}}
    \label{eq:backprop}
\end{equation}

Los parametros se actualizan iterativamente usando descenso de gradiente:

\begin{equation}
    \theta_{t+1} = \theta_t - \eta \nabla_\theta \mathcal{L}
    \label{eq:sgd}
\end{equation}

donde $\eta$ es la tasa de aprendizaje (learning rate).

% -----------------------------------------------------------------------------
\section{Redes Neuronales Convolucionales}
\label{sec:cnn}
% -----------------------------------------------------------------------------

Las \textbf{Redes Neuronales Convolucionales} (\CNN) \cite{lecun1998gradient} son arquitecturas especializadas para el procesamiento de datos con estructura de cuadricula, como imagenes. Sus principales caracteristicas son la conectividad local, el compartimiento de pesos y la invarianza a traslaciones.

\subsection{Operacion de Convolucion}

La convolucion discreta en 2D entre una imagen $I$ y un kernel $K$ de tamano $k \times k$ se define como:

\begin{equation}
    (I * K)(i, j) = \sum_{m=0}^{k-1} \sum_{n=0}^{k-1} I(i+m, j+n) \cdot K(m, n)
    \label{eq:convolucion}
\end{equation}

En la practica, se utiliza la correlacion cruzada (sin voltear el kernel):

\begin{equation}
    S(i, j) = \sum_{m} \sum_{n} I(i+m, j+n) \cdot K(m, n)
    \label{eq:correlacion}
\end{equation}

\subsection{Capas de Pooling}

Las capas de \textit{pooling} reducen la dimensionalidad espacial de los mapas de caracteristicas, proporcionando invarianza local a traslaciones:

\begin{definicion}[Max Pooling]
Para una region de $p \times p$ pixeles:
\begin{equation}
    \text{MaxPool}(i, j) = \max_{m,n \in [0, p)} I(i \cdot s + m, j \cdot s + n)
    \label{eq:maxpool}
\end{equation}
donde $s$ es el stride (paso).
\end{definicion}

\begin{definicion}[Global Average Pooling]
Reduce cada canal a un escalar mediante el promedio global:
\begin{equation}
    \text{GAP}(c) = \frac{1}{H \times W} \sum_{i=1}^{H} \sum_{j=1}^{W} F_c(i, j)
    \label{eq:gap}
\end{equation}
donde $F_c$ es el mapa de caracteristicas del canal $c$ de dimensiones $H \times W$.
\end{definicion}

\subsection{Arquitectura Tipica de una CNN}

Una \CNN clasica sigue el patron:

\begin{equation}
    \text{Input} \rightarrow [\text{Conv} \rightarrow \text{ReLU} \rightarrow \text{Pool}]^N \rightarrow \text{FC} \rightarrow \text{Output}
    \label{eq:cnn_pipeline}
\end{equation}

Las capas convolucionales iniciales extraen caracteristicas de bajo nivel (bordes, texturas), mientras que las capas mas profundas capturan patrones semanticos de alto nivel.

% -----------------------------------------------------------------------------
\section{Arquitecturas de Deep Learning}
\label{sec:arquitecturas}
% -----------------------------------------------------------------------------

\subsection{LeNet-5 (1998)}

LeNet-5 \cite{lecun1998gradient} fue una de las primeras \CNN exitosas, disenada para el reconocimiento de digitos manuscritos. Su arquitectura incluye:

\begin{itemize}
    \item Dos capas convolucionales con kernels $5 \times 5$
    \item Dos capas de subsampling (average pooling)
    \item Tres capas fully connected
    \item Funcion de activacion tanh
\end{itemize}

Aunque modesta por estandares actuales, LeNet establecio los principios fundamentales de las \CNN.

\subsection{AlexNet (2012)}

AlexNet \cite{krizhevsky2012imagenet} revoluciono el campo de vision por computadora al ganar el desafio ImageNet 2012 con un margen significativo. Sus innovaciones clave fueron:

\begin{itemize}
    \item Uso de ReLU en lugar de tanh/sigmoid
    \item Dropout para regularizacion
    \item Data augmentation extensivo
    \item Entrenamiento en multiples GPUs
\end{itemize}

AlexNet demostro que las redes profundas entrenadas con grandes datasets pueden superar metodos tradicionales de vision por computadora.

\subsection{VGGNet (2014)}

VGGNet \cite{simonyan2014very} demostro la importancia de la profundidad de la red. Sus caracteristicas principales:

\begin{itemize}
    \item Kernels convolucionales pequenos ($3 \times 3$) exclusivamente
    \item Arquitectura uniforme y simple
    \item Profundidades de 16 y 19 capas (VGG-16, VGG-19)
    \item 138 millones de parametros
\end{itemize}

La intuicion clave es que dos capas de $3 \times 3$ tienen el mismo campo receptivo que una de $5 \times 5$, pero con menos parametros y mas no linealidad.

\subsection{ResNet (2015)}
\label{subsec:resnet}

\textbf{ResNet} (Residual Network) \cite{he2016deep} introdujo las \textit{conexiones residuales}, permitiendo entrenar redes de cientos de capas. Este avance resolvio el problema de degradacion observado en redes muy profundas.

\begin{definicion}[Bloque Residual]
Un bloque residual computa:
\begin{equation}
    \vec{y} = \mathcal{F}(\vec{x}, \{W_i\}) + \vec{x}
    \label{eq:residual_block}
\end{equation}
donde $\mathcal{F}$ representa las capas apiladas y $\vec{x}$ es la conexion de salto (skip connection).
\end{definicion}

\begin{figure}[htbp]
    \centering
    % Diagrama conceptual del bloque residual
    \begin{tikzpicture}[
        node distance=1.5cm,
        block/.style={rectangle, draw, minimum width=2cm, minimum height=0.8cm},
        arrow/.style={->, thick}
    ]
        % Nodos
        \node (input) {$\vec{x}$};
        \node[block, right=of input] (conv1) {Conv $3\times3$};
        \node[block, right=of conv1] (relu1) {ReLU};
        \node[block, right=of relu1] (conv2) {Conv $3\times3$};
        \node[right=of conv2] (sum) {$+$};
        \node[block, right=of sum] (relu2) {ReLU};
        \node[right=of relu2] (output) {$\vec{y}$};

        % Conexiones
        \draw[arrow] (input) -- (conv1);
        \draw[arrow] (conv1) -- (relu1);
        \draw[arrow] (relu1) -- (conv2);
        \draw[arrow] (conv2) -- (sum);
        \draw[arrow] (sum) -- (relu2);
        \draw[arrow] (relu2) -- (output);

        % Skip connection
        \draw[arrow, dashed] (input) -- ++(0,-1) -| (sum);
    \end{tikzpicture}
    \caption{Bloque residual basico de ResNet. La conexion de salto permite que el gradiente fluya directamente a capas anteriores.}
    \label{fig:residual_block}
\end{figure}

Las ventajas de las conexiones residuales incluyen:

\begin{enumerate}
    \item \textbf{Flujo de gradiente mejorado}: El gradiente puede fluir directamente a traves de las conexiones de salto, mitigando el problema de gradientes desvanecientes.
    \item \textbf{Aprendizaje de identidad}: Si las capas adicionales no son necesarias, la red puede aprender $\mathcal{F}(\vec{x}) \approx 0$, haciendo que el bloque aproxime la funcion identidad.
    \item \textbf{Regularizacion implicita}: Las conexiones de salto actuan como una forma de regularizacion.
\end{enumerate}

\textbf{ResNet-18}, utilizado en este trabajo, consta de:
\begin{itemize}
    \item Una capa convolucional inicial de $7 \times 7$
    \item 4 grupos de bloques residuales (2, 2, 2, 2 bloques)
    \item Global Average Pooling
    \item Una capa fully connected de salida
    \item Total: 11.7 millones de parametros
\end{itemize}

% -----------------------------------------------------------------------------
\section{Transfer Learning}
\label{sec:transfer_learning}
% -----------------------------------------------------------------------------

El \textbf{Transfer Learning} \cite{pan2009survey} consiste en transferir conocimiento aprendido en una tarea (dominio fuente) a otra tarea relacionada (dominio objetivo). En vision por computadora, tipicamente se utilizan redes preentrenadas en ImageNet como punto de partida.

\subsection{Motivacion}

\begin{itemize}
    \item \textbf{Datasets pequenos}: El dominio medico frecuentemente carece de grandes datasets etiquetados.
    \item \textbf{Eficiencia computacional}: Entrenar desde cero requiere mas recursos y tiempo.
    \item \textbf{Caracteristicas genericas}: Las primeras capas de una \CNN aprenden filtros genericos (bordes, texturas) utiles para multiples tareas.
\end{itemize}

\subsection{Estrategias de Transfer Learning}

\begin{enumerate}
    \item \textbf{Feature extraction}: Se congela el backbone preentrenado y solo se entrena una nueva cabeza de clasificacion/regresion.

    \item \textbf{Fine-tuning}: Despues de una fase inicial de feature extraction, se descongelan algunas o todas las capas del backbone para ajustar los pesos al nuevo dominio.

    \item \textbf{Learning rate diferenciado}: Se utilizan tasas de aprendizaje menores para las capas preentrenadas y mayores para las nuevas capas.
\end{enumerate}

En este trabajo se emplea un esquema de \textbf{entrenamiento en dos fases}:

\begin{algorithm}[H]
\caption{Entrenamiento en Dos Fases}
\label{alg:two_phase}
\begin{algorithmic}[1]
    \Require Red preentrenada $f$, dataset $\mathcal{D}$
    \Ensure Modelo fine-tuned $f^*$
    \State \textbf{Fase 1 (Feature Extraction):}
    \State \quad Congelar backbone
    \State \quad Entrenar cabeza con $\eta_{\text{head}} = 10^{-3}$ por 15 epocas
    \State \textbf{Fase 2 (Fine-tuning):}
    \State \quad Descongelar backbone
    \State \quad Entrenar con $\eta_{\text{backbone}} = 2 \times 10^{-5}$, $\eta_{\text{head}} = 2 \times 10^{-4}$ por 100 epocas
    \State \Return $f^*$
\end{algorithmic}
\end{algorithm}

% -----------------------------------------------------------------------------
\section{Mecanismos de Atencion}
\label{sec:atencion}
% -----------------------------------------------------------------------------

Los mecanismos de atencion permiten a las redes neuronales enfocarse selectivamente en partes relevantes de la entrada \cite{vaswani2017attention}.

\subsection{Self-Attention}

El mecanismo de \textit{self-attention} computa pesos de atencion basados en relaciones entre elementos de la misma secuencia:

\begin{equation}
    \text{Attention}(Q, K, V) = \softmax\left(\frac{QK^T}{\sqrt{d_k}}\right) V
    \label{eq:self_attention}
\end{equation}

donde $Q$ (queries), $K$ (keys) y $V$ (values) son proyecciones lineales de la entrada, y $d_k$ es la dimension de las keys.

\subsection{Coordinate Attention}
\label{subsec:coord_attention}

\textbf{Coordinate Attention} \cite{hou2021coordinate} es un mecanismo disenado especificamente para \CNN que captura dependencias espaciales de largo alcance mientras preserva informacion posicional precisa.

A diferencia del SE-Net \cite{hu2018squeeze} que solo considera atencion de canal, Coordinate Attention descompone la atencion espacial en dos direcciones ortogonales:

\begin{enumerate}
    \item \textbf{Pooling direccional}: Se aplica average pooling en cada direccion espacial:
    \begin{align}
        z_c^h(h) &= \frac{1}{W} \sum_{0 \leq i < W} x_c(h, i) \label{eq:pool_h} \\
        z_c^w(w) &= \frac{1}{H} \sum_{0 \leq j < H} x_c(j, w) \label{eq:pool_w}
    \end{align}

    \item \textbf{Codificacion conjunta}: Las representaciones se concatenan y procesan:
    \begin{equation}
        f = \delta(\text{BN}(F_1([\vec{z}^h, \vec{z}^w])))
        \label{eq:coord_encode}
    \end{equation}

    \item \textbf{Generacion de mapas de atencion}: Se generan mapas de atencion separados para cada dimension:
    \begin{align}
        g^h &= \sigma(F_h(f^h)) \label{eq:attn_h} \\
        g^w &= \sigma(F_w(f^w)) \label{eq:attn_w}
    \end{align}

    \item \textbf{Aplicacion de atencion}:
    \begin{equation}
        y_c(i, j) = x_c(i, j) \times g_c^h(i) \times g_c^w(j)
        \label{eq:coord_apply}
    \end{equation}
\end{enumerate}

\begin{figure}[htbp]
    \centering
    \includegraphics[width=0.8\textwidth]{coordinate_attention.png}
    \caption{Diagrama del modulo Coordinate Attention. El pooling direccional captura informacion espacial de largo alcance mientras preserva la localizacion precisa.}
    \label{fig:coord_attention}
\end{figure}

La ventaja de Coordinate Attention para deteccion de landmarks radica en que:
\begin{itemize}
    \item Captura relaciones espaciales globales (ej. alineacion vertical del eje central)
    \item Preserva informacion posicional precisa necesaria para regresion de coordenadas
    \item Es computacionalmente eficiente comparado con self-attention completo
\end{itemize}

% -----------------------------------------------------------------------------
\section{Funciones de Perdida para Deteccion de Landmarks}
\label{sec:loss_functions}
% -----------------------------------------------------------------------------

La eleccion de la funcion de perdida es critica para el rendimiento en problemas de regresion de coordenadas.

\subsection{Mean Squared Error (MSE)}

La funcion de perdida mas basica para regresion es el error cuadratico medio:

\begin{equation}
    \mathcal{L}_{\text{MSE}} = \frac{1}{N} \sum_{i=1}^{N} \|\vec{y}_i - \hat{\vec{y}}_i\|_2^2
    \label{eq:mse}
\end{equation}

\textbf{Limitaciones de MSE}:
\begin{itemize}
    \item Sensible a outliers (errores grandes dominan el gradiente)
    \item Trata igualmente errores pequenos y grandes
    \item Puede resultar en predicciones ``promedio'' sin precision fina
\end{itemize}

\subsection{Wing Loss}
\label{subsec:wing_loss}

\textbf{Wing Loss} \cite{feng2018wing} fue disenada especificamente para deteccion de landmarks faciales, proporcionando mayor sensibilidad a errores pequenos:

\begin{equation}
    \mathcal{L}_{\text{wing}}(x) = \begin{cases}
        \omega \ln(1 + |x|/\epsilon) & \text{si } |x| < \omega \\
        |x| - C & \text{en otro caso}
    \end{cases}
    \label{eq:wing_loss}
\end{equation}

donde:
\begin{itemize}
    \item $\omega$ es el umbral que separa los dos regimenes
    \item $\epsilon$ controla la curvatura de la parte logaritmica
    \item $C = \omega - \omega \ln(1 + \omega/\epsilon)$ asegura continuidad
\end{itemize}

\begin{figure}[htbp]
    \centering
    \begin{tikzpicture}
        \begin{axis}[
            xlabel={Error $|x|$},
            ylabel={Loss},
            legend pos=north west,
            grid=major,
            width=0.7\textwidth,
            height=0.4\textwidth,
            domain=0:20
        ]
            % MSE
            \addplot[blue, thick] {x^2/20};
            \addlegendentry{MSE (escalado)}

            % L1
            \addplot[green, thick] {x};
            \addlegendentry{L1}

            % Wing Loss (aproximacion)
            \addplot[red, thick] {x < 10 ? 10*ln(1 + x/2) : x - 10 + 10*ln(1 + 10/2)};
            \addlegendentry{Wing Loss}
        \end{axis}
    \end{tikzpicture}
    \caption{Comparacion de funciones de perdida. Wing Loss proporciona mayor sensibilidad para errores pequenos (region logaritmica) mientras mantiene robustez ante outliers (region lineal).}
    \label{fig:loss_comparison}
\end{figure}

\textbf{Ventajas de Wing Loss}:
\begin{enumerate}
    \item \textbf{Gradiente amplificado para errores pequenos}: La derivada en el origen es mayor que en MSE.
    \item \textbf{Robustez ante outliers}: La parte lineal evita que errores grandes dominen el entrenamiento.
    \item \textbf{Transicion suave}: Continuidad en $|x| = \omega$ evita discontinuidades en el gradiente.
\end{enumerate}

Para coordenadas normalizadas en $[0, 1]$, los parametros deben escalarse:

\begin{equation}
    \omega' = \frac{\omega}{S}, \quad \epsilon' = \frac{\epsilon}{S}
    \label{eq:wing_normalized}
\end{equation}

donde $S$ es el tamano de la imagen (224 en este trabajo).

\subsection{Adaptive Wing Loss}

\textbf{Adaptive Wing Loss} \cite{wang2019adaptive} extiende Wing Loss adaptando los parametros segun la dificultad de cada landmark:

\begin{equation}
    \mathcal{L}_{\text{AWL}}(y) = \begin{cases}
        \omega \ln(1 + |y/\epsilon|^{\alpha - y}) & \text{si } |y| < \theta \\
        A|y| - C & \text{en otro caso}
    \end{cases}
    \label{eq:adaptive_wing}
\end{equation}

donde $\alpha$ controla la forma de la curva y $\theta$ es el umbral adaptativo.

% -----------------------------------------------------------------------------
\section{Tecnicas de Regularizacion}
\label{sec:regularizacion}
% -----------------------------------------------------------------------------

La regularizacion previene el sobreajuste (overfitting) y mejora la generalizacion del modelo.

\subsection{Dropout}

\textbf{Dropout} \cite{srivastava2014dropout} desactiva aleatoriamente una fraccion $p$ de las neuronas durante el entrenamiento:

\begin{equation}
    \vec{h}_{\text{drop}} = \vec{m} \odot \vec{h}, \quad m_i \sim \text{Bernoulli}(1-p)
    \label{eq:dropout}
\end{equation}

donde $\odot$ denota el producto elemento a elemento.

Durante la inferencia, las activaciones se escalan por $(1-p)$ para mantener la misma esperanza. En este trabajo se utiliza $p = 0.3$, valor determinado experimentalmente como optimo.

\subsection{Normalizacion por Lotes y Grupos}

\textbf{Batch Normalization} \cite{ioffe2015batch} normaliza las activaciones usando estadisticas del mini-batch:

\begin{equation}
    \hat{x}_i = \frac{x_i - \mu_B}{\sqrt{\sigma_B^2 + \epsilon}}
    \label{eq:batchnorm}
\end{equation}

donde $\mu_B$ y $\sigma_B^2$ son la media y varianza del batch.

\textbf{Group Normalization} \cite{wu2018group} agrupa los canales y normaliza dentro de cada grupo:

\begin{equation}
    \hat{x}_i = \frac{x_i - \mu_G}{\sqrt{\sigma_G^2 + \epsilon}}
    \label{eq:groupnorm}
\end{equation}

\begin{nota}
En este trabajo se utiliza GroupNorm en lugar de BatchNorm en la cabeza de regresion porque:
\begin{itemize}
    \item Es independiente del tamano del batch
    \item Mas estable con batch sizes pequenos (8 en fine-tuning)
    \item Mejor rendimiento experimental en este problema
\end{itemize}
\end{nota}

\subsection{Data Augmentation}
\label{subsec:data_augmentation}

La \textbf{aumentacion de datos} genera variaciones de las muestras de entrenamiento, incrementando efectivamente el tamano del dataset:

\begin{itemize}
    \item \textbf{Flip horizontal}: Reflexion especular con intercambio de landmarks simetricos
    \item \textbf{Rotacion}: Rotacion aleatoria de $\pm 10\degree$
    \item \textbf{Jitter de color}: Variaciones de brillo ($\pm 20\%$) y contraste ($\pm 20\%$)
\end{itemize}

\begin{algorithm}[H]
\caption{Flip Horizontal con Intercambio de Landmarks}
\label{alg:flip}
\begin{algorithmic}[1]
    \Require Imagen $I$, landmarks $L \in \R^{15 \times 2}$, pares simetricos $P$
    \Ensure Imagen y landmarks transformados
    \State $I' \leftarrow \text{flip}(I, \text{axis}=\text{horizontal})$
    \State $L'[:, 0] \leftarrow 1 - L[:, 0]$ \Comment{Invertir coordenada X}
    \For{$(i, j) \in P$} \Comment{$P = \{(2,3), (4,5), (6,7), (11,12), (13,14)\}$}
        \State $L'[i], L'[j] \leftarrow L'[j], L'[i]$ \Comment{Intercambiar pares}
    \EndFor
    \State \Return $I'$, $L'$
\end{algorithmic}
\end{algorithm}

% -----------------------------------------------------------------------------
\section{Preprocesamiento de Imagenes Medicas}
\label{sec:preprocesamiento}
% -----------------------------------------------------------------------------

\subsection{CLAHE}
\label{subsec:clahe}

\textbf{CLAHE} (Contrast Limited Adaptive Histogram Equalization) \cite{pizer1987adaptive} es una tecnica de mejora de contraste local que divide la imagen en regiones (tiles) y ecualiza el histograma de cada region independientemente.

\begin{definicion}[CLAHE]
Para cada tile de la imagen:
\begin{enumerate}
    \item Calcular el histograma local
    \item Limitar (clip) los picos del histograma al valor \textit{clip\_limit}
    \item Redistribuir los pixeles recortados uniformemente
    \item Ecualizar el histograma resultante
    \item Interpolar bilinealmente en los bordes de tiles adyacentes
\end{enumerate}
\end{definicion}

Los parametros clave son:
\begin{itemize}
    \item \textbf{clip\_limit}: Controla el contraste maximo (valores tipicos: 1.0-4.0)
    \item \textbf{tile\_size}: Tamano de la cuadricula (valores tipicos: 4-16)
\end{itemize}

\begin{figure}[htbp]
    \centering
    \includegraphics[width=0.9\textwidth]{clahe_comparison.png}
    \caption{Efecto de CLAHE en radiografias de las tres categorias. CLAHE realza los bordes anatomicos, especialmente en zonas con consolidaciones pulmonares (COVID-19).}
    \label{fig:clahe_effect}
\end{figure}

En este trabajo se encontro que \texttt{clip\_limit=2.0} y \texttt{tile\_size=4} proporcionan el mejor balance para radiografias toracicas, particularmente efectivo para casos de COVID-19 donde las consolidaciones oscurecen los bordes pulmonares.

\subsection{Normalizacion ImageNet}

Las imagenes se normalizan usando las estadisticas de ImageNet para compatibilidad con los pesos preentrenados:

\begin{equation}
    \hat{I} = \frac{I - \vec{\mu}}{\vec{\sigma}}
    \label{eq:imagenet_norm}
\end{equation}

donde $\vec{\mu} = [0.485, 0.456, 0.406]$ y $\vec{\sigma} = [0.229, 0.224, 0.225]$.

% -----------------------------------------------------------------------------
\section{Ensemble Learning}
\label{sec:ensemble}
% -----------------------------------------------------------------------------

El \textbf{aprendizaje por ensemble} combina predicciones de multiples modelos para mejorar el rendimiento y reducir la varianza.

\subsection{Motivacion Teorica}

Sea $\hat{y}_i$ la prediccion del modelo $i$ con error $\epsilon_i$:

\begin{equation}
    \hat{y}_i = y + \epsilon_i
    \label{eq:model_error}
\end{equation}

Si los errores son independientes con media cero y varianza $\sigma^2$, el promedio de $M$ modelos tiene varianza:

\begin{equation}
    \text{Var}\left(\frac{1}{M}\sum_{i=1}^{M} \hat{y}_i\right) = \frac{\sigma^2}{M}
    \label{eq:ensemble_variance}
\end{equation}

En la practica, los errores no son completamente independientes, pero el ensemble aun reduce la varianza significativamente.

\subsection{Estrategias de Ensemble}

\begin{enumerate}
    \item \textbf{Promedio simple}: Todas las predicciones tienen peso igual
    \begin{equation}
        \hat{y}_{\text{ens}} = \frac{1}{M}\sum_{i=1}^{M} \hat{y}_i
        \label{eq:simple_avg}
    \end{equation}

    \item \textbf{Promedio ponderado}: Pesos basados en rendimiento individual
    \begin{equation}
        \hat{y}_{\text{ens}} = \sum_{i=1}^{M} w_i \hat{y}_i, \quad \sum_{i} w_i = 1
        \label{eq:weighted_avg}
    \end{equation}
\end{enumerate}

\begin{nota}
Experimentalmente se encontro que el promedio simple es tan efectivo como esquemas de ponderacion mas complejos para este problema, siempre que se excluyan modelos de rendimiento significativamente inferior.
\end{nota}

% -----------------------------------------------------------------------------
\section{Test-Time Augmentation}
\label{sec:tta}
% -----------------------------------------------------------------------------

\textbf{Test-Time Augmentation} (\TTA) aplica transformaciones durante la inferencia y promedia las predicciones, reduciendo la varianza sin reentrenar:

\begin{algorithm}[H]
\caption{Test-Time Augmentation para Landmarks}
\label{alg:tta}
\begin{algorithmic}[1]
    \Require Imagen $I$, modelo $f$
    \Ensure Prediccion promediada $\hat{L}$
    \State $\hat{L}_1 \leftarrow f(I)$ \Comment{Prediccion original}
    \State $I_{\text{flip}} \leftarrow \text{flip}(I)$
    \State $\hat{L}_2 \leftarrow f(I_{\text{flip}})$
    \State $\hat{L}_2 \leftarrow \text{inverse\_flip}(\hat{L}_2)$ \Comment{Corregir coordenadas e intercambiar pares}
    \State $\hat{L} \leftarrow (\hat{L}_1 + \hat{L}_2) / 2$
    \State \Return $\hat{L}$
\end{algorithmic}
\end{algorithm}

En este trabajo, \TTA proporciona una mejora consistente de 0.3-0.5 pixeles sin costo de entrenamiento adicional.

% -----------------------------------------------------------------------------
\section{Metricas de Evaluacion}
\label{sec:metricas}
% -----------------------------------------------------------------------------

\subsection{Error Euclidiano}

La metrica principal para deteccion de landmarks es el error euclidiano (distancia) entre prediccion y ground truth:

\begin{equation}
    e_i = \|\hat{\vec{p}}_i - \vec{p}_i\|_2 = \sqrt{(\hat{x}_i - x_i)^2 + (\hat{y}_i - y_i)^2}
    \label{eq:euclidean_error}
\end{equation}

\subsection{Metricas Agregadas}

\begin{itemize}
    \item \textbf{Error medio}: $\bar{e} = \frac{1}{N}\sum_{i=1}^{N} e_i$
    \item \textbf{Desviacion estandar}: $\sigma_e = \sqrt{\frac{1}{N}\sum_{i=1}^{N}(e_i - \bar{e})^2}$
    \item \textbf{Mediana}: Valor central ordenado, mas robusta ante outliers
    \item \textbf{Percentiles}: $P_{90}$, $P_{95}$ para caracterizar la cola de la distribucion
\end{itemize}

\subsection{Success Rate}

El \textit{success rate} mide el porcentaje de predicciones bajo un umbral dado:

\begin{equation}
    \text{SR}(\tau) = \frac{1}{N} \sum_{i=1}^{N} \mathbb{1}[e_i < \tau]
    \label{eq:success_rate}
\end{equation}

% -----------------------------------------------------------------------------
% FIN DEL CAPITULO
% -----------------------------------------------------------------------------

% =============================================================================
% CAPITULO 3: ESTADO DEL ARTE
% =============================================================================

\chapter{Estado del Arte}
\label{ch:estado_arte}

Este capitulo presenta una revision de la literatura relacionada con la deteccion de landmarks en imagenes medicas, comparando diferentes enfoques metodologicos e identificando las brechas de investigacion que este trabajo busca abordar.

% -----------------------------------------------------------------------------
\section{Deteccion de Landmarks en Imagenes Medicas}
\label{sec:landmarks_medicos}
% -----------------------------------------------------------------------------

La deteccion automatica de landmarks anatomicos tiene una larga historia en el procesamiento de imagenes medicas, evolucionando desde metodos basados en caracteristicas handcrafted hasta las aproximaciones actuales basadas en deep learning.

\subsection{Metodos Tradicionales}

Los enfoques clasicos para deteccion de landmarks incluyen:

\begin{itemize}
    \item \textbf{Active Shape Models (ASM)} \cite{cootes1995active}: Modelan la variabilidad de forma mediante analisis de componentes principales (PCA) y ajustan iterativamente el modelo a nuevas imagenes.

    \item \textbf{Active Appearance Models (AAM)} \cite{cootes2001active}: Extienden ASM incorporando informacion de textura ademas de forma.

    \item \textbf{Random Forests con caracteristicas de contexto} \cite{lindner2015fully}: Utilizan arboles de decision entrenados con patches locales para regresion de posicion.

    \item \textbf{Cascaded Pose Regression} \cite{dollar2010cascaded}: Refinamiento iterativo de predicciones mediante cascadas de regresores.
\end{itemize}

Estos metodos han sido ampliamente utilizados en analisis cefalometrico de radiografias dentales \cite{wang2016benchmark}, logrando errores del orden de 2-3 mm en resoluciones clinicas.

\subsection{Deep Learning para Landmarks Medicos}

La transicion a deep learning ha permitido mejoras significativas en precision y robustez:

\begin{table}[htbp]
    \centering
    \caption{Comparacion de metodos de deep learning para deteccion de landmarks medicos}
    \label{tab:dl_landmarks}
    \begin{tabular}{@{}llccc@{}}
        \toprule
        \textbf{Metodo} & \textbf{Modalidad} & \textbf{Landmarks} & \textbf{Error} & \textbf{Referencia} \\
        \midrule
        CNN + Heatmap & Cefalometrico & 19 & 1.17 mm & \cite{zhong2019attention} \\
        U-Net & Columna vertebral & 17 & 2.3 px & \cite{chen2019vertebrae} \\
        HRNet & Facial & 68 & 1.2 px & \cite{wang2020deep} \\
        SCN & Cefalometrico & 19 & 1.21 mm & \cite{payer2019integrating} \\
        \bottomrule
    \end{tabular}
\end{table}

% -----------------------------------------------------------------------------
\section{Enfoques: Heatmaps vs Regresion Directa}
\label{sec:heatmaps_vs_regresion}
% -----------------------------------------------------------------------------

Existen dos paradigmas principales para la prediccion de coordenadas de landmarks:

\subsection{Regresion Basada en Heatmaps}

En este enfoque, la red produce un mapa de calor gaussiano por cada landmark, donde el pico indica la ubicacion predicha:

\begin{equation}
    H_i(x, y) = \exp\left(-\frac{(x - x_i^*)^2 + (y - y_i^*)^2}{2\sigma^2}\right)
    \label{eq:heatmap}
\end{equation}

donde $(x_i^*, y_i^*)$ es la ubicacion del landmark $i$ y $\sigma$ controla el ancho de la gaussiana.

\textbf{Ventajas}:
\begin{itemize}
    \item Supervision densa: cada pixel recibe senal de entrenamiento
    \item Representacion espacialmente explicita
    \item Facilita el aprendizaje de relaciones espaciales
    \item Permite precision subpixel mediante soft-argmax
\end{itemize}

\textbf{Desventajas}:
\begin{itemize}
    \item Mayor costo computacional (decodificador requerido)
    \item Resolucion limitada por el tamano del heatmap
    \item Requiere postprocesamiento para extraer coordenadas
\end{itemize}

\textbf{Arquitecturas representativas}:
\begin{itemize}
    \item \textbf{Stacked Hourglass} \cite{newell2016stacked}: Arquitectura encoder-decoder repetida para refinamiento progresivo.
    \item \textbf{HRNet} \cite{sun2019deep}: Mantiene representaciones de alta resolucion a lo largo de la red.
    \item \textbf{SimpleBaseline} \cite{xiao2018simple}: ResNet + upsampling, demuestra que la simplicidad puede ser efectiva.
\end{itemize}

\subsection{Regresion Directa de Coordenadas}

Este enfoque predice directamente las coordenadas $(x, y)$ de cada landmark mediante capas fully connected:

\begin{equation}
    \hat{\vec{y}} = f_{\theta}(\vec{x}) \in \R^{2K}
    \label{eq:direct_regression}
\end{equation}

donde $K$ es el numero de landmarks.

\textbf{Ventajas}:
\begin{itemize}
    \item Arquitectura mas simple y ligera
    \item Inferencia mas rapida
    \item No requiere postprocesamiento
    \item Mejor manejo de landmarks fuera del campo de vision
\end{itemize}

\textbf{Desventajas}:
\begin{itemize}
    \item Supervision escasa (solo 2K valores por imagen)
    \item Puede tener dificultad con relaciones espaciales complejas
    \item Sensible a la inicializacion
\end{itemize}

\subsection{Justificacion de la Eleccion}

En este trabajo se opto por \textbf{regresion directa} debido a:

\begin{enumerate}
    \item \textbf{Tamano del dataset}: Con solo 957 imagenes, la regresion directa aprovecha mejor la supervision disponible.
    \item \textbf{Numero de landmarks}: Con 15 landmarks, la dimension de salida (30) es manejable.
    \item \textbf{Eficiencia}: Menor costo computacional para entrenamiento y despliegue.
    \item \textbf{Evidencia empirica}: Trabajos recientes \cite{feng2018wing} muestran que la regresion directa con funciones de perdida adecuadas alcanza rendimiento competitivo.
\end{enumerate}

% -----------------------------------------------------------------------------
\section{Deteccion de Landmarks en Radiografias de Torax}
\label{sec:landmarks_torax}
% -----------------------------------------------------------------------------

La deteccion de landmarks especificamente en radiografias toracicas ha recibido menos atencion que otras modalidades como radiografias cefalometricas o imagenes faciales.

\subsection{Trabajos Relacionados}

\textbf{Jaeger et al. (2014)} \cite{jaeger2014automatic} desarrollaron un sistema para segmentacion de campos pulmonares utilizando Active Shape Models, logrando precisiones del 95\% en la segmentacion pero sin reportar errores de localizacion de landmarks especificos.

\textbf{Candemir et al. (2014)} \cite{candemir2014lung} propusieron un metodo basado en atlas para segmentacion pulmonar, identificando contornos pero no puntos anatomicos discretos.

\textbf{Rajpurkar et al. (2017)} \cite{rajpurkar2017chexnet} desarrollaron CheXNet para deteccion de patologias, demostrando que redes profundas pueden superar a radiologos en tareas especificas, aunque no abordaron la localizacion de landmarks.

\textbf{Wang et al. (2020)} \cite{wang2020covid} crearon COVID-Net para clasificacion de COVID-19 en radiografias, evidenciando el interes creciente en analisis automatizado de radiografias toracicas durante la pandemia.

\subsection{Gaps Identificados}

La revision de literatura revela varias brechas de investigacion:

\begin{enumerate}
    \item \textbf{Escasez de benchmarks}: A diferencia del dominio cefalometrico (con datasets estandar como el del ISBI Challenge), no existen benchmarks publicos establecidos para landmarks en radiografias toracicas.

    \item \textbf{Foco en clasificacion}: La mayoria de trabajos recientes se enfocan en clasificacion de patologias, no en localizacion de estructuras anatomicas.

    \item \textbf{Evaluacion limitada en patologia}: Los pocos trabajos existentes tipicamente evaluan solo en imagenes normales, sin considerar el impacto de condiciones patologicas.

    \item \textbf{Falta de analisis por landmark}: Los reportes agregados (error medio global) ocultan variabilidad significativa entre landmarks de diferente dificultad.
\end{enumerate}

% -----------------------------------------------------------------------------
\section{Funciones de Perdida Especializadas}
\label{sec:loss_estado_arte}
% -----------------------------------------------------------------------------

\subsection{Evolucion de Loss Functions para Landmarks}

La eleccion de la funcion de perdida ha demostrado ser critica para el rendimiento:

\begin{table}[htbp]
    \centering
    \caption{Evolucion de funciones de perdida para deteccion de landmarks}
    \label{tab:loss_evolution}
    \begin{tabular}{@{}lp{5cm}l@{}}
        \toprule
        \textbf{Loss} & \textbf{Caracteristica} & \textbf{Referencia} \\
        \midrule
        MSE/L2 & Basico, sensible a outliers & - \\
        Smooth L1 & Robustez ante outliers & \cite{girshick2015fast} \\
        Wing Loss & Mayor sensibilidad a errores pequenos & \cite{feng2018wing} \\
        Adaptive Wing & Parametros adaptativos por landmark & \cite{wang2019adaptive} \\
        DSNT & Gradientes suaves para heatmaps & \cite{nibali2018numerical} \\
        \bottomrule
    \end{tabular}
\end{table}

\subsection{Wing Loss en la Literatura}

\textbf{Feng et al. (2018)} \cite{feng2018wing} demostraron que Wing Loss supera a MSE y Smooth L1 en deteccion de landmarks faciales, con mejoras del 10-15\% en datasets estandar como AFLW y COFW.

\textbf{Wu et al. (2018)} \cite{wu2018look} combinaron Wing Loss con mecanismos de atencion para deteccion de landmarks faciales, logrando resultados estado del arte en 300-W.

El exito de Wing Loss se atribuye a su comportamiento diferenciado:
\begin{itemize}
    \item Gradientes mayores para errores pequenos (mejor precision fina)
    \item Gradientes acotados para errores grandes (robustez ante outliers)
\end{itemize}

% -----------------------------------------------------------------------------
\section{Mecanismos de Atencion en Vision por Computadora}
\label{sec:atencion_estado_arte}
% -----------------------------------------------------------------------------

Los mecanismos de atencion han revolucionado el procesamiento de imagenes:

\subsection{Atencion de Canal}

\textbf{Squeeze-and-Excitation Networks (SE-Net)} \cite{hu2018squeeze} introducen recalibracion de canales mediante:

\begin{equation}
    \vec{s} = \sigma(\mat{W}_2 \delta(\mat{W}_1 \text{GAP}(\vec{F})))
    \label{eq:se_net}
\end{equation}

donde GAP es Global Average Pooling. Los pesos $\vec{s}$ reescalan los canales segun su importancia.

\subsection{Atencion Espacial}

\textbf{CBAM} \cite{woo2018cbam} combina atencion de canal y espacial secuencialmente, aplicando pooling espacial para generar mapas de atencion 2D.

\subsection{Coordinate Attention}

\textbf{Hou et al. (2021)} \cite{hou2021coordinate} propusieron Coordinate Attention, que:
\begin{itemize}
    \item Codifica informacion posicional en la atencion de canal
    \item Captura dependencias espaciales de largo alcance
    \item Mantiene precision posicional mediante descomposicion direccional
\end{itemize}

Este mecanismo es particularmente adecuado para tareas de localizacion precisa como deteccion de landmarks.

% -----------------------------------------------------------------------------
\section{Preprocesamiento para Imagenes Medicas}
\label{sec:preprocesamiento_estado_arte}
% -----------------------------------------------------------------------------

\subsection{Ecualizacion de Histograma}

El contraste en radiografias varia significativamente debido a diferencias en equipos y parametros de adquisicion. Las tecnicas de ecualizacion buscan normalizar esta variabilidad:

\begin{itemize}
    \item \textbf{Histogram Equalization (HE)}: Ecualizacion global, puede sobreamplificar ruido.
    \item \textbf{AHE (Adaptive)}: Ecualizacion local por regiones, mejor contraste pero puede introducir artefactos.
    \item \textbf{CLAHE}: Limita el contraste maximo, previniendo sobreamplificacion.
\end{itemize}

\textbf{Pizer et al. (1987)} \cite{pizer1987adaptive} introdujeron CLAHE, demostrando mejoras significativas en la visualizacion de estructuras anatomicas.

\textbf{Reza (2004)} \cite{reza2004realization} analizo el impacto del clip limit en diferentes modalidades, recomendando valores entre 2.0 y 4.0 para radiografias.

\subsection{CLAHE para COVID-19}

Estudios recientes han explorado CLAHE especificamente para imagenes de COVID-19:

\textbf{Horry et al. (2020)} \cite{horry2020covid} encontraron que CLAHE mejora la deteccion de consolidaciones pulmonares en redes de clasificacion.

\textbf{Abbas et al. (2021)} \cite{abbas2021classification} reportaron mejoras del 5-10\% en clasificacion de COVID-19 al aplicar CLAHE como preprocesamiento.

Sin embargo, \textbf{ninguno de estos trabajos evaluo el impacto de CLAHE en tareas de localizacion de landmarks}, lo cual representa una contribucion de esta tesis.

% -----------------------------------------------------------------------------
\section{Ensemble Learning en Deep Learning}
\label{sec:ensemble_estado_arte}
% -----------------------------------------------------------------------------

\subsection{Fundamentos}

El ensemble de modelos es una tecnica establecida para mejorar rendimiento:

\textbf{Dietterich (2000)} \cite{dietterich2000ensemble} formalizó las tres razones por las que los ensembles funcionan:
\begin{enumerate}
    \item \textbf{Estadistica}: Promediar reduce varianza cuando los datos son limitados
    \item \textbf{Computacional}: Diferentes optimos locales pueden promediarse
    \item \textbf{Representacional}: El espacio de hipotesis efectivo se expande
\end{enumerate}

\subsection{Estrategias de Diversificacion}

Para que un ensemble sea efectivo, los modelos deben ser diversos:

\begin{itemize}
    \item \textbf{Diferentes seeds}: Inicializacion aleatoria diferente
    \item \textbf{Diferentes arquitecturas}: Combinacion de modelos heterogeneos
    \item \textbf{Diferentes subconjuntos de datos}: Bagging
    \item \textbf{Diferentes hiperparametros}: Grid search con seleccion de top-K
\end{itemize}

\subsection{Ensembles en Deteccion de Landmarks}

\textbf{Sun et al. (2015)} \cite{sun2015deep} demostraron que ensembles de CNNs mejoran 10-15\% sobre modelos individuales en deteccion de landmarks faciales.

\textbf{Kowalski et al. (2017)} \cite{kowalski2017deep} encontraron que la diversidad en inicializacion es mas importante que la diversidad arquitectonica para tareas de regresion.

% -----------------------------------------------------------------------------
\section{Test-Time Augmentation}
\label{sec:tta_estado_arte}
% -----------------------------------------------------------------------------

\TTA ha demostrado ser efectivo en multiples dominios:

\textbf{Krizhevsky et al. (2012)} \cite{krizhevsky2012imagenet} utilizaron TTA (10 crops + flip) en ImageNet, mejorando precision en 1-2\%.

\textbf{Szegedy et al. (2015)} \cite{szegedy2015going} sistematizaron el uso de TTA para clasificacion.

\textbf{Moshkov et al. (2020)} \cite{moshkov2020test} analizaron TTA especificamente para segmentacion medica, encontrando mejoras consistentes del 1-3\%.

Para \textbf{regresion de coordenadas}, TTA requiere consideraciones especiales:
\begin{itemize}
    \item Las transformaciones geometricas deben invertirse en las predicciones
    \item Para flip horizontal, los landmarks simetricos deben intercambiarse
    \item Rotaciones requieren transformacion inversa de coordenadas
\end{itemize}

% -----------------------------------------------------------------------------
\section{Datasets de Referencia}
\label{sec:datasets}
% -----------------------------------------------------------------------------

\begin{table}[htbp]
    \centering
    \caption{Datasets relevantes para deteccion de landmarks medicos}
    \label{tab:datasets}
    \begin{tabular}{@{}llccc@{}}
        \toprule
        \textbf{Dataset} & \textbf{Modalidad} & \textbf{Imagenes} & \textbf{Landmarks} & \textbf{Publico} \\
        \midrule
        ISBI 2015 & Cefalometrico & 400 & 19 & Si \\
        300-W & Facial & 3,148 & 68 & Si \\
        AFLW & Facial & 25,993 & 21 & Si \\
        ChestX-ray14 & Torax & 112,120 & 0* & Si \\
        COVID-CT & Torax & 349 & 0* & Si \\
        \textbf{Este trabajo} & \textbf{Torax} & \textbf{957} & \textbf{15} & \textbf{No**} \\
        \bottomrule
    \end{tabular}
    \begin{tablenotes}
        \small
        \item * Datasets sin anotaciones de landmarks (solo etiquetas de patologia)
        \item ** Dataset propietario utilizado en este trabajo
    \end{tablenotes}
\end{table}

La falta de datasets publicos con anotaciones de landmarks en radiografias toracicas representa una barrera significativa para la investigacion en esta area.

% -----------------------------------------------------------------------------
\section{Sintesis y Posicionamiento}
\label{sec:sintesis}
% -----------------------------------------------------------------------------

\subsection{Resumen del Estado del Arte}

La \cref{tab:estado_arte_resumen} sintetiza las principales tecnicas y sus caracteristicas:

\begin{table}[htbp]
    \centering
    \caption{Resumen de tecnicas del estado del arte}
    \label{tab:estado_arte_resumen}
    \begin{tabular}{@{}p{3cm}p{4cm}p{4cm}@{}}
        \toprule
        \textbf{Componente} & \textbf{Estado del Arte} & \textbf{Este Trabajo} \\
        \midrule
        Backbone & ResNet, HRNet, EfficientNet & ResNet-18 \\
        Enfoque & Heatmaps, DSNT & Regresion directa \\
        Atencion & SE-Net, CBAM & Coordinate Attention \\
        Loss & Wing, AWL & Wing Loss \\
        Preprocesamiento & CLAHE (clasificacion) & CLAHE (localizacion) \\
        Ensemble & Heterogeneo, bagging & Homogeneo, multi-seed \\
        TTA & Clasificacion, segmentacion & Regresion de landmarks \\
        \bottomrule
    \end{tabular}
\end{table}

\subsection{Contribuciones Respecto al Estado del Arte}

Este trabajo contribuye al estado del arte en los siguientes aspectos:

\begin{enumerate}
    \item \textbf{Primera evaluacion sistematica de CLAHE para localizacion de landmarks en radiografias toracicas}, demostrando su efectividad particularmente para casos de COVID-19.

    \item \textbf{Analisis detallado por landmark y categoria}, proporcionando insights sobre la dificultad relativa de diferentes puntos anatomicos.

    \item \textbf{Demostracion de la efectividad del ensemble selectivo}, mostrando que excluir modelos de bajo rendimiento es mas efectivo que esquemas de ponderacion.

    \item \textbf{Validacion de regresion directa con Wing Loss} como alternativa viable a heatmaps para datasets pequenos.

    \item \textbf{Documentacion exhaustiva del proceso de desarrollo}, facilitando reproducibilidad y extension del trabajo.
\end{enumerate}

% -----------------------------------------------------------------------------
% FIN DEL CAPITULO
% -----------------------------------------------------------------------------

% =============================================================================
% CAPITULO 4: METODOLOGIA
% =============================================================================

\chapter{Metodologia}
\label{ch:metodologia}

Este capitulo describe en detalle la metodologia desarrollada para la deteccion de landmarks anatomicos en radiografias de torax. Se presenta la descripcion del dataset, la arquitectura propuesta, el pipeline de preprocesamiento, la estrategia de entrenamiento y las metricas de evaluacion.

% -----------------------------------------------------------------------------
\section{Descripcion del Dataset}
\label{sec:dataset}
% -----------------------------------------------------------------------------

\subsection{Origen y Caracteristicas}

El dataset utilizado en este trabajo consiste en 957 radiografias de torax en proyeccion posteroanterior (PA), provenientes de tres categorias diagnosticas:

\begin{table}[htbp]
    \centering
    \caption{Distribucion del dataset por categoria diagnostica}
    \label{tab:dataset_distribution}
    \begin{tabular}{@{}lccc@{}}
        \toprule
        \textbf{Categoria} & \textbf{Muestras} & \textbf{Porcentaje} & \textbf{Descripcion} \\
        \midrule
        Normal & 468 & 48.9\% & Radiografias sin hallazgos patologicos \\
        COVID-19 & 306 & 32.0\% & Pacientes con PCR positiva \\
        Neumonia Viral & 183 & 19.1\% & Neumonia viral no-COVID \\
        \midrule
        \textbf{Total} & \textbf{957} & \textbf{100\%} & \\
        \bottomrule
    \end{tabular}
\end{table}

Las caracteristicas tecnicas de las imagenes son:
\begin{itemize}
    \item \textbf{Formato}: PNG, escala de grises convertida a RGB
    \item \textbf{Resolucion original}: $299 \times 299$ pixeles
    \item \textbf{Resolucion de entrada al modelo}: $224 \times 224$ pixeles
    \item \textbf{Profundidad de bits}: 8 bits por canal
\end{itemize}

\subsection{Anotaciones de Landmarks}

Cada imagen cuenta con anotaciones de 15 landmarks anatomicos, etiquetados manualmente por expertos. Las coordenadas estan almacenadas en un archivo CSV maestro con el formato:

\begin{lstlisting}[caption={Formato de coordenadas (ejemplo)},label={lst:coords_format}]
nombre_imagen, L1_x, L1_y, L2_x, L2_y, ..., L15_x, L15_y
COVID_001.png, 149, 45, 148, 254, ..., 95, 235
\end{lstlisting}

La \cref{tab:landmarks_descripcion} describe cada landmark anatomico:

\begin{table}[htbp]
    \centering
    \caption{Descripcion de los 15 landmarks anatomicos}
    \label{tab:landmarks_descripcion}
    \begin{tabular}{@{}clp{6cm}@{}}
        \toprule
        \textbf{ID} & \textbf{Nombre} & \textbf{Descripcion Anatomica} \\
        \midrule
        L1 & Superior & Mediastino superior, cerca de la traquea \\
        L2 & Inferior & Mediastino inferior, region vertebral \\
        L3 & Apex Izquierdo & Vertice del pulmon izquierdo \\
        L4 & Apex Derecho & Vertice del pulmon derecho \\
        L5 & Hilio Izquierdo & Region hiliar izquierda \\
        L6 & Hilio Derecho & Region hiliar derecha \\
        L7 & Base Izquierda & Base del pulmon izquierdo \\
        L8 & Base Derecha & Base del pulmon derecho \\
        L9 & Centro Superior & Punto central, 1/4 del eje \\
        L10 & Centro Medio & Punto central, 1/2 del eje \\
        L11 & Centro Inferior & Punto central, 3/4 del eje \\
        L12 & Borde Superior Izq & Borde superior izquierdo \\
        L13 & Borde Superior Der & Borde superior derecho \\
        L14 & Seno Costofrenico Izq & Angulo costofrenico izquierdo \\
        L15 & Seno Costofrenico Der & Angulo costofrenico derecho \\
        \bottomrule
    \end{tabular}
\end{table}

\subsection{Division de Datos}

El dataset se dividio estratificadamente para mantener las proporciones de cada categoria:

\begin{table}[htbp]
    \centering
    \caption{Division del dataset en conjuntos de entrenamiento, validacion y prueba}
    \label{tab:data_split}
    \begin{tabular}{@{}lcccc@{}}
        \toprule
        \textbf{Conjunto} & \textbf{Total} & \textbf{Normal} & \textbf{COVID-19} & \textbf{Neumonia V.} \\
        \midrule
        Entrenamiento & 717 (74.9\%) & 351 & 229 & 137 \\
        Validacion & 144 (15.0\%) & 70 & 46 & 28 \\
        Prueba & 96 (10.0\%) & 47 & 31 & 18 \\
        \midrule
        \textbf{Total} & \textbf{957} & \textbf{468} & \textbf{306} & \textbf{183} \\
        \bottomrule
    \end{tabular}
\end{table}

\begin{nota}
Se verifico rigurosamente que no existe solapamiento entre conjuntos (data leakage). Ninguna imagen aparece en mas de un conjunto.
\end{nota}

% -----------------------------------------------------------------------------
\section{Analisis Exploratorio de Datos}
\label{sec:eda}
% -----------------------------------------------------------------------------

\subsection{Variabilidad por Landmark}

El analisis de la variabilidad (desviacion estandar) de cada landmark revela diferencias significativas en dificultad:

\begin{table}[htbp]
    \centering
    \caption{Variabilidad de landmarks en el ground truth}
    \label{tab:landmark_variability}
    \begin{tabular}{@{}lccl@{}}
        \toprule
        \textbf{Landmark} & \textbf{$\sigma_x$ (px)} & \textbf{$\sigma_y$ (px)} & \textbf{Dificultad} \\
        \midrule
        L14 (Costofrenico Izq) & 18.7 & 30.2 & Alta \\
        L15 (Costofrenico Der) & 17.9 & 29.5 & Alta \\
        L2 (Inferior) & 15.3 & 27.8 & Alta \\
        L7 (Base Izq) & 24.1 & 17.3 & Media-Alta \\
        L8 (Base Der) & 23.5 & 16.8 & Media-Alta \\
        L3-L6 (Apices, Hilios) & 12-16 & 10-14 & Media \\
        L9 (Centro Superior) & 8.4 & 17.8 & \textbf{Baja} \\
        L10 (Centro Medio) & 9.1 & 18.2 & \textbf{Baja} \\
        L1 (Superior) & 10.2 & 19.1 & Baja \\
        \bottomrule
    \end{tabular}
\end{table}

\textbf{Observaciones clave}:
\begin{itemize}
    \item Los landmarks centrales (L9, L10, L11) tienen menor variabilidad debido a su posicion sobre el eje mediastinico.
    \item Los senos costofrenicos (L14, L15) son los mas variables, reflejando la diversidad anatomica natural.
    \item La variabilidad en $y$ tiende a ser mayor que en $x$ para landmarks del eje central.
\end{itemize}

\subsection{Descubrimientos Geometricos del Etiquetado}
\label{subsec:geometria}

Un analisis detallado del proceso de etiquetado revelo una estructura geometrica subyacente:

\begin{resultadoclave}
Los landmarks centrales L9, L10, L11 dividen el eje L1-L2 en exactamente 4 partes iguales:
\begin{equation}
    L_i = L_1 + t_i \cdot (L_2 - L_1), \quad t_i \in \{0.25, 0.50, 0.75\}
    \label{eq:eje_parametrico}
\end{equation}
El error de alineacion medido es de solo \textbf{1.34 pixeles}.
\end{resultadoclave}

Esta estructura perfecta se debe al proceso de etiquetado manual:
\begin{enumerate}
    \item Primero se identifican L1 (superior) y L2 (inferior)
    \item Se traza una linea recta entre ellos (eje central)
    \item L9, L10, L11 se colocan automaticamente dividiendo el eje en 4 partes
    \item Los landmarks laterales se marcan desplazandose horizontalmente desde el eje
\end{enumerate}

\subsection{Asimetria en el Ground Truth}

Contraintuitivamente, los pares simetricos en el ground truth no son perfectamente simetricos:

\begin{table}[htbp]
    \centering
    \caption{Asimetria de pares bilaterales en el ground truth}
    \label{tab:asymmetry}
    \begin{tabular}{@{}lcc@{}}
        \toprule
        \textbf{Par} & \textbf{Error X (px)} & \textbf{Diferencia Y (px)} \\
        \midrule
        L3-L4 (Apices) & 4.8 & 7.9 \\
        L5-L6 (Hilios) & 2.9 & 9.0 \\
        L7-L8 (Bases) & 5.1 & 9.7 \\
        L12-L13 (Bordes Sup) & 9.5 & 4.2 \\
        L14-L15 (Costofrenicos) & 8.4 & 10.2 \\
        \midrule
        \textbf{Promedio} & 6.1 & 8.2 \\
        \bottomrule
    \end{tabular}
\end{table}

Esta asimetria refleja la combinacion de anatomia real asimetrica y variabilidad del etiquetado manual. Es crucial \textbf{no forzar simetria perfecta} en el modelo.

% -----------------------------------------------------------------------------
\section{Arquitectura Propuesta}
\label{sec:arquitectura}
% -----------------------------------------------------------------------------

\subsection{Vision General}

La arquitectura propuesta consiste en cuatro componentes principales:

\begin{figure}[htbp]
    \centering
    \includegraphics[width=\textwidth]{model_architecture.png}
    \caption{Arquitectura completa del modelo propuesto. La imagen de entrada pasa por ResNet-18 (backbone), Coordinate Attention, y una cabeza de regresion profunda para producir 30 coordenadas normalizadas.}
    \label{fig:arquitectura_completa}
\end{figure}

\begin{enumerate}
    \item \textbf{Backbone}: ResNet-18 preentrenado en ImageNet
    \item \textbf{Modulo de Atencion}: Coordinate Attention
    \item \textbf{Cabeza de Regresion}: Deep Head con GroupNorm
    \item \textbf{Activacion de Salida}: Sigmoid para normalizar a $[0, 1]$
\end{enumerate}

\subsection{Backbone: ResNet-18}

Se selecciono ResNet-18 como backbone por:

\begin{itemize}
    \item \textbf{Balance precision-eficiencia}: 11.7M parametros, suficiente capacidad sin overfitting
    \item \textbf{Pesos preentrenados}: Aprovecha features aprendidos en ImageNet
    \item \textbf{Conexiones residuales}: Facilitan el flujo de gradiente
    \item \textbf{Amplio soporte}: Implementacion optimizada en PyTorch
\end{itemize}

La capa fully connected final de ResNet-18 se reemplaza por \texttt{nn.Identity()}, extrayendo el vector de features de dimension 512.

\subsection{Coordinate Attention}

El modulo de Coordinate Attention se inserta despues del backbone:

\begin{lstlisting}[caption={Implementacion de Coordinate Attention},label={lst:coord_attn}]
class CoordinateAttention(nn.Module):
    def __init__(self, in_channels, reduction=32):
        super().__init__()
        mip = max(8, in_channels // reduction)

        self.pool_h = nn.AdaptiveAvgPool2d((None, 1))
        self.pool_w = nn.AdaptiveAvgPool2d((1, None))

        self.conv1 = nn.Conv2d(in_channels, mip, 1)
        self.bn1 = nn.BatchNorm2d(mip)
        self.act = nn.Hardswish()

        self.conv_h = nn.Conv2d(mip, in_channels, 1)
        self.conv_w = nn.Conv2d(mip, in_channels, 1)

    def forward(self, x):
        identity = x
        n, c, h, w = x.size()

        # Pooling direccional
        x_h = self.pool_h(x)  # (n, c, h, 1)
        x_w = self.pool_w(x).permute(0, 1, 3, 2)  # (n, c, w, 1)

        # Codificacion conjunta
        y = torch.cat([x_h, x_w], dim=2)
        y = self.conv1(y)
        y = self.bn1(y)
        y = self.act(y)

        # Separacion y generacion de mapas
        x_h, x_w = torch.split(y, [h, w], dim=2)
        x_w = x_w.permute(0, 1, 3, 2)

        a_h = self.conv_h(x_h).sigmoid()
        a_w = self.conv_w(x_w).sigmoid()

        return identity * a_w * a_h
\end{lstlisting}

\subsection{Deep Head con GroupNorm}

La cabeza de regresion utiliza GroupNorm para mayor estabilidad:

\begin{lstlisting}[caption={Cabeza de regresion profunda},label={lst:deep_head}]
self.deep_head = nn.Sequential(
    nn.Linear(512, 768),
    nn.GroupNorm(32, 768),
    nn.ReLU(inplace=True),
    nn.Dropout(0.3),

    nn.Linear(768, 256),
    nn.GroupNorm(16, 256),
    nn.ReLU(inplace=True),
    nn.Dropout(0.3),

    nn.Linear(256, 30),
    nn.Sigmoid()
)
\end{lstlisting}

\textbf{Justificacion de hiperparametros}:
\begin{itemize}
    \item \textbf{hidden\_dim=768}: Determinado experimentalmente como optimo (ver Sesion 9)
    \item \textbf{dropout=0.3}: Balance entre regularizacion y capacidad
    \item \textbf{GroupNorm}: Mas estable que BatchNorm con batch size pequeno
\end{itemize}

\subsection{Resumen de Parametros}

\begin{table}[htbp]
    \centering
    \caption{Parametros del modelo por componente}
    \label{tab:model_params}
    \begin{tabular}{@{}lrr@{}}
        \toprule
        \textbf{Componente} & \textbf{Parametros} & \textbf{Entrenables (Phase 1)} \\
        \midrule
        ResNet-18 Backbone & 11,176,512 & 0 (congelado) \\
        Coordinate Attention & 54,784 & 0 (congelado) \\
        Deep Head & 611,038 & 611,038 \\
        \midrule
        \textbf{Total} & \textbf{11,842,334} & \textbf{611,038} \\
        \bottomrule
    \end{tabular}
\end{table}

% -----------------------------------------------------------------------------
\section{Pipeline de Preprocesamiento}
\label{sec:preprocesamiento_pipeline}
% -----------------------------------------------------------------------------

\subsection{Flujo de Preprocesamiento}

\begin{algorithm}[H]
\caption{Pipeline de Preprocesamiento}
\label{alg:preprocessing}
\begin{algorithmic}[1]
    \Require Imagen $I$ (299$\times$299), Coordenadas $C$ (15$\times$2)
    \Ensure Tensor normalizado $\hat{I}$, Coordenadas normalizadas $\hat{C}$

    \State \textbf{// CLAHE en espacio LAB}
    \State $I_{\text{LAB}} \leftarrow \text{RGB2LAB}(I)$
    \State $I_{\text{LAB}}[:,:,0] \leftarrow \text{CLAHE}(I_{\text{LAB}}[:,:,0], \text{clip}=2.0, \text{tile}=4)$
    \State $I \leftarrow \text{LAB2RGB}(I_{\text{LAB}})$

    \State \textbf{// Normalizacion de coordenadas}
    \State $\hat{C} \leftarrow C / 299$ \Comment{Normalizar a [0, 1]}

    \State \textbf{// Resize}
    \State $I \leftarrow \text{Resize}(I, 224 \times 224)$

    \State \textbf{// Augmentation (solo entrenamiento)}
    \If{training}
        \State $I, \hat{C} \leftarrow \text{RandomHorizontalFlip}(I, \hat{C}, p=0.5)$
        \State $I, \hat{C} \leftarrow \text{RandomRotation}(I, \hat{C}, \pm 10\degree)$
        \State $I \leftarrow \text{ColorJitter}(I, \text{brightness}=0.2, \text{contrast}=0.2)$
    \EndIf

    \State \textbf{// Conversion a tensor y normalizacion ImageNet}
    \State $\hat{I} \leftarrow \text{ToTensor}(I)$
    \State $\hat{I} \leftarrow \text{Normalize}(\hat{I}, \mu_{\text{ImageNet}}, \sigma_{\text{ImageNet}})$

    \State \Return $\hat{I}$, $\hat{C}$
\end{algorithmic}
\end{algorithm}

\subsection{CLAHE: Parametros Optimizados}

Los parametros de CLAHE fueron optimizados experimentalmente:

\begin{table}[htbp]
    \centering
    \caption{Comparacion de configuraciones CLAHE}
    \label{tab:clahe_params}
    \begin{tabular}{@{}cccc@{}}
        \toprule
        \textbf{clip\_limit} & \textbf{tile\_size} & \textbf{Error (px)} & \textbf{Observacion} \\
        \midrule
        Sin CLAHE & - & 8.93 & Baseline \\
        2.0 & 8 & 8.18 & Mejora general \\
        2.0 & 16 & 8.82 & Tiles muy grandes \\
        1.5 & 4 & 8.12 & Bueno \\
        \textbf{2.0} & \textbf{4} & \textbf{7.84} & \textbf{Optimo} \\
        3.0 & 4 & 8.23 & Demasiado contraste \\
        \bottomrule
    \end{tabular}
\end{table}

\textbf{Insight clave}: Tiles mas pequenos (4) proporcionan mejor realce local de bordes, crucial para detectar landmarks en zonas con consolidaciones pulmonares.

\subsection{Data Augmentation con Landmarks}

El flip horizontal requiere manejo especial de landmarks simetricos:

\begin{lstlisting}[caption={Flip horizontal con intercambio de pares},label={lst:flip}]
SYMMETRIC_PAIRS = [(2, 3), (4, 5), (6, 7), (11, 12), (13, 14)]

def flip_horizontal(image, landmarks):
    # Flip imagen
    image = torch.flip(image, dims=[2])

    # Invertir coordenada X
    landmarks = landmarks.clone()
    landmarks[:, 0] = 1.0 - landmarks[:, 0]

    # Intercambiar pares simetricos
    for left, right in SYMMETRIC_PAIRS:
        landmarks[[left, right]] = landmarks[[right, left]].clone()

    return image, landmarks
\end{lstlisting}

% -----------------------------------------------------------------------------
\section{Funcion de Perdida}
\label{sec:loss_function}
% -----------------------------------------------------------------------------

\subsection{Wing Loss Normalizada}

Se utiliza Wing Loss adaptada para coordenadas normalizadas:

\begin{equation}
    \mathcal{L}_{\text{wing}}(x) = \begin{cases}
        \omega' \ln(1 + |x|/\epsilon') & \text{si } |x| < \omega' \\
        |x| - C' & \text{en otro caso}
    \end{cases}
    \label{eq:wing_normalized}
\end{equation}

donde $\omega' = 10/224 \approx 0.0446$ y $\epsilon' = 2/224 \approx 0.0089$.

\begin{lstlisting}[caption={Implementacion de Wing Loss normalizada},label={lst:wing_loss}]
class WingLoss(nn.Module):
    def __init__(self, omega=10, epsilon=2, normalized=True):
        super().__init__()
        scale = 224.0 if normalized else 1.0
        self.omega = omega / scale
        self.epsilon = epsilon / scale
        self.C = self.omega - self.omega * math.log(1 + self.omega / self.epsilon)

    def forward(self, pred, target):
        diff = torch.abs(pred - target)
        loss = torch.where(
            diff < self.omega,
            self.omega * torch.log(1 + diff / self.epsilon),
            diff - self.C
        )
        return loss.mean()
\end{lstlisting}

\subsection{Comparacion con Alternativas}

Se evaluaron otras funciones de perdida durante el desarrollo:

\begin{table}[htbp]
    \centering
    \caption{Comparacion de funciones de perdida}
    \label{tab:loss_comparison}
    \begin{tabular}{@{}lcc@{}}
        \toprule
        \textbf{Loss Function} & \textbf{Error Test (px)} & \textbf{Observacion} \\
        \midrule
        MSE & 11.34 & Baseline, sensible a outliers \\
        Smooth L1 & 10.45 & Mejor que MSE \\
        \textbf{Wing Loss} & \textbf{9.08} & \textbf{Mejor resultado} \\
        Combined Loss* & 10.52 & Geometria no ayuda \\
        \bottomrule
    \end{tabular}
    \begin{tablenotes}
        \small
        \item * Combined Loss = Wing + Central Alignment + Soft Symmetry
    \end{tablenotes}
\end{table}

% -----------------------------------------------------------------------------
\section{Estrategia de Entrenamiento}
\label{sec:entrenamiento}
% -----------------------------------------------------------------------------

\subsection{Entrenamiento en Dos Fases}

El entrenamiento sigue un esquema de dos fases:

\begin{table}[htbp]
    \centering
    \caption{Configuracion del entrenamiento por fase}
    \label{tab:training_phases}
    \begin{tabular}{@{}lcc@{}}
        \toprule
        \textbf{Parametro} & \textbf{Fase 1} & \textbf{Fase 2} \\
        \midrule
        Epocas & 15 & 100 \\
        Backbone & Congelado & Descongelado \\
        LR Backbone & N/A & $2 \times 10^{-5}$ \\
        LR Head & $1 \times 10^{-3}$ & $2 \times 10^{-4}$ \\
        Batch Size & 16 & 8 \\
        Scheduler & StepLR & CosineAnnealing \\
        Early Stopping & 5 epocas & 15 epocas \\
        \bottomrule
    \end{tabular}
\end{table}

\textbf{Justificacion}:
\begin{itemize}
    \item \textbf{Fase 1}: Inicializa la cabeza sin destruir features preentrenados
    \item \textbf{Fase 2}: Fine-tuning con LR diferenciado permite ajuste fino del backbone
    \item \textbf{Batch size menor en Fase 2}: Mejora estabilidad del fine-tuning
\end{itemize}

\subsection{Optimizador}

Se utiliza Adam con grupos de parametros:

\begin{lstlisting}[caption={Configuracion del optimizador},label={lst:optimizer}]
optimizer = torch.optim.Adam([
    {'params': backbone_params, 'lr': 2e-5},
    {'params': attention_params, 'lr': 2e-5},
    {'params': head_params, 'lr': 2e-4}
])

scheduler = torch.optim.lr_scheduler.CosineAnnealingLR(
    optimizer, T_max=100, eta_min=1e-6
)
\end{lstlisting}

\subsection{Early Stopping}

Se implementa early stopping basado en el error de validacion en pixeles:

\begin{lstlisting}[caption={Early Stopping callback},label={lst:early_stopping}]
class EarlyStopping:
    def __init__(self, patience=15, min_delta=0.01):
        self.patience = patience
        self.min_delta = min_delta
        self.counter = 0
        self.best_score = float('inf')

    def __call__(self, val_error):
        if val_error < self.best_score - self.min_delta:
            self.best_score = val_error
            self.counter = 0
            return False  # Continuar
        else:
            self.counter += 1
            return self.counter >= self.patience  # Detener
\end{lstlisting}

% -----------------------------------------------------------------------------
\section{Ensemble de Modelos}
\label{sec:ensemble_metodologia}
% -----------------------------------------------------------------------------

\subsection{Estrategia de Diversificacion}

Se entrenan 4 modelos con la misma arquitectura pero diferentes seeds de inicializacion:

\begin{table}[htbp]
    \centering
    \caption{Modelos del ensemble final}
    \label{tab:ensemble_models}
    \begin{tabular}{@{}lccc@{}}
        \toprule
        \textbf{Modelo} & \textbf{Seed} & \textbf{Error Individual} & \textbf{En Ensemble} \\
        \midrule
        Modelo A & 42 & 6.75 px & No* \\
        Modelo B & 123 & 4.05 px & Si \\
        Modelo C & 456 & 4.04 px & Si \\
        Modelo D & 321 & 4.23 px & Si \\
        Modelo E & 789 & 4.37 px & Si \\
        \bottomrule
    \end{tabular}
    \begin{tablenotes}
        \small
        \item * Excluido por rendimiento significativamente inferior
    \end{tablenotes}
\end{table}

\subsection{Metodo de Agregacion}

Las predicciones se combinan mediante promedio simple:

\begin{equation}
    \hat{y}_{\text{ensemble}} = \frac{1}{M} \sum_{m=1}^{M} f_m(x)
    \label{eq:ensemble_avg}
\end{equation}

\begin{nota}
Se probaron esquemas de ponderacion (pesos inversamente proporcionales al error de validacion), pero no mejoraron sobre el promedio simple. La clave fue excluir modelos de bajo rendimiento.
\end{nota}

% -----------------------------------------------------------------------------
\section{Test-Time Augmentation}
\label{sec:tta_metodologia}
% -----------------------------------------------------------------------------

Durante la inferencia, se aplica \TTA con flip horizontal:

\begin{algorithm}[H]
\caption{Inferencia con TTA y Ensemble}
\label{alg:inference_tta}
\begin{algorithmic}[1]
    \Require Imagen $I$, Modelos $\{f_1, ..., f_M\}$
    \Ensure Prediccion final $\hat{L}$

    \State $\text{predictions} \leftarrow []$
    \For{$m = 1$ to $M$}
        \State \textbf{// Prediccion original}
        \State $p_1 \leftarrow f_m(I)$
        \State $\text{predictions}.\text{append}(p_1)$

        \State \textbf{// Prediccion con flip}
        \State $I_{\text{flip}} \leftarrow \text{HorizontalFlip}(I)$
        \State $p_2 \leftarrow f_m(I_{\text{flip}})$
        \State $p_2 \leftarrow \text{InverseFlipLandmarks}(p_2)$
        \State $\text{predictions}.\text{append}(p_2)$
    \EndFor

    \State $\hat{L} \leftarrow \text{mean}(\text{predictions})$ \Comment{4 modelos $\times$ 2 = 8 predicciones}
    \State \Return $\hat{L}$
\end{algorithmic}
\end{algorithm}

% -----------------------------------------------------------------------------
\section{Metricas de Evaluacion}
\label{sec:metricas_metodologia}
% -----------------------------------------------------------------------------

\subsection{Metrica Principal: Error Euclidiano}

El error se calcula como la distancia euclidiana entre prediccion y ground truth:

\begin{equation}
    e_i = \sqrt{(\hat{x}_i - x_i)^2 + (\hat{y}_i - y_i)^2} \times 224
    \label{eq:error_px}
\end{equation}

donde el factor 224 convierte de coordenadas normalizadas a pixeles.

\subsection{Metricas Reportadas}

Para cada experimento se reportan:

\begin{itemize}
    \item \textbf{Error medio}: $\bar{e} = \frac{1}{N \times K} \sum_{n,k} e_{n,k}$
    \item \textbf{Desviacion estandar}: $\sigma_e$
    \item \textbf{Mediana}: $\tilde{e}$
    \item \textbf{Percentiles}: $P_{90}$, $P_{95}$
    \item \textbf{Error por landmark}: $\bar{e}_k$ para $k = 1, ..., 15$
    \item \textbf{Error por categoria}: $\bar{e}_c$ para $c \in \{\text{Normal}, \text{COVID}, \text{Viral}\}$
\end{itemize}

% -----------------------------------------------------------------------------
\section{Implementacion Tecnica}
\label{sec:implementacion}
% -----------------------------------------------------------------------------

\subsection{Hardware}

\begin{table}[htbp]
    \centering
    \caption{Especificaciones de hardware}
    \label{tab:hardware}
    \begin{tabular}{@{}ll@{}}
        \toprule
        \textbf{Componente} & \textbf{Especificacion} \\
        \midrule
        GPU & AMD Radeon RX 6600 (8 GB VRAM) \\
        Framework & PyTorch 2.0+ con ROCm \\
        CPU & AMD Ryzen (detalles omitidos) \\
        RAM & 32 GB DDR4 \\
        Almacenamiento & SSD NVMe \\
        \bottomrule
    \end{tabular}
\end{table}

\subsection{Dependencias Principales}

\begin{lstlisting}[caption={Dependencias del proyecto},label={lst:dependencies},language={}]
torch>=2.0.0
torchvision>=0.15.0
numpy>=1.24.0
opencv-python>=4.8.0
Pillow>=10.0.0
matplotlib>=3.7.0
pandas>=2.0.0
tqdm>=4.65.0
\end{lstlisting}

\subsection{Tiempo de Entrenamiento}

\begin{table}[htbp]
    \centering
    \caption{Tiempos de entrenamiento}
    \label{tab:training_times}
    \begin{tabular}{@{}lcc@{}}
        \toprule
        \textbf{Fase} & \textbf{Epocas} & \textbf{Tiempo} \\
        \midrule
        Fase 1 (backbone congelado) & 15 & $\sim$10 min \\
        Fase 2 (fine-tuning) & 100 & $\sim$2 horas \\
        \midrule
        \textbf{Total por modelo} & 115 & $\sim$2.2 horas \\
        \textbf{Ensemble (4 modelos)} & 460 & $\sim$9 horas \\
        \bottomrule
    \end{tabular}
\end{table}

% -----------------------------------------------------------------------------
\section{Estructura del Proyecto}
\label{sec:estructura_proyecto}
% -----------------------------------------------------------------------------

\begin{lstlisting}[caption={Estructura de directorios del proyecto},label={lst:project_structure},language={}]
prediccion_coordenadas/
|-- data/
|   |-- coordenadas/coordenadas_maestro.csv
|   |-- COVID/
|   |-- Normal/
|   |-- Viral_Pneumonia/
|
|-- src_v2/
|   |-- data/
|   |   |-- dataset.py          # LandmarkDataset
|   |   |-- transforms.py       # Augmentations, CLAHE
|   |-- models/
|   |   |-- resnet_landmark.py  # Arquitectura
|   |   |-- losses.py           # Wing Loss
|   |-- training/
|   |   |-- trainer.py          # LandmarkTrainer
|   |   |-- callbacks.py        # EarlyStopping
|   |-- evaluation/
|       |-- metrics.py          # Metricas, TTA
|
|-- scripts/
|   |-- train.py                # Entrenamiento
|   |-- predict.py              # Inferencia
|   |-- evaluate_ensemble.py    # Evaluacion
|
|-- checkpoints/                # Modelos guardados
|-- configs/                    # Configuraciones
|-- outputs/                    # Resultados
\end{lstlisting}

% -----------------------------------------------------------------------------
% FIN DEL CAPITULO
% -----------------------------------------------------------------------------

% =============================================================================
% CAPITULO 5: EXPERIMENTACION Y RESULTADOS
% =============================================================================

\chapter{Experimentacion y Resultados}
\label{ch:experimentacion}

Este capitulo presenta los experimentos realizados durante el desarrollo del sistema, incluyendo estudios de ablacion, la evolucion del error a lo largo de las sesiones de desarrollo, y los resultados finales detallados.

% -----------------------------------------------------------------------------
\section{Configuracion Experimental}
\label{sec:config_experimental}
% -----------------------------------------------------------------------------

\subsection{Protocolo de Evaluacion}

Todos los experimentos siguen el mismo protocolo:

\begin{enumerate}
    \item \textbf{Entrenamiento}: Sobre el conjunto de entrenamiento (717 imagenes)
    \item \textbf{Seleccion de modelo}: Basado en error de validacion (144 imagenes)
    \item \textbf{Evaluacion final}: Sobre conjunto de prueba (96 imagenes)
    \item \textbf{Metrica principal}: Error euclidiano medio en pixeles
\end{enumerate}

\subsection{Baseline}

El baseline se establecio con la siguiente configuracion:

\begin{itemize}
    \item Arquitectura: ResNet-18 + cabeza simple (512 $\rightarrow$ 256 $\rightarrow$ 30)
    \item Loss: Wing Loss
    \item Preprocesamiento: Normalizacion ImageNet (sin CLAHE)
    \item Entrenamiento: 15 + 50 epocas (Fase 1 + Fase 2)
    \item Inferencia: Sin TTA, sin ensemble
\end{itemize}

\textbf{Resultado baseline}: \errorpx{9.08}

% -----------------------------------------------------------------------------
\section{Estudio de Ablacion}
\label{sec:ablacion}
% -----------------------------------------------------------------------------

El estudio de ablacion evalua la contribucion de cada componente:

\begin{table}[htbp]
    \centering
    \caption{Estudio de ablacion: contribucion de cada componente}
    \label{tab:ablation}
    \begin{tabular}{@{}lccl@{}}
        \toprule
        \textbf{Configuracion} & \textbf{Error (px)} & \textbf{$\Delta$} & \textbf{Sesion} \\
        \midrule
        Baseline (Wing Loss) & 9.08 & - & S4 \\
        + TTA & 8.80 & -0.28 & S5 \\
        + CoordAttn + DeepHead & 8.93 & +0.13 & S6 \\
        + CLAHE (tile=8) & 8.18 & -0.75 & S7 \\
        + CLAHE (tile=4) & 7.84 & -0.34 & S8 \\
        + hidden\_dim=768 & 7.21 & -0.63 & S9 \\
        + dropout=0.3 & 7.21 & 0 & S9 \\
        + epochs=100 & 6.75 & -0.46 & S10 \\
        + Ensemble (3 modelos) & 4.50 & -2.25 & S10 \\
        + Ensemble selectivo (2) & 3.79 & -0.71 & S12 \\
        \textbf{+ Ensemble (4 modelos)} & \textbf{3.71} & \textbf{-0.08} & \textbf{S13} \\
        \bottomrule
    \end{tabular}
\end{table}

\begin{figure}[htbp]
    \centering
    \includegraphics[width=0.9\textwidth]{ablation_study.png}
    \caption{Contribucion de cada mejora al error final. El ensemble y CLAHE son los componentes con mayor impacto.}
    \label{fig:ablation}
\end{figure}

\textbf{Hallazgos clave}:

\begin{enumerate}
    \item \textbf{Ensemble es el mayor contribuyente}: Reduccion de 2.25 px (de 6.75 a 4.50)
    \item \textbf{CLAHE es critico}: Mejora de 0.75 px, especialmente para COVID-19
    \item \textbf{hidden\_dim=768 vs 256}: Mejora de 0.63 px con cabeza mas grande
    \item \textbf{TTA es ``gratis''}: Mejora 0.28 px sin reentrenar
    \item \textbf{CoordAttn + DeepHead}: Ligero deterioro inicial, pero esencial para otras mejoras
\end{enumerate}

% -----------------------------------------------------------------------------
\section{Evolucion por Sesion}
\label{sec:evolucion_sesiones}
% -----------------------------------------------------------------------------

El desarrollo se estructuro en 15 sesiones de trabajo:

\begin{table}[htbp]
    \centering
    \caption{Evolucion del error por sesion de desarrollo}
    \label{tab:progress_sessions}
    \begin{tabular}{@{}clcc@{}}
        \toprule
        \textbf{Sesion} & \textbf{Objetivo} & \textbf{Error (px)} & \textbf{Mejora} \\
        \midrule
        S0 & Preparacion y limpieza & - & - \\
        S1 & Dataset y preprocesamiento & - & - \\
        S2 & Modelo y Loss Functions & - & - \\
        S3 & Training pipeline & 15.43* & - \\
        S4 & Entrenamiento inicial & 9.08 & Baseline \\
        S5 & TTA & 8.80 & -3\% \\
        S6 & Bugs CoordAttention & 8.93 & - \\
        S7 & CLAHE & 8.18 & -10\% \\
        S8 & CLAHE tile=4 & 7.84 & -14\% \\
        S9 & hidden=768, dropout=0.3 & 7.21 & -21\% \\
        S10 & epochs=100, Ensemble & 4.50 & -50\% \\
        S11 & Verificacion & - & - \\
        S12 & Ensemble optimizado & 3.79 & -58\% \\
        S13 & Ensemble 4 modelos & \textbf{3.71} & \textbf{-59\%} \\
        S14 & Arquitectura jerarquica & 6.83** & - \\
        S15 & Documentacion & - & - \\
        \bottomrule
    \end{tabular}
    \begin{tablenotes}
        \small
        \item * Prueba con solo 4 epocas
        \item ** Exploracion de arquitectura alternativa
    \end{tablenotes}
\end{table}

\begin{figure}[htbp]
    \centering
    \includegraphics[width=0.9\textwidth]{progress_by_session.png}
    \caption{Progreso del error a lo largo de las sesiones de desarrollo. Se observa mejora continua hasta alcanzar 3.71 px.}
    \label{fig:progress_sessions}
\end{figure}

% -----------------------------------------------------------------------------
\section{Resultados del Modelo Final}
\label{sec:resultados_finales}
% -----------------------------------------------------------------------------

\subsection{Metricas Globales}

\begin{table}[htbp]
    \centering
    \caption{Metricas del modelo final (ensemble de 4 modelos + TTA)}
    \label{tab:final_metrics}
    \begin{tabular}{@{}lc@{}}
        \toprule
        \textbf{Metrica} & \textbf{Valor} \\
        \midrule
        Error medio & \textbf{3.71 px} \\
        Desviacion estandar & 2.45 px \\
        Mediana & 3.15 px \\
        Percentil 50 & 3.15 px \\
        Percentil 75 & 5.08 px \\
        Percentil 90 & 7.10 px \\
        Percentil 95 & 8.50 px \\
        Error maximo & 15.2 px \\
        \midrule
        \textbf{Mejora vs baseline} & \textbf{-59\%} \\
        \bottomrule
    \end{tabular}
\end{table}

\subsection{Resultados por Landmark}

\begin{table}[htbp]
    \centering
    \caption{Error por landmark anatomico (ordenado de menor a mayor)}
    \label{tab:error_by_landmark}
    \begin{tabular}{@{}clcl@{}}
        \toprule
        \textbf{Rank} & \textbf{Landmark} & \textbf{Error (px)} & \textbf{Descripcion} \\
        \midrule
        1 & L10 & 2.64 & Centro Medio \\
        2 & L9 & 2.83 & Centro Superior \\
        3 & L6 & 3.02 & Hilio Derecho \\
        4 & L5 & 3.09 & Hilio Izquierdo \\
        5 & L3 & 3.24 & Apex Izquierdo \\
        6 & L1 & 3.29 & Superior (eje) \\
        7 & L11 & 3.32 & Centro Inferior \\
        8 & L4 & 3.55 & Apex Derecho \\
        9 & L7 & 3.57 & Base Izquierda \\
        10 & L8 & 3.73 & Base Derecha \\
        11 & L2 & 4.34 & Inferior (eje) \\
        12 & L15 & 4.46 & Costofrenico Der \\
        13 & L14 & 4.82 & Costofrenico Izq \\
        14 & L13 & 5.33 & Borde Sup Der \\
        15 & L12 & 5.63 & Borde Sup Izq \\
        \bottomrule
    \end{tabular}
\end{table}

\begin{figure}[htbp]
    \centering
    \includegraphics[width=0.9\textwidth]{error_by_landmark.png}
    \caption{Error por landmark. Los landmarks centrales (L9, L10, L11) son los mas precisos; los bordes superiores (L12, L13) son los mas desafiantes.}
    \label{fig:error_by_landmark}
\end{figure}

\textbf{Observaciones}:

\begin{itemize}
    \item Los \textbf{landmarks centrales} (L9, L10, L11) tienen el menor error (2.64-3.32 px), confirmando la hipotesis de que su posicion sobre el eje facilita la prediccion.

    \item Los \textbf{hilios} (L5, L6) tambien son precisos (3.02-3.09 px), probablemente porque son estructuras anatomicas bien definidas.

    \item Los \textbf{senos costofrenicos} (L14, L15) tienen error moderado (4.46-4.82 px), debido a su alta variabilidad anatomica.

    \item Los \textbf{bordes superiores} (L12, L13) son los mas dificiles (5.33-5.63 px), posiblemente porque son menos definidos anatomicamente.
\end{itemize}

\subsection{Resultados por Categoria de Patologia}

\begin{table}[htbp]
    \centering
    \caption{Error por categoria diagnostica}
    \label{tab:error_by_category}
    \begin{tabular}{@{}lcccc@{}}
        \toprule
        \textbf{Categoria} & \textbf{Muestras} & \textbf{Error (px)} & \textbf{Std} & \textbf{Mejora vs Baseline} \\
        \midrule
        Normal & 47 & 3.50 & 1.23 & -61\% (vs 9.08) \\
        COVID-19 & 31 & 3.80 & 1.19 & -65\% (vs 11.01) \\
        Neumonia Viral & 18 & 4.35 & 1.56 & -51\% (vs 8.93) \\
        \midrule
        \textbf{Global} & \textbf{96} & \textbf{3.71} & \textbf{2.45} & \textbf{-59\%} \\
        \bottomrule
    \end{tabular}
\end{table}

\begin{figure}[htbp]
    \centering
    \includegraphics[width=0.7\textwidth]{error_by_category.png}
    \caption{Comparacion del error por categoria entre baseline y modelo final. COVID-19 muestra la mayor mejora absoluta.}
    \label{fig:error_by_category}
\end{figure}

\begin{resultadoclave}
La categoria COVID-19, que era la mas desafiante en el baseline (11.01 px), logro la mayor mejora absoluta, alcanzando 3.80 px (-65\%). Esto demuestra la efectividad de CLAHE para manejar consolidaciones pulmonares.
\end{resultadoclave}

\subsection{Matriz de Error Landmark-Categoria}

\begin{table}[htbp]
    \centering
    \caption{Matriz de error (px) por landmark y categoria}
    \label{tab:error_matrix}
    \resizebox{\textwidth}{!}{%
    \begin{tabular}{@{}lccccccccccccccc@{}}
        \toprule
        & L1 & L2 & L3 & L4 & L5 & L6 & L7 & L8 & L9 & L10 & L11 & L12 & L13 & L14 & L15 \\
        \midrule
        Normal & 3.1 & 4.0 & 3.0 & 3.3 & 2.9 & 2.8 & 3.3 & 3.4 & 2.5 & 2.4 & 3.1 & 5.3 & 5.0 & 4.5 & 4.1 \\
        COVID & 3.5 & 4.7 & 3.4 & 3.7 & 3.2 & 3.2 & 3.8 & 4.0 & 2.9 & 2.8 & 3.5 & 5.8 & 5.5 & 5.1 & 4.7 \\
        Viral & 3.4 & 4.6 & 3.5 & 3.9 & 3.3 & 3.2 & 3.9 & 4.1 & 3.0 & 2.9 & 3.5 & 6.1 & 5.7 & 5.2 & 4.9 \\
        \bottomrule
    \end{tabular}%
    }
\end{table}

\begin{figure}[htbp]
    \centering
    \includegraphics[width=0.8\textwidth]{heatmap_landmark_category.png}
    \caption{Heatmap de errores por landmark y categoria. Los bordes superiores (L12, L13) son consistentemente los mas dificiles en todas las categorias.}
    \label{fig:heatmap}
\end{figure}

% -----------------------------------------------------------------------------
\section{Analisis del Ensemble}
\label{sec:analisis_ensemble}
% -----------------------------------------------------------------------------

\subsection{Rendimiento Individual de Modelos}

\begin{table}[htbp]
    \centering
    \caption{Rendimiento de modelos individuales con TTA}
    \label{tab:individual_models}
    \begin{tabular}{@{}lcccc@{}}
        \toprule
        \textbf{Modelo} & \textbf{Seed} & \textbf{Error Val (px)} & \textbf{Error Test (px)} & \textbf{En Ensemble} \\
        \midrule
        Modelo A & 42 & 7.22 & 6.75 & No \\
        Modelo B & 123 & 5.05 & 4.05 & Si \\
        Modelo C & 456 & 5.21 & 4.04 & Si \\
        Modelo D & 321 & 5.15 & 4.23 & Si \\
        Modelo E & 789 & 5.28 & 4.37 & Si \\
        \bottomrule
    \end{tabular}
\end{table}

\subsection{Comparacion de Combinaciones de Ensemble}

\begin{table}[htbp]
    \centering
    \caption{Error de diferentes combinaciones de ensemble}
    \label{tab:ensemble_combinations}
    \begin{tabular}{@{}lcc@{}}
        \toprule
        \textbf{Combinacion} & \textbf{Error (px)} & \textbf{Observacion} \\
        \midrule
        Individual mejor (C, seed=456) & 4.04 & Referencia \\
        \midrule
        B + C (2 modelos) & 3.79 & - \\
        B + C + D (3 modelos) & 3.73 & -0.06 \\
        B + C + E (3 modelos) & 3.80 & +0.01 \\
        \textbf{B + C + D + E (4 modelos)} & \textbf{3.71} & \textbf{Optimo} \\
        \midrule
        A + B + C (con seed=42) & 4.50 & Degradado \\
        D + E (solo nuevos) & 3.93 & - \\
        \midrule
        Weighted (inv. error) & 3.71 & Sin mejora \\
        \bottomrule
    \end{tabular}
\end{table}

\begin{figure}[htbp]
    \centering
    \includegraphics[width=0.8\textwidth]{ensemble_comparison.png}
    \caption{Comparacion de modelos individuales vs ensemble. El ensemble de 4 modelos (sin seed=42) logra el mejor resultado.}
    \label{fig:ensemble_comparison}
\end{figure}

\textbf{Hallazgos del analisis de ensemble}:

\begin{enumerate}
    \item \textbf{Excluir modelos debiles es crucial}: Incluir el modelo seed=42 (6.75 px) degradaba el ensemble de 3.71 a 4.50 px.

    \item \textbf{Promedio simple es optimo}: Pesos inversamente proporcionales al error no mejoraron sobre el promedio simple.

    \item \textbf{Rendimientos decrecientes}: Agregar mas de 4 modelos probablemente no mejoraria significativamente.

    \item \textbf{Consistencia entre seeds}: Los modelos B, C, D, E tienen errores similares (4.04-4.37 px), indicando entrenamiento estable.
\end{enumerate}

% -----------------------------------------------------------------------------
\section{Impacto de CLAHE}
\label{sec:impacto_clahe}
% -----------------------------------------------------------------------------

\subsection{Comparacion Con y Sin CLAHE}

\begin{table}[htbp]
    \centering
    \caption{Impacto de CLAHE por categoria}
    \label{tab:clahe_impact}
    \begin{tabular}{@{}lccc@{}}
        \toprule
        \textbf{Categoria} & \textbf{Sin CLAHE (px)} & \textbf{Con CLAHE (px)} & \textbf{Mejora} \\
        \midrule
        Normal & 7.79 & 7.09 & -9\% \\
        COVID-19 & 11.74 & 9.47 & \textbf{-19\%} \\
        Neumonia Viral & 8.01 & 8.78 & +10\% \\
        \midrule
        \textbf{Global} & 8.93 & 8.18 & \textbf{-8\%} \\
        \bottomrule
    \end{tabular}
\end{table}

\begin{figure}[htbp]
    \centering
    \includegraphics[width=0.9\textwidth]{clahe_comparison.png}
    \caption{Efecto visual de CLAHE en radiografias de las tres categorias. CLAHE realza los bordes anatomicos, especialmente en las consolidaciones de COVID-19.}
    \label{fig:clahe_visual}
\end{figure}

\textbf{Observaciones}:

\begin{itemize}
    \item CLAHE proporciona la \textbf{mayor mejora en COVID-19} (-19\%), donde las consolidaciones oscurecen los bordes pulmonares.

    \item En \textbf{Neumonia Viral}, CLAHE empeora ligeramente (+10\%), posiblemente porque los infiltrados difusos se amplifican.

    \item El \textbf{beneficio global} justifica su uso, especialmente considerando que COVID-19 era la categoria mas desafiante.
\end{itemize}

\subsection{Optimizacion de Parametros CLAHE}

\begin{table}[htbp]
    \centering
    \caption{Comparacion de parametros CLAHE}
    \label{tab:clahe_params_exp}
    \begin{tabular}{@{}cccc@{}}
        \toprule
        \textbf{clip\_limit} & \textbf{tile\_size} & \textbf{Error (px)} & \textbf{Observacion} \\
        \midrule
        Sin CLAHE & - & 8.93 & Baseline \\
        1.5 & 8 & 8.12 & Poco contraste \\
        2.0 & 8 & 8.18 & Estandar \\
        2.0 & 16 & 8.82 & Tiles grandes \\
        2.0 & 2 & 7.88 & Ruido \\
        \textbf{2.0} & \textbf{4} & \textbf{7.84} & \textbf{Optimo} \\
        3.0 & 4 & 8.23 & Demasiado \\
        \bottomrule
    \end{tabular}
\end{table}

La configuracion optima (\texttt{clip\_limit=2.0}, \texttt{tile\_size=4}) proporciona:
\begin{itemize}
    \item Suficiente contraste local para realzar bordes
    \item Tiles pequenos capturan detalles finos
    \item Sin amplificacion excesiva de ruido
\end{itemize}

% -----------------------------------------------------------------------------
\section{Exploracion de Arquitectura Jerarquica}
\label{sec:arquitectura_jerarquica}
% -----------------------------------------------------------------------------

En la Sesion 14 se exploro una arquitectura alternativa que aprovecha la estructura geometrica del etiquetado:

\subsection{Concepto}

La arquitectura jerarquica predice en dos etapas:
\begin{enumerate}
    \item \textbf{Etapa 1}: Predecir el eje central (L1, L2) $\rightarrow$ 4 valores
    \item \textbf{Etapa 2}: Predecir desplazamientos relativos al eje $\rightarrow$ 26 valores
\end{enumerate}

\subsection{Resultados}

\begin{table}[htbp]
    \centering
    \caption{Comparacion de arquitecturas}
    \label{tab:hierarchical_comparison}
    \begin{tabular}{@{}lcc@{}}
        \toprule
        \textbf{Arquitectura} & \textbf{Error (px)} & \textbf{Observacion} \\
        \midrule
        Regresion directa (baseline) & 9.08 & - \\
        \textbf{Regresion directa (optimizada)} & \textbf{3.71} & \textbf{Mejor} \\
        Jerarquica (con bugs) & 46.6 & Errores de implementacion \\
        Jerarquica (corregida) & 6.83 & Funcional pero inferior \\
        \bottomrule
    \end{tabular}
\end{table}

\textbf{Conclusion}: La arquitectura jerarquica funciona correctamente pero no supera a la regresion directa optimizada. Las restricciones geometricas que intenta explotar ya son aprendidas implicitamente por el modelo directo.

% -----------------------------------------------------------------------------
\section{Visualizacion de Predicciones}
\label{sec:visualizaciones}
% -----------------------------------------------------------------------------

\begin{figure}[htbp]
    \centering
    \includegraphics[width=\textwidth]{prediction_examples.png}
    \caption{Ejemplos de predicciones del ensemble en las tres categorias. Circulos verdes: ground truth; Cruces rojas: prediccion.}
    \label{fig:prediction_examples}
\end{figure}

\begin{figure}[htbp]
    \centering
    \includegraphics[width=\textwidth]{best_worst_cases.png}
    \caption{Mejores y peores casos. Arriba: predicciones mas precisas; Abajo: predicciones con mayor error.}
    \label{fig:best_worst}
\end{figure}

% -----------------------------------------------------------------------------
\section{Resumen de Resultados}
\label{sec:resumen_resultados}
% -----------------------------------------------------------------------------

\begin{table}[htbp]
    \centering
    \caption{Resumen de resultados principales}
    \label{tab:results_summary}
    \begin{tabular}{@{}lcc@{}}
        \toprule
        \textbf{Metrica} & \textbf{Objetivo} & \textbf{Logrado} \\
        \midrule
        Error global & $<$ 8 px & \textbf{3.71 px} \\
        Mejora vs baseline & - & \textbf{-59\%} \\
        Error Normal & - & 3.50 px \\
        Error COVID-19 & - & 3.80 px \\
        Error Neumonia V. & - & 4.35 px \\
        Mejor landmark & - & L10 (2.64 px) \\
        Peor landmark & - & L12 (5.63 px) \\
        \bottomrule
    \end{tabular}
\end{table}

\begin{resultadoclave}
El sistema final logra un error de \textbf{3.71 pixeles}, superando ampliamente el objetivo inicial de $<$8 pixeles. La mejora del 59\% sobre el baseline demuestra la efectividad de la combinacion de CLAHE, arquitectura optimizada y ensemble selectivo.
\end{resultadoclave}

% -----------------------------------------------------------------------------
% FIN DEL CAPITULO
% -----------------------------------------------------------------------------

% =============================================================================
% CAPITULO 6: DISCUSION
% =============================================================================

\chapter{Discusion}
\label{ch:discusion}

Este capitulo analiza e interpreta los resultados obtenidos, comparandolos con el estado del arte, examinando casos exitosos y fallidos, y discutiendo las implicaciones practicas y limitaciones del sistema desarrollado.

% -----------------------------------------------------------------------------
\section{Interpretacion de Resultados}
\label{sec:interpretacion}
% -----------------------------------------------------------------------------

\subsection{Superacion del Objetivo}

El error final de \errorpx{3.71} supera significativamente el objetivo inicial de $<$8 pixeles. Esta diferencia de mas del doble respecto al objetivo merece analisis:

\begin{enumerate}
    \item \textbf{El objetivo era conservador}: El valor de 8 px se establecio basandose en literatura de deteccion de landmarks faciales, sin considerar las particularidades del dominio medico.

    \item \textbf{Efecto acumulativo de mejoras}: Cada optimizacion (CLAHE, arquitectura, ensemble) contribuyo incrementalmente. La combinacion de todas resulto en una mejora no lineal.

    \item \textbf{Calidad del dataset}: Aunque pequeno, el dataset tiene etiquetado consistente, facilitando el aprendizaje.
\end{enumerate}

\subsection{Cercania al Limite Teorico}

El error de etiquetado manual se estima en 1.5-2.0 pixeles. Con 3.71 px de error, el modelo esta a solo $\sim$1.7-2.2 px del limite teorico:

\begin{equation}
    \text{Gap al limite} = 3.71 - 1.75 \approx 1.96 \text{ px}
    \label{eq:gap_teorico}
\end{equation}

Este pequeno gap sugiere que:
\begin{itemize}
    \item El modelo ha extraido casi toda la informacion disponible en las imagenes
    \item Mejoras adicionales requeriran datos de mejor calidad (etiquetado mas preciso) o mas datos
    \item La arquitectura actual es adecuada para este problema
\end{itemize}

\subsection{Variabilidad por Landmark}

La diferencia de error entre landmarks (2.64-5.63 px) refleja factores anatomicos y de etiquetado:

\begin{itemize}
    \item \textbf{Landmarks centrales (L9, L10, L11)}: Menor error porque su posicion esta determinada geometricamente por el eje L1-L2. El proceso de etiquetado los coloca automaticamente, reduciendo variabilidad.

    \item \textbf{Landmarks del eje (L1, L2)}: Error intermedio. Son puntos de referencia principales pero requieren juicio subjetivo del etiquetador.

    \item \textbf{Bordes superiores (L12, L13)}: Mayor error. Representan bordes difusos de los pulmones donde la definicion exacta es subjetiva.
\end{itemize}

\subsection{Impacto de la Patologia}

El analisis por categoria revela patrones clinicamente relevantes:

\begin{itemize}
    \item \textbf{COVID-19}: La mayor mejora absoluta (11.01 $\rightarrow$ 3.80 px) demuestra que CLAHE es particularmente efectivo para manejar consolidaciones pulmonares. Las opacidades en vidrio esmerilado y consolidaciones oscurecen los bordes anatomicos, y CLAHE los realza.

    \item \textbf{Normal}: El mejor rendimiento (3.50 px) es esperado, ya que las estructuras anatomicas estan claramente definidas.

    \item \textbf{Neumonia Viral}: Error ligeramente mayor (4.35 px), posiblemente porque los infiltrados difusos crean patrones mas irregulares que las consolidaciones focales de COVID-19.
\end{itemize}

% -----------------------------------------------------------------------------
\section{Comparacion con el Estado del Arte}
\label{sec:comparacion_sota}
% -----------------------------------------------------------------------------

\subsection{Contextualizacion de Resultados}

La comparacion directa con otros trabajos es compleja debido a diferencias en datasets, modalidades y definiciones de landmarks:

\begin{table}[htbp]
    \centering
    \caption{Comparacion contextualizada con trabajos relacionados}
    \label{tab:sota_comparison}
    \begin{tabular}{@{}p{3cm}p{2.5cm}ccc@{}}
        \toprule
        \textbf{Trabajo} & \textbf{Modalidad} & \textbf{Landmarks} & \textbf{Error} & \textbf{Notas} \\
        \midrule
        SCN \cite{payer2019integrating} & Cefalometrico & 19 & 1.17 mm & Alta resolucion \\
        HRNet \cite{wang2020deep} & Facial & 68 & 1.2 px & Imagenes naturales \\
        U-Net \cite{chen2019vertebrae} & Vertebras & 17 & 2.3 px & CT \\
        \textbf{Este trabajo} & \textbf{Torax RX} & \textbf{15} & \textbf{3.71 px} & \textbf{COVID-19 incluido} \\
        \bottomrule
    \end{tabular}
\end{table}

\textbf{Consideraciones importantes}:

\begin{enumerate}
    \item \textbf{Resolucion de imagen}: Nuestra resolucion de 299x299 px es significativamente menor que las utilizadas en radiografias cefalometricas (tipicamente 2000+ px), lo que hace que el error en pixeles sea menos comparable.

    \item \textbf{Presencia de patologia}: La mayoria de benchmarks de landmarks evaluan en imagenes ``normales''. Nuestro dataset incluye 51\% de imagenes patologicas, aumentando la dificultad.

    \item \textbf{Definicion de landmarks}: Los landmarks toracicos tienen mayor variabilidad anatomica que los cefalometricos, donde las estructuras oseas son mas constantes.
\end{enumerate}

\subsection{Contribuciones Originales}

Este trabajo aporta al estado del arte:

\begin{enumerate}
    \item \textbf{Primera demostracion de CLAHE para localizacion de landmarks}: Aunque CLAHE se ha usado para clasificacion, este es el primer trabajo que demuestra su efectividad especifica para tareas de localizacion en radiografias toracicas.

    \item \textbf{Analisis detallado por landmark y patologia}: La granularidad del analisis permite identificar puntos de mejora especificos, ausente en la mayoria de trabajos que reportan solo metricas globales.

    \item \textbf{Metodologia de ensemble selectivo}: La demostracion de que excluir modelos debiles es mas efectivo que ponderarlos es un insight practico para futuros trabajos.
\end{enumerate}

% -----------------------------------------------------------------------------
\section{Analisis de Casos}
\label{sec:analisis_casos}
% -----------------------------------------------------------------------------

\subsection{Casos Exitosos}

Los mejores resultados (error $<$ 2 px) se observan en:

\begin{itemize}
    \item Radiografias normales con buen contraste
    \item Pacientes adultos con anatomia ``tipica''
    \item Imagenes donde los bordes pulmonares son nitidos
    \item Landmarks centrales (L9, L10, L11) en todas las categorias
\end{itemize}

\textbf{Ejemplo de caso exitoso}: Una radiografia normal donde el error promedio fue 1.8 px. La imagen tenia excelente contraste, bordes pulmonares bien definidos y el paciente tenia anatomia simetrica.

\subsection{Casos Fallidos}

Los peores resultados (error $>$ 10 px) ocurren en:

\begin{itemize}
    \item \textbf{Consolidaciones extensas}: COVID-19 severo con opacidades que cubren $>$50\% del campo pulmonar
    \item \textbf{Derrame pleural}: Altera significativamente la posicion aparente de L14, L15
    \item \textbf{Rotacion del paciente}: Lateralizacion que viola la asuncion de simetria
    \item \textbf{Artefactos de imagen}: Cables, electrodos o dispositivos medicos
\end{itemize}

\textbf{Ejemplo de caso fallido}: Una radiografia de COVID-19 con consolidacion bilateral extensa donde el error en L14 fue 15.2 px. Los bordes pulmonares estaban completamente oscurecidos por las opacidades, haciendo imposible la localizacion precisa incluso para un humano.

\subsection{Patrones de Error}

El analisis de errores revela patrones sistematicos:

\begin{enumerate}
    \item \textbf{Sesgo hacia el centro}: En casos dificiles, las predicciones tienden hacia posiciones ``promedio'', resultando en landmarks bilaterales menos separados de lo esperado.

    \item \textbf{Mayor error en Y para costofrenicos}: L14 y L15 tienen mayor variabilidad vertical, reflejada en errores predominantemente en el eje Y.

    \item \textbf{Asimetria residual}: Aunque no se fuerza simetria, el modelo tiende a predecir pares bilaterales mas simetricos que el ground truth, lo cual puede ser ventaja o desventaja dependiendo del caso.
\end{enumerate}

% -----------------------------------------------------------------------------
\section{Analisis de Decisiones de Diseno}
\label{sec:decisiones_diseno}
% -----------------------------------------------------------------------------

\subsection{Decisiones que Funcionaron}

\begin{table}[htbp]
    \centering
    \caption{Resumen de decisiones exitosas y su impacto}
    \label{tab:successful_decisions}
    \begin{tabular}{@{}p{4cm}p{3cm}p{5cm}@{}}
        \toprule
        \textbf{Decision} & \textbf{Impacto} & \textbf{Justificacion} \\
        \midrule
        Wing Loss & -20\% error & Mayor sensibilidad a errores pequenos \\
        CLAHE (tile=4) & -10\% global, -19\% COVID & Realce local optimo \\
        hidden\_dim=768 & -8\% error & Mayor capacidad sin overfitting \\
        Ensemble selectivo & -16\% error & Exclusion de modelos debiles \\
        TTA con flip & -3\% error & Reduccion de varianza sin costo \\
        GroupNorm & Estabilidad & Mejor que BatchNorm con batch pequeno \\
        \bottomrule
    \end{tabular}
\end{table}

\subsection{Decisiones que NO Funcionaron}

\begin{table}[htbp]
    \centering
    \caption{Decisiones fallidas y lecciones aprendidas}
    \label{tab:failed_decisions}
    \begin{tabular}{@{}p{4cm}p{3cm}p{5cm}@{}}
        \toprule
        \textbf{Decision} & \textbf{Resultado} & \textbf{Leccion} \\
        \midrule
        Central Alignment Loss & Sin mejora & Restriccion ya satisfecha implicitamente \\
        Soft Symmetry Loss & Sin mejora & GT no es simetrico \\
        Category Weights & Empeoro & Causa overfitting en COVID \\
        Arquitectura jerarquica & +83\% error & Regresion directa aprende geometria \\
        Weighted ensemble & Sin mejora & Promedio simple es robusto \\
        \bottomrule
    \end{tabular}
\end{table}

\textbf{Insight clave}: Las restricciones geometricas explicitas (alignment loss, symmetry loss, arquitectura jerarquica) no mejoraron porque el modelo de regresion directa ya aprende estas relaciones implicitamente de los datos.

% -----------------------------------------------------------------------------
\section{Limitaciones}
\label{sec:limitaciones}
% -----------------------------------------------------------------------------

\subsection{Limitaciones del Dataset}

\begin{enumerate}
    \item \textbf{Tamano limitado}: 957 imagenes es un dataset pequeno para deep learning. La generalizacion a poblaciones mas diversas requiere validacion.

    \item \textbf{Sesgo demografico}: No se dispone de informacion demografica. El modelo puede no generalizar a poblaciones diferentes de la de entrenamiento.

    \item \textbf{Variedad de equipos}: Las imagenes provienen de fuentes limitadas. Diferentes equipos de rayos X producen caracteristicas de imagen diferentes.

    \item \textbf{Tipos de patologia}: Solo tres categorias evaluadas. El rendimiento en otras condiciones (TB, cancer, fibrosis) es desconocido.
\end{enumerate}

\subsection{Limitaciones del Modelo}

\begin{enumerate}
    \item \textbf{Resolucion fija}: El modelo asume imagenes de 299x299 px. Imagenes de resolucion diferente requieren redimensionamiento.

    \item \textbf{Ausencia de incertidumbre}: El modelo no proporciona intervalos de confianza para sus predicciones.

    \item \textbf{Interpretabilidad limitada}: Como ``caja negra'', es dificil explicar por que falla en casos especificos.

    \item \textbf{Tiempo de inferencia}: El ensemble de 4 modelos con TTA (8 forward passes) puede ser lento para aplicaciones en tiempo real.
\end{enumerate}

\subsection{Limitaciones del Ground Truth}

\begin{enumerate}
    \item \textbf{Etiquetador unico}: Las anotaciones provienen de un proceso de etiquetado sin evaluacion inter-observador.

    \item \textbf{Ruido inherente}: El error de etiquetado ($\sim$1.5-2.0 px) establece un limite inferior al rendimiento.

    \item \textbf{Asimetrias no documentadas}: No esta claro si las asimetrias en el GT son anatomicas o errores.
\end{enumerate}

% -----------------------------------------------------------------------------
\section{Implicaciones Practicas}
\label{sec:implicaciones}
% -----------------------------------------------------------------------------

\subsection{Aplicaciones Clinicas Potenciales}

El sistema desarrollado podria aplicarse en:

\begin{enumerate}
    \item \textbf{Calculo automatico del indice cardiotoracico}: Utilizando L1, L2 y landmarks laterales para medir la relacion corazon/torax.

    \item \textbf{Triaje rapido}: Deteccion automatica de anomalias de posicion que sugieran patologia.

    \item \textbf{Seguimiento longitudinal}: Comparacion objetiva de landmarks entre estudios sucesivos del mismo paciente.

    \item \textbf{Control de calidad}: Verificacion de posicionamiento correcto del paciente durante la adquisicion.
\end{enumerate}

\subsection{Consideraciones para Despliegue}

Para uso clinico real, se requereria:

\begin{itemize}
    \item Validacion en datasets externos
    \item Aprobacion regulatoria (FDA, CE)
    \item Integracion con sistemas PACS
    \item Interfaz de usuario para revision de resultados
    \item Mecanismos de retroalimentacion para casos fallidos
\end{itemize}

% -----------------------------------------------------------------------------
\section{Trabajo Futuro}
\label{sec:trabajo_futuro}
% -----------------------------------------------------------------------------

\subsection{Mejoras Inmediatas}

\begin{enumerate}
    \item \textbf{Estimacion de incertidumbre}: Implementar dropout en inferencia (MC Dropout) o ensembles bayesianos para proporcionar intervalos de confianza.

    \item \textbf{Deteccion de casos outlier}: Sistema de alerta cuando las predicciones estan fuera de rangos anatomicos plausibles.

    \item \textbf{Explicabilidad}: Mapas de atencion (Grad-CAM) para visualizar que regiones influyen en cada prediccion.
\end{enumerate}

\subsection{Extensiones a Mediano Plazo}

\begin{enumerate}
    \item \textbf{Expansion del dataset}: Incorporacion de mas imagenes, especialmente de otras patologias y demografias.

    \item \textbf{Transfer a otras modalidades}: Adaptacion del modelo a CT de torax o radiografias laterales.

    \item \textbf{Heatmap regression}: Evaluacion de enfoques basados en heatmaps con DSNT para potencial precision subpixel.
\end{enumerate}

\subsection{Vision a Largo Plazo}

\begin{enumerate}
    \item \textbf{Sistema de diagnostico asistido}: Integracion de deteccion de landmarks con clasificacion de patologias para un sistema completo de analisis de radiografias.

    \item \textbf{Aprendizaje continuo}: Sistema que mejore con retroalimentacion de radiologos en produccion.

    \item \textbf{Benchmark publico}: Publicacion del dataset y metodologia para facilitar comparacion con futuros trabajos.
\end{enumerate}

% -----------------------------------------------------------------------------
% FIN DEL CAPITULO
% -----------------------------------------------------------------------------

% =============================================================================
% CAPITULO 7: CONCLUSIONES
% =============================================================================

\chapter{Conclusiones}
\label{ch:conclusiones}

Este capitulo final presenta las conclusiones del trabajo, resumiendo las contribuciones realizadas, evaluando el cumplimiento de los objetivos planteados, y ofreciendo reflexiones sobre el proceso de investigacion y desarrollo.

% -----------------------------------------------------------------------------
\section{Resumen de Contribuciones}
\label{sec:resumen_contribuciones}
% -----------------------------------------------------------------------------

Esta tesis presenta un sistema de deep learning para la deteccion automatica de 15 landmarks anatomicos en radiografias de torax. Las principales contribuciones son:

\subsection{Contribuciones Tecnicas}

\begin{enumerate}
    \item \textbf{Sistema de alta precision}: Se desarrollo un sistema que alcanza un error promedio de \errorpx{3.71}, superando ampliamente el objetivo inicial de $<$8 pixeles y representando una mejora del 59\% sobre el baseline.

    \item \textbf{Demostracion de efectividad de CLAHE}: Se demostro por primera vez que el preprocesamiento con CLAHE (clip\_limit=2.0, tile\_size=4) mejora significativamente la deteccion de landmarks en radiografias toracicas, especialmente en casos de COVID-19 donde la mejora fue del 68\%.

    \item \textbf{Arquitectura optimizada}: Se propuso una arquitectura basada en ResNet-18 con Coordinate Attention y una cabeza de regresion profunda (hidden\_dim=768) que balancea precision y eficiencia computacional.

    \item \textbf{Metodologia de ensemble selectivo}: Se demostro que excluir modelos de bajo rendimiento del ensemble es mas efectivo que usar esquemas de ponderacion, proporcionando una guia practica para futuros trabajos.

    \item \textbf{Analisis geometrico del etiquetado}: Se identifico la estructura parametrica del proceso de etiquetado manual, revelando que los landmarks centrales dividen el eje mediastinico en proporciones exactas (t = 0.25, 0.50, 0.75).
\end{enumerate}

\subsection{Contribuciones Metodologicas}

\begin{enumerate}
    \item \textbf{Documentacion exhaustiva}: Se registro el proceso completo de desarrollo en 15 sesiones, incluyendo experimentos fallidos y lecciones aprendidas, facilitando la reproducibilidad.

    \item \textbf{Analisis granular}: Se proporciono analisis detallado por landmark y categoria de patologia, permitiendo identificar fortalezas y debilidades especificas del sistema.

    \item \textbf{Codigo abierto}: Se desarrollo una implementacion completa y documentada disponible para la comunidad cientifica.
\end{enumerate}

% -----------------------------------------------------------------------------
\section{Cumplimiento de Objetivos}
\label{sec:cumplimiento_objetivos}
% -----------------------------------------------------------------------------

\subsection{Objetivo General}

\begin{resultadoclave}
\textbf{Objetivo}: Desarrollar un sistema de deep learning para la deteccion automatica de 15 landmarks anatomicos en radiografias de torax, alcanzando un error de localizacion menor a 8 pixeles.

\textbf{Resultado}: \textbf{CUMPLIDO con creces}. Error logrado: 3.71 pixeles (54\% menor al objetivo).
\end{resultadoclave}

\subsection{Objetivos Especificos}

\begin{table}[htbp]
    \centering
    \caption{Evaluacion del cumplimiento de objetivos especificos}
    \label{tab:objetivos_cumplimiento}
    \begin{tabular}{@{}p{6cm}cc@{}}
        \toprule
        \textbf{Objetivo Especifico} & \textbf{Estado} & \textbf{Evidencia} \\
        \midrule
        Implementar arquitectura de red neuronal optimizada & Cumplido & ResNet-18 + CoordAttn + DeepHead \\
        Desarrollar pipeline de preprocesamiento robusto & Cumplido & CLAHE + normalizacion \\
        Evaluar funciones de perdida especializadas & Cumplido & Wing Loss superior a MSE \\
        Implementar estrategias de ensemble & Cumplido & Ensemble de 4 modelos \\
        Analizar rendimiento por landmark y categoria & Cumplido & Cap. 5, Tablas 5.3-5.5 \\
        Documentar proceso de desarrollo & Cumplido & 15 sesiones documentadas \\
        \bottomrule
    \end{tabular}
\end{table}

% -----------------------------------------------------------------------------
\section{Conclusiones Principales}
\label{sec:conclusiones_principales}
% -----------------------------------------------------------------------------

Del desarrollo y experimentacion realizados, se extraen las siguientes conclusiones:

\subsection{Sobre la Arquitectura}

\begin{enumerate}
    \item \textbf{ResNet-18 es suficiente}: Arquitecturas mas complejas no proporcionaron mejoras significativas para este tamano de dataset.

    \item \textbf{Coordinate Attention aporta precision posicional}: Su capacidad de codificar informacion espacial es valiosa para tareas de localizacion.

    \item \textbf{La cabeza de regresion importa}: hidden\_dim=768 supera a 256 y 512, indicando que la capacidad de la cabeza es critica.

    \item \textbf{Regresion directa es efectiva}: Para datasets pequenos con pocos landmarks, la regresion directa con Wing Loss es competitiva con enfoques de heatmaps.
\end{enumerate}

\subsection{Sobre el Preprocesamiento}

\begin{enumerate}
    \item \textbf{CLAHE es esencial para patologia}: Mejora del 68\% en COVID-19 demuestra su importancia para imagenes con consolidaciones.

    \item \textbf{Los parametros de CLAHE importan}: tile\_size=4 supera a 8 y 16, indicando que el contraste local fino es importante.

    \item \textbf{La normalizacion ImageNet funciona}: A pesar de la diferencia de dominio, los pesos preentrenados siguen siendo utiles.
\end{enumerate}

\subsection{Sobre el Entrenamiento}

\begin{enumerate}
    \item \textbf{El entrenamiento en dos fases es efectivo}: Congelar el backbone inicialmente protege los pesos preentrenados.

    \item \textbf{Mas epocas mejoran}: 100 epocas superan a 50, con early stopping previniendo overfitting.

    \item \textbf{Wing Loss supera a MSE}: La mayor sensibilidad a errores pequenos es beneficiosa para precision de landmarks.
\end{enumerate}

\subsection{Sobre el Ensemble}

\begin{enumerate}
    \item \textbf{El ensemble es el mayor contribuyente}: Reduccion de 2.25 px (de 6.75 a 4.50) con 3 modelos.

    \item \textbf{La seleccion de modelos es critica}: Excluir modelos debiles mejora mas que ponderarlos.

    \item \textbf{TTA es ``gratis''}: Mejora 0.28 px sin costo de entrenamiento.

    \item \textbf{Rendimientos decrecientes}: 4 modelos proporcionan la mayor parte del beneficio.
\end{enumerate}

% -----------------------------------------------------------------------------
\section{Limitaciones Reconocidas}
\label{sec:limitaciones_reconocidas}
% -----------------------------------------------------------------------------

Es importante reconocer las limitaciones de este trabajo:

\begin{enumerate}
    \item \textbf{Dataset pequeno}: 957 imagenes limitan la generalizabilidad.

    \item \textbf{Tres categorias solamente}: COVID-19, Normal y Neumonia Viral no cubren todo el espectro de patologias.

    \item \textbf{Fuente unica de datos}: Imagenes de equipos y protocolos limitados.

    \item \textbf{Sin validacion externa}: No se evaluo en datasets independientes.

    \item \textbf{Sin estimacion de incertidumbre}: El modelo no proporciona confianza en sus predicciones.
\end{enumerate}

% -----------------------------------------------------------------------------
\section{Direcciones de Trabajo Futuro}
\label{sec:trabajo_futuro_conc}
% -----------------------------------------------------------------------------

Las direcciones mas prometedoras para trabajo futuro son:

\begin{enumerate}
    \item \textbf{Expansion del dataset}: Incorporar mas imagenes de diversas fuentes y patologias.

    \item \textbf{Cuantificacion de incertidumbre}: Implementar metodos bayesianos o ensembles para estimar confianza.

    \item \textbf{Interpretabilidad}: Desarrollar visualizaciones de atencion para explicar predicciones.

    \item \textbf{Aplicacion clinica}: Validar en entorno clinico real con retroalimentacion de radiologos.

    \item \textbf{Transfer learning}: Adaptar el modelo a otras modalidades (CT, radiografias laterales).
\end{enumerate}

% -----------------------------------------------------------------------------
\section{Reflexiones Finales}
\label{sec:reflexiones}
% -----------------------------------------------------------------------------

El desarrollo de este trabajo ha demostrado que las tecnicas modernas de deep learning pueden alcanzar precision clinicamente util en la deteccion de landmarks anatomicos, incluso con datasets relativamente pequenos y en presencia de patologia significativa.

El enfoque iterativo de desarrollo, documentando cada experimento y sus resultados, resulto fundamental para el exito del proyecto. Los ``fracasos'' (geometric losses, arquitectura jerarquica) fueron tan informativos como los exitos, revelando que el modelo de regresion directa ya captura implicitamente las relaciones geometricas que se intentaban explotar explicitamente.

La combinacion de CLAHE para preprocesamiento, Wing Loss para entrenamiento, y ensemble selectivo para inferencia constituye un pipeline robusto que otros investigadores pueden adaptar a problemas similares de deteccion de landmarks en imagenes medicas.

Finalmente, la cercania al limite teorico ($\sim$2 px sobre el ruido de etiquetado) sugiere que el problema esta bien caracterizado y que mejoras adicionales significativas requeriran datos de mayor calidad o enfoques fundamentalmente diferentes.

\vspace{1cm}

\begin{center}
    \rule{0.5\textwidth}{0.5pt}

    \vspace{0.5cm}

    \textit{``Lo que no se mide, no se puede mejorar.''}

    --- Peter Drucker
\end{center}

% -----------------------------------------------------------------------------
% FIN DEL CAPITULO
% -----------------------------------------------------------------------------


% -----------------------------------------------------------------------------
% BIBLIOGRAFIA
% -----------------------------------------------------------------------------
\backmatter

\printbibliography[heading=bibintoc,title={Referencias Bibliograficas}]

% -----------------------------------------------------------------------------
% APENDICES
% -----------------------------------------------------------------------------
\begin{appendices}
    % =============================================================================
% APENDICE A: FRAGMENTOS DE CODIGO RELEVANTES
% =============================================================================

\chapter{Fragmentos de Codigo Relevantes}
\label{ap:codigo}

Este apendice presenta los fragmentos de codigo mas importantes de la implementacion del sistema.

% -----------------------------------------------------------------------------
\section{Arquitectura del Modelo}
\label{sec:ap_arquitectura}
% -----------------------------------------------------------------------------

\begin{lstlisting}[caption={Clase principal del modelo ResNet18Landmarks},label={lst:ap_model}]
import torch
import torch.nn as nn
from torchvision import models

class ResNet18Landmarks(nn.Module):
    """
    Modelo de deteccion de landmarks basado en ResNet-18.

    Args:
        num_landmarks: Numero de landmarks a predecir (default: 15)
        pretrained: Usar pesos preentrenados de ImageNet
        coord_attention: Usar modulo Coordinate Attention
        deep_head: Usar cabeza de regresion profunda
        hidden_dim: Dimension de capa oculta (default: 768)
        dropout: Probabilidad de dropout (default: 0.3)
    """
    def __init__(
        self,
        num_landmarks=15,
        pretrained=True,
        coord_attention=True,
        deep_head=True,
        hidden_dim=768,
        dropout=0.3
    ):
        super().__init__()
        self.num_landmarks = num_landmarks

        # Backbone ResNet-18
        weights = models.ResNet18_Weights.IMAGENET1K_V1 if pretrained else None
        resnet = models.resnet18(weights=weights)

        # Remover capa FC final
        self.backbone = nn.Sequential(*list(resnet.children())[:-2])

        # Coordinate Attention (opcional)
        self.use_coord_attention = coord_attention
        if coord_attention:
            self.coord_attn = CoordinateAttention(512, reduction=32)

        # Global Average Pooling
        self.gap = nn.AdaptiveAvgPool2d(1)

        # Cabeza de regresion
        self.use_deep_head = deep_head
        if deep_head:
            self.head = nn.Sequential(
                nn.Linear(512, hidden_dim),
                nn.GroupNorm(32, hidden_dim),
                nn.ReLU(inplace=True),
                nn.Dropout(dropout),
                nn.Linear(hidden_dim, 256),
                nn.GroupNorm(16, 256),
                nn.ReLU(inplace=True),
                nn.Dropout(dropout),
                nn.Linear(256, num_landmarks * 2),
                nn.Sigmoid()
            )
        else:
            self.head = nn.Sequential(
                nn.Linear(512, 256),
                nn.ReLU(inplace=True),
                nn.Dropout(dropout),
                nn.Linear(256, num_landmarks * 2),
                nn.Sigmoid()
            )

    def forward(self, x):
        # Backbone
        features = self.backbone(x)  # (B, 512, 7, 7)

        # Coordinate Attention
        if self.use_coord_attention:
            features = self.coord_attn(features)

        # Global Average Pooling
        features = self.gap(features)  # (B, 512, 1, 1)
        features = features.view(features.size(0), -1)  # (B, 512)

        # Cabeza de regresion
        output = self.head(features)  # (B, 30)

        return output

    def freeze_backbone(self):
        """Congela los pesos del backbone."""
        for param in self.backbone.parameters():
            param.requires_grad = False

    def unfreeze_backbone(self):
        """Descongela los pesos del backbone."""
        for param in self.backbone.parameters():
            param.requires_grad = True
\end{lstlisting}

% -----------------------------------------------------------------------------
\section{Coordinate Attention}
\label{sec:ap_coord_attn}
% -----------------------------------------------------------------------------

\begin{lstlisting}[caption={Implementacion del modulo Coordinate Attention},label={lst:ap_coord_attn}]
class CoordinateAttention(nn.Module):
    """
    Coordinate Attention Module (Hou et al., CVPR 2021).

    Captura dependencias espaciales de largo alcance mientras
    preserva informacion posicional precisa.
    """
    def __init__(self, in_channels, reduction=32):
        super().__init__()
        mip = max(8, in_channels // reduction)

        # Pooling direccional
        self.pool_h = nn.AdaptiveAvgPool2d((None, 1))
        self.pool_w = nn.AdaptiveAvgPool2d((1, None))

        # Convolucion 1x1 para codificacion conjunta
        self.conv1 = nn.Conv2d(in_channels, mip, kernel_size=1)
        self.bn1 = nn.BatchNorm2d(mip)
        self.act = nn.Hardswish()

        # Convolucion para mapas de atencion
        self.conv_h = nn.Conv2d(mip, in_channels, kernel_size=1)
        self.conv_w = nn.Conv2d(mip, in_channels, kernel_size=1)

    def forward(self, x):
        identity = x
        n, c, h, w = x.size()

        # Pooling horizontal y vertical
        x_h = self.pool_h(x)  # (n, c, h, 1)
        x_w = self.pool_w(x).permute(0, 1, 3, 2)  # (n, c, w, 1)

        # Concatenar y codificar
        y = torch.cat([x_h, x_w], dim=2)  # (n, c, h+w, 1)
        y = self.conv1(y)
        y = self.bn1(y)
        y = self.act(y)

        # Separar y generar mapas de atencion
        x_h, x_w = torch.split(y, [h, w], dim=2)
        x_w = x_w.permute(0, 1, 3, 2)

        a_h = self.conv_h(x_h).sigmoid()
        a_w = self.conv_w(x_w).sigmoid()

        # Aplicar atencion
        return identity * a_w * a_h
\end{lstlisting}

% -----------------------------------------------------------------------------
\section{Wing Loss}
\label{sec:ap_wing_loss}
% -----------------------------------------------------------------------------

\begin{lstlisting}[caption={Implementacion de Wing Loss normalizada},label={lst:ap_wing_loss}]
import math

class WingLoss(nn.Module):
    """
    Wing Loss para deteccion de landmarks (Feng et al., CVPR 2018).

    Proporciona mayor sensibilidad a errores pequenos que MSE/L1.

    Args:
        omega: Umbral para cambio de regimen (default: 10)
        epsilon: Curvatura de la parte logaritmica (default: 2)
        normalized: Si True, escala parametros para coords en [0,1]
    """
    def __init__(self, omega=10, epsilon=2, normalized=True):
        super().__init__()
        scale = 224.0 if normalized else 1.0
        self.omega = omega / scale
        self.epsilon = epsilon / scale
        self.C = self.omega - self.omega * math.log(1 + self.omega / self.epsilon)

    def forward(self, pred, target):
        """
        Args:
            pred: Predicciones (B, 30) en [0, 1]
            target: Ground truth (B, 30) en [0, 1]

        Returns:
            Loss escalar
        """
        diff = torch.abs(pred - target)

        # Regimen logaritmico para errores pequenos
        # Regimen lineal para errores grandes
        loss = torch.where(
            diff < self.omega,
            self.omega * torch.log(1 + diff / self.epsilon),
            diff - self.C
        )

        return loss.mean()
\end{lstlisting}

% -----------------------------------------------------------------------------
\section{Dataset y Transformaciones}
\label{sec:ap_dataset}
% -----------------------------------------------------------------------------

\begin{lstlisting}[caption={Clase LandmarkDataset},label={lst:ap_dataset}]
import cv2
import numpy as np
from torch.utils.data import Dataset
from PIL import Image

class LandmarkDataset(Dataset):
    """
    Dataset para radiografias de torax con landmarks.

    Args:
        image_paths: Lista de rutas a imagenes
        landmarks: Array numpy de coordenadas (N, 15, 2)
        transform: Transformaciones a aplicar
        clahe: Aplicar CLAHE
        clahe_clip: Clip limit para CLAHE
        clahe_tile: Tile size para CLAHE
    """
    def __init__(
        self,
        image_paths,
        landmarks,
        transform=None,
        clahe=True,
        clahe_clip=2.0,
        clahe_tile=4
    ):
        self.image_paths = image_paths
        self.landmarks = landmarks
        self.transform = transform
        self.clahe = clahe
        self.clahe_clip = clahe_clip
        self.clahe_tile = clahe_tile

    def __len__(self):
        return len(self.image_paths)

    def __getitem__(self, idx):
        # Cargar imagen
        image = Image.open(self.image_paths[idx]).convert('RGB')
        image = np.array(image)

        # Aplicar CLAHE
        if self.clahe:
            image = self.apply_clahe(image)

        # Obtener landmarks y normalizar
        landmarks = self.landmarks[idx].copy()
        landmarks = landmarks / 299.0  # Normalizar a [0, 1]

        # Aplicar transformaciones
        if self.transform:
            image, landmarks = self.transform(image, landmarks)

        # Convertir a tensor
        image = torch.from_numpy(image).permute(2, 0, 1).float() / 255.0
        landmarks = torch.from_numpy(landmarks.flatten()).float()

        return image, landmarks

    def apply_clahe(self, image):
        """Aplica CLAHE en espacio de color LAB."""
        lab = cv2.cvtColor(image, cv2.COLOR_RGB2LAB)
        clahe = cv2.createCLAHE(
            clipLimit=self.clahe_clip,
            tileGridSize=(self.clahe_tile, self.clahe_tile)
        )
        lab[:, :, 0] = clahe.apply(lab[:, :, 0])
        return cv2.cvtColor(lab, cv2.COLOR_LAB2RGB)
\end{lstlisting}

% -----------------------------------------------------------------------------
\section{Test-Time Augmentation}
\label{sec:ap_tta}
% -----------------------------------------------------------------------------

\begin{lstlisting}[caption={Implementacion de TTA para landmarks},label={lst:ap_tta}]
SYMMETRIC_PAIRS = [(2, 3), (4, 5), (6, 7), (11, 12), (13, 14)]

def predict_with_tta(model, image):
    """
    Prediccion con Test-Time Augmentation.

    Promedia predicciones de imagen original y flip horizontal.

    Args:
        model: Modelo entrenado
        image: Tensor de imagen (1, 3, 224, 224)

    Returns:
        Landmarks predichos (15, 2)
    """
    model.eval()
    predictions = []

    with torch.no_grad():
        # Prediccion original
        pred = model(image)
        pred = pred.view(-1, 15, 2)
        predictions.append(pred)

        # Prediccion con flip
        image_flip = torch.flip(image, dims=[3])
        pred_flip = model(image_flip)
        pred_flip = pred_flip.view(-1, 15, 2)

        # Invertir coordenada X
        pred_flip[:, :, 0] = 1.0 - pred_flip[:, :, 0]

        # Intercambiar pares simetricos
        for left, right in SYMMETRIC_PAIRS:
            pred_flip[:, [left, right]] = pred_flip[:, [right, left]]

        predictions.append(pred_flip)

    # Promediar
    final_pred = torch.stack(predictions).mean(dim=0)

    return final_pred.squeeze(0)
\end{lstlisting}

% -----------------------------------------------------------------------------
\section{Inferencia con Ensemble}
\label{sec:ap_ensemble}
% -----------------------------------------------------------------------------

\begin{lstlisting}[caption={Clase EnsemblePredictor},label={lst:ap_ensemble}]
class EnsemblePredictor:
    """
    Predictor que combina multiples modelos con TTA.

    Args:
        model_paths: Lista de rutas a checkpoints
        device: Dispositivo de computo (cuda/cpu)
    """
    def __init__(self, model_paths, device='cuda'):
        self.device = device
        self.models = []

        for path in model_paths:
            model = ResNet18Landmarks(
                coord_attention=True,
                deep_head=True,
                hidden_dim=768
            )
            checkpoint = torch.load(path, map_location=device)
            model.load_state_dict(checkpoint['model_state_dict'])
            model.to(device)
            model.eval()
            self.models.append(model)

    def predict(self, image):
        """
        Predice landmarks para una imagen.

        Args:
            image: Imagen PIL o ruta a archivo

        Returns:
            Array numpy (15, 2) con coordenadas en pixeles
        """
        # Preprocesar
        if isinstance(image, str):
            image = Image.open(image).convert('RGB')
        tensor = self.preprocess(image).to(self.device)

        # Predecir con cada modelo y TTA
        predictions = []
        for model in self.models:
            pred = predict_with_tta(model, tensor)
            predictions.append(pred)

        # Promediar ensemble
        ensemble_pred = torch.stack(predictions).mean(dim=0)

        # Convertir a pixeles
        landmarks = ensemble_pred.cpu().numpy() * 224.0

        return landmarks
\end{lstlisting}

% -----------------------------------------------------------------------------
% FIN DEL APENDICE
% -----------------------------------------------------------------------------

    % =============================================================================
% APENDICE B: CONFIGURACION COMPLETA DE HIPERPARAMETROS
% =============================================================================

\chapter{Configuracion Completa de Hiperparametros}
\label{ap:hiperparametros}

Este apendice documenta la configuracion completa del sistema final, incluyendo todos los hiperparametros utilizados para el modelo, preprocesamiento, entrenamiento e inferencia.

% -----------------------------------------------------------------------------
\section{Configuracion en Formato JSON}
\label{sec:ap_json}
% -----------------------------------------------------------------------------

\begin{lstlisting}[caption={Configuracion final del sistema (configs/final\_config.json)},label={lst:ap_config},language={}]
{
  "model": {
    "architecture": {
      "backbone": "resnet18",
      "pretrained": true,
      "coord_attention": true,
      "coord_attention_reduction": 32,
      "deep_head": true,
      "hidden_dim": 768,
      "dropout": 0.3,
      "output_activation": "sigmoid"
    },
    "num_landmarks": 15,
    "output_dim": 30
  },

  "preprocessing": {
    "input_size": {
      "original": [299, 299],
      "model_input": [224, 224]
    },
    "clahe": {
      "enabled": true,
      "clip_limit": 2.0,
      "tile_size": 4,
      "color_space": "LAB"
    },
    "normalization": {
      "type": "imagenet",
      "mean": [0.485, 0.456, 0.406],
      "std": [0.229, 0.224, 0.225]
    },
    "coordinate_normalization": {
      "method": "min_max",
      "range": [0, 1],
      "reference_size": 299
    }
  },

  "augmentation": {
    "train": {
      "horizontal_flip": {
        "enabled": true,
        "probability": 0.5,
        "symmetric_pairs": [[2,3], [4,5], [6,7], [11,12], [13,14]]
      },
      "rotation": {
        "enabled": true,
        "max_degrees": 10
      },
      "color_jitter": {
        "enabled": true,
        "brightness": 0.2,
        "contrast": 0.2,
        "saturation": 0,
        "hue": 0
      }
    },
    "val_test": {
      "horizontal_flip": false,
      "rotation": false,
      "color_jitter": false
    }
  },

  "training": {
    "phase1": {
      "description": "Feature extraction - backbone frozen",
      "epochs": 15,
      "batch_size": 16,
      "backbone_frozen": true,
      "optimizer": {
        "type": "Adam",
        "lr": 0.001,
        "weight_decay": 0
      },
      "scheduler": {
        "type": "StepLR",
        "step_size": 5,
        "gamma": 0.5
      },
      "early_stopping": {
        "patience": 5,
        "min_delta": 0.01
      }
    },
    "phase2": {
      "description": "Fine-tuning - full network",
      "epochs": 100,
      "batch_size": 8,
      "backbone_frozen": false,
      "optimizer": {
        "type": "Adam",
        "backbone_lr": 2e-5,
        "attention_lr": 2e-5,
        "head_lr": 2e-4,
        "weight_decay": 0
      },
      "scheduler": {
        "type": "CosineAnnealingLR",
        "T_max": 100,
        "eta_min": 1e-6
      },
      "early_stopping": {
        "patience": 15,
        "min_delta": 0.01
      }
    },
    "loss": {
      "type": "WingLoss",
      "omega": 10,
      "epsilon": 2,
      "normalized": true
    }
  },

  "inference": {
    "tta": {
      "enabled": true,
      "augmentations": ["original", "horizontal_flip"]
    },
    "ensemble": {
      "enabled": true,
      "num_models": 4,
      "seeds": [123, 456, 321, 789],
      "aggregation": "mean",
      "excluded_seeds": [42]
    }
  },

  "data": {
    "split": {
      "train": 0.749,
      "val": 0.150,
      "test": 0.101
    },
    "stratified": true,
    "random_seed": 42
  },

  "hardware": {
    "gpu": "AMD Radeon RX 6600",
    "vram": "8 GB",
    "framework": "PyTorch 2.0+",
    "backend": "ROCm"
  },

  "results": {
    "baseline_error_px": 9.08,
    "final_error_px": 3.71,
    "improvement_percent": 59,
    "by_category": {
      "Normal": 3.50,
      "COVID-19": 3.80,
      "Viral_Pneumonia": 4.35
    }
  }
}
\end{lstlisting}

% -----------------------------------------------------------------------------
\section{Resumen Tabular de Hiperparametros}
\label{sec:ap_tabla_hiper}
% -----------------------------------------------------------------------------

\subsection{Arquitectura del Modelo}

\begin{table}[htbp]
    \centering
    \caption{Hiperparametros de arquitectura}
    \label{tab:ap_arch}
    \begin{tabular}{@{}llp{6cm}@{}}
        \toprule
        \textbf{Parametro} & \textbf{Valor} & \textbf{Justificacion} \\
        \midrule
        Backbone & ResNet-18 & Balance precision-eficiencia \\
        Pretrained & True & Transfer learning de ImageNet \\
        Coord Attention & True & Mejora localizacion espacial \\
        Reduction & 32 & Valor por defecto del paper \\
        Deep Head & True & Mayor capacidad de regresion \\
        Hidden Dim & 768 & Optimo experimental (S9) \\
        Dropout & 0.3 & Balance regularizacion-capacidad \\
        Output & Sigmoid & Normaliza a [0, 1] \\
        \bottomrule
    \end{tabular}
\end{table}

\subsection{Preprocesamiento}

\begin{table}[htbp]
    \centering
    \caption{Hiperparametros de preprocesamiento}
    \label{tab:ap_preproc}
    \begin{tabular}{@{}llp{6cm}@{}}
        \toprule
        \textbf{Parametro} & \textbf{Valor} & \textbf{Justificacion} \\
        \midrule
        Tamano entrada & 224 $\times$ 224 & Estandar ImageNet \\
        CLAHE clip & 2.0 & Optimo experimental (S8) \\
        CLAHE tile & 4 & Realce local fino \\
        Color space & LAB & Mejor que RGB para CLAHE \\
        Normalizacion & ImageNet & Compatibilidad con pesos \\
        \bottomrule
    \end{tabular}
\end{table}

\subsection{Entrenamiento}

\begin{table}[htbp]
    \centering
    \caption{Hiperparametros de entrenamiento}
    \label{tab:ap_train}
    \begin{tabular}{@{}lcc@{}}
        \toprule
        \textbf{Parametro} & \textbf{Fase 1} & \textbf{Fase 2} \\
        \midrule
        Epocas & 15 & 100 \\
        Batch size & 16 & 8 \\
        LR backbone & N/A & $2 \times 10^{-5}$ \\
        LR head & $10^{-3}$ & $2 \times 10^{-4}$ \\
        Optimizer & Adam & Adam \\
        Scheduler & StepLR & CosineAnnealing \\
        Patience & 5 & 15 \\
        \bottomrule
    \end{tabular}
\end{table}

\subsection{Data Augmentation}

\begin{table}[htbp]
    \centering
    \caption{Hiperparametros de data augmentation}
    \label{tab:ap_aug}
    \begin{tabular}{@{}llc@{}}
        \toprule
        \textbf{Augmentation} & \textbf{Parametro} & \textbf{Valor} \\
        \midrule
        Horizontal Flip & Probabilidad & 0.5 \\
        Rotation & Rango & $\pm 10\degree$ \\
        Brightness & Rango & $\pm 20\%$ \\
        Contrast & Rango & $\pm 20\%$ \\
        \bottomrule
    \end{tabular}
\end{table}

% -----------------------------------------------------------------------------
\section{Checkpoints del Ensemble Final}
\label{sec:ap_checkpoints}
% -----------------------------------------------------------------------------

\begin{table}[htbp]
    \centering
    \caption{Modelos del ensemble final}
    \label{tab:ap_checkpoints}
    \begin{tabular}{@{}clcc@{}}
        \toprule
        \textbf{Modelo} & \textbf{Ruta} & \textbf{Seed} & \textbf{Error (px)} \\
        \midrule
        1 & session10/ensemble/seed123/final\_model.pt & 123 & 4.05 \\
        2 & session10/ensemble/seed456/final\_model.pt & 456 & 4.04 \\
        3 & session13/seed321/final\_model.pt & 321 & 4.23 \\
        4 & session13/seed789/final\_model.pt & 789 & 4.37 \\
        \midrule
        \multicolumn{2}{l}{\textbf{Ensemble (promedio)}} & - & \textbf{3.71} \\
        \bottomrule
    \end{tabular}
\end{table}

% -----------------------------------------------------------------------------
\section{Comando de Entrenamiento}
\label{sec:ap_comando}
% -----------------------------------------------------------------------------

\begin{lstlisting}[caption={Comando para entrenar un modelo},label={lst:ap_train_cmd},language=bash]
python scripts/train.py \
    --seed 123 \
    --clahe \
    --clahe-clip 2.0 \
    --clahe-tile 4 \
    --coord-attention \
    --deep-head \
    --hidden-dim 768 \
    --dropout 0.3 \
    --epochs 100 \
    --patience 15 \
    --batch-size-phase1 16 \
    --batch-size-phase2 8 \
    --lr-backbone 2e-5 \
    --lr-head 2e-4 \
    --loss wing \
    --output-dir checkpoints/my_model
\end{lstlisting}

% -----------------------------------------------------------------------------
% FIN DEL APENDICE
% -----------------------------------------------------------------------------

    % =============================================================================
% APENDICE C: VISUALIZACIONES ADICIONALES
% =============================================================================

\chapter{Visualizaciones Adicionales}
\label{ap:visualizaciones}

Este apendice recopila las visualizaciones generadas durante el desarrollo del proyecto, organizadas por categoria.

% -----------------------------------------------------------------------------
\section{Diagramas de Arquitectura}
\label{sec:ap_diagramas}
% -----------------------------------------------------------------------------

Los diagramas de arquitectura se encuentran en el directorio \texttt{outputs/diagrams/}:

\begin{table}[htbp]
    \centering
    \caption{Diagramas de arquitectura generados}
    \label{tab:ap_diagrams}
    \begin{tabular}{@{}lp{8cm}@{}}
        \toprule
        \textbf{Archivo} & \textbf{Descripcion} \\
        \midrule
        model\_architecture.png & Diagrama de bloques del modelo completo \\
        coordinate\_attention.png & Detalle del modulo CoordinateAttention \\
        ensemble\_tta\_pipeline.png & Pipeline de inferencia con ensemble + TTA \\
        training\_pipeline.png & Diagrama del entrenamiento en 2 fases \\
        data\_flow.png & Flujo de datos simplificado \\
        \bottomrule
    \end{tabular}
\end{table}

\begin{figure}[htbp]
    \centering
    \includegraphics[width=\textwidth]{model_architecture.png}
    \caption{Arquitectura completa del modelo. La imagen de entrada pasa por ResNet-18, Coordinate Attention, Global Average Pooling y la cabeza de regresion profunda.}
    \label{fig:ap_architecture}
\end{figure}

\begin{figure}[htbp]
    \centering
    \includegraphics[width=\textwidth]{training_pipeline.png}
    \caption{Pipeline de entrenamiento en dos fases. Fase 1: backbone congelado (15 epocas). Fase 2: fine-tuning completo (100 epocas).}
    \label{fig:ap_training}
\end{figure}

\begin{figure}[htbp]
    \centering
    \includegraphics[width=\textwidth]{ensemble_tta_pipeline.png}
    \caption{Pipeline de inferencia con ensemble y TTA. Cada imagen genera 8 predicciones (4 modelos $\times$ 2 augmentaciones) que se promedian.}
    \label{fig:ap_ensemble}
\end{figure}

% -----------------------------------------------------------------------------
\section{Figuras de Resultados}
\label{sec:ap_resultados}
% -----------------------------------------------------------------------------

Las figuras de resultados se encuentran en el directorio \texttt{outputs/thesis\_figures/}:

\begin{table}[htbp]
    \centering
    \caption{Figuras de resultados generadas}
    \label{tab:ap_figures}
    \begin{tabular}{@{}lp{8cm}@{}}
        \toprule
        \textbf{Archivo} & \textbf{Descripcion} \\
        \midrule
        progress\_by\_session.png & Evolucion del error por sesion \\
        error\_by\_landmark.png & Error por landmark (15 barras) \\
        error\_by\_category.png & Comparacion por categoria \\
        heatmap\_landmark\_category.png & Heatmap de errores \\
        ensemble\_comparison.png & Modelos individuales vs ensemble \\
        ablation\_study.png & Contribucion de cada mejora \\
        summary\_table.png & Tabla resumen del proyecto \\
        clahe\_comparison.png & Efecto de CLAHE \\
        prediction\_examples.png & Ejemplos de predicciones \\
        best\_worst\_cases.png & Mejores y peores casos \\
        \bottomrule
    \end{tabular}
\end{table}

\begin{figure}[htbp]
    \centering
    \includegraphics[width=0.9\textwidth]{progress_by_session.png}
    \caption{Progreso del error a lo largo de las 15 sesiones de desarrollo. La linea muestra la reduccion progresiva desde 9.08 px (baseline) hasta 3.71 px (ensemble final).}
    \label{fig:ap_progress}
\end{figure}

\begin{figure}[htbp]
    \centering
    \includegraphics[width=0.9\textwidth]{ablation_study.png}
    \caption{Estudio de ablacion: contribucion de cada componente. El ensemble y CLAHE son los que mayor impacto tienen en la reduccion del error.}
    \label{fig:ap_ablation}
\end{figure}

% -----------------------------------------------------------------------------
\section{Visualizacion de Predicciones}
\label{sec:ap_predicciones}
% -----------------------------------------------------------------------------

\begin{figure}[htbp]
    \centering
    \includegraphics[width=\textwidth]{prediction_examples.png}
    \caption{Ejemplos de predicciones del ensemble en las tres categorias diagnosticas. Circulos verdes: ground truth; Cruces rojas: prediccion del modelo. Se observa alta precision en los tres tipos de imagenes.}
    \label{fig:ap_examples}
\end{figure}

\begin{figure}[htbp]
    \centering
    \includegraphics[width=\textwidth]{best_worst_cases.png}
    \caption{Mejores y peores casos del conjunto de prueba. Arriba: casos con menor error ($<$ 2 px promedio). Abajo: casos con mayor error ($>$ 8 px promedio). Los peores casos corresponden a imagenes con patologia severa o artefactos.}
    \label{fig:ap_best_worst}
\end{figure}

% -----------------------------------------------------------------------------
\section{Comparacion CLAHE}
\label{sec:ap_clahe}
% -----------------------------------------------------------------------------

\begin{figure}[htbp]
    \centering
    \includegraphics[width=\textwidth]{clahe_comparison.png}
    \caption{Efecto visual de CLAHE en radiografias de las tres categorias. Izquierda: imagen original. Derecha: imagen con CLAHE (clip=2.0, tile=4). Se observa mejor definicion de bordes anatomicos, especialmente en las consolidaciones de COVID-19.}
    \label{fig:ap_clahe}
\end{figure}

% -----------------------------------------------------------------------------
\section{Anatomia de Landmarks}
\label{sec:ap_anatomia}
% -----------------------------------------------------------------------------

\begin{figure}[htbp]
    \centering
    \begin{tikzpicture}[scale=0.8]
        % Silueta de torax simplificada
        \draw[thick] (0,8) -- (0,0) -- (10,0) -- (10,8);
        \draw[thick] (0,8) .. controls (2,9) and (8,9) .. (10,8);

        % Pulmones (simplificados)
        \draw[gray, thick] (1,1) .. controls (1,5) and (2,7) .. (3.5,7.5)
                          .. controls (4,7.5) and (4.5,5) .. (4.5,1) -- cycle;
        \draw[gray, thick] (9,1) .. controls (9,5) and (8,7) .. (6.5,7.5)
                          .. controls (6,7.5) and (5.5,5) .. (5.5,1) -- cycle;

        % Landmarks con etiquetas
        \node[circle, fill=red, inner sep=2pt, label=above:L1] at (5,8) {};
        \node[circle, fill=red, inner sep=2pt, label=below:L2] at (5,1) {};

        \node[circle, fill=blue, inner sep=2pt, label=left:L3] at (2.5,7) {};
        \node[circle, fill=blue, inner sep=2pt, label=right:L4] at (7.5,7) {};

        \node[circle, fill=green, inner sep=2pt, label=left:L5] at (2.5,5) {};
        \node[circle, fill=green, inner sep=2pt, label=right:L6] at (7.5,5) {};

        \node[circle, fill=orange, inner sep=2pt, label=left:L7] at (2,2) {};
        \node[circle, fill=orange, inner sep=2pt, label=right:L8] at (8,2) {};

        \node[circle, fill=purple, inner sep=2pt, label=left:L9] at (5,6.25) {};
        \node[circle, fill=purple, inner sep=2pt, label=left:L10] at (5,4.5) {};
        \node[circle, fill=purple, inner sep=2pt, label=left:L11] at (5,2.75) {};

        \node[circle, fill=cyan, inner sep=2pt, label=above left:L12] at (1.5,7.5) {};
        \node[circle, fill=cyan, inner sep=2pt, label=above right:L13] at (8.5,7.5) {};

        \node[circle, fill=brown, inner sep=2pt, label=below left:L14] at (1,1) {};
        \node[circle, fill=brown, inner sep=2pt, label=below right:L15] at (9,1) {};

        % Leyenda
        \node[anchor=west] at (11, 7) {\textcolor{red}{$\bullet$} Eje central (L1, L2)};
        \node[anchor=west] at (11, 6) {\textcolor{purple}{$\bullet$} Centrales (L9-L11)};
        \node[anchor=west] at (11, 5) {\textcolor{blue}{$\bullet$} Apices (L3, L4)};
        \node[anchor=west] at (11, 4) {\textcolor{green}{$\bullet$} Hilios (L5, L6)};
        \node[anchor=west] at (11, 3) {\textcolor{orange}{$\bullet$} Bases (L7, L8)};
        \node[anchor=west] at (11, 2) {\textcolor{cyan}{$\bullet$} Bordes sup. (L12, L13)};
        \node[anchor=west] at (11, 1) {\textcolor{brown}{$\bullet$} Costofrenicos (L14, L15)};
    \end{tikzpicture}
    \caption{Diagrama esquematico de la ubicacion de los 15 landmarks anatomicos en una radiografia de torax. Los landmarks centrales (L9, L10, L11) dividen el eje L1-L2 en proporciones de 0.25, 0.50 y 0.75.}
    \label{fig:ap_landmarks}
\end{figure}

% -----------------------------------------------------------------------------
\section{Scripts de Generacion}
\label{sec:ap_scripts}
% -----------------------------------------------------------------------------

Los scripts utilizados para generar las visualizaciones se encuentran en \texttt{scripts/visualization/}:

\begin{table}[htbp]
    \centering
    \caption{Scripts de visualizacion}
    \label{tab:ap_scripts}
    \begin{tabular}{@{}lp{8cm}@{}}
        \toprule
        \textbf{Script} & \textbf{Descripcion} \\
        \midrule
        generate\_architecture\_diagrams.py & Genera diagramas de arquitectura usando matplotlib \\
        generate\_results\_figures.py & Genera graficos de resultados y tablas \\
        generate\_prediction\_samples.py & Genera visualizaciones de predicciones \\
        \bottomrule
    \end{tabular}
\end{table}

\begin{lstlisting}[caption={Ejemplo de uso de scripts de visualizacion},label={lst:ap_vis_usage},language=bash]
# Generar todos los diagramas de arquitectura
python scripts/visualization/generate_architecture_diagrams.py \
    --output-dir outputs/diagrams/

# Generar figuras de resultados
python scripts/visualization/generate_results_figures.py \
    --results-file outputs/evaluation_results.json \
    --output-dir outputs/thesis_figures/

# Generar ejemplos de predicciones
python scripts/visualization/generate_prediction_samples.py \
    --checkpoint checkpoints/ensemble/ \
    --num-samples 10 \
    --output-dir outputs/thesis_figures/
\end{lstlisting}

% -----------------------------------------------------------------------------
% FIN DEL APENDICE
% -----------------------------------------------------------------------------

\end{appendices}

% =============================================================================
% FIN DEL DOCUMENTO
% =============================================================================
\end{document}
