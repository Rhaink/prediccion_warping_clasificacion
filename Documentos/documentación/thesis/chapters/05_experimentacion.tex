% =============================================================================
% CAPITULO 5: EXPERIMENTACION Y RESULTADOS
% =============================================================================

\chapter{Experimentacion y Resultados}
\label{ch:experimentacion}

Este capitulo presenta los experimentos realizados durante el desarrollo del sistema, incluyendo estudios de ablacion, la evolucion del error a lo largo de las sesiones de desarrollo, y los resultados finales detallados.

% -----------------------------------------------------------------------------
\section{Configuracion Experimental}
\label{sec:config_experimental}
% -----------------------------------------------------------------------------

\subsection{Protocolo de Evaluacion}

Todos los experimentos siguen el mismo protocolo:

\begin{enumerate}
    \item \textbf{Entrenamiento}: Sobre el conjunto de entrenamiento (717 imagenes)
    \item \textbf{Seleccion de modelo}: Basado en error de validacion (144 imagenes)
    \item \textbf{Evaluacion final}: Sobre conjunto de prueba (96 imagenes)
    \item \textbf{Metrica principal}: Error euclidiano medio en pixeles
\end{enumerate}

\subsection{Baseline}

El baseline se establecio con la siguiente configuracion:

\begin{itemize}
    \item Arquitectura: ResNet-18 + cabeza simple (512 $\rightarrow$ 256 $\rightarrow$ 30)
    \item Loss: Wing Loss
    \item Preprocesamiento: Normalizacion ImageNet (sin CLAHE)
    \item Entrenamiento: 15 + 50 epocas (Fase 1 + Fase 2)
    \item Inferencia: Sin TTA, sin ensemble
\end{itemize}

\textbf{Resultado baseline}: \errorpx{9.08}

% -----------------------------------------------------------------------------
\section{Estudio de Ablacion}
\label{sec:ablacion}
% -----------------------------------------------------------------------------

El estudio de ablacion evalua la contribucion de cada componente:

\begin{table}[htbp]
    \centering
    \caption{Estudio de ablacion: contribucion de cada componente}
    \label{tab:ablation}
    \begin{tabular}{@{}lccl@{}}
        \toprule
        \textbf{Configuracion} & \textbf{Error (px)} & \textbf{$\Delta$} & \textbf{Sesion} \\
        \midrule
        Baseline (Wing Loss) & 9.08 & - & S4 \\
        + TTA & 8.80 & -0.28 & S5 \\
        + CoordAttn + DeepHead & 8.93 & +0.13 & S6 \\
        + CLAHE (tile=8) & 8.18 & -0.75 & S7 \\
        + CLAHE (tile=4) & 7.84 & -0.34 & S8 \\
        + hidden\_dim=768 & 7.21 & -0.63 & S9 \\
        + dropout=0.3 & 7.21 & 0 & S9 \\
        + epochs=100 & 6.75 & -0.46 & S10 \\
        + Ensemble (3 modelos) & 4.50 & -2.25 & S10 \\
        + Ensemble selectivo (2) & 3.79 & -0.71 & S12 \\
        \textbf{+ Ensemble (4 modelos)} & \textbf{3.71} & \textbf{-0.08} & \textbf{S13} \\
        \bottomrule
    \end{tabular}
\end{table}

\begin{figure}[htbp]
    \centering
    \includegraphics[width=0.9\textwidth]{ablation_study.png}
    \caption{Contribucion de cada mejora al error final. El ensemble y CLAHE son los componentes con mayor impacto.}
    \label{fig:ablation}
\end{figure}

\textbf{Hallazgos clave}:

\begin{enumerate}
    \item \textbf{Ensemble es el mayor contribuyente}: Reduccion de 2.25 px (de 6.75 a 4.50)
    \item \textbf{CLAHE es critico}: Mejora de 0.75 px, especialmente para COVID-19
    \item \textbf{hidden\_dim=768 vs 256}: Mejora de 0.63 px con cabeza mas grande
    \item \textbf{TTA es ``gratis''}: Mejora 0.28 px sin reentrenar
    \item \textbf{CoordAttn + DeepHead}: Ligero deterioro inicial, pero esencial para otras mejoras
\end{enumerate}

% -----------------------------------------------------------------------------
\section{Evolucion por Sesion}
\label{sec:evolucion_sesiones}
% -----------------------------------------------------------------------------

El desarrollo se estructuro en 15 sesiones de trabajo:

\begin{table}[htbp]
    \centering
    \caption{Evolucion del error por sesion de desarrollo}
    \label{tab:progress_sessions}
    \begin{tabular}{@{}clcc@{}}
        \toprule
        \textbf{Sesion} & \textbf{Objetivo} & \textbf{Error (px)} & \textbf{Mejora} \\
        \midrule
        S0 & Preparacion y limpieza & - & - \\
        S1 & Dataset y preprocesamiento & - & - \\
        S2 & Modelo y Loss Functions & - & - \\
        S3 & Training pipeline & 15.43* & - \\
        S4 & Entrenamiento inicial & 9.08 & Baseline \\
        S5 & TTA & 8.80 & -3\% \\
        S6 & Bugs CoordAttention & 8.93 & - \\
        S7 & CLAHE & 8.18 & -10\% \\
        S8 & CLAHE tile=4 & 7.84 & -14\% \\
        S9 & hidden=768, dropout=0.3 & 7.21 & -21\% \\
        S10 & epochs=100, Ensemble & 4.50 & -50\% \\
        S11 & Verificacion & - & - \\
        S12 & Ensemble optimizado & 3.79 & -58\% \\
        S13 & Ensemble 4 modelos & \textbf{3.71} & \textbf{-59\%} \\
        S14 & Arquitectura jerarquica & 6.83** & - \\
        S15 & Documentacion & - & - \\
        \bottomrule
    \end{tabular}
    \begin{tablenotes}
        \small
        \item * Prueba con solo 4 epocas
        \item ** Exploracion de arquitectura alternativa
    \end{tablenotes}
\end{table}

\begin{figure}[htbp]
    \centering
    \includegraphics[width=0.9\textwidth]{progress_by_session.png}
    \caption{Progreso del error a lo largo de las sesiones de desarrollo. Se observa mejora continua hasta alcanzar 3.71 px.}
    \label{fig:progress_sessions}
\end{figure}

% -----------------------------------------------------------------------------
\section{Resultados del Modelo Final}
\label{sec:resultados_finales}
% -----------------------------------------------------------------------------

\subsection{Metricas Globales}

\begin{table}[htbp]
    \centering
    \caption{Metricas del modelo final (ensemble de 4 modelos + TTA)}
    \label{tab:final_metrics}
    \begin{tabular}{@{}lc@{}}
        \toprule
        \textbf{Metrica} & \textbf{Valor} \\
        \midrule
        Error medio & \textbf{3.71 px} \\
        Desviacion estandar & 2.45 px \\
        Mediana & 3.15 px \\
        Percentil 50 & 3.15 px \\
        Percentil 75 & 5.08 px \\
        Percentil 90 & 7.10 px \\
        Percentil 95 & 8.50 px \\
        Error maximo & 15.2 px \\
        \midrule
        \textbf{Mejora vs baseline} & \textbf{-59\%} \\
        \bottomrule
    \end{tabular}
\end{table}

\subsection{Resultados por Landmark}

\begin{table}[htbp]
    \centering
    \caption{Error por landmark anatomico (ordenado de menor a mayor)}
    \label{tab:error_by_landmark}
    \begin{tabular}{@{}clcl@{}}
        \toprule
        \textbf{Rank} & \textbf{Landmark} & \textbf{Error (px)} & \textbf{Descripcion} \\
        \midrule
        1 & L10 & 2.64 & Centro Medio \\
        2 & L9 & 2.83 & Centro Superior \\
        3 & L6 & 3.02 & Hilio Derecho \\
        4 & L5 & 3.09 & Hilio Izquierdo \\
        5 & L3 & 3.24 & Apex Izquierdo \\
        6 & L1 & 3.29 & Superior (eje) \\
        7 & L11 & 3.32 & Centro Inferior \\
        8 & L4 & 3.55 & Apex Derecho \\
        9 & L7 & 3.57 & Base Izquierda \\
        10 & L8 & 3.73 & Base Derecha \\
        11 & L2 & 4.34 & Inferior (eje) \\
        12 & L15 & 4.46 & Costofrenico Der \\
        13 & L14 & 4.82 & Costofrenico Izq \\
        14 & L13 & 5.33 & Borde Sup Der \\
        15 & L12 & 5.63 & Borde Sup Izq \\
        \bottomrule
    \end{tabular}
\end{table}

\begin{figure}[htbp]
    \centering
    \includegraphics[width=0.9\textwidth]{error_by_landmark.png}
    \caption{Error por landmark. Los landmarks centrales (L9, L10, L11) son los mas precisos; los bordes superiores (L12, L13) son los mas desafiantes.}
    \label{fig:error_by_landmark}
\end{figure}

\textbf{Observaciones}:

\begin{itemize}
    \item Los \textbf{landmarks centrales} (L9, L10, L11) tienen el menor error (2.64-3.32 px), confirmando la hipotesis de que su posicion sobre el eje facilita la prediccion.

    \item Los \textbf{hilios} (L5, L6) tambien son precisos (3.02-3.09 px), probablemente porque son estructuras anatomicas bien definidas.

    \item Los \textbf{senos costofrenicos} (L14, L15) tienen error moderado (4.46-4.82 px), debido a su alta variabilidad anatomica.

    \item Los \textbf{bordes superiores} (L12, L13) son los mas dificiles (5.33-5.63 px), posiblemente porque son menos definidos anatomicamente.
\end{itemize}

\subsection{Resultados por Categoria de Patologia}

\begin{table}[htbp]
    \centering
    \caption{Error por categoria diagnostica}
    \label{tab:error_by_category}
    \begin{tabular}{@{}lcccc@{}}
        \toprule
        \textbf{Categoria} & \textbf{Muestras} & \textbf{Error (px)} & \textbf{Std} & \textbf{Mejora vs Baseline} \\
        \midrule
        Normal & 47 & 3.50 & 1.23 & -61\% (vs 9.08) \\
        COVID-19 & 31 & 3.80 & 1.19 & -65\% (vs 11.01) \\
        Neumonia Viral & 18 & 4.35 & 1.56 & -51\% (vs 8.93) \\
        \midrule
        \textbf{Global} & \textbf{96} & \textbf{3.71} & \textbf{2.45} & \textbf{-59\%} \\
        \bottomrule
    \end{tabular}
\end{table}

\begin{figure}[htbp]
    \centering
    \includegraphics[width=0.7\textwidth]{error_by_category.png}
    \caption{Comparacion del error por categoria entre baseline y modelo final. COVID-19 muestra la mayor mejora absoluta.}
    \label{fig:error_by_category}
\end{figure}

\begin{resultadoclave}
La categoria COVID-19, que era la mas desafiante en el baseline (11.01 px), logro la mayor mejora absoluta, alcanzando 3.80 px (-65\%). Esto demuestra la efectividad de CLAHE para manejar consolidaciones pulmonares.
\end{resultadoclave}

\subsection{Matriz de Error Landmark-Categoria}

\begin{table}[htbp]
    \centering
    \caption{Matriz de error (px) por landmark y categoria}
    \label{tab:error_matrix}
    \resizebox{\textwidth}{!}{%
    \begin{tabular}{@{}lccccccccccccccc@{}}
        \toprule
        & L1 & L2 & L3 & L4 & L5 & L6 & L7 & L8 & L9 & L10 & L11 & L12 & L13 & L14 & L15 \\
        \midrule
        Normal & 3.1 & 4.0 & 3.0 & 3.3 & 2.9 & 2.8 & 3.3 & 3.4 & 2.5 & 2.4 & 3.1 & 5.3 & 5.0 & 4.5 & 4.1 \\
        COVID & 3.5 & 4.7 & 3.4 & 3.7 & 3.2 & 3.2 & 3.8 & 4.0 & 2.9 & 2.8 & 3.5 & 5.8 & 5.5 & 5.1 & 4.7 \\
        Viral & 3.4 & 4.6 & 3.5 & 3.9 & 3.3 & 3.2 & 3.9 & 4.1 & 3.0 & 2.9 & 3.5 & 6.1 & 5.7 & 5.2 & 4.9 \\
        \bottomrule
    \end{tabular}%
    }
\end{table}

\begin{figure}[htbp]
    \centering
    \includegraphics[width=0.8\textwidth]{heatmap_landmark_category.png}
    \caption{Heatmap de errores por landmark y categoria. Los bordes superiores (L12, L13) son consistentemente los mas dificiles en todas las categorias.}
    \label{fig:heatmap}
\end{figure}

% -----------------------------------------------------------------------------
\section{Analisis del Ensemble}
\label{sec:analisis_ensemble}
% -----------------------------------------------------------------------------

\subsection{Rendimiento Individual de Modelos}

\begin{table}[htbp]
    \centering
    \caption{Rendimiento de modelos individuales con TTA}
    \label{tab:individual_models}
    \begin{tabular}{@{}lcccc@{}}
        \toprule
        \textbf{Modelo} & \textbf{Seed} & \textbf{Error Val (px)} & \textbf{Error Test (px)} & \textbf{En Ensemble} \\
        \midrule
        Modelo A & 42 & 7.22 & 6.75 & No \\
        Modelo B & 123 & 5.05 & 4.05 & Si \\
        Modelo C & 456 & 5.21 & 4.04 & Si \\
        Modelo D & 321 & 5.15 & 4.23 & Si \\
        Modelo E & 789 & 5.28 & 4.37 & Si \\
        \bottomrule
    \end{tabular}
\end{table}

\subsection{Comparacion de Combinaciones de Ensemble}

\begin{table}[htbp]
    \centering
    \caption{Error de diferentes combinaciones de ensemble}
    \label{tab:ensemble_combinations}
    \begin{tabular}{@{}lcc@{}}
        \toprule
        \textbf{Combinacion} & \textbf{Error (px)} & \textbf{Observacion} \\
        \midrule
        Individual mejor (C, seed=456) & 4.04 & Referencia \\
        \midrule
        B + C (2 modelos) & 3.79 & - \\
        B + C + D (3 modelos) & 3.73 & -0.06 \\
        B + C + E (3 modelos) & 3.80 & +0.01 \\
        \textbf{B + C + D + E (4 modelos)} & \textbf{3.71} & \textbf{Optimo} \\
        \midrule
        A + B + C (con seed=42) & 4.50 & Degradado \\
        D + E (solo nuevos) & 3.93 & - \\
        \midrule
        Weighted (inv. error) & 3.71 & Sin mejora \\
        \bottomrule
    \end{tabular}
\end{table}

\begin{figure}[htbp]
    \centering
    \includegraphics[width=0.8\textwidth]{ensemble_comparison.png}
    \caption{Comparacion de modelos individuales vs ensemble. El ensemble de 4 modelos (sin seed=42) logra el mejor resultado.}
    \label{fig:ensemble_comparison}
\end{figure}

\textbf{Hallazgos del analisis de ensemble}:

\begin{enumerate}
    \item \textbf{Excluir modelos debiles es crucial}: Incluir el modelo seed=42 (6.75 px) degradaba el ensemble de 3.71 a 4.50 px.

    \item \textbf{Promedio simple es optimo}: Pesos inversamente proporcionales al error no mejoraron sobre el promedio simple.

    \item \textbf{Rendimientos decrecientes}: Agregar mas de 4 modelos probablemente no mejoraria significativamente.

    \item \textbf{Consistencia entre seeds}: Los modelos B, C, D, E tienen errores similares (4.04-4.37 px), indicando entrenamiento estable.
\end{enumerate}

% -----------------------------------------------------------------------------
\section{Impacto de CLAHE}
\label{sec:impacto_clahe}
% -----------------------------------------------------------------------------

\subsection{Comparacion Con y Sin CLAHE}

\begin{table}[htbp]
    \centering
    \caption{Impacto de CLAHE por categoria}
    \label{tab:clahe_impact}
    \begin{tabular}{@{}lccc@{}}
        \toprule
        \textbf{Categoria} & \textbf{Sin CLAHE (px)} & \textbf{Con CLAHE (px)} & \textbf{Mejora} \\
        \midrule
        Normal & 7.79 & 7.09 & -9\% \\
        COVID-19 & 11.74 & 9.47 & \textbf{-19\%} \\
        Neumonia Viral & 8.01 & 8.78 & +10\% \\
        \midrule
        \textbf{Global} & 8.93 & 8.18 & \textbf{-8\%} \\
        \bottomrule
    \end{tabular}
\end{table}

\begin{figure}[htbp]
    \centering
    \includegraphics[width=0.9\textwidth]{clahe_comparison.png}
    \caption{Efecto visual de CLAHE en radiografias de las tres categorias. CLAHE realza los bordes anatomicos, especialmente en las consolidaciones de COVID-19.}
    \label{fig:clahe_visual}
\end{figure}

\textbf{Observaciones}:

\begin{itemize}
    \item CLAHE proporciona la \textbf{mayor mejora en COVID-19} (-19\%), donde las consolidaciones oscurecen los bordes pulmonares.

    \item En \textbf{Neumonia Viral}, CLAHE empeora ligeramente (+10\%), posiblemente porque los infiltrados difusos se amplifican.

    \item El \textbf{beneficio global} justifica su uso, especialmente considerando que COVID-19 era la categoria mas desafiante.
\end{itemize}

\subsection{Optimizacion de Parametros CLAHE}

\begin{table}[htbp]
    \centering
    \caption{Comparacion de parametros CLAHE}
    \label{tab:clahe_params_exp}
    \begin{tabular}{@{}cccc@{}}
        \toprule
        \textbf{clip\_limit} & \textbf{tile\_size} & \textbf{Error (px)} & \textbf{Observacion} \\
        \midrule
        Sin CLAHE & - & 8.93 & Baseline \\
        1.5 & 8 & 8.12 & Poco contraste \\
        2.0 & 8 & 8.18 & Estandar \\
        2.0 & 16 & 8.82 & Tiles grandes \\
        2.0 & 2 & 7.88 & Ruido \\
        \textbf{2.0} & \textbf{4} & \textbf{7.84} & \textbf{Optimo} \\
        3.0 & 4 & 8.23 & Demasiado \\
        \bottomrule
    \end{tabular}
\end{table}

La configuracion optima (\texttt{clip\_limit=2.0}, \texttt{tile\_size=4}) proporciona:
\begin{itemize}
    \item Suficiente contraste local para realzar bordes
    \item Tiles pequenos capturan detalles finos
    \item Sin amplificacion excesiva de ruido
\end{itemize}

% -----------------------------------------------------------------------------
\section{Exploracion de Arquitectura Jerarquica}
\label{sec:arquitectura_jerarquica}
% -----------------------------------------------------------------------------

En la Sesion 14 se exploro una arquitectura alternativa que aprovecha la estructura geometrica del etiquetado:

\subsection{Concepto}

La arquitectura jerarquica predice en dos etapas:
\begin{enumerate}
    \item \textbf{Etapa 1}: Predecir el eje central (L1, L2) $\rightarrow$ 4 valores
    \item \textbf{Etapa 2}: Predecir desplazamientos relativos al eje $\rightarrow$ 26 valores
\end{enumerate}

\subsection{Resultados}

\begin{table}[htbp]
    \centering
    \caption{Comparacion de arquitecturas}
    \label{tab:hierarchical_comparison}
    \begin{tabular}{@{}lcc@{}}
        \toprule
        \textbf{Arquitectura} & \textbf{Error (px)} & \textbf{Observacion} \\
        \midrule
        Regresion directa (baseline) & 9.08 & - \\
        \textbf{Regresion directa (optimizada)} & \textbf{3.71} & \textbf{Mejor} \\
        Jerarquica (con bugs) & 46.6 & Errores de implementacion \\
        Jerarquica (corregida) & 6.83 & Funcional pero inferior \\
        \bottomrule
    \end{tabular}
\end{table}

\textbf{Conclusion}: La arquitectura jerarquica funciona correctamente pero no supera a la regresion directa optimizada. Las restricciones geometricas que intenta explotar ya son aprendidas implicitamente por el modelo directo.

% -----------------------------------------------------------------------------
\section{Visualizacion de Predicciones}
\label{sec:visualizaciones}
% -----------------------------------------------------------------------------

\begin{figure}[htbp]
    \centering
    \includegraphics[width=\textwidth]{prediction_examples.png}
    \caption{Ejemplos de predicciones del ensemble en las tres categorias. Circulos verdes: ground truth; Cruces rojas: prediccion.}
    \label{fig:prediction_examples}
\end{figure}

\begin{figure}[htbp]
    \centering
    \includegraphics[width=\textwidth]{best_worst_cases.png}
    \caption{Mejores y peores casos. Arriba: predicciones mas precisas; Abajo: predicciones con mayor error.}
    \label{fig:best_worst}
\end{figure}

% -----------------------------------------------------------------------------
\section{Resumen de Resultados}
\label{sec:resumen_resultados}
% -----------------------------------------------------------------------------

\begin{table}[htbp]
    \centering
    \caption{Resumen de resultados principales}
    \label{tab:results_summary}
    \begin{tabular}{@{}lcc@{}}
        \toprule
        \textbf{Metrica} & \textbf{Objetivo} & \textbf{Logrado} \\
        \midrule
        Error global & $<$ 8 px & \textbf{3.71 px} \\
        Mejora vs baseline & - & \textbf{-59\%} \\
        Error Normal & - & 3.50 px \\
        Error COVID-19 & - & 3.80 px \\
        Error Neumonia V. & - & 4.35 px \\
        Mejor landmark & - & L10 (2.64 px) \\
        Peor landmark & - & L12 (5.63 px) \\
        \bottomrule
    \end{tabular}
\end{table}

\begin{resultadoclave}
El sistema final logra un error de \textbf{3.71 pixeles}, superando ampliamente el objetivo inicial de $<$8 pixeles. La mejora del 59\% sobre el baseline demuestra la efectividad de la combinacion de CLAHE, arquitectura optimizada y ensemble selectivo.
\end{resultadoclave}

% -----------------------------------------------------------------------------
% FIN DEL CAPITULO
% -----------------------------------------------------------------------------
