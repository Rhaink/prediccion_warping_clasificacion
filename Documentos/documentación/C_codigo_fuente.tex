% ==============================================================================
% APÉNDICE C - CÓDIGO FUENTE DE FUNCIONES CLAVE
% Proyecto: Detección de COVID-19 mediante Landmarks Anatómicos
% Nivel: Doctoral/Científico - Completo y Detallado
% ==============================================================================

\documentclass[12pt,a4paper]{article}
% ==============================================================================
% PREÁMBULO LATEX PARA DOCUMENTACIÓN CIENTÍFICA DE NIVEL DOCTORAL
% Proyecto: Detección de COVID-19 mediante Landmarks Anatómicos y Warping Geométrico
% ==============================================================================

% --- Codificación y idioma ---
\usepackage[utf8]{inputenc}
\usepackage[T1]{fontenc}
\usepackage[spanish,es-tabla]{babel}

% --- Matemáticas ---
\usepackage{amsmath,amssymb,amsfonts,amsthm}
\usepackage{mathtools}
\usepackage{bm}  % Negritas en matemáticas

% --- Tablas profesionales ---
\usepackage{booktabs}
\usepackage{multirow}
\usepackage{array}
\usepackage{longtable}
\usepackage{tabularx}

% --- Figuras y gráficos ---
\usepackage{graphicx}
\usepackage{subfig}
\usepackage{float}
\usepackage{caption}

% --- Algoritmos y pseudocódigo ---
\usepackage{algorithm}
\usepackage{algorithmic}

% --- Colores (debe cargarse antes de listings) ---
\usepackage{xcolor}
\definecolor{covidred}{RGB}{220,53,69}
\definecolor{normalgreen}{RGB}{40,167,69}
\definecolor{viralblue}{RGB}{0,123,255}
\definecolor{codeblue}{RGB}{0,0,180}
\definecolor{codegray}{RGB}{128,128,128}

% --- Código fuente ---
\usepackage{listings}
\lstset{
    language=Python,
    basicstyle=\ttfamily\small,
    keywordstyle=\color{blue}\bfseries,
    commentstyle=\color{gray}\itshape,
    stringstyle=\color{red},
    numbers=left,
    numberstyle=\tiny\color{gray},
    stepnumber=1,
    numbersep=5pt,
    backgroundcolor=\color{white},
    frame=single,
    rulecolor=\color{black},
    tabsize=4,
    captionpos=b,
    breaklines=true,
    breakatwhitespace=false,
    showspaces=false,
    showstringspaces=false,
    showtabs=false,
    literate={á}{{\'a}}1 {é}{{\'e}}1 {í}{{\'i}}1 {ó}{{\'o}}1 {ú}{{\'u}}1
             {Á}{{\'A}}1 {É}{{\'E}}1 {Í}{{\'I}}1 {Ó}{{\'O}}1 {Ú}{{\'U}}1
             {ñ}{{\~n}}1 {Ñ}{{\~N}}1
             {ü}{{\"u}}1 {Ü}{{\"U}}1
             {¿}{{?`}}1 {¡}{{!`}}1
}

% --- Referencias y enlaces ---
\usepackage{hyperref}
\hypersetup{
    colorlinks=true,
    linkcolor=blue,
    filecolor=magenta,
    urlcolor=cyan,
    citecolor=blue,
    pdftitle={Documentación del Proyecto de Tesis},
    pdfauthor={},
}
\usepackage{cleveref}

% --- Bibliografía ---
\usepackage{natbib}
\bibliographystyle{plainnat}

% --- Geometría de página ---
\usepackage[a4paper,margin=2.5cm]{geometry}

% --- Espaciado ---
\usepackage{setspace}
\onehalfspacing

% --- Encabezados y pies ---
\usepackage{fancyhdr}
\pagestyle{fancy}
\fancyhf{}
\rhead{\rightmark}
\lhead{\leftmark}
\cfoot{\thepage}

% --- Entornos personalizados ---
\theoremstyle{definition}
\newtheorem{definicion}{Definición}[section]
\newtheorem{proposicion}{Proposición}[section]
\newtheorem{teorema}{Teorema}[section]
\newtheorem{lema}{Lema}[section]
\newtheorem{corolario}{Corolario}[section]

\theoremstyle{remark}
\newtheorem{observacion}{Observación}[section]
\newtheorem{nota}{Nota}[section]
\newtheorem{hipotesis}{Hipótesis}[section]

% --- Comandos personalizados ---
% Vectores y matrices
\newcommand{\vect}[1]{\mathbf{#1}}
\newcommand{\mat}[1]{\mathbf{#1}}

% Operadores
\DeclareMathOperator*{\argmin}{arg\,min}
\DeclareMathOperator*{\argmax}{arg\,max}
\DeclareMathOperator{\sgn}{sgn}
\DeclareMathOperator{\diag}{diag}
\DeclareMathOperator{\trace}{tr}

% Normas y productos
\newcommand{\norm}[1]{\left\|#1\right\|}
\newcommand{\abs}[1]{\left|#1\right|}
\newcommand{\inner}[2]{\langle #1, #2 \rangle}

% Conjuntos
\newcommand{\R}{\mathbb{R}}
\newcommand{\N}{\mathbb{N}}
\newcommand{\Z}{\mathbb{Z}}

% Espacios de imágenes y landmarks
\newcommand{\imgspace}{\mathcal{I}}
\newcommand{\landmarkspace}{\mathcal{L}}
\newcommand{\classspace}{\mathcal{Y}}

% Landmarks específicos
\newcommand{\landmark}[1]{L_{#1}}
\newcommand{\landmarkpair}[2]{(L_{#1}, L_{#2})}

% Funciones de pérdida
\newcommand{\loss}{\mathcal{L}}
\newcommand{\wingloss}{\mathcal{L}_{\text{wing}}}
\newcommand{\mseloss}{\mathcal{L}_{\text{MSE}}}

% Métricas
\newcommand{\accuracy}{\text{Acc}}
\newcommand{\precision}{\text{Prec}}
\newcommand{\recall}{\text{Rec}}
\newcommand{\fscore}{F_1}

% Abreviaciones
\newcommand{\ie}{\textit{i.e.}}
\newcommand{\eg}{\textit{e.g.}}
\newcommand{\etal}{\textit{et al.}}
\newcommand{\etc}{\textit{etc.}}

% Referencias a archivos del proyecto
\newcommand{\archivo}[1]{\texttt{#1}}
\newcommand{\funcion}[1]{\texttt{#1()}}
\newcommand{\clase}[1]{\texttt{#1}}
\newcommand{\parametro}[1]{\texttt{#1}}

% Unidades
\newcommand{\px}{\,\text{px}}
\newcommand{\epoch}{\,\text{época}}
\newcommand{\epochs}{\,\text{épocas}}

% --- Información del documento ---
\newcommand{\proyectotitulo}{Detección de COVID-19 en Radiografías de Tórax mediante Landmarks Anatómicos y Normalización Geométrica}
\newcommand{\proyectosubtitulo}{Documentación Científica del Proceso de Desarrollo}

% --- Formato de secciones ---
\usepackage{titlesec}
\titleformat{\section}
  {\normalfont\Large\bfseries}{\thesection}{1em}{}
\titleformat{\subsection}
  {\normalfont\large\bfseries}{\thesubsection}{1em}{}
\titleformat{\subsubsection}
  {\normalfont\normalsize\bfseries}{\thesubsubsection}{1em}{}

% --- Notas al margen para figuras sugeridas ---
\usepackage{marginnote}
\newcommand{\figurasugerida}[1]{\marginnote{\scriptsize\textcolor{blue}{[Figura: #1]}}}

% --- Cajas para resultados importantes ---
\usepackage{tcolorbox}
\newtcolorbox{resultadoimportante}[1][]{
    colback=green!5!white,
    colframe=green!75!black,
    fonttitle=\bfseries,
    title=Resultado Importante,
    #1
}

\newtcolorbox{hallazgo}[1][]{
    colback=blue!5!white,
    colframe=blue!75!black,
    fonttitle=\bfseries,
    title=Hallazgo,
    #1
}

\newtcolorbox{metodologia}[1][]{
    colback=yellow!5!white,
    colframe=yellow!75!black,
    fonttitle=\bfseries,
    title=Metodología,
    #1
}

\newtcolorbox{figuradescripcion}[1][]{
    colback=gray!5!white,
    colframe=gray!75!black,
    fonttitle=\bfseries,
    title=Descripción de Figura,
    #1
}

% ==============================================================================
% FIN DEL PREÁMBULO
% ==============================================================================


\title{Apéndice C:\\
Código Fuente de Funciones Clave}
\author{Documentación del Proceso de Desarrollo}
\date{Apéndice Técnico}

\begin{document}
\maketitle

\begin{abstract}
Este apéndice presenta el código fuente de las funciones más importantes
del proyecto. Se incluyen implementaciones de análisis Procrustes, GPA,
warping piecewise affine, y utilidades de entrenamiento. El código está
documentado con comentarios explicativos para facilitar su comprensión
y reproducibilidad.
\end{abstract}

\tableofcontents
\newpage

% ==============================================================================
\section{Análisis Procrustes}
% ==============================================================================

\subsection{Superposición Procrustes Ordinaria}

\begin{lstlisting}[language=Python, caption={Implementación de Procrustes ordinario}]
import numpy as np
from scipy.linalg import svd, det

def procrustes_analysis(X, Y, scaling=True, reflection=False):
    """
    Alinea la configuracion Y a la referencia X mediante
    analisis Procrustes.

    Parameters
    ----------
    X : ndarray, shape (k, m)
        Configuracion de referencia (k landmarks, m dimensiones)
    Y : ndarray, shape (k, m)
        Configuracion a alinear
    scaling : bool
        Si True, permite escalado
    reflection : bool
        Si True, permite reflexiones

    Returns
    -------
    Y_aligned : ndarray, shape (k, m)
        Configuracion Y alineada a X
    d : float
        Distancia Procrustes (suma de cuadrados residual)
    transform : dict
        Diccionario con parametros de transformacion
    """
    k, m = X.shape

    # Paso 1: Centrar ambas configuraciones
    mu_X = X.mean(axis=0)
    mu_Y = Y.mean(axis=0)
    X_centered = X - mu_X
    Y_centered = Y - mu_Y

    # Paso 2: Normalizar escala (centroid size)
    ss_X = np.sum(X_centered ** 2)
    ss_Y = np.sum(Y_centered ** 2)
    norm_X = np.sqrt(ss_X)
    norm_Y = np.sqrt(ss_Y)

    X_normalized = X_centered / norm_X
    Y_normalized = Y_centered / norm_Y

    # Paso 3: Calcular rotacion optima via SVD
    # M = Y^T @ X
    M = Y_normalized.T @ X_normalized
    U, S, Vt = svd(M)

    # Rotacion optima: R = V @ U^T
    R = Vt.T @ U.T

    # Verificar reflexion
    if det(R) < 0 and not reflection:
        # Corregir para obtener rotacion propia
        Vt[-1, :] *= -1
        S[-1] *= -1
        R = Vt.T @ U.T

    # Paso 4: Calcular escala optima
    if scaling:
        scale = S.sum()
    else:
        scale = 1.0

    # Paso 5: Aplicar transformacion
    Y_aligned = norm_X * scale * (Y_normalized @ R) + mu_X

    # Calcular distancia Procrustes
    d = np.sum((X_normalized - scale * Y_normalized @ R) ** 2)

    transform = {
        'rotation': R,
        'scale': scale * norm_X / norm_Y,
        'translation': mu_X - scale * (mu_Y @ R)
    }

    return Y_aligned, d, transform
\end{lstlisting}

% ==============================================================================
\section{Análisis Generalizado de Procrustes (GPA)}
% ==============================================================================

\subsection{Implementación del GPA}

\begin{lstlisting}[language=Python, caption={Implementación del algoritmo GPA}]
import numpy as np
from scipy.linalg import svd

def generalized_procrustes_analysis(shapes, tol=1e-6, max_iter=100):
    """
    Calcula la forma canonica mediante GPA iterativo.

    Parameters
    ----------
    shapes : ndarray, shape (n, k, m)
        n configuraciones de k landmarks en m dimensiones
    tol : float
        Tolerancia de convergencia
    max_iter : int
        Maximo numero de iteraciones

    Returns
    -------
    canonical : ndarray, shape (k, m)
        Forma canonica (consenso)
    aligned_shapes : ndarray, shape (n, k, m)
        Formas alineadas al consenso
    transforms : list of dict
        Transformaciones aplicadas a cada forma
    """
    n, k, m = shapes.shape

    # Paso 1: Centrar y normalizar todas las formas
    aligned = np.zeros_like(shapes)
    for i in range(n):
        # Centrar
        centered = shapes[i] - shapes[i].mean(axis=0)
        # Normalizar (centroid size)
        norm = np.sqrt(np.sum(centered ** 2))
        aligned[i] = centered / norm

    # Paso 2: Inicializar consenso con la primera forma
    consensus = aligned[0].copy()

    prev_error = np.inf
    transforms = [None] * n

    for iteration in range(max_iter):
        # Paso 3: Alinear cada forma al consenso actual
        for i in range(n):
            # SVD de Y^T @ X
            M = aligned[i].T @ consensus
            U, S, Vt = svd(M)
            R = Vt.T @ U.T

            # Asegurar rotacion propia (no reflexion)
            if np.linalg.det(R) < 0:
                Vt[-1, :] *= -1
                R = Vt.T @ U.T

            # Aplicar rotacion
            aligned[i] = aligned[i] @ R
            transforms[i] = {'rotation': R, 'iteration': iteration}

        # Paso 4: Calcular nuevo consenso (promedio)
        consensus = aligned.mean(axis=0)

        # Paso 5: Re-normalizar consenso
        consensus = consensus / np.sqrt(np.sum(consensus ** 2))

        # Paso 6: Calcular error de alineacion
        error = 0
        for i in range(n):
            error += np.sum((aligned[i] - consensus) ** 2)
        error /= n

        # Verificar convergencia
        if abs(prev_error - error) < tol:
            print(f"GPA convergio en {iteration + 1} iteraciones")
            break

        prev_error = error

    return consensus, aligned, transforms


def compute_canonical_shape(landmarks_list, image_size=224):
    """
    Wrapper para calcular forma canonica en coordenadas de pixel.

    Parameters
    ----------
    landmarks_list : list of ndarray
        Lista de arrays (k, 2) con landmarks de cada imagen
    image_size : int
        Tamano de imagen para desnormalizar

    Returns
    -------
    canonical_normalized : ndarray, shape (k, 2)
        Forma canonica normalizada (centroide=0, escala=1)
    canonical_pixels : ndarray, shape (k, 2)
        Forma canonica en coordenadas de pixel
    """
    # Convertir lista a array 3D
    shapes = np.array(landmarks_list)  # (n, k, 2)

    # Ejecutar GPA
    canonical_normalized, _, _ = generalized_procrustes_analysis(shapes)

    # Convertir a coordenadas de pixel
    # Escalar al tamano de imagen y centrar
    center = image_size / 2
    scale = image_size * 0.4  # Usar 40% del tamano como escala

    canonical_pixels = canonical_normalized * scale + center

    return canonical_normalized, canonical_pixels
\end{lstlisting}

% ==============================================================================
\section{Triangulación de Delaunay}
% ==============================================================================

\subsection{Generación de Triangulación}

\begin{lstlisting}[language=Python, caption={Triangulación Delaunay con puntos de borde}]
import numpy as np
from scipy.spatial import Delaunay

def create_delaunay_triangulation(landmarks, image_size=224,
                                   add_boundary=True):
    """
    Crea triangulacion Delaunay sobre landmarks.

    Parameters
    ----------
    landmarks : ndarray, shape (k, 2)
        Coordenadas de landmarks
    image_size : int
        Tamano de la imagen
    add_boundary : bool
        Si True, anade puntos en el borde de la imagen

    Returns
    -------
    triangulation : Delaunay
        Objeto de triangulacion scipy
    points : ndarray, shape (n_points, 2)
        Todos los puntos (landmarks + borde)
    triangles : ndarray, shape (n_triangles, 3)
        Indices de vertices de cada triangulo
    """
    points = landmarks.copy()

    if add_boundary:
        # Anadir 8 puntos en el borde
        boundary_points = np.array([
            [0, 0],                          # Esquina superior izquierda
            [image_size // 2, 0],            # Centro superior
            [image_size - 1, 0],             # Esquina superior derecha
            [image_size - 1, image_size // 2],  # Centro derecho
            [image_size - 1, image_size - 1],   # Esquina inferior derecha
            [image_size // 2, image_size - 1],  # Centro inferior
            [0, image_size - 1],             # Esquina inferior izquierda
            [0, image_size // 2]             # Centro izquierdo
        ])
        points = np.vstack([points, boundary_points])

    # Crear triangulacion
    triangulation = Delaunay(points)

    return triangulation, points, triangulation.simplices


def get_internal_triangles(triangles, n_landmarks):
    """
    Filtra triangulos que solo usan landmarks anatomicos.

    Parameters
    ----------
    triangles : ndarray, shape (n_triangles, 3)
        Todos los triangulos
    n_landmarks : int
        Numero de landmarks anatomicos

    Returns
    -------
    internal : ndarray
        Triangulos internos (solo landmarks anatomicos)
    """
    # Un triangulo es interno si todos sus vertices son landmarks
    mask = np.all(triangles < n_landmarks, axis=1)
    return triangles[mask]
\end{lstlisting}

% ==============================================================================
\section{Coordenadas Baricéntricas}
% ==============================================================================

\subsection{Cálculo de Coordenadas Baricéntricas}

\begin{lstlisting}[language=Python, caption={Cálculo eficiente de coordenadas baricéntricas}]
import numpy as np

def compute_barycentric_coords(point, triangle):
    """
    Calcula coordenadas baricentricas de un punto respecto a un triangulo.

    Parameters
    ----------
    point : tuple or ndarray
        Coordenadas (x, y) del punto
    triangle : ndarray, shape (3, 2)
        Vertices del triangulo [(x1,y1), (x2,y2), (x3,y3)]

    Returns
    -------
    lambdas : ndarray, shape (3,)
        Coordenadas baricentricas (lambda1, lambda2, lambda3)
    inside : bool
        True si el punto esta dentro del triangulo
    """
    x, y = point
    x1, y1 = triangle[0]
    x2, y2 = triangle[1]
    x3, y3 = triangle[2]

    # Calcular determinante (2 * area del triangulo)
    det = (y2 - y3) * (x1 - x3) + (x3 - x2) * (y1 - y3)

    if abs(det) < 1e-10:
        # Triangulo degenerado
        return np.array([0, 0, 0]), False

    # Coordenadas baricentricas
    lambda1 = ((y2 - y3) * (x - x3) + (x3 - x2) * (y - y3)) / det
    lambda2 = ((y3 - y1) * (x - x3) + (x1 - x3) * (y - y3)) / det
    lambda3 = 1.0 - lambda1 - lambda2

    lambdas = np.array([lambda1, lambda2, lambda3])

    # El punto esta dentro si todas las coordenadas son >= 0
    inside = np.all(lambdas >= -1e-10)  # Pequena tolerancia

    return lambdas, inside


def compute_barycentric_vectorized(points, triangle):
    """
    Version vectorizada para multiples puntos.

    Parameters
    ----------
    points : ndarray, shape (n, 2)
        Coordenadas de n puntos
    triangle : ndarray, shape (3, 2)
        Vertices del triangulo

    Returns
    -------
    lambdas : ndarray, shape (n, 3)
        Coordenadas baricentricas de cada punto
    inside : ndarray, shape (n,)
        Mascara de puntos dentro del triangulo
    """
    x1, y1 = triangle[0]
    x2, y2 = triangle[1]
    x3, y3 = triangle[2]

    det = (y2 - y3) * (x1 - x3) + (x3 - x2) * (y1 - y3)

    if abs(det) < 1e-10:
        n = len(points)
        return np.zeros((n, 3)), np.zeros(n, dtype=bool)

    x = points[:, 0]
    y = points[:, 1]

    lambda1 = ((y2 - y3) * (x - x3) + (x3 - x2) * (y - y3)) / det
    lambda2 = ((y3 - y1) * (x - x3) + (x1 - x3) * (y - y3)) / det
    lambda3 = 1.0 - lambda1 - lambda2

    lambdas = np.stack([lambda1, lambda2, lambda3], axis=1)
    inside = np.all(lambdas >= -1e-10, axis=1)

    return lambdas, inside
\end{lstlisting}

% ==============================================================================
\section{Warping Piecewise Affine}
% ==============================================================================

\subsection{Implementación Completa del Warping}

\begin{lstlisting}[language=Python, caption={Warping piecewise affine completo}]
import numpy as np
from scipy.spatial import Delaunay
from scipy.ndimage import map_coordinates

def compute_affine_transform(src_tri, dst_tri):
    """
    Calcula matriz de transformacion afin entre dos triangulos.

    Parameters
    ----------
    src_tri : ndarray, shape (3, 2)
        Triangulo fuente
    dst_tri : ndarray, shape (3, 2)
        Triangulo destino

    Returns
    -------
    M : ndarray, shape (2, 3)
        Matriz de transformacion afin [A | t]
    """
    # Construir sistema de ecuaciones
    # dst = src @ A.T + t
    # Resolver para A y t

    src_h = np.hstack([src_tri, np.ones((3, 1))])  # (3, 3)

    # Resolver src_h @ [a11, a21, t1].T = dst[:, 0]
    # y     src_h @ [a12, a22, t2].T = dst[:, 1]

    M = np.linalg.solve(src_h, dst_tri).T  # (2, 3)

    return M


def piecewise_affine_warp(image, src_landmarks, dst_landmarks,
                           output_size=None):
    """
    Aplica warping piecewise affine a una imagen.

    Parameters
    ----------
    image : ndarray, shape (H, W) or (H, W, C)
        Imagen a transformar
    src_landmarks : ndarray, shape (k, 2)
        Landmarks en la imagen fuente
    dst_landmarks : ndarray, shape (k, 2)
        Landmarks en la imagen destino (forma canonica)
    output_size : tuple, optional
        Tamano de salida (H, W). Por defecto igual que entrada.

    Returns
    -------
    warped : ndarray
        Imagen warpeada
    """
    if output_size is None:
        output_size = image.shape[:2]

    H_out, W_out = output_size
    H_in, W_in = image.shape[:2]

    # Anadir puntos de borde
    boundary = np.array([
        [0, 0], [W_out//2, 0], [W_out-1, 0],
        [W_out-1, H_out//2], [W_out-1, H_out-1],
        [W_out//2, H_out-1], [0, H_out-1], [0, H_out//2]
    ])

    src_points = np.vstack([src_landmarks, boundary])
    dst_points = np.vstack([dst_landmarks, boundary])

    # Crear triangulacion sobre puntos destino
    tri = Delaunay(dst_points)

    # Crear grid de coordenadas de salida
    y_coords, x_coords = np.mgrid[0:H_out, 0:W_out]
    output_coords = np.stack([x_coords.ravel(), y_coords.ravel()], axis=1)

    # Encontrar triangulo para cada pixel
    triangle_indices = tri.find_simplex(output_coords)

    # Inicializar mapa de coordenadas fuente
    src_x = np.zeros(H_out * W_out)
    src_y = np.zeros(H_out * W_out)

    # Procesar cada triangulo
    for tri_idx in range(len(tri.simplices)):
        # Mascara de pixeles en este triangulo
        mask = triangle_indices == tri_idx
        if not np.any(mask):
            continue

        # Obtener vertices del triangulo
        vertex_indices = tri.simplices[tri_idx]
        dst_tri = dst_points[vertex_indices]
        src_tri = src_points[vertex_indices]

        # Calcular transformacion afin inversa (dst -> src)
        M = compute_affine_transform(dst_tri, src_tri)

        # Aplicar transformacion a pixeles de este triangulo
        pixels = output_coords[mask]  # (n, 2)
        pixels_h = np.hstack([pixels, np.ones((len(pixels), 1))])
        src_pixels = pixels_h @ M.T  # (n, 2)

        src_x[mask] = src_pixels[:, 0]
        src_y[mask] = src_pixels[:, 1]

    # Reshape a imagen
    src_x = src_x.reshape(H_out, W_out)
    src_y = src_y.reshape(H_out, W_out)

    # Interpolar imagen fuente
    if image.ndim == 2:
        warped = map_coordinates(image, [src_y, src_x],
                                  order=1, mode='constant', cval=0)
    else:
        # Imagen multicanal
        warped = np.zeros((H_out, W_out, image.shape[2]))
        for c in range(image.shape[2]):
            warped[:, :, c] = map_coordinates(
                image[:, :, c], [src_y, src_x],
                order=1, mode='constant', cval=0
            )

    return warped
\end{lstlisting}

% ==============================================================================
\section{Modelo de Predicción de Landmarks}
% ==============================================================================

\subsection{Definición del Modelo}

\begin{lstlisting}[language=Python, caption={Modelo de regresión de landmarks}]
import torch
import torch.nn as nn
from torchvision import models

class LandmarkRegressor(nn.Module):
    """
    Red neuronal para prediccion de landmarks anatomicos.
    """

    def __init__(self, backbone='resnet18', num_landmarks=15,
                 pretrained=True):
        """
        Parameters
        ----------
        backbone : str
            Arquitectura del backbone
        num_landmarks : int
            Numero de landmarks a predecir
        pretrained : bool
            Usar pesos pre-entrenados de ImageNet
        """
        super().__init__()

        self.num_landmarks = num_landmarks
        self.output_dim = num_landmarks * 2  # (x, y) por landmark

        # Cargar backbone
        if backbone == 'resnet18':
            self.backbone = models.resnet18(pretrained=pretrained)
            in_features = self.backbone.fc.in_features
            self.backbone.fc = nn.Identity()
        elif backbone == 'resnet50':
            self.backbone = models.resnet50(pretrained=pretrained)
            in_features = self.backbone.fc.in_features
            self.backbone.fc = nn.Identity()
        elif backbone == 'efficientnet_b0':
            self.backbone = models.efficientnet_b0(pretrained=pretrained)
            in_features = self.backbone.classifier[1].in_features
            self.backbone.classifier = nn.Identity()
        elif backbone == 'densenet121':
            self.backbone = models.densenet121(pretrained=pretrained)
            in_features = self.backbone.classifier.in_features
            self.backbone.classifier = nn.Identity()
        else:
            raise ValueError(f"Backbone no soportado: {backbone}")

        # Cabeza de regresion
        self.regressor = nn.Sequential(
            nn.Linear(in_features, 512),
            nn.ReLU(),
            nn.Dropout(0.3),
            nn.Linear(512, 256),
            nn.ReLU(),
            nn.Dropout(0.3),
            nn.Linear(256, self.output_dim)
        )

    def forward(self, x):
        """
        Forward pass.

        Parameters
        ----------
        x : torch.Tensor, shape (B, 3, H, W)
            Batch de imagenes

        Returns
        -------
        landmarks : torch.Tensor, shape (B, num_landmarks, 2)
            Coordenadas predichas de landmarks
        """
        features = self.backbone(x)
        output = self.regressor(features)

        # Reshape a (B, num_landmarks, 2)
        landmarks = output.view(-1, self.num_landmarks, 2)

        return landmarks
\end{lstlisting}

\subsection{Función de Entrenamiento}

\begin{lstlisting}[language=Python, caption={Loop de entrenamiento para landmarks}]
import torch
from torch.optim import AdamW
from torch.optim.lr_scheduler import CosineAnnealingWarmRestarts

def train_landmark_model(model, train_loader, val_loader,
                          num_epochs=100, device='cuda'):
    """
    Entrena modelo de prediccion de landmarks.
    """
    model = model.to(device)

    # Optimizador
    optimizer = AdamW(model.parameters(), lr=1e-4, weight_decay=1e-4)

    # Scheduler
    scheduler = CosineAnnealingWarmRestarts(
        optimizer, T_0=10, T_mult=2, eta_min=1e-6
    )

    # Funcion de perdida
    criterion = nn.MSELoss()

    best_val_loss = float('inf')
    patience_counter = 0
    patience = 15

    for epoch in range(num_epochs):
        # Entrenamiento
        model.train()
        train_loss = 0

        for images, landmarks in train_loader:
            images = images.to(device)
            landmarks = landmarks.to(device)

            optimizer.zero_grad()
            predictions = model(images)
            loss = criterion(predictions, landmarks)
            loss.backward()

            # Gradient clipping
            torch.nn.utils.clip_grad_norm_(model.parameters(), 1.0)

            optimizer.step()
            train_loss += loss.item()

        train_loss /= len(train_loader)

        # Validacion
        model.eval()
        val_loss = 0

        with torch.no_grad():
            for images, landmarks in val_loader:
                images = images.to(device)
                landmarks = landmarks.to(device)
                predictions = model(images)
                loss = criterion(predictions, landmarks)
                val_loss += loss.item()

        val_loss /= len(val_loader)

        # Actualizar scheduler
        scheduler.step()

        # Early stopping
        if val_loss < best_val_loss:
            best_val_loss = val_loss
            patience_counter = 0
            torch.save(model.state_dict(), 'best_model.pth')
        else:
            patience_counter += 1
            if patience_counter >= patience:
                print(f"Early stopping en epoca {epoch}")
                break

        print(f"Epoch {epoch}: train_loss={train_loss:.4f}, "
              f"val_loss={val_loss:.4f}")

    return model
\end{lstlisting}

% ==============================================================================
\section{Modelo de Clasificación}
% ==============================================================================

\subsection{Definición del Clasificador}

\begin{lstlisting}[language=Python, caption={Clasificador de COVID-19}]
import torch
import torch.nn as nn
from torchvision import models

class CovidClassifier(nn.Module):
    """
    Clasificador de COVID-19 basado en radiografias de torax.
    """

    def __init__(self, backbone='mobilenet_v2', num_classes=3,
                 pretrained=True, dropout=0.5):
        super().__init__()

        self.num_classes = num_classes

        # Cargar backbone
        if backbone == 'mobilenet_v2':
            self.backbone = models.mobilenet_v2(pretrained=pretrained)
            in_features = self.backbone.classifier[1].in_features
            self.backbone.classifier = nn.Sequential(
                nn.Dropout(dropout),
                nn.Linear(in_features, num_classes)
            )
        elif backbone == 'resnet18':
            self.backbone = models.resnet18(pretrained=pretrained)
            in_features = self.backbone.fc.in_features
            self.backbone.fc = nn.Sequential(
                nn.Dropout(dropout),
                nn.Linear(in_features, num_classes)
            )
        elif backbone == 'efficientnet_b0':
            self.backbone = models.efficientnet_b0(pretrained=pretrained)
            in_features = self.backbone.classifier[1].in_features
            self.backbone.classifier = nn.Sequential(
                nn.Dropout(dropout),
                nn.Linear(in_features, num_classes)
            )
        else:
            raise ValueError(f"Backbone no soportado: {backbone}")

    def forward(self, x):
        return self.backbone(x)


def train_classifier(model, train_loader, val_loader,
                      num_epochs=50, device='cuda'):
    """
    Entrena clasificador de COVID-19.
    """
    model = model.to(device)

    # Optimizador
    optimizer = torch.optim.SGD(
        model.parameters(), lr=1e-3, momentum=0.9,
        weight_decay=1e-4, nesterov=True
    )

    # Scheduler
    scheduler = torch.optim.lr_scheduler.StepLR(
        optimizer, step_size=10, gamma=0.1
    )

    # Funcion de perdida con label smoothing
    criterion = nn.CrossEntropyLoss(label_smoothing=0.1)

    best_val_acc = 0

    for epoch in range(num_epochs):
        # Entrenamiento
        model.train()
        train_loss = 0
        correct = 0
        total = 0

        for images, labels in train_loader:
            images = images.to(device)
            labels = labels.to(device)

            optimizer.zero_grad()
            outputs = model(images)
            loss = criterion(outputs, labels)
            loss.backward()
            optimizer.step()

            train_loss += loss.item()
            _, predicted = outputs.max(1)
            total += labels.size(0)
            correct += predicted.eq(labels).sum().item()

        train_acc = 100. * correct / total

        # Validacion
        model.eval()
        val_correct = 0
        val_total = 0

        with torch.no_grad():
            for images, labels in val_loader:
                images = images.to(device)
                labels = labels.to(device)
                outputs = model(images)
                _, predicted = outputs.max(1)
                val_total += labels.size(0)
                val_correct += predicted.eq(labels).sum().item()

        val_acc = 100. * val_correct / val_total

        scheduler.step()

        if val_acc > best_val_acc:
            best_val_acc = val_acc
            torch.save(model.state_dict(), 'best_classifier.pth')

        print(f"Epoch {epoch}: train_acc={train_acc:.2f}%, "
              f"val_acc={val_acc:.2f}%")

    return model
\end{lstlisting}

% ==============================================================================
\section{Evaluación y Métricas}
% ==============================================================================

\subsection{Cálculo de Métricas}

\begin{lstlisting}[language=Python, caption={Funciones de evaluación}]
import numpy as np
from sklearn.metrics import (accuracy_score, precision_recall_fscore_support,
                             confusion_matrix, roc_auc_score)

def evaluate_landmarks(predictions, ground_truth):
    """
    Evalua predicciones de landmarks.

    Parameters
    ----------
    predictions : ndarray, shape (n, k, 2)
        Landmarks predichos
    ground_truth : ndarray, shape (n, k, 2)
        Ground truth

    Returns
    -------
    metrics : dict
        Diccionario con metricas
    """
    # Error por imagen
    errors = np.sqrt(np.sum((predictions - ground_truth) ** 2, axis=2))

    # Metricas globales
    mse = np.mean((predictions - ground_truth) ** 2)
    mae = np.mean(np.abs(predictions - ground_truth))
    med = np.mean(errors)  # Mean Euclidean Distance

    # Error por landmark
    error_per_landmark = np.mean(errors, axis=0)

    return {
        'mse': mse,
        'mae': mae,
        'med': med,
        'error_per_landmark': error_per_landmark,
        'std': np.std(errors)
    }


def evaluate_classification(predictions, ground_truth, num_classes=3):
    """
    Evalua predicciones de clasificacion.

    Parameters
    ----------
    predictions : ndarray, shape (n,)
        Predicciones de clase
    ground_truth : ndarray, shape (n,)
        Ground truth
    num_classes : int
        Numero de clases

    Returns
    -------
    metrics : dict
        Diccionario con metricas
    """
    accuracy = accuracy_score(ground_truth, predictions)

    precision, recall, f1, _ = precision_recall_fscore_support(
        ground_truth, predictions, average='macro'
    )

    cm = confusion_matrix(ground_truth, predictions)

    # Per-class metrics
    precision_per_class, recall_per_class, f1_per_class, _ = \
        precision_recall_fscore_support(
            ground_truth, predictions, average=None
        )

    return {
        'accuracy': accuracy,
        'precision_macro': precision,
        'recall_macro': recall,
        'f1_macro': f1,
        'confusion_matrix': cm,
        'precision_per_class': precision_per_class,
        'recall_per_class': recall_per_class,
        'f1_per_class': f1_per_class
    }


def mcnemar_test(y_true, pred_a, pred_b):
    """
    Test de McNemar para comparar dos clasificadores.

    Returns
    -------
    chi2 : float
        Estadistico chi-cuadrado
    p_value : float
        Valor p
    """
    from scipy.stats import chi2

    # Tabla de contingencia
    # n01: A incorrecto, B correcto
    # n10: A correcto, B incorrecto
    correct_a = (pred_a == y_true)
    correct_b = (pred_b == y_true)

    n01 = np.sum(~correct_a & correct_b)
    n10 = np.sum(correct_a & ~correct_b)

    if n01 + n10 == 0:
        return 0, 1.0

    chi2_stat = (n01 - n10) ** 2 / (n01 + n10)
    p_value = 1 - chi2.cdf(chi2_stat, df=1)

    return chi2_stat, p_value
\end{lstlisting}

% ==============================================================================
\section{Análisis de Robustez}
% ==============================================================================

\subsection{Aplicación de Perturbaciones}

\begin{lstlisting}[language=Python, caption={Funciones de perturbación}]
import cv2
import numpy as np
from PIL import Image
import io

def apply_jpeg_compression(image, quality):
    """Aplica compresion JPEG."""
    img_pil = Image.fromarray(image)
    buffer = io.BytesIO()
    img_pil.save(buffer, format='JPEG', quality=quality)
    buffer.seek(0)
    return np.array(Image.open(buffer))


def apply_gaussian_blur(image, sigma):
    """Aplica desenfoque gaussiano."""
    kernel_size = int(6 * sigma + 1) | 1  # Asegurar impar
    return cv2.GaussianBlur(image, (kernel_size, kernel_size), sigma)


def apply_gaussian_noise(image, sigma):
    """Anade ruido gaussiano."""
    noise = np.random.normal(0, sigma, image.shape)
    noisy = image.astype(float) + noise
    return np.clip(noisy, 0, 255).astype(np.uint8)


def apply_rotation(image, angle):
    """Aplica rotacion."""
    h, w = image.shape[:2]
    center = (w // 2, h // 2)
    M = cv2.getRotationMatrix2D(center, angle, 1.0)
    return cv2.warpAffine(image, M, (w, h))


def apply_scale(image, scale_factor):
    """Aplica escalado."""
    h, w = image.shape[:2]
    new_h, new_w = int(h * scale_factor), int(w * scale_factor)
    resized = cv2.resize(image, (new_w, new_h))

    # Recortar o rellenar para mantener tamano original
    if scale_factor > 1:
        start_h = (new_h - h) // 2
        start_w = (new_w - w) // 2
        return resized[start_h:start_h+h, start_w:start_w+w]
    else:
        result = np.zeros_like(image)
        start_h = (h - new_h) // 2
        start_w = (w - new_w) // 2
        result[start_h:start_h+new_h, start_w:start_w+new_w] = resized
        return result


def evaluate_robustness(model, test_loader, perturbation_fn,
                         param_values, device='cuda'):
    """
    Evalua robustez bajo perturbaciones.

    Returns
    -------
    results : dict
        Accuracy para cada nivel de perturbacion
    """
    model.eval()
    results = {}

    for param in param_values:
        correct = 0
        total = 0

        with torch.no_grad():
            for images, labels in test_loader:
                # Aplicar perturbacion
                images_np = images.numpy()
                perturbed = np.array([
                    perturbation_fn(img.transpose(1, 2, 0), param)
                    for img in images_np
                ])
                perturbed = torch.from_numpy(
                    perturbed.transpose(0, 3, 1, 2)
                ).float()

                perturbed = perturbed.to(device)
                labels = labels.to(device)

                outputs = model(perturbed)
                _, predicted = outputs.max(1)
                total += labels.size(0)
                correct += predicted.eq(labels).sum().item()

        results[param] = 100. * correct / total

    return results
\end{lstlisting}

% ==============================================================================
\section{Conclusiones}
% ==============================================================================

Este apéndice ha presentado el código fuente de las funciones clave del
proyecto, incluyendo:

\begin{enumerate}
    \item \textbf{Análisis Procrustes}: Alineación de configuraciones de landmarks
    \item \textbf{GPA}: Cálculo iterativo de forma canónica
    \item \textbf{Triangulación}: Construcción de malla Delaunay
    \item \textbf{Coordenadas baricéntricas}: Cálculo eficiente vectorizado
    \item \textbf{Warping}: Implementación completa de piecewise affine
    \item \textbf{Modelos}: Arquitecturas de landmarks y clasificación
    \item \textbf{Evaluación}: Métricas y tests estadísticos
    \item \textbf{Robustez}: Aplicación y evaluación de perturbaciones
\end{enumerate}

Todo el código está disponible en el repositorio del proyecto para
facilitar la reproducibilidad.

\end{document}
