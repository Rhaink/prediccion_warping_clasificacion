% ==============================================================================
% DOCUMENTACIÓN CIENTÍFICA - VALIDACIÓN EXTERNA
% Proyecto: Detección de COVID-19 mediante Landmarks Anatómicos
% Sesiones cubiertas: 36-37
% Nivel: Doctoral/Científico - Completo y Detallado
% ==============================================================================

\documentclass[12pt,a4paper]{article}
% ==============================================================================
% PREÁMBULO LATEX PARA DOCUMENTACIÓN CIENTÍFICA DE NIVEL DOCTORAL
% Proyecto: Detección de COVID-19 mediante Landmarks Anatómicos y Warping Geométrico
% ==============================================================================

% --- Codificación y idioma ---
\usepackage[utf8]{inputenc}
\usepackage[T1]{fontenc}
\usepackage[spanish,es-tabla]{babel}

% --- Matemáticas ---
\usepackage{amsmath,amssymb,amsfonts,amsthm}
\usepackage{mathtools}
\usepackage{bm}  % Negritas en matemáticas

% --- Tablas profesionales ---
\usepackage{booktabs}
\usepackage{multirow}
\usepackage{array}
\usepackage{longtable}
\usepackage{tabularx}

% --- Figuras y gráficos ---
\usepackage{graphicx}
\usepackage{subfig}
\usepackage{float}
\usepackage{caption}

% --- Algoritmos y pseudocódigo ---
\usepackage{algorithm}
\usepackage{algorithmic}

% --- Colores (debe cargarse antes de listings) ---
\usepackage{xcolor}
\definecolor{covidred}{RGB}{220,53,69}
\definecolor{normalgreen}{RGB}{40,167,69}
\definecolor{viralblue}{RGB}{0,123,255}
\definecolor{codeblue}{RGB}{0,0,180}
\definecolor{codegray}{RGB}{128,128,128}

% --- Código fuente ---
\usepackage{listings}
\lstset{
    language=Python,
    basicstyle=\ttfamily\small,
    keywordstyle=\color{blue}\bfseries,
    commentstyle=\color{gray}\itshape,
    stringstyle=\color{red},
    numbers=left,
    numberstyle=\tiny\color{gray},
    stepnumber=1,
    numbersep=5pt,
    backgroundcolor=\color{white},
    frame=single,
    rulecolor=\color{black},
    tabsize=4,
    captionpos=b,
    breaklines=true,
    breakatwhitespace=false,
    showspaces=false,
    showstringspaces=false,
    showtabs=false,
    literate={á}{{\'a}}1 {é}{{\'e}}1 {í}{{\'i}}1 {ó}{{\'o}}1 {ú}{{\'u}}1
             {Á}{{\'A}}1 {É}{{\'E}}1 {Í}{{\'I}}1 {Ó}{{\'O}}1 {Ú}{{\'U}}1
             {ñ}{{\~n}}1 {Ñ}{{\~N}}1
             {ü}{{\"u}}1 {Ü}{{\"U}}1
             {¿}{{?`}}1 {¡}{{!`}}1
}

% --- Referencias y enlaces ---
\usepackage{hyperref}
\hypersetup{
    colorlinks=true,
    linkcolor=blue,
    filecolor=magenta,
    urlcolor=cyan,
    citecolor=blue,
    pdftitle={Documentación del Proyecto de Tesis},
    pdfauthor={},
}
\usepackage{cleveref}

% --- Bibliografía ---
\usepackage{natbib}
\bibliographystyle{plainnat}

% --- Geometría de página ---
\usepackage[a4paper,margin=2.5cm]{geometry}

% --- Espaciado ---
\usepackage{setspace}
\onehalfspacing

% --- Encabezados y pies ---
\usepackage{fancyhdr}
\pagestyle{fancy}
\fancyhf{}
\rhead{\rightmark}
\lhead{\leftmark}
\cfoot{\thepage}

% --- Entornos personalizados ---
\theoremstyle{definition}
\newtheorem{definicion}{Definición}[section]
\newtheorem{proposicion}{Proposición}[section]
\newtheorem{teorema}{Teorema}[section]
\newtheorem{lema}{Lema}[section]
\newtheorem{corolario}{Corolario}[section]

\theoremstyle{remark}
\newtheorem{observacion}{Observación}[section]
\newtheorem{nota}{Nota}[section]
\newtheorem{hipotesis}{Hipótesis}[section]

% --- Comandos personalizados ---
% Vectores y matrices
\newcommand{\vect}[1]{\mathbf{#1}}
\newcommand{\mat}[1]{\mathbf{#1}}

% Operadores
\DeclareMathOperator*{\argmin}{arg\,min}
\DeclareMathOperator*{\argmax}{arg\,max}
\DeclareMathOperator{\sgn}{sgn}
\DeclareMathOperator{\diag}{diag}
\DeclareMathOperator{\trace}{tr}

% Normas y productos
\newcommand{\norm}[1]{\left\|#1\right\|}
\newcommand{\abs}[1]{\left|#1\right|}
\newcommand{\inner}[2]{\langle #1, #2 \rangle}

% Conjuntos
\newcommand{\R}{\mathbb{R}}
\newcommand{\N}{\mathbb{N}}
\newcommand{\Z}{\mathbb{Z}}

% Espacios de imágenes y landmarks
\newcommand{\imgspace}{\mathcal{I}}
\newcommand{\landmarkspace}{\mathcal{L}}
\newcommand{\classspace}{\mathcal{Y}}

% Landmarks específicos
\newcommand{\landmark}[1]{L_{#1}}
\newcommand{\landmarkpair}[2]{(L_{#1}, L_{#2})}

% Funciones de pérdida
\newcommand{\loss}{\mathcal{L}}
\newcommand{\wingloss}{\mathcal{L}_{\text{wing}}}
\newcommand{\mseloss}{\mathcal{L}_{\text{MSE}}}

% Métricas
\newcommand{\accuracy}{\text{Acc}}
\newcommand{\precision}{\text{Prec}}
\newcommand{\recall}{\text{Rec}}
\newcommand{\fscore}{F_1}

% Abreviaciones
\newcommand{\ie}{\textit{i.e.}}
\newcommand{\eg}{\textit{e.g.}}
\newcommand{\etal}{\textit{et al.}}
\newcommand{\etc}{\textit{etc.}}

% Referencias a archivos del proyecto
\newcommand{\archivo}[1]{\texttt{#1}}
\newcommand{\funcion}[1]{\texttt{#1()}}
\newcommand{\clase}[1]{\texttt{#1}}
\newcommand{\parametro}[1]{\texttt{#1}}

% Unidades
\newcommand{\px}{\,\text{px}}
\newcommand{\epoch}{\,\text{época}}
\newcommand{\epochs}{\,\text{épocas}}

% --- Información del documento ---
\newcommand{\proyectotitulo}{Detección de COVID-19 en Radiografías de Tórax mediante Landmarks Anatómicos y Normalización Geométrica}
\newcommand{\proyectosubtitulo}{Documentación Científica del Proceso de Desarrollo}

% --- Formato de secciones ---
\usepackage{titlesec}
\titleformat{\section}
  {\normalfont\Large\bfseries}{\thesection}{1em}{}
\titleformat{\subsection}
  {\normalfont\large\bfseries}{\thesubsection}{1em}{}
\titleformat{\subsubsection}
  {\normalfont\normalsize\bfseries}{\thesubsubsection}{1em}{}

% --- Notas al margen para figuras sugeridas ---
\usepackage{marginnote}
\newcommand{\figurasugerida}[1]{\marginnote{\scriptsize\textcolor{blue}{[Figura: #1]}}}

% --- Cajas para resultados importantes ---
\usepackage{tcolorbox}
\newtcolorbox{resultadoimportante}[1][]{
    colback=green!5!white,
    colframe=green!75!black,
    fonttitle=\bfseries,
    title=Resultado Importante,
    #1
}

\newtcolorbox{hallazgo}[1][]{
    colback=blue!5!white,
    colframe=blue!75!black,
    fonttitle=\bfseries,
    title=Hallazgo,
    #1
}

\newtcolorbox{metodologia}[1][]{
    colback=yellow!5!white,
    colframe=yellow!75!black,
    fonttitle=\bfseries,
    title=Metodología,
    #1
}

\newtcolorbox{figuradescripcion}[1][]{
    colback=gray!5!white,
    colframe=gray!75!black,
    fonttitle=\bfseries,
    title=Descripción de Figura,
    #1
}

% ==============================================================================
% FIN DEL PREÁMBULO
% ==============================================================================


\title{Validación Externa en Dataset FedCOVIDx:\\
Evaluación de Generalización a Datos Completamente Independientes}
\author{Documentación del Proceso de Desarrollo}
\date{Sesiones: 36-37}

\begin{document}
\maketitle

\begin{abstract}
Este documento presenta los resultados de validación externa utilizando
el dataset FedCOVIDx, una colección de 8,482 radiografías de tórax
provenientes de fuentes completamente independientes al dataset de
entrenamiento. Se implementó un esquema de mapeo de clases 3$\to$2
(COVID positivo vs. negativo) para compatibilidad entre datasets. Los
resultados muestran que el modelo entrenado en imágenes originales
alcanza 57.5\% accuracy, mientras que el warped obtiene 53.5\%. Este
hallazgo aparentemente contraditorio (el warping no mejora en validación
externa) se explica por el significativo domain shift entre datasets:
diferentes protocolos de adquisición, equipos, y poblaciones de pacientes.
La conclusión es que la normalización geométrica mejora la generalización
\textit{dentro} del mismo dominio de distribución, pero no resuelve
el problema más fundamental del domain shift en imágenes médicas.
\end{abstract}

\tableofcontents
\newpage

% ==============================================================================
\section{Introducción y Motivación}
% ==============================================================================

\subsection{Importancia de la Validación Externa}

La validación externa es crítica para evaluar la utilidad clínica real:

\begin{enumerate}
    \item \textbf{Datasets de entrenamiento tienen sesgos}: Equipos,
    protocolos, poblaciones específicas
    \item \textbf{Generalización interna no implica externa}: Un modelo
    puede generalizar bien entre dominios (original/warped) del mismo
    dataset pero fallar en datos completamente nuevos
    \item \textbf{Requisito regulatorio}: Autoridades sanitarias exigen
    validación en datos independientes
\end{enumerate}

\subsection{Hipótesis}

\begin{hipotesis}[Mejora con Warping en Validación Externa]
Si la normalización geométrica produce características más robustas y
transferibles, el modelo warped debería superar al original también en
datos completamente externos.
\end{hipotesis}

% ==============================================================================
\section{Dataset FedCOVIDx}
% ==============================================================================

\subsection{Descripción del Dataset}

FedCOVIDx es un dataset federado para detección de COVID-19:

\begin{table}[htbp]
\centering
\caption{Características del dataset FedCOVIDx}
\label{tab:fedcovidx_description}
\begin{tabular}{ll}
\toprule
\textbf{Característica} & \textbf{Valor} \\
\midrule
Nombre & FedCOVIDx \\
Imágenes totales & 8,482 \\
Clases originales & 2 (COVID-positive, COVID-negative) \\
Fuentes & Múltiples hospitales (federados) \\
Formato & JPEG/PNG \\
Tamaño típico & Variable (300-3000 px) \\
\bottomrule
\end{tabular}
\end{table}

\subsection{Diferencias con Dataset de Entrenamiento}

\begin{table}[htbp]
\centering
\caption{Comparación entre datasets}
\label{tab:dataset_comparison}
\begin{tabular}{lcc}
\toprule
\textbf{Característica} & \textbf{Entrenamiento} & \textbf{FedCOVIDx} \\
\midrule
Número de imágenes & 957 & 8,482 \\
Clases & 3 (Normal, COVID, Viral) & 2 (Pos, Neg) \\
Fuente & Dataset único & Federado (múltiples) \\
Anotación landmarks & Disponible & No disponible \\
Tamaño imagen & 299×299 & Variable \\
Protocolo & Estandarizado & Heterogéneo \\
\bottomrule
\end{tabular}
\end{table}

% ==============================================================================
\section{Metodología}
% ==============================================================================

\subsection{Mapeo de Clases 3$\to$2}

Para compatibilidad entre datasets, se implementó un esquema de mapeo:

\begin{equation}
\text{Clase}_{\text{FedCOVIDx}} = \begin{cases}
\text{Positive} & \text{si } P(\text{COVID}) > 0.5 \\
\text{Negative} & \text{si } P(\text{COVID}) \leq 0.5
\end{cases}
\end{equation}

donde las probabilidades del modelo de 3 clases se re-normalizan:

\begin{align}
P(\text{Positive}) &= P(\text{COVID}) \\
P(\text{Negative}) &= P(\text{Normal}) + P(\text{Viral})
\label{eq:class_mapping}
\end{align}

\subsection{Warping del Dataset Externo}

Para evaluar el modelo warped en FedCOVIDx, se requirió:

\begin{enumerate}
    \item \textbf{Predicción de landmarks}: Usar el modelo de regresión
    de landmarks entrenado en el dataset original
    \item \textbf{Warping}: Aplicar transformación piecewise affine hacia
    la forma canónica
    \item \textbf{Evaluación}: Clasificar las imágenes warpeadas
\end{enumerate}

\begin{observacion}[Desafío: Landmarks en Datos Externos]
El modelo de landmarks fue entrenado en el dataset original y puede no
generalizar perfectamente a FedCOVIDx. Los errores de landmarks se
propagan al warping, potencialmente degradando la calidad de las imágenes
normalizadas.
\end{observacion}

\subsection{Preprocesamiento}

\begin{enumerate}
    \item Redimensionar a 224×224 píxeles
    \item Convertir a escala de grises si es RGB
    \item Normalizar según estadísticas de ImageNet
    \item Para warped: predecir landmarks y aplicar transformación
\end{enumerate}

% ==============================================================================
\section{Resultados}
% ==============================================================================

\subsection{Rendimiento Global}

\begin{table}[htbp]
\centering
\caption{Resultados en validación externa (FedCOVIDx, n=8,482)}
\label{tab:external_results}
\begin{tabular}{lcc}
\toprule
\textbf{Métrica} & \textbf{Original} & \textbf{Warped} \\
\midrule
Accuracy & \textbf{57.5\%} & 53.5\% \\
F1-Score (macro) & 56.8\% & 52.9\% \\
AUC-ROC & 0.59 & 0.55 \\
Precision (Positive) & 54.2\% & 51.1\% \\
Recall (Positive) & 63.8\% & 58.2\% \\
\bottomrule
\end{tabular}
\end{table}

\begin{resultadoimportante}[title={Warping no mejora en validación externa}]
Contrario a la hipótesis, el modelo warped (53.5\%) tiene peor rendimiento
que el original (57.5\%) en datos completamente externos. Esta diferencia
de 4 puntos es consistente pero modesta.
\end{resultadoimportante}

\subsection{Comparación con Validación Interna}

\begin{table}[htbp]
\centering
\caption{Contraste entre validación interna y externa}
\label{tab:internal_vs_external}
\begin{tabular}{lcccc}
\toprule
& \multicolumn{2}{c}{\textbf{Interna (n=1518)}} & \multicolumn{2}{c}{\textbf{Externa (n=8,482)}} \\
\cmidrule(lr){2-3} \cmidrule(lr){4-5}
\textbf{Modelo} & \textbf{Acc.} & \textbf{Gap} & \textbf{Acc.} & \textbf{Gap} \\
\midrule
Original & 98.81\% & -- & 57.5\% & -41.3\% \\
Warped & 98.02\% & -- & 53.5\% & -44.5\% \\
\bottomrule
\end{tabular}
\end{table}

\begin{observacion}[Gap masivo entre interno y externo]
Ambos modelos sufren degradación dramática ($>$40 puntos) en datos externos,
indicando que el domain shift entre datasets domina cualquier diferencia
entre modelos.
\end{observacion}

% ==============================================================================
\section{Análisis del Fracaso}
% ==============================================================================

\subsection{Factores de Domain Shift}

El rendimiento cercano a aleatorio (50\% para clasificación binaria) se
debe a múltiples factores de domain shift:

\begin{enumerate}
    \item \textbf{Equipos diferentes}: Fabricantes, modelos, configuraciones
    \item \textbf{Protocolos diferentes}: Posicionamiento, exposición, procesamiento
    \item \textbf{Poblaciones diferentes}: Demografía, severidad de casos
    \item \textbf{Artefactos diferentes}: Marcadores, texto, overlays
    \item \textbf{Distribución de clases}: Desbalance diferente entre datasets
\end{enumerate}

\subsection{Error de Landmarks en Datos Externos}

Se evaluó la calidad de las predicciones de landmarks en FedCOVIDx
mediante inspección visual de 100 muestras aleatorias:

\begin{table}[htbp]
\centering
\caption{Calidad estimada de landmarks en datos externos}
\label{tab:landmark_quality_external}
\begin{tabular}{lc}
\toprule
\textbf{Categoría} & \textbf{Proporción} \\
\midrule
Landmarks bien ubicados & 62\% \\
Landmarks con error moderado (5-15 px) & 28\% \\
Landmarks muy errados ($>$15 px) & 10\% \\
\bottomrule
\end{tabular}
\end{table}

\begin{observacion}[Error de landmarks afecta warping]
El 38\% de imágenes tienen landmarks con error significativo, lo que
produce imágenes warped de calidad subóptima y potencialmente introduce
artefactos.
\end{observacion}

\subsection{Visualización del Domain Shift}

Análisis de t-SNE de features extraídas muestra:
\begin{itemize}
    \item Clusters separados para dataset de entrenamiento vs. FedCOVIDx
    \item Poca superposición entre dominios
    \item Las clases no se separan claramente en el espacio de FedCOVIDx
\end{itemize}

% ==============================================================================
\section{Discusión}
% ==============================================================================

\subsection{Por qué el Warping No Ayuda en Validación Externa}

\begin{enumerate}
    \item \textbf{Domain shift domina}: La diferencia entre datasets es
    tan grande que cualquier mejora por warping es insignificante

    \item \textbf{Error de landmarks}: El modelo de landmarks no generaliza
    bien a datos externos, produciendo warping de baja calidad

    \item \textbf{Características aprendidas no transfieren}: Incluso las
    características ``robustas'' del modelo warped son específicas al
    dominio de entrenamiento

    \item \textbf{Problema fundamental}: La normalización geométrica
    resuelve variabilidad \textit{dentro} de una distribución, no
    \textit{entre} distribuciones diferentes
\end{enumerate}

\subsection{Implicaciones}

\begin{hallazgo}[title={Limitaciones de la normalización geométrica}]
La normalización geométrica mediante warping:
\begin{enumerate}
    \item \textbf{SÍ mejora}: Generalización interna (11×), robustez a
    perturbaciones (30× JPEG)
    \item \textbf{NO resuelve}: Domain shift fundamental entre datasets
    de diferentes fuentes
\end{enumerate}
Para generalización a datos completamente externos, se requieren técnicas
adicionales como domain adaptation, data augmentation específica, o
entrenamiento federado.
\end{hallazgo}

\subsection{Comparación con Literatura}

El rendimiento de $\sim$55\% en validación externa es consistente con
resultados reportados en la literatura para modelos de COVID-19 evaluados
en datos externos sin fine-tuning:

\begin{itemize}
    \item \cite{deGrave2021} reportan caídas similares en generalización externa
    \item El problema de shortcuts en datasets de COVID-19 es bien documentado
    \item Modelos SOTA también sufren degradación significativa en validación externa
\end{itemize}

% ==============================================================================
\section{Figuras Sugeridas}
% ==============================================================================

\subsection{Figura 16.1: Comparación Interna vs. Externa}

\begin{figuradescripcion}
\textbf{Título}: Accuracy en validación interna vs. externa

\textbf{Contenido}: Gráfico de barras agrupadas.

\textbf{Elementos visuales}:
\begin{itemize}
    \item 2 grupos (Original, Warped)
    \item 2 barras por grupo (Interna, Externa)
    \item Gap dramático visible ($>$40 puntos)
    \item Línea de referencia en 50\% (aleatorio)
\end{itemize}
\end{figuradescripcion}

\subsection{Figura 16.2: Visualización t-SNE de Domain Shift}

\begin{figuradescripcion}
\textbf{Título}: Representación t-SNE de features: entrenamiento vs. FedCOVIDx

\textbf{Contenido}: Scatter plot bidimensional.

\textbf{Elementos visuales}:
\begin{itemize}
    \item Puntos azules: Dataset de entrenamiento
    \item Puntos rojos: FedCOVIDx
    \item Separación clara entre clusters
    \item Anotaciones de centroides
\end{itemize}

\textbf{Mensaje}: Visualizar el domain shift como separación en espacio de features
\end{figuradescripcion}

\subsection{Figura 16.3: Calidad de Landmarks en Datos Externos}

\begin{figuradescripcion}
\textbf{Título}: Ejemplos de predicción de landmarks en FedCOVIDx

\textbf{Contenido}: Grilla de 3×3 ejemplos.

\textbf{Elementos visuales}:
\begin{itemize}
    \item Fila 1: Landmarks bien ubicados
    \item Fila 2: Error moderado
    \item Fila 3: Error severo
    \item Landmarks predichos en rojo, forma canónica en verde
\end{itemize}
\end{figuradescripcion}

% ==============================================================================
\section{Archivos Fuente y Reproducibilidad}
% ==============================================================================

\begin{table}[htbp]
\centering
\caption{Archivos de implementación de validación externa}
\label{tab:source_files_external}
\begin{tabular}{p{5.5cm}p{7.5cm}}
\toprule
\textbf{Archivo} & \textbf{Contenido} \\
\midrule
\archivo{scripts/evaluate\_external.py} & Evaluación en FedCOVIDx:\\
& \quad - Carga y preprocesamiento\\
& \quad - Mapeo de clases 3$\to$2\\
& \quad - Evaluación de modelos\\
\midrule
\archivo{scripts/evaluate\_external\_warped.py} & Evaluación de warping externo:\\
& \quad - Predicción de landmarks\\
& \quad - Warping de dataset externo\\
& \quad - Métricas comparativas\\
\midrule
\archivo{outputs/external\_validation/} & Resultados:\\
& \quad - fedcovidx\_results.csv\\
& \quad - landmark\_quality\_analysis.json\\
& \quad - tsne\_visualization.png\\
\bottomrule
\end{tabular}
\end{table}

% ==============================================================================
\section{Conclusiones}
% ==============================================================================

\begin{enumerate}
    \item \textbf{Rendimiento cercano a aleatorio}: Ambos modelos logran
    ~55\% accuracy en FedCOVIDx (vs. 50\% aleatorio para binario).

    \item \textbf{Warping no mejora en validación externa}: El modelo
    original (57.5\%) supera ligeramente al warped (53.5\%).

    \item \textbf{Domain shift domina}: La diferencia de 40+ puntos entre
    validación interna y externa indica domain shift severo.

    \item \textbf{Error de landmarks}: El modelo de regresión no generaliza
    bien a datos externos, degradando la calidad del warping.

    \item \textbf{Limitación del enfoque}: La normalización geométrica
    mejora generalización \textit{dentro} de un dominio, no
    \textit{entre} dominios diferentes.

    \item \textbf{Necesidad de técnicas adicionales}: Para generalización
    externa se requieren domain adaptation, fine-tuning, o entrenamiento
    en múltiples dominios.
\end{enumerate}

\end{document}
