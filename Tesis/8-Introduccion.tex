\section*{Introducción}

% JUSTIFICACIÓN DE ESTA SECCIÓN:
% - Establece el contexto global del problema
% - Conecta con los objetivos ya definidos
% - Prepara al lector para la hipótesis

La radiografía de tórax constituye el estudio de imagenología médica más frecuente a nivel mundial, con aproximadamente 2 mil millones de procedimientos realizados anualmente. Esta modalidad diagnóstica es fundamental para la evaluación de patologías pulmonares, cardiovasculares y torácicas, proporcionando información crítica para la toma de decisiones clínicas.

\subsection*{Planteamiento del Problema}

El análisis automatizado de radiografías de tórax enfrenta un desafío fundamental: la \textbf{variabilidad geométrica inherente} a las condiciones de adquisición. Las diferencias en posicionamiento del paciente, distancia foco-detector, angulación del haz de rayos X y características anatómicas individuales introducen variaciones significativas en la representación espacial de las estructuras torácicas.

Esta variabilidad impacta negativamente en los sistemas de diagnóstico asistido por computadora (CAD), ya que:

\begin{enumerate}
    \item Los clasificadores aprenden características dependientes de la geometría específica del conjunto de entrenamiento.
    \item La robustez ante artefactos de compresión (común en entornos hospitalarios) se ve comprometida.
    \item La generalización a datos de diferentes instituciones es limitada.
\end{enumerate}

\subsection*{Pregunta de Investigación}

El presente trabajo aborda la siguiente pregunta central:

\begin{quote}
\textit{¿Cómo diseñar un sistema de normalización geométrica basado en landmarks anatómicos que mejore la robustez y precisión de clasificadores de patologías pulmonares en radiografías de tórax?}
\end{quote}

\subsection*{Enfoque Propuesto}

Se propone un pipeline de tres etapas:

\begin{enumerate}
    \item \textbf{Predicción de landmarks}: Localización automática de 15 puntos anatómicos mediante redes neuronales convolucionales con restricciones geométricas.
    \item \textbf{Normalización geométrica}: Transformación de la imagen a una forma canónica mediante warping afín por partes.
    \item \textbf{Clasificación}: Detección de patologías (COVID-19, neumonía viral, normal) sobre imágenes normalizadas.
\end{enumerate}

Este enfoque permite separar la variabilidad geométrica extrínseca de las características patológicas intrínsecas, mejorando tanto la precisión como la robustez del sistema.
