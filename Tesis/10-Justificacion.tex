\section*{Justificación}

% JUSTIFICACIÓN DE ESTA SECCIÓN:
% - Demuestra relevancia clínica y científica
% - Presenta contribuciones originales
% - Prepara al lector para el marco teórico

\subsection*{Relevancia Clínica}

La pandemia de COVID-19 evidenció dramáticamente la brecha entre la demanda de análisis radiológico y la capacidad disponible de especialistas. En el pico de la crisis sanitaria, los sistemas de salud enfrentaron:

\begin{itemize}
    \item Incremento exponencial en volumen de radiografías de tórax
    \item Necesidad de triaje rápido para priorización de pacientes
    \item Variabilidad en equipos y protocolos entre instituciones
    \item Compresión de imágenes para transmisión y almacenamiento
\end{itemize}

Los sistemas de diagnóstico asistido por computadora (CAD) ofrecen una solución potencial, pero su efectividad depende críticamente de la robustez ante las condiciones reales de operación hospitalaria.

\subsection*{Contribuciones Científicas}

Este trabajo aporta las siguientes contribuciones:

\begin{enumerate}
    \item \textbf{Función de pérdida geométrica multi-componente}: Integración de Wing Loss con restricciones de alineación central y simetría bilateral, respetando la anatomía torácica.

    \item \textbf{Análisis del mecanismo de robustez}: Identificación experimental de los factores causales que contribuyen a la mejora de robustez (reducción de información vs. normalización geométrica).

    \item \textbf{Validación rigurosa con limitaciones documentadas}: Evaluación en dataset externo (8,482 muestras) con análisis honesto del domain shift.

    \item \textbf{Pipeline reproducible}: Implementación completa con código fuente, configuraciones y checkpoints disponibles para replicación.
\end{enumerate}

\subsection*{Impacto Potencial}

La normalización geométrica propuesta puede servir como:

\begin{itemize}
    \item Preprocesamiento estándar para sistemas CAD pulmonares
    \item Base para segmentación automática mediante Active Shape Models
    \item Método de regularización implícita para mejorar generalización
    \item Técnica de reducción de sensibilidad a artefactos de compresión
\end{itemize}

\subsection*{Alcance y Limitaciones}

Es importante delimitar que este trabajo:
\begin{itemize}
    \item Se enfoca en radiografías PA de tórax (proyección posteroanterior)
    \item Valida dentro del dominio de entrenamiento con alta precisión
    \item Documenta honestamente las limitaciones de generalización cross-institucional
    \item No pretende reemplazar el diagnóstico médico profesional
\end{itemize}
