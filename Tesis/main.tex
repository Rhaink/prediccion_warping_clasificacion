\documentclass[12pt,letterpaper]{article}
\usepackage[spanish]{babel}
\usepackage[letterpaper, margin=2cm]{geometry}
\usepackage{graphicx}
\usepackage{float}
\usepackage{booktabs}
\usepackage{tabularx}
\usepackage{multirow}
\usepackage{array}
\usepackage{enumitem}
\usepackage{setspace}
\usepackage[hidelinks]{hyperref}
\usepackage{amsmath}  % Para ecuaciones

% Nombres en español
\renewcommand{\tablename}{Tabla}

\begin{document}

\setstretch{1.15}

% Página 1: Portada
%----------------------------------------------------------------------------------------
%       PÁGINA DE TÍTULO PERSONALIZADA (Universidad)
%----------------------------------------------------------------------------------------
\makeatletter
\begin{titlepage}
    \begin{center}
    \includegraphics[width=1.9in]{Figures/Logo_de_la_BUAP.png}~\\[0.5cm]
    
    {\Large{BENEMÉRITA UNIVERSIDAD AUTÓNOMA DE PUEBLA}} \\[0.4cm]
    \Large{FACULTAD DE CIENCIAS DE LA ELECTRÓNICA}\\
    {\Large{MAESTRÍA EN INGENIERÍA ELECTRÓNICA,}} \\
    {\Large{OPCIÓN INSTRUMENTACIÓN ELECTRÓNICA}} \\[0.8cm]
    
    {\Large{Tesis para obtener el grado de:}} \\
    {\Large{MAESTRO EN INGENIERÍA ELECTRÓNICA}} \\
    \vspace{0.5cm}

    \hrule
    \vspace{15pt}
    {\Large Normalización y alineación automática de la forma de la región pulmonar integrada con selección de características discriminantes para detección de neumonía y COVID-19} 
    \vspace{15pt}
    \hrule
    \vspace{0.5cm}

    {\large{Presenta:}} \\
    {\large{Lic. Rafael Alejandro Cruz Ovando*}} \\
    \vspace{0.5cm}
    
    {\large{Directores:}} \\
    {\large{Dr. Salvador Eugenio Ayala Raggi}} \\
    {\large{Dr. Aldrin Barreto Flores}} \\
        
    \vfill
        
    \small *Becario SECIHTI \hfill Puebla, Pue., Noviembre 2025
    
    \end{center}
%     \begin{figure}
%     \centering
%     \includegraphics[width=1\linewidth]{imagen3.png}
% \end{figure}
\end{titlepage}
\makeatother

% Página 2: Objetivos
\newpage
\input{5-Objetivos}

% Página 3: Cronograma
\newpage
\section*{Calendarización de actividades}

\begin{table}[H]
\centering
\small
\label{tab:calendarizacion}
\begin{tabular}{|p{4.5cm}|c|c|c|c|}
\hline
\textbf{Actividad} & \textbf{Otoño 2023} & \textbf{Primavera 2024} & \textbf{Otoño 2024} & \textbf{Primavera 2025} \\ \hline
Evaluación de programas relacionados & Oct - Ene &  &  &  \\ \hline
Revisión de trabajos relacionados &  & Ene - Feb &  &  \\ \hline
Adquisición y preprocesamiento de imágenes & & Feb - Mar  &  &  \\ \hline
Investigación de algoritmos deformables & & Mar - May  &  &  \\ \hline
Implementación de la segmentación &  &  & May - Oct &  \\ \hline
Normalización de imágenes &  &  & Oct - Dic  &  \\ \hline
Alineación de imágenes &  &  &  &  Ene - Mar \\ \hline
Extracción y selección de características &  &  &   & Mar - Abr \\ \hline
Desarrollo del clasificador &  &  &  & Abr - May  \\ \hline
Validación del clasificador &  &  &  & Jun \\ \hline
Análisis comparativo con otros métodos &  &  &  & Jul  \\ \hline
\end{tabular}
\end{table}


% Página 4: Porcentajes
\newpage
\section*{Porcentaje de avance de actividades}

\begin{table}[H]
\centering
\small
\label{tab:porcentajes}
\begin{tabular}{|p{4.5cm}|c|c|c|}
\hline
\textbf{Objetivo} & \textbf{\% Objetivo} & \textbf{\% Avance} & \textbf{\% Total} \\ \hline
Evaluación de programas relacionados &5\%  &100\%  & 5\%  \\ \hline
Revisión de trabajos relacionados &5\%  &100\%  &  5\%  \\ \hline
Adquisición y preprocesamiento de imágenes &10\% &100\%   &  10\%  \\ \hline
Investigación de algoritmos deformables &5\% &100\%   & 5\%   \\ \hline
Implementación de la segmentación &15\%  &100\%  &  15\%  \\ \hline
Normalización de imágenes &20\%  &100\%  & 20\%    \\ \hline
Alineación de imágenes &20\%  &100\%  &  20\%   \\ \hline
Extracción y selección de características &5\%  &100\%  & 5\%    \\ \hline
Desarrollo del clasificador &5\%  & 0\% & 0\%   \\ \hline
Validación del clasificador & 5\% & 0\% & 0\%   \\ \hline
Análisis comparativo con otros métodos &5\%  &0\%  & 0\%  \\ \hline
\textbf{Total} &  &  &  \textbf{85\%} \\ \hline
\end{tabular}
\end{table}


% Página 5: Introducción
\newpage
\section*{Introducción}

% JUSTIFICACIÓN DE ESTA SECCIÓN:
% - Establece el contexto global del problema
% - Conecta con los objetivos ya definidos
% - Prepara al lector para la hipótesis

La radiografía de tórax constituye el estudio de imagenología médica más frecuente a nivel mundial, con aproximadamente 2 mil millones de procedimientos realizados anualmente. Esta modalidad diagnóstica es fundamental para la evaluación de patologías pulmonares, cardiovasculares y torácicas, proporcionando información crítica para la toma de decisiones clínicas.

\subsection*{Planteamiento del Problema}

El análisis automatizado de radiografías de tórax enfrenta un desafío fundamental: la \textbf{variabilidad geométrica inherente} a las condiciones de adquisición. Las diferencias en posicionamiento del paciente, distancia foco-detector, angulación del haz de rayos X y características anatómicas individuales introducen variaciones significativas en la representación espacial de las estructuras torácicas.

Esta variabilidad impacta negativamente en los sistemas de diagnóstico asistido por computadora (CAD), ya que:

\begin{enumerate}
    \item Los clasificadores aprenden características dependientes de la geometría específica del conjunto de entrenamiento.
    \item La robustez ante artefactos de compresión (común en entornos hospitalarios) se ve comprometida.
    \item La generalización a datos de diferentes instituciones es limitada.
\end{enumerate}

\subsection*{Pregunta de Investigación}

El presente trabajo aborda la siguiente pregunta central:

\begin{quote}
\textit{¿Cómo diseñar un sistema de normalización geométrica basado en landmarks anatómicos que mejore la robustez y precisión de clasificadores de patologías pulmonares en radiografías de tórax?}
\end{quote}

\subsection*{Enfoque Propuesto}

Se propone un pipeline de tres etapas:

\begin{enumerate}
    \item \textbf{Predicción de landmarks}: Localización automática de 15 puntos anatómicos mediante redes neuronales convolucionales con restricciones geométricas.
    \item \textbf{Normalización geométrica}: Transformación de la imagen a una forma canónica mediante warping afín por partes.
    \item \textbf{Clasificación}: Detección de patologías (COVID-19, neumonía viral, normal) sobre imágenes normalizadas.
\end{enumerate}

Este enfoque permite separar la variabilidad geométrica extrínseca de las características patológicas intrínsecas, mejorando tanto la precisión como la robustez del sistema.


% Página 6: Hipótesis
\newpage
\section*{Hipótesis}

% JUSTIFICACIÓN DE ESTA SECCIÓN:
% - Define claramente qué se va a demostrar
% - Establece variables y predicciones cuantificables
% - Esencial para rigor científico ante jurado exigente

\subsection*{Hipótesis Principal}

La incorporación de restricciones geométricas anatómicas en la función de pérdida de redes neuronales convolucionales, combinada con normalización espacial mediante warping afín por partes, mejorará significativamente la robustez de clasificadores de patologías pulmonares ante artefactos de compresión y variabilidad geométrica.

\subsection*{Variables de Investigación}

\textbf{Variables Independientes:}
\begin{itemize}
    \item Función de pérdida: MSE, Wing Loss, Wing Loss + restricciones geométricas
    \item Estrategia de entrenamiento: backbone congelado vs. fine-tuning completo
    \item Porcentaje de cobertura (fill rate) del warping: 47\%, 96\%, 99\%
\end{itemize}

\textbf{Variables Dependientes:}
\begin{itemize}
    \item Error euclidiano de predicción de landmarks (píxeles)
    \item Accuracy de clasificación (\%)
    \item Degradación de accuracy bajo compresión JPEG (\%)
    \item Gap de generalización en validación cruzada (\%)
\end{itemize}

\subsection*{Predicciones Cuantificables}

\begin{enumerate}
    \item \textbf{Predicción de landmarks}: El ensemble de modelos alcanzará un error medio menor a 5 píxeles en imágenes de 224$\times$224.

    \item \textbf{Robustez a compresión}: El clasificador entrenado con imágenes normalizadas presentará al menos 5$\times$ menor degradación bajo compresión JPEG Q50 comparado con el baseline sin normalización.

    \item \textbf{Generalización}: El gap de accuracy en validación cruzada (modelo evaluado en dataset diferente al de entrenamiento) será al menos 2$\times$ menor para el modelo con normalización geométrica.
\end{enumerate}

\subsection*{Métrica de Evaluación Principal}

El error de localización de landmarks se define como:

\begin{equation}
E_k = \sqrt{(\hat{x}_k - x_k)^2 + (\hat{y}_k - y_k)^2}
\end{equation}

donde $(\hat{x}_k, \hat{y}_k)$ son las coordenadas predichas y $(x_k, y_k)$ las coordenadas de referencia (ground truth) para el landmark $k$.


% Página 7: Justificación
\newpage
\section*{Justificación}

% JUSTIFICACIÓN DE ESTA SECCIÓN:
% - Demuestra relevancia clínica y científica
% - Presenta contribuciones originales
% - Prepara al lector para el marco teórico

\subsection*{Relevancia Clínica}

La pandemia de COVID-19 evidenció dramáticamente la brecha entre la demanda de análisis radiológico y la capacidad disponible de especialistas. En el pico de la crisis sanitaria, los sistemas de salud enfrentaron:

\begin{itemize}
    \item Incremento exponencial en volumen de radiografías de tórax
    \item Necesidad de triaje rápido para priorización de pacientes
    \item Variabilidad en equipos y protocolos entre instituciones
    \item Compresión de imágenes para transmisión y almacenamiento
\end{itemize}

Los sistemas de diagnóstico asistido por computadora (CAD) ofrecen una solución potencial, pero su efectividad depende críticamente de la robustez ante las condiciones reales de operación hospitalaria.

\subsection*{Contribuciones Científicas}

Este trabajo aporta las siguientes contribuciones:

\begin{enumerate}
    \item \textbf{Función de pérdida geométrica multi-componente}: Integración de Wing Loss con restricciones de alineación central y simetría bilateral, respetando la anatomía torácica.

    \item \textbf{Análisis del mecanismo de robustez}: Identificación experimental de los factores causales que contribuyen a la mejora de robustez (reducción de información vs. normalización geométrica).

    \item \textbf{Validación rigurosa con limitaciones documentadas}: Evaluación en dataset externo (8,482 muestras) con análisis honesto del domain shift.

    \item \textbf{Pipeline reproducible}: Implementación completa con código fuente, configuraciones y checkpoints disponibles para replicación.
\end{enumerate}

\subsection*{Impacto Potencial}

La normalización geométrica propuesta puede servir como:

\begin{itemize}
    \item Preprocesamiento estándar para sistemas CAD pulmonares
    \item Base para segmentación automática mediante Active Shape Models
    \item Método de regularización implícita para mejorar generalización
    \item Técnica de reducción de sensibilidad a artefactos de compresión
\end{itemize}

\subsection*{Alcance y Limitaciones}

Es importante delimitar que este trabajo:
\begin{itemize}
    \item Se enfoca en radiografías PA de tórax (proyección posteroanterior)
    \item Valida dentro del dominio de entrenamiento con alta precisión
    \item Documenta honestamente las limitaciones de generalización cross-institucional
    \item No pretende reemplazar el diagnóstico médico profesional
\end{itemize}


% Páginas 8-9: Marco Teórico
\newpage
\section*{Marco Teórico}

% JUSTIFICACIÓN DE ESTA SECCIÓN:
% - Proporciona fundamentos físicos y matemáticos
% - Establece base para entender la metodología
% - Énfasis ingenieril con fórmulas relevantes

\subsection*{Radiografías de Tórax}

La radiografía de tórax se fundamenta en la atenuación diferencial de rayos X al atravesar tejidos de distinta densidad. La intensidad transmitida $I(x)$ sigue la ley de Beer-Lambert:

\begin{equation}
I(x) = I_0 \exp\left(-\int_0^x \mu(s) \, ds\right)
\end{equation}

donde $I_0$ es la intensidad incidente, $\mu(s)$ el coeficiente de atenuación lineal (dependiente del tejido), y $x$ la distancia recorrida.

Los coeficientes típicos de atenuación son:
\begin{itemize}
    \item Aire alveolar: $\mu \approx 0.0001$ cm$^{-1}$
    \item Tejidos blandos: $\mu \approx 0.20$ cm$^{-1}$
    \item Hueso cortical: $\mu \approx 0.50$ cm$^{-1}$
\end{itemize}

\subsubsection*{Landmarks Anatómicos}

Se definen 15 puntos de referencia que caracterizan la geometría torácica:

\begin{itemize}
    \item \textbf{Eje central} (L1, L2): Define la línea media vertical
    \item \textbf{Puntos centrales} (L9, L10, L11): Dividen el eje en cuartos
    \item \textbf{Pares bilaterales}: (L3-L4), (L5-L6), (L7-L8), (L12-L13), (L14-L15)
\end{itemize}

Estos landmarks presentan propiedades geométricas verificables: los puntos centrales se ubican exactamente en $t \in \{0.25, 0.50, 0.75\}$ a lo largo del eje, con desviación menor a 1.5 píxeles.

\subsection*{Redes Neuronales Convolucionales}

Las CNNs realizan extracción jerárquica de características mediante la operación de convolución discreta:

\begin{equation}
Y[i,j] = \sum_{m=0}^{M-1} \sum_{n=0}^{N-1} X[i+m, j+n] \cdot W[m,n] + b
\end{equation}

donde $X$ es la entrada, $W$ el kernel de convolución, y $b$ el sesgo.

\subsubsection*{Arquitectura ResNet}

Las redes residuales introducen conexiones de salto (skip connections) que permiten entrenar redes más profundas:

\begin{equation}
\mathbf{y} = \mathcal{F}(\mathbf{x}, \{W_i\}) + \mathbf{x}
\end{equation}

donde $\mathcal{F}$ representa las capas residuales y $\mathbf{x}$ la conexión directa.

ResNet-18, con 11.7 millones de parámetros, ofrece un balance óptimo entre capacidad representacional y eficiencia computacional para tareas de regresión de coordenadas.

\subsubsection*{Transfer Learning}

El preentrenamiento en ImageNet (1.2 millones de imágenes, 1000 clases) proporciona:
\begin{itemize}
    \item Filtros de bajo nivel (bordes, texturas) altamente transferibles
    \item Inicialización superior a pesos aleatorios
    \item Convergencia más rápida con menos datos médicos
\end{itemize}

Para tareas con menos de 10,000 imágenes etiquetadas, el transfer learning mejora consistentemente el rendimiento comparado con entrenamiento desde cero.


\end{document}
