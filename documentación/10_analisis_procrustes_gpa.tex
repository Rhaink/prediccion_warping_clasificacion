% ==============================================================================
% DOCUMENTACIÓN CIENTÍFICA - ANÁLISIS PROCRUSTES Y GPA
% Proyecto: Detección de COVID-19 mediante Landmarks Anatómicos
% Sesiones cubiertas: 18-19
% Nivel: Doctoral/Científico - Completo y Detallado
% ==============================================================================

\documentclass[12pt,a4paper]{article}
% ==============================================================================
% PREÁMBULO LATEX PARA DOCUMENTACIÓN CIENTÍFICA DE NIVEL DOCTORAL
% Proyecto: Detección de COVID-19 mediante Landmarks Anatómicos y Warping Geométrico
% ==============================================================================

% --- Codificación y idioma ---
\usepackage[utf8]{inputenc}
\usepackage[T1]{fontenc}
\usepackage[spanish,es-tabla]{babel}

% --- Matemáticas ---
\usepackage{amsmath,amssymb,amsfonts,amsthm}
\usepackage{mathtools}
\usepackage{bm}  % Negritas en matemáticas

% --- Tablas profesionales ---
\usepackage{booktabs}
\usepackage{multirow}
\usepackage{array}
\usepackage{longtable}
\usepackage{tabularx}

% --- Figuras y gráficos ---
\usepackage{graphicx}
\usepackage{subfig}
\usepackage{float}
\usepackage{caption}

% --- Algoritmos y pseudocódigo ---
\usepackage{algorithm}
\usepackage{algorithmic}

% --- Colores (debe cargarse antes de listings) ---
\usepackage{xcolor}
\definecolor{covidred}{RGB}{220,53,69}
\definecolor{normalgreen}{RGB}{40,167,69}
\definecolor{viralblue}{RGB}{0,123,255}
\definecolor{codeblue}{RGB}{0,0,180}
\definecolor{codegray}{RGB}{128,128,128}

% --- Código fuente ---
\usepackage{listings}
\lstset{
    language=Python,
    basicstyle=\ttfamily\small,
    keywordstyle=\color{blue}\bfseries,
    commentstyle=\color{gray}\itshape,
    stringstyle=\color{red},
    numbers=left,
    numberstyle=\tiny\color{gray},
    stepnumber=1,
    numbersep=5pt,
    backgroundcolor=\color{white},
    frame=single,
    rulecolor=\color{black},
    tabsize=4,
    captionpos=b,
    breaklines=true,
    breakatwhitespace=false,
    showspaces=false,
    showstringspaces=false,
    showtabs=false,
    literate={á}{{\'a}}1 {é}{{\'e}}1 {í}{{\'i}}1 {ó}{{\'o}}1 {ú}{{\'u}}1
             {Á}{{\'A}}1 {É}{{\'E}}1 {Í}{{\'I}}1 {Ó}{{\'O}}1 {Ú}{{\'U}}1
             {ñ}{{\~n}}1 {Ñ}{{\~N}}1
             {ü}{{\"u}}1 {Ü}{{\"U}}1
             {¿}{{?`}}1 {¡}{{!`}}1
}

% --- Referencias y enlaces ---
\usepackage{hyperref}
\hypersetup{
    colorlinks=true,
    linkcolor=blue,
    filecolor=magenta,
    urlcolor=cyan,
    citecolor=blue,
    pdftitle={Documentación del Proyecto de Tesis},
    pdfauthor={},
}
\usepackage{cleveref}

% --- Bibliografía ---
\usepackage{natbib}
\bibliographystyle{plainnat}

% --- Geometría de página ---
\usepackage[a4paper,margin=2.5cm]{geometry}

% --- Espaciado ---
\usepackage{setspace}
\onehalfspacing

% --- Encabezados y pies ---
\usepackage{fancyhdr}
\pagestyle{fancy}
\fancyhf{}
\rhead{\rightmark}
\lhead{\leftmark}
\cfoot{\thepage}

% --- Entornos personalizados ---
\theoremstyle{definition}
\newtheorem{definicion}{Definición}[section]
\newtheorem{proposicion}{Proposición}[section]
\newtheorem{teorema}{Teorema}[section]
\newtheorem{lema}{Lema}[section]
\newtheorem{corolario}{Corolario}[section]

\theoremstyle{remark}
\newtheorem{observacion}{Observación}[section]
\newtheorem{nota}{Nota}[section]
\newtheorem{hipotesis}{Hipótesis}[section]

% --- Comandos personalizados ---
% Vectores y matrices
\newcommand{\vect}[1]{\mathbf{#1}}
\newcommand{\mat}[1]{\mathbf{#1}}

% Operadores
\DeclareMathOperator*{\argmin}{arg\,min}
\DeclareMathOperator*{\argmax}{arg\,max}
\DeclareMathOperator{\sgn}{sgn}
\DeclareMathOperator{\diag}{diag}
\DeclareMathOperator{\trace}{tr}

% Normas y productos
\newcommand{\norm}[1]{\left\|#1\right\|}
\newcommand{\abs}[1]{\left|#1\right|}
\newcommand{\inner}[2]{\langle #1, #2 \rangle}

% Conjuntos
\newcommand{\R}{\mathbb{R}}
\newcommand{\N}{\mathbb{N}}
\newcommand{\Z}{\mathbb{Z}}

% Espacios de imágenes y landmarks
\newcommand{\imgspace}{\mathcal{I}}
\newcommand{\landmarkspace}{\mathcal{L}}
\newcommand{\classspace}{\mathcal{Y}}

% Landmarks específicos
\newcommand{\landmark}[1]{L_{#1}}
\newcommand{\landmarkpair}[2]{(L_{#1}, L_{#2})}

% Funciones de pérdida
\newcommand{\loss}{\mathcal{L}}
\newcommand{\wingloss}{\mathcal{L}_{\text{wing}}}
\newcommand{\mseloss}{\mathcal{L}_{\text{MSE}}}

% Métricas
\newcommand{\accuracy}{\text{Acc}}
\newcommand{\precision}{\text{Prec}}
\newcommand{\recall}{\text{Rec}}
\newcommand{\fscore}{F_1}

% Abreviaciones
\newcommand{\ie}{\textit{i.e.}}
\newcommand{\eg}{\textit{e.g.}}
\newcommand{\etal}{\textit{et al.}}
\newcommand{\etc}{\textit{etc.}}

% Referencias a archivos del proyecto
\newcommand{\archivo}[1]{\texttt{#1}}
\newcommand{\funcion}[1]{\texttt{#1()}}
\newcommand{\clase}[1]{\texttt{#1}}
\newcommand{\parametro}[1]{\texttt{#1}}

% Unidades
\newcommand{\px}{\,\text{px}}
\newcommand{\epoch}{\,\text{época}}
\newcommand{\epochs}{\,\text{épocas}}

% --- Información del documento ---
\newcommand{\proyectotitulo}{Detección de COVID-19 en Radiografías de Tórax mediante Landmarks Anatómicos y Normalización Geométrica}
\newcommand{\proyectosubtitulo}{Documentación Científica del Proceso de Desarrollo}

% --- Formato de secciones ---
\usepackage{titlesec}
\titleformat{\section}
  {\normalfont\Large\bfseries}{\thesection}{1em}{}
\titleformat{\subsection}
  {\normalfont\large\bfseries}{\thesubsection}{1em}{}
\titleformat{\subsubsection}
  {\normalfont\normalsize\bfseries}{\thesubsubsection}{1em}{}

% --- Notas al margen para figuras sugeridas ---
\usepackage{marginnote}
\newcommand{\figurasugerida}[1]{\marginnote{\scriptsize\textcolor{blue}{[Figura: #1]}}}

% --- Cajas para resultados importantes ---
\usepackage{tcolorbox}
\newtcolorbox{resultadoimportante}[1][]{
    colback=green!5!white,
    colframe=green!75!black,
    fonttitle=\bfseries,
    title=Resultado Importante,
    #1
}

\newtcolorbox{hallazgo}[1][]{
    colback=blue!5!white,
    colframe=blue!75!black,
    fonttitle=\bfseries,
    title=Hallazgo,
    #1
}

\newtcolorbox{metodologia}[1][]{
    colback=yellow!5!white,
    colframe=yellow!75!black,
    fonttitle=\bfseries,
    title=Metodología,
    #1
}

\newtcolorbox{figuradescripcion}[1][]{
    colback=gray!5!white,
    colframe=gray!75!black,
    fonttitle=\bfseries,
    title=Descripción de Figura,
    #1
}

% ==============================================================================
% FIN DEL PREÁMBULO
% ==============================================================================


\title{Generalized Procrustes Analysis para\\Cálculo de Forma Canónica:\\
Fundamentos Teóricos y Aplicación a Landmarks Torácicos}
\author{Documentación del Proceso de Desarrollo}
\date{Sesiones: 18-19}

\begin{document}
\maketitle

\begin{abstract}
Este documento presenta una implementación rigurosa del Generalized Procrustes
Analysis (GPA) para calcular la forma canónica de los 15 landmarks anatómicos
en radiografías de tórax. Se desarrollan los fundamentos matemáticos completos,
incluyendo la derivación de la solución óptima mediante descomposición en valores
singulares (SVD) y las propiedades de convergencia del algoritmo iterativo.
El análisis aplicado a 957 formas demuestra convergencia en solo 2 iteraciones,
eliminando el 99.9\% de la variabilidad debida a traslación, escala y rotación.
El análisis de componentes principales (PCA) sobre las formas alineadas revela
que únicamente 2 componentes capturan el 93.4\% de la varianza residual,
indicando que la variabilidad intrínseca de forma anatómica es extremadamente baja.
Un hallazgo crítico para el proyecto es que las formas correspondientes a las
categorías Normal, COVID y Viral Pneumonia son estadísticamente indistinguibles
(ANOVA: $F = 5.72$, $p = 0.003$, $\eta^2 = 1.2\%$), donde aunque existe
significancia estadística, el tamaño del efecto es despreciable, lo que
justifica el uso de una única forma canónica universal para el warping
geométrico de todas las categorías diagnósticas.
\end{abstract}

\tableofcontents
\newpage

% ==============================================================================
\section{Introducción y Motivación}
% ==============================================================================

\subsection{Contexto del Problema}

En el análisis de formas anatómicas, los landmarks extraídos de diferentes
pacientes presentan variabilidad debida a múltiples factores:

\begin{enumerate}
    \item \textbf{Variabilidad intrínseca}: Diferencias anatómicas reales
    entre individuos (morfología, tamaño de órganos, etc.)
    \item \textbf{Variabilidad extrínseca}:
    \begin{itemize}
        \item \textit{Traslación}: Posicionamiento del paciente en la radiografía
        \item \textit{Escala}: Distancia del paciente al detector, magnificación
        \item \textit{Rotación}: Inclinación del paciente, rotación del torso
    \end{itemize}
    \item \textbf{Ruido de medición}: Error en la localización de landmarks
\end{enumerate}

Para el warping geométrico normalizado, necesitamos una \textit{forma canónica}
que represente la anatomía promedio libre de variabilidad extrínseca. El
Generalized Procrustes Analysis (GPA) es el método estándar para obtener
esta forma canónica.

\subsection{Objetivos del Análisis}

El GPA se utiliza en este proyecto para:
\begin{enumerate}
    \item Calcular una forma canónica que represente la anatomía torácica promedio
    \item Eliminar variabilidad espuria (traslación, escala, rotación)
    \item Proporcionar la referencia geométrica para el warping piecewise affine
    \item Cuantificar la variabilidad residual de forma entre pacientes
    \item Determinar si existen diferencias de forma entre categorías diagnósticas
\end{enumerate}

\subsection{Decisión Metodológica: GPA vs Active Shape Models}

Antes de implementar, se evaluó si era necesario un modelo estadístico de forma
completo como Active Shape Models (ASM) \citep{cootes1995} o si GPA era suficiente:

\begin{table}[htbp]
\centering
\caption{Análisis comparativo GPA vs ASM para el problema}
\label{tab:gpa_vs_asm_detailed}
\begin{tabular}{p{4cm}cc}
\toprule
\textbf{Característica} & \textbf{GPA} & \textbf{ASM} \\
\midrule
Varianza explicada & 99.9\% & 100\% \\
Complejidad computacional & $O(nm^2)$ & $O(nm^2k)$ \\
Requiere modelo estadístico & No & Sí \\
Genera formas nuevas & No & Sí \\
Suficiente para warping & \textbf{Sí} & Sí \\
Restricción de variabilidad & No & Sí \\
\bottomrule
\end{tabular}
\end{table}

donde $n$ es el número de landmarks, $m$ es el número de formas, y $k$ es el
número de modos de variación en ASM.

\begin{hallazgo}[title={GPA es suficiente para nuestro objetivo}]
Dado que GPA captura el 99.9\% de la variabilidad y la varianza residual
es de muy baja dimensionalidad (2 componentes capturan 93.4\%), ASM no
aporta beneficios significativos para el objetivo de warping geométrico.
La complejidad adicional de ASM solo se justifica cuando se necesita
generar formas sintéticas o imponer restricciones estadísticas durante
la segmentación automática.
\end{hallazgo}

% ==============================================================================
\section{Fundamentos Teóricos del Análisis Procrustes}
% ==============================================================================

\subsection{Representación Matemática de Formas}

\begin{definicion}[Configuración de Landmarks]
Una forma se representa como una matriz de configuración $X \in \R^{n \times d}$
donde $n$ es el número de landmarks y $d$ es la dimensionalidad del espacio
(en nuestro caso, $d=2$ para coordenadas planares). Cada fila $\vect{x}_i = (x_i, y_i)$
representa las coordenadas del $i$-ésimo landmark.
\end{definicion}

Para nuestro problema específico:
\begin{equation}
X = \begin{pmatrix}
x_1 & y_1 \\
x_2 & y_2 \\
\vdots & \vdots \\
x_{15} & y_{15}
\end{pmatrix} \in \R^{15 \times 2}
\end{equation}

donde los 15 landmarks corresponden a las estructuras anatómicas definidas
en el Documento 01.

\subsection{Espacio de Formas y Transformaciones de Similitud}

\begin{definicion}[Grupo de Transformaciones de Similitud]
El grupo de similitud $\mathcal{S}$ en $\R^2$ consiste en todas las transformaciones
que preservan proporciones geométricas:
\begin{equation}
\mathcal{S} = \{(s, R, \vect{t}) : s > 0, R \in SO(2), \vect{t} \in \R^2\}
\end{equation}
donde $s$ es el factor de escala, $R$ es una matriz de rotación, y $\vect{t}$
es un vector de traslación.
\end{definicion}

La acción del grupo sobre una configuración $X$ está dada por:
\begin{equation}
\mathcal{T}(X; s, R, \vect{t}) = sXR + \vect{1}_n \vect{t}^T
\label{eq:similarity_transform}
\end{equation}
donde $\vect{1}_n$ es un vector de unos de dimensión $n$.

La matriz de rotación 2D tiene la forma:
\begin{equation}
R(\theta) = \begin{pmatrix}
\cos\theta & -\sin\theta \\
\sin\theta & \cos\theta
\end{pmatrix}
\end{equation}

\begin{definicion}[Espacio de Formas de Kendall]
El espacio de formas de Kendall $\Sigma_n^d$ se define como el espacio cociente
de configuraciones módulo el grupo de similitud:
\begin{equation}
\Sigma_n^d = \R^{n \times d}_* / \mathcal{S}
\end{equation}
donde $\R^{n \times d}_*$ denota configuraciones no degeneradas (rango $\geq 2$).
Dos configuraciones $X$ y $Y$ representan la misma forma si y solo si
existe una transformación de similitud que las relaciona.
\end{definicion}

\subsection{Análisis Procrustes Ordinario (OPA)}

El análisis Procrustes ordinario alinea dos formas minimizando la suma de
distancias cuadráticas entre landmarks correspondientes.

\begin{definicion}[Problema de Procrustes]
Dadas dos configuraciones $X, Y \in \R^{n \times d}$, el problema de Procrustes
consiste en encontrar:
\begin{equation}
\min_{s, R, \vect{t}} \|\mathcal{T}(X; s, R, \vect{t}) - Y\|_F^2 =
\min_{s, R, \vect{t}} \|sXR + \vect{1}_n\vect{t}^T - Y\|_F^2
\label{eq:procrustes_problem}
\end{equation}
donde $\|\cdot\|_F$ denota la norma de Frobenius.
\end{definicion}

\subsubsection{Solución Óptima: Paso 1 - Eliminación de Traslación}

\begin{teorema}[Centrado Óptimo]
La traslación óptima $\vect{t}^*$ que minimiza \eqref{eq:procrustes_problem}
es aquella que alinea los centroides de ambas configuraciones. Equivalentemente,
el problema se simplifica trabajando con configuraciones centradas.
\end{teorema}

\begin{proof}
Expandiendo el funcional de la ecuación \eqref{eq:procrustes_problem}:
\begin{align}
L(s, R, \vect{t}) &= \|sXR + \vect{1}_n\vect{t}^T - Y\|_F^2 \\
&= \text{tr}\left[(sXR + \vect{1}_n\vect{t}^T - Y)^T(sXR + \vect{1}_n\vect{t}^T - Y)\right]
\end{align}

Derivando respecto a $\vect{t}$ e igualando a cero:
\begin{equation}
\frac{\partial L}{\partial \vect{t}} = 2n(s\bar{\vect{x}}R + \vect{t} - \bar{\vect{y}}) = 0
\end{equation}

donde $\bar{\vect{x}} = \frac{1}{n}\sum_{i=1}^n \vect{x}_i$ y
$\bar{\vect{y}} = \frac{1}{n}\sum_{i=1}^n \vect{y}_i$ son los centroides.

Por tanto:
\begin{equation}
\vect{t}^* = \bar{\vect{y}} - s\bar{\vect{x}}R
\end{equation}

Sustituyendo en el funcional original, el problema se reduce a:
\begin{equation}
\min_{s, R} \|s\tilde{X}R - \tilde{Y}\|_F^2
\end{equation}
donde $\tilde{X} = X - \vect{1}_n\bar{\vect{x}}^T$ y
$\tilde{Y} = Y - \vect{1}_n\bar{\vect{y}}^T$ son las configuraciones centradas.
\end{proof}

La operación de centrado se implementa como:
\begin{equation}
\tilde{X} = (I_n - \frac{1}{n}\vect{1}_n\vect{1}_n^T)X = HX
\label{eq:centering}
\end{equation}
donde $H = I_n - \frac{1}{n}\vect{1}_n\vect{1}_n^T$ es la matriz de centrado
(idempotente y simétrica).

\subsubsection{Solución Óptima: Paso 2 - Normalización de Escala}

\begin{definicion}[Tamaño de Centroide]
El tamaño de centroide de una configuración centrada $\tilde{X}$ se define como:
\begin{equation}
S(X) = \|\tilde{X}\|_F = \sqrt{\sum_{i=1}^n \|\vect{x}_i - \bar{\vect{x}}\|^2}
\end{equation}
\end{definicion}

\begin{teorema}[Escala Óptima]
Para configuraciones centradas $\tilde{X}$ y $\tilde{Y}$, y rotación fija $R$,
la escala óptima es:
\begin{equation}
s^* = \frac{\text{tr}(\tilde{Y}^T \tilde{X} R)}{\|\tilde{X}\|_F^2}
\label{eq:optimal_scale}
\end{equation}
\end{teorema}

\begin{proof}
El funcional a minimizar (con $\tilde{X}$ y $\tilde{Y}$ centradas) es:
\begin{align}
L(s, R) &= \|s\tilde{X}R - \tilde{Y}\|_F^2 \\
&= s^2\|\tilde{X}\|_F^2 - 2s\,\text{tr}(\tilde{Y}^T\tilde{X}R) + \|\tilde{Y}\|_F^2
\end{align}

Derivando respecto a $s$:
\begin{equation}
\frac{\partial L}{\partial s} = 2s\|\tilde{X}\|_F^2 - 2\,\text{tr}(\tilde{Y}^T\tilde{X}R) = 0
\end{equation}

Resolviendo:
\begin{equation}
s^* = \frac{\text{tr}(\tilde{Y}^T\tilde{X}R)}{\|\tilde{X}\|_F^2}
\end{equation}
\end{proof}

Para el análisis de formas puras (sin considerar escala), se trabaja con
configuraciones pre-normalizadas:
\begin{equation}
\hat{X} = \frac{\tilde{X}}{\|\tilde{X}\|_F}
\label{eq:scaling}
\end{equation}

Esto sitúa todas las formas en la pre-forma esférica (esfera unitaria en
$\R^{n \times d}$).

\subsubsection{Solución Óptima: Paso 3 - Rotación Óptima via SVD}

\begin{teorema}[Rotación Óptima de Procrustes]
\label{thm:optimal_rotation}
Para configuraciones centradas y escaladas $\hat{X}$ y $\hat{Y}$, la matriz
de rotación óptima $R^*$ que minimiza $\|\hat{X}R - \hat{Y}\|_F^2$ está dada por:
\begin{equation}
R^* = VU^T
\label{eq:optimal_rotation}
\end{equation}
donde $U\Sigma V^T$ es la descomposición en valores singulares (SVD) de la
matriz de correlación cruzada $H = \hat{Y}^T\hat{X}$.
\end{teorema}

\begin{proof}
Para configuraciones normalizadas ($\|\hat{X}\|_F = \|\hat{Y}\|_F = 1$),
el funcional de Procrustes se simplifica:
\begin{align}
L(R) &= \|\hat{X}R - \hat{Y}\|_F^2 \\
&= \|\hat{X}\|_F^2 - 2\,\text{tr}(\hat{Y}^T\hat{X}R) + \|\hat{Y}\|_F^2 \\
&= 2 - 2\,\text{tr}(\hat{Y}^T\hat{X}R) \\
&= 2(1 - \text{tr}(HR))
\end{align}
donde $H = \hat{Y}^T\hat{X} \in \R^{d \times d}$.

Minimizar $L(R)$ equivale a maximizar $\text{tr}(HR)$ sujeto a $R \in SO(d)$.

Sea $H = U\Sigma V^T$ la SVD de $H$, donde $U, V \in SO(d)$ y
$\Sigma = \text{diag}(\sigma_1, \ldots, \sigma_d)$ con $\sigma_i \geq 0$.

Entonces:
\begin{equation}
\text{tr}(HR) = \text{tr}(U\Sigma V^T R) = \text{tr}(\Sigma V^T R U) = \text{tr}(\Sigma Q)
\end{equation}
donde $Q = V^T R U$ es también una matriz ortogonal.

Por la desigualdad de von Neumann:
\begin{equation}
\text{tr}(\Sigma Q) \leq \sum_{i=1}^d \sigma_i
\end{equation}
con igualdad si y solo si $Q = I$, es decir, $R = VU^T$.

Para asegurar $R \in SO(d)$ (rotación propia, no reflexión):
\begin{equation}
R^* = V \, \text{diag}(1, \ldots, 1, \det(VU^T)) \, U^T
\end{equation}
\end{proof}

\begin{corolario}[Distancia de Procrustes]
La distancia de Procrustes entre dos formas normalizadas es:
\begin{equation}
d_P(\hat{X}, \hat{Y}) = \sqrt{2(1 - \sum_{i=1}^d \sigma_i)}
\label{eq:procrustes_distance}
\end{equation}
donde $\sigma_i$ son los valores singulares de $H = \hat{Y}^T\hat{X}$.
\end{corolario}

\subsection{Generalized Procrustes Analysis (GPA)}

El GPA extiende el análisis Procrustes ordinario a múltiples formas
simultáneamente.

\begin{definicion}[Problema de GPA]
Dado un conjunto de $m$ configuraciones $\{X_1, \ldots, X_m\}$, el GPA
busca encontrar:
\begin{equation}
\min_{\{s_i, R_i, \vect{t}_i\}, \bar{X}} \sum_{i=1}^m \|s_i X_i R_i + \vect{1}_n\vect{t}_i^T - \bar{X}\|_F^2
\label{eq:gpa_problem}
\end{equation}
donde $\bar{X}$ es la forma consenso (media de Procrustes).
\end{definicion}

\subsubsection{Algoritmo Iterativo de GPA}

El problema \eqref{eq:gpa_problem} no tiene solución cerrada, pero puede
resolverse mediante un algoritmo iterativo que alterna entre:
\begin{enumerate}
    \item Fijar $\bar{X}$ y optimizar cada $(s_i, R_i, \vect{t}_i)$
    (problema de Procrustes ordinario)
    \item Fijar las transformaciones y calcular la nueva media $\bar{X}$
\end{enumerate}

\begin{algorithm}[H]
\caption{Generalized Procrustes Analysis Iterativo}
\label{alg:gpa_detailed}
\begin{algorithmic}[1]
\REQUIRE Formas $\{X_1, \ldots, X_m\}$, tolerancia $\epsilon$, máx. iteraciones $K$
\ENSURE Forma canónica $\bar{X}$, formas alineadas $\{\hat{X}_1, \ldots, \hat{X}_m\}$
\STATE \COMMENT{Paso 1: Pre-procesamiento}
\FOR{$i = 1, \ldots, m$}
    \STATE $\tilde{X}_i \leftarrow X_i - \vect{1}_n\bar{\vect{x}}_i^T$ \COMMENT{Centrar}
    \STATE $\hat{X}_i \leftarrow \tilde{X}_i / \|\tilde{X}_i\|_F$ \COMMENT{Normalizar escala}
\ENDFOR
\STATE \COMMENT{Paso 2: Inicializar referencia}
\STATE $\bar{X}^{(0)} \leftarrow \frac{1}{m}\sum_{i=1}^m \hat{X}_i$
\STATE $\bar{X}^{(0)} \leftarrow \bar{X}^{(0)} / \|\bar{X}^{(0)}\|_F$ \COMMENT{Re-normalizar}
\STATE \COMMENT{Paso 3: Iteración principal}
\FOR{$k = 1, \ldots, K$}
    \STATE \COMMENT{3a: Alinear cada forma con la referencia actual}
    \FOR{$i = 1, \ldots, m$}
        \STATE $H_i \leftarrow (\bar{X}^{(k-1)})^T \hat{X}_i$ \COMMENT{Correlación cruzada}
        \STATE $U_i \Sigma_i V_i^T \leftarrow \text{SVD}(H_i)$
        \STATE $R_i^* \leftarrow V_i U_i^T$ \COMMENT{Rotación óptima}
        \IF{$\det(R_i^*) < 0$}
            \STATE $V_i[:, -1] \leftarrow -V_i[:, -1]$
            \STATE $R_i^* \leftarrow V_i U_i^T$ \COMMENT{Corregir reflexión}
        \ENDIF
        \STATE $\hat{X}_i^{\text{aligned}} \leftarrow \hat{X}_i R_i^*$
    \ENDFOR
    \STATE \COMMENT{3b: Calcular nueva referencia}
    \STATE $\bar{X}^{(k)} \leftarrow \frac{1}{m}\sum_{i=1}^m \hat{X}_i^{\text{aligned}}$
    \STATE $\bar{X}^{(k)} \leftarrow \bar{X}^{(k)} / \|\bar{X}^{(k)}\|_F$ \COMMENT{Re-normalizar}
    \STATE \COMMENT{3c: Verificar convergencia}
    \STATE $\Delta \leftarrow \|\bar{X}^{(k)} - \bar{X}^{(k-1)}\|_F$
    \IF{$\Delta < \epsilon$}
        \STATE \textbf{break} \COMMENT{Convergencia alcanzada}
    \ENDIF
\ENDFOR
\RETURN $\bar{X}^{(k)}$, $\{\hat{X}_i^{\text{aligned}}\}_{i=1}^m$
\end{algorithmic}
\end{algorithm}

\subsubsection{Propiedades de Convergencia}

\begin{teorema}[Convergencia del GPA]
El algoritmo de GPA iterativo converge a un mínimo local del funcional
\eqref{eq:gpa_problem}. Si las formas iniciales tienen variabilidad limitada,
la convergencia es típicamente a un mínimo global.
\end{teorema}

\begin{proof}[Esquema de Demostración]
La convergencia se garantiza porque:
\begin{enumerate}
    \item Cada paso de alineación (Procrustes ordinario) reduce o mantiene
    la suma de distancias cuadráticas
    \item El cálculo de la media es el estimador que minimiza la suma de
    distancias cuadráticas para transformaciones fijas
    \item El funcional está acotado inferiormente por 0
    \item La secuencia de valores del funcional es monótona decreciente y acotada
\end{enumerate}
Por el teorema de convergencia monótona, la secuencia converge.
\end{proof}

\begin{proposicion}[Tasa de Convergencia]
Para formas con variabilidad moderada, el GPA típicamente exhibe convergencia
superlineal en las primeras iteraciones, seguida de convergencia lineal
cerca del punto fijo.
\end{proposicion}

\begin{observacion}[Convergencia Rápida en Nuestro Dataset]
En nuestro dataset de 957 formas torácicas, el GPA converge en solo
\textbf{2 iteraciones} con tolerancia $10^{-4}$. Esto indica que:
\begin{enumerate}
    \item La variabilidad de rotación en el dataset original es muy baja
    \item Las formas están bien definidas y consistentes
    \item La mayor parte de la variabilidad se elimina con centrado y escalado
\end{enumerate}
\end{observacion}

\subsection{Análisis de Componentes Principales Post-GPA}

Después de la alineación GPA, las formas alineadas $\{\hat{X}_i^{\text{aligned}}\}$
pueden analizarse mediante PCA para estudiar los modos de variación residual.

\begin{definicion}[PCA sobre Formas Alineadas]
Sea $\vect{z}_i = \text{vec}(\hat{X}_i^{\text{aligned}}) \in \R^{nd}$ la
vectorización de la forma alineada $i$. El PCA se realiza sobre la matriz
de datos $Z = [\vect{z}_1, \ldots, \vect{z}_m]^T \in \R^{m \times nd}$.
\end{definicion}

Los pasos son:
\begin{enumerate}
    \item Calcular la media: $\bar{\vect{z}} = \frac{1}{m}\sum_{i=1}^m \vect{z}_i$
    \item Centrar: $\tilde{Z} = Z - \vect{1}_m\bar{\vect{z}}^T$
    \item Matriz de covarianza: $C = \frac{1}{m-1}\tilde{Z}^T\tilde{Z} \in \R^{nd \times nd}$
    \item Eigendescomposición: $C = P\Lambda P^T$ donde $\Lambda = \text{diag}(\lambda_1, \ldots, \lambda_{nd})$
\end{enumerate}

Los eigenvectores $\vect{p}_j$ (modos de variación) pueden interpretarse
como deformaciones de la forma media:
\begin{equation}
X(\alpha_j) = \bar{X} + \alpha_j \cdot \text{reshape}(\vect{p}_j, n, d)
\end{equation}
donde típicamente $\alpha_j \in [-3\sqrt{\lambda_j}, +3\sqrt{\lambda_j}]$
para capturar el 99.7\% de la variación bajo normalidad.

% ==============================================================================
\section{Implementación y Resultados Experimentales}
% ==============================================================================

\subsection{Configuración Experimental}

\begin{table}[htbp]
\centering
\caption{Parámetros de configuración del GPA}
\label{tab:gpa_config}
\begin{tabular}{ll}
\toprule
\textbf{Parámetro} & \textbf{Valor} \\
\midrule
Número de formas ($m$) & 957 \\
Landmarks por forma ($n$) & 15 \\
Dimensión ($d$) & 2 \\
Tolerancia de convergencia ($\epsilon$) & $10^{-4}$ \\
Máximo iteraciones ($K$) & 100 \\
Normalización & Norma Frobenius unitaria \\
Tamaño imagen original & 299 $\times$ 299 px \\
Tamaño imagen objetivo & 224 $\times$ 224 px \\
\bottomrule
\end{tabular}
\end{table}

\subsection{Resultados de Convergencia}

\begin{table}[htbp]
\centering
\caption{Métricas de convergencia del GPA}
\label{tab:gpa_convergence}
\begin{tabular}{lc}
\toprule
\textbf{Métrica} & \textbf{Valor} \\
\midrule
Formas procesadas & 957 \\
Iteraciones hasta convergencia & 2 \\
Tolerancia objetivo & $10^{-4}$ \\
Cambio final ($\Delta$) & $7.33 \times 10^{-5}$ \\
Distancia promedio al consenso & 0.1307 \\
Tiempo de ejecución & $< 1$ segundo \\
\bottomrule
\end{tabular}
\end{table}

\begin{resultadoimportante}[title={Convergencia Excepcionalmente Rápida}]
La convergencia en solo 2 iteraciones (con $\Delta_{\text{final}} = 7.33 \times 10^{-5}$)
indica que la variabilidad de rotación en el dataset original es mínima.
Esto sugiere que:
\begin{enumerate}
    \item Los pacientes fueron posicionados consistentemente durante la adquisición
    \item La variabilidad principal era de escala y traslación, ya eliminada
    en el pre-procesamiento
    \item Las anotaciones de landmarks fueron realizadas con criterios geométricos
    consistentes
\end{enumerate}
\end{resultadoimportante}

\subsection{Forma Canónica Resultante}

La forma canónica calculada representa la anatomía torácica promedio
normalizada. Las coordenadas se presentan tanto en forma normalizada
(centrada, escala unitaria) como en píxeles para una imagen de 224×224.

\begin{table}[htbp]
\centering
\caption{Coordenadas de la forma canónica en espacio normalizado y píxeles}
\label{tab:canonical_coords_full}
\begin{tabular}{lcccl}
\toprule
\textbf{ID} & \textbf{X norm.} & \textbf{Y norm.} & \textbf{(X, Y) px} & \textbf{Estructura} \\
\midrule
L1 & 0.0010 & -0.2458 & (112.3, 33.7) & Mediastino superior \\
L2 & -0.0006 & 0.2466 & (111.8, 190.5) & Vértebra inferior \\
L3 & -0.2155 & -0.1251 & (43.4, 72.2) & Ápice pulmonar izq. \\
L4 & 0.2173 & -0.1228 & (181.2, 72.9) & Ápice pulmonar der. \\
L5 & -0.2485 & -0.0011 & (32.9, 111.6) & Hilio izquierdo \\
L6 & 0.2480 & 0.0011 & (191.0, 112.3) & Hilio derecho \\
L7 & -0.2676 & 0.1219 & (26.8, 150.8) & Base pulmonar izq. \\
L8 & 0.2657 & 0.1245 & (196.6, 151.6) & Base pulmonar der. \\
L9 & -0.0002 & -0.1232 & (111.9, 72.8) & Centro superior \\
L10 & -0.0004 & 0.0003 & (111.9, 112.1) & Centro medio \\
L11 & -0.0008 & 0.1232 & (111.7, 151.2) & Centro inferior \\
L12 & -0.1129 & -0.2470 & (76.1, 33.4) & Borde sup. izq. \\
L13 & 0.1184 & -0.2458 & (149.7, 33.7) & Borde sup. der. \\
L14 & -0.2833 & 0.2453 & (21.8, 190.1) & Ángulo costofrénico izq. \\
L15 & 0.2794 & 0.2481 & (201.0, 191.0) & Ángulo costofrénico der. \\
\bottomrule
\end{tabular}
\end{table}

\begin{observacion}[Simetría de la Forma Canónica]
La forma canónica exhibe alta simetría bilateral:
\begin{itemize}
    \item Los landmarks centrales (L9, L10, L11) tienen coordenada X $\approx 0$
    (desviación $< 0.001$), confirmando su posición sobre el eje medial
    \item Los pares bilaterales (L3-L4, L5-L6, L7-L8, L12-L13, L14-L15) tienen
    coordenadas X casi simétricas respecto al eje
    \item El eje central L1-L2 es casi perfectamente vertical
    ($X_{L1} - X_{L2} \approx 0.0016$, ángulo $\approx 0.2°$)
\end{itemize}
\end{observacion}

\subsection{Análisis de Componentes Principales}

\subsubsection{Varianza Explicada}

\begin{table}[htbp]
\centering
\caption{Descomposición de varianza por componentes principales}
\label{tab:pca_variance_full}
\begin{tabular}{lcccc}
\toprule
\textbf{Componente} & \textbf{Eigenvalor} & \textbf{Varianza (\%)} & \textbf{Acumulada (\%)} \\
\midrule
PC1 & $3.48 \times 10^{-3}$ & 69.57 & 69.57 \\
PC2 & $1.19 \times 10^{-3}$ & 23.86 & 93.43 \\
PC3 & $1.41 \times 10^{-4}$ & 2.82 & 96.25 \\
PC4 & $1.04 \times 10^{-4}$ & 2.08 & 98.33 \\
PC5 & $8.35 \times 10^{-5}$ & 1.67 & 100.00 \\
\bottomrule
\end{tabular}
\end{table}

\begin{resultadoimportante}[title={Baja Dimensionalidad Intrínseca}]
Solo 2 componentes principales capturan el 93.43\% de la varianza residual
post-GPA. Esto indica que:
\begin{enumerate}
    \item La variabilidad anatómica intrínseca es de muy baja dimensionalidad
    \item Las formas pueden representarse con muy pocos parámetros
    \item El espacio de formas válidas es altamente restringido
\end{enumerate}
La dimensionalidad efectiva del espacio de formas es $\approx 2$, lo cual
es extraordinariamente bajo para una configuración de 15 landmarks (30 DOF).
\end{resultadoimportante}

\subsubsection{Interpretación Anatómica de los Modos de Variación}

\begin{descripcion}[Primer Modo de Variación (PC1 - 69.57\%)]
El primer componente principal corresponde principalmente a variaciones en
la \textit{expansión/contracción bilateral} de la caja torácica:
\begin{itemize}
    \item $+2\sigma$: Tórax más ancho, landmarks laterales más separados
    \item $-2\sigma$: Tórax más estrecho, landmarks laterales más cercanos
\end{itemize}
Este modo captura diferencias en el tamaño relativo del tórax que no se
eliminan completamente con la normalización de escala global (que preserva
proporciones).
\end{descripcion}

\begin{descripcion}[Segundo Modo de Variación (PC2 - 23.86\%)]
El segundo componente corresponde a variaciones en la \textit{elongación vertical}:
\begin{itemize}
    \item $+2\sigma$: Tórax más alargado verticalmente
    \item $-2\sigma$: Tórax más achatado
\end{itemize}
Este modo captura diferencias en la relación de aspecto del tórax
(ancho vs. alto).
\end{descripcion}

% ==============================================================================
\section{Análisis Estadístico por Categoría Diagnóstica}
% ==============================================================================

Un objetivo central del análisis es determinar si existen diferencias
sistemáticas de forma entre las categorías diagnósticas (Normal, COVID,
Viral Pneumonia).

\subsection{Distribución de Muestras}

\begin{table}[htbp]
\centering
\caption{Distribución del dataset por categoría}
\label{tab:category_distribution}
\begin{tabular}{lcc}
\toprule
\textbf{Categoría} & \textbf{$n$} & \textbf{Proporción (\%)} \\
\midrule
Normal & 468 & 48.9 \\
COVID-19 & 306 & 32.0 \\
Viral Pneumonia & 183 & 19.1 \\
\midrule
\textbf{Total} & \textbf{957} & \textbf{100.0} \\
\bottomrule
\end{tabular}
\end{table}

\subsection{ANOVA sobre Componentes Principales}

Para evaluar diferencias de forma entre categorías, se realizó un ANOVA
unidireccional sobre las puntuaciones de los componentes principales.

\subsubsection{Análisis del Primer Componente (PC1)}

\begin{table}[htbp]
\centering
\caption{Estadísticas descriptivas de PC1 por categoría}
\label{tab:anova_pc1_descriptive}
\begin{tabular}{lcccc}
\toprule
\textbf{Categoría} & \textbf{$n$} & \textbf{Media} & \textbf{Desv. Est.} & \textbf{IC 95\%} \\
\midrule
Normal & 468 & -0.013 & 0.115 & $[-0.024, -0.003]$ \\
COVID-19 & 306 & 0.019 & 0.150 & $[0.002, 0.036]$ \\
Viral Pneumonia & 183 & 0.002 & 0.116 & $[-0.015, 0.019]$ \\
\bottomrule
\end{tabular}
\end{table}

\begin{table}[htbp]
\centering
\caption{Tabla ANOVA para PC1}
\label{tab:anova_pc1_table}
\begin{tabular}{lccccc}
\toprule
\textbf{Fuente} & \textbf{SS} & \textbf{df} & \textbf{MS} & \textbf{$F$} & \textbf{$p$-valor} \\
\midrule
Entre grupos & 0.1861 & 2 & 0.0931 & 5.72 & 0.003 \\
Dentro grupos & 15.5110 & 954 & 0.0163 & & \\
\midrule
Total & 15.6971 & 956 & & & \\
\bottomrule
\end{tabular}
\end{table}

\begin{resultadoimportante}[title={Diferencias Estadísticas Mínimas entre Categorías}]
El ANOVA sobre PC1 encuentra diferencias estadísticamente significativas entre
las tres categorías diagnósticas:
\begin{equation}
F(2, 954) = 5.72, \quad p = 0.003
\end{equation}
Sin embargo, el tamaño del efecto es muy pequeño ($\eta^2 = 1.2\%$), indicando
que las diferencias, aunque detectables estadísticamente con una muestra grande,
son prácticamente irrelevantes para el objetivo de warping geométrico.
\end{resultadoimportante}

\subsubsection{Tamaño del Efecto}

\begin{equation}
\eta^2 = \frac{SS_{\text{entre}}}{SS_{\text{total}}} = \frac{0.1861}{15.6971} = 0.0119 \approx 1.2\%
\end{equation}

El tamaño del efecto es pequeño ($\eta^2 \approx 0.01$), indicando que
las diferencias entre categorías explican apenas el 1.2\% de la variabilidad
total en la forma. Según la clasificación de Cohen, este efecto es considerado
``pequeño'' y prácticamente despreciable para propósitos de normalización
geométrica.

\subsection{Análisis de Potencia Estadística}

El tamaño muestral grande ($n = 957$) proporciona alta potencia estadística
para detectar diferencias pequeñas:

\begin{equation}
\text{Potencia} = 1 - \beta = P(\text{rechazar } H_0 | H_1 \text{ verdadera})
\end{equation}

Con $n_{\text{total}} = 957$, $k = 3$ grupos, y $\alpha = 0.05$:
\begin{itemize}
    \item Para detectar un efecto pequeño ($f = 0.10$): Potencia $> 0.99$
    \item Para detectar un efecto medio ($f = 0.25$): Potencia $> 0.99$
    \item Efecto mínimo detectable con potencia 0.80: $f \approx 0.06$
\end{itemize}

\begin{observacion}[Significancia vs. Relevancia Práctica]
La alta potencia del test con $n = 957$ permite detectar diferencias muy
pequeñas que, aunque estadísticamente significativas ($p = 0.003$), tienen
relevancia práctica mínima ($\eta^2 = 1.2\%$). Esto ilustra la diferencia
entre significancia estadística y significancia práctica: con muestras
suficientemente grandes, cualquier diferencia no nula se vuelve detectable.
\end{observacion}

\subsection{Implicaciones para el Proyecto}

\begin{hallazgo}[title={Las Formas son Prácticamente Equivalentes entre Categorías}]
Aunque el ANOVA detecta diferencias estadísticamente significativas
($p = 0.003$), el tamaño del efecto es muy pequeño ($\eta^2 = 1.2\%$),
indicando que las formas torácicas de pacientes Normal, COVID y Viral
Pneumonia son prácticamente equivalentes. Esto tiene importantes implicaciones:
\begin{enumerate}
    \item \textbf{Una única forma canónica es válida}: Las diferencias son
    tan pequeñas que no justifican calcular formas canónicas separadas
    \item \textbf{El warping no introduce sesgo significativo}: La normalización
    geométrica trata de manera esencialmente equivalente a todas las categorías
    \item \textbf{Las diferencias son de textura, no de forma}: Las patologías
    (consolidaciones, opacidades) afectan la apariencia pero no alteran
    significativamente la geometría del tórax a nivel de landmarks
    \item \textbf{Validación del enfoque}: El warping a una forma canónica
    común es metodológicamente válido dado el efecto prácticamente nulo
\end{enumerate}
\end{hallazgo}

% ==============================================================================
\section{Análisis de Variabilidad Pre y Post GPA}
% ==============================================================================

\subsection{Descomposición de Variabilidad}

La variabilidad total en el dataset original puede descomponerse en
componentes explicados por diferentes factores:

\begin{table}[htbp]
\centering
\caption{Descomposición de variabilidad en el dataset}
\label{tab:variability_decomposition}
\begin{tabular}{lcc}
\toprule
\textbf{Fuente de Variabilidad} & \textbf{Varianza Explicada (\%)} \\
\midrule
Traslación (posición del centroide) & 4.3 \\
Escala (tamaño del centroide) & $\sim$95.0 \\
Rotación & $< 0.5$ \\
\midrule
\textbf{Total eliminado por GPA} & \textbf{99.9} \\
\midrule
Variabilidad residual (forma pura) & 0.1 \\
\bottomrule
\end{tabular}
\end{table}

\subsection{Análisis de Escala Original}

\begin{table}[htbp]
\centering
\caption{Estadísticas de tamaño de centroide antes de normalización}
\label{tab:scale_analysis}
\begin{tabular}{lc}
\toprule
\textbf{Estadística} & \textbf{Valor} \\
\midrule
Media de $S(X)$ & 145.2 px \\
Desviación estándar & 16.9 px \\
Coeficiente de variación (CV) & 11.6\% \\
Mínimo & 98.7 px \\
Máximo & 294.1 px \\
Ratio máx/mín & 2.98 \\
\bottomrule
\end{tabular}
\end{table}

El coeficiente de variación de 11.6\% y el ratio de casi 3:1 entre el
tórax más grande y el más pequeño confirman que la escala es la principal
fuente de variabilidad que debe eliminarse.

\subsection{Análisis de Rotación Residual}

Después del centrado y escalado, pero antes de la alineación rotacional:

\begin{table}[htbp]
\centering
\caption{Distribución de rotaciones relativas}
\label{tab:rotation_analysis}
\begin{tabular}{lc}
\toprule
\textbf{Métrica} & \textbf{Valor} \\
\midrule
Rotación media & $0.21° \pm 4.2°$ \\
Porcentaje con $|\theta| > 5°$ & 14.9\% \\
Porcentaje con $|\theta| > 10°$ & 2.3\% \\
Rotación máxima & $18.7°$ \\
\bottomrule
\end{tabular}
\end{table}

La baja variabilidad rotacional explica la convergencia rápida del GPA.

% ==============================================================================
\section{Figuras Sugeridas}
% ==============================================================================

\subsection{Figura 10.1: Convergencia del GPA}

\begin{figuradescripcion}
\textbf{Título}: Convergencia del algoritmo GPA iterativo

\textbf{Contenido}: Gráfico de línea mostrando la evolución del cambio
en la forma consenso ($\Delta = \|\bar{X}^{(k)} - \bar{X}^{(k-1)}\|_F$)
versus el número de iteración.

\textbf{Ejes}:
\begin{itemize}
    \item Eje X: Número de iteración (0, 1, 2)
    \item Eje Y: Cambio $\Delta$ (escala logarítmica)
\end{itemize}

\textbf{Elementos visuales}:
\begin{itemize}
    \item Línea azul con marcadores circulares
    \item Línea horizontal punteada verde indicando tolerancia ($10^{-4}$)
    \item Anotación en iteración 2 mostrando "$\Delta = 7.33 \times 10^{-5}$"
\end{itemize}

\textbf{Mensaje}: Demostrar la convergencia excepcionalmente rápida (2 iteraciones).
\end{figuradescripcion}

\subsection{Figura 10.2: Forma Canónica con Landmarks}

\begin{figuradescripcion}
\textbf{Título}: Forma canónica de 15 landmarks anatómicos

\textbf{Contenido}: Visualización de los landmarks sobre un canvas de
224×224 píxeles, con conexiones anatómicas.

\textbf{Elementos visuales}:
\begin{itemize}
    \item Puntos azules de tamaño proporcional a la importancia del landmark
    \item Línea roja vertical conectando L1-L9-L10-L11-L2 (eje central)
    \item Líneas verdes punteadas conectando pares bilaterales
    \item Etiquetas con identificadores (L1-L15)
    \item Cuadrícula de fondo con divisiones cada 50 píxeles
\end{itemize}

\textbf{Anotaciones}:
\begin{itemize}
    \item Regiones anatómicas: "Mediastino", "Hilios", "Bases", "Ángulos costofrénicos"
\end{itemize}
\end{figuradescripcion}

\subsection{Figura 10.3: Modos de Variación PCA}

\begin{figuradescripcion}
\textbf{Título}: Modos de variación principal (PC1 y PC2) sobre forma canónica

\textbf{Contenido}: Matriz de 2×3 subplots mostrando:
\begin{itemize}
    \item Fila 1: PC1 con $-2\sigma$, $0$, $+2\sigma$
    \item Fila 2: PC2 con $-2\sigma$, $0$, $+2\sigma$
\end{itemize}

\textbf{Para cada subplot}:
\begin{itemize}
    \item Forma deformada como puntos conectados
    \item Vectores de desplazamiento (flechas) desde forma media
    \item Porcentaje de varianza explicada en título
\end{itemize}

\textbf{Mensaje}: Ilustrar que PC1 corresponde a expansión lateral y PC2
a elongación vertical.
\end{figuradescripcion}

\subsection{Figura 10.4: Distribución por Categoría en Espacio PCA}

\begin{figuradescripcion}
\textbf{Título}: Distribución de formas por categoría diagnóstica (PC1 vs PC2)

\textbf{Contenido}: Scatter plot bidimensional con puntos coloreados por categoría.

\textbf{Elementos visuales}:
\begin{itemize}
    \item Puntos verdes: Normal (n=468)
    \item Puntos rojos: COVID (n=306)
    \item Puntos azules: Viral Pneumonia (n=183)
    \item Elipses de confianza al 95\% para cada grupo
    \item Ejes centrados con líneas de referencia
\end{itemize}

\textbf{Anotaciones}:
\begin{itemize}
    \item Leyenda con $n$ por categoría
    \item Texto: "ANOVA: $F = 5.72$, $p = 0.003$, $\eta^2 = 1.2\%$"
\end{itemize}

\textbf{Mensaje}: Demostrar visualmente que las distribuciones se superponen
sustancialmente, confirmando que las diferencias de forma entre categorías
son mínimas a pesar de la significancia estadística.
\end{figuradescripcion}

\subsection{Figura 10.5: Superposición de Formas Alineadas}

\begin{figuradescripcion}
\textbf{Título}: Superposición de 100 formas alineadas por GPA

\textbf{Contenido}: Visualización de múltiples formas superpuestas para
mostrar la variabilidad residual.

\textbf{Elementos visuales}:
\begin{itemize}
    \item 100 formas aleatorias en gris claro (alpha=0.3)
    \item Forma canónica en rojo sólido
    \item Landmarks numerados en la forma canónica
\end{itemize}

\textbf{Mensaje}: Mostrar que la variabilidad residual post-GPA es mínima.
\end{figuradescripcion}

% ==============================================================================
\section{Archivos Fuente y Reproducibilidad}
% ==============================================================================

\begin{table}[htbp]
\centering
\caption{Archivos de implementación del análisis GPA}
\label{tab:source_files_detailed}
\begin{tabular}{p{5.5cm}p{7.5cm}}
\toprule
\textbf{Archivo} & \textbf{Contenido} \\
\midrule
\archivo{scripts/gpa\_analysis.py} & Implementación completa del GPA:\\
& \quad - \texttt{center\_shape()}: Centrado de forma\\
& \quad - \texttt{scale\_shape()}: Normalización de escala\\
& \quad - \texttt{optimal\_rotation\_matrix()}: Rotación óptima via SVD\\
& \quad - \texttt{gpa\_iterative()}: Algoritmo GPA completo\\
& \quad - \texttt{pca\_on\_aligned\_shapes()}: PCA post-GPA\\
\midrule
\archivo{scripts/verify\_canonical\_delaunay.py} & Verificación de triangulación sobre forma canónica \\
\midrule
\archivo{outputs/shape\_analysis/\\canonical\_shape\_gpa.json} & Forma canónica resultante:\\
& \quad - Coordenadas normalizadas\\
& \quad - Coordenadas en píxeles (224×224)\\
& \quad - Información de convergencia\\
\midrule
\archivo{outputs/shape\_analysis/\\aligned\_shapes.npz} & Formas alineadas (957, 15, 2) + metadatos \\
\midrule
\archivo{outputs/shape\_analysis/figures/} & Visualizaciones generadas:\\
& \quad - \texttt{gpa\_convergence.png}\\
& \quad - \texttt{canonical\_shape.png}\\
& \quad - \texttt{pca\_modes\_variation.png}\\
& \quad - \texttt{pca\_category\_scatter.png}\\
\bottomrule
\end{tabular}
\end{table}

% ==============================================================================
\section{Discusión}
% ==============================================================================

\subsection{Sobre la Convergencia Rápida}

La convergencia del GPA en solo 2 iteraciones es notable y tiene varias
implicaciones:

\begin{enumerate}
    \item \textbf{Consistencia del dataset}: Las radiografías fueron
    adquiridas con protocolos estandarizados que minimizan la variabilidad
    de posicionamiento

    \item \textbf{Calidad de las anotaciones}: Los landmarks fueron marcados
    siguiendo criterios geométricos consistentes, resultando en formas
    con variabilidad rotacional mínima

    \item \textbf{Dominio bien definido}: La anatomía torácica tiene
    restricciones geométricas fuertes (simetría bilateral, proporciones
    consistentes) que limitan la variabilidad de forma
\end{enumerate}

\subsection{Sobre la Ausencia de Diferencias entre Categorías}

El resultado de que las formas son estadísticamente indistinguibles entre
categorías diagnósticas tiene profundas implicaciones metodológicas:

\begin{enumerate}
    \item \textbf{Validación del enfoque de warping}: La normalización
    geométrica a una forma canónica común no introduce sesgo hacia ninguna
    categoría, lo cual es crítico para la validez del clasificador

    \item \textbf{Naturaleza de las patologías}: Las neumonías (viral y
    COVID-19) afectan primariamente la textura pulmonar (consolidaciones,
    opacidades, infiltrados) pero no alteran detectablemente la geometría
    macroscópica del tórax a nivel de landmarks

    \item \textbf{Robustez del método}: El warping puede aplicarse
    uniformemente a todas las categorías sin ajustes específicos
\end{enumerate}

\subsection{Limitaciones}

\begin{enumerate}
    \item \textbf{Dataset específico}: Los resultados son válidos para el
    dataset COVID-19 Radiography; la generalización a otros datasets
    requiere verificación

    \item \textbf{Resolución de landmarks}: Con 15 landmarks, solo se captura
    geometría macroscópica; patologías que afecten estructuras locales
    podrían no reflejarse

    \item \textbf{Poblacion homogénea}: El dataset puede no incluir casos
    con deformidades torácicas severas que sí alterarían la forma
\end{enumerate}

% ==============================================================================
\section{Conclusiones}
% ==============================================================================

\begin{enumerate}
    \item \textbf{GPA converge excepcionalmente rápido}: Solo 2 iteraciones
    son necesarias para alcanzar la forma canónica con tolerancia $10^{-4}$,
    indicando baja variabilidad rotacional en el dataset original.

    \item \textbf{Variabilidad residual es de muy baja dimensionalidad}:
    Solo 2 componentes principales capturan el 93.4\% de la varianza
    post-alineación, indicando que el espacio de formas válidas es
    altamente restringido.

    \item \textbf{Las formas son prácticamente equivalentes entre categorías}:
    Aunque existen diferencias estadísticamente significativas ($F = 5.72$,
    $p = 0.003$), el tamaño del efecto es despreciable ($\eta^2 = 1.2\%$).

    \item \textbf{Una forma canónica universal es válida}: El hallazgo
    anterior justifica el uso de una única forma canónica para el warping
    de todas las categorías diagnósticas.

    \item \textbf{GPA es suficiente para nuestro propósito}: No se requiere
    la complejidad adicional de Active Shape Models dado que GPA elimina
    el 99.9\% de la variabilidad extrínseca.

    \item \textbf{El método es computacionalmente eficiente}: El tiempo
    de ejecución es inferior a 1 segundo para 957 formas, permitiendo
    su uso en pipelines de producción.
\end{enumerate}

% ==============================================================================
% Referencias sugeridas (no incluidas en este documento standalone)
% ==============================================================================

% \begin{thebibliography}{9}
% \bibitem{cootes1995} Cootes, T. F., Taylor, C. J., Cooper, D. H., & Graham, J. (1995).
% Active shape models-their training and application. Computer Vision and Image Understanding, 61(1), 38-59.
% \bibitem{dryden2016} Dryden, I. L., & Mardia, K. V. (2016). Statistical shape analysis:
% with applications in R (Vol. 995). John Wiley & Sons.
% \bibitem{gower1975} Gower, J. C. (1975). Generalized procrustes analysis. Psychometrika, 40(1), 33-51.
% \end{thebibliography}

\end{document}
