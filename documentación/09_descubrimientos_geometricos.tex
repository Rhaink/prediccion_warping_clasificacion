% ==============================================================================
% DOCUMENTACIÓN CIENTÍFICA - DESCUBRIMIENTOS GEOMÉTRICOS DEL ETIQUETADO
% Proyecto: Detección de COVID-19 mediante Landmarks Anatómicos
% Sesiones cubiertas: 19
% ==============================================================================

\documentclass[12pt,a4paper]{article}
% ==============================================================================
% PREÁMBULO LATEX PARA DOCUMENTACIÓN CIENTÍFICA DE NIVEL DOCTORAL
% Proyecto: Detección de COVID-19 mediante Landmarks Anatómicos y Warping Geométrico
% ==============================================================================

% --- Codificación y idioma ---
\usepackage[utf8]{inputenc}
\usepackage[T1]{fontenc}
\usepackage[spanish,es-tabla]{babel}

% --- Matemáticas ---
\usepackage{amsmath,amssymb,amsfonts,amsthm}
\usepackage{mathtools}
\usepackage{bm}  % Negritas en matemáticas

% --- Tablas profesionales ---
\usepackage{booktabs}
\usepackage{multirow}
\usepackage{array}
\usepackage{longtable}
\usepackage{tabularx}

% --- Figuras y gráficos ---
\usepackage{graphicx}
\usepackage{subfig}
\usepackage{float}
\usepackage{caption}

% --- Algoritmos y pseudocódigo ---
\usepackage{algorithm}
\usepackage{algorithmic}

% --- Colores (debe cargarse antes de listings) ---
\usepackage{xcolor}
\definecolor{covidred}{RGB}{220,53,69}
\definecolor{normalgreen}{RGB}{40,167,69}
\definecolor{viralblue}{RGB}{0,123,255}
\definecolor{codeblue}{RGB}{0,0,180}
\definecolor{codegray}{RGB}{128,128,128}

% --- Código fuente ---
\usepackage{listings}
\lstset{
    language=Python,
    basicstyle=\ttfamily\small,
    keywordstyle=\color{blue}\bfseries,
    commentstyle=\color{gray}\itshape,
    stringstyle=\color{red},
    numbers=left,
    numberstyle=\tiny\color{gray},
    stepnumber=1,
    numbersep=5pt,
    backgroundcolor=\color{white},
    frame=single,
    rulecolor=\color{black},
    tabsize=4,
    captionpos=b,
    breaklines=true,
    breakatwhitespace=false,
    showspaces=false,
    showstringspaces=false,
    showtabs=false,
    literate={á}{{\'a}}1 {é}{{\'e}}1 {í}{{\'i}}1 {ó}{{\'o}}1 {ú}{{\'u}}1
             {Á}{{\'A}}1 {É}{{\'E}}1 {Í}{{\'I}}1 {Ó}{{\'O}}1 {Ú}{{\'U}}1
             {ñ}{{\~n}}1 {Ñ}{{\~N}}1
             {ü}{{\"u}}1 {Ü}{{\"U}}1
             {¿}{{?`}}1 {¡}{{!`}}1
}

% --- Referencias y enlaces ---
\usepackage{hyperref}
\hypersetup{
    colorlinks=true,
    linkcolor=blue,
    filecolor=magenta,
    urlcolor=cyan,
    citecolor=blue,
    pdftitle={Documentación del Proyecto de Tesis},
    pdfauthor={},
}
\usepackage{cleveref}

% --- Bibliografía ---
\usepackage{natbib}
\bibliographystyle{plainnat}

% --- Geometría de página ---
\usepackage[a4paper,margin=2.5cm]{geometry}

% --- Espaciado ---
\usepackage{setspace}
\onehalfspacing

% --- Encabezados y pies ---
\usepackage{fancyhdr}
\pagestyle{fancy}
\fancyhf{}
\rhead{\rightmark}
\lhead{\leftmark}
\cfoot{\thepage}

% --- Entornos personalizados ---
\theoremstyle{definition}
\newtheorem{definicion}{Definición}[section]
\newtheorem{proposicion}{Proposición}[section]
\newtheorem{teorema}{Teorema}[section]
\newtheorem{lema}{Lema}[section]
\newtheorem{corolario}{Corolario}[section]

\theoremstyle{remark}
\newtheorem{observacion}{Observación}[section]
\newtheorem{nota}{Nota}[section]
\newtheorem{hipotesis}{Hipótesis}[section]

% --- Comandos personalizados ---
% Vectores y matrices
\newcommand{\vect}[1]{\mathbf{#1}}
\newcommand{\mat}[1]{\mathbf{#1}}

% Operadores
\DeclareMathOperator*{\argmin}{arg\,min}
\DeclareMathOperator*{\argmax}{arg\,max}
\DeclareMathOperator{\sgn}{sgn}
\DeclareMathOperator{\diag}{diag}
\DeclareMathOperator{\trace}{tr}

% Normas y productos
\newcommand{\norm}[1]{\left\|#1\right\|}
\newcommand{\abs}[1]{\left|#1\right|}
\newcommand{\inner}[2]{\langle #1, #2 \rangle}

% Conjuntos
\newcommand{\R}{\mathbb{R}}
\newcommand{\N}{\mathbb{N}}
\newcommand{\Z}{\mathbb{Z}}

% Espacios de imágenes y landmarks
\newcommand{\imgspace}{\mathcal{I}}
\newcommand{\landmarkspace}{\mathcal{L}}
\newcommand{\classspace}{\mathcal{Y}}

% Landmarks específicos
\newcommand{\landmark}[1]{L_{#1}}
\newcommand{\landmarkpair}[2]{(L_{#1}, L_{#2})}

% Funciones de pérdida
\newcommand{\loss}{\mathcal{L}}
\newcommand{\wingloss}{\mathcal{L}_{\text{wing}}}
\newcommand{\mseloss}{\mathcal{L}_{\text{MSE}}}

% Métricas
\newcommand{\accuracy}{\text{Acc}}
\newcommand{\precision}{\text{Prec}}
\newcommand{\recall}{\text{Rec}}
\newcommand{\fscore}{F_1}

% Abreviaciones
\newcommand{\ie}{\textit{i.e.}}
\newcommand{\eg}{\textit{e.g.}}
\newcommand{\etal}{\textit{et al.}}
\newcommand{\etc}{\textit{etc.}}

% Referencias a archivos del proyecto
\newcommand{\archivo}[1]{\texttt{#1}}
\newcommand{\funcion}[1]{\texttt{#1()}}
\newcommand{\clase}[1]{\texttt{#1}}
\newcommand{\parametro}[1]{\texttt{#1}}

% Unidades
\newcommand{\px}{\,\text{px}}
\newcommand{\epoch}{\,\text{época}}
\newcommand{\epochs}{\,\text{épocas}}

% --- Información del documento ---
\newcommand{\proyectotitulo}{Detección de COVID-19 en Radiografías de Tórax mediante Landmarks Anatómicos y Normalización Geométrica}
\newcommand{\proyectosubtitulo}{Documentación Científica del Proceso de Desarrollo}

% --- Formato de secciones ---
\usepackage{titlesec}
\titleformat{\section}
  {\normalfont\Large\bfseries}{\thesection}{1em}{}
\titleformat{\subsection}
  {\normalfont\large\bfseries}{\thesubsection}{1em}{}
\titleformat{\subsubsection}
  {\normalfont\normalsize\bfseries}{\thesubsubsection}{1em}{}

% --- Notas al margen para figuras sugeridas ---
\usepackage{marginnote}
\newcommand{\figurasugerida}[1]{\marginnote{\scriptsize\textcolor{blue}{[Figura: #1]}}}

% --- Cajas para resultados importantes ---
\usepackage{tcolorbox}
\newtcolorbox{resultadoimportante}[1][]{
    colback=green!5!white,
    colframe=green!75!black,
    fonttitle=\bfseries,
    title=Resultado Importante,
    #1
}

\newtcolorbox{hallazgo}[1][]{
    colback=blue!5!white,
    colframe=blue!75!black,
    fonttitle=\bfseries,
    title=Hallazgo,
    #1
}

\newtcolorbox{metodologia}[1][]{
    colback=yellow!5!white,
    colframe=yellow!75!black,
    fonttitle=\bfseries,
    title=Metodología,
    #1
}

\newtcolorbox{figuradescripcion}[1][]{
    colback=gray!5!white,
    colframe=gray!75!black,
    fonttitle=\bfseries,
    title=Descripción de Figura,
    #1
}

% ==============================================================================
% FIN DEL PREÁMBULO
% ==============================================================================


\title{Descubrimientos Geométricos de la\\Estructura del Etiquetado Manual}
\author{Documentación del Proceso de Desarrollo}
\date{Sesión: 19}

\begin{document}
\maketitle

\begin{abstract}
Este documento presenta los descubrimientos geométricos críticos realizados
al analizar la estructura del etiquetado manual de 957 radiografías. Se
descubrió que el proceso de anotación crea una estructura paramétrica
casi perfecta: L9, L10 y L11 dividen el eje central L1-L2 en \textbf{exactamente}
4 partes iguales ($t = 0.25, 0.50, 0.75$) con un error de solo 1.3 píxeles.
También se identificó que el ground truth presenta asimetría natural de
5.5-7.9 píxeles entre pares bilaterales, estableciendo un error mínimo
teórico de 5-6 píxeles. Estos descubrimientos fundamentaron el diseño
de funciones de pérdida geométricas y la arquitectura jerárquica, y
explican por qué el modelo directo aprende estas restricciones implícitamente.
\end{abstract}

\tableofcontents
\newpage

% ==============================================================================
\section{Introducción y Metodología de Análisis}
% ==============================================================================

\subsection{Motivación del Análisis}

El análisis geométrico del ground truth se realizó con dos objetivos:

\begin{enumerate}
    \item Comprender la estructura implícita del proceso de etiquetado manual
    \item Identificar restricciones que pudieran explotarse para mejorar
    las predicciones del modelo
\end{enumerate}

\subsection{Metodología}

Se analizaron las 957 muestras del dataset completo, extrayendo para cada
imagen:

\begin{itemize}
    \item Coordenadas de los 15 landmarks en píxeles
    \item Geometría del eje central (L1-L2)
    \item Parámetros relativos ($t$, $d$) de cada landmark
    \item Métricas de simetría y alineación
\end{itemize}

% ==============================================================================
\section{Estructura del Eje Central}
% ==============================================================================

\subsection{Definición del Eje}

El eje central se define por los landmarks L1 (ápice superior) y L2
(ápice inferior) de la columna vertebral torácica:

\begin{equation}
\text{Eje} = \overrightarrow{L_1 L_2} = L_2 - L_1
\label{eq:axis_def}
\end{equation}

\subsection{Estadísticas del Eje}

\begin{table}[htbp]
\centering
\caption{Estadísticas del eje central L1-L2}
\label{tab:axis_stats}
\begin{tabular}{lcc}
\toprule
\textbf{Métrica} & \textbf{Valor} & \textbf{Interpretación} \\
\midrule
Longitud promedio & $198.2 \pm 32.8$ px & Alta variabilidad \\
Ángulo con vertical & $-0.21° \pm 4.00°$ & \textbf{Casi perfecto} \\
\bottomrule
\end{tabular}
\end{table}

\begin{hallazgo}[title={El eje es prácticamente vertical}]
El ángulo medio del eje con la vertical es de solo $-0.21°$, con desviación
estándar de $4°$. Esto significa que el proceso de anotación tiende a
alinear L1 y L2 casi perfectamente en vertical, probablemente guiado por
la anatomía de la columna vertebral.
\end{hallazgo}

% ==============================================================================
\section{Estructura Paramétrica del Etiquetado}
% ==============================================================================

\subsection{Parametrización de Landmarks}

Cada landmark puede representarse con dos parámetros relativos al eje L1-L2:

\begin{definicion}[Parámetro $t$ (Posición en el Eje)]
\begin{equation}
t = \frac{(P - L_1) \cdot (L_2 - L_1)}{\|L_2 - L_1\|^2}
\label{eq:t_param}
\end{equation}
donde $t = 0$ corresponde a $L_1$ y $t = 1$ a $L_2$.
\end{definicion}

\begin{definicion}[Parámetro $d$ (Distancia Perpendicular)]
\begin{equation}
d = (P - L_1) \cdot \hat{u}^\perp
\label{eq:d_param}
\end{equation}
donde $\hat{u}^\perp$ es el vector unitario perpendicular al eje.
Valores negativos indican posición a la izquierda, positivos a la derecha.
\end{definicion}

\subsection{Landmarks Centrales: División Exacta del Eje}

\begin{table}[htbp]
\centering
\caption{Posición de landmarks centrales sobre el eje}
\label{tab:central_landmarks}
\begin{tabular}{lcccc}
\toprule
\textbf{Landmark} & \textbf{$t$ Observado} & \textbf{$t$ Teórico} & \textbf{Error $t$} & \textbf{Dist. al eje} \\
\midrule
L9 & $0.249 \pm 0.010$ & 0.25 & 0.001 & $1.37 \pm 1.13$ px \\
L10 & $0.500 \pm 0.010$ & 0.50 & 0.000 & $1.30 \pm 1.13$ px \\
L11 & $0.749 \pm 0.010$ & 0.75 & 0.001 & $1.35 \pm 1.11$ px \\
\bottomrule
\end{tabular}
\end{table}

\begin{resultadoimportante}[title={División exacta en 4 partes}]
Los landmarks centrales L9, L10, L11 dividen el eje L1-L2 en
\textbf{exactamente 4 partes iguales}:
\begin{itemize}
    \item L9 en $t = 0.249$ (teórico: 0.25) $\to$ Error: 0.4\%
    \item L10 en $t = 0.500$ (teórico: 0.50) $\to$ Error: 0.0\%
    \item L11 en $t = 0.749$ (teórico: 0.75) $\to$ Error: 0.1\%
\end{itemize}
La distancia perpendicular al eje es de solo $\sim$1.3 px, indicando
que estos puntos están \textbf{casi perfectamente alineados} con el eje.
\end{resultadoimportante}

\subsection{Representación Paramétrica Completa}

\begin{table}[htbp]
\centering
\caption{Parámetros $(t, d)$ de todos los landmarks}
\label{tab:all_params}
\begin{tabular}{lcrl}
\toprule
\textbf{Landmark} & \textbf{$t$} & \textbf{$d$ (px)} & \textbf{Ubicación Anatómica} \\
\midrule
L1 & 0.000 & 0.0 & Origen del eje (ápice superior) \\
L2 & 1.000 & 0.0 & Fin del eje (ápice inferior) \\
\midrule
L3 & 0.247 & $-86.2$ & Ápice pulmonar izquierdo \\
L4 & 0.248 & $+86.5$ & Ápice pulmonar derecho \\
\midrule
L5 & 0.499 & $-99.2$ & Hilio izquierdo \\
L6 & 0.499 & $+98.8$ & Hilio derecho \\
\midrule
L7 & 0.748 & $-106.6$ & Base pulmonar izquierda \\
L8 & 0.749 & $+106.0$ & Base pulmonar derecha \\
\midrule
L9 & 0.249 & $-0.3$ & Centro superior (1/4 eje) \\
L10 & 0.500 & $-0.2$ & Centro medio (1/2 eje) \\
L11 & 0.749 & $-0.3$ & Centro inferior (3/4 eje) \\
\midrule
L12 & $-0.001$ & $-45.4$ & Borde superior izquierdo \\
L13 & $-0.001$ & $+46.8$ & Borde superior derecho \\
\midrule
L14 & 0.998 & $-112.7$ & Ángulo costofrénico izquierdo \\
L15 & 0.999 & $+111.6$ & Ángulo costofrénico derecho \\
\bottomrule
\end{tabular}
\end{table}

\subsection{Proporcionalidad de Distancias}

Las distancias perpendiculares $d$ son proporcionales a la longitud del eje:

\begin{table}[htbp]
\centering
\caption{Ratio $|d|/\text{longitud\_eje}$ por par bilateral}
\label{tab:d_ratio}
\begin{tabular}{lcc}
\toprule
\textbf{Par} & \textbf{Ratio} & \textbf{Correlación con longitud} \\
\midrule
L3-L4 (ápices) & $0.444 \pm 0.076$ & $r = 0.41$ \\
L5-L6 (hilios) & $0.511 \pm 0.084$ & $r = 0.42$ \\
L7-L8 (bases) & $0.549 \pm 0.091$ & $r = 0.43$ \\
L14-L15 (costof.) & $0.581 \pm 0.098$ & $r = 0.44$ \\
\bottomrule
\end{tabular}
\end{table}

\begin{observacion}
La correlación significativa ($p < 10^{-39}$) indica que radiografías
con ejes más largos tienden a tener landmarks bilaterales más separados,
preservando proporciones anatómicas.
\end{observacion}

% ==============================================================================
\section{Error Irreducible en Anotaciones}
% ==============================================================================

\subsection{Fuentes de Error Identificadas}

\subsubsection{Ruido Base del Etiquetado}

El ruido fundamental del proceso de anotación se estima en $\sim$1.3-1.5 px:

\begin{itemize}
    \item \textbf{Evidencia}: L9, L10, L11 están a $\sim$1.3 px del eje perfecto,
    cuando deberían estar exactamente sobre él
    \item \textbf{Causa}: Movimiento discreto con teclado, imprecisión del cursor,
    fatiga del anotador
\end{itemize}

\subsubsection{Asimetría Natural del Ground Truth}

\begin{table}[htbp]
\centering
\caption{Asimetría entre pares bilaterales en el ground truth}
\label{tab:asymmetry}
\begin{tabular}{lcc}
\toprule
\textbf{Par} & \textbf{Asimetría Media (px)} & \textbf{Desv. Est. (px)} \\
\midrule
L3-L4 & 5.51 & 4.58 \\
L5-L6 & 5.55 & 5.20 \\
L7-L8 & 6.82 & 5.85 \\
L14-L15 & 7.89 & 6.84 \\
\bottomrule
\end{tabular}
\end{table}

\begin{hallazgo}[title={El ground truth NO es simétrico}]
El ground truth presenta asimetría natural de 5.5-7.9 píxeles entre pares
bilaterales. Esta asimetría puede ser:
\begin{itemize}
    \item Asimetría anatómica real del paciente
    \item Variabilidad del proceso de anotación
    \item Combinación de ambas
\end{itemize}
\textbf{Implicación crítica}: Forzar simetría perfecta introduce error.
\end{hallazgo}

\subsection{Error Mínimo Teórico}

\begin{table}[htbp]
\centering
\caption{Estimación del error mínimo alcanzable}
\label{tab:min_error}
\begin{tabular}{lll}
\toprule
\textbf{Escenario} & \textbf{Error Esperado} & \textbf{Descripción} \\
\midrule
Perfecto & 1.5-2 px & Aprendizaje ideal del ruido \\
Excelente & 3-4 px & Mejor resultado realista \\
Muy bueno & 4-6 px & Objetivo alcanzable \\
Resultado obtenido & 3.71 px & Ensemble final \\
\bottomrule
\end{tabular}
\end{table}

% ==============================================================================
\section{Implicaciones para el Modelado}
% ==============================================================================

\subsection{Restricciones NO Forzables}

\begin{observacion}[Simetría perfecta perjudica]
Evaluación de Soft Symmetry Loss en el ground truth:
\begin{itemize}
    \item Margen 0 px: Loss = 72.31 (GT penalizado severamente)
    \item Margen 5 px: Loss = 28.96
    \item Margen 8 px: Loss = 16.51 (GT casi sin penalización)
\end{itemize}
\textbf{Conclusión}: Usar margen de 5-8 px para permitir asimetría natural.
\end{observacion}

\subsection{Restricciones Explotables}

\begin{enumerate}
    \item \textbf{Alineación central}: L9, L10, L11 deben estar sobre el eje.
    Error actual en GT: 1.34 $\pm$ 0.87 px

    \item \textbf{División del eje}: L9, L10, L11 en $t = 0.25, 0.50, 0.75$
    exactamente

    \item \textbf{Proporcionalidad}: $|d|$ proporcional a longitud del eje
\end{enumerate}

\subsection{Funciones de Pérdida Propuestas}

\subsubsection{Central Alignment Loss}

\begin{equation}
\mathcal{L}_{central} = \frac{1}{3} \sum_{i \in \{9, 10, 11\}}
\text{dist}(L_i, \overrightarrow{L_1 L_2})
\label{eq:central_loss}
\end{equation}

donde $\text{dist}(P, \ell)$ es la distancia perpendicular del punto $P$
a la línea $\ell$.

\subsubsection{Soft Symmetry Loss}

\begin{equation}
\mathcal{L}_{sym} = \frac{1}{5} \sum_{(i,j) \in \mathcal{P}}
\max(0, |d_i| - |d_j| - \text{margin})^2
\label{eq:soft_sym_loss}
\end{equation}

donde $\mathcal{P} = \{(3,4), (5,6), (7,8), (12,13), (14,15)\}$ son los
pares bilaterales y $\text{margin} = 6$ px.

% ==============================================================================
\section{Variabilidad por Categoría}
% ==============================================================================

\begin{table}[htbp]
\centering
\caption{Variabilidad de landmarks por categoría diagnóstica}
\label{tab:category_variability}
\begin{tabular}{lccc}
\toprule
\textbf{Categoría} & \textbf{Muestras} & \textbf{Variabilidad ($\sigma$)} & \textbf{Outliers} \\
\midrule
COVID & 306 (32\%) & 20.1 px & 10.5\% \\
Normal & 468 (49\%) & 17.0 px & 4.5\% \\
Viral & 183 (19\%) & 12.5 px & 1.1\% \\
\bottomrule
\end{tabular}
\end{table}

\begin{observacion}
Las imágenes de neumonía viral tienen significativamente menor variabilidad
(12.5 px vs 17-20 px). Esto implica que un modelo único debe
``comprometerse'' entre precisiones diferentes por categoría.
\end{observacion}

\subsection{Muestras Problemáticas}

El 20.7\% de las muestras tienen geometría inconsistente (asimetría $>$ 8 px
en algún par). Las 5 muestras más problemáticas:

\begin{table}[htbp]
\centering
\caption{Muestras con mayor asimetría}
\label{tab:outliers}
\begin{tabular}{lccc}
\toprule
\textbf{ID} & \textbf{Categoría} & \textbf{Asimetría Máx (px)} \\
\midrule
601 & Normal-8317 & 39.2 \\
369 & COVID-1258 & 33.9 \\
373 & COVID-1558 & 32.9 \\
379 & COVID-2281 & 25.7 \\
285 & COVID-2933 & 24.1 \\
\bottomrule
\end{tabular}
\end{table}

% ==============================================================================
\section{Correlaciones Descubiertas}
% ==============================================================================

\subsection{Predicibilidad de Coordenadas}

\begin{hallazgo}[title={Coordenadas Y son altamente predecibles}]
Las coordenadas Y de todos los landmarks tienen correlación $R^2 > 0.98$
con las coordenadas de L1 y L2:
\begin{itemize}
    \item $L_{12,Y} \approx L_{1,Y}$ (correlación 0.98)
    \item $L_{14,Y} \approx L_{2,Y}$ (correlación 0.97)
    \item $L_{9,X} \approx L_{1,X}$ (correlación 0.96)
\end{itemize}
Esto sugiere que predecir correctamente L1 y L2 condiciona fuertemente
el resto de landmarks.
\end{hallazgo}

\subsection{Orden de Dificultad}

Los landmarks ordenados por variabilidad (más fácil a más difícil):

\begin{enumerate}
    \item L9, L10 ($\sigma \sim 20$ px) --- \textbf{Más fáciles}
    \item L1, L4, L13, L3, L12 ($\sigma \sim 22$-24 px)
    \item L6, L5, L11 ($\sigma \sim 24$-25 px)
    \item L8, L7 ($\sigma \sim 28$-30 px)
    \item L2, L15, L14 ($\sigma \sim 32$-35 px) --- \textbf{Más difíciles}
\end{enumerate}

% ==============================================================================
\section{Figuras Sugeridas}
% ==============================================================================

\subsection{Figura 9.1: Estructura Paramétrica del Etiquetado}
\textit{Descripción}: Diagrama anatómico mostrando:
\begin{itemize}
    \item Eje L1-L2 con parámetro $t$ marcado en 0.25, 0.50, 0.75
    \item L9, L10, L11 sobre el eje
    \item Flechas mostrando distancias perpendiculares $d$ para pares bilaterales
    \item Proporciones anotadas
\end{itemize}

\subsection{Figura 9.2: Distribución de Posición $t$}
\textit{Descripción}: Histograma mostrando la distribución de $t$ para
L9, L10, L11, con líneas verticales en 0.25, 0.50, 0.75 teóricos.
Demostrar la precisión de la división en 4 partes.

\subsection{Figura 9.3: Asimetría en Ground Truth}
\textit{Descripción}: Boxplot mostrando distribución de asimetría por par
bilateral (L3-L4, L5-L6, etc.). Incluir línea horizontal en 6 px (margen
recomendado).

\subsection{Figura 9.4: Variabilidad por Categoría}
\textit{Descripción}: Gráfico de violín o boxplot comparando la distribución
de variabilidad ($\sigma$) por categoría (COVID, Normal, Viral).

\subsection{Figura 9.5: Orden de Dificultad de Landmarks}
\textit{Descripción}: Gráfico de barras horizontales mostrando variabilidad
por landmark, ordenados de menor a mayor $\sigma$.

% ==============================================================================
\section{Archivos Fuente}
% ==============================================================================

\begin{table}[htbp]
\centering
\caption{Archivos de análisis geométrico}
\label{tab:source_files}
\begin{tabular}{ll}
\toprule
\textbf{Archivo} & \textbf{Contenido} \\
\midrule
\archivo{DESCUBRIMIENTOS\_GEOMETRICOS.md} & Documento completo de hallazgos \\
\archivo{scripts/gpa\_analysis.py} & Análisis de Procrustes \\
\archivo{scripts/analyze\_data.py} & Estadísticas del dataset \\
\bottomrule
\end{tabular}
\end{table}

% ==============================================================================
\section{Conclusiones}
% ==============================================================================

\begin{enumerate}
    \item \textbf{Estructura paramétrica exacta}: El etiquetado manual crea
    una estructura geométrica casi perfecta, con L9, L10, L11 dividiendo
    el eje en exactamente 4 partes.

    \item \textbf{Error mínimo de 1.3-1.5 px}: El ruido base del etiquetado
    establece un límite inferior práctico para cualquier modelo.

    \item \textbf{No forzar simetría perfecta}: El GT tiene asimetría natural
    de 5.5-7.9 px; forzar simetría introduce error.

    \item \textbf{Proporcionalidad preservada}: Las distancias perpendiculares
    son proporcionales a la longitud del eje.

    \item \textbf{COVID es más variable}: Mayor variabilidad en COVID
    (20.1 px) vs Normal (17.0 px) vs Viral (12.5 px).

    \item \textbf{Predicibilidad jerárquica}: Coordenadas Y altamente
    predecibles desde L1, L2 ($R^2 > 0.98$).

    \item \textbf{Error obtenido: 3.71 px}: Resultado excelente cercano al
    límite teórico, considerando ruido de etiquetado y asimetría natural.

    \item \textbf{Fundamento para arquitectura jerárquica}: Aunque el modelo
    directo resultó mejor, estos descubrimientos explican por qué las
    restricciones geométricas son aprendidas implícitamente.
\end{enumerate}

\end{document}
