% =============================================================================
% CAPITULO 7: CONCLUSIONES
% =============================================================================

\chapter{Conclusiones}
\label{ch:conclusiones}

Este capitulo final presenta las conclusiones del trabajo, resumiendo las contribuciones realizadas, evaluando el cumplimiento de los objetivos planteados, y ofreciendo reflexiones sobre el proceso de investigacion y desarrollo.

% -----------------------------------------------------------------------------
\section{Resumen de Contribuciones}
\label{sec:resumen_contribuciones}
% -----------------------------------------------------------------------------

Esta tesis presenta un sistema de deep learning para la deteccion automatica de 15 landmarks anatomicos en radiografias de torax. Las principales contribuciones son:

\subsection{Contribuciones Tecnicas}

\begin{enumerate}
    \item \textbf{Sistema de alta precision}: Se desarrollo un sistema que alcanza un error promedio de \errorpx{3.71}, superando ampliamente el objetivo inicial de $<$8 pixeles y representando una mejora del 59\% sobre el baseline.

    \item \textbf{Demostracion de efectividad de CLAHE}: Se demostro por primera vez que el preprocesamiento con CLAHE (clip\_limit=2.0, tile\_size=4) mejora significativamente la deteccion de landmarks en radiografias toracicas, especialmente en casos de COVID-19 donde la mejora fue del 68\%.

    \item \textbf{Arquitectura optimizada}: Se propuso una arquitectura basada en ResNet-18 con Coordinate Attention y una cabeza de regresion profunda (hidden\_dim=768) que balancea precision y eficiencia computacional.

    \item \textbf{Metodologia de ensemble selectivo}: Se demostro que excluir modelos de bajo rendimiento del ensemble es mas efectivo que usar esquemas de ponderacion, proporcionando una guia practica para futuros trabajos.

    \item \textbf{Analisis geometrico del etiquetado}: Se identifico la estructura parametrica del proceso de etiquetado manual, revelando que los landmarks centrales dividen el eje mediastinico en proporciones exactas (t = 0.25, 0.50, 0.75).
\end{enumerate}

\subsection{Contribuciones Metodologicas}

\begin{enumerate}
    \item \textbf{Documentacion exhaustiva}: Se registro el proceso completo de desarrollo en 15 sesiones, incluyendo experimentos fallidos y lecciones aprendidas, facilitando la reproducibilidad.

    \item \textbf{Analisis granular}: Se proporciono analisis detallado por landmark y categoria de patologia, permitiendo identificar fortalezas y debilidades especificas del sistema.

    \item \textbf{Codigo abierto}: Se desarrollo una implementacion completa y documentada disponible para la comunidad cientifica.
\end{enumerate}

% -----------------------------------------------------------------------------
\section{Cumplimiento de Objetivos}
\label{sec:cumplimiento_objetivos}
% -----------------------------------------------------------------------------

\subsection{Objetivo General}

\begin{resultadoclave}
\textbf{Objetivo}: Desarrollar un sistema de deep learning para la deteccion automatica de 15 landmarks anatomicos en radiografias de torax, alcanzando un error de localizacion menor a 8 pixeles.

\textbf{Resultado}: \textbf{CUMPLIDO con creces}. Error logrado: 3.71 pixeles (54\% menor al objetivo).
\end{resultadoclave}

\subsection{Objetivos Especificos}

\begin{table}[htbp]
    \centering
    \caption{Evaluacion del cumplimiento de objetivos especificos}
    \label{tab:objetivos_cumplimiento}
    \begin{tabular}{@{}p{6cm}cc@{}}
        \toprule
        \textbf{Objetivo Especifico} & \textbf{Estado} & \textbf{Evidencia} \\
        \midrule
        Implementar arquitectura de red neuronal optimizada & Cumplido & ResNet-18 + CoordAttn + DeepHead \\
        Desarrollar pipeline de preprocesamiento robusto & Cumplido & CLAHE + normalizacion \\
        Evaluar funciones de perdida especializadas & Cumplido & Wing Loss superior a MSE \\
        Implementar estrategias de ensemble & Cumplido & Ensemble de 4 modelos \\
        Analizar rendimiento por landmark y categoria & Cumplido & Cap. 5, Tablas 5.3-5.5 \\
        Documentar proceso de desarrollo & Cumplido & 15 sesiones documentadas \\
        \bottomrule
    \end{tabular}
\end{table}

% -----------------------------------------------------------------------------
\section{Conclusiones Principales}
\label{sec:conclusiones_principales}
% -----------------------------------------------------------------------------

Del desarrollo y experimentacion realizados, se extraen las siguientes conclusiones:

\subsection{Sobre la Arquitectura}

\begin{enumerate}
    \item \textbf{ResNet-18 es suficiente}: Arquitecturas mas complejas no proporcionaron mejoras significativas para este tamano de dataset.

    \item \textbf{Coordinate Attention aporta precision posicional}: Su capacidad de codificar informacion espacial es valiosa para tareas de localizacion.

    \item \textbf{La cabeza de regresion importa}: hidden\_dim=768 supera a 256 y 512, indicando que la capacidad de la cabeza es critica.

    \item \textbf{Regresion directa es efectiva}: Para datasets pequenos con pocos landmarks, la regresion directa con Wing Loss es competitiva con enfoques de heatmaps.
\end{enumerate}

\subsection{Sobre el Preprocesamiento}

\begin{enumerate}
    \item \textbf{CLAHE es esencial para patologia}: Mejora del 68\% en COVID-19 demuestra su importancia para imagenes con consolidaciones.

    \item \textbf{Los parametros de CLAHE importan}: tile\_size=4 supera a 8 y 16, indicando que el contraste local fino es importante.

    \item \textbf{La normalizacion ImageNet funciona}: A pesar de la diferencia de dominio, los pesos preentrenados siguen siendo utiles.
\end{enumerate}

\subsection{Sobre el Entrenamiento}

\begin{enumerate}
    \item \textbf{El entrenamiento en dos fases es efectivo}: Congelar el backbone inicialmente protege los pesos preentrenados.

    \item \textbf{Mas epocas mejoran}: 100 epocas superan a 50, con early stopping previniendo overfitting.

    \item \textbf{Wing Loss supera a MSE}: La mayor sensibilidad a errores pequenos es beneficiosa para precision de landmarks.
\end{enumerate}

\subsection{Sobre el Ensemble}

\begin{enumerate}
    \item \textbf{El ensemble es el mayor contribuyente}: Reduccion de 2.25 px (de 6.75 a 4.50) con 3 modelos.

    \item \textbf{La seleccion de modelos es critica}: Excluir modelos debiles mejora mas que ponderarlos.

    \item \textbf{TTA es ``gratis''}: Mejora 0.28 px sin costo de entrenamiento.

    \item \textbf{Rendimientos decrecientes}: 4 modelos proporcionan la mayor parte del beneficio.
\end{enumerate}

% -----------------------------------------------------------------------------
\section{Limitaciones Reconocidas}
\label{sec:limitaciones_reconocidas}
% -----------------------------------------------------------------------------

Es importante reconocer las limitaciones de este trabajo:

\begin{enumerate}
    \item \textbf{Dataset pequeno}: 957 imagenes limitan la generalizabilidad.

    \item \textbf{Tres categorias solamente}: COVID-19, Normal y Neumonia Viral no cubren todo el espectro de patologias.

    \item \textbf{Fuente unica de datos}: Imagenes de equipos y protocolos limitados.

    \item \textbf{Sin validacion externa}: No se evaluo en datasets independientes.

    \item \textbf{Sin estimacion de incertidumbre}: El modelo no proporciona confianza en sus predicciones.
\end{enumerate}

% -----------------------------------------------------------------------------
\section{Direcciones de Trabajo Futuro}
\label{sec:trabajo_futuro_conc}
% -----------------------------------------------------------------------------

Las direcciones mas prometedoras para trabajo futuro son:

\begin{enumerate}
    \item \textbf{Expansion del dataset}: Incorporar mas imagenes de diversas fuentes y patologias.

    \item \textbf{Cuantificacion de incertidumbre}: Implementar metodos bayesianos o ensembles para estimar confianza.

    \item \textbf{Interpretabilidad}: Desarrollar visualizaciones de atencion para explicar predicciones.

    \item \textbf{Aplicacion clinica}: Validar en entorno clinico real con retroalimentacion de radiologos.

    \item \textbf{Transfer learning}: Adaptar el modelo a otras modalidades (CT, radiografias laterales).
\end{enumerate}

% -----------------------------------------------------------------------------
\section{Reflexiones Finales}
\label{sec:reflexiones}
% -----------------------------------------------------------------------------

El desarrollo de este trabajo ha demostrado que las tecnicas modernas de deep learning pueden alcanzar precision clinicamente util en la deteccion de landmarks anatomicos, incluso con datasets relativamente pequenos y en presencia de patologia significativa.

El enfoque iterativo de desarrollo, documentando cada experimento y sus resultados, resulto fundamental para el exito del proyecto. Los ``fracasos'' (geometric losses, arquitectura jerarquica) fueron tan informativos como los exitos, revelando que el modelo de regresion directa ya captura implicitamente las relaciones geometricas que se intentaban explotar explicitamente.

La combinacion de CLAHE para preprocesamiento, Wing Loss para entrenamiento, y ensemble selectivo para inferencia constituye un pipeline robusto que otros investigadores pueden adaptar a problemas similares de deteccion de landmarks en imagenes medicas.

Finalmente, la cercania al limite teorico ($\sim$2 px sobre el ruido de etiquetado) sugiere que el problema esta bien caracterizado y que mejoras adicionales significativas requeriran datos de mayor calidad o enfoques fundamentalmente diferentes.

\vspace{1cm}

\begin{center}
    \rule{0.5\textwidth}{0.5pt}

    \vspace{0.5cm}

    \textit{``Lo que no se mide, no se puede mejorar.''}

    --- Peter Drucker
\end{center}

% -----------------------------------------------------------------------------
% FIN DEL CAPITULO
% -----------------------------------------------------------------------------
