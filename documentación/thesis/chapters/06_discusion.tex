% =============================================================================
% CAPITULO 6: DISCUSION
% =============================================================================

\chapter{Discusion}
\label{ch:discusion}

Este capitulo analiza e interpreta los resultados obtenidos, comparandolos con el estado del arte, examinando casos exitosos y fallidos, y discutiendo las implicaciones practicas y limitaciones del sistema desarrollado.

% -----------------------------------------------------------------------------
\section{Interpretacion de Resultados}
\label{sec:interpretacion}
% -----------------------------------------------------------------------------

\subsection{Superacion del Objetivo}

El error final de \errorpx{3.71} supera significativamente el objetivo inicial de $<$8 pixeles. Esta diferencia de mas del doble respecto al objetivo merece analisis:

\begin{enumerate}
    \item \textbf{El objetivo era conservador}: El valor de 8 px se establecio basandose en literatura de deteccion de landmarks faciales, sin considerar las particularidades del dominio medico.

    \item \textbf{Efecto acumulativo de mejoras}: Cada optimizacion (CLAHE, arquitectura, ensemble) contribuyo incrementalmente. La combinacion de todas resulto en una mejora no lineal.

    \item \textbf{Calidad del dataset}: Aunque pequeno, el dataset tiene etiquetado consistente, facilitando el aprendizaje.
\end{enumerate}

\subsection{Cercania al Limite Teorico}

El error de etiquetado manual se estima en 1.5-2.0 pixeles. Con 3.71 px de error, el modelo esta a solo $\sim$1.7-2.2 px del limite teorico:

\begin{equation}
    \text{Gap al limite} = 3.71 - 1.75 \approx 1.96 \text{ px}
    \label{eq:gap_teorico}
\end{equation}

Este pequeno gap sugiere que:
\begin{itemize}
    \item El modelo ha extraido casi toda la informacion disponible en las imagenes
    \item Mejoras adicionales requeriran datos de mejor calidad (etiquetado mas preciso) o mas datos
    \item La arquitectura actual es adecuada para este problema
\end{itemize}

\subsection{Variabilidad por Landmark}

La diferencia de error entre landmarks (2.64-5.63 px) refleja factores anatomicos y de etiquetado:

\begin{itemize}
    \item \textbf{Landmarks centrales (L9, L10, L11)}: Menor error porque su posicion esta determinada geometricamente por el eje L1-L2. El proceso de etiquetado los coloca automaticamente, reduciendo variabilidad.

    \item \textbf{Landmarks del eje (L1, L2)}: Error intermedio. Son puntos de referencia principales pero requieren juicio subjetivo del etiquetador.

    \item \textbf{Bordes superiores (L12, L13)}: Mayor error. Representan bordes difusos de los pulmones donde la definicion exacta es subjetiva.
\end{itemize}

\subsection{Impacto de la Patologia}

El analisis por categoria revela patrones clinicamente relevantes:

\begin{itemize}
    \item \textbf{COVID-19}: La mayor mejora absoluta (11.01 $\rightarrow$ 3.80 px) demuestra que CLAHE es particularmente efectivo para manejar consolidaciones pulmonares. Las opacidades en vidrio esmerilado y consolidaciones oscurecen los bordes anatomicos, y CLAHE los realza.

    \item \textbf{Normal}: El mejor rendimiento (3.50 px) es esperado, ya que las estructuras anatomicas estan claramente definidas.

    \item \textbf{Neumonia Viral}: Error ligeramente mayor (4.35 px), posiblemente porque los infiltrados difusos crean patrones mas irregulares que las consolidaciones focales de COVID-19.
\end{itemize}

% -----------------------------------------------------------------------------
\section{Comparacion con el Estado del Arte}
\label{sec:comparacion_sota}
% -----------------------------------------------------------------------------

\subsection{Contextualizacion de Resultados}

La comparacion directa con otros trabajos es compleja debido a diferencias en datasets, modalidades y definiciones de landmarks:

\begin{table}[htbp]
    \centering
    \caption{Comparacion contextualizada con trabajos relacionados}
    \label{tab:sota_comparison}
    \begin{tabular}{@{}p{3cm}p{2.5cm}ccc@{}}
        \toprule
        \textbf{Trabajo} & \textbf{Modalidad} & \textbf{Landmarks} & \textbf{Error} & \textbf{Notas} \\
        \midrule
        SCN \cite{payer2019integrating} & Cefalometrico & 19 & 1.17 mm & Alta resolucion \\
        HRNet \cite{wang2020deep} & Facial & 68 & 1.2 px & Imagenes naturales \\
        U-Net \cite{chen2019vertebrae} & Vertebras & 17 & 2.3 px & CT \\
        \textbf{Este trabajo} & \textbf{Torax RX} & \textbf{15} & \textbf{3.71 px} & \textbf{COVID-19 incluido} \\
        \bottomrule
    \end{tabular}
\end{table}

\textbf{Consideraciones importantes}:

\begin{enumerate}
    \item \textbf{Resolucion de imagen}: Nuestra resolucion de 299x299 px es significativamente menor que las utilizadas en radiografias cefalometricas (tipicamente 2000+ px), lo que hace que el error en pixeles sea menos comparable.

    \item \textbf{Presencia de patologia}: La mayoria de benchmarks de landmarks evaluan en imagenes ``normales''. Nuestro dataset incluye 51\% de imagenes patologicas, aumentando la dificultad.

    \item \textbf{Definicion de landmarks}: Los landmarks toracicos tienen mayor variabilidad anatomica que los cefalometricos, donde las estructuras oseas son mas constantes.
\end{enumerate}

\subsection{Contribuciones Originales}

Este trabajo aporta al estado del arte:

\begin{enumerate}
    \item \textbf{Primera demostracion de CLAHE para localizacion de landmarks}: Aunque CLAHE se ha usado para clasificacion, este es el primer trabajo que demuestra su efectividad especifica para tareas de localizacion en radiografias toracicas.

    \item \textbf{Analisis detallado por landmark y patologia}: La granularidad del analisis permite identificar puntos de mejora especificos, ausente en la mayoria de trabajos que reportan solo metricas globales.

    \item \textbf{Metodologia de ensemble selectivo}: La demostracion de que excluir modelos debiles es mas efectivo que ponderarlos es un insight practico para futuros trabajos.
\end{enumerate}

% -----------------------------------------------------------------------------
\section{Analisis de Casos}
\label{sec:analisis_casos}
% -----------------------------------------------------------------------------

\subsection{Casos Exitosos}

Los mejores resultados (error $<$ 2 px) se observan en:

\begin{itemize}
    \item Radiografias normales con buen contraste
    \item Pacientes adultos con anatomia ``tipica''
    \item Imagenes donde los bordes pulmonares son nitidos
    \item Landmarks centrales (L9, L10, L11) en todas las categorias
\end{itemize}

\textbf{Ejemplo de caso exitoso}: Una radiografia normal donde el error promedio fue 1.8 px. La imagen tenia excelente contraste, bordes pulmonares bien definidos y el paciente tenia anatomia simetrica.

\subsection{Casos Fallidos}

Los peores resultados (error $>$ 10 px) ocurren en:

\begin{itemize}
    \item \textbf{Consolidaciones extensas}: COVID-19 severo con opacidades que cubren $>$50\% del campo pulmonar
    \item \textbf{Derrame pleural}: Altera significativamente la posicion aparente de L14, L15
    \item \textbf{Rotacion del paciente}: Lateralizacion que viola la asuncion de simetria
    \item \textbf{Artefactos de imagen}: Cables, electrodos o dispositivos medicos
\end{itemize}

\textbf{Ejemplo de caso fallido}: Una radiografia de COVID-19 con consolidacion bilateral extensa donde el error en L14 fue 15.2 px. Los bordes pulmonares estaban completamente oscurecidos por las opacidades, haciendo imposible la localizacion precisa incluso para un humano.

\subsection{Patrones de Error}

El analisis de errores revela patrones sistematicos:

\begin{enumerate}
    \item \textbf{Sesgo hacia el centro}: En casos dificiles, las predicciones tienden hacia posiciones ``promedio'', resultando en landmarks bilaterales menos separados de lo esperado.

    \item \textbf{Mayor error en Y para costofrenicos}: L14 y L15 tienen mayor variabilidad vertical, reflejada en errores predominantemente en el eje Y.

    \item \textbf{Asimetria residual}: Aunque no se fuerza simetria, el modelo tiende a predecir pares bilaterales mas simetricos que el ground truth, lo cual puede ser ventaja o desventaja dependiendo del caso.
\end{enumerate}

% -----------------------------------------------------------------------------
\section{Analisis de Decisiones de Diseno}
\label{sec:decisiones_diseno}
% -----------------------------------------------------------------------------

\subsection{Decisiones que Funcionaron}

\begin{table}[htbp]
    \centering
    \caption{Resumen de decisiones exitosas y su impacto}
    \label{tab:successful_decisions}
    \begin{tabular}{@{}p{4cm}p{3cm}p{5cm}@{}}
        \toprule
        \textbf{Decision} & \textbf{Impacto} & \textbf{Justificacion} \\
        \midrule
        Wing Loss & -20\% error & Mayor sensibilidad a errores pequenos \\
        CLAHE (tile=4) & -10\% global, -19\% COVID & Realce local optimo \\
        hidden\_dim=768 & -8\% error & Mayor capacidad sin overfitting \\
        Ensemble selectivo & -16\% error & Exclusion de modelos debiles \\
        TTA con flip & -3\% error & Reduccion de varianza sin costo \\
        GroupNorm & Estabilidad & Mejor que BatchNorm con batch pequeno \\
        \bottomrule
    \end{tabular}
\end{table}

\subsection{Decisiones que NO Funcionaron}

\begin{table}[htbp]
    \centering
    \caption{Decisiones fallidas y lecciones aprendidas}
    \label{tab:failed_decisions}
    \begin{tabular}{@{}p{4cm}p{3cm}p{5cm}@{}}
        \toprule
        \textbf{Decision} & \textbf{Resultado} & \textbf{Leccion} \\
        \midrule
        Central Alignment Loss & Sin mejora & Restriccion ya satisfecha implicitamente \\
        Soft Symmetry Loss & Sin mejora & GT no es simetrico \\
        Category Weights & Empeoro & Causa overfitting en COVID \\
        Arquitectura jerarquica & +83\% error & Regresion directa aprende geometria \\
        Weighted ensemble & Sin mejora & Promedio simple es robusto \\
        \bottomrule
    \end{tabular}
\end{table}

\textbf{Insight clave}: Las restricciones geometricas explicitas (alignment loss, symmetry loss, arquitectura jerarquica) no mejoraron porque el modelo de regresion directa ya aprende estas relaciones implicitamente de los datos.

% -----------------------------------------------------------------------------
\section{Limitaciones}
\label{sec:limitaciones}
% -----------------------------------------------------------------------------

\subsection{Limitaciones del Dataset}

\begin{enumerate}
    \item \textbf{Tamano limitado}: 957 imagenes es un dataset pequeno para deep learning. La generalizacion a poblaciones mas diversas requiere validacion.

    \item \textbf{Sesgo demografico}: No se dispone de informacion demografica. El modelo puede no generalizar a poblaciones diferentes de la de entrenamiento.

    \item \textbf{Variedad de equipos}: Las imagenes provienen de fuentes limitadas. Diferentes equipos de rayos X producen caracteristicas de imagen diferentes.

    \item \textbf{Tipos de patologia}: Solo tres categorias evaluadas. El rendimiento en otras condiciones (TB, cancer, fibrosis) es desconocido.
\end{enumerate}

\subsection{Limitaciones del Modelo}

\begin{enumerate}
    \item \textbf{Resolucion fija}: El modelo asume imagenes de 299x299 px. Imagenes de resolucion diferente requieren redimensionamiento.

    \item \textbf{Ausencia de incertidumbre}: El modelo no proporciona intervalos de confianza para sus predicciones.

    \item \textbf{Interpretabilidad limitada}: Como ``caja negra'', es dificil explicar por que falla en casos especificos.

    \item \textbf{Tiempo de inferencia}: El ensemble de 4 modelos con TTA (8 forward passes) puede ser lento para aplicaciones en tiempo real.
\end{enumerate}

\subsection{Limitaciones del Ground Truth}

\begin{enumerate}
    \item \textbf{Etiquetador unico}: Las anotaciones provienen de un proceso de etiquetado sin evaluacion inter-observador.

    \item \textbf{Ruido inherente}: El error de etiquetado ($\sim$1.5-2.0 px) establece un limite inferior al rendimiento.

    \item \textbf{Asimetrias no documentadas}: No esta claro si las asimetrias en el GT son anatomicas o errores.
\end{enumerate}

% -----------------------------------------------------------------------------
\section{Implicaciones Practicas}
\label{sec:implicaciones}
% -----------------------------------------------------------------------------

\subsection{Aplicaciones Clinicas Potenciales}

El sistema desarrollado podria aplicarse en:

\begin{enumerate}
    \item \textbf{Calculo automatico del indice cardiotoracico}: Utilizando L1, L2 y landmarks laterales para medir la relacion corazon/torax.

    \item \textbf{Triaje rapido}: Deteccion automatica de anomalias de posicion que sugieran patologia.

    \item \textbf{Seguimiento longitudinal}: Comparacion objetiva de landmarks entre estudios sucesivos del mismo paciente.

    \item \textbf{Control de calidad}: Verificacion de posicionamiento correcto del paciente durante la adquisicion.
\end{enumerate}

\subsection{Consideraciones para Despliegue}

Para uso clinico real, se requereria:

\begin{itemize}
    \item Validacion en datasets externos
    \item Aprobacion regulatoria (FDA, CE)
    \item Integracion con sistemas PACS
    \item Interfaz de usuario para revision de resultados
    \item Mecanismos de retroalimentacion para casos fallidos
\end{itemize}

% -----------------------------------------------------------------------------
\section{Trabajo Futuro}
\label{sec:trabajo_futuro}
% -----------------------------------------------------------------------------

\subsection{Mejoras Inmediatas}

\begin{enumerate}
    \item \textbf{Estimacion de incertidumbre}: Implementar dropout en inferencia (MC Dropout) o ensembles bayesianos para proporcionar intervalos de confianza.

    \item \textbf{Deteccion de casos outlier}: Sistema de alerta cuando las predicciones estan fuera de rangos anatomicos plausibles.

    \item \textbf{Explicabilidad}: Mapas de atencion (Grad-CAM) para visualizar que regiones influyen en cada prediccion.
\end{enumerate}

\subsection{Extensiones a Mediano Plazo}

\begin{enumerate}
    \item \textbf{Expansion del dataset}: Incorporacion de mas imagenes, especialmente de otras patologias y demografias.

    \item \textbf{Transfer a otras modalidades}: Adaptacion del modelo a CT de torax o radiografias laterales.

    \item \textbf{Heatmap regression}: Evaluacion de enfoques basados en heatmaps con DSNT para potencial precision subpixel.
\end{enumerate}

\subsection{Vision a Largo Plazo}

\begin{enumerate}
    \item \textbf{Sistema de diagnostico asistido}: Integracion de deteccion de landmarks con clasificacion de patologias para un sistema completo de analisis de radiografias.

    \item \textbf{Aprendizaje continuo}: Sistema que mejore con retroalimentacion de radiologos en produccion.

    \item \textbf{Benchmark publico}: Publicacion del dataset y metodologia para facilitar comparacion con futuros trabajos.
\end{enumerate}

% -----------------------------------------------------------------------------
% FIN DEL CAPITULO
% -----------------------------------------------------------------------------
