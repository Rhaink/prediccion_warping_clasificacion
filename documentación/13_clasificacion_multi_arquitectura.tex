% ==============================================================================
% DOCUMENTACIÓN CIENTÍFICA - CLASIFICACIÓN MULTI-ARQUITECTURA
% Proyecto: Detección de COVID-19 mediante Landmarks Anatómicos
% Sesiones cubiertas: 22-23, 31
% Nivel: Doctoral/Científico - Completo y Detallado
% ==============================================================================

\documentclass[12pt,a4paper]{article}
% ==============================================================================
% PREÁMBULO LATEX PARA DOCUMENTACIÓN CIENTÍFICA DE NIVEL DOCTORAL
% Proyecto: Detección de COVID-19 mediante Landmarks Anatómicos y Warping Geométrico
% ==============================================================================

% --- Codificación y idioma ---
\usepackage[utf8]{inputenc}
\usepackage[T1]{fontenc}
\usepackage[spanish,es-tabla]{babel}

% --- Matemáticas ---
\usepackage{amsmath,amssymb,amsfonts,amsthm}
\usepackage{mathtools}
\usepackage{bm}  % Negritas en matemáticas

% --- Tablas profesionales ---
\usepackage{booktabs}
\usepackage{multirow}
\usepackage{array}
\usepackage{longtable}
\usepackage{tabularx}

% --- Figuras y gráficos ---
\usepackage{graphicx}
\usepackage{subfig}
\usepackage{float}
\usepackage{caption}

% --- Algoritmos y pseudocódigo ---
\usepackage{algorithm}
\usepackage{algorithmic}

% --- Colores (debe cargarse antes de listings) ---
\usepackage{xcolor}
\definecolor{covidred}{RGB}{220,53,69}
\definecolor{normalgreen}{RGB}{40,167,69}
\definecolor{viralblue}{RGB}{0,123,255}
\definecolor{codeblue}{RGB}{0,0,180}
\definecolor{codegray}{RGB}{128,128,128}

% --- Código fuente ---
\usepackage{listings}
\lstset{
    language=Python,
    basicstyle=\ttfamily\small,
    keywordstyle=\color{blue}\bfseries,
    commentstyle=\color{gray}\itshape,
    stringstyle=\color{red},
    numbers=left,
    numberstyle=\tiny\color{gray},
    stepnumber=1,
    numbersep=5pt,
    backgroundcolor=\color{white},
    frame=single,
    rulecolor=\color{black},
    tabsize=4,
    captionpos=b,
    breaklines=true,
    breakatwhitespace=false,
    showspaces=false,
    showstringspaces=false,
    showtabs=false,
    literate={á}{{\'a}}1 {é}{{\'e}}1 {í}{{\'i}}1 {ó}{{\'o}}1 {ú}{{\'u}}1
             {Á}{{\'A}}1 {É}{{\'E}}1 {Í}{{\'I}}1 {Ó}{{\'O}}1 {Ú}{{\'U}}1
             {ñ}{{\~n}}1 {Ñ}{{\~N}}1
             {ü}{{\"u}}1 {Ü}{{\"U}}1
             {¿}{{?`}}1 {¡}{{!`}}1
}

% --- Referencias y enlaces ---
\usepackage{hyperref}
\hypersetup{
    colorlinks=true,
    linkcolor=blue,
    filecolor=magenta,
    urlcolor=cyan,
    citecolor=blue,
    pdftitle={Documentación del Proyecto de Tesis},
    pdfauthor={},
}
\usepackage{cleveref}

% --- Bibliografía ---
\usepackage{natbib}
\bibliographystyle{plainnat}

% --- Geometría de página ---
\usepackage[a4paper,margin=2.5cm]{geometry}

% --- Espaciado ---
\usepackage{setspace}
\onehalfspacing

% --- Encabezados y pies ---
\usepackage{fancyhdr}
\pagestyle{fancy}
\fancyhf{}
\rhead{\rightmark}
\lhead{\leftmark}
\cfoot{\thepage}

% --- Entornos personalizados ---
\theoremstyle{definition}
\newtheorem{definicion}{Definición}[section]
\newtheorem{proposicion}{Proposición}[section]
\newtheorem{teorema}{Teorema}[section]
\newtheorem{lema}{Lema}[section]
\newtheorem{corolario}{Corolario}[section]

\theoremstyle{remark}
\newtheorem{observacion}{Observación}[section]
\newtheorem{nota}{Nota}[section]
\newtheorem{hipotesis}{Hipótesis}[section]

% --- Comandos personalizados ---
% Vectores y matrices
\newcommand{\vect}[1]{\mathbf{#1}}
\newcommand{\mat}[1]{\mathbf{#1}}

% Operadores
\DeclareMathOperator*{\argmin}{arg\,min}
\DeclareMathOperator*{\argmax}{arg\,max}
\DeclareMathOperator{\sgn}{sgn}
\DeclareMathOperator{\diag}{diag}
\DeclareMathOperator{\trace}{tr}

% Normas y productos
\newcommand{\norm}[1]{\left\|#1\right\|}
\newcommand{\abs}[1]{\left|#1\right|}
\newcommand{\inner}[2]{\langle #1, #2 \rangle}

% Conjuntos
\newcommand{\R}{\mathbb{R}}
\newcommand{\N}{\mathbb{N}}
\newcommand{\Z}{\mathbb{Z}}

% Espacios de imágenes y landmarks
\newcommand{\imgspace}{\mathcal{I}}
\newcommand{\landmarkspace}{\mathcal{L}}
\newcommand{\classspace}{\mathcal{Y}}

% Landmarks específicos
\newcommand{\landmark}[1]{L_{#1}}
\newcommand{\landmarkpair}[2]{(L_{#1}, L_{#2})}

% Funciones de pérdida
\newcommand{\loss}{\mathcal{L}}
\newcommand{\wingloss}{\mathcal{L}_{\text{wing}}}
\newcommand{\mseloss}{\mathcal{L}_{\text{MSE}}}

% Métricas
\newcommand{\accuracy}{\text{Acc}}
\newcommand{\precision}{\text{Prec}}
\newcommand{\recall}{\text{Rec}}
\newcommand{\fscore}{F_1}

% Abreviaciones
\newcommand{\ie}{\textit{i.e.}}
\newcommand{\eg}{\textit{e.g.}}
\newcommand{\etal}{\textit{et al.}}
\newcommand{\etc}{\textit{etc.}}

% Referencias a archivos del proyecto
\newcommand{\archivo}[1]{\texttt{#1}}
\newcommand{\funcion}[1]{\texttt{#1()}}
\newcommand{\clase}[1]{\texttt{#1}}
\newcommand{\parametro}[1]{\texttt{#1}}

% Unidades
\newcommand{\px}{\,\text{px}}
\newcommand{\epoch}{\,\text{época}}
\newcommand{\epochs}{\,\text{épocas}}

% --- Información del documento ---
\newcommand{\proyectotitulo}{Detección de COVID-19 en Radiografías de Tórax mediante Landmarks Anatómicos y Normalización Geométrica}
\newcommand{\proyectosubtitulo}{Documentación Científica del Proceso de Desarrollo}

% --- Formato de secciones ---
\usepackage{titlesec}
\titleformat{\section}
  {\normalfont\Large\bfseries}{\thesection}{1em}{}
\titleformat{\subsection}
  {\normalfont\large\bfseries}{\thesubsection}{1em}{}
\titleformat{\subsubsection}
  {\normalfont\normalsize\bfseries}{\thesubsubsection}{1em}{}

% --- Notas al margen para figuras sugeridas ---
\usepackage{marginnote}
\newcommand{\figurasugerida}[1]{\marginnote{\scriptsize\textcolor{blue}{[Figura: #1]}}}

% --- Cajas para resultados importantes ---
\usepackage{tcolorbox}
\newtcolorbox{resultadoimportante}[1][]{
    colback=green!5!white,
    colframe=green!75!black,
    fonttitle=\bfseries,
    title=Resultado Importante,
    #1
}

\newtcolorbox{hallazgo}[1][]{
    colback=blue!5!white,
    colframe=blue!75!black,
    fonttitle=\bfseries,
    title=Hallazgo,
    #1
}

\newtcolorbox{metodologia}[1][]{
    colback=yellow!5!white,
    colframe=yellow!75!black,
    fonttitle=\bfseries,
    title=Metodología,
    #1
}

\newtcolorbox{figuradescripcion}[1][]{
    colback=gray!5!white,
    colframe=gray!75!black,
    fonttitle=\bfseries,
    title=Descripción de Figura,
    #1
}

% ==============================================================================
% FIN DEL PREÁMBULO
% ==============================================================================


\title{Evaluación Comparativa de Arquitecturas CNN\\para Clasificación de Neumonía:\\
Análisis Original vs. Geométricamente Normalizado}
\author{Documentación del Proceso de Desarrollo}
\date{Sesiones: 22-23, 31}

\begin{document}
\maketitle

\begin{abstract}
Este documento presenta una evaluación sistemática de 7 arquitecturas de
redes neuronales convolucionales para la clasificación de radiografías de
tórax en tres categorías: Normal, COVID-19 y Neumonía Viral. Se compara
el rendimiento de cada arquitectura entrenada en el dataset original
versus el dataset geométricamente normalizado mediante warping. Los resultados
demuestran que 6 de 7 arquitecturas logran mayor accuracy en imágenes
originales que en warpeadas, con una diferencia promedio del 4.2\%.
MobileNetV2 alcanza el mejor rendimiento absoluto (98.96\% en original),
mientras que ResNet-18 muestra la mayor sensibilidad a la normalización
geométrica (gap del 10.42\%). La excepción notable es AlexNet, que mejora
en el dataset warpeado, posiblemente debido a su limitada capacidad para
aprender patrones complejos. Este análisis sugiere la presencia de
\textit{shortcut learning} relacionado con características del dataset
más que con deficiencias arquitecturales específicas.
\end{abstract}

\tableofcontents
\newpage

% ==============================================================================
\section{Introducción y Motivación}
% ==============================================================================

\subsection{Objetivo del Estudio Comparativo}

El estudio multi-arquitectura tiene dos objetivos principales:

\begin{enumerate}
    \item \textbf{Encontrar la mejor arquitectura}: Identificar qué red CNN
    logra el mejor rendimiento para la tarea de clasificación de neumonía

    \item \textbf{Evaluar el impacto de la normalización}: Determinar si
    la normalización geométrica mejora, empeora o no afecta el rendimiento
    de clasificación
\end{enumerate}

\subsection{Hipótesis a Evaluar}

\begin{hipotesis}[H1: Alto rendimiento en original]
Las arquitecturas CNN entrenadas en el dataset original lograrán accuracy
superior al 90\%, debido a la presencia de patrones discriminativos
(potencialmente incluyendo shortcuts).
\end{hipotesis}

\begin{hipotesis}[H2: Rendimiento ``honesto'' en warped]
El dataset warpeado, al eliminar variabilidad geométrica irrelevante,
mostrará rendimiento entre 85-92\%, reflejando la dificultad intrínseca
del problema sin shortcuts.
\end{hipotesis}

\begin{hipotesis}[H3: Gap consistente entre arquitecturas]
Si el gap (original $-$ warped) es consistente entre arquitecturas,
el fenómeno es atribuible al dataset (shortcuts) más que a características
específicas de cada modelo.
\end{hipotesis}

% ==============================================================================
\section{Arquitecturas Evaluadas}
% ==============================================================================

\subsection{Selección de Arquitecturas}

Se seleccionaron 7 arquitecturas representativas que cubren diferentes
paradigmas de diseño y épocas del desarrollo de CNNs:

\begin{table}[htbp]
\centering
\caption{Arquitecturas CNN evaluadas}
\label{tab:architectures}
\begin{tabular}{llcc}
\toprule
\textbf{Arquitectura} & \textbf{Paradigma} & \textbf{Parámetros} & \textbf{Año} \\
\midrule
AlexNet & Clásica profunda & 57M & 2012 \\
VGG-16 & Muy profunda uniforme & 134M & 2014 \\
ResNet-18 & Conexiones residuales & 11M & 2015 \\
ResNet-50 & Residual profunda & 23.5M & 2015 \\
DenseNet-121 & Conexiones densas & 7M & 2016 \\
MobileNetV2 & Eficiente (depthwise) & 2.2M & 2018 \\
EfficientNet-B0 & Compound scaling & 4M & 2019 \\
\bottomrule
\end{tabular}
\end{table}

\subsection{Descripción de Arquitecturas}

\subsubsection{AlexNet (2012)}

Arquitectura pionera que demostró la viabilidad de CNNs profundas:
\begin{itemize}
    \item 8 capas con pesos entrenables (5 conv + 3 FC)
    \item Introdujo ReLU, dropout, data augmentation
    \item 57M parámetros pero arquitectura relativamente simple
\end{itemize}

\subsubsection{VGG-16 (2014)}

Arquitectura uniforme con filtros pequeños:
\begin{itemize}
    \item 16 capas con pesos (13 conv + 3 FC)
    \item Todos los filtros de $3 \times 3$
    \item 134M parámetros, muy costosa computacionalmente
\end{itemize}

\subsubsection{ResNet-18/50 (2015)}

Introdujo las conexiones residuales:
\begin{equation}
\vect{y} = \mathcal{F}(\vect{x}, \{W_i\}) + \vect{x}
\end{equation}
\begin{itemize}
    \item Permite entrenar redes muy profundas
    \item ResNet-18: 11M params, ResNet-50: 23.5M params
    \item Ampliamente usado como backbone en transfer learning
\end{itemize}

\subsubsection{DenseNet-121 (2016)}

Conexiones densas entre todas las capas:
\begin{equation}
\vect{x}_l = H_l([\vect{x}_0, \vect{x}_1, \ldots, \vect{x}_{l-1}])
\end{equation}
\begin{itemize}
    \item Reutilización de features a través de concatenación
    \item 7M parámetros, eficiente en uso de parámetros
    \item Excelente para imágenes médicas
\end{itemize}

\subsubsection{MobileNetV2 (2018)}

Arquitectura eficiente para dispositivos móviles:
\begin{itemize}
    \item Convoluciones depthwise separables
    \item Inverted residual blocks
    \item Solo 2.2M parámetros, muy rápida
\end{itemize}

\subsubsection{EfficientNet-B0 (2019)}

Compound scaling de profundidad, ancho y resolución:
\begin{equation}
\text{depth} = \alpha^\phi, \quad \text{width} = \beta^\phi, \quad \text{resolution} = \gamma^\phi
\end{equation}
\begin{itemize}
    \item Búsqueda arquitectural automática (NAS)
    \item 4M parámetros, estado del arte en eficiencia
\end{itemize}

% ==============================================================================
\section{Metodología Experimental}
% ==============================================================================

\subsection{Configuración de Entrenamiento}

\begin{table}[htbp]
\centering
\caption{Parámetros de entrenamiento}
\label{tab:training_params}
\begin{tabular}{ll}
\toprule
\textbf{Parámetro} & \textbf{Valor} \\
\midrule
Épocas máximas & 100 \\
Early stopping patience & 10 \\
Métrica de early stopping & F1-score (val) \\
Batch size & 32 \\
Optimizer & Adam \\
Learning rate inicial & $10^{-4}$ \\
LR scheduler & ReduceLROnPlateau \\
Weight decay & $10^{-4}$ \\
Loss function & CrossEntropy con class weights \\
\bottomrule
\end{tabular}
\end{table}

\subsection{Transfer Learning}

Todas las arquitecturas fueron inicializadas con pesos pre-entrenados en
ImageNet:

\begin{enumerate}
    \item \textbf{Fase 1}: Backbone congelado, solo entrenar clasificador (10 épocas)
    \item \textbf{Fase 2}: Fine-tuning completo de toda la red
\end{enumerate}

\subsection{Adaptación para 3 Clases}

La capa de clasificación final fue reemplazada:
\begin{equation}
\text{FC}_{\text{original}}(d_{\text{feat}}, 1000) \to \text{FC}_{\text{nuevo}}(d_{\text{feat}}, 3)
\end{equation}

donde $d_{\text{feat}}$ es la dimensión del feature vector de cada arquitectura.

\subsection{Class Weights}

Dado el desbalance de clases:
\begin{equation}
w_c = \frac{N_{\text{total}}}{k \cdot N_c}
\end{equation}

\begin{table}[htbp]
\centering
\caption{Class weights aplicados}
\label{tab:class_weights_clf}
\begin{tabular}{lcc}
\toprule
\textbf{Clase} & \textbf{$N$ (train)} & \textbf{Weight} \\
\midrule
Normal & 350 & 0.68 \\
COVID & 229 & 1.04 \\
Viral & 138 & 1.73 \\
\bottomrule
\end{tabular}
\end{table}

\subsection{Data Augmentation}

\begin{table}[htbp]
\centering
\caption{Transformaciones de data augmentation}
\label{tab:augmentation}
\begin{tabular}{lll}
\toprule
\textbf{Transformación} & \textbf{Parámetro} & \textbf{Probabilidad} \\
\midrule
Flip horizontal & -- & 0.5 \\
Rotación & $\pm 10°$ & 0.5 \\
Affine & translate 5\%, scale 95-105\% & 0.5 \\
Color jitter & brillo/contraste $\pm 10\%$ & 0.3 \\
\bottomrule
\end{tabular}
\end{table}

\subsection{Métricas de Evaluación}

\begin{enumerate}
    \item \textbf{Accuracy}: Proporción de predicciones correctas
    \begin{equation}
    \text{Accuracy} = \frac{1}{N}\sum_{c=1}^{C} \text{TP}_c = \frac{\sum_{i=1}^{N} \mathbb{1}[\hat{y}_i = y_i]}{N}
    \end{equation}
    donde $C$ es el número de clases y $\text{TP}_c$ son los verdaderos positivos de la clase $c$.

    \item \textbf{F1-Score macro}: Media de F1 por clase
    \begin{equation}
    \text{F1} = \frac{2 \cdot \text{Precision} \cdot \text{Recall}}{\text{Precision} + \text{Recall}}
    \end{equation}

    \item \textbf{Matriz de confusión}: Distribución de errores por clase
\end{enumerate}

% ==============================================================================
\section{Resultados}
% ==============================================================================

\subsection{Comparación de Accuracy}

\begin{table}[htbp]
\centering
\caption{Resultados de accuracy y F1-score en test set (n=96)}
\label{tab:results_accuracy}
\begin{tabular}{lcccc}
\toprule
\textbf{Modelo} & \textbf{Acc. Original} & \textbf{Acc. Warped} & \textbf{F1 Orig.} & \textbf{F1 Warp.} \\
\midrule
\textbf{MobileNetV2} & \textbf{98.96\%} & 92.71\% & 98.74\% & 92.99\% \\
ResNet-18 & 95.83\% & 85.42\% & 96.02\% & 85.61\% \\
EfficientNet-B0 & 95.83\% & 91.67\% & 95.65\% & 92.50\% \\
DenseNet-121 & 94.79\% & 89.58\% & 95.16\% & 90.46\% \\
ResNet-50 & 93.75\% & 89.58\% & 94.19\% & 90.42\% \\
VGG-16 & 93.75\% & 90.62\% & 93.89\% & 91.22\% \\
AlexNet & 86.46\% & \textbf{90.62\%} & 87.02\% & 91.17\% \\
\midrule
\textbf{Promedio} & \textbf{94.2\%} & \textbf{90.0\%} & -- & -- \\
\bottomrule
\end{tabular}
\end{table}

\subsection{Análisis del Gap Original-Warped}

\begin{table}[htbp]
\centering
\caption{Gap de accuracy entre original y warped}
\label{tab:gap_analysis}
\begin{tabular}{lcc}
\toprule
\textbf{Modelo} & \textbf{Gap (Orig $-$ Warp)} & \textbf{Interpretación} \\
\midrule
ResNet-18 & +10.42\% & Mayor sensibilidad a shortcuts \\
MobileNetV2 & +6.25\% & Sensibilidad moderada \\
DenseNet-121 & +5.21\% & Sensibilidad moderada \\
EfficientNet-B0 & +4.17\% & Sensibilidad baja \\
ResNet-50 & +4.17\% & Sensibilidad baja \\
VGG-16 & +3.12\% & Sensibilidad muy baja \\
AlexNet & \textbf{-4.17\%} & \textbf{Mejor en warped (anomalía)} \\
\midrule
\textbf{Promedio} & \textbf{+4.17\%} & Gap consistente \\
\textbf{Std} & 4.1\% & \\
\bottomrule
\end{tabular}
\end{table}

\begin{resultadoimportante}[title={Gap consistente indica shortcut learning}]
El gap promedio de 4.17\% (std=4.1\%) entre datasets es relativamente
consistente entre arquitecturas, sugiriendo que el fenómeno está relacionado
con características del dataset (shortcuts) más que con propiedades
específicas de los modelos.
\end{resultadoimportante}

\subsection{Validación de Hipótesis}

\begin{table}[htbp]
\centering
\caption{Validación de hipótesis experimentales}
\label{tab:hypothesis_validation}
\begin{tabular}{llcc}
\toprule
\textbf{Hipótesis} & \textbf{Condición} & \textbf{Resultado} & \textbf{Validada} \\
\midrule
H1: Original $>$ 90\% & 6/7 modelos & 6/7 (86\% AlexNet) & Sí \\
H2: Warped 85-92\% & 7/7 modelos & 7/7 cumple & Sí \\
H3: Gap consistente & std $<$ 5\% & std = 4.1\% & Sí \\
\bottomrule
\end{tabular}
\end{table}

\subsection{Caso Anómalo: AlexNet}

AlexNet presenta un comportamiento opuesto al resto de arquitecturas:
mejor rendimiento en warped (+4.17\%).

\begin{hallazgo}[title={Hipótesis sobre anomalía de AlexNet}]
Posibles explicaciones:
\begin{enumerate}
    \item \textbf{Capacidad limitada}: Solo 8 capas con pesos, posiblemente
    insuficiente para aprender shortcuts complejos
    \item \textbf{Features de bajo nivel}: Los shortcuts pueden requerir
    features de alto nivel que AlexNet no captura efectivamente
    \item \textbf{Regularización implícita}: La simplicidad de la arquitectura
    actúa como regularizador contra overfitting a artefactos
\end{enumerate}
\end{hallazgo}

% ==============================================================================
\section{Análisis por Arquitectura}
% ==============================================================================

\subsection{MobileNetV2: Mejor Rendimiento Absoluto}

\begin{table}[htbp]
\centering
\caption{Desglose de rendimiento de MobileNetV2}
\label{tab:mobilenet_detail}
\begin{tabular}{lcc}
\toprule
\textbf{Métrica} & \textbf{Original} & \textbf{Warped} \\
\midrule
Accuracy & 98.96\% & 92.71\% \\
F1-score & 98.74\% & 92.99\% \\
Precision (macro) & 98.8\% & 93.1\% \\
Recall (macro) & 98.7\% & 93.0\% \\
\bottomrule
\end{tabular}
\end{table}

MobileNetV2 destaca por:
\begin{itemize}
    \item Menor número de parámetros (2.2M) con alto rendimiento
    \item Inverted residual blocks eficientes
    \item Buena transferencia desde ImageNet
\end{itemize}

\subsection{ResNet-18: Mayor Sensibilidad}

ResNet-18 muestra el mayor gap (10.42\%), sugiriendo mayor dependencia
de shortcuts. Paradójicamente, es la arquitectura más usada como backbone
en el modelo de landmarks del proyecto.

\begin{observacion}[Implicación para el proyecto]
Aunque ResNet-18 muestra mayor sensibilidad a shortcuts en clasificación,
su uso en el modelo de landmarks (regresión de coordenadas) puede ser
apropiado dado que la tarea de regresión no se beneficia de los mismos
shortcuts de clasificación.
\end{observacion}

\subsection{EfficientNet-B0: Balance Óptimo}

EfficientNet-B0 presenta el mejor trade-off:
\begin{itemize}
    \item Alto rendimiento (95.83\% original, 91.67\% warped)
    \item Gap moderado (4.17\%)
    \item Solo 4M parámetros
\end{itemize}

% ==============================================================================
\section{Implicaciones para Shortcut Learning}
% ==============================================================================

\subsection{Definición de Shortcut Learning}

\begin{definicion}[Shortcut Learning]
Fenómeno donde los modelos de deep learning aprenden a resolver tareas
utilizando ``atajos'' (shortcuts) no genuinos en lugar de características
semánticamente relevantes \citep{geirhos2020shortcut}.
\end{definicion}

En nuestro contexto, posibles shortcuts incluyen:
\begin{itemize}
    \item Artefactos de adquisición correlacionados con diagnóstico
    \item Marcadores de texto o equipamiento en las radiografías
    \item Características de posicionamiento del paciente
    \item Parámetros de procesamiento de imagen específicos por institución
\end{itemize}

\subsection{Evidencia de Shortcuts en Nuestro Dataset}

\begin{enumerate}
    \item \textbf{Gap consistente}: 6/7 arquitecturas muestran el mismo patrón
    \item \textbf{Mejor en original}: El dataset original contiene ``algo''
    que facilita la clasificación
    \item \textbf{El warping elimina información}: La normalización geométrica
    elimina algunas características que los modelos explotaban
\end{enumerate}

\begin{hallazgo}[title={El problema es de datos, no de arquitectura}]
La consistencia del gap entre arquitecturas de diseños muy diferentes
(desde AlexNet 2012 hasta EfficientNet 2019) indica que el shortcut learning
es un problema del dataset, no una deficiencia de arquitecturas específicas.
Cualquier CNN con suficiente capacidad aprenderá estos shortcuts.
\end{hallazgo}

\subsection{Interpretación de Resultados}

\begin{table}[htbp]
\centering
\caption{Interpretación de rendimiento por dataset}
\label{tab:interpretation}
\begin{tabular}{lll}
\toprule
\textbf{Dataset} & \textbf{Rendimiento} & \textbf{Interpretación} \\
\midrule
Original & $\sim$94\% & Aprende shortcuts + patología (posiblemente) \\
Warped & $\sim$90\% & Rendimiento más ``honesto'' sin shortcuts geométricos \\
\bottomrule
\end{tabular}
\end{table}

% ==============================================================================
\section{Figuras Sugeridas}
% ==============================================================================

\subsection{Figura 13.1: Comparación de Accuracy por Arquitectura}

\begin{figuradescripcion}
\textbf{Título}: Accuracy en test set: Original vs. Warped por arquitectura

\textbf{Contenido}: Gráfico de barras agrupadas.

\textbf{Elementos visuales}:
\begin{itemize}
    \item Eje X: 7 arquitecturas ordenadas por accuracy en original
    \item Eje Y: Accuracy (\%)
    \item Barras azules: Dataset original
    \item Barras naranjas: Dataset warped
    \item Línea de referencia en 90\%
\end{itemize}

\textbf{Anotaciones}: Delta (gap) sobre cada par de barras
\end{figuradescripcion}

\subsection{Figura 13.2: Gap Original-Warped}

\begin{figuradescripcion}
\textbf{Título}: Diferencia de accuracy (Original $-$ Warped) por arquitectura

\textbf{Contenido}: Gráfico de barras horizontales.

\textbf{Elementos visuales}:
\begin{itemize}
    \item Barras ordenadas por magnitud del gap
    \item AlexNet resaltado (único negativo)
    \item Línea vertical en 0
    \item Media y std indicados
\end{itemize}
\end{figuradescripcion}

\subsection{Figura 13.3: Matrices de Confusión}

\begin{figuradescripcion}
\textbf{Título}: Matrices de confusión: mejor modelo en cada dataset

\textbf{Contenido}: 2 matrices lado a lado.

\textbf{Elementos visuales}:
\begin{itemize}
    \item Izquierda: MobileNetV2 en Original (98.96\%)
    \item Derecha: MobileNetV2 en Warped (92.71\%)
    \item Heatmap con valores absolutos y porcentajes
    \item Clases: Normal, COVID, Viral
\end{itemize}

\textbf{Observación clave}: Mostrar qué errores introduce el warping
\end{figuradescripcion}

\subsection{Figura 13.4: Eficiencia vs. Accuracy}

\begin{figuradescripcion}
\textbf{Título}: Trade-off entre número de parámetros y accuracy

\textbf{Contenido}: Scatter plot.

\textbf{Elementos visuales}:
\begin{itemize}
    \item Eje X: Millones de parámetros (log scale)
    \item Eje Y: Accuracy (\%)
    \item Puntos por arquitectura, dos colores (original/warped)
    \item MobileNetV2 y EfficientNet-B0 resaltados como óptimos
\end{itemize}
\end{figuradescripcion}

% ==============================================================================
\section{Archivos Fuente y Reproducibilidad}
% ==============================================================================

\begin{table}[htbp]
\centering
\caption{Archivos de implementación del estudio multi-arquitectura}
\label{tab:source_files_multiarch}
\begin{tabular}{p{5.5cm}p{7.5cm}}
\toprule
\textbf{Archivo} & \textbf{Contenido} \\
\midrule
\archivo{scripts/train\_all\_architectures.py} & Script de entrenamiento multi-arquitectura:\\
& \quad - Carga de 7 arquitecturas pretrained\\
& \quad - Configuración de transfer learning\\
& \quad - Loop de entrenamiento con early stopping\\
& \quad - Guardado de métricas y checkpoints\\
\midrule
\archivo{scripts/train\_classifier.py} & Clasificador para dataset warped\\
\midrule
\archivo{scripts/train\_classifier\_original.py} & Clasificador para dataset original\\
\midrule
\archivo{outputs/classifier\_comparison/} & Resultados comparativos:\\
& \quad - comparison\_chart.png\\
& \quad - comparison\_table.csv\\
& \quad - Checkpoints de modelos\\
\bottomrule
\end{tabular}
\end{table}

% ==============================================================================
\section{Discusión}
% ==============================================================================

\subsection{Limitaciones del Estudio}

\begin{enumerate}
    \item \textbf{Tamaño del test set}: Con solo 96 muestras, las diferencias
    de $\pm$5\% pueden no ser estadísticamente significativas

    \item \textbf{Single random seed}: Los resultados podrían variar con
    diferentes inicializaciones

    \item \textbf{Hiperparámetros fijos}: No se optimizaron hiperparámetros
    por arquitectura

    \item \textbf{Dataset específico}: Resultados pueden no generalizar
    a otros datasets de radiografía
\end{enumerate}

\subsection{Implicaciones Prácticas}

\begin{enumerate}
    \item \textbf{Para clasificación pura}: MobileNetV2 o EfficientNet-B0
    ofrecen el mejor balance rendimiento/eficiencia

    \item \textbf{Para robustez}: El dataset warped proporciona rendimiento
    más confiable, potencialmente menos dependiente de artefactos

    \item \textbf{Para interpretabilidad}: El menor gap de AlexNet sugiere
    que puede ser útil para estudios de interpretabilidad
\end{enumerate}

% ==============================================================================
\section{Conclusiones}
% ==============================================================================

\begin{enumerate}
    \item \textbf{MobileNetV2 logra el mejor rendimiento absoluto}: 98.96\%
    accuracy en el dataset original, con solo 2.2M parámetros.

    \item \textbf{El gap Original-Warped es consistente}: Promedio 4.17\%,
    std 4.1\%, indicando que el fenómeno es del dataset, no de arquitecturas.

    \item \textbf{6/7 arquitecturas muestran el mismo patrón}: Mejor
    rendimiento en original, consistente con shortcut learning.

    \item \textbf{AlexNet es la excepción}: Mejor en warped, posiblemente
    por capacidad limitada para aprender shortcuts complejos.

    \item \textbf{ResNet-18 es más sensible}: Gap del 10.42\%, sugiriendo
    mayor dependencia de características geométricas.

    \item \textbf{El warping no mejora la clasificación directa}: Pero
    proporciona un baseline más ``honesto'' del problema.

    \item \textbf{Las hipótesis fueron validadas}: El estudio confirma
    la presencia de shortcuts y cuantifica su impacto.
\end{enumerate}

\end{document}
