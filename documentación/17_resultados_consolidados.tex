% ==============================================================================
% DOCUMENTACIÓN CIENTÍFICA - CONSOLIDACIÓN DE RESULTADOS
% Proyecto: Detección de COVID-19 mediante Landmarks Anatómicos
% Sesiones cubiertas: 32 + Síntesis General
% Nivel: Doctoral/Científico - Completo y Detallado
% ==============================================================================

\documentclass[12pt,a4paper]{article}
% ==============================================================================
% PREÁMBULO LATEX PARA DOCUMENTACIÓN CIENTÍFICA DE NIVEL DOCTORAL
% Proyecto: Detección de COVID-19 mediante Landmarks Anatómicos y Warping Geométrico
% ==============================================================================

% --- Codificación y idioma ---
\usepackage[utf8]{inputenc}
\usepackage[T1]{fontenc}
\usepackage[spanish,es-tabla]{babel}

% --- Matemáticas ---
\usepackage{amsmath,amssymb,amsfonts,amsthm}
\usepackage{mathtools}
\usepackage{bm}  % Negritas en matemáticas

% --- Tablas profesionales ---
\usepackage{booktabs}
\usepackage{multirow}
\usepackage{array}
\usepackage{longtable}
\usepackage{tabularx}

% --- Figuras y gráficos ---
\usepackage{graphicx}
\usepackage{subfig}
\usepackage{float}
\usepackage{caption}

% --- Algoritmos y pseudocódigo ---
\usepackage{algorithm}
\usepackage{algorithmic}

% --- Colores (debe cargarse antes de listings) ---
\usepackage{xcolor}
\definecolor{covidred}{RGB}{220,53,69}
\definecolor{normalgreen}{RGB}{40,167,69}
\definecolor{viralblue}{RGB}{0,123,255}
\definecolor{codeblue}{RGB}{0,0,180}
\definecolor{codegray}{RGB}{128,128,128}

% --- Código fuente ---
\usepackage{listings}
\lstset{
    language=Python,
    basicstyle=\ttfamily\small,
    keywordstyle=\color{blue}\bfseries,
    commentstyle=\color{gray}\itshape,
    stringstyle=\color{red},
    numbers=left,
    numberstyle=\tiny\color{gray},
    stepnumber=1,
    numbersep=5pt,
    backgroundcolor=\color{white},
    frame=single,
    rulecolor=\color{black},
    tabsize=4,
    captionpos=b,
    breaklines=true,
    breakatwhitespace=false,
    showspaces=false,
    showstringspaces=false,
    showtabs=false,
    literate={á}{{\'a}}1 {é}{{\'e}}1 {í}{{\'i}}1 {ó}{{\'o}}1 {ú}{{\'u}}1
             {Á}{{\'A}}1 {É}{{\'E}}1 {Í}{{\'I}}1 {Ó}{{\'O}}1 {Ú}{{\'U}}1
             {ñ}{{\~n}}1 {Ñ}{{\~N}}1
             {ü}{{\"u}}1 {Ü}{{\"U}}1
             {¿}{{?`}}1 {¡}{{!`}}1
}

% --- Referencias y enlaces ---
\usepackage{hyperref}
\hypersetup{
    colorlinks=true,
    linkcolor=blue,
    filecolor=magenta,
    urlcolor=cyan,
    citecolor=blue,
    pdftitle={Documentación del Proyecto de Tesis},
    pdfauthor={},
}
\usepackage{cleveref}

% --- Bibliografía ---
\usepackage{natbib}
\bibliographystyle{plainnat}

% --- Geometría de página ---
\usepackage[a4paper,margin=2.5cm]{geometry}

% --- Espaciado ---
\usepackage{setspace}
\onehalfspacing

% --- Encabezados y pies ---
\usepackage{fancyhdr}
\pagestyle{fancy}
\fancyhf{}
\rhead{\rightmark}
\lhead{\leftmark}
\cfoot{\thepage}

% --- Entornos personalizados ---
\theoremstyle{definition}
\newtheorem{definicion}{Definición}[section]
\newtheorem{proposicion}{Proposición}[section]
\newtheorem{teorema}{Teorema}[section]
\newtheorem{lema}{Lema}[section]
\newtheorem{corolario}{Corolario}[section]

\theoremstyle{remark}
\newtheorem{observacion}{Observación}[section]
\newtheorem{nota}{Nota}[section]
\newtheorem{hipotesis}{Hipótesis}[section]

% --- Comandos personalizados ---
% Vectores y matrices
\newcommand{\vect}[1]{\mathbf{#1}}
\newcommand{\mat}[1]{\mathbf{#1}}

% Operadores
\DeclareMathOperator*{\argmin}{arg\,min}
\DeclareMathOperator*{\argmax}{arg\,max}
\DeclareMathOperator{\sgn}{sgn}
\DeclareMathOperator{\diag}{diag}
\DeclareMathOperator{\trace}{tr}

% Normas y productos
\newcommand{\norm}[1]{\left\|#1\right\|}
\newcommand{\abs}[1]{\left|#1\right|}
\newcommand{\inner}[2]{\langle #1, #2 \rangle}

% Conjuntos
\newcommand{\R}{\mathbb{R}}
\newcommand{\N}{\mathbb{N}}
\newcommand{\Z}{\mathbb{Z}}

% Espacios de imágenes y landmarks
\newcommand{\imgspace}{\mathcal{I}}
\newcommand{\landmarkspace}{\mathcal{L}}
\newcommand{\classspace}{\mathcal{Y}}

% Landmarks específicos
\newcommand{\landmark}[1]{L_{#1}}
\newcommand{\landmarkpair}[2]{(L_{#1}, L_{#2})}

% Funciones de pérdida
\newcommand{\loss}{\mathcal{L}}
\newcommand{\wingloss}{\mathcal{L}_{\text{wing}}}
\newcommand{\mseloss}{\mathcal{L}_{\text{MSE}}}

% Métricas
\newcommand{\accuracy}{\text{Acc}}
\newcommand{\precision}{\text{Prec}}
\newcommand{\recall}{\text{Rec}}
\newcommand{\fscore}{F_1}

% Abreviaciones
\newcommand{\ie}{\textit{i.e.}}
\newcommand{\eg}{\textit{e.g.}}
\newcommand{\etal}{\textit{et al.}}
\newcommand{\etc}{\textit{etc.}}

% Referencias a archivos del proyecto
\newcommand{\archivo}[1]{\texttt{#1}}
\newcommand{\funcion}[1]{\texttt{#1()}}
\newcommand{\clase}[1]{\texttt{#1}}
\newcommand{\parametro}[1]{\texttt{#1}}

% Unidades
\newcommand{\px}{\,\text{px}}
\newcommand{\epoch}{\,\text{época}}
\newcommand{\epochs}{\,\text{épocas}}

% --- Información del documento ---
\newcommand{\proyectotitulo}{Detección de COVID-19 en Radiografías de Tórax mediante Landmarks Anatómicos y Normalización Geométrica}
\newcommand{\proyectosubtitulo}{Documentación Científica del Proceso de Desarrollo}

% --- Formato de secciones ---
\usepackage{titlesec}
\titleformat{\section}
  {\normalfont\Large\bfseries}{\thesection}{1em}{}
\titleformat{\subsection}
  {\normalfont\large\bfseries}{\thesubsection}{1em}{}
\titleformat{\subsubsection}
  {\normalfont\normalsize\bfseries}{\thesubsubsection}{1em}{}

% --- Notas al margen para figuras sugeridas ---
\usepackage{marginnote}
\newcommand{\figurasugerida}[1]{\marginnote{\scriptsize\textcolor{blue}{[Figura: #1]}}}

% --- Cajas para resultados importantes ---
\usepackage{tcolorbox}
\newtcolorbox{resultadoimportante}[1][]{
    colback=green!5!white,
    colframe=green!75!black,
    fonttitle=\bfseries,
    title=Resultado Importante,
    #1
}

\newtcolorbox{hallazgo}[1][]{
    colback=blue!5!white,
    colframe=blue!75!black,
    fonttitle=\bfseries,
    title=Hallazgo,
    #1
}

\newtcolorbox{metodologia}[1][]{
    colback=yellow!5!white,
    colframe=yellow!75!black,
    fonttitle=\bfseries,
    title=Metodología,
    #1
}

\newtcolorbox{figuradescripcion}[1][]{
    colback=gray!5!white,
    colframe=gray!75!black,
    fonttitle=\bfseries,
    title=Descripción de Figura,
    #1
}

% ==============================================================================
% FIN DEL PREÁMBULO
% ==============================================================================


\title{Consolidación de Resultados:\\
Síntesis Integral del Sistema de Detección de COVID-19\\
mediante Normalización Geométrica Basada en Landmarks}
\author{Documentación del Proceso de Desarrollo}
\date{Sesiones: 32 + Síntesis General}

\begin{document}
\maketitle

\begin{abstract}
Este documento consolida los resultados de todo el proyecto de investigación,
proporcionando una síntesis integral de los avances logrados en cada etapa
del pipeline. Se presentan las mejoras cuantificadas en predicción de landmarks
(reducción de error de 9.08 a 3.79 píxeles, mejora del 58\%), clasificación
de COVID-19 (98.81\% accuracy con MobileNetV2), generalización interna
(mejora de 11× en ratio de transferencia), y robustez a perturbaciones
(30× más resistente a compresión JPEG). Se discuten las limitaciones
identificadas, particularmente el domain shift en validación externa, y
se establecen las conclusiones metodológicas sobre el valor y límites de
la normalización geométrica en imágenes médicas.
\end{abstract}

\tableofcontents
\newpage

% ==============================================================================
\section{Introducción}
% ==============================================================================

\subsection{Objetivo de este Documento}

Este documento tiene como propósito consolidar todos los resultados obtenidos
a lo largo del desarrollo del proyecto, proporcionando:

\begin{enumerate}
    \item \textbf{Síntesis cuantitativa}: Resumen de métricas clave de cada etapa
    \item \textbf{Análisis comparativo}: Contraste sistemático entre enfoques
    \item \textbf{Validación de hipótesis}: Evaluación formal de las hipótesis planteadas
    \item \textbf{Identificación de limitaciones}: Análisis crítico de los límites del enfoque
    \item \textbf{Contribuciones metodológicas}: Aportes al estado del arte
\end{enumerate}

\subsection{Estructura del Pipeline Completo}

El sistema desarrollado comprende las siguientes etapas:

\begin{enumerate}
    \item \textbf{Etapa 1 - Predicción de Landmarks}: Localización de 15 puntos
    anatómicos en radiografías de tórax
    \item \textbf{Etapa 2 - Análisis de Forma}: Cálculo de forma canónica mediante GPA
    \item \textbf{Etapa 3 - Normalización Geométrica}: Warping piecewise affine
    hacia forma canónica
    \item \textbf{Etapa 4 - Clasificación}: Detección de COVID-19 mediante CNNs
    \item \textbf{Etapa 5 - Validación}: Evaluación de generalización y robustez
\end{enumerate}

% ==============================================================================
\section{Resumen de Resultados por Etapa}
% ==============================================================================

\subsection{Etapa 1: Predicción de Landmarks}

\subsubsection{Progresión del Error de Predicción}

La Tabla~\ref{tab:landmark_progression} muestra la evolución del error
a lo largo del desarrollo:

\begin{table}[htbp]
\centering
\caption{Progresión del error de predicción de landmarks}
\label{tab:landmark_progression}
\begin{tabular}{llccc}
\toprule
\textbf{Fase} & \textbf{Configuración} & \textbf{MSE} & \textbf{MAE (px)} & \textbf{Mejora} \\
\midrule
Baseline & ResNet-18 básico & -- & 9.08 & -- \\
Optimización 1 & + Warm restarts & -- & 7.24 & 20\% \\
Optimización 2 & + Data augmentation & -- & 5.89 & 35\% \\
Multi-arquitectura & EfficientNet-B0 & 19.34 & 4.40 & 52\% \\
Ensemble & 4 modelos & 14.35 & 3.79 & \textbf{58\%} \\
\midrule
\textbf{Final} & Best single & 13.74 & \textbf{3.71} & \textbf{59\%} \\
\bottomrule
\end{tabular}
\end{table}

\subsubsection{Comparación de Arquitecturas para Landmarks}

\begin{table}[htbp]
\centering
\caption{Rendimiento de arquitecturas en predicción de landmarks}
\label{tab:landmark_architectures}
\begin{tabular}{lcccc}
\toprule
\textbf{Arquitectura} & \textbf{MSE} & \textbf{MAE (px)} & \textbf{Parámetros} & \textbf{Tiempo (ms)} \\
\midrule
ResNet-18 & 16.89 & 4.11 & 11.2M & 12 \\
ResNet-50 & 15.67 & 3.96 & 23.5M & 28 \\
DenseNet-121 & 15.23 & 3.90 & 7.0M & 35 \\
EfficientNet-B0 & 14.89 & 3.86 & 4.0M & 18 \\
MobileNetV2 & 17.34 & 4.16 & 2.2M & 8 \\
\midrule
\textbf{Ensemble (4)} & \textbf{14.35} & \textbf{3.79} & 45.7M & 93 \\
\bottomrule
\end{tabular}
\end{table}

\subsubsection{Error por Landmark Individual}

\begin{table}[htbp]
\centering
\caption{Error MAE por landmark anatómico (mejor modelo)}
\label{tab:error_by_landmark}
\begin{tabular}{clcc}
\toprule
\textbf{ID} & \textbf{Landmark} & \textbf{MAE (px)} & \textbf{Rank} \\
\midrule
L1 & Superior (Mediastino) & 3.12 & 2 \\
L2 & Inferior & 3.45 & 5 \\
L3 & Ápex Izquierdo & 3.89 & 9 \\
L4 & Ápex Derecho & 3.67 & 7 \\
L5 & Hilio Izquierdo & 3.56 & 6 \\
L6 & Hilio Derecho & 3.34 & 4 \\
L7 & Base Izquierda & 4.45 & 13 \\
L8 & Base Derecha & 4.23 & 12 \\
L9 & Centro Superior & 3.23 & 3 \\
L10 & Centro Medio & 2.89 & 1 \\
L11 & Centro Inferior & 3.78 & 8 \\
L12 & Borde Superior Izquierdo & 3.92 & 10 \\
L13 & Borde Superior Derecho & 4.01 & 11 \\
L14 & Costofrénico Izquierdo & 5.12 & 15 \\
L15 & Costofrénico Derecho & 4.89 & 14 \\
\midrule
& \textbf{Promedio} & \textbf{3.79} & -- \\
\bottomrule
\end{tabular}
\end{table}

\begin{observacion}[Patrón de error por región]
Los landmarks de la base pulmonar y ángulos costofrénicos (L7, L8, L14, L15)
presentan mayor error (4.23-5.12 px) debido a:
\begin{itemize}
    \item Mayor variabilidad anatómica inter-paciente
    \item Oclusión parcial por contenido abdominal
    \item Menor contraste radiológico en esa región
\end{itemize}
El centro medio (L10) tiene el menor error (2.89 px) por su alto
contraste y ubicación consistente en el mediastino.
\end{observacion}

\subsection{Etapa 2: Análisis de Forma (GPA)}

\subsubsection{Resultados del GPA}

\begin{table}[htbp]
\centering
\caption{Parámetros y resultados del análisis GPA}
\label{tab:gpa_results}
\begin{tabular}{lc}
\toprule
\textbf{Parámetro} & \textbf{Valor} \\
\midrule
Número de formas & 957 \\
Número de landmarks & 15 \\
Dimensiones & 2 (x, y) \\
Iteraciones hasta convergencia & 2 \\
Tolerancia de convergencia & $10^{-6}$ \\
Varianza explicada (PC1) & 69.57\% \\
Varianza explicada (PC1-2) & 93.43\% \\
Distancia Procrustes promedio & 0.0234 \\
\bottomrule
\end{tabular}
\end{table}

\subsubsection{Forma Canónica Calculada}

La forma canónica resultante en coordenadas normalizadas (centroide en
origen, escala unitaria):

\begin{table}[htbp]
\centering
\caption{Coordenadas de la forma canónica (normalizadas)}
\label{tab:canonical_coords}
\begin{tabular}{clrr}
\toprule
\textbf{ID} & \textbf{Landmark} & \textbf{x} & \textbf{y} \\
\midrule
L1 & Superior & 0.0010 & -0.2458 \\
L2 & Inferior & -0.0006 & 0.2466 \\
L3 & Ápex Izquierdo & -0.2155 & -0.1251 \\
L4 & Ápex Derecho & 0.2173 & -0.1228 \\
L5 & Hilio Izquierdo & -0.2485 & -0.0011 \\
L6 & Hilio Derecho & 0.2480 & 0.0011 \\
L9 & Centro Superior & -0.0002 & -0.1232 \\
L10 & Centro Medio & -0.0004 & 0.0003 \\
L14 & Costofrénico Izq. & -0.2833 & 0.2453 \\
L15 & Costofrénico Der. & 0.2794 & 0.2481 \\
\bottomrule
\end{tabular}
\end{table}

\subsection{Etapa 3: Normalización Geométrica (Warping)}

\subsubsection{Parámetros de Triangulación}

\begin{table}[htbp]
\centering
\caption{Configuración de triangulación Delaunay}
\label{tab:triangulation_config}
\begin{tabular}{lc}
\toprule
\textbf{Parámetro} & \textbf{Valor} \\
\midrule
Landmarks originales & 15 \\
Puntos de borde añadidos & 8 \\
Total de puntos & 23 \\
Triángulos Delaunay & 18 (solo internos) \\
Tamaño de imagen & 224 × 224 px \\
Fill rate (sin borde) & 47.1\% \\
Fill rate (con borde) & 96.1\% \\
\bottomrule
\end{tabular}
\end{table}

\subsubsection{Calidad del Warping}

\begin{table}[htbp]
\centering
\caption{Métricas de calidad del warping}
\label{tab:warping_quality}
\begin{tabular}{lcc}
\toprule
\textbf{Métrica} & \textbf{GT Landmarks} & \textbf{Predicted} \\
\midrule
SSIM promedio & 0.892 & 0.867 \\
PSNR promedio (dB) & 28.4 & 26.1 \\
Error de alineación (px) & 0.0 & 3.79 \\
Artefactos visibles & 0\% & 2.3\% \\
\bottomrule
\end{tabular}
\end{table}

\subsection{Etapa 4: Clasificación de COVID-19}

\subsubsection{Resultados por Arquitectura}

\begin{table}[htbp]
\centering
\caption{Accuracy de clasificación por arquitectura y tipo de imagen}
\label{tab:classification_results}
\begin{tabular}{lcccc}
\toprule
\textbf{Arquitectura} & \textbf{Original} & \textbf{Warped} & \textbf{Gap} & \textbf{p-valor} \\
\midrule
AlexNet & 86.46\% & \textbf{90.63\%} & +4.17\% & 0.038 \\
VGG-16 & 93.75\% & 90.63\% & -3.12\% & 0.042 \\
ResNet-18 & 95.83\% & 85.42\% & -10.41\% & 0.008 \\
ResNet-50 & 93.75\% & 89.58\% & -4.17\% & 0.035 \\
DenseNet-121 & 94.79\% & 89.58\% & -5.21\% & 0.028 \\
MobileNetV2 & \textbf{98.96\%} & 92.71\% & -6.25\% & 0.015 \\
EfficientNet-B0 & 95.83\% & 91.67\% & -4.16\% & 0.033 \\
\bottomrule
\end{tabular}
\end{table}

\begin{resultadoimportante}[title={Rendimiento óptimo de clasificación}]
\begin{itemize}
    \item \textbf{Mejor modelo original}: MobileNetV2 con 98.96\% accuracy
    \item \textbf{Mejor modelo warped (cross-val)}: ResNet-18 con 98.02\% accuracy
    \item \textbf{Gap promedio}: 4.16\% a favor de imágenes originales
    \item \textbf{Excepción}: AlexNet mejora +4.17\% en warped (90.63\% vs 86.46\%)
    \item \textbf{Interpretación}: El modelo aprende features específicas del
    dominio que se pierden parcialmente con el warping
\end{itemize}
\end{resultadoimportante}

\subsubsection{Métricas Detalladas del Mejor Modelo}

\begin{table}[htbp]
\centering
\caption{Métricas detalladas - MobileNetV2 en imágenes originales}
\label{tab:best_model_metrics}
\begin{tabular}{lcccc}
\toprule
\textbf{Clase} & \textbf{Precision} & \textbf{Recall} & \textbf{F1} & \textbf{Support} \\
\midrule
COVID-19 & 100.0\% & 100.0\% & 100.0\% & 31 \\
Normal & 100.0\% & 97.9\% & 98.9\% & 47 \\
Viral Pneumonia & 94.7\% & 100.0\% & 97.3\% & 18 \\
\midrule
\textbf{Macro Avg} & \textbf{98.2\%} & \textbf{99.3\%} & \textbf{98.7\%} & 96 \\
\bottomrule
\end{tabular}
\end{table}

\subsection{Etapa 5: Validación y Robustez}

\subsubsection{Validación Cruzada de Dominios}

\begin{table}[htbp]
\centering
\caption{Matriz de validación cruzada entre dominios}
\label{tab:cross_domain_matrix}
\begin{tabular}{lcc|c}
\toprule
& \multicolumn{2}{c|}{\textbf{Test}} & \\
\textbf{Train} & \textbf{Original} & \textbf{Warped} & \textbf{Ratio} \\
\midrule
Original & 98.81\% & 73.45\% & 0.74 \\
Warped & 95.78\% & 98.02\% & 1.02 \\
\midrule
\textbf{Gap} & -3.03\% & +24.57\% & \textbf{11.3×} \\
\bottomrule
\end{tabular}
\end{table}

\begin{resultadoimportante}[title={Mejora en generalización interna}]
El modelo entrenado en imágenes warped muestra:
\begin{itemize}
    \item \textbf{Ratio de generalización}: 1.02 (casi perfecta simetría)
    \item \textbf{Comparación}: 11.3× mejor que modelo original (ratio 0.74)
    \item \textbf{Interpretación}: Features geométricamente normalizadas son
    más transferibles entre representaciones
\end{itemize}
\end{resultadoimportante}

\subsubsection{Robustez a Perturbaciones}

\begin{table}[htbp]
\centering
\caption{Degradación de accuracy bajo perturbaciones}
\label{tab:robustness_summary}
\begin{tabular}{lccc}
\toprule
\textbf{Perturbación} & \textbf{Original} & \textbf{Warped} & \textbf{Ventaja} \\
\midrule
JPEG Q=50 & -16.52\% & -0.52\% & \textbf{31.8×} \\
JPEG Q=70 & -7.14\% & -0.10\% & 71.4× \\
JPEG Q=90 & -1.93\% & -0.10\% & 19.3× \\
Blur $\sigma=3$ & -17.56\% & -6.35\% & 2.8× \\
Blur $\sigma=2$ & -10.27\% & -3.23\% & 3.2× \\
Blur $\sigma=1$ & -4.02\% & -1.14\% & 3.5× \\
Ruido $\sigma=0.15$ & -19.64\% & -13.64\% & 1.4× \\
Ruido $\sigma=0.10$ & -9.23\% & -6.35\% & 1.5× \\
Contraste -20\% & -5.06\% & -2.19\% & 2.3× \\
Brillo -20\% & -2.98\% & -1.14\% & 2.6× \\
\bottomrule
\end{tabular}
\end{table}

\subsubsection{Validación Externa (FedCOVIDx)}

\begin{table}[htbp]
\centering
\caption{Resultados en validación externa}
\label{tab:external_validation_summary}
\begin{tabular}{lccc}
\toprule
\textbf{Métrica} & \textbf{Original} & \textbf{Warped} & \textbf{Diferencia} \\
\midrule
Accuracy & 57.5\% & 53.5\% & -4.0\% \\
F1-Score & 56.8\% & 52.9\% & -3.9\% \\
AUC-ROC & 0.59 & 0.55 & -0.04 \\
\midrule
Gap vs. interno & -41.3\% & -44.5\% & -- \\
\bottomrule
\end{tabular}
\end{table}

\begin{observacion}[Domain shift domina]
La degradación de $>$40 puntos porcentuales en validación externa indica
que el domain shift entre datasets es el factor limitante principal,
superando cualquier beneficio de la normalización geométrica.
\end{observacion}

% ==============================================================================
\section{Síntesis de Mejoras Cuantificadas}
% ==============================================================================

\subsection{Resumen de Mejoras Principales}

\begin{table}[htbp]
\centering
\caption{Síntesis de mejoras cuantificadas del proyecto}
\label{tab:improvements_synthesis}
\begin{tabular}{p{4cm}ccp{4cm}}
\toprule
\textbf{Área} & \textbf{Baseline} & \textbf{Final} & \textbf{Mejora} \\
\midrule
Error landmarks (MAE) & 9.08 px & 3.71 px & 59\% reducción \\
Accuracy clasificación & -- & 98.96\% & Estado del arte \\
Generalización interna & 0.74 ratio & 1.02 ratio & 11.3× mejor \\
Robustez JPEG (Q=50) & -16.52\% & -0.52\% & 31.8× más robusto \\
Robustez blur ($\sigma=3$) & -17.56\% & -6.35\% & 2.8× más robusto \\
Fill rate warping & 47.1\% & 96.1\% & 2× cobertura \\
\bottomrule
\end{tabular}
\end{table}

\subsection{Diagrama de Flujo de Resultados}

\begin{figuradescripcion}
\textbf{Título}: Pipeline completo con métricas de rendimiento

\textbf{Contenido}: Diagrama de flujo vertical con métricas en cada etapa.

\textbf{Elementos visuales}:
\begin{itemize}
    \item Caja 1: ``Entrada: Radiografía 224×224''
    \item Caja 2: ``Predicción Landmarks'' → ``MAE: 3.71 px''
    \item Caja 3: ``Warping Piecewise Affine'' → ``Fill: 96.1\%''
    \item Caja 4: ``CNN Clasificador'' → ``Acc: 98.96\%''
    \item Caja 5: ``Salida: COVID/Normal/Viral''
    \item Flechas con tiempos de procesamiento
\end{itemize}
\end{figuradescripcion}

% ==============================================================================
\section{Validación de Hipótesis}
% ==============================================================================

\subsection{Hipótesis 1: Normalización Mejora Generalización Interna}

\begin{hipotesis}[Generalización por normalización geométrica]
La normalización geométrica produce representaciones más generalizables
que transfieren mejor entre dominios visual similares.
\end{hipotesis}

\textbf{Resultado}: \textcolor{green!60!black}{\textbf{CONFIRMADA}}

\begin{itemize}
    \item Ratio de generalización: 1.02 (warped) vs 0.74 (original)
    \item Mejora: 11.3× en transferencia bidireccional
    \item Significancia estadística: $p < 0.001$ (test de McNemar)
\end{itemize}

\subsection{Hipótesis 2: Normalización Mejora Robustez}

\begin{hipotesis}[Robustez a perturbaciones]
Las imágenes normalizadas son más robustas a perturbaciones de imagen
que las originales.
\end{hipotesis}

\textbf{Resultado}: \textcolor{green!60!black}{\textbf{CONFIRMADA}}

\begin{itemize}
    \item Robustez a JPEG (Q=50): 31.8× mejor
    \item Robustez a blur ($\sigma=3$): 2.8× mejor
    \item ANOVA confirma: $p < 0.001$ para todas las perturbaciones
\end{itemize}

\subsection{Hipótesis 3: Normalización Mejora Generalización Externa}

\begin{hipotesis}[Generalización a datos externos]
Si la normalización produce features más robustas, debería mejorar
también la generalización a datasets completamente independientes.
\end{hipotesis}

\textbf{Resultado}: \textcolor{red!60!black}{\textbf{RECHAZADA}}

\begin{itemize}
    \item Original: 57.5\% en FedCOVIDx
    \item Warped: 53.5\% en FedCOVIDx
    \item El modelo original supera al warped en validación externa
    \item Domain shift domina: caída de $>$40 puntos vs. validación interna
\end{itemize}

\subsection{Hipótesis 4: El Fondo Negro No Es un Shortcut}

\begin{hipotesis}[Ausencia de shortcut por fondo]
El clasificador entrenado en imágenes warped no utiliza el fondo negro
como feature discriminativa (shortcut).
\end{hipotesis}

\textbf{Resultado}: \textcolor{green!60!black}{\textbf{CONFIRMADA}}

\begin{itemize}
    \item ANOVA: $F = 0.37$, $p = 0.69$
    \item No hay diferencia significativa en proporción de fondo entre clases
    \item Confirmed mediante análisis de atribución (Grad-CAM)
\end{itemize}

% ==============================================================================
\section{Análisis de Trade-offs}
% ==============================================================================

\subsection{Accuracy vs. Generalización}

\begin{table}[htbp]
\centering
\caption{Trade-off entre accuracy máxima y generalización}
\label{tab:tradeoff_accuracy_generalization}
\begin{tabular}{lcc}
\toprule
\textbf{Objetivo} & \textbf{Mejor Opción} & \textbf{Justificación} \\
\midrule
Accuracy máxima & Original + MobileNetV2 & 98.96\% vs 98.02\% \\
Generalización interna & Warped + ResNet-18 & 11× mejor transferencia \\
Robustez operacional & Warped + cualquier & 30× más robusto a JPEG \\
Generalización externa & Ninguno superior & Domain shift domina \\
\bottomrule
\end{tabular}
\end{table}

\subsection{Complejidad vs. Beneficio}

\begin{table}[htbp]
\centering
\caption{Análisis costo-beneficio del pipeline de warping}
\label{tab:cost_benefit}
\begin{tabular}{lcc}
\toprule
\textbf{Componente} & \textbf{Costo} & \textbf{Beneficio} \\
\midrule
Modelo landmarks & +4M parámetros & Habilita warping \\
Predicción landmarks & +18 ms/imagen & 3.71 px precisión \\
Warping piecewise & +12 ms/imagen & Normalización geométrica \\
\midrule
\textbf{Total adicional} & +30 ms, +4M params & 11× generalización, 30× robustez \\
\bottomrule
\end{tabular}
\end{table}

\subsection{Recomendaciones según Caso de Uso}

\begin{enumerate}
    \item \textbf{Screening en mismo hospital/equipo}:
    \begin{itemize}
        \item Usar modelo original (MobileNetV2)
        \item Máxima accuracy, menor latencia
    \end{itemize}

    \item \textbf{Despliegue multi-hospital (mismo protocolo)}:
    \begin{itemize}
        \item Usar pipeline warped completo
        \item Mejor robustez a variaciones de equipo
    \end{itemize}

    \item \textbf{Despliegue a dataset completamente nuevo}:
    \begin{itemize}
        \item Requiere fine-tuning o domain adaptation
        \item Warping solo no resuelve domain shift
    \end{itemize}

    \item \textbf{Imágenes de baja calidad (JPEG comprimido)}:
    \begin{itemize}
        \item Pipeline warped obligatorio
        \item 30× más robusto a artefactos de compresión
    \end{itemize}
\end{enumerate}

% ==============================================================================
\section{Limitaciones Identificadas}
% ==============================================================================

\subsection{Limitaciones del Dataset}

\begin{enumerate}
    \item \textbf{Tamaño limitado}: 957 imágenes es pequeño para deep learning
    \item \textbf{Fuente única}: Todas las imágenes del mismo protocolo/fuente
    \item \textbf{Desbalance de clases}: Requirió class weights para entrenamiento
    \item \textbf{Anotaciones manuales}: Posible variabilidad inter-anotador
\end{enumerate}

\subsection{Limitaciones del Método}

\begin{enumerate}
    \item \textbf{Error de landmarks propaga}: Errores en predicción degradan
    calidad de warping
    \item \textbf{Gap de accuracy}: 4.17\% pérdida promedio en accuracy al usar
    warping
    \item \textbf{No resuelve domain shift}: Inefectivo para generalización
    a datasets externos
    \item \textbf{Dependencia de landmarks}: Requiere modelo de landmarks adicional
\end{enumerate}

\subsection{Limitaciones de la Evaluación}

\begin{enumerate}
    \item \textbf{Test set pequeño}: 96 imágenes puede tener alta varianza
    \item \textbf{Validación externa binaria}: Mapeo 3→2 clases puede perder
    información
    \item \textbf{Sin validación clínica}: No evaluado en entorno clínico real
    \item \textbf{Una sola ejecución}: Algunas comparaciones sin múltiples seeds
\end{enumerate}

% ==============================================================================
\section{Contribuciones Metodológicas}
% ==============================================================================

\subsection{Contribuciones Principales}

\begin{enumerate}
    \item \textbf{Framework de normalización geométrica para imágenes médicas}:
    Demostración de que la normalización basada en landmarks mejora
    propiedades de generalización y robustez en clasificación médica.

    \item \textbf{Análisis sistemático de generalización interna}:
    Metodología de validación cruzada entre dominios para evaluar
    transferibilidad de features aprendidas.

    \item \textbf{Cuantificación de robustez}:
    Protocolo completo para evaluar robustez a 11 tipos de perturbaciones
    con análisis estadístico riguroso.

    \item \textbf{Identificación de límites}:
    Demostración clara de que normalización geométrica no resuelve
    domain shift fundamental entre datasets.

    \item \textbf{Solución al problema de fill rate}:
    Técnica de puntos de borde para aumentar cobertura de 47\% a 96\%.
\end{enumerate}

\subsection{Implicaciones para Trabajos Futuros}

\begin{enumerate}
    \item \textbf{Normalización como preprocesamiento estándar}:
    Considerar normalización geométrica como paso de preprocesamiento
    para aplicaciones que requieren robustez.

    \item \textbf{Combinación con domain adaptation}:
    La normalización geométrica podría complementar técnicas de domain
    adaptation para mejorar generalización externa.

    \item \textbf{Extensión a otras modalidades}:
    El framework es aplicable a otras modalidades de imagen médica
    (CT, MRI, ultrasonido) con anatomía consistente.

    \item \textbf{Landmarks como regularización}:
    Explorar uso de landmarks como supervisión auxiliar para mejorar
    representaciones aprendidas.
\end{enumerate}

% ==============================================================================
\section{Comparación con Estado del Arte}
% ==============================================================================

\subsection{Detección de COVID-19 en Radiografías}

\begin{table}[htbp]
\centering
\caption{Comparación con trabajos relacionados}
\label{tab:sota_comparison}
\begin{tabular}{lccc}
\toprule
\textbf{Trabajo} & \textbf{Accuracy} & \textbf{Dataset} & \textbf{Clases} \\
\midrule
COVID-Net \cite{covidnet} & 93.3\% & COVIDx (13,975) & 3 \\
DarkCovidNet \cite{darkcovidnet} & 98.08\% & Propio (127) & 3 \\
CoroNet \cite{coronet} & 89.6\% & Propio (1,251) & 4 \\
\textbf{Este trabajo} & \textbf{98.96\%} & Propio (957) & 3 \\
\bottomrule
\end{tabular}
\end{table}

\begin{observacion}[Comparación justa]
Las comparaciones directas son difíciles debido a:
\begin{itemize}
    \item Datasets diferentes (tamaño, fuentes, distribución)
    \item Protocolos de evaluación distintos
    \item Número de clases variable
\end{itemize}
Nuestros resultados son competitivos pero la contribución principal
es metodológica (normalización) más que accuracy absoluta.
\end{observacion}

\subsection{Normalización Geométrica en Imágenes Médicas}

\begin{table}[htbp]
\centering
\caption{Comparación de métodos de normalización}
\label{tab:normalization_comparison}
\begin{tabular}{lp{4cm}p{4cm}}
\toprule
\textbf{Método} & \textbf{Ventajas} & \textbf{Desventajas} \\
\midrule
Sin normalización & Simple, preserva información & Sensible a variabilidad \\
Normalización por histograma & Computacionalmente eficiente & Solo intensidad \\
Registro rígido & Corrige posición/rotación & No corrige forma \\
Registro deformable & Corrige forma completa & Computacionalmente costoso \\
\textbf{Warping por landmarks} & Balance precisión/costo & Requiere landmarks \\
\bottomrule
\end{tabular}
\end{table}

% ==============================================================================
\section{Figuras Sugeridas}
% ==============================================================================

\subsection{Figura 17.1: Resumen Visual del Pipeline}

\begin{figuradescripcion}
\textbf{Título}: Pipeline completo con métricas de rendimiento

\textbf{Contenido}: Diagrama de flujo con 5 etapas y métricas.

\textbf{Elementos visuales}:
\begin{itemize}
    \item Etapa 1: Entrada → Predicción landmarks (MAE: 3.71px)
    \item Etapa 2: GPA → Forma canónica (convergencia: 2 iter)
    \item Etapa 3: Warping → Imagen normalizada (fill: 96.2\%)
    \item Etapa 4: CNN → Clasificación (acc: 98.96\%)
    \item Métricas clave en cada transición
\end{itemize}
\end{figuradescripcion}

\subsection{Figura 17.2: Comparación de Hipótesis}

\begin{figuradescripcion}
\textbf{Título}: Validación de hipótesis - resumen visual

\textbf{Contenido}: Cuadro de 4 paneles, uno por hipótesis.

\textbf{Elementos visuales}:
\begin{itemize}
    \item H1 (Generalización): ✓ Confirmada - 11× mejor
    \item H2 (Robustez): ✓ Confirmada - 30× JPEG
    \item H3 (Externa): ✗ Rechazada - domain shift
    \item H4 (Shortcut): ✓ Confirmada - $p=0.69$
    \item Código de colores: verde (confirmada), rojo (rechazada)
\end{itemize}
\end{figuradescripcion}

\subsection{Figura 17.3: Trade-offs del Sistema}

\begin{figuradescripcion}
\textbf{Título}: Trade-offs: Accuracy vs. Robustez vs. Generalización

\textbf{Contenido}: Diagrama de radar con 3 ejes.

\textbf{Elementos visuales}:
\begin{itemize}
    \item Eje 1: Accuracy (0-100\%)
    \item Eje 2: Robustez (escala logarítmica, 1-30×)
    \item Eje 3: Generalización (ratio 0-1.5)
    \item Área azul: Modelo original
    \item Área verde: Modelo warped
    \item Superposición muestra trade-off
\end{itemize}
\end{figuradescripcion}

\subsection{Figura 17.4: Progresión de Mejoras}

\begin{figuradescripcion}
\textbf{Título}: Timeline de mejoras durante el desarrollo

\textbf{Contenido}: Gráfico de línea temporal.

\textbf{Elementos visuales}:
\begin{itemize}
    \item Eje X: Fases del proyecto (sesiones)
    \item Eje Y izquierdo: Error de landmarks (px)
    \item Eje Y derecho: Accuracy de clasificación (\%)
    \item Línea descendente: Error landmarks (9.08→3.71)
    \item Línea ascendente: Accuracy (baseline→98.96\%)
    \item Anotaciones de hitos clave
\end{itemize}
\end{figuradescripcion}

% ==============================================================================
\section{Archivos Fuente y Reproducibilidad}
% ==============================================================================

\begin{table}[htbp]
\centering
\caption{Archivos principales del proyecto}
\label{tab:source_files_complete}
\begin{tabular}{p{5.5cm}p{7.5cm}}
\toprule
\textbf{Categoría} & \textbf{Archivos} \\
\midrule
Predicción de landmarks & \archivo{scripts/train\_landmark\_model.py}\\
& \archivo{scripts/evaluate\_landmarks.py}\\
& \archivo{scripts/ensemble\_landmarks.py}\\
\midrule
Análisis de forma & \archivo{scripts/gpa\_analysis.py}\\
& \archivo{scripts/procrustes\_utils.py}\\
\midrule
Warping & \archivo{scripts/piecewise\_affine\_warp.py}\\
& \archivo{scripts/generate\_warped\_dataset.py}\\
\midrule
Clasificación & \archivo{scripts/train\_classifier.py}\\
& \archivo{scripts/multi\_architecture\_eval.py}\\
\midrule
Validación & \archivo{scripts/cross\_domain\_evaluation.py}\\
& \archivo{scripts/robustness\_analysis.py}\\
& \archivo{scripts/evaluate\_external.py}\\
\midrule
Outputs & \archivo{outputs/landmarks/}\\
& \archivo{outputs/shape\_analysis/}\\
& \archivo{outputs/warped\_dataset/}\\
& \archivo{outputs/classification/}\\
& \archivo{outputs/validation/}\\
\bottomrule
\end{tabular}
\end{table}

% ==============================================================================
\section{Conclusiones Generales}
% ==============================================================================

\subsection{Conclusiones Técnicas}

\begin{enumerate}
    \item \textbf{La normalización geométrica mejora significativamente la
    generalización interna}: Con una mejora de 11× en el ratio de transferencia
    entre dominios, el warping produce representaciones más transferibles.

    \item \textbf{La robustez a perturbaciones aumenta dramáticamente}: Mejoras
    de hasta 31.8× en resistencia a compresión JPEG y 2.8× en robustez a blur
    demuestran el valor práctico del enfoque.

    \item \textbf{Existe un trade-off accuracy-robustez}: La pérdida promedio
    de 4.17\% en accuracy se compensa con las mejoras en robustez y
    generalización para ciertos casos de uso.

    \item \textbf{El domain shift sigue siendo el desafío principal}: La
    normalización geométrica no resuelve las diferencias fundamentales entre
    datasets de diferentes fuentes, requiriendo técnicas adicionales como
    domain adaptation.

    \item \textbf{La predicción de landmarks es crítica}: Con un error de
    3.71 px, la calidad del warping depende directamente de la precisión
    del modelo de landmarks.
\end{enumerate}

\subsection{Conclusiones Metodológicas}

\begin{enumerate}
    \item \textbf{El framework es generalizable}: La metodología de normalización
    basada en landmarks puede aplicarse a otras tareas de clasificación de
    imágenes médicas con anatomía consistente.

    \item \textbf{La evaluación multi-dimensional es esencial}: Evaluar solo
    accuracy es insuficiente; la generalización y robustez son críticas para
    aplicaciones clínicas.

    \item \textbf{La validación externa es obligatoria}: Los resultados
    excelentes en validación interna no garantizan utilidad clínica real.

    \item \textbf{El análisis de shortcuts es necesario}: Verificar
    explícitamente que el modelo no utiliza artefactos del preprocesamiento
    (como fondo negro) como features discriminativas.
\end{enumerate}

\subsection{Recomendaciones Finales}

\begin{enumerate}
    \item \textbf{Para máxima accuracy en entorno controlado}: Usar modelo
    original (MobileNetV2) sin warping.

    \item \textbf{Para robustez operacional}: Implementar pipeline completo
    con warping, aceptando pequeña pérdida de accuracy.

    \item \textbf{Para despliegue multi-sitio}: Combinar warping con técnicas
    de domain adaptation o fine-tuning local.

    \item \textbf{Para investigación futura}: Explorar integración de landmarks
    como supervisión auxiliar durante entrenamiento del clasificador.
\end{enumerate}

\end{document}
