\section{Bloques de RF para el transmisor FM-UWB en CppSim}

La transmisión de datos digitales se basan en las modulación digitales, donde las más usadas son la modulación QPSK, BFSK y QAM, debido a su fácil integración y diseño en tecnología CMOS, el esquema de modulación elegido es el \textbf{BFSK} por el diseño del sub-carrier. El modelo propuesto se observa en la figura \ref{fig:003:002:001}.


\begin{figure}[h!]
    \centering
    \includegraphics[width=15cm]{chapters/3-chapter/figuras/transmisor_fm_bloques.png}
    \caption{Bloque propuesto para el transmisor FM-UWB.}
    \label{fig:003:002:001}
\end{figure}

La configuración con la que se desea trabajar es el siguiente:

\begin{enumerate}
    \item La tasa de transmisión (baude raute en inglés) de la entrada de datos digitales es de \SI{125}{\kilo\bit/\second}.
    \item El \textit{sub-carrier} opera como un modulador \textbf{BFSK}, donde la frecuencia de operación son: \SI{2}{\mega \hertz} cuando recibe un '1' lógico y 1 MHz cuando se recibe un '0' lógico.
    \item Las características del VCO son los siguientes:
    \begin{itemize}
        \item Frecuencia central ($F_{VCO}$) de \SI{4}{\giga \hertz}.
        \item Ganancia de frecuencia ($K_{VCO}$) de \SI{2}{\giga \hertz}.
    \end{itemize}
    \item El amplificador de potencia debe tener voltaje pico-pico de salida máximo de \SI{160}{\milli \volt}, esto es para garantizar las características que se define la UWB.
\end{enumerate}

Una vez cumplido estás características se tiene el modelo comportamental del transmisor.

\subsection{Modelo comportamental del \textit{Sub-carrier}}

Un \textit{sub-carrier} es un modulador digital cuya característica principal es convertir los n datos digitales en una onda triangular de n frecuencias, por lo tanto, de acuerdo a su tipo de implementación se puede realizar todos los tipos de modulación digital, sin embargo tiene sus desventajas. La primer desventaja es el método más común de diseño: sintetizador digital directo (o por sus siglas en inglés DDS).

La estructura básica de un DDS incluye un registro de fase acumulativa, una tabla de onda (almacenada en una ROM), y un convertidor digital-analógico (DAC). La frecuencia de salida de un DDS está determinada por el valor en el acumulador de fase, que se incrementa con cada ciclo de reloj. Al controlar la tasa de incremento del acumulador, el DDS genera señales de frecuencia ajustable en tiempo real. Esta capacidad hace que los DDS sean populares en aplicaciones de RF, comunicaciones, y pruebas electrónicas, donde se requieren señales de frecuencia ajustable y precisión \cite{Libro_DDS}. No obstante, la frecuencia máxima que genera la onda el DDS depende de la frecuencia de reloj, por ende, la frecuencia de un DDS no pasa de los \SI{5}{\mega \hertz}. Existe otra técnica de diseño \textit{sub-carrier} y es a través de compradores Schmitt-trigger, técnica utilizada en este trabajo de tesis que se mencionará en el capítulo 4.

CppSim permite simular ondas triangulares con su descripción matemática, la ecuación de una onda triangular se observa en \ref{eq:003:002:001}.

\begin{equation}
    v_{tri}(t) = \frac{4A}{T} \left| t - \frac{T}{2} \left( 2 \left\lfloor \frac{t}{T} + \frac{1}{2} \right\rfloor + 1 \right) \right| - A
    \label{eq:003:002:001}
\end{equation}

Donde: 

\begin{enumerate}
    \item $A$ es la amplitud máxima.
    \item $T$ es el periodo de muestreo de la onda, se define como $T = \frac{1}{f}$, donde $f$ es la frecuencia.
    \item $t$ es el tiempo.
\end{enumerate}

Los resultados de la simulación se observa en la figura \ref{fig:003:002:002}.

\begin{figure}[h!]
    \centering
    \includegraphics[width=12cm]{chapters/3-chapter/figuras/onda_triangular.jpg}
    \caption{Ondas triangulares de \SI{1}{\mega \hertz} y \SI{2}{\mega \hertz} en CPPSIM.}
    \label{fig:003:002:002}
\end{figure}

\subsection{Modelo comportamental del Oscilador Controlado por Voltaje}

Un Oscilador Controlado por Voltaje (o por sus siglas en inglés VCO)  es un dispositivo electrónico que genera una señal periódica cuya frecuencia es controlada por un voltaje de entrada. En otras palabras, la frecuencia de oscilación de un VCO varía en función del voltaje aplicado, permitiendo ajustar la frecuencia de salida de forma continua y precisa \cite{Libro_Razavi_Rf}.

Los VCOs son ampliamente utilizados en sistemas de comunicación y en aplicaciones de control, como en bucles de enganche de fase (Phase-Locked Loops, PLLs), donde se sincronizan señales de alta frecuencia. También son esenciales en la modulación de frecuencia y de fase en sistemas de transmisión de datos.

La relación matemática se describe en la ecuación:

\begin{equation}
    v_{VCO} (t) = A\text{ }cos(w_{out}t)
\end{equation}

Donde:

\begin{itemize}
    \item $w_{out} = K_{VCO}V_{cont}+w_0$ , se denomina $K_{VCO}$ como la ganancia del VCO (sus unidades son rad/s/V o \SI{}{\hertz/\volt}) y $w_0$ es la frecuencia central en la que opera el VCO.
\end{itemize}

En la figura \ref{fig:003:002:003} se observa el VCO con $V_{cont} =$ \SI{0.5}{\volt}, por lo tanto la salida de frecuencia es \SI{5}{\giga\hertz}. 

\begin{figure}[h!]
    \centering
    \includegraphics[width=12cm]{chapters/3-chapter/figuras/vco_5GHz.jpg}
    \caption{VCO a \SI{5}{\giga\hertz}.}
    \label{fig:003:002:003}
\end{figure}

\subsection{Modelo comportamental del Amplificador de Potencia}

La implementación de un amplificador de potencia (o por sus siglas en inglés PA) depende de dos factores, la carga (en sistemas RF la carga $Z$ se establece como \SI{50}{\ohm}) y la eficiencia energética \cite{Libro_Razavi_Rf}, en el caso de simulación en CPPSIM solo basta con definir la salida de voltaje que entregará.

Para la obtención de la densidad de potencia espectral (PSD) se calcula con la ecuación \ref{eq:003:002:003}

\begin{eqnarray*}
P(\text{\SI{}{\watt/\hertz}}) &=& P(\SI{}{\volt^2/\hertz})/Z \\
P(\text{\SI{}{\decibel/\hertz}}) &=& 10\log_{10}\left(P(\text{\SI{}{\watt/\hertz}})\right) \\
P(\text{\SI{}{\decibel\milli/\hertz}}) &=& 10\log_{10} (P(\text{\SI{}{\decibel/\hertz}})) + 30 \\
P(\text{\SI{}{\decibel\milli/\mega\hertz}}) &=& 10\log_{10} (P(\text{\SI{}{\decibel\milli/\hertz}})) + 60 
\end{eqnarray*}

\begin{equation}
    P(\text{\SI{}{\decibel\milli/\mega\hertz}}) = 10\log_{10} (P(\text{\SI{}{\decibel\milli/\hertz}})/Z)) + 90 
    \label{eq:003:002:003}
\end{equation}

Y la potencia de una señal se calcula con la ecuación \ref{eq:003:002:004}

\begin{equation}
    P(f) =  \dfrac{\left\vert X(f)^2\right\vert}{N} 
    \label{eq:003:002:004}
\end{equation}

Donde:
   \begin{itemize}
        \item $\left\vert X(f)^2\right\vert$ es el valor absoluto de los coeficientes de Fourier.
        \item $N$ es el número de muestras de la señal.
        \item Sus unidades normalmente son \SI{}{\volt^2/\hertz} o \SI{}{\ampere^2/\hertz}.
    \end{itemize}

    Para calcular $X(f)$ se realiza el siguiente calculo:

    \begin{equation}
        X(f) = \text{FFT}(x(n))
        \label{eq:003:002:005}
    \end{equation}
    
    Donde:
    \begin{itemize}
        \item FFT es la transformada rápida de Fourier.
        \item $x(n)$ es la señal discreta de la señal $x(t)$ en intervalos $n$.
    \end{itemize}

Para cumplir con las característica de UWB el voltaje pico-pico ($V_{pp}$) para el PA no debe exceder los \SI{160}{\milli\volt}, como se observa en la figura \ref{fig:003:002:004}.

\begin{figure}[h!]
    \centering
    \includegraphics[width=12cm]{chapters/3-chapter/figuras/pa_160.jpg}
    \caption{Señal generada por el PA a una frecuencia \SI{5}{\giga\hertz}.}
    \label{fig:003:002:004}
\end{figure}

\section{Simulación del transmisor FM-UWB en CppSim}


La simulación de la etapa del \textit{sub-carrier} con una señal de entrada digital de \SI{125}{\kbps} se presenta en la Figura \ref{fig:003:003:001}. En esta figura, se observan claramente los cambios de frecuencia asociados a los valores lógicos de la señal digital de entrada, lo cual demuestra cómo la modulación de frecuencia responde a los niveles lógicos. Los resultados permiten visualizar el comportamiento del \textit{sub-carrier} en función de los cambios de estado digital, confirmando su capacidad para modular la señal según los valores binarios a \SI{125}{\kbps}.

\begin{figure}[h!]
    \centering
    \includegraphics[width=16cm]{chapters/3-chapter/figuras/onda_subcarrier.jpg}
    \caption{Señal digital y señal triangular generada por el \textit{sub-carrier} simulados a \SI{100}{\micro\second}.}
    \label{fig:003:003:001}
\end{figure}

En la Figura \ref{fig:003:003:002} se muestra la señal con el aumento de potencia dirigido hacia la antena, donde, para facilitar su visualización, la simulación se limitó al intervalo de \SI{5}{\nano\second}. La principal ventaja de utilizar una onda senoidal es que su amplitud permanece constante, mientras que la frecuencia varía en el tiempo. Esta variación de frecuencia se debe a la forma de onda triangular del \textit{sub-carrier}, la cual controla el VCO. El comportamiento de la frecuencia en función del tiempo permite observar cómo la modulación en frecuencia se ajusta de acuerdo con la señal del \textit{sub-carrier}, asegurando la transmisión adecuada a través de la antena.

\begin{figure}[h!]
    \centering
    \includegraphics[width=16cm]{chapters/3-chapter/figuras/pa_5ns.jpg}
    \caption{Señal del PA simulada a \SI{5}{\nano\second}.}
    \label{fig:003:003:002}
\end{figure}


La PSD de la señal obtenida a través del PA se muestra en la Figura \ref{fig:003:003:003}. Al analizar el espectro electromagnético, se observa que cumple con las normas establecidas por la FCC en el año 2002, las cuales especifican que la PSD no debe exceder el límite de \SI{-41.3}{\decibel\milli/\mega\hertz} en aplicaciones de UWB. 

El análisis revela que el espectro de la señal se encuentra entre los 4 GHz y los 6 GHz, proporcionando un ancho de banda de \SI{2}{\giga\hertz}, que satisface el mínimo requerido de \SI{500}{\mega\hertz} para la tecnología UWB. Esto asegura que la señal cumpla tanto con el límite de PSD como con el ancho de banda necesario para las aplicaciones UWB, garantizando así la compatibilidad y el uso seguro en el espectro, sin interferencias con otros sistemas vecinos.
 
\begin{figure}[h!]
    \centering
    \includegraphics[width=16cm]{chapters/3-chapter/figuras/psd_final.jpg}
    \caption{PSD de la señal transmitida.}
    \label{fig:003:003:003}
\end{figure}

