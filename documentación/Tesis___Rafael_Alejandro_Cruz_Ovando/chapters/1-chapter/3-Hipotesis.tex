\section{Hipótesis}
\label{sec:hipotesis}

La hipótesis central de esta investigación se formula de la siguiente manera:

\begin{quote}
\textit{La incorporación explícita de restricciones geométricas anatómicas (simetría bilateral y preservación de distancias) como componentes de la función de pérdida en redes neuronales convolucionales mejorará significativamente la precisión de detección de \textit{landmarks} anatómicos en radiografías de tórax, comparado con funciones de pérdida estándar basadas en Error Cuadrático Medio (Mean Squared Error, MSE), alcanzando niveles de excelencia clínica (error radial medio $<$8.5 píxeles) establecidos en la literatura internacional \cite{Payer2016}.}
\end{quote}

Esta hipótesis se sustenta en el principio fundamental de que el conocimiento del dominio anatómico, cuando se integra adecuadamente en los mecanismos de aprendizaje profundo mediante regularizaciones geométricas, supera en efectividad al incremento de complejidad arquitectural en tareas especializadas con datos limitados \cite{Donner2013, Thaler2021, Zeng2020}. Las restricciones geométricas actúan como sesgos inductivos que guían el proceso de optimización hacia soluciones anatómicamente plausibles, reduciendo el espacio de búsqueda y mejorando la generalización del modelo.

\subsection{Formulación Formal}

\textbf{Hipótesis nula ($H_0$):} La adición de restricciones geométricas (simetría bilateral y preservación de distancias anatómicas) a la función de pérdida no produce mejora estadísticamente significativa en la precisión de detección de \textit{landmarks}, comparado con el uso exclusivo de funciones de pérdida estándar para regresión de coordenadas.

\textbf{Hipótesis alternativa ($H_1$):} La incorporación de restricciones geométricas en la función de pérdida produce una reducción estadísticamente significativa del error radial medio de localización de \textit{landmarks}, permitiendo alcanzar el umbral de excelencia clínica de $<$8.5 píxeles.

\subsection{Variables y Predicciones Específicas}

\textbf{Variables independientes:}
\begin{itemize}
    \item Función de pérdida empleada: (1) MSE como línea base (\textit{baseline}), (2) Wing Loss \cite{Feng2018}, (3) Wing Loss + Symmetry Loss \cite{Donner2013}, (4) Wing Loss + Symmetry Loss + Distance Preservation Loss \cite{Thaler2021}.
    \item Estrategia de entrenamiento: congelamiento de columna vertebral de la red (en adelante referido como \textit{backbone}) vs. ajuste fino (\textit{fine-tuning}) completo \cite{Yosinski2014}.
    \item Arquitectura base: ResNet-18 con y sin mecanismos de atención espacial \cite{He2016, Hou2021}.
\end{itemize}

\textbf{Variables dependientes (métricas de evaluación):}
\begin{itemize}
    \item Error radial medio (píxeles) entre \textit{landmarks} predichos y anotaciones de referencia.
    \item Consistencia bilateral: error de simetría entre \textit{landmarks} correspondientes de pulmones izquierdo y derecho.
    \item Validez anatómica: desviación porcentual en distancias anatómicas críticas (ancho torácico, altura mediastínica).
    \item Porcentaje de predicciones con excelencia clínica (error $<$8.5 píxeles).
    \item Porcentaje de predicciones clínicamente útiles (error $<$15 píxeles).
\end{itemize}

\textbf{Predicciones específicas testeables:}

\begin{enumerate}
    \item La función de pérdida Wing Loss reducirá el error radial medio en al menos 10\% comparado con MSE \textit{baseline}, debido a su diseño específico para amplificar gradientes en el régimen de errores pequeños, permitiendo refinamiento iterativo de predicciones \cite{Feng2018}.

    \item La adición de Symmetry Loss mejorará la consistencia bilateral en al menos 15\%, reduciendo la asimetría artificial entre \textit{landmarks} correspondientes de ambos pulmones y forzando al modelo a respetar la simetría anatómica inherente \cite{Donner2013}.

    \item La incorporación de Distance Preservation Loss reducirá la desviación en distancias anatómicas críticas en al menos 20\%, garantizando que el modelo preserve relaciones espaciales fundamentales del tórax \cite{Thaler2021}.

    \item La función de pérdida completa (Wing Loss + Symmetry Loss + Distance Preservation Loss) logrará error radial medio $<$8.5 píxeles, cumpliendo con el estándar internacional de excelencia clínica para detección de \textit{landmarks} anatómicos \cite{Payer2016}.

    \item El modelo entrenado con restricciones geométricas demostrará robustez superior ante variabilidad patológica (COVID-19, neumonía viral), con degradación de desempeño $<$15\% comparado con casos normales \cite{Jacobi2020}.

    \item Los mecanismos de atención arquitecturales (Coordinate Attention) no proporcionarán mejora significativa comparados con restricciones geométricas en la función de pérdida, validando que el conocimiento del dominio es más efectivo que la complejidad arquitectural en tareas especializadas con datos limitados \cite{Hou2021}.
\end{enumerate}

\subsection{Alcance de la Validación Experimental}

La validación de esta hipótesis se realizará mediante:

\begin{itemize}
    \item \textbf{Diseño experimental controlado:} Entrenamiento de modelos con configuraciones sistemáticamente variadas (funciones de pérdida, estrategias de optimización, componentes arquitecturales) sobre el mismo conjunto de datos de 956 radiografías dividido consistentemente en conjuntos de entrenamiento (70\%), validación (15\%) y prueba (15\%).

    \item \textbf{Validación cruzada estratificada:} División que preserva distribución de categorías médicas (COVID-19, neumonía viral, normal) para garantizar representatividad.

    \item \textbf{Análisis estadístico riguroso:} Pruebas de significancia estadística (t-test pareado, ANOVA) para comparar desempeño entre configuraciones, con nivel de significancia $\alpha = 0.05$.

    \item \textbf{Evaluación multi-dimensional:} Métricas complementarias (error radial, consistencia bilateral, validez anatómica) que evalúan diferentes aspectos de la calidad de predicción.

    \item \textbf{Análisis de ablación:} Remoción sistemática de componentes (restricciones geométricas, \textit{fine-tuning}, aprendizaje por transferencia) para cuantificar la contribución individual de cada elemento metodológico.
\end{itemize}

La confirmación de esta hipótesis demostrará que la integración de conocimiento anatómico mediante restricciones geométricas constituye una estrategia efectiva y generalizable para mejorar la precisión de modelos de aprendizaje profundo en tareas de localización anatómica, estableciendo una metodología reproducible aplicable a otros problemas de análisis de imágenes médicas \cite{Litjens2017}.
