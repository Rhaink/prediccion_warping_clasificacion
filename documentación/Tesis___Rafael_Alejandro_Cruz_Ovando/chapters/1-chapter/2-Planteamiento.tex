\section{Planteamiento del problema}

La interpretación de radiografías de tórax representa uno de los procedimientos de diagnóstico más frecuentes en la práctica clínica a nivel mundial, con más de 2 mil millones de estudios realizados anualmente \cite{WHO2020}. La localización precisa de estructuras anatómicas clave mediante la identificación de \textit{landmarks} es fundamental para el análisis cuantitativo y la toma de decisiones clínicas \cite{Tang2019}. Estos \textit{landmarks} anatómicos permiten el cálculo de índices diagnósticos como el índice cardiotorácico, la detección de asimetrías patológicas y el establecimiento de sistemas de coordenadas consistentes para análisis longitudinales \cite{Sogancioglu2021}. Sin embargo, la anotación manual de \textit{landmarks} requiere aproximadamente 15 minutos por imagen y está sujeta a variabilidad inter e intra-observador de hasta 5-10 píxeles, limitando su aplicabilidad en escenarios clínicos de alto volumen \cite{Payer2016}.

Los enfoques tradicionales para la detección de \textit{landmarks} anatómicos se basan en métodos de visión por computadora que utilizan características diseñadas manualmente (\textit{hand-crafted features}) combinadas con modelos estadísticos de forma \cite{Cootes1995, Cootes2001}. Aunque estos métodos han demostrado efectividad en condiciones controladas, enfrentan limitaciones significativas: (1) requieren ingeniería manual de características específicas del dominio, proceso que resulta costoso y poco generalizable; (2) dependen de alineamientos geométricos previos (como el Análisis de Procrustes Generalizado, Generalized Procrustes Analysis, GPA) que pueden fallar ante deformaciones anatómicas severas; (3) modelan relaciones lineales mediante Análisis de Componentes Principales (Principal Component Analysis, PCA), incapaces de capturar la naturaleza no lineal de las variaciones anatómicas; y (4) presentan sensibilidad elevada a condiciones de imagen como bajo contraste, ruido y artefactos \cite{Shen2017, Heimann2009}.

El surgimiento del aprendizaje profundo (\textit{deep learning}) ha transformado radicalmente el análisis de imágenes médicas \cite{Litjens2017}, demostrando capacidad para aprender representaciones jerárquicas de características directamente desde datos sin necesidad de ingeniería manual \cite{Krizhevsky2012}. Las redes neuronales convolucionales (Convolutional Neural Networks, CNNs) han alcanzado niveles de desempeño comparables o superiores al de especialistas humanos en diversas tareas de imagenología médica \cite{Esteva2017, Gulshan2016}. Sin embargo, la detección precisa de landmarks anatómicos mediante CNNs presenta desafíos específicos que requieren soluciones especializadas más allá de las arquitecturas estándar de clasificación o segmentación.

El problema central que aborda esta investigación se formula de la siguiente manera: \textbf{¿Cómo diseñar un sistema automatizado basado en redes neuronales convolucionales que detecte \textit{landmarks} anatómicos en radiografías de tórax con precisión clínicamente útil (error $<$8.5 píxeles), incorporando conocimiento anatómico del dominio médico y resultando computacionalmente eficiente para integración en flujos de trabajo hospitalarios?}

Este problema general se descompone en los siguientes desafíos técnicos específicos:

\textbf{Desafío 1: Alta precisión en regresión de coordenadas.} A diferencia de tareas de clasificación donde pequeños errores son tolerables, la localización de \textit{landmarks} constituye un problema de regresión donde el modelo debe predecir coordenadas continuas $(x, y)$ con alta precisión para resultar clínicamente útil. Los estándares internacionales establecen que un error inferior a 8.5 píxeles representa excelencia clínica \cite{Payer2016}. Las funciones de pérdida estándar como el Error Cuadrático Medio (Mean Squared Error, MSE) tratan todos los errores de manera uniforme, penalizando excesivamente valores atípicos (\textit{outliers}) pero proporcionando gradientes insuficientes para refinar predicciones ya cercanas al objetivo. Este comportamiento dificulta el logro de la alta precisión requerida en aplicaciones clínicas \cite{Feng2018}.

\textbf{Desafío 2: Incorporación de conocimiento anatómico.} El cuerpo humano exhibe restricciones geométricas inherentes que no son explotadas por enfoques estándar de aprendizaje profundo. Específicamente, las radiografías de tórax presentan simetría bilateral aproximada entre pulmones izquierdo y derecho, relaciones de distancia fijas entre estructuras anatómicas (ancho torácico, altura mediastínica), y restricciones de ordenamiento espacial (los ápices pulmonares siempre se localizan superiormente a las bases) \cite{Donner2013}. Integrar explícitamente este conocimiento anatómico como restricciones geométricas en el proceso de optimización constituye un desafío metodológico no resuelto completamente en la literatura existente \cite{Thaler2021, Zeng2020}.

\textbf{Desafío 3: Generalización ante variabilidad patológica.} El sistema debe mantener precisión robusta en presencia de condiciones patológicas que alteran significativamente la apariencia radiográfica. Las manifestaciones de COVID-19 (opacidades en vidrio esmerilado, consolidaciones), neumonía viral (infiltrados intersticiales) y otras patologías torácicas pueden oscurecer parcialmente referencias anatómicas, reduciendo el contraste local y dificultando la localización precisa de \textit{landmarks} \cite{Jacobi2020, Wang2020COVID}. El modelo debe aprender representaciones suficientemente robustas para localizar estructuras anatómicas incluso cuando los límites no resultan claramente visibles.

\textbf{Desafío 4: Eficiencia computacional para despliegue clínico.} Para resultar práctico en entornos hospitalarios, el sistema debe ejecutar inferencia en equipo físico (\textit{hardware}) de consumo general (sin requerir GPUs de alta gama) en tiempos de respuesta cercanos al tiempo real (menos de 1 segundo por imagen). Esta restricción limita la complejidad arquitectural viable y motiva el uso de modelos eficientes con \textit{transfer learning} desde dominios de datos abundantes \cite{Raghu2019}.

\textbf{Desafío 5: Escasez de datos médicos etiquetados.} A diferencia de aplicaciones de visión por computadora en dominios generales donde existen millones de imágenes etiquetadas (ImageNet: 1.2M imágenes), los conjuntos de datos (\textit{datasets}) médicos típicamente contienen cientos o pocos miles de imágenes anotadas debido al costo y tiempo requerido para anotación experta \cite{Ker2018}. Esta escasez de datos incrementa el riesgo de sobreajuste (\textit{overfitting}) y limita la capacidad de generalización de modelos entrenados desde cero, motivando estrategias de \textit{transfer learning} y regularización especializada \cite{Tajbakhsh2016}.

La solución a estos desafíos interconectados requiere una aproximación metodológica que integre: (1) arquitecturas de redes neuronales eficientes con capacidad de extracción de características robustas, (2) funciones de pérdida especializadas diseñadas para alta precisión mediante amplificación de gradientes en el régimen de errores pequeños, (3) mecanismos de regularización geométrica que incorporen conocimiento anatómico del dominio médico, y (4) estrategias de optimización progresiva que balanceen convergencia rápida con precisión final. El Capítulo 3 presenta en detalle la metodología propuesta que aborda sistemáticamente cada uno de estos desafíos.
