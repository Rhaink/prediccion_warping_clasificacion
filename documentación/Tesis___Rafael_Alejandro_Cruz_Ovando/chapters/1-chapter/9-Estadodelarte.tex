\section{Estado del Arte}
En los últimos años, la detección automática de enfermedades pulmonares, como la neumonía y el COVID-19, a partir de imágenes de tórax ha constituido un área de investigación activa y de gran importancia clínica. Los avances en visión por computadora y aprendizaje automático han permitido el desarrollo de sistemas cada vez más precisos y eficientes para este propósito. A continuación, se presenta una revisión del estado del arte en esta área, abarcando los enfoques más recientes y relevantes según las publicaciones científicas.

\subsection{Métodos existentes para la detección de neumonía y COVID-19 en radiografías de tórax}
Los enfoques para la detección automática de neumonía y COVID-19 en radiografías de tórax pueden dividirse en dos categorías principales: técnicas de procesamiento de imágenes tradicional y métodos basados en aprendizaje profundo. Entre los enfoques tradicionales, Bharati et al. \cite{bharati2020hybrid} propusieron un método híbrido que combina la transformada de wavelet discreta (DWT) para la extracción de características y una máquina de soporte vectorial (SVM) para la clasificación, logrando una precisión del 96.39% en la detección de neumonía. Narin et al. \cite{narin2020automatic} utilizaron la transformada de coseno discreta (DCT) y compararon diferentes clasificadores, alcanzando una precisión del 98% con ResNet50 en la detección de COVID-19.

Por otro lado, los métodos basados en aprendizaje profundo han demostrado un rendimiento excepcional. Wang et al. \cite{wang2020covid} propusieron COVID-Net, una red neuronal convolucional (CNN) diseñada específicamente para la detección de COVID-19, logrando una precisión del 92.4%. Ozturk et al. \cite{ozturk2020automated} desarrollaron DarkCovidNet, basada en la arquitectura DarkNet, obteniendo una precisión del 98.08% para la clasificación binaria y del 87.02% para la clasificación multiclase. Rajpurkar et al. \cite{rajpurkar2017chexnet} introdujeron CheXNet, una CNN basada en DenseNet121, para la detección de neumonía, demostrando un rendimiento comparable al de radiólogos expertos.

\subsection{Técnicas de segmentación y normalización de la región pulmonar}
La segmentación y normalización de la región pulmonar son pasos cruciales en el análisis de imágenes de tórax. Los métodos de segmentación pulmonar incluyen técnicas basadas en umbralización \cite{saad2014lung}, regiones \cite{dai2017region}, contornos activos \cite{xu2012lung} y aprendizaje profundo \cite{ait2018lung}. Para la normalización de la región pulmonar, los modelos deformables han demostrado ser efectivos. Van Ginneken et al. \cite{van2006segmentation} utilizaron modelos de forma activa (ASM) para segmentar y normalizar los pulmones en radiografías de tórax, mientras que Shi et al. \cite{shi2008hierarchical} propusieron un modelo de apariencia activa (AAM) jerárquico para imágenes de TAC. Recientemente, Zheng et al. \cite{zheng2020shape} desarrollaron un modelo de forma pulmonar basado en una red generativa adversaria (GAN) para normalizar la región pulmonar en radiografías de tórax.

\subsection{Extracción y selección de características discriminantes}
La extracción y selección de características discriminantes es fundamental para representar de manera compacta y relevante la información contenida en las imágenes de tórax. Las técnicas de análisis de textura incluyen matrices de co-ocurrencia de niveles de gris (GLCM) \cite{haralick1973textural}, patrones binarios locales (LBP) \cite{ojala2002multiresolution} y transformadas de wavelet \cite{bharati2020hybrid}. Para el análisis de bordes y formas, se han empleado descriptores de Fourier \cite{karargyris2016automated}, momentos de Hu \cite{xu2014texture} y patrones locales binarios codificados radialmente (RLBP) \cite{xu2014texture}. Además, técnicas de reducción de dimensionalidad y selección de características supervisadas, como el análisis de componentes principales (PCA) \cite{jolliffe2016principal}, el análisis discriminante lineal (LDA) \cite{guyon2003introduction} y la selección basada en información mutua \cite{guyon2003introduction}, han demostrado ser efectivas para obtener un conjunto compacto y discriminante de características.

\subsection{Clasificadores utilizados en la detección automática de enfermedades pulmonares}
Los clasificadores más relevantes y recientes utilizados en la detección automática de enfermedades pulmonares incluyen redes neuronales convolucionales (CNN), máquinas de soporte vectorial (SVM) y ensambles de clasificadores. Wang et al. \cite{wang2020covid} propusieron COVID-Net, una CNN diseñada específicamente para la detección de COVID-19, logrando una precisión del 93.3\%\. Zhang et al. \cite{zhang2020clinically} presentaron DeCoVNet, que incorpora módulos de atención y una función de pérdida ponderada, alcanzando una precisión del 95.7\%\. Liu et al. \cite{erdaw2021} utilizaron una SVM con kernels de base radial (RBF) para clasificar radiografías de tórax en clases de neumonía y no neumonía, logrando una precisión del 96.2\%\. Rajaraman et al. \cite{erdaw2021} propusieron un ensemble de CNN y árboles de decisión para la detección de neumonía, alcanzando una precisión del 97.8\%\.

\subsection{Métricas de evaluación y comparación de rendimiento}
La evaluación adecuada del rendimiento de los sistemas de detección de enfermedades pulmonares es esencial para comparar diferentes enfoques. Las métricas comunes incluyen la precisión \cite{goyal2021}, la sensibilidad \cite{shi2021review}, la especificidad \cite{goyal2021} y el F1-score \cite{goyal2021}. Las curvas ROC y el área bajo la curva (AUC) también son ampliamente utilizadas \cite{shi2021review}, \cite{dong2021role}. Además, la validación cruzada y las pruebas en conjuntos de datos externos son esenciales para evaluar la capacidad de generalización de los modelos \cite{litjens2017survey}, \cite{dong2021role}.
