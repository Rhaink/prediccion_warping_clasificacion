\chapter{Marco Teórico y Estado del Arte}

La detección automática de puntos de referencia anatómicos (\textit{landmarks}) en radiografías de tórax constituye un problema fundamental en el análisis computarizado de imágenes médicas. Como se estableció en el Capítulo 1, la localización precisa de estos puntos es esencial para la cuantificación de estructuras anatómicas, el cálculo de índices diagnósticos como el índice cardiotorácico, y la normalización espacial de radiografías para sistemas de clasificación automática. El presente capítulo tiene como objetivo establecer los fundamentos teóricos que sustentan el desarrollo de métodos basados en aprendizaje profundo para la detección automática de \textit{landmarks} anatómicos, revisando tanto los principios fundamentales como el estado del arte actual en esta área de investigación.

La detección de \textit{landmarks} en imágenes médicas ha experimentado una evolución significativa en las últimas dos décadas. Tradicionalmente, este problema se abordó mediante métodos estadísticos y geométricos, tales como los Modelos Activos de Forma (\textit{Active Shape Models}, ASM) \cite{Cootes1995} y los Modelos Activos de Apariencia (\textit{Active Appearance Models}, AAM) \cite{Cootes2001}, que representan variaciones anatómicas mediante descomposición lineal basada en Análisis de Componentes Principales. Si bien estos métodos clásicos demostraron utilidad en escenarios controlados, presentan limitaciones fundamentales relacionadas con la linealidad de sus representaciones y su dependencia de características diseñadas manualmente (\textit{hand-crafted features}) \cite{Heimann2009}. La irrupción del aprendizaje profundo en visión por computadora, particularmente tras el trabajo seminal de Krizhevsky et al. \cite{Krizhevsky2012}, ha revolucionado el análisis de imágenes médicas al permitir el aprendizaje automático de representaciones jerárquicas directamente desde los datos \cite{Litjens2017, Shen2017}. En el contexto específico de la detección de \textit{landmarks}, las Redes Neuronales Convolucionales (CNNs) han demostrado capacidad superior para capturar patrones complejos y no lineales en estructuras anatómicas, superando consistentemente el desempeño de métodos tradicionales.

El presente capítulo se estructura en ocho secciones que abarcan desde los fundamentos físicos de las radiografías de tórax hasta el estado del arte en métodos basados en aprendizaje profundo. La Sección~2.1 introduce los principios físicos de las radiografías torácicas y define los quince \textit{landmarks} anatómicos relevantes para este trabajo. La Sección~2.2 establece los fundamentos matemáticos de las redes neuronales convolucionales, incluyendo la operación de convolución, funciones de activación, y el algoritmo de retropropagación (\textit{backpropagation}). La Sección~2.3 analiza en detalle las arquitecturas residuales, particularmente la familia ResNet, que han demostrado ser especialmente efectivas para el entrenamiento de redes profundas mediante el uso de conexiones residuales (\textit{skip connections}). La Sección~2.4 examina el paradigma de aprendizaje por transferencia (\textit{transfer learning}), un componente crucial cuando se trabaja con conjuntos de datos médicos de tamaño limitado. La Sección~2.5 presenta una revisión exhaustiva de funciones de pérdida especializadas para la regresión de coordenadas, con énfasis particular en \textit{Wing Loss} y funciones de pérdida basadas en restricciones geométricas. La Sección~2.6 contrasta los enfoques de regresión directa de coordenadas versus regresión de mapas de calor (\textit{heatmap regression}), justificando la elección metodológica adoptada en esta tesis. La Sección~2.7 ofrece un análisis comparativo exhaustivo del estado del arte en detección de \textit{landmarks} anatómicos, identificando las brechas que motivan el presente trabajo. Finalmente, la Sección~2.8 sintetiza los conceptos presentados y establece la conexión con la metodología propuesta que se desarrollará en el Capítulo 3.
