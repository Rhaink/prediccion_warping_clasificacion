\section{Enfoques de Regresión para Detección de Landmarks}

Las funciones de pérdida presentadas en la Sección~2.5 permiten el entrenamiento supervisado de redes neuronales profundas para la tarea de detección de \textit{landmarks} anatómicos. Sin embargo, la arquitectura de salida de la red y la representación de las predicciones constituyen decisiones fundamentales que determinan la eficiencia computacional, la precisión sub-píxel, y la robustez del modelo. Existen dos paradigmas principales para la predicción de localizaciones de \textit{landmarks}: la regresión directa de coordenadas (del inglés, \textit{coordinate regression}), que predice directamente las coordenadas $(x, y)$ de cada punto como valores continuos, y la regresión de mapas de calor (del inglés, \textit{heatmap regression}), que genera mapas de probabilidad espacial bidimensionales que representan la localización de cada \textit{landmark}. Esta sección presenta el análisis matemático y arquitectónico de ambos enfoques, sus ventajas y limitaciones, y proporciona la justificación técnica para la selección del enfoque de regresión directa de coordenadas en el contexto de detección de \textit{landmarks} en radiografías de tórax.

\subsection{Regresión Directa de Coordenadas}

El enfoque de regresión directa de coordenadas formula la detección de \textit{landmarks} como un problema de regresión multi-salida donde la red neuronal aprende un mapeo directo desde la imagen de entrada hasta las coordenadas de todos los \textit{landmarks} \cite{Sun2013, Zhang2014}. Formalmente, dada una imagen $I \in \mathbb{R}^{H \times W}$ (donde $H$ y $W$ representan altura y ancho en píxeles), y un conjunto de $K$ \textit{landmarks}, el objetivo es aprender una función parametrizada:
\begin{equation}
f_\theta: \mathbb{R}^{H \times W} \to \mathbb{R}^{2K}
\label{eq:coordinate_regression}
\end{equation}
donde $\theta$ representa los parámetros de la red, y la salida es un vector de $2K$ valores continuos que representan las coordenadas $\hat{p} = [\hat{x}_1, \hat{y}_1, \hat{x}_2, \hat{y}_2, \ldots, \hat{x}_K, \hat{y}_K]^T$.

La arquitectura típica para regresión de coordenadas consta de tres componentes principales: (1) una red neuronal convolucional profunda que actúa como extractor de características (por ejemplo, ResNet-18, como se describió en la Sección~2.3), que transforma la imagen de entrada en mapas de características de alta dimensionalidad; (2) una capa de \textit{global average pooling} (GAP) que reduce cada canal de características espaciales ($h \times w$) a un valor escalar mediante promediación, generando un vector de características de dimensión fija independiente de la resolución espacial de entrada; y (3) una o dos capas completamente conectadas que mapean el vector de características a las $2K$ coordenadas de salida.

Matemáticamente, si $\phi(I; \theta_{\text{conv}}) \in \mathbb{R}^{C \times h \times w}$ representa los mapas de características generados por la red convolucional (con $C$ canales), el \textit{global average pooling} calcula:
\begin{equation}
z_c = \frac{1}{h \cdot w} \sum_{i=1}^{h} \sum_{j=1}^{w} \phi_c(I)_{i,j}, \quad c = 1, \ldots, C
\label{eq:gap}
\end{equation}
generando un vector de características $z \in \mathbb{R}^{C}$. Las capas completamente conectadas finales realizan la transformación afín:
\begin{equation}
\hat{p} = W_{\text{FC}} z + b_{\text{FC}}
\label{eq:fc_output}
\end{equation}
donde $W_{\text{FC}} \in \mathbb{R}^{2K \times C}$ y $b_{\text{FC}} \in \mathbb{R}^{2K}$ son parámetros aprendibles. Durante el entrenamiento, estos parámetros se optimizan mediante minimización de funciones de pérdida basadas en distancias euclidianas, como se discutió en la Sección~2.5.

\textbf{Ventajas de la regresión directa de coordenadas:}

\textbf{1. Eficiencia de memoria y computacional:} La salida de la red es un vector compacto de $2K$ valores, en contraste con representaciones espacialmente extensas. Para $K = 15$ \textit{landmarks}, la salida es un vector de 30 valores escalares. Esta compacidad reduce significativamente los requerimientos de memoria GPU durante el entrenamiento y la inferencia, permitiendo tamaños de lote (\textit{batch size}) más grandes y convergencia más rápida.

\textbf{2. Precisión sub-píxel inherente:} Las coordenadas se predicen como valores continuos en el espacio real $\mathbb{R}^2$, proporcionando capacidad intrínseca para localización sub-píxel sin necesidad de técnicas de refinamiento adicionales. Esta propiedad es crítica en aplicaciones médicas donde errores de fracción de píxel pueden tener relevancia diagnóstica.

\textbf{3. Arquitectura simple y estándar:} El enfoque de regresión directa es compatible con arquitecturas de clasificación estándar (como ResNet) mediante el simple reemplazo de la capa completamente conectada final, facilitando la utilización de modelos pre-entrenados en ImageNet mediante \textit{transfer learning}, como se discutió en la Sección~2.4.

\textbf{4. Reducción de hiperparámetros:} A diferencia del enfoque de mapas de calor, no requiere la selección de parámetros relacionados con la representación espacial de las predicciones (como el ancho de las Gaussianas o la resolución de salida).

\textbf{Limitaciones:}

\textbf{1. Pérdida de información espacial explícita:} El \textit{global average pooling} colapsa completamente la estructura espacial de los mapas de características. Esto puede dificultar el aprendizaje de relaciones espaciales complejas entre \textit{landmarks}, aunque este efecto puede mitigarse mediante la incorporación de restricciones geométricas explícitas en la función de pérdida, como las restricciones de simetría y preservación de distancia discutidas en la Sección~2.5.

\textbf{2. Sensibilidad a oclusiones y ambigüedad:} En presencia de oclusiones parciales o artefactos de imagen, la red debe producir una única predicción de coordenada, sin capacidad de representar incertidumbre espacial distribuida.

\subsection{Regresión de Mapas de Calor}

El enfoque de regresión de mapas de calor representa cada \textit{landmark} mediante un mapa de probabilidad espacial bidimensional que indica la probabilidad de presencia del punto en cada localización de la imagen \cite{Tompson2014, Newell2016}. Para cada \textit{landmark} $k$, la red genera un mapa de calor $H_k \in \mathbb{R}^{h \times w}$, donde $H_k(i, j) \in [0, 1]$ representa la probabilidad de que el \textit{landmark} $k$ esté localizado en la posición $(i, j)$.

Durante el entrenamiento, los mapas de calor objetivo (\textit{ground truth}) se construyen típicamente como Gaussianas bidimensionales centradas en las coordenadas anotadas $(x_k, y_k)$:
\begin{equation}
H_k^{\text{gt}}(i, j) = \exp\left(-\frac{(i - y_k)^2 + (j - x_k)^2}{2\sigma^2}\right)
\label{eq:heatmap_gaussian}
\end{equation}
donde $\sigma$ controla el ancho de la Gaussiana. La red aprende a predecir estos mapas de calor mediante minimización de funciones de pérdida como el error cuadrático medio píxel-a-píxel o entropía cruzada binaria:
\begin{equation}
\mathcal{L}_{\text{heatmap}} = \frac{1}{K \cdot h \cdot w} \sum_{k=1}^{K} \sum_{i=1}^{h} \sum_{j=1}^{w} \left(H_k(i, j) - H_k^{\text{gt}}(i, j)\right)^2
\label{eq:heatmap_loss}
\end{equation}

Las arquitecturas típicas para regresión de mapas de calor emplean diseños codificador-decodificador que preservan o reconstruyen resoluciones espaciales altas. Las arquitecturas \textit{U-Net} \cite{Ronneberger2015} y \textit{Stacked Hourglass Networks} \cite{Newell2016} son ejemplos representativos ampliamente utilizados en tareas de estimación de pose humana y análisis de imágenes médicas.

Durante la inferencia, las coordenadas de los \textit{landmarks} se extraen de los mapas de calor predichos mediante dos estrategias principales: (1) \textit{Hard argmax}, que selecciona la posición del píxel con valor máximo $\arg\max_{i,j} H_k(i, j)$, limitando la precisión a la resolución de píxel, o (2) \textit{Soft argmax} diferenciable \cite{Payer2016}, que calcula el centro de masa ponderado del mapa de calor:
\begin{equation}
\hat{x}_k = \sum_{i=1}^{h} \sum_{j=1}^{w} j \cdot \text{softmax}(H_k(i, j)), \quad \hat{y}_k = \sum_{i=1}^{h} \sum_{j=1}^{w} i \cdot \text{softmax}(H_k(i, j))
\label{eq:soft_argmax}
\end{equation}
proporcionando capacidad de localización sub-píxel y diferenciabilidad completa para entrenamiento de extremo a extremo.

\textbf{Ventajas de la regresión de mapas de calor:}

\textbf{1. Información espacial explícita:} Los mapas de calor preservan la estructura espacial bidimensional de las predicciones, facilitando el aprendizaje de relaciones espaciales complejas y contexto anatómico.

\textbf{2. Robustez a oclusiones y ambigüedad:} El enfoque puede representar distribuciones de probabilidad multimodales o difusas, capturando incertidumbre en la localización de \textit{landmarks} parcialmente ocluidos o ambiguos.

\textbf{3. Supervisión densa:} La función de pérdida proporciona señales de gradiente en todas las localizaciones espaciales, potencialmente facilitando la convergencia del entrenamiento.

\textbf{Limitaciones:}

\textbf{1. Costo computacional y de memoria:} La generación de $K$ mapas de calor de resolución $h \times w$ requiere memoria proporcional a $K \cdot h \cdot w$, que puede ser substancialmente mayor que el vector de $2K$ coordenadas. Para $K = 15$ \textit{landmarks} y resolución de salida $64 \times 64$, se requiere almacenar 61,440 valores en comparación con 30 valores del enfoque de coordenadas directas.

\textbf{2. Velocidad de entrenamiento reducida:} Las arquitecturas codificador-decodificador con conexiones de salto y múltiples etapas de deconvolución son computacionalmente más costosas que las arquitecturas estándar con \textit{global pooling}.

\textbf{3. Hiperparámetros adicionales:} El ancho de la Gaussiana $\sigma$ en la Ecuación~\ref{eq:heatmap_gaussian} es un hiperparámetro crítico que debe seleccionarse cuidadosamente: valores pequeños proporcionan supervisión más precisa pero pueden dificultar la convergencia, mientras que valores grandes facilitan el aprendizaje pero reducen la precisión de localización.

\subsection{Comparación y Selección de Enfoque}

La Tabla~\ref{tab:coordinate_vs_heatmap} presenta una comparación sistemática de los aspectos técnicos y prácticos de ambos enfoques.

\begin{table}[h]
\centering
\caption{Comparación de enfoques de regresión directa de coordenadas y mapas de calor para detección de \textit{landmarks}.}
\label{tab:coordinate_vs_heatmap}
\small
\begin{tabular}{|l|c|c|}
\hline
\textbf{Aspecto} & \textbf{Regresión de Coordenadas} & \textbf{Regresión de Mapas de Calor} \\
\hline
Memoria (salida) & $2K$ valores & $K \cdot h \cdot w$ valores \\
\hline
Precisión sub-píxel & Inherente (continuo) & Requiere soft-argmax \\
\hline
Arquitectura & ResNet + GAP + FC & U-Net / Hourglass \\
\hline
Velocidad entrenamiento & Rápida & Moderada-Lenta \\
\hline
Información espacial & Implícita (colapsada) & Explícita (preservada) \\
\hline
Robustez a oclusiones & Limitada & Alta \\
\hline
Hiperparámetros & Mínimos & $\sigma$ (ancho Gaussiana) \\
\hline
Transfer learning & Directo (ResNet) & Requiere adaptación \\
\hline
Restricciones geométricas & Fácil (en espacio de coord.) & Complejo (en heatmaps) \\
\hline
\end{tabular}
\end{table}

Investigaciones recientes han explorado enfoques híbridos que combinan ambos paradigmas. Li et al.~\cite{Li2023} propusieron una arquitectura multi-tarea que predice simultáneamente mapas de calor y coordenadas, demostrando mejoras en precisión de localización en análisis de radiografías de tórax mediante la combinación de supervisión espacialmente densa y predicción directa. Jeong et al.~\cite{Jeong2023} presentaron un enfoque de regresión de coordenadas guiado por atención en mapas de características espaciales, preservando parcialmente información espacial sin la sobrecarga completa de generación de mapas de calor. Adicionalmente, enfoques basados en \textit{transformers} de visión \cite{Li2022CVPR} han demostrado capacidad para capturar relaciones espaciales de largo alcance sin la necesidad de representaciones espaciales explícitas, representando una dirección prometedora para futuras investigaciones.

\textbf{Justificación de la selección de regresión directa de coordenadas:}

En el contexto específico de detección de 15 \textit{landmarks} anatómicos en radiografías de tórax con el conjunto de datos del presente trabajo (956 imágenes de entrenamiento, como se establecerá en el Capítulo 3), el enfoque de regresión directa de coordenadas se selecciona por las siguientes razones técnicas:

\textbf{1. Eficiencia computacional en régimen de datos moderados:} Con un conjunto de datos de tamaño moderado, la eficiencia de entrenamiento y la capacidad de utilizar arquitecturas estándar pre-entrenadas en ImageNet (como se discutió en la Sección~2.4) proporcionan ventajas significativas. La regresión directa de coordenadas permite \textit{fine-tuning} directo de ResNet-18 pre-entrenado, aprovechando conocimiento transferido sin modificaciones arquitectónicas substanciales.

\textbf{2. Precisión sub-píxel natural:} La aplicación clínica requiere localización precisa de estructuras anatómicas, y el enfoque de coordenadas continuas proporciona capacidad sub-píxel inherente sin técnicas de refinamiento adicionales.

\textbf{3. Restricciones de hardware:} Las restricciones de memoria GPU (8 GB VRAM en la infraestructura utilizada, como se describirá en el Capítulo 3) favorecen el enfoque de coordenadas compactas, permitiendo tamaños de lote más grandes que aceleran la convergencia y mejoran la estimación de estadísticas de normalización por lotes.

\textbf{4. Incorporación de conocimiento anatómico mediante restricciones geométricas:} La limitación principal del enfoque de coordenadas (pérdida de información espacial explícita) se mitiga mediante la incorporación de las funciones de pérdida de simetría bilateral y preservación de distancias anatómicas presentadas en la Sección~2.5. Estas restricciones geométricas imponen explícitamente conocimiento anatómico sobre la geometría del tórax, compensando la falta de supervisión espacialmente densa de los mapas de calor.

\textbf{5. Reducción de hiperparámetros:} La eliminación del hiperparámetro $\sigma$ de ancho de Gaussiana reduce el espacio de búsqueda de hiperparámetros, simplificando el proceso de validación experimental que se presentará en el Capítulo 4.

La combinación de regresión directa de coordenadas con la función de pérdida compuesta $\mathcal{L}_{\text{total}} = \lambda_1 \mathcal{L}_{\text{Wing}} + \lambda_2 \mathcal{L}_{\text{sym}} + \lambda_3 \mathcal{L}_{\text{dist}}$ (Ecuación~2.31, Sección~2.5) constituye el enfoque metodológico adoptado en este trabajo, balanceando eficiencia computacional, precisión de localización, y aprovechamiento de conocimiento anatómico estructurado. El estado del arte de métodos de detección de \textit{landmarks} en imágenes médicas, incluyendo enfoques basados en coordenadas, mapas de calor, y técnicas híbridas, se analiza exhaustivamente en la sección subsecuente.
