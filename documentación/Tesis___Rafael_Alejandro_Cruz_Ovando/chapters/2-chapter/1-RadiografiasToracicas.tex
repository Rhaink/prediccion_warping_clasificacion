\section{Radiografías de Tórax en Diagnóstico Médico}

Las radiografías de tórax constituyen el estudio de imagen médica más frecuentemente utilizado en la práctica clínica. Como se estableció en el Capítulo 1, este método diagnóstico es fundamental para la evaluación de patologías pulmonares, cardiovasculares y mediastinales. La interpretación radiológica se fundamenta en la identificación de estructuras anatómicas de referencia (\textit{landmarks}) y en la evaluación de sus relaciones espaciales. En el contexto del análisis computarizado de imágenes médicas, la detección automática de estos puntos anatómicos representa una tarea fundamental para sistemas de diagnóstico asistido por computadora y para la cuantificación objetiva de hallazgos radiológicos.

\subsection{Principios Físicos de la Radiografía Torácica}

La formación de imágenes radiográficas se basa en la interacción de fotones de rayos X con tejidos biológicos, específicamente mediante los fenómenos de dispersión Compton y absorción fotoeléctrica. Los sistemas de radiografía torácica convencionales operan con voltajes de aceleración de 110-120 kVp para proyecciones posteroanterior (PA), generando radiación electromagnética con energías en el rango de 40 a 150 keV \cite{Bushberg2020}.

La atenuación del haz de rayos X al atravesar tejido biológico se describe mediante la Ley de Beer-Lambert:
\begin{equation}
I(x) = I_0 \exp\left(-\int_0^x \mu(s) \, ds\right)
\label{eq:beer_lambert}
\end{equation}
donde $I(x)$ representa la intensidad del haz transmitido después de atravesar un espesor $x$ de tejido, $I_0$ es la intensidad del haz incidente, y $\mu(s)$ es el coeficiente de atenuación lineal en función de la posición. Para tejidos homogéneos con coeficiente de atenuación constante, la ecuación se simplifica a:
\begin{equation}
I = I_0 e^{-\mu x}
\label{eq:beer_lambert_simple}
\end{equation}

El contraste radiográfico resulta de las diferencias en los coeficientes de atenuación entre los tejidos que componen la anatomía torácica. Los campos pulmonares, compuestos predominantemente por aire alveolar ($\mu \approx 0.0001$ cm$^{-1}$), presentan baja atenuación y aparecen radiolúcidos (oscuros) en las radiografías. Por el contrario, las estructuras mediastinales y la silueta cardíaca, constituidas por tejidos blandos con coeficientes de atenuación superiores ($\mu \approx 0.20$ cm$^{-1}$), presentan mayor radioopacidad (tonos claros). Las estructuras óseas de la caja torácica (costillas, clavículas, columna vertebral) exhiben la mayor atenuación ($\mu \approx 0.50$ cm$^{-1}$ para hueso cortical) \cite{Bushberg2020}. Esta diferenciación inherente de densidades radiográficas entre estructuras anatómicas adyacentes genera los bordes y contornos que definen los \textit{landmarks} anatómicos de interés para este trabajo.

\subsection{Anatomía Torácica y Definición de Landmarks Anatómicos}

La anatomía torácica en proyección posteroanterior comprende tres compartimentos principales: los campos pulmonares bilaterales, el mediastino central, y la caja torácica ósea \cite{Webb2015, Hansell2008}. La interpretación sistemática de radiografías de tórax requiere la identificación de estructuras anatómicas de referencia cuya localización precisa permite la evaluación de normalidad anatómica y la detección de alteraciones patológicas.

Los \textit{landmarks} anatómicos se definen como puntos de referencia específicos que corresponden a estructuras anatómicas con significado clínico establecido y criterios de identificación reproducibles entre observadores expertos. A diferencia de regiones de interés arbitrarias, los \textit{landmarks} representan localizaciones anatómicas con propiedades geométricas consistentes que pueden explotarse mediante restricciones geométricas en algoritmos de detección automática \cite{Li2022}.

El presente trabajo aborda la detección automática de 15 \textit{landmarks} anatómicos distribuidos en las estructuras pulmonares, mediastinales y óseas de la radiografía de tórax. Como se ilustra en la Figura~\ref{fig:landmarks_anotados}, estos puntos de referencia se seleccionaron considerando tres criterios fundamentales: (1) detectabilidad visual consistente en radiografías de calidad diagnóstica, (2) relevancia anatómica para la caracterización de la geometría torácica, y (3) distribución espacial que captura la estructura global del tórax. La Tabla~\ref{tab:landmarks_descripcion} presenta la nomenclatura y localización anatómica de cada punto.

\begin{table}[h]
\centering
\caption{Descripción de los 15 \textit{landmarks} anatómicos en radiografías de tórax}
\label{tab:landmarks_descripcion}
\small
\begin{tabular}{|c|p{5.5cm}|p{7.5cm}|}
\hline
\textbf{Nº} & \textbf{Nombre anatómico} & \textbf{Localización / Descripción} \\
\hline
1 & Escotadura yugular & Punto superior en línea media, entre articulaciones esternoclaviculares \\
\hline
2 & Ángulo cardiofrénico izquierdo & Unión del borde inferior izquierdo de la silueta cardíaca con la cúpula diafragmática \\
\hline
3 & Borde costal lateral superior izquierdo & Contorno lateral alto del hemitórax izquierdo, nivel de 2ª-3ª costilla posterior \\
\hline
4 & Borde costal lateral superior derecho & Homólogo del \textit{landmark} \#3 en el hemitórax derecho \\
\hline
5 & Borde costal lateral medio izquierdo & Contorno medio lateral del pulmón izquierdo, tercio medio del hemitórax \\
\hline
6 & Borde costal lateral medio derecho & Homólogo del \textit{landmark} \#5 en el hemitórax derecho \\
\hline
7 & Borde costal lateral inferior izquierdo & Contorno lateral inferior del pulmón izquierdo, inmediatamente superior al diafragma \\
\hline
8 & Borde costal lateral inferior derecho & Homólogo del \textit{landmark} \#7 en el hemitórax derecho \\
\hline
9 & Carina traqueal & Bifurcación de la tráquea en bronquios principales, mediastino medio \\
\hline
10 & Borde cardíaco derecho medio & Límite lateral derecho de la silueta cardíaca, correspondiente a la aurícula derecha \\
\hline
11 & Borde cardíaco izquierdo inferior & Límite lateral inferior izquierdo de la silueta cardíaca, ventrículo izquierdo \\
\hline
12 & Ápice pulmonar izquierdo subclavicular & Punto más alto del campo pulmonar izquierdo, bajo el extremo medial de la clavícula \\
\hline
13 & Ápice pulmonar derecho subclavicular & Homólogo del \textit{landmark} \#12 en el hemitórax derecho \\
\hline
14 & Ángulo costofrénico izquierdo & Receso pleural posterolateral izquierdo, unión diafragma-pared torácica costal \\
\hline
15 & Ángulo costofrénico derecho & Homólogo del \textit{landmark} \#14 en el hemitórax derecho \\
\hline
\end{tabular}
\end{table}

Esta configuración de \textit{landmarks} presenta características geométricas de particular relevancia para el análisis computarizado: siete pares de puntos con simetría bilateral (\#3-4, \#5-6, \#7-8, \#10-11, \#12-13, \#14-15, y \#2 respecto al eje de simetría), y dos puntos localizados en la línea media que definen el eje vertical de simetría (\#1 y \#9). La simetría bilateral es una propiedad anatómica fundamental del tórax normal que puede explotarse mediante restricciones geométricas en algoritmos de aprendizaje profundo. Esta propiedad geométrica se incorpora explícitamente en la función de pérdida propuesta en este trabajo, como se discutirá en detalle en la Sección~2.5. Adicionalmente, las distancias entre pares de \textit{landmarks} específicos (por ejemplo, entre \#3 y \#4, o entre \#12 y \#13) representan medidas anatómicas con variabilidad limitada que pueden utilizarse como restricciones de preservación de distancias.

\subsection{Aplicaciones Clínicas de la Localización de Landmarks}

La localización precisa de \textit{landmarks} anatómicos en radiografías de tórax tiene múltiples aplicaciones en la práctica clínica y en sistemas de análisis automatizado. Las aplicaciones diagnósticas incluyen la cuantificación de parámetros anatómicos (como índice cardiotorácico, altura pulmonar, y evaluación de simetría bilateral), la detección de asimetrías patológicas mediante comparación de distancias entre pares de \textit{landmarks} homólogos, y la caracterización de deformaciones anatómicas asociadas a patologías específicas \cite{Sardanelli2022}.

En el contexto de sistemas de diagnóstico asistido por computadora, la detección automática de \textit{landmarks} constituye una etapa de preprocesamiento fundamental para dos aplicaciones principales \cite{Oakden-Rayner2020, Sogancioglu2021}: (1) la normalización espacial de radiografías con variabilidad en posicionamiento del paciente, distancia foco-detector, y grado de inspiración, y (2) la segmentación automática de regiones anatómicas mediante la definición de límites anatómicos iniciales. Como se estableció en el Capítulo 1, la detección manual de \textit{landmarks} por radiólogos expertos presenta variabilidad inter-observador significativa y requiere tiempo considerable, motivando el desarrollo de métodos automatizados basados en aprendizaje profundo.

Estudios recientes han demostrado que la incorporación explícita de \textit{landmarks} anatómicos en arquitecturas de aprendizaje profundo mejora significativamente la interpretabilidad de los modelos, permitiendo a los clínicos comprender qué regiones anatómicas contribuyen a las predicciones diagnósticas \cite{Liu2024, Li2022}. Esta interpretabilidad representa un aspecto crítico para la adopción clínica de sistemas de inteligencia artificial en medicina, particularmente en contextos de alta demanda diagnóstica donde la confiabilidad y la transparencia algorítmica son requisitos esenciales \cite{Rubin2018}. Los fundamentos matemáticos y computacionales de las arquitecturas de aprendizaje profundo que permiten la detección automática de estos \textit{landmarks} se desarrollan en las secciones subsecuentes.
