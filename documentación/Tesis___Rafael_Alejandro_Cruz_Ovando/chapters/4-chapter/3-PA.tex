\section{Amplificador de Potencia (PA)}

Un amplificador de potencia (o por sus siglas en inglés PA) es un bloque esencial en sistemas electrónicos que requieren transmitir señales con suficiente energía hacia una carga, típicamente una antena o un altavoz. Su función principal es incrementar la potencia de una señal eléctrica, manteniendo su forma y características espectrales, para que pueda ser útil en aplicaciones como transmisión de radiofrecuencia (RF), audio de alta fidelidad o sistemas de comunicación inalámbrica \cite{Libro_Razavi_Rf}.

A diferencia de los amplificadores de pequeña señal, donde la linealidad y ganancia son prioritarias, en los amplificadores de potencia también es crucial la eficiencia energética, dado que manejan niveles altos de corriente y voltaje. Esto implica un diseño cuidadoso para minimizar pérdidas, controlar el calor generado y evitar distorsiones significativas.

A nivel de implementación, en tecnologías CMOS modernas, el diseño de amplificadores de potencia debe considerar aspectos como el ancho de banda operativo, la adaptación de impedancia (típicamente a \SI{50}{\ohm}), la tensión de salida requerida, y la capacidad del diseño para integrarse con otros bloques analógicos. Los transistores de salida suelen trabajar en región de saturación para maximizar la ganancia de potencia, y se emplean técnicas como etapas diferenciales, polarización adaptativa, o redes de salida sintonizadas para mejorar el rendimiento.

\subsection{Fundamentos de amplificadores de potencia}

En un PA las principales métricas de desempeño son:

\begin{itemize}
    \item Potencia de salida ($P_{out}$)
    \item Ganancia de potencia ($G_p$)
    \item Eficiencia de potencia añadida (PAE)
    \item Linealidad 
    \item Estabilidad (factor $K>1$)
\end{itemize}

En la etapa de diseño, se utilizan herramientas como simulaciones de carga-pull o análisis de envolvente para optimizar la respuesta del circuito bajo distintas condiciones de carga y señal.

Adicionalmente, los amplificadores de potencia se clasifican comúnmente por su clase de operación: Clase A, B, AB, C, D, entre otras, cada una con un compromiso distinto entre eficiencia, linealidad y complejidad. Por ejemplo, la Clase A ofrece excelente linealidad pero baja eficiencia, mientras que la Clase AB logra un mejor equilibrio entre ambos aspectos, siendo común en transmisores \cite{Articulo_FMUWB_Trigger}.

\begin{table}[H]
    \centering
    \begin{tabular}{|c|c|c|c|c|c|}
    \hline
    \textbf{Clase} &  \begin{tabular}[c]{@{}c@{}}\textbf{Ángulo de}\\ \textbf{conducción}\end{tabular} &\begin{tabular}[c]{@{}c@{}}\textbf{Eficiencia}\\ \textbf{teórica}\end{tabular} & \textbf{Linealidad} &\begin{tabular}[c]{@{}c@{}}\textbf{Corriente}\\ \textbf{en reposo}\end{tabular} &\begin{tabular}[c]{@{}c@{}}\textbf{Aplicaciones}\\ \textbf{típicas}\end{tabular} \\
    \hline
    A  & $360^\circ$                     & $\sim$25--30\%         & Excelente      & Alta      & Audio, RF, UWB \\
    B  & $180^\circ$                     & $\sim$78.5\%                  & Baja           & Cero      & RF \\
    AB & $180^\circ$--$360^\circ$       & $\sim$35--55\%                & Buena          & Moderada  & RF, UWB \\
    C  & $<$180$^\circ$                 & $>$80\%                       & Muy baja       & Cero      & AM/FM \\
    \hline
    \end{tabular}
    \caption{Comparación entre clases de amplificadores de potencia \cite{Libro_Cripps_RFPA,Libro_Razavi_Rf}.}
    \label{tab:004:003:001}
\end{table}

Los amplificadores de potencia (PA) de tipo A y tipo AB son ampliamente utilizados en circuitos integrados CMOS debido a su balance entre linealidad, eficiencia y simplicidad de diseño. En la Tabla \ref{tab:004:003:001} se presentan los principales tipos de amplificadores de potencia. Se observa que, para aplicaciones UWB, es recomendable emplear un amplificador de tipo A o AB como diseño principal. En la figura \ref{fig:004:003:001} se observa los PA tipo A y tipo AB.

\begin{figure}[H]
    \centering
    \begin{subfigure}[b]{0.29\textwidth}
        \centering
        \includegraphics[width=\textwidth]{chapters/4-chapter/figuras/PA_CLASEA.png}
        \caption{Clase A.}
        \label{fig:004:003:001a}
    \end{subfigure}
    \hspace{0.1\textwidth} % Espacio controlado entre imágenes
    \begin{subfigure}[b]{0.32\textwidth}
        \centering
        \includegraphics[width=\textwidth]{chapters/4-chapter/figuras/PA_CLASEAB.png}
        \caption{Clase AB.}
        \label{fig:004:003:001b}
    \end{subfigure}
    \caption{Tipos de amplificadores con CMOS.}
    \label{fig:004:003:001}
\end{figure}

Un amplificador tipo A en CMOS se caracteriza por polarizar el transistor activo (generalmente un MOSFET) de forma que conduzca durante todo el ciclo de la señal de entrada, es decir, durante los $360^\circ$. Esto garantiza una alta linealidad y baja distorsión, ya que el dispositivo nunca se apaga. Sin embargo, el inconveniente principal de esta clase es su baja eficiencia energética, típicamente alrededor del $25-30\%$ \cite{Libro_Cripps_RFPA,Libro_PA_YellowBlack}, debido a la corriente de polarización constante que circula incluso en ausencia de señal de entrada. En aplicaciones de RF y, en particular, en sistemas UWB, esta clase puede ser adecuada cuando se prioriza la fidelidad de la señal sobre la eficiencia.

Por otro lado, un amplificador tipo AB representa un compromiso entre la linealidad del tipo A y la eficiencia del tipo B. En esta configuración, el transistor conduce por más de la mitad pero menos de todo el ciclo (generalmente alrededor de $180^\circ–360^\circ$). Esto se logra mediante una polarización adecuada que permite que el dispositivo se mantenga apenas encendido en reposo, reduciendo así la corriente de polarización y mejorando significativamente la eficiencia (que puede llegar a alrededor de 50-60\% \cite{Libro_Cripps_RFPA,Libro_PA_YellowBlack}), mientras se mantiene una distorsión aceptablemente baja. En tecnología CMOS, los PA tipo AB son muy populares para sistemas UWB, ya que permiten manejar variaciones rápidas de la señal con buena eficiencia y sin degradar excesivamente la calidad de la transmisión.

En resumen, en PA en CMOS para UWB, la clase A es preferida cuando la linealidad extrema es crítica, mientras que la clase AB se elige cuando se requiere un balance entre eficiencia y fidelidad, especialmente en aplicaciones donde la duración de la batería o el consumo de energía son factores importantes.

\subsection{Amplificador de potencia tipo A}

El PA tipo A (ver figura \ref{fig:004:003:001a}) se caracteriza por su operación en la cual el dispositivo activo, generalmente un transistor NMOS en tecnología CMOS, conduce durante la totalidad del ciclo de la señal de entrada, es decir, durante los $360^\circ$) \cite{Libro_PA_YellowBlack}. Esta condición se logra mediante una polarización adecuada que mantiene el transistor siempre en la región de operación activa, permitiendo una alta linealidad y minimizando la distorsión armónica, es decir mientras el voltaje de la compuerta a la fuente ($V_{GS}$) sea superior al voltaje del umbral ($V_{TH}$). 

\begin{figure}[H]
    \centering
    \includegraphics[width=5cm]{chapters/4-chapter/figuras/PA_CLASEAL.png}
    \caption{PA tipo A con inductor.}
    \label{fig:004:003:002}
\end{figure}

En la figura \ref{fig:004:003:002} se sustituye la fuente de corriente $I_b$ con un inductor $L$, a este elemento pasivo se le conoce como inductor de \textit{choke}, la principal razón del uso es el transistor CMOS, en particular, tiene una limitación en cuanto a la capacidad de generar una fuente de corriente estable durante todo el ciclo de la señal de entrada. Para sortear esta limitación, se utiliza un inductor en la etapa de carga.

En la implementación CMOS, el diseño de un amplificador tipo A enfrenta desafíos adicionales, como las limitaciones de voltaje de umbral de los dispositivos y el manejo del calor generado debido a las pérdidas continuas de potencia \cite{Libro_PA_YellowBlack}. Sin embargo, su ventaja principal radica en la excelente fidelidad de la señal, lo cual es crucial en aplicaciones de comunicación de alta precisión, tales como transmisores de ultra banda ancha (UWB), donde las características de baja distorsión son prioritarias.

En resumen, aunque los amplificadores de clase A no son óptimos en términos de eficiencia, su uso se justifica en escenarios donde la calidad de la señal es más importante que el consumo energético, particularmente en etapas de transmisión de baja potencia donde las exigencias de linealidad son estrictas.

\subsection{Amplificador de potencia tipo AB}
Un PA tipo AB combina características claves de el PA tipo A; tal como la linealidad, y del tipo B; la eficiencia. Las propiedades clave del diseño, se observa que en condiciones estáticas ambos transistores permanecen conduciendo, con corrientes iguales a $I_s = I_b$, como en un amplificador de clase A. Esto evita la distorsión por cruce. Además, el uso de un desplazamiento de nivel en continua permite transferir eficazmente las variaciones de la señal de entrada al transistor superior.

Como consecuencia, las corrientes que fluyen hacia y desde el terminal de salida superan la corriente de polarización. Esto hace que los amplificadores en clase AB no solo sean adecuados para señales de gran amplitud debido a su rápida respuesta, sino que también ofrezcan bajo consumo energético y una distorsión mínima, lo que los convierte en una solución eficiente y atractiva para aplicaciones modernas que demandan alta eficiencia energética.

En la figura \ref{fig:004:003:003} se observa un PA tipo AB con sistema de polarización del NMOS y PMOS por voltajes ($V_{b1}$ y $V_{b2}$) y una red resitiva con un capacitor en paralelo, esta hace que los transistores NMOS y PMOS estén encendidos y así el PA puedan entregar el punto de potencia adecuado para que el sistema suministre un valor adecuado hacia la carga.


\begin{figure}[H]
    \centering
    \includegraphics[width=8cm]{chapters/4-chapter/figuras/PA_CLASEABR.png}
    \caption{Amplificador de potencia tipo AB con sistema de polarización.}
    \label{fig:004:003:003}
\end{figure}

\subsection{Propuesta de diseño 1: PA tipo A}

Como propuesta de diseño se tiene que considerar las características de UWB:

\begin{enumerate}
    \item La PSD se tiene como límite de \SI{-41.3}{\decibel\milli/\mega\hertz}.
    \item La carga ($R_L$) del PA se establece como \SI{50}{\ohm}.
    \item La potencia máxima sobre la carga se establece como  \SI{-14}{\decibel m} \cite{Articulo_FMUWB_Trigger}.
    \item El ancho de banda debe ser de al menos \SI{500}{\mega\hertz}.
    \item El inductor de \textit{choke} no debe ser mayor a \SI{1}{\henry}.
    \item Etapa de transformación de potencia para maximar la corriente distribuida.
    \item Potencia de consumo no mayor 
\end{enumerate}

Suponiendo una potencia de sobre la carga de \SI{-14}{\decibel m}, se obtiene la cantidad de voltaje/corriente de salida:

\begin{eqnarray*}
P &=& 10^{\dfrac{\SI{}{[\decibel m]}}{10}}  \\
 &=& 10^{\frac{\SI{}{[\decibel m]}}{10}}  \\
 &=& 10^{\frac{-14}{10}}  \\
P &=& 0.0398x10^{-5}\\
\end{eqnarray*}
\begin{eqnarray*}
V_p &=& \sqrt{P*2*R_L}\\
&=& \sqrt{3.98x10^{-5}*2*50}\\
&=& \SI{63.1}{\milli\volt}\\
\end{eqnarray*}

Por lo tanto, el voltaje y la corriente de la onda senoidal que tendrá la carga se observa en las ecuaciones \ref{eq:004:003:001} y \ref{eq:004:003:002}, donde la corriente pico de la carga es $I_{p}=$\SI{1.262}{\milli\ampere}.

\begin{equation}
    v_{RL}(t) = 0.0631*sin(w_0t)
    \label{eq:004:003:001}
\end{equation}

\begin{equation}
    i_{RL}(t) = 0.001262*sin(w_0t)
    \label{eq:004:003:002}
\end{equation}

\begin{figure}[H]
    \centering
    \includegraphics[width=16cm]{chapters/4-chapter/figuras/corriente_cmos_pa.jpg}
    \caption{Curvas $I_D$ vs $V_{GS}$ con diferentes anchos de canal.}
    \label{fig:004:003:004}
\end{figure}

Para la selección del tamaño del ancho de canal para una tecnología de \SI{65}{\nano\meter}, se obtiene con la caracterización del transistor NMOS se observa en la figura \ref{fig:004:003:004}, donde se usa un voltaje de alimentación $V_{DD}$ de \SI{1.8}{\volt}. Se selecciono un ancho de canal $W=1.2-1.4$ \SI{}{\micro\meter} debido a que su corriente del transistor no excede los \SI{1.3}{\milli\ampere}.

En la figura \ref{fig:004:003:005} se muestra la propuesta del PA clase A implementado un transistor NMOS $M_1$ polarizado mediante un inductor RF $L_m$ el cual actua como una carga activa de alta impedancia en frecuencias altas, permitiendo el paso de corriente continua. 

%La señal de entrada (\(V_{in}\)) se aplica a la compuerta de \(M_1\), el cual opera en la región activa durante todo el ciclo de la señal.

El capacitor de acoplamiento ($C_m$) bloquea la componente de corriente directa mientras permite que la señal amplificada llegue a la etapa de salida.

\begin{figure}[H]
    \centering
    \includegraphics[width=16cm]{chapters/4-chapter/figuras/PA_CLASEAFINAL.png}
    \caption{Propuesta de amplificador de potencia clase A.}
    \label{fig:004:003:005}
\end{figure}

La red de salida está compuesta por una celda tanque resonante formada por la inductancia ($L_1$) y el capacitor ($C_1$), la cual sintoniza la frecuencia de operación deseada, maximizando la transferencia de potencia hacia la carga resistiva ($R_L=$\SI{50}{\ohm}). Esta topología garantiza una amplificación lineal con mínima distorsión, lo cual es característico de los amplificadores clase %A. Sin embargo, su eficiencia sigue siendo limitada, típicamente menor al 35\%, debido a que el transistor conduce corriente constantemente, incluso sin señal de entrada. A pesar de ello, esta arquitectura es utilizada en aplicaciones donde la linealidad y la integridad de la señal son críticas, como en etapas de salida de transmisores de radiofrecuencia de baja potencia.

