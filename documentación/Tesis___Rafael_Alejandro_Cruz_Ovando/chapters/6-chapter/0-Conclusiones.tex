\chapter{Conclusiones}
\label{cap:conclusiones_trabajo_futuro}

El sistema desarrollado demuestra la viabilidad de utilizar modelos de apariencia basados en PCA en combinación con técnicas geométricas y análisis de regiones para la detección automática de un conjunto predefinido de puntos anatómicos en radiografías de tórax. La metodología propuesta aborda las variaciones en la forma y apariencia de las estructuras anatómicas mediante un proceso estructurado que incluye preprocesamiento, alineamiento de formas, análisis de regiones de búsqueda y templates, entrenamiento de modelos de apariencia y predicción localizada.

Los resultados de las pruebas preliminares, aunque enfocados en un subconjunto de puntos y sin la aplicación completa del alineamiento de formas, han sido cruciales para comprender el comportamiento del sistema. Se observó que la presencia de imágenes con variaciones significativas de traslación, rotación y escala en el dataset de entrenamiento impacta negativamente la precisión de las predicciones, aumentando el error. Esto subraya la importancia crítica de la etapa de alineamiento de formas (GPA) para normalizar la variabilidad geométrica y permitir que los modelos de apariencia se centren en las características visuales intrínsecas de los puntos. La normalización de contraste, como la aplicada con el algoritmo SAHS, demostró ser beneficiosa para reducir el error en regiones con alta variabilidad de contraste, como la Coordenada 2.

Las principales fortalezas del enfoque incluyen su base matemática sólida, así como su modularidad y la capacidad de adaptarse a diferentes puntos anatómicos. Las limitaciones identificadas incluyen la dependencia de la calidad y cantidad de las anotaciones manuales para el entrenamiento, la sensibilidad a variaciones de imagen no capturadas por los modelos de apariencia (ej. patologías severas o artefactos), y la complejidad computacional de la búsqueda exhaustiva en la región definida.

En base a los hallazgos y las observaciones, se concluye que es necesario implementar un método más robusto para atacar el problema de las imágenes muy variantes en su forma. Esto implica no solo la aplicación rigurosa de métodos de alineamiento como GPA, sino también la consideración de enfoques más avanzados.

\section{Trabajo Futuro}

\begin{itemize}
    \item La implementación y evaluación completa del proceso de alineamiento de formas (GPA) en todo el dataset para cuantificar su impacto en la reducción de errores de predicción en todos los puntos anatómicos.
    \item La incorporación de modelos de forma activa (como Active Shape Models - ASM o Active Appearance Models - AAM) que utilicen las relaciones espaciales entre los puntos para imponer restricciones y guiar la búsqueda de apariencia, mejorando la robustez y la coherencia de las predicciones.
    \item La evaluación de técnicas de aprendizaje profundo, como las Redes Neuronales Convolucionales (CNNs), para la extracción de características de apariencia más robustas y discriminativas. Estas podrían ofrecer una representación más potente que PCA, especialmente en presencia de variaciones complejas.
    \item La adaptación del sistema para manejar imágenes con resoluciones variables o diferentes modalidades de imagen, posiblemente mediante el uso de técnicas de normalización más avanzadas.
    \item La implementación de métodos de búsqueda más eficientes que no requieran evaluar exhaustivamente todos los puntos en la región de búsqueda, como la búsqueda jerárquica o el uso de clasificadores rápidos para descartar candidatos poco probables, lo que podría reducir la complejidad computacional.
\end{itemize}
En conclusión, este trabajo establece una base sólida para la detección automática de puntos anatómicos en radiografías, proporcionando una metodología matemáticamente fundamentada y un proceso funcional. Los resultados preliminares han identificado áreas clave de mejora, y las líneas de investigación futuras prometen avanzar significativamente en la robustez y precisión del sistema frente a la alta variabilidad de las imágenes médicas.

\section{Conclusiones Segmentación Automática}
Los resultados del diseño experimental demuestran que el modelo híbrido exhibe una capacidad de segmentación con una precisión promedio del $70.10\%$. La variabilidad en la precisión, indicada por una desviación estándar de $0.0753$, sugiere que el rendimiento del modelo puede fluctuar dependiendo del modo de entrenamiento o de las características específicas de los datos en cada partición.

El modo k0 se destacó significativamente con una precisión de evaluación final del $88.70\%$ y una pérdida considerablemente menor ($0.37881$) en comparación con los otros modos. Esto podría indicar que ciertas configuraciones de datos o condiciones iniciales son más favorables para la convergencia y el rendimiento óptimo del modelo.

La pérdida de evaluación final promedio de $0.86882$ es consistente con la precisión observada. La dispersión de los resultados entre los modos subraya la importancia de la evaluación en múltiples configuraciones para comprender la robustez y la generalización del modelo. Futuras investigaciones podrían enfocarse en identificar los factores que contribuyen a la alta precisión en modos específicos y en reducir la variabilidad para mejorar la consistencia del rendimiento del modelo en diversas condiciones.

Esto confirma que usar un método más robusto para la extracción de características mejora la precisión al predecir landmarks y segmentar adecuadamente, lo cual nos da una base sólida para explorar estos nuevos métodos e incorporarlos al sistema principal, además de continuar con la investigación y la experimentación para obtener un método que pueda cumplir con los objetivos de la tesis. 

\section{Conclusiones y trabajo futuro SAHS}
Los resultados demuestran que SAHS (Statistical Asymmetrical Histogram Stretching) es efectivo para mejorar la precisión de la clasificación de imágenes radiográficas de tórax en una variedad de arquitecturas CNN. En la mayoría de los casos, supera al método CLAHE convencional. La eficacia de SAHS puede atribuirse a su capacidad para adaptar la mejora de contraste a la naturaleza asimétrica de los histogramas en imágenes radiográficas de tórax, preservando información crítica para el diagnóstico mientras mejora la visibilidad de estructuras relevantes. Es importante notar que mientras SAHS mostró mejoras consistentes, la magnitud de la mejora varió entre las diferentes arquitecturas. Esto sugiere que la elección del método de preprocesamiento debe considerarse en conjunto con la selección de la arquitectura CNN para optimizar el rendimiento general del sistema de clasificación.  Las principales contribuciones de este artículo son dos: un método de ajuste de contraste para histogramas asimétricos SAHS como los que presentan las imágenes radiográficas de tórax, y la integración exitosa de SAHS con el Algoritmo Localizador de Pulmones (ALP) para una clasificación más precisa de la neumonía en imágenes radiográficas de tórax. En conclusión, estos resultados respaldan la eficacia del método propuesto SAHS en este artículo como una técnica de preprocesamiento valiosa para mejorar la precisión en la clasificación de imágenes radiográficas de tórax utilizando diversas arquitecturas de CNN. Como trabajo futuro se propone analizar aún más detalladamente la forma del histograma en este tipo de radiografías para desarrollar un método de ajuste de contraste basado completamente en la forma típica de esta clase de histogramas. Así mismo, para realizar dicho ajuste de contraste de toda la imagen podrían tomarse como referencia ciertas zonas o subregiones de brillo estable tales como la columna vertebral.
