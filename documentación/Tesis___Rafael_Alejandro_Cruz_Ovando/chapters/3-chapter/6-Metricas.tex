\section{Métricas de evaluación}
\label{sec:metricas_eval}

La Sección~\ref{sec:implementacion} documentó detalles técnicos de reproducibilidad computacional que garantizan replicabilidad determinística del entrenamiento. La presente sección define formalmente las métricas de evaluación empleadas para cuantificar rendimiento del sistema desarrollado, estableciendo criterios objetivos que permiten comparación rigurosa con trabajos previos en detección automática de \textit{landmarks} (puntos de referencia anatómicos) y valoración de idoneidad clínica del sistema. La definición matemática precisa de métricas constituye componente fundamental de metodología científica rigurosa: sin especificación formal inequívoca, reportes numéricos de rendimiento carecen de interpretabilidad y comparabilidad, impidiendo reproducción y validación independiente de resultados \cite{Payer2016}.

El diseño del sistema de evaluación implementa jerarquía de métricas complementarias que caracterizan diferentes aspectos de calidad de predicciones. La métrica primaria, Error Radial Medio (MRE, \textit{Mean Radial Error}), cuantifica precisión de localización espacial promedio, constituyendo estándar universal en literatura de detección de \textit{landmarks} anatómicos que permite comparación directa con trabajos previos \cite{Payer2016, Feng2018, Ibragimov2017}. Las métricas geométricas complementarias evalúan consistencia estructural de predicciones: error de simetría bilateral, preservación de distancias anatómicas críticas, y validez de ordenamiento espacial fisiológico. Estas métricas capturan aspectos de coherencia anatómica que métricas de error puntual aisladas no detectan, siendo críticas para valoración de aceptabilidad clínica. Finalmente, el sistema de clasificación por umbrales de calidad clínica traduce mediciones continuas de error a categorías discretas interpretables por profesionales médicos (excelente, bueno, aceptable, inaceptable), facilitando comunicación de capacidades del sistema a audiencias no técnicas y estableciendo criterios de decisión para aprobación de uso clínico.


\subsection{Error Radial Medio (MRE)}
\label{subsec:mre}

El Error Radial Medio (MRE, \textit{Mean Radial Error}) constituye la métrica primaria de evaluación, cuantificando precisión de localización espacial mediante distancia euclidiana promedio entre coordenadas predichas y anotaciones de referencia expertas. El MRE se define formalmente como:

\begin{equation}
\label{eq:mre_definicion}
\text{MRE} = \frac{1}{N \cdot K} \sum_{i=1}^{N} \sum_{k=1}^{K} \sqrt{(x_{i,k}^{pred} - x_{i,k}^{gt})^2 + (y_{i,k}^{pred} - y_{i,k}^{gt})^2} \cdot s_i,
\end{equation}

donde:
\begin{itemize}
    \item $N$ es el número total de muestras en conjunto de evaluación (143 en validación, 144 en prueba).
    \item $K = 15$ es el número de \textit{landmarks} anatómicos por radiografía.
    \item $(x_{i,k}^{pred}, y_{i,k}^{pred}) \in [0,1]^2$ son coordenadas normalizadas predichas por el modelo para \textit{landmark} $k$ en muestra $i$.
    \item $(x_{i,k}^{gt}, y_{i,k}^{gt}) \in [0,1]^2$ son coordenadas normalizadas de referencia (\textit{ground truth}) anotadas por experto.
    \item $s_i = 224$ píxeles es el factor de escala que convierte coordenadas normalizadas a coordenadas absolutas en píxeles en espacio de imagen de entrada a la red ($224 \times 224$ píxeles).
\end{itemize}

La formulación expresa error en unidades de píxeles absolutas en lugar de coordenadas normalizadas, facilitando interpretabilidad física directa: un MRE de 5.0 píxeles indica que, en promedio, predicciones del modelo se localizan a 5 píxeles de distancia de posiciones verdaderas en imágenes de $224 \times 224$ píxeles. Esta convención de reportar errores en píxeles constituye estándar universal en literatura de detección de \textit{landmarks}, permitiendo comparación directa con trabajos previos independientemente de resolución de imagen específica empleada por cada método \cite{Payer2016, Feng2018}.

El MRE posee propiedades estadísticas deseables como métrica de localización. Primero, invariancia ante permutaciones de muestras o \textit{landmarks}: el orden de suma no afecta valor final. Segundo, sensibilidad uniforme a errores en todas direcciones espaciales: distancia euclidiana penaliza desplazamientos horizontales y verticales equitativamente mediante norma $L_2$. Tercero, interpretabilidad intuitiva: errores mayores contribuyen proporcionalmente más al promedio, y un MRE de cero indica concordancia perfecta entre predicciones y referencia.

La limitación principal del MRE como métrica aislada es insensibilidad a patrones de error: un modelo que distribuye errores uniformemente entre todos los \textit{landmarks} produce el mismo MRE que un modelo con errores concentrados en subconjunto específico de \textit{landmarks} difíciles, aunque el segundo pueda ser más útil clínicamente si localiza correctamente estructuras críticas (por ejemplo, carinas, ápices pulmonares) aunque falle en estructuras auxiliares. Esta limitación motiva introducción de métricas complementarias que caracterizan distribución espacial y consistencia estructural de errores.


\subsection{Error por landmark individual}
\label{subsec:error_por_landmark}

El análisis de error desagregado por \textit{landmark} individual proporciona diagnóstico detallado de capacidades y limitaciones del modelo, identificando estructuras anatómicas específicas que presentan mayor dificultad de localización. El error radial medio por \textit{landmark} $k$ se define como:

\begin{equation}
\label{eq:mre_por_landmark}
\text{MRE}_k = \frac{1}{N} \sum_{i=1}^{N} \sqrt{(x_{i,k}^{pred} - x_{i,k}^{gt})^2 + (y_{i,k}^{pred} - y_{i,k}^{gt})^2} \cdot s_i.
\end{equation}

Esta métrica permite identificación de patrones sistemáticos de error. Por ejemplo, estructuras de alto contraste bien definidas como carinas traqueales típicamente exhiben $\text{MRE}_k < 3$ píxeles, mientras que estructuras difusas de bajo contraste como ángulos costofrénicos pueden presentar $\text{MRE}_k > 8$ píxeles debido a ambigüedad anatómica inherente. El análisis por \textit{landmark} informa decisiones de refinamiento arquitectural: errores concentrados en \textit{landmarks} específicos sugieren necesidad de mecanismos de atención espacial que enfoquen capacidad representacional en regiones de interés \cite{Hou2021}, mientras que errores uniformemente distribuidos indican limitaciones de capacidad global del modelo que requieren aumento de profundidad o anchura arquitectural.

La desviación estándar del error por \textit{landmark} cuantifica variabilidad de predicciones:
\begin{equation}
\label{eq:std_por_landmark}
\sigma_k = \sqrt{\frac{1}{N} \sum_{i=1}^{N} \left(\sqrt{(x_{i,k}^{pred} - x_{i,k}^{gt})^2 + (y_{i,k}^{pred} - y_{i,k}^{gt})^2} \cdot s_i - \text{MRE}_k\right)^2}.
\end{equation}

Desviaciones estándar elevadas ($\sigma_k > 0.5 \times \text{MRE}_k$) indican predicciones inconsistentes con alta variabilidad caso-a-caso, sugiriendo sensibilidad excesiva a variaciones en apariencia de imagen (artefactos, patologías superpuestas) que degradan robustez clínica. Desviaciones estándar bajas indican predicciones estables, propiedad deseable para confiabilidad en aplicaciones médicas.


\subsection{Métricas de consistencia geométrica}
\label{subsec:metricas_geometricas}

Las métricas de consistencia geométrica evalúan validez estructural de configuraciones predichas de \textit{landmarks}, cuantificando adherencia a restricciones anatómicas fundamentales: simetría bilateral del tórax, preservación de distancias entre estructuras emparejadas, y ordenamiento espacial fisiológico. Estas métricas capturan aspectos de coherencia anatómica críticos para aceptabilidad clínica que el MRE, enfocado exclusivamente en precisión de localización puntual, no detecta.


\subsubsection{Error de simetría bilateral}
\label{subsubsec:error_simetria}

El error de simetría bilateral cuantifica violaciones de la restricción de simetría aproximada del tórax humano, formalizada mediante la función de pérdida de simetría $\mathcal{L}_{sym}$ introducida en la Fase 3 de entrenamiento (Sección~\ref{sec:phase3_symmetry}). La métrica de error de simetría se define como:

\begin{equation}
\label{eq:metrica_simetria}
E_{sym} = \frac{1}{N \cdot |\mathcal{P}_{sym}|} \sum_{i=1}^{N} \sum_{(j,k) \in \mathcal{P}_{sym}} \left| |x_{i,j}^{pred} - x_{axis}| - |x_{i,k}^{pred} - x_{axis}| \right| \cdot s_i,
\end{equation}

donde $\mathcal{P}_{sym} = \{(2,3), (4,5), (6,7), (11,12), (13,14)\}$ es el conjunto de cinco pares de \textit{landmarks} bilateralmente simétricos (ápices pulmonares, ángulos costofrénicos, hilios, etc.), y $x_{axis} = 0.5$ es la coordenada horizontal normalizada del eje de simetría mediastínico central (Sección~\ref{subsec:simetria_bilateral}). La métrica cuantifica discrepancia promedio en distancias de \textit{landmarks} emparejados respecto al eje central, expresada en píxeles.

Valores bajos de error de simetría ($E_{sym} < 3$ píxeles) indican configuraciones predichas que respetan simetría bilateral, sugiriendo que el modelo ha internalizado restricciones anatómicas geométricas durante entrenamiento. Valores elevados ($E_{sym} > 6$ píxeles) indican predicciones asimétricas anatómicamente implausibles, frecuentemente causadas por confusión entre estructuras bilaterales homólogas (por ejemplo, intercambio de ápice pulmonar izquierdo y derecho) o sensibilidad excesiva a asimetrías patológicas reales (derrames pleurales unilaterales, neumonías lobares) que deben distinguirse cuidadosamente de errores de predicción.

La métrica de simetría complementa el MRE: un modelo puede exhibir MRE bajo mediante predicciones precisas en promedio pero violar simetría sistemáticamente (por ejemplo, consistentemente desplazando estructuras derechas hacia el centro), produciendo configuraciones anatómicamente inválidas. La incorporación explícita de $\mathcal{L}_{sym}$ en función de pérdida de entrenamiento (Fase 3) busca minimizar $E_{sym}$ simultáneamente con MRE, optimizando tanto precisión como coherencia estructural.


\subsubsection{Error de preservación de distancias}
\label{subsubsec:error_distancias}

El error de preservación de distancias cuantifica violaciones de restricciones de proporciones anatómicas, evaluando cuán fielmente las configuraciones predichas mantienen distancias relativas entre \textit{landmarks} observadas en anotaciones de referencia. La métrica se fundamenta en el conjunto de pares críticos de \textit{landmarks} $\mathcal{D}_{critical}$ definido en la Fase 4 de entrenamiento (Sección~\ref{sec:phase4_complete}), que incluye distancias anatómicamente significativas como altura pulmonar (distancia vertical entre ápices y bases), amplitud torácica (distancia horizontal entre ángulos costofrénicos), y dimensiones mediastínicas.

La métrica de error de preservación de distancias se define como error relativo porcentual promedio:

\begin{equation}
\label{eq:metrica_distancias}
E_{dist} = \frac{100\%}{N \cdot |\mathcal{D}_{critical}|} \sum_{i=1}^{N} \sum_{(j,k) \in \mathcal{D}_{critical}} \left| \frac{d_{i,jk}^{pred} - d_{i,jk}^{gt}}{d_{i,jk}^{gt}} \right|,
\end{equation}

donde:
\begin{align}
d_{i,jk}^{pred} &= \sqrt{(x_{i,j}^{pred} - x_{i,k}^{pred})^2 + (y_{i,j}^{pred} - y_{i,k}^{pred})^2} \label{eq:distancia_pred} \\
d_{i,jk}^{gt} &= \sqrt{(x_{i,j}^{gt} - x_{i,k}^{gt})^2 + (y_{i,j}^{gt} - y_{i,k}^{gt})^2} \label{eq:distancia_gt}
\end{align}

son las distancias euclidianas normalizadas entre \textit{landmarks} $j$ y $k$ en configuración predicha y de referencia, respectivamente. La formulación como error relativo porcentual normaliza por magnitud de distancia verdadera, evitando que distancias grandes dominen la métrica y permitiendo interpretación intuitiva: $E_{dist} = 10\%$ indica que, en promedio, distancias predichas difieren un 10\% de distancias verdaderas.

Valores bajos de error de preservación ($E_{dist} < 5\%$) indican que el modelo mantiene proporciones anatómicas correctamente, sugiriendo capacidad de capturar relaciones espaciales globales entre estructuras. Valores elevados ($E_{dist} > 15\%$) indican distorsiones geométricas sistemáticas que, aunque cada \textit{landmark} individual pueda tener error de localización bajo, la configuración global presenta proporciones anatómicas incorrectas (por ejemplo, tórax excesivamente estrecho o anormalmente alto), inaceptable para aplicaciones clínicas donde proporciones informan valoraciones diagnósticas.

La complementariedad entre MRE y $E_{dist}$ es sutil pero crítica: un modelo puede lograr MRE bajo mediante errores individuales pequeños que, al estar correlacionados sistemáticamente (por ejemplo, todos los \textit{landmarks} desplazados uniformemente hacia arriba), preservan distancias relativas y producen $E_{dist}$ bajo a pesar de configuración globalmente incorrecta. Por tanto, evaluación rigurosa requiere consideración simultánea de ambas métricas junto con métricas de simetría para caracterización completa de calidad de predicciones.


\subsubsection{Tasa de validez anatómica}
\label{subsubsec:validez_anatomica}

La tasa de validez anatómica cuantifica la proporción de predicciones que satisfacen restricciones de ordenamiento espacial fisiológico básicas, criterios binarios de aceptabilidad que toda configuración anatómicamente plausible debe cumplir. Se definen cuatro restricciones fundamentales:

\begin{enumerate}
    \item \textbf{Ordenamiento vertical de estructuras pulmonares}: Ápices pulmonares (landmarks 2, 3) deben localizarse superiormente (menor coordenada $y$, dado que origen es esquina superior izquierda) respecto a ángulos costofrénicos (landmarks 4, 5):
    \begin{equation}
    y_{\text{apex}}^{pred} < y_{\text{angulo}}^{pred}.
    \end{equation}

    \item \textbf{Centrado de estructuras mediastínicas}: Estructuras mediastínicas centrales (carina traqueal, ápice cardiaco, landmarks 1, 8) deben localizarse dentro de banda central del tórax, definida como $x \in [0.35, 0.65]$ en coordenadas normalizadas:
    \begin{equation}
    0.35 \leq x_{\text{mediastino}}^{pred} \leq 0.65.
    \end{equation}

    \item \textbf{No-inversión de estructuras bilaterales}: \textit{Landmarks} derechos (convencionalmente numerados pares: 2, 4, 6, etc.) deben localizarse a la derecha del eje de simetría ($x > 0.5$), y \textit{landmarks} izquierdos (impares: 3, 5, 7, etc.) a la izquierda ($x < 0.5$):
    \begin{equation}
    x_{\text{derecho}}^{pred} > 0.5, \quad x_{\text{izquierdo}}^{pred} < 0.5.
    \end{equation}

    \item \textbf{Contención dentro de campo de visión}: Todas las coordenadas predichas deben residir dentro de rango válido $[0,1]^2$:
    \begin{equation}
    0 \leq x_k^{pred} \leq 1, \quad 0 \leq y_k^{pred} \leq 1, \quad \forall k.
    \end{equation}
\end{enumerate}

Una muestra se clasifica como anatómicamente válida si satisface simultáneamente las cuatro restricciones para todos los \textit{landmarks} aplicables. La tasa de validez anatómica se define como:

\begin{equation}
\label{eq:tasa_validez}
\text{TVA} = \frac{\text{Número de muestras válidas}}{N} \times 100\%.
\end{equation}

Sistemas de calidad clínica deben alcanzar $\text{TVA} \geq 95\%$, garantizando que la vasta mayoría de predicciones son anatómicamente plausibles incluso si exhiben errores de localización moderados. Tasas de validez inferiores indican inestabilidad del modelo que produce ocasionalmente configuraciones absurdas (por ejemplo, ápices pulmonares inferiores a bases, estructuras mediastínicas desplazadas a periferia torácica), inaceptable para confianza clínica.


\subsection{Sistema de clasificación por calidad clínica}
\label{subsec:clasificacion_clinica}

El sistema de clasificación por umbrales de calidad clínica traduce mediciones continuas de MRE a categorías discretas interpretables, facilitando comunicación de capacidades del sistema a profesionales médicos y estableciendo criterios de decisión para aprobación de uso clínico. El sistema implementa cuatro categorías de calidad basadas en umbrales de error establecidos en literatura mediante análisis de variabilidad inter-observador entre radiólogos expertos y evaluación de precisión requerida para tareas diagnósticas específicas \cite{Payer2016, Ibragimov2017}.


\subsubsection{Definición de categorías}
\label{subsubsec:categorias_calidad}

Las cuatro categorías de calidad clínica se definen según rangos de MRE medido en imágenes de $224 \times 224$ píxeles:

\begin{enumerate}
    \item \textbf{Excelente} (MRE $< 2.0$ mm): Precisión equivalente a concordancia inter-observador entre radiólogos expertos experimentados. Errores de esta magnitud son imperceptibles en práctica clínica y no afectan interpretación diagnóstica. Sistemas en esta categoría alcanzan rendimiento humano experto, siendo candidatos ideales para integración en flujos de trabajo clínicos automatizados sin supervisión adicional.

    \item \textbf{Bueno} (2.0 mm $\leq$ MRE $< 4.0$ mm): Precisión suficiente para mayoría de aplicaciones clínicas de asistencia diagnóstica. Errores pueden ser detectables por observadores entrenados pero raramente alteran conclusiones diagnósticas. Sistemas en esta categoría son apropiados para despliegue clínico con supervisión ocasional por especialistas.

    \item \textbf{Aceptable} (4.0 mm $\leq$ MRE $< 8.5$ mm): Precisión marginal para aplicaciones clínicas, cercana al límite de aceptabilidad. Errores son frecuentemente visibles y pueden ocasionalmente afectar interpretación de hallazgos sutiles. Sistemas en esta categoría requieren validación extensiva caso-a-caso por especialistas antes de uso clínico, siendo más apropiados para aplicaciones de investigación, priorización de casos, o inicialización de segmentaciones manuales.

    \item \textbf{Inaceptable} (MRE $\geq 8.5$ mm): Precisión insuficiente para uso clínico. Errores son sistemáticos y de magnitud que compromete interpretabilidad de resultados. Sistemas en esta categoría no deben emplearse en contextos clínicos, requiriendo refinamiento metodológico fundamental antes de consideración para aplicaciones médicas.
\end{enumerate}

El umbral crítico de 8.5 mm (equivalente a aproximadamente 8.5 píxeles en imágenes de $224 \times 224$ para tórax adulto estándar con campo de visión de 40 cm, asumiendo resolución espacial de $\sim$1.8 mm/píxel) fue establecido por Payer et al. \cite{Payer2016} mediante análisis de variabilidad inter-observador: errores superiores a este umbral exceden discrepancias típicas entre anotaciones de múltiples radiólogos expertos, indicando que predicciones automáticas son menos confiables que juicio humano. Este umbral constituye estándar de facto en literatura de detección automática de \textit{landmarks} en radiografías de tórax, siendo referencia para comparación de métodos \cite{Payer2016, Ibragimov2017, Urschler2018}.


\subsubsection{Conversión de MRE a milímetros}
\label{subsubsec:conversion_mm}

La conversión de MRE expresado en píxeles a unidades físicas de milímetros requiere conocimiento de resolución espacial de imágenes, definida como distancia física representada por cada píxel. En el conjunto de datos empleado (Sección~\ref{sec:dataset}), las radiografías corresponden a tórax adultos con campo de visión típico de 40 cm, procesadas a resolución de $224 \times 224$ píxeles. La resolución espacial aproximada es:

\begin{equation}
\label{eq:resolucion_espacial}
r = \frac{400 \text{ mm}}{224 \text{ píxeles}} \approx 1.79 \text{ mm/píxel}.
\end{equation}

Por tanto, un MRE de $E_{pix}$ píxeles corresponde a error físico de:

\begin{equation}
\label{eq:mre_mm}
E_{mm} = E_{pix} \times 1.79 \text{ mm/píxel}.
\end{equation}

Siguiendo esta conversión, el umbral de aceptabilidad clínica de 8.5 píxeles corresponde a aproximadamente 15.2 mm de error físico, valor que coincide con definiciones de literatura médica \cite{Payer2016}. Los umbrales de categorías de calidad expresados en píxeles son:

\begin{itemize}
    \item Excelente: MRE $< 1.12$ píxeles (2.0 mm).
    \item Bueno: 1.12 píxeles $\leq$ MRE $< 2.23$ píxeles (4.0 mm).
    \item Aceptable: 2.23 píxeles $\leq$ MRE $< 4.75$ píxeles (8.5 mm).
    \item Inaceptable: MRE $\geq 4.75$ píxeles (8.5 mm).
\end{itemize}

Debe notarse que esta conversión asume campo de visión uniforme de 40 cm, aproximación válida para radiografías de tórax posteroanterior estándar de adultos. Variaciones en tamaño corporal del paciente, distancia foco-detector, y magnificación geométrica introducen variabilidad en resolución espacial efectiva, por lo que umbrales absolutos deben interpretarse como guías aproximadas sujetas a calibración específica por protocolo de adquisición clínico.


\subsection{Protocolo de validación}
\label{subsec:validacion}

El protocolo de validación implementa separación estricta de conjuntos de entrenamiento, validación y prueba para garantizar evaluación no sesgada de capacidad de generalización. El conjunto de prueba de 144 muestras (15\% del total, Sección~\ref{subsec:division_dataset}) permanece completamente no visto durante todo el proceso de entrenamiento de las cuatro fases, utilizado exclusivamente para evaluación final tras selección de modelo óptimo basado en rendimiento en conjunto de validación.

La estrategia de \textit{early stopping} (detención temprana) descrita en la Sección~\ref{sec:estrategia_entrenamiento} monitorea pérdida de validación tras cada época, almacenando \textit{checkpoint} (punto de control) cuando se observa nuevo mínimo. El modelo final seleccionado para evaluación en conjunto de prueba corresponde al \textit{checkpoint} con menor pérdida de validación observada durante Fase 4 (entrenamiento con función de pérdida completa), garantizando que métricas reportadas en conjunto de prueba reflejan mejor capacidad de generalización alcanzada sin optimización directa sobre datos de prueba.

El análisis por subgrupos diagnósticos evalúa robustez del modelo ante variabilidad patológica, reportando MRE desagregado por categoría diagnóstica (COVID-19, Viral Pneumonia, Normal) para identificar sensibilidad diferencial a tipos específicos de patología. Desviaciones significativas de rendimiento entre subgrupos (por ejemplo, MRE sustancialmente mayor en casos COVID-19 respecto a radiografías normales) indicarían limitaciones de generalización que requerirían aumentación de datos específica por patología o estrategias de aprendizaje multidominio \cite{Raghu2019}.


\subsection{Síntesis de métricas}
\label{subsec:sintesis_metricas}

El sistema de evaluación documentado en esta sección implementa jerarquía de métricas complementarias que caracterizan precisión de localización (MRE), consistencia estructural (simetría, preservación de distancias, validez anatómica), e idoneidad clínica (clasificación por umbrales) de predicciones del modelo desarrollado. La definición formal matemática de cada métrica, acompañada de justificación teórica y umbrales de interpretación basados en estándares internacionales de literatura médica, garantiza reproducibilidad completa de evaluación y comparabilidad rigurosa con trabajos previos \cite{Payer2016, Feng2018, Ibragimov2017}.

La aplicación sistemática de estas métricas sobre conjunto de prueba independiente, cuyos resultados se presentan exhaustivamente en el Capítulo~\ref{cap:resultados}, constituye validación experimental definitiva de la metodología desarrollada en este capítulo, permitiendo valoración objetiva de idoneidad del sistema para eventual aplicación clínica en detección automática de estructuras anatómicas en radiografías de tórax.
