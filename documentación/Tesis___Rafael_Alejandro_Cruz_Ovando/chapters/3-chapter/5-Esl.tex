% % Asegúrate de que estos comandos están definidos en tu preámbulo:
% % \newcommand{\mat}[1]{\mathbf{#1}}   
% % \newcommand{\vect}[1]{\bm{#1}} % Requiere \usepackage{bm}
% % \newcommand{\transpose}{\mathsf{T}}
% % \newcommand{\R}{\mathbb{R}} 

% \section{Estimación de Pose Inicial (ESL)}
% \label{sec:esl_matematica}

% La inicialización precisa de los parámetros de pose (posición, orientación y escala) de un modelo de forma es un prerrequisito fundamental para la convergencia robusta y eficiente de procesos subsecuentes de ajuste iterativo, como los aplicados a Modelos Estadísticos de Forma (SSM). Una estimación inicial subóptima puede conducir a la convergencia del modelo hacia mínimos locales espurios o a un incremento computacionalmente prohibitivo en el número de iteraciones requeridas. Para mitigar estos riesgos, se introduce una etapa de Estimación de Pose Inicial (ESL). Esta etapa se fundamenta en el uso de un conjunto de funciones discriminativas, denotadas genéricamente como $C$, capaces de inferir los parámetros de una transformación global, $\mathcal{T}_{ESL}$, que relaciona el espacio canónico del modelo con el espacio de la imagen.

% El objetivo primordial de la ESL es la determinación de una transformación afín $\mathcal{T}_{ESL}: \R^2 \to \R^2$. Esta transformación mapea puntos $\vect{p} = (p_x, p_y)^\transpose$ desde un sistema de coordenadas canónico del modelo de forma hacia sus correspondientes localizaciones $\vect{x} = (x,y)^\transpose$ en la imagen de entrada $I(\vect{x})$. Dicha transformación se parametriza como:
% \begin{equation}
% \hspace*{\fill}
% \vect{x} = \mathcal{T}_{ESL}(\vect{p}) = \mat{S}_{ESL} \mat{R}(\theta_{ESL}) \vect{p} + \vect{t}_{ESL} = \begin{pmatrix} s_x & 0 \\ 0 & s_y \end{pmatrix} \begin{pmatrix} \cos \theta_{ESL} & -\sin \theta_{ESL} \\ \sin \theta_{ESL} & \cos \theta_{ESL} \end{pmatrix} \vect{p} + \begin{pmatrix} t_x \\ t_y \end{pmatrix},
% \label{eq:esl_transform_mat}
% \hspace*{\fill}
% \end{equation}
% donde $\mat{S}_{ESL} = \text{diag}(s_x, s_y)$ es la matriz de escala diagonal con factores de escala anisotrópica $s_x, s_y$; $\mat{R}(\theta_{ESL})$ es la matriz de rotación bidimensional correspondiente al ángulo de orientación $\theta_{ESL}$; y $\vect{t}_{ESL} = (t_x, t_y)^\transpose$ es el vector de traslación. La tarea de la ESL es, por lo tanto, estimar el conjunto de parámetros $(\hat{\vect{t}}_{ESL}, \hat{s}_x, \hat{s}_y, \hat{\theta}_{ESL})$. Si definimos el vector de parámetros de escala como $\hat{\vect{s}}_{ESL\_params} = (\hat{s}_x, \hat{s}_y)^\transpose$ (para distinguirlo de un posible vector de forma $\vect{s}$), el objetivo es estimar $(\hat{\vect{t}}_{ESL}, \hat{\vect{s}}_{ESL\_params}, \hat{\theta}_{ESL})$.

% La estimación de los componentes de esta transformación se descompone en tareas de inferencia acometidas por funciones discriminativas especializadas:

% \begin{itemize}
%     \item \textbf{Estimación de Límites de Contorno:}
%     Un conjunto de cuatro funciones discriminativas, $\{C_{L1}, C_{L2}, C_{L3}, C_{L4}\}$, se emplea para la detección de las posiciones óptimas de cuatro líneas canónicas: $L_1$ (línea vertical izquierda, correspondiente a una abscisa $x_{L1}$), $L_2$ (línea horizontal superior, ordenada $y_{L2}$), $L_3$ (línea vertical derecha, abscisa $x_{L3}$), y $L_4$ (línea horizontal inferior, ordenada $y_{L4}$). Estas líneas definen una caja delimitadora $B_{img}$ (AABB: Axis-Aligned Bounding Box), alineada con los ejes de la imagen, que se presume encierra la estructura de interés.

%     Para cada línea $L_k$ (donde $k \in \{1,2,3,4\}$), su función discriminativa asociada $C_{Lk}$ opera sobre una colección de subregiones o parches $\Omega_j \subset I(\vect{x})$, de dimensiones predefinidas $N \times N$ píxeles. Cada parche $\Omega_j$ se extrae a lo largo de una trayectoria de búsqueda discreta $\mathcal{P}_k$, definida perpendicularmente a la orientación esperada de la línea $L_k$. Por ejemplo, para $L_1$ (vertical), $\mathcal{P}_1$ es un conjunto de abscisas candidatas $\{x_j\}$ evaluadas a una ordenada fija $y_c$ (e.g., el centro vertical de la imagen), dentro de un rango de búsqueda $[x_{\text{search\_min}}, x_{\text{search\_max}}]$ determinado en función de las dimensiones de $I$.
%     La función $C_{Lk}$ asigna a cada parche $\Omega_j(pos_j)$, extraído en la posición $pos_j \in \mathcal{P}_k$, una puntuación $f_{Lk}(\Omega_j(pos_j)) \in [0, 1]$ que cuantifica la verosimilitud de que la línea $L_k$ esté presente en $\Omega_j$. La posición óptima estimada para la línea $L_k$, denotada $\widehat{pos}_{Lk}$, se obtiene como:
%     \begin{equation}
%     \hspace*{\fill}
%     \widehat{pos}_{Lk} = \arg\max_{pos_j \in \mathcal{P}_k} \{ f_{Lk}(\Omega_j(pos_j)) \}.
%     \label{eq:esl_pos_opt_mat}
%     \hspace*{\fill}
%     \end{equation}
%     Una vez estimadas las posiciones de las cuatro líneas $\{\hat{x}_{L1}, \hat{y}_{L2}, \hat{x}_{L3}, \hat{y}_{L4}\}$, los parámetros de traslación y las dimensiones de la caja $B_{img}$ se derivan:
%     \begin{align}
%     \hat{t}_x &= (\hat{x}_{L1} + \hat{x}_{L3}) / 2, \label{eq:esl_tx_estimate} \\
%     \hat{t}_y &= (\hat{y}_{L2} + \hat{y}_{L4}) / 2, \label{eq:esl_ty_estimate} \\
%     \hat{w}_{B} &= \hat{x}_{L3} - \hat{x}_{L1}, \label{eq:esl_wB_mat} \\
%     \hat{h}_{B} &= \hat{y}_{L4} - \hat{y}_{L2}. \label{eq:esl_hB_mat}
%     \end{align}
%     Estos definen el centro estimado $\hat{\vect{t}}_{ESL} = (\hat{t}_x, \hat{t}_y)^\transpose$ y las dimensiones (ancho $\hat{w}_{B}$, alto $\hat{h}_{B}$) de la caja delimitadora $B_{img}$. Los factores de escala $s_x, s_y$ de la Ecuación~\eqref{eq:esl_transform_mat} se relacionan con estas dimensiones, usualmente mediante una normalización respecto a las dimensiones de un modelo de forma canónico de referencia ($s_x = \hat{w}_B / w_{\text{ref}}$, $s_y = \hat{h}_B / h_{\text{ref}}$). El proceso esquemático para la estimación de los límites de contorno se ilustra en la Figura~\ref{fig:esl_line_detection_mat}.

% \begin{figure}[htbp] % Posicionamiento flexible
%     \centering
%     \includegraphics[width=\columnwidth]{Figures/fig_esl_line_detection.png}
%     \caption{Proceso esquemático de estimación de la caja delimitadora mediante funciones discriminativas de línea. Se ilustran las trayectorias de búsqueda y los parches muestreados en la imagen de entrada para las líneas horizontales y verticales. También se muestra un ejemplo del perfil de puntuaciones (verosimilitud) obtenido por una función $C_{Lk}$ a lo largo de su trayectoria de búsqueda, donde el máximo indica la posición óptima estimada. Finalmente, se visualiza la caja delimitadora resultante, definida por los parámetros $(\hat{\vect{t}}_{ESL}, \hat{w}_B, \hat{h}_B)$, superpuesta en la imagen original.}
%     \label{fig:esl_line_detection_mat}
% \end{figure}

%     \item \textbf{Estimación de Orientación:}
%     Una función discriminativa adicional, $C_{\theta}$, se emplea para estimar el ángulo de rotación global $\hat{\theta}_{ESL}$. Esta función opera sobre una región de interés $\Omega_{\theta}$ extraída de la imagen $I$, centrada en $\hat{\vect{t}}_{ESL}$ y cuyas dimensiones son proporcionales a $(\hat{w}_{B}, \hat{h}_{B})$ (e.g., escaladas por un factor $\kappa > 1$ para asegurar la inclusión completa del objeto bajo rotación).
%     Se genera un conjunto de parches candidatos $\{\Omega_{\theta}^{(r)}\}_{r=1}^M$ rotando la región $\Omega_{\theta}$ mediante un conjunto discreto de ángulos de prueba $\{\alpha_r\}_{r=1}^M$, donde $\alpha_r \in [-\alpha_{\text{max}}, \alpha_{\text{max}}]$. Cada parche rotado $\Omega_{\theta}^{(r)}$ es reescalado a las dimensiones $N \times N$ píxeles. La función $C_{\theta}$ asigna una puntuación $f_{\theta}(\Omega_{\theta}^{(r)})$ a cada parche. El ángulo de orientación estimado $\hat{\theta}_{ESL}$ es aquel que maximiza esta puntuación:
%     \begin{equation}
%     \hspace*{\fill}
%     \hat{\theta}_{ESL} = \arg\max_{\alpha_r \in \{\alpha_j\}} \{ f_{\theta}(\Omega_{\theta}^{(r)}(\alpha_r)) \}.
%     \label{eq:esl_theta_opt_mat}
%     \hspace*{\fill}
%     \end{equation}
%     El procedimiento para la estimación de la orientación se detalla esquemáticamente en la Figura~\ref{fig:esl_orientation_detection_mat}.

% \begin{figure}[htbp] % Posicionamiento flexible
%     \centering
%     \includegraphics[width=0.8\columnwidth]{Figures/fig_esl_orientation_detection.png} 
%     \caption{Proceso esquemático de estimación de la orientación. Se extrae una región de interés $\Omega_{\theta}$ de la imagen, centrada y escalada según la AABB previamente estimada. Se genera un conjunto de parches candidatos mediante la rotación de $\Omega_{\theta}$ a través de un rango de ángulos discretos. Se muestra un ejemplo del perfil de puntuaciones (verosimilitud) de la función $C_{\theta}$ para cada ángulo de rotación evaluado, donde el máximo indica el ángulo $\hat{\theta}_{ESL}$ estimado. Finalmente, se visualiza la pose final estimada $\mathcal{T}_{ESL}$ (caja delimitadora rotada) superpuesta en la imagen original.}
%     \label{fig:esl_orientation_detection_mat}
% \end{figure}
% \end{itemize}

% La construcción (entrenamiento) de las funciones discriminativas $C_{Lk}$ y $C_{\theta}$ se realiza mediante un proceso de aprendizaje supervisado. Se parte de un conjunto de $N_S$ muestras de entrenamiento, donde cada muestra $i$ consiste en una imagen $I_i$ y la representación de la forma de interés mediante $N_P$ puntos de referencia (landmarks), $\{\vect{p}_{i,j}^*\}_{j=1}^{N_P}$, que definen la verdad terreno (ground truth).

% Para cada muestra de entrenamiento $i$:
% \begin{enumerate}
%     \item Se calcula la caja delimitadora alineada a los ejes (AABB) ground truth, $B_{gt,i}$, a partir de sus landmarks $\{\vect{p}_{i,j}^*\}$. Las coordenadas de sus límites $(x_{\text{min},i}, y_{\text{min},i}, x_{\text{max},i}, y_{\text{max},i})$ definen las posiciones verdaderas $L_{k,gt,i}$ para las cuatro líneas.
%     \item Para el entrenamiento de cada $C_{Lk}$:
%         \begin{itemize}
%             \item Se extraen parches \textit{positivos} $\Omega_{\text{pos}}$ de $I_i$. El centro de $\Omega_{\text{pos}}$ se sitúa en la posición $L_{k,gt,i}$ (con la otra coordenada del centro del parche tomada del centro de $B_{gt,i}$). Estos parches se asocian a una etiqueta de clase positiva (e.g., $y=1$).
%             \item Se extraen parches \textit{negativos} $\Omega_{\text{neg}}$ de $I_i$. Estos se obtienen de posiciones $pos_{\text{neg}}$ a lo largo de la trayectoria de búsqueda $\mathcal{P}_k$, tales que la distancia entre $pos_{\text{neg}}$ y $L_{k,gt,i}$ sea mayor que un umbral $\delta_L$. Dicho umbral $\delta_L$ es típicamente proporcional a la dimensión relevante de $B_{gt,i}$. Estos parches se asocian a una etiqueta de clase negativa (e.g., $y=0$).
%         \end{itemize}
%     \item Para el entrenamiento de $C_{\theta}$:
%         \begin{itemize}
%             \item La orientación ground truth de la AABB, $\theta_{gt,i}$, se considera $0$ radianes por definición (alineada con los ejes).
%             \item Se extrae una región base $\Omega_{\theta,\text{base}}$ de $I_i$, centrada en el centro de $B_{gt,i}$ y con dimensiones que aseguren contener $B_{gt,i}$ con un margen contextual.
%             \item Para cada ángulo de rotación de prueba $\alpha_r$ de un conjunto discreto, se rota $\Omega_{\theta,\text{base}}$ por $\alpha_r$ para obtener $\Omega_{\theta}^{(r)}$.
%             \item El parche $\Omega_{\theta}^{(r)}$ se etiqueta como positivo si $|\alpha_r - \theta_{gt,i}| \le \delta_{\theta}$ (donde $\delta_{\theta}$ es un umbral angular pequeño, y $\theta_{gt,i}=0$), y negativo en caso contrario.
%         \end{itemize}
% \end{enumerate}
% Todos los parches extraídos son reescalados a dimensiones $N \times N$ y sometidos a un proceso de normalización de intensidad antes de ser utilizados para entrenar las funciones discriminativas.

% En la fase de inferencia, dada una nueva imagen $I(\vect{x})$, las funciones $C_{L1}-C_{L4}$ y $C_{\theta}$ se aplican secuencialmente como se describió para estimar $(\hat{\vect{t}}_{ESL}, \hat{w}_{B}, \hat{h}_{B}, \hat{\theta}_{ESL})$. Estos parámetros, junto con la relación entre $(\hat{w}_{B}, \hat{h}_{B})$ y los factores de escala $s_x, s_y$ (que parametrizan $\mat{S}_{ESL}$ en la Ecuación~\eqref{eq:esl_transform_mat}), definen completamente la transformación $\mathcal{T}_{ESL}$. Esta transformación proporciona una estimación global robusta de la pose del objeto, fundamental para inicializar subsecuentes procesos de ajuste fino del modelo de forma.

% \begin{figure}[htbp] % Posicionamiento flexible
%     \centering
%     \includegraphics[width=1\columnwidth]{Figures/visualizacion_entrenamiento_esl.png} 
%     \caption{Estimación de Pose Inicial (ESL): Ilustración del proceso de generación de datos de entrenamiento para sus clasificadores. (Arriba-Izquierda) Imagen de muestra $I_i$ con sus landmarks ground truth $\{\vect{p}_{i,j}^*\}$ y la AABB $B_{gt,i}$ derivada. (Arriba-Centro, Arriba-Derecha) Ejemplos de un parche positivo $\Omega_{\text{pos}}$ y uno negativo $\Omega_{\text{neg}}$ para el clasificador de línea $C_{L1}$, extraídos en la posición $L_{1,gt}$ y a una distancia $>\delta_L$, respectivamente. (Abajo-Izquierda) Región base $\Omega_{\theta,\text{base}}$ para el clasificador de orientación $C_{\theta}$. (Abajo-Centro, Abajo-Derecha) Ejemplos de parches rotados $\Omega_{\theta}^{(r)}$ generados a partir de $\Omega_{\theta,\text{base}}$, con una rotación pequeña (etiqueta positiva, e.g., $\alpha_r \approx 0$) y una mayor (etiqueta negativa).}
%     \label{fig:esl_training_data_generation}
% \end{figure}

\section{Estimación de Pose Inicial (ESL)}
\label{sec:esl_simplified}

Una estimación inicial precisa de la pose (posición, orientación y escala) de un modelo de forma es crucial para el ajuste robusto del modelo a una imagen. La Estimación de Pose Inicial (ESL) busca determinar una transformación afín global $\mathcal{T}_{ESL}$ que alinee el modelo con el objeto en la imagen.

Esta transformación mapea puntos $\vect{p}$ del modelo a puntos $\vect{x}$ en la imagen:
\begin{equation}
\vect{x} = \mathcal{T}_{ESL}(\vect{p}) = \mat{S}_{ESL} \mat{R}(\theta_{ESL}) \vect{p} + \vect{t}_{ESL},
\label{eq:esl_transform_simplified}
\end{equation}
donde $\mat{S}_{ESL}$ es una matriz de escala (con factores $s_x, s_y$), $\mat{R}(\theta_{ESL})$ es una matriz de rotación (con ángulo $\theta_{ESL}$), y $\vect{t}_{ESL}$ es un vector de traslación (con componentes $t_x, t_y$). El objetivo de ESL es estimar estos parámetros: $(\hat{\vect{t}}_{ESL}, \hat{s}_x, \hat{s}_y, \hat{\theta}_{ESL})$.

\begin{figure}[htbp]
    \centering
    \includegraphics[width=0.7\columnwidth]{Figures/fig_esl_line_detection.png}
    \caption{Estimación de la caja delimitadora mediante funciones de línea. Se buscan las posiciones óptimas para las líneas delimitadoras, definiendo el centro y las dimensiones de la caja.}
    \label{fig:esl_line_detection_simplified}
\end{figure}

La estimación se realiza mediante funciones discriminativas especializadas:

\begin{itemize}
    \item \textbf{Estimación de Límites de Contorno (para $\hat{\vect{t}}_{ESL}, \hat{s}_x, \hat{s}_y$):}
    Se utilizan funciones para detectar cuatro líneas (izquierda, superior, derecha, inferior) que definen una caja delimitadora (Bounding Box) alrededor del objeto en la imagen. Cada función $C_{Lk}$ busca la posición óptima $\widehat{pos}_{Lk}$ para su línea $L_k$ evaluando parches de imagen:
    \begin{equation}
    \widehat{pos}_{Lk} = \arg\max_{pos_j} \{ f_{Lk}(\Omega_j(pos_j)) \}.
    \label{eq:esl_pos_opt_simplified}
    \end{equation}
    A partir de estas posiciones ($\hat{x}_{L1}, \hat{y}_{L2}, \hat{x}_{L3}, \hat{y}_{L4}$), se calculan el centro de la caja (que da $\hat{\vect{t}}_{ESL}$) y sus dimensiones (ancho $\hat{w}_{B}$, alto $\hat{h}_{B}$), que se usan para estimar los factores de escala $\hat{s}_x, \hat{s}_y$.
    \begin{align}
    \hat{t}_x &= (\hat{x}_{L1} + \hat{x}_{L3}) / 2 \label{eq:esl_tx_estimate_simplified} \\
    \hat{t}_y &= (\hat{y}_{L2} + \hat{y}_{L4}) / 2 \label{eq:esl_ty_estimate_simplified} \\
    \hat{w}_{B} &= \hat{x}_{L3} - \hat{x}_{L1} \label{eq:esl_wB_simplified} \\
    \hat{h}_{B} &= \hat{y}_{L4} - \hat{y}_{L2} \label{eq:esl_hB_simplified}
    \end{align}

    \item \textbf{Estimación de Orientación (para $\hat{\theta}_{ESL}$):}
    Otra función $C_{\theta}$ estima el ángulo de rotación $\hat{\theta}_{ESL}$. Opera sobre una región de la imagen centrada según la caja delimitadora. Se evalúan varias rotaciones de esta región:
    \begin{equation}
    \hat{\theta}_{ESL} = \arg\max_{\alpha_r} \{ f_{\theta}(\Omega_{\theta}^{(r)}(\alpha_r)) \}.
    \label{eq:esl_theta_opt_simplified}
    \end{equation}
    El ángulo $\alpha_r$ que maximiza la puntuación de la función $f_{\theta}$ es la orientación estimada.

\begin{figure}[htbp]
    \centering
    \includegraphics[width=0.6\columnwidth]{Figures/fig_esl_orientation_detection.png} 
    \caption{Estimación de la orientación. Se evalúan rotaciones de una región de interés para encontrar el ángulo que mejor se ajusta según una función discriminativa.}
    \label{fig:esl_orientation_detection_simplified}
\end{figure}
\end{itemize}

Las funciones discriminativas ($C_{Lk}, C_{\theta}$) se entrenan previamente con imágenes de ejemplo. En la fase de inferencia, estas funciones estiman los parámetros de pose $(\hat{\vect{t}}_{ESL}, \hat{s}_x, \hat{s}_y, \hat{\theta}_{ESL})$, definiendo la transformación $\mathcal{T}_{ESL}$ para una inicialización robusta del modelo de forma.