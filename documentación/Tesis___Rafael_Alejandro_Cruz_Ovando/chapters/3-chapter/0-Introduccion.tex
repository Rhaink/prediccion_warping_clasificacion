\chapter{Metodología}
\label{cap:metodologia}

\section{Introducción}
\label{sec:intro_metodologia}

El Capítulo~\ref{cap:marco_teorico} estableció los fundamentos teóricos de las arquitecturas residuales profundas, específicamente la familia ResNet \cite{He2016}, y demostró formalmente la superioridad de funciones de pérdida especializadas como \textit{Wing Loss} \cite{Feng2018} para tareas de localización de \textit{landmarks} (puntos de referencia anatómicos) con precisión sub-píxel. Asimismo, se fundamentó teóricamente el paradigma de \textit{transfer learning} (aprendizaje por transferencia) \cite{Raghu2019, Tajbakhsh2016} como estrategia óptima para dominio médico con datos limitados, y se presentó el marco matemático de restricciones geométricas aplicadas a predicción de estructuras anatómicas. El presente capítulo constituye la transición de teoría a práctica: describe exhaustivamente la metodología experimental implementada para desarrollar el sistema de detección automática de 15 \textit{landmarks} anatómicos en radiografías de tórax, especificando cada decisión de diseño arquitectural, configuración de hiperparámetros, protocolo de entrenamiento progresivo, y estrategia de evaluación rigurosa.

La metodología desarrollada aborda el problema de regresión de coordenadas mediante una arquitectura neuronal profunda que implementa \textit{coordinate regression} (regresión directa de coordenadas) en lugar de generación de \textit{heatmaps} (mapas de calor espaciales), evitando así la costosa decodificación espacial y aprovechando eficientemente la representación compacta de coordenadas $(x, y) \in [0,1]^2$ para cada \textit{landmark}. % Como se argumentó en la Sección~\ref{sec:coordinate_vs_heatmap} del marco teórico (NOTA: label no existe en Cap 2 actual)
Esta elección metodológica permite procesamiento en tiempo real, reduce demandas de memoria computacional, y facilita la incorporación de restricciones geométricas globales mediante formulaciones de pérdida diferenciables. La arquitectura seleccionada, ResNet-18 preentrenada en ImageNet con un módulo de regresión especializado de tres capas completamente conectadas, fue diseñada para balancear capacidad representacional, eficiencia computacional en hardware de consumo general (GPU con 8GB VRAM), y facilidad de entrenamiento mediante \textit{transfer learning}. El protocolo de entrenamiento progresivo en cuatro fases incorpora gradualmente restricciones geométricas inspiradas en conocimiento anatómico humano: simetría bilateral del tórax, preservación de distancias anatómicas críticas, y consistencia estructural, transformando conocimiento anatómico cualitativo en restricciones cuantificables mediante funciones de pérdida diferenciables.

El objetivo metodológico central es alcanzar el estándar de excelencia clínica establecido internacionalmente para sistemas automáticos de detección de \textit{landmarks} anatómicos: error medio inferior a 8.5 píxeles en imágenes de 224$\times$224 píxeles, umbral definido por Payer et al. \cite{Payer2016} basándose en análisis de variabilidad inter-observador entre radiólogos expertos. Este umbral distingue sistemas de precisión suficiente para asistencia diagnóstica real de aquellos limitados a investigación académica. La metodología busca no solo minimizar error de localización, sino garantizar coherencia geométrica: predicciones anatómicamente válidas que respeten simetría bilateral, preserven proporciones estructurales, y mantengan ordenamiento espacial fisiológico (ápices pulmonares superiores a bases, estructuras mediastínicas centradas), aspectos críticos para aceptabilidad clínica que métricas de error puntual aisladas no capturan.

La estructura del presente capítulo organiza la documentación metodológica en siete componentes esenciales. La Sección~\ref{sec:dataset} caracteriza exhaustivamente el conjunto de datos de 956 radiografías de tórax en proyección posteroanterior con anotaciones expertas de 15 \textit{landmarks}, describiendo protocolo de adquisición, definición anatómica de cada punto de referencia, identificación de pares simétricos bilaterales, cálculo del eje de simetría mediastínico, protocolo de división estratificada para conjuntos de entrenamiento, validación y prueba, y validación de calidad de anotaciones. La Sección~\ref{sec:arquitectura} detalla la arquitectura neuronal implementada: modificaciones realizadas a ResNet-18 estándar, diseño del módulo de regresión especializado con \textit{dropout} (regularización estocástica) progresivo, distribución de parámetros entrenables, y justificación de cada elección arquitectural. La Sección~\ref{sec:pipeline_datos} especifica el \textit{pipeline} (secuencia de procesamiento) completo de transformación de datos: conversión de espacio de color, redimensionamiento con compensación de coordenadas, normalización según estadísticas de ImageNet, y protocolo de \textit{data augmentation} (aumentación de datos) geométrico compatible con \textit{landmarks}, implementado mediante transformaciones afines que preservan correspondencias punto-a-punto.

La Sección~\ref{sec:estrategia_entrenamiento} constituye el núcleo metodológico: presenta la estrategia de entrenamiento progresivo en cuatro fases que incorpora restricciones geométricas gradualmente. La Fase 1 adapta únicamente el módulo de regresión preservando representaciones preentrenadas del \textit{backbone} (extractor de características). La Fase 2 optimiza todos los parámetros mediante \textit{fine-tuning} (ajuste fino) con tasas de aprendizaje diferenciadas, introduciendo \textit{Wing Loss} para precisión sub-píxel. La Fase 3 incorpora \textit{Symmetry Loss} (pérdida de simetría) que penaliza inconsistencias bilaterales. La Fase 4 implementa la función de pérdida completa, agregando \textit{Distance Preservation Loss} (pérdida de preservación de distancias) que mantiene proporciones anatómicas críticas. Cada fase se documenta con hiperparámetros específicos (tasas de aprendizaje, tamaño de \textit{batch}, número de épocas, parámetros de regularización), función de pérdida matemáticamente formalizada, y estrategia de inicialización mediante \textit{warm-start} (inicialización con pesos de fase previa). La Sección~\ref{sec:implementacion} documenta detalles de reproducibilidad: \textit{frameworks} (entornos de desarrollo) y librerías específicas empleadas \cite{Paszke2019, Bradski2000, Pedregosa2011}, especificaciones de \textit{hardware} utilizado, tiempo de entrenamiento por fase, y configuración de semillas aleatorias para reproducibilidad determinística completa. La Sección~\ref{sec:metricas_eval} define matemáticamente las métricas de evaluación implementadas: Error Radial Medio en píxeles como métrica principal, métricas geométricas complementarias (error de simetría, consistencia bilateral, validez anatómica), y sistema de clasificación por calidad clínica basado en umbrales internacionales. Finalmente, la Sección~\ref{subsec:validacion} establece el protocolo de validación experimental riguroso: separación estricta de conjuntos de datos, estrategia de \textit{early stopping} (detención temprana) basada en validación, evaluación final sobre conjunto de prueba completamente no visto, y análisis por subgrupos diagnósticos para evaluar robustez ante variabilidad patológica.

Esta organización metodológica permite reproducibilidad completa del trabajo: cada parámetro, ecuación, transformación, y decisión de diseño está especificada con precisión suficiente para replicación independiente. La transparencia metodológica es requisito fundamental para validación científica y eventual traducción clínica de sistemas basados en aprendizaje profundo aplicados a diagnóstico médico. Los protocolos documentados en este capítulo constituyen la base experimental para los resultados presentados en el Capítulo~\ref{cap:resultados}.

La siguiente sección describe detalladamente el conjunto de datos empleado, componente fundamental que determina tanto las capacidades como las limitaciones del sistema desarrollado.
