\section{Adquisición y Representación Inicial de Datos Geométricos}
\label{subsec:AdquisicionDatos}

El punto de partida es la representación discreta de los contornos pulmonares mediante un conjunto de puntos de referencia, o \textit{landmarks}. Para cada imagen, se dispone de un conjunto inicial de $k = 15$ \textit{landmarks}. Cada \textit{landmark} $\mathbf{p}_i$ se define por sus coordenadas en el plano de la imagen:

\begin{equation}
\label{eq:landmark_definition}
\hspace*{\fill} \mathbf{p}_i = (x_i, y_i) \in \mathbb{R}^2, \quad \text{donde } i = 1, \dots, k. \hspace*{\fill}
\end{equation}

Estos \textit{landmarks} definen puntos clave a lo largo de los contornos de interés. Consecuentemente, dichos puntos se organizan en una matriz de forma, $\mathbf{S}$, para cada instancia, definida como:

\begin{equation}
\label{eq:shape_matrix}
\hspace*{\fill}
\mathbf{S} = \begin{bmatrix} x_1 & y_1 \\ x_2 & y_2 \\ \vdots & \vdots \\ x_k & y_k \end{bmatrix} % Cambio k_0 por k
\hspace*{\fill}
\end{equation}

Esta matriz $\mathbf{S}$, como se muestra en la Ecuación~\eqref{eq:shape_matrix}, encapsula la configuración geométrica de los \textit{landmarks} (definidos en la Ecuación~\eqref{eq:landmark_definition}) para una imagen específica.

\begin{figure}[htbp] 
\centering 
\includegraphics[width=1\linewidth]{Figures/initial_landmarks_visualization.png} 
\caption{Representación inicial de los contornos pulmonares mediante 15 \textit{landmarks} (ver Ecuación~\eqref{eq:landmark_definition}). Estos puntos discretos sirven como base para la representación de forma.}
\label{fig:initial_landmarks_visualization} 
\end{figure}