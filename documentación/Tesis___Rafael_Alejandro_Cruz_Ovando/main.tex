\documentclass[12pt,fleqn,openany,twoside,letterpape]{book}
\usepackage[spanish]{babel}
\usepackage{todonotes}
\usepackage{subcaption}
%----------------------------------------------------------------------------------------
%       CONFIGURACIÓN BÁSICA DE LA TESIS
%----------------------------------------------------------------------------------------

% Bibliografía
\usepackage[
    citestyle=ieee,
    bibstyle=ieee,
    style=numeric-comp,
    sorting=none,
    doi=false,
    url=true,
    maxbibnames=99,
]{biblatex}

% Paquetes esenciales
\usepackage{amsmath}
\usepackage{amssymb}
\usepackage{bm}
\usepackage{enumitem}

% Geometría y layout
\usepackage[letterpaper, margin=2.5cm, centering]{geometry}
\usepackage{graphicx}
\usepackage{setspace}
\usepackage{float}

% Tablas
\usepackage{booktabs}
\usepackage{tabularx}
\usepackage{multirow}
\usepackage{array}
\usepackage{longtable}
\usepackage{makecell}

% Headers y footers
\usepackage{fancyhdr}

% Enlaces
\usepackage[hidelinks]{hyperref}
\usepackage{nameref}

% Párrafos
\usepackage[parfill]{parskip}

% Caption
\usepackage{caption}
\usepackage{subcaption}
\captionsetup{font=small}

% Títulos
\usepackage{titlesec}

% Formato de capítulos
\titleformat{\chapter}[display]
  {\normalfont\sffamily\huge\bfseries}
  {Capítulo \thechapter}{17pt}{\Huge}

\titlespacing*{\chapter}{0pt}{0pt}{10pt}

% Contadores
\usepackage{chngcntr}
\counterwithin{figure}{section}
\counterwithin{table}{section}

% Nombres en español
\renewcommand{\figurename}{Figura}
\addto\captionsspanish{\renewcommand{\tablename}{Tabla}}

% TOC
\usepackage{tocloft}
\renewcommand*\contentsname{Tabla de Contenido}

% Estilos de página
\fancypagestyle{myfancy}{%
    \fancyhf{}
    \fancyhead[OL]{\selectfont\leftmark}
    \fancyhead[OR]{\selectfont\thepage}
    \fancyhead[ER]{\selectfont\rightmark}
    \fancyhead[EL]{\selectfont\thepage}
    \setlength{\headheight}{15.5pt}
    \renewcommand{\headrulewidth}{0.4pt}
}

\fancypagestyle{plain}{
    \fancyhf{}
    \setlength{\headheight}{0em}
    \renewcommand{\headrulewidth}{0pt}
    \fancyfoot[C]{\selectfont\thepage}
}

% Evitar hyphenation excesiva
\pretolerance=10000
\tolerance=2000
\emergencystretch=50pt

% Numeración de ecuaciones por sección
\numberwithin{equation}{section}


% Defining files for bibliography
\addbibresource{reference.bib}
% Add a second bibliography file for the second author to allow
% both to update it through the mendeley integration.

% Defining document information
\title{Template}
\newcommand{\subtitle}{Tesis}

\author{Rafael Alejandro Cruz Ovando>}
% Asegúrate de que estos comandos están definidos en tu preámbulo:

% Definir comandos personalizados si es necesario (ej. para notación específica)
\newcommand{\vectornorm}[1]{\left\|#1\right\|}
\newcommand{\matrixnorm}[1]{\left\|#1\right\|_F} % Norma de Frobenius
\newcommand{\mean}[1]{\overline{#1}}
\newcommand{\transpose}[1]{{#1}^T} % Transpuesta
\newcommand{\pseudoinverse}[1]{{#1}^+} % Pseudoinversa (si se usa)
\newcommand{\trace}[1]{\text{Tr}(#1)} % Traza
\newcommand{\determinant}[1]{\det(#1)} % Determinante
\newcommand{\mat}[1]{\mathbf{#1}}   
\newcommand{\vect}[1]{\bm{#1}} % Requiere \usepackage{bm}
\newcommand{\R}{\mathbb{R}} 
\newcommand{\Ktotal}{K_{\text{total}}} % Si definiste K_total como macro (e.g., \newcommand{\Ktotal}{144})
\newcommand{\dval}{d} % Si definiste d para la dimensionalidad (e.g., \newcommand{\dval}{2})
\newcommand{\matS}{\mathbf{S}}
\newcommand{\vecs}{\mathbf{s}}
\newcommand{\vecp}{\mathbf{p}}
\newcommand{\matM}{\mathbf{M}}
\newcommand{\matR}{\mathbf{R}}
\newcommand{\matI}{\mathbf{I}}
\newcommand{\matC}{\mathbf{C}}
\newcommand{\matU}{\mathbf{U}}
\newcommand{\matV}{\mathbf{V}}
\newcommand{\matsigma}{\bm{\Sigma}} % Usar bm para griego en negrita
\newcommand{\vecuno}{\mathbf{1}}
\newcommand{\normF}[1]{\lVert#1\rVert_F} % Norma de Frobenius
\newcommand{\landmark}{\textit{landmark}}
\newcommand{\landmarks}{\textit{landmarks}}
\newcommand{\matSigma}{\bm{\Sigma}} % Usar bm para griego en negrita
\newcommand{\set}[1]{\mathcal{#1}} % Para conjuntos
\newfloat{tableH}{thp}{lop} % Para el entorno tableH
\DeclareMathOperator*{\argmin}{arg\,min} % Argmin
\DeclareMathOperator{\tr}{tr}           % Traza de una matriz
\DeclareMathOperator{\SO}{SO}

\begin{document}

\setstretch{1.25}

% The front page of the document
\pagenumbering{Roman}
%----------------------------------------------------------------------------------------
%       PÁGINA DE TÍTULO PERSONALIZADA (Universidad)
%----------------------------------------------------------------------------------------
\makeatletter
\begin{titlepage}
    \begin{center}
    \includegraphics[width=1.9in]{Figures/Logo_de_la_BUAP.png}~\\[0.5cm]
    
    {\Large{BENEMÉRITA UNIVERSIDAD AUTÓNOMA DE PUEBLA}} \\[0.4cm]
    \Large{FACULTAD DE CIENCIAS DE LA ELECTRÓNICA}\\
    {\Large{MAESTRÍA EN INGENIERÍA ELECTRÓNICA,}} \\
    {\Large{OPCIÓN INSTRUMENTACIÓN ELECTRÓNICA}} \\[0.8cm]
    
    {\Large{Tesis para obtener el grado de:}} \\
    {\Large{MAESTRO EN INGENIERÍA ELECTRÓNICA}} \\
    \vspace{0.5cm}

    \hrule
    \vspace{15pt}
    {\Large Normalización y alineación automática de la forma de la región pulmonar integrada con selección de características discriminantes para detección de neumonía y COVID-19} 
    \vspace{15pt}
    \hrule
    \vspace{0.5cm}

    {\large{Presenta:}} \\
    {\large{Lic. Rafael Alejandro Cruz Ovando*}} \\
    \vspace{0.5cm}
    
    {\large{Directores:}} \\
    {\large{Dr. Salvador Eugenio Ayala Raggi}} \\
    {\large{Dr. Aldrin Barreto Flores}} \\
        
    \vfill
        
    \small *Becario SECIHTI \hfill Puebla, Pue., Noviembre 2025
    
    \end{center}
%     \begin{figure}
%     \centering
%     \includegraphics[width=1\linewidth]{imagen3.png}
% \end{figure}
\end{titlepage}
\makeatother
\include{setup/title-page-2}
%\include{sections/Glosario}
\newpage
\titleformat
{\chapter} % command
[display] % shape
{\normalfont\huge\bfseries} % format
%{\hfill}
{\hfill  Capítulo \thechapter} % Formato de encabezado al inicio del capitulo
{-2ex} % sep
{
    % \noindent
    \makebox[0pt][l]{\rule[.6ex]{\linewidth}{2.5pt}}%
    \rule[.3ex]{\linewidth}{.6pt}
    % \vspace{-0.5ex}
} % before-code
[
\vspace{-2ex}%
\rule{\textwidth}{.6pt}
] % after-code

\newpage

%\input{sections/0.1-acronyms}
%\printnomenclature

%\newpage

\etocdepthtag.toc{mtchapter}
\etocsettagdepth{mtchapter}{subsection}
\etocsettagdepth{mtappendix}{none}

\renewcommand{\contentsname}{Tabla de Contenido}
\tableofcontents
\newpage
%TC:endignorep
\setstretch{1.25}
\pagestyle{myfancy}
%\renewcommand{\nomname}{Lista de Nomenclatura}
% \nomenclature[A]{FCC}{Federal Communications Commission}
% \nomenclature[A]{UWB}{Ultra-wideband}
% \nomenclature[A]{CMOS}{Complementary Metal-Oxide-Semiconductor}
% \nomenclature[A]{RF}{Radio Frequency}
% \nomenclature[A]{IR-UWB}{Impulse-Radio Ultra-wideband}
% \nomenclature[A]{FM-UWB}{Frequency-Modulation Ultra-wideband}
% \nomenclature[A]{PSD}{Power Spectral Density}
% \nomenclature[A]{IoT}{Internet of Things}
% \nomenclature[A]{ECG}{Electrocardiogram}
% \nomenclature[A]{LO}{Local Oscillator}
% \nomenclature[A]{VCO}{Voltage-Controlled Oscillator}
% \nomenclature[A]{LNA}{Low Noise Amplifier}
% \nomenclature[A]{FSK}{Frequency Shift Keying}
% \nomenclature[A]{DDS}{Direct Digital Synthesis}
% \nomenclature[A]{PA}{Power Amplifiker}
% \nomenclature[A]{ASK}{Amplitude Shift Keying}
% \nomenclature[A]{PSK}{Phase Shift Keying}
% \nomenclature[A]{FSK}{Frequency Shift Keying}
% \nomenclature[A]{BFSK}{ABinary Frequency Shift Keying}
% %\nomenclature[A]{GMSK}{Gaussian Modulation Shift Keying}
% %\nomenclature[A]{VCO}{Voltage-Controlled Oscillator}
% %\nomenclature[P]{\(c\)}{Speed of light in a vacuum}
% %\nomenclature[N]{\(\mathbb{R}\)}{Real numbers}
% %\nomenclature[N]{\(\mathbb{C}\)}{Complex numbers}
% %\nomenclature[N]{\(\mathbb{H}\)}{Quaternions}
% %\nomenclature[C]{\(e\)}{2.71828}
% %\nomenclature[C]{\(\pi\)}{3.14159}
\addcontentsline{toc}{chapter}{Lista de Nomenclatura}
\printnomenclature
\renewcommand{\listfigurename}{Lista de Figuras}
\addcontentsline{toc}{chapter}{Lista de Figuras}
\listoffigures
%\renewcommand{\listtablename}{Lista de Tablas}
%\addcontentsline{toc}{chapter}{Lista de Tablas}
%\listoftables
\renewcommand{\listtablename}{Lista de Tablas}
\addcontentsline{toc}{chapter}{Lista de Tablas}
\listoftables
\pagenumbering{arabic}

% %Capitulo 1
\chapter{Introducción}

El análisis preciso de radiografías de tórax es fundamental en el diagnóstico médico, siendo estas imágenes una herramienta ampliamente utilizada debido a su disponibilidad y bajo costo. Sin embargo, la interpretación manual es un proceso subjetivo, susceptible a errores y variabilidad inter-observador, especialmente en contextos de alta demanda \cite{mansoor2015segmentation}. Para mejorar la objetividad y eficiencia, la investigación en diagnóstico médico asistido por computadora se ha enfocado en desarrollar herramientas automáticas que asistan en el análisis de estas imágenes. Un paso crucial en este proceso es la correcta identificación y delimitación de estructuras anatómicas relevantes, como los campos pulmonares y puntos de referencia clave.

Esta tesis aborda el desarrollo de un sistema automatizado para la detección de puntos de referencia (en adelante referidos como \textit{landmarks}) anatómicos  en radiografías de tórax, componente fundamental para el análisis cuantitativo de imágenes médicas. Se propone una metodología basada en aprendizaje profundo (\textit{deep learning}) que utiliza redes neuronales convolucionales (Convolutional Neural Networks, CNNs) que incorporan conocimiento anatómico del dominio médico mediante restricciones geométricas \cite{Litjens2017, Shen2017}. El sistema predice de manera directa las coordenadas de 15 puntos de referencia anatómicos clave, aprovechando aprendizaje por transferencia (\textit{transfer learning}) desde dominios de imágenes naturales \cite{Raghu2019}. Los resultados experimentales demuestran que el enfoque propuesto alcanza niveles de precisión que cumplen con los estándares internacionales de excelencia clínica establecidos para tareas de localización anatómica \cite{Payer2016}, validado sobre un conjunto de datos que incluye casos de COVID-19, neumonía viral y pacientes saludables.

La detección precisa de \textit{landmarks} anatómicos constituye un componente fundamental para el desarrollo futuro de sistemas completos de diagnóstico asistido por computadora. Los puntos de referencia detectados automáticamente proporcionan una base para posteriores etapas de análisis, incluyendo la segmentación automática de regiones anatómicas, la normalización geométrica de imágenes y la clasificación de patologías torácicas. Esta investigación se enfoca específicamente en la primera etapa: la localización robusta y precisa de \textit{landmarks} anatómicos mediante técnicas de aprendizaje profundo. Las líneas de investigación futuras derivadas de este trabajo incluyen el desarrollo de modelos de segmentación pulmonar, sistemas de normalización espacial y clasificadores de patologías que aprovechen los \textit{landmarks} detectados automáticamente.

Esta tesis se organiza de la siguiente manera: el Capítulo 2 presenta el marco teórico y el estado del arte en detección de \textit{landmarks} anatómicos y aprendizaje profundo aplicado a imágenes médicas; el Capítulo 3 detalla la metodología propuesta, incluyendo la arquitectura de red neuronal, las funciones de pérdida especializadas y las estrategias de entrenamiento; el Capítulo 4 describe el conjunto de datos utilizado, las métricas de evaluación y el protocolo experimental; el Capítulo 5 presenta los resultados obtenidos y su análisis comparativo; finalmente, el Capítulo 6 discute las conclusiones, limitaciones y líneas futuras de investigación.

% \begin{figure}
%     \centering
%     \includegraphics[width=1\linewidth]{0.png}
%     \caption{Diagrama de bloques del sistema propuesto para localización de landmarks y segmentación pulmonar.} % Modificado para reflejar el alcance real
%     \label{fig:enter-label}
% \end{figure}

\section{Planteamiento del problema}

La interpretación de radiografías de tórax representa uno de los procedimientos de diagnóstico más frecuentes en la práctica clínica a nivel mundial, con más de 2 mil millones de estudios realizados anualmente \cite{WHO2020}. La localización precisa de estructuras anatómicas clave mediante la identificación de \textit{landmarks} es fundamental para el análisis cuantitativo y la toma de decisiones clínicas \cite{Tang2019}. Estos \textit{landmarks} anatómicos permiten el cálculo de índices diagnósticos como el índice cardiotorácico, la detección de asimetrías patológicas y el establecimiento de sistemas de coordenadas consistentes para análisis longitudinales \cite{Sogancioglu2021}. Sin embargo, la anotación manual de \textit{landmarks} requiere aproximadamente 15 minutos por imagen y está sujeta a variabilidad inter e intra-observador de hasta 5-10 píxeles, limitando su aplicabilidad en escenarios clínicos de alto volumen \cite{Payer2016}.

Los enfoques tradicionales para la detección de \textit{landmarks} anatómicos se basan en métodos de visión por computadora que utilizan características diseñadas manualmente (\textit{hand-crafted features}) combinadas con modelos estadísticos de forma \cite{Cootes1995, Cootes2001}. Aunque estos métodos han demostrado efectividad en condiciones controladas, enfrentan limitaciones significativas: (1) requieren ingeniería manual de características específicas del dominio, proceso que resulta costoso y poco generalizable; (2) dependen de alineamientos geométricos previos (como el Análisis de Procrustes Generalizado, Generalized Procrustes Analysis, GPA) que pueden fallar ante deformaciones anatómicas severas; (3) modelan relaciones lineales mediante Análisis de Componentes Principales (Principal Component Analysis, PCA), incapaces de capturar la naturaleza no lineal de las variaciones anatómicas; y (4) presentan sensibilidad elevada a condiciones de imagen como bajo contraste, ruido y artefactos \cite{Shen2017, Heimann2009}.

El surgimiento del aprendizaje profundo (\textit{deep learning}) ha transformado radicalmente el análisis de imágenes médicas \cite{Litjens2017}, demostrando capacidad para aprender representaciones jerárquicas de características directamente desde datos sin necesidad de ingeniería manual \cite{Krizhevsky2012}. Las redes neuronales convolucionales (Convolutional Neural Networks, CNNs) han alcanzado niveles de desempeño comparables o superiores al de especialistas humanos en diversas tareas de imagenología médica \cite{Esteva2017, Gulshan2016}. Sin embargo, la detección precisa de landmarks anatómicos mediante CNNs presenta desafíos específicos que requieren soluciones especializadas más allá de las arquitecturas estándar de clasificación o segmentación.

El problema central que aborda esta investigación se formula de la siguiente manera: \textbf{¿Cómo diseñar un sistema automatizado basado en redes neuronales convolucionales que detecte \textit{landmarks} anatómicos en radiografías de tórax con precisión clínicamente útil (error $<$8.5 píxeles), incorporando conocimiento anatómico del dominio médico y resultando computacionalmente eficiente para integración en flujos de trabajo hospitalarios?}

Este problema general se descompone en los siguientes desafíos técnicos específicos:

\textbf{Desafío 1: Alta precisión en regresión de coordenadas.} A diferencia de tareas de clasificación donde pequeños errores son tolerables, la localización de \textit{landmarks} constituye un problema de regresión donde el modelo debe predecir coordenadas continuas $(x, y)$ con alta precisión para resultar clínicamente útil. Los estándares internacionales establecen que un error inferior a 8.5 píxeles representa excelencia clínica \cite{Payer2016}. Las funciones de pérdida estándar como el Error Cuadrático Medio (Mean Squared Error, MSE) tratan todos los errores de manera uniforme, penalizando excesivamente valores atípicos (\textit{outliers}) pero proporcionando gradientes insuficientes para refinar predicciones ya cercanas al objetivo. Este comportamiento dificulta el logro de la alta precisión requerida en aplicaciones clínicas \cite{Feng2018}.

\textbf{Desafío 2: Incorporación de conocimiento anatómico.} El cuerpo humano exhibe restricciones geométricas inherentes que no son explotadas por enfoques estándar de aprendizaje profundo. Específicamente, las radiografías de tórax presentan simetría bilateral aproximada entre pulmones izquierdo y derecho, relaciones de distancia fijas entre estructuras anatómicas (ancho torácico, altura mediastínica), y restricciones de ordenamiento espacial (los ápices pulmonares siempre se localizan superiormente a las bases) \cite{Donner2013}. Integrar explícitamente este conocimiento anatómico como restricciones geométricas en el proceso de optimización constituye un desafío metodológico no resuelto completamente en la literatura existente \cite{Thaler2021, Zeng2020}.

\textbf{Desafío 3: Generalización ante variabilidad patológica.} El sistema debe mantener precisión robusta en presencia de condiciones patológicas que alteran significativamente la apariencia radiográfica. Las manifestaciones de COVID-19 (opacidades en vidrio esmerilado, consolidaciones), neumonía viral (infiltrados intersticiales) y otras patologías torácicas pueden oscurecer parcialmente referencias anatómicas, reduciendo el contraste local y dificultando la localización precisa de \textit{landmarks} \cite{Jacobi2020, Wang2020COVID}. El modelo debe aprender representaciones suficientemente robustas para localizar estructuras anatómicas incluso cuando los límites no resultan claramente visibles.

\textbf{Desafío 4: Eficiencia computacional para despliegue clínico.} Para resultar práctico en entornos hospitalarios, el sistema debe ejecutar inferencia en equipo físico (\textit{hardware}) de consumo general (sin requerir GPUs de alta gama) en tiempos de respuesta cercanos al tiempo real (menos de 1 segundo por imagen). Esta restricción limita la complejidad arquitectural viable y motiva el uso de modelos eficientes con \textit{transfer learning} desde dominios de datos abundantes \cite{Raghu2019}.

\textbf{Desafío 5: Escasez de datos médicos etiquetados.} A diferencia de aplicaciones de visión por computadora en dominios generales donde existen millones de imágenes etiquetadas (ImageNet: 1.2M imágenes), los conjuntos de datos (\textit{datasets}) médicos típicamente contienen cientos o pocos miles de imágenes anotadas debido al costo y tiempo requerido para anotación experta \cite{Ker2018}. Esta escasez de datos incrementa el riesgo de sobreajuste (\textit{overfitting}) y limita la capacidad de generalización de modelos entrenados desde cero, motivando estrategias de \textit{transfer learning} y regularización especializada \cite{Tajbakhsh2016}.

La solución a estos desafíos interconectados requiere una aproximación metodológica que integre: (1) arquitecturas de redes neuronales eficientes con capacidad de extracción de características robustas, (2) funciones de pérdida especializadas diseñadas para alta precisión mediante amplificación de gradientes en el régimen de errores pequeños, (3) mecanismos de regularización geométrica que incorporen conocimiento anatómico del dominio médico, y (4) estrategias de optimización progresiva que balanceen convergencia rápida con precisión final. El Capítulo 3 presenta en detalle la metodología propuesta que aborda sistemáticamente cada uno de estos desafíos.

%\section{Hipótesis}
\label{sec:hipotesis}

La hipótesis central de esta investigación se formula de la siguiente manera:

\begin{quote}
\textit{La incorporación explícita de restricciones geométricas anatómicas (simetría bilateral y preservación de distancias) como componentes de la función de pérdida en redes neuronales convolucionales mejorará significativamente la precisión de detección de \textit{landmarks} anatómicos en radiografías de tórax, comparado con funciones de pérdida estándar basadas en Error Cuadrático Medio (Mean Squared Error, MSE), alcanzando niveles de excelencia clínica (error radial medio $<$8.5 píxeles) establecidos en la literatura internacional \cite{Payer2016}.}
\end{quote}

Esta hipótesis se sustenta en el principio fundamental de que el conocimiento del dominio anatómico, cuando se integra adecuadamente en los mecanismos de aprendizaje profundo mediante regularizaciones geométricas, supera en efectividad al incremento de complejidad arquitectural en tareas especializadas con datos limitados \cite{Donner2013, Thaler2021, Zeng2020}. Las restricciones geométricas actúan como sesgos inductivos que guían el proceso de optimización hacia soluciones anatómicamente plausibles, reduciendo el espacio de búsqueda y mejorando la generalización del modelo.

\subsection{Formulación Formal}

\textbf{Hipótesis nula ($H_0$):} La adición de restricciones geométricas (simetría bilateral y preservación de distancias anatómicas) a la función de pérdida no produce mejora estadísticamente significativa en la precisión de detección de \textit{landmarks}, comparado con el uso exclusivo de funciones de pérdida estándar para regresión de coordenadas.

\textbf{Hipótesis alternativa ($H_1$):} La incorporación de restricciones geométricas en la función de pérdida produce una reducción estadísticamente significativa del error radial medio de localización de \textit{landmarks}, permitiendo alcanzar el umbral de excelencia clínica de $<$8.5 píxeles.

\subsection{Variables y Predicciones Específicas}

\textbf{Variables independientes:}
\begin{itemize}
    \item Función de pérdida empleada: (1) MSE como línea base (\textit{baseline}), (2) Wing Loss \cite{Feng2018}, (3) Wing Loss + Symmetry Loss \cite{Donner2013}, (4) Wing Loss + Symmetry Loss + Distance Preservation Loss \cite{Thaler2021}.
    \item Estrategia de entrenamiento: congelamiento de columna vertebral de la red (en adelante referido como \textit{backbone}) vs. ajuste fino (\textit{fine-tuning}) completo \cite{Yosinski2014}.
    \item Arquitectura base: ResNet-18 con y sin mecanismos de atención espacial \cite{He2016, Hou2021}.
\end{itemize}

\textbf{Variables dependientes (métricas de evaluación):}
\begin{itemize}
    \item Error radial medio (píxeles) entre \textit{landmarks} predichos y anotaciones de referencia.
    \item Consistencia bilateral: error de simetría entre \textit{landmarks} correspondientes de pulmones izquierdo y derecho.
    \item Validez anatómica: desviación porcentual en distancias anatómicas críticas (ancho torácico, altura mediastínica).
    \item Porcentaje de predicciones con excelencia clínica (error $<$8.5 píxeles).
    \item Porcentaje de predicciones clínicamente útiles (error $<$15 píxeles).
\end{itemize}

\textbf{Predicciones específicas testeables:}

\begin{enumerate}
    \item La función de pérdida Wing Loss reducirá el error radial medio en al menos 10\% comparado con MSE \textit{baseline}, debido a su diseño específico para amplificar gradientes en el régimen de errores pequeños, permitiendo refinamiento iterativo de predicciones \cite{Feng2018}.

    \item La adición de Symmetry Loss mejorará la consistencia bilateral en al menos 15\%, reduciendo la asimetría artificial entre \textit{landmarks} correspondientes de ambos pulmones y forzando al modelo a respetar la simetría anatómica inherente \cite{Donner2013}.

    \item La incorporación de Distance Preservation Loss reducirá la desviación en distancias anatómicas críticas en al menos 20\%, garantizando que el modelo preserve relaciones espaciales fundamentales del tórax \cite{Thaler2021}.

    \item La función de pérdida completa (Wing Loss + Symmetry Loss + Distance Preservation Loss) logrará error radial medio $<$8.5 píxeles, cumpliendo con el estándar internacional de excelencia clínica para detección de \textit{landmarks} anatómicos \cite{Payer2016}.

    \item El modelo entrenado con restricciones geométricas demostrará robustez superior ante variabilidad patológica (COVID-19, neumonía viral), con degradación de desempeño $<$15\% comparado con casos normales \cite{Jacobi2020}.

    \item Los mecanismos de atención arquitecturales (Coordinate Attention) no proporcionarán mejora significativa comparados con restricciones geométricas en la función de pérdida, validando que el conocimiento del dominio es más efectivo que la complejidad arquitectural en tareas especializadas con datos limitados \cite{Hou2021}.
\end{enumerate}

\subsection{Alcance de la Validación Experimental}

La validación de esta hipótesis se realizará mediante:

\begin{itemize}
    \item \textbf{Diseño experimental controlado:} Entrenamiento de modelos con configuraciones sistemáticamente variadas (funciones de pérdida, estrategias de optimización, componentes arquitecturales) sobre el mismo conjunto de datos de 956 radiografías dividido consistentemente en conjuntos de entrenamiento (70\%), validación (15\%) y prueba (15\%).

    \item \textbf{Validación cruzada estratificada:} División que preserva distribución de categorías médicas (COVID-19, neumonía viral, normal) para garantizar representatividad.

    \item \textbf{Análisis estadístico riguroso:} Pruebas de significancia estadística (t-test pareado, ANOVA) para comparar desempeño entre configuraciones, con nivel de significancia $\alpha = 0.05$.

    \item \textbf{Evaluación multi-dimensional:} Métricas complementarias (error radial, consistencia bilateral, validez anatómica) que evalúan diferentes aspectos de la calidad de predicción.

    \item \textbf{Análisis de ablación:} Remoción sistemática de componentes (restricciones geométricas, \textit{fine-tuning}, aprendizaje por transferencia) para cuantificar la contribución individual de cada elemento metodológico.
\end{itemize}

La confirmación de esta hipótesis demostrará que la integración de conocimiento anatómico mediante restricciones geométricas constituye una estrategia efectiva y generalizable para mejorar la precisión de modelos de aprendizaje profundo en tareas de localización anatómica, estableciendo una metodología reproducible aplicable a otros problemas de análisis de imágenes médicas \cite{Litjens2017}.

\section{Justificación}
\label{sec:justificacion}

\subsection{Contexto Global y Necesidad Clínica}

Las radiografías de tórax constituyen el estudio de imagenología más frecuentemente realizado a nivel mundial, representando la primera línea de evaluación para enfermedades pulmonares y cardíacas en servicios de urgencias, unidades de cuidados intensivos y consulta ambulatoria. La localización precisa de estructuras anatómicas clave mediante la identificación de puntos de referencia (\textit{landmarks}) es esencial para el análisis cuantitativo, la toma de decisiones clínicas y el seguimiento longitudinal de pacientes \cite{Tang2019, Sogancioglu2021}.

El proceso tradicional de anotación manual de \textit{landmarks} anatómicos enfrenta limitaciones críticas en el contexto clínico contemporáneo: (1) el tiempo requerido resulta prohibitivo en escenarios de alta demanda; (2) la variabilidad inter e intra-observador afecta la reproducibilidad de mediciones cuantitativas \cite{Payer2016}; (3) la fatiga del observador incrementa errores en sesiones prolongadas de anotación; y (4) el crecimiento exponencial en volumen de estudios radiológicos supera la disponibilidad de especialistas capacitados, particularmente en regiones con recursos limitados. La pandemia de COVID-19 ha evidenciado dramáticamente esta brecha, con incrementos sostenidos en demanda de interpretación de radiografías torácicas que exceden la capacidad de respuesta del personal médico disponible \cite{Jacobi2020}.

\subsection{Estado del Arte y Limitaciones de Enfoques Existentes}

El aprendizaje profundo (\textit{deep learning}) ha transformado el análisis de imágenes médicas en la última década, alcanzando desempeño comparable o superior a especialistas humanos en diversas tareas de clasificación y segmentación \cite{Litjens2017}. Trabajos seminales han demostrado este potencial en dermatología \cite{Esteva2017}, oftalmología \cite{Gulshan2016} y radiología torácica \cite{Wang2020COVID}. Sin embargo, la detección precisa de \textit{landmarks} anatómicos mediante redes neuronales convolucionales presenta desafíos específicos que no se resuelven simplemente mediante el escalamiento de arquitecturas o el incremento de datos de entrenamiento.

Los métodos existentes para detección de \textit{landmarks} en radiografías de tórax exhiben limitaciones significativas: (1) enfoques basados en regresión de mapas de calor (\textit{heatmap regression}) requieren resolución espacial elevada y memoria computacional sustancial, dificultando su despliegue en equipo físico (\textit{hardware}) de consumo \cite{Newell2016}; (2) métodos de regresión coordinada (\textit{coordinate regression}) con funciones de pérdida estándar como el Error Cuadrático Medio (Mean Squared Error, MSE) no alcanzan los niveles de precisión requeridos para excelencia clínica (error < 8.5 píxeles) \cite{Zhang2014}; (3) sistemas que ignoran restricciones geométricas anatómicas producen predicciones anatómicamente implausibles (asimetrías artificiales, violación de relaciones espaciales fundamentales); y (4) la mayoría de trabajos reportan validación en conjuntos de datos (\textit{datasets}) con más de 10,000 imágenes, dejando sin resolver el problema de entrenamiento efectivo con \textit{datasets} médicos de tamaño limitado (típicamente cientos o pocos miles de imágenes) \cite{Ker2018}.

Revisiones sistemáticas de sistemas de inteligencia artificial para COVID-19 han identificado riesgo de sesgo elevado, falta de validación externa y reporte inadecuado de metodología en la mayoría de publicaciones \cite{Roberts2021, Wynants2020}. Estas limitaciones metodológicas subrayan la necesidad de investigación rigurosa que establezca estándares reproducibles para desarrollo y validación de sistemas de análisis automatizado de imágenes médicas.

\subsection{Contribución Científica y Técnica del Trabajo}

Esta tesis aborda las limitaciones identificadas mediante las siguientes contribuciones:

\textbf{1. Función de pérdida geométrica con conocimiento anatómico.} Se desarrolla una función de pérdida multi-componente que integra: (a) Wing Loss \cite{Feng2018} para mejora de precisión mediante amplificación de gradientes en el régimen de errores pequeños (refinando predicciones cercanas al objetivo), (b) Symmetry Loss para imponer simetría bilateral anatómica \cite{Donner2013}, y (c) Distance Preservation Loss para preservar relaciones espaciales críticas entre estructuras \cite{Thaler2021}. Esta integración de conocimiento del dominio médico mediante restricciones geométricas constituye una contribución metodológica que supera el enfoque tradicional de incrementar complejidad arquitectural.

\textbf{2. Estrategia de entrenamiento progresivo en cuatro fases.} Se propone una metodología sistemática que progresa desde congelamiento de columna vertebral de la red (\textit{backbone}) hasta optimización completa con restricciones geométricas incrementales. Esta estrategia permite convergencia estable y mejora progresiva del desempeño, demostrando ser superior al entrenamiento extremo a extremo (\textit{end-to-end}) directo en datasets médicos de tamaño limitado \cite{Yosinski2014, Raghu2019}.

\textbf{3. Validación empírica rigurosa.} El sistema se valida sobre 956 radiografías que incluyen casos de COVID-19, neumonía viral y pacientes normales, con evaluación multi-dimensional mediante métricas estándar (error radial medio) y métricas geométricas especializadas (consistencia bilateral, validez anatómica). El diseño experimental incluye análisis de ablación sistemático que cuantifica la contribución individual de cada componente metodológico, proporcionando evidencia empírica del valor de restricciones geométricas sobre complejidad arquitectural.

\textbf{4. Eficiencia computacional y reproducibilidad.} El sistema alcanza precisión de excelencia clínica con inferencia en menos de 1 segundo por imagen en \textit{hardware} de consumo (GPU de gama media con 8GB VRAM), demostrando viabilidad para despliegue en entornos con recursos limitados. Todo el código, configuraciones experimentales y resultados se documentan exhaustivamente para facilitar reproducción y validación independiente, abordando las deficiencias metodológicas identificadas en revisiones sistemáticas \cite{Roberts2021}.

\subsection{Impacto y Aplicaciones Potenciales}

Los \textit{landmarks} anatómicos detectados automáticamente por el sistema propuesto constituyen la base para una secuencia de procesamiento (en adelante referida como \textit{pipeline}) de análisis completa que permitirá, como trabajo futuro, desarrollar:

\textbf{1. Segmentación automática precisa.} Los 15 \textit{landmarks} pueden inicializar Modelos Activos de Forma (Active Shape Models, ASM) para delineación automatizada de contornos pulmonares con modelado de forma anatómicamente plausible \cite{Cootes1995, Heimann2009}.

\textbf{2. Normalización geométrica robusta.} Las coordenadas de \textit{landmarks} permiten calcular transformaciones geométricas que estandaricen pose, escala y orientación, eliminando variaciones extrínsecas y facilitando análisis cuantitativo reproducible.

\textbf{3. Extracción de ROI normalizadas.} Regiones de interés (Regions of Interest, ROI) estandarizadas geométricamente reducen variabilidad inter-sujeto no relacionada con patología, mejorando la sensibilidad y especificidad de análisis posteriores.

\textbf{4. Sistemas de clasificación de patologías.} Representaciones normalizadas pueden alimentar clasificadores de aprendizaje profundo para detección automática de neumonía, COVID-19 y otras patologías torácicas \cite{Wang2020COVID, Apostolopoulos2020}.

La metodología desarrollada no se limita a radiografías de tórax; es generalizable a otros problemas de localización anatómica en imágenes médicas donde existen restricciones geométricas inherentes (simetría en imágenes cerebrales, proporciones anatómicas en radiografías pediátricas, etc.). Esta generalización amplifica el impacto potencial del trabajo más allá del dominio específico de aplicación.

\subsection{Relevancia en el Contexto de COVID-19 y Salud Pública}

La pandemia de COVID-19 ha incrementado exponencialmente la demanda de herramientas de diagnóstico asistido por computadora para triaje rápido y seguimiento de pacientes \cite{Jacobi2020}. Las manifestaciones radiológicas de COVID-19 (opacidades en vidrio esmerilado, consolidaciones, distribución periférica) requieren evaluación cuantitativa de extensión y distribución que se beneficiaría significativamente de \textit{landmarks} anatómicos localizados automáticamente. El sistema propuesto demuestra robustez ante variabilidad patológica, manteniendo precisión clínicamente útil en casos de COVID-19 y neumonía viral, validando su aplicabilidad en escenarios clínicos reales.

Más allá de COVID-19, las enfermedades respiratorias crónicas (EPOC, fibrosis pulmonar, asma severa) y agudas (neumonía bacteriana, tuberculosis) requieren seguimiento longitudinal mediante radiografías seriadas. Sistemas automatizados de análisis cuantitativo basados en \textit{landmarks} precisos permitirían monitoreo objetivo de progresión de enfermedad y respuesta a tratamiento, mejorando la calidad de atención médica especialmente en entornos con acceso limitado a especialistas.

\subsection{Justificación Metodológica}

La elección de ResNet-18 como arquitectura base se justifica por su balance óptimo entre capacidad de representación (11.7M parámetros) y eficiencia computacional, permitiendo entrenamiento efectivo con conjuntos de datos (\textit{datasets}) de tamaño limitado mediante aprendizaje por transferencia (\textit{transfer learning}) desde ImageNet \cite{He2016, Raghu2019}. Estudios sistemáticos han demostrado que \textit{transfer learning} desde ImageNet beneficia tareas médicas especialmente con menos de 10,000 imágenes, siendo las capas iniciales altamente transferibles entre dominios \cite{Tajbakhsh2016}.

La formulación del problema como regresión coordinada directa (\textit{coordinate regression}) en lugar de regresión de mapas de calor (\textit{heatmap regression}) se justifica por: (1) eficiencia computacional (30 salidas vs. 15 mapas de calor de alta resolución), (2) predicciones con valores de coordenadas continuos (ej. 120.37px) sin necesidad de post-procesamiento de mapas de calor, y (3) facilidad de integración con restricciones geométricas en la función de pérdida \cite{Zhang2014}.

En conclusión, esta investigación se justifica por su contribución metodológica (funciones de pérdida geométricas, entrenamiento progresivo), validación rigurosa (análisis multi-dimensional con datasets multi-categoría), eficiencia computacional (despliegue viable en hardware de consumo), y potencial de impacto en salud pública (base para sistemas de diagnóstico asistido accesibles globalmente). Los resultados establecen una metodología reproducible para integración de conocimiento del dominio en sistemas de aprendizaje profundo médico, principio generalizable más allá del problema específico abordado \cite{Litjens2017}.

\include{chapters/1-chapter/5-Objetivos}
%\section{Diagrama de Bloques}
\label{sec:diagrama_bloques}

\begin{figure}[htbp]
    \centering
    \includegraphics[width=1\linewidth]{Figures/diagrama_bloques_metodologia.png}
    \caption{Diagrama de bloques de la visión general de la metodología}
    \label{fig:diagrama_bloques_metodologia_cap1}
\end{figure}

\begin{figure}[htbp]
    \centering
    \includegraphics[width=1\linewidth]{Figures/diagrama_bloques_1.png}
    \caption{Diagrama de bloques de la metodología de segmentación automática}
    \label{fig:diagrama_bloques_1_cap1}
\end{figure}

%\section{Contribuciones Principales}
\label{sec:contribuciones}

Las contribuciones fundamentales de esta tesis se centran en el desarrollo y la evaluación de una nueva metodología para el análisis de radiografías de tórax, con un énfasis particular en la normalización robusta de la forma pulmonar como prerrequisito para una mejor detección de patologías:

\begin{enumerate}
\item \textbf{Desarrollo y validación del enfoque MaShDL-CNN Hybrid para la alineación y normalización de la forma pulmonar:} La principal contribución consiste en un novedoso método híbrido que integra Modelos Estadísticos de Forma (SSM) con Redes Neuronales Convolucionales (CNNs) de manera sinérgica. Este sistema utiliza CNNs, entrenadas para analizar la información contenida en parches 2D de la imagen, con el fin de predecir los coeficientes de deformación de un SSM pulmonar. Esta predicción permite una adaptación precisa y automática de la forma pulmonar a la variabilidad geométrica presente en las radiografías de tórax. El desarrollo incluye el diseño y la optimización de la arquitectura CNN específica para la extracción de características relevantes de los parches y la arquitectura de la Red Neuronal Densa (DNN) para la regresión de los coeficientes de forma.
\item \textbf{Implementación y evaluación de un pipeline completo y reproducible:} Se ha implementado un pipeline completo que abarca desde la estimación de pose inicial de la región pulmonar (utilizando ESL), la extracción de parches 2D basada en la pose, el entrenamiento y la inferencia del modelo MaShDL-CNN, la desdiscretización de los parámetros de forma predichos, hasta la reconstrucción final de la forma pulmonar segmentada y su evaluación cuantitativa rigurosa mediante el coeficiente de Dice. Un componente crucial de este pipeline es la generación de máscaras Ground Truth consistentes con la definición del SSM, asegurando una evaluación justa y precisa del método de segmentación.
\item \textbf{Avance en la estimación de parámetros de forma mediante aprendizaje profundo en imágenes médicas:} Se demuestra empíricamente que el uso de CNNs operando sobre parches 2D mejora significativamente la precisión en la estimación de los parámetros de forma del SSM en comparación con enfoques previos que se basaban en características de perfiles de intensidad 1D. Esta mejora aborda las limitaciones de precisión previamente identificadas y permite una modelización más fiel de las variaciones sutiles de la forma pulmonar.
\item \textbf{Cuantificación del impacto de la normalización de forma avanzada en la detección de patologías pulmonares:} Se investigará y demostrará el beneficio de la alineación y normalización de forma obtenida con el método MaShDL-CNN Hybrid en la tarea subsecuente de clasificación de neumonía y COVID-19. El rendimiento se comparará sistemáticamente con sistemas que carecen de esta etapa de normalización explícita o que utilizan métodos de alineación más simples o menos precisos, estableciendo la importancia de una correcta estandarización geométrica.
% \item \textbf{Normalización de contraste (si se mantiene como contribución distintiva):** Una técnica de expansión estadística del histograma para la normalización del contraste de las imágenes de tórax, mejorando la consistencia visual y facilitando el análisis posterior (si esta técnica es una parte significativa y novedosa del preprocesamiento, mantenerla; si no, podría ser solo un paso de preprocesamiento mencionado).
\end{enumerate}

Estas contribuciones, en conjunto, buscan ofrecer un sistema más preciso, robusto y clínicamente relevante para la interpretación de radiografías de tórax, con el potencial de mejorar significativamente el diagnóstico asistido por computadora de enfermedades pulmonares críticas y de alta prevalencia.
%\section*{Calendarización de actividades}

\begin{table}[H]
\centering
\small
\label{tab:calendarizacion}
\begin{tabular}{|p{4.5cm}|c|c|c|c|}
\hline
\textbf{Actividad} & \textbf{Otoño 2023} & \textbf{Primavera 2024} & \textbf{Otoño 2024} & \textbf{Primavera 2025} \\ \hline
Evaluación de programas relacionados & Oct - Ene &  &  &  \\ \hline
Revisión de trabajos relacionados &  & Ene - Feb &  &  \\ \hline
Adquisición y preprocesamiento de imágenes & & Feb - Mar  &  &  \\ \hline
Investigación de algoritmos deformables & & Mar - May  &  &  \\ \hline
Implementación de la segmentación &  &  & May - Oct &  \\ \hline
Normalización de imágenes &  &  & Oct - Dic  &  \\ \hline
Alineación de imágenes &  &  &  &  Ene - Mar \\ \hline
Extracción y selección de características &  &  &   & Mar - Abr \\ \hline
Desarrollo del clasificador &  &  &  & Abr - May  \\ \hline
Validación del clasificador &  &  &  & Jun \\ \hline
Análisis comparativo con otros métodos &  &  &  & Jul  \\ \hline
\end{tabular}
\end{table}

%\section*{Porcentaje de avance de actividades}

\begin{table}[H]
\centering
\small
\label{tab:porcentajes}
\begin{tabular}{|p{4.5cm}|c|c|c|}
\hline
\textbf{Objetivo} & \textbf{\% Objetivo} & \textbf{\% Avance} & \textbf{\% Total} \\ \hline
Evaluación de programas relacionados &5\%  &100\%  & 5\%  \\ \hline
Revisión de trabajos relacionados &5\%  &100\%  &  5\%  \\ \hline
Adquisición y preprocesamiento de imágenes &10\% &100\%   &  10\%  \\ \hline
Investigación de algoritmos deformables &5\% &100\%   & 5\%   \\ \hline
Implementación de la segmentación &15\%  &100\%  &  15\%  \\ \hline
Normalización de imágenes &20\%  &100\%  & 20\%    \\ \hline
Alineación de imágenes &20\%  &100\%  &  20\%   \\ \hline
Extracción y selección de características &5\%  &100\%  & 5\%    \\ \hline
Desarrollo del clasificador &5\%  & 0\% & 0\%   \\ \hline
Validación del clasificador & 5\% & 0\% & 0\%   \\ \hline
Análisis comparativo con otros métodos &5\%  &0\%  & 0\%  \\ \hline
\textbf{Total} &  &  &  \textbf{85\%} \\ \hline
\end{tabular}
\end{table}

%\section{Metodología y Cronograma de actividades}

\subsection{Adquisición de Imágenes y Preprocesamiento}
El primer paso en este enfoque consiste en adquirir un conjunto de datos de radiografías de tórax que incluya imágenes de pacientes sanos, con neumonía y con COVID-19. Estas imágenes se someten a un preprocesamiento para mejorar su calidad visual y facilitar la extracción de características relevantes. Las técnicas de preprocesamiento utilizadas incluyen la normalización de la intensidad y el ajuste de contraste, lo cual es crucial para asegurar que las imágenes sean comparables entre sí y que las características importantes sean fácilmente identificables \cite{koonsanit2017, prokop2003}.

\subsection{Normalización y Alineación de la Región Pulmonar}
Debido a la variabilidad en tamaño, forma y orientación de los pulmones en las imágenes radiográficas, es esencial normalizar y alinear la región pulmonar. Este proceso consta de dos fases principales:

\begin{enumerate}
\item \textbf{Segmentación de la Región Pulmonar:} Se utilizan algoritmos avanzados de visión por computadora para identificar y extraer la región de interés (ROI) correspondiente a los pulmones en cada imagen. Esto incluye técnicas de segmentación basadas en modelos deformables que se adaptan a la forma específica de los pulmones \cite{Shi2008}. La precisión en esta etapa es crucial, ya que una segmentación incorrecta puede afectar negativamente las etapas posteriores del análisis.

      
\item \textbf{Normalización y Alineación:} Posteriormente, **se aplica** un conjunto de transformaciones geométricas para normalizar la forma, tamaño y orientación de la ROI. Esto asegura que las variaciones no relacionadas con las patologías de interés no afecten el proceso de clasificación. Las transformaciones incluyen ajustes en escala, rotación y deformaciones no rígidas, **alineando** la ROI a una plantilla estándar para facilitar una comparación uniforme entre imágenes \cite{coselmon2004, rueckert1999nonrigid}.

\end{enumerate}

\subsection{Extracción y Selección de Características Discriminantes}
Una vez normalizadas las imágenes, se procede a la extracción de un conjunto de características significativas para la detección de neumonía y COVID-19. Este proceso incluye técnicas de análisis de textura, bordes y formas, con el objetivo de identificar patrones específicos asociados a cada condición \cite{wu2020deep}. La selección de características se realiza mediante métodos de aprendizaje automático supervisado, como el análisis de componentes principales (PCA), para determinar las características más relevantes para la clasificación \cite{jolliffe2016principal}. Esta etapa es vital para reducir la dimensionalidad del problema y mejorar la precisión del clasificador.

\subsection{Diseño e Implementación del Clasificador}
Las características seleccionadas se utilizan como entrada para un clasificador basado en aprendizaje automático. Se explorarán varios modelos, incluyendo redes neuronales, máquinas de soporte vectorial (SVM) y bosques aleatorios, para determinar el más efectivo. El clasificador será entrenado, validado y probado utilizando técnicas de validación cruzada para asegurar su generalizabilidad y rendimiento robusto en conjuntos de datos no vistos \cite{goyal2021}. La elección del modelo y su configuración se optimizarán para maximizar la precisión y minimizar los errores de clasificación.

\subsection{Validación y Comparación del Sistema}
Finalmente, se validará el rendimiento del clasificador desarrollado comparándolo con los estándares actuales y otros métodos existentes utilizando un conjunto de datos independiente. Las métricas de evaluación incluirán precisión, sensibilidad, especificidad y el área bajo la curva ROC (AUC). Además, se llevará a cabo un análisis de los casos de fallo para identificar oportunidades de mejora \cite{erdaw2021}.

En resumen, la solución propuesta integra técnicas de visión por computadora y aprendizaje automático para desarrollar un sistema robusto y eficiente para la detección automática de neumonía y COVID-19 en imágenes de tórax. La combinación de normalización y alineación de la región pulmonar, extracción y selección de características discriminantes, y la implementación de clasificadores avanzados permitirá mejorar significativamente la precisión y fiabilidad del diagnóstico asistido por computadora.
%\section{Estado del Arte}
En los últimos años, la detección automática de enfermedades pulmonares, como la neumonía y el COVID-19, a partir de imágenes de tórax ha constituido un área de investigación activa y de gran importancia clínica. Los avances en visión por computadora y aprendizaje automático han permitido el desarrollo de sistemas cada vez más precisos y eficientes para este propósito. A continuación, se presenta una revisión del estado del arte en esta área, abarcando los enfoques más recientes y relevantes según las publicaciones científicas.

\subsection{Métodos existentes para la detección de neumonía y COVID-19 en radiografías de tórax}
Los enfoques para la detección automática de neumonía y COVID-19 en radiografías de tórax pueden dividirse en dos categorías principales: técnicas de procesamiento de imágenes tradicional y métodos basados en aprendizaje profundo. Entre los enfoques tradicionales, Bharati et al. \cite{bharati2020hybrid} propusieron un método híbrido que combina la transformada de wavelet discreta (DWT) para la extracción de características y una máquina de soporte vectorial (SVM) para la clasificación, logrando una precisión del 96.39% en la detección de neumonía. Narin et al. \cite{narin2020automatic} utilizaron la transformada de coseno discreta (DCT) y compararon diferentes clasificadores, alcanzando una precisión del 98% con ResNet50 en la detección de COVID-19.

Por otro lado, los métodos basados en aprendizaje profundo han demostrado un rendimiento excepcional. Wang et al. \cite{wang2020covid} propusieron COVID-Net, una red neuronal convolucional (CNN) diseñada específicamente para la detección de COVID-19, logrando una precisión del 92.4%. Ozturk et al. \cite{ozturk2020automated} desarrollaron DarkCovidNet, basada en la arquitectura DarkNet, obteniendo una precisión del 98.08% para la clasificación binaria y del 87.02% para la clasificación multiclase. Rajpurkar et al. \cite{rajpurkar2017chexnet} introdujeron CheXNet, una CNN basada en DenseNet121, para la detección de neumonía, demostrando un rendimiento comparable al de radiólogos expertos.

\subsection{Técnicas de segmentación y normalización de la región pulmonar}
La segmentación y normalización de la región pulmonar son pasos cruciales en el análisis de imágenes de tórax. Los métodos de segmentación pulmonar incluyen técnicas basadas en umbralización \cite{saad2014lung}, regiones \cite{dai2017region}, contornos activos \cite{xu2012lung} y aprendizaje profundo \cite{ait2018lung}. Para la normalización de la región pulmonar, los modelos deformables han demostrado ser efectivos. Van Ginneken et al. \cite{van2006segmentation} utilizaron modelos de forma activa (ASM) para segmentar y normalizar los pulmones en radiografías de tórax, mientras que Shi et al. \cite{shi2008hierarchical} propusieron un modelo de apariencia activa (AAM) jerárquico para imágenes de TAC. Recientemente, Zheng et al. \cite{zheng2020shape} desarrollaron un modelo de forma pulmonar basado en una red generativa adversaria (GAN) para normalizar la región pulmonar en radiografías de tórax.

\subsection{Extracción y selección de características discriminantes}
La extracción y selección de características discriminantes es fundamental para representar de manera compacta y relevante la información contenida en las imágenes de tórax. Las técnicas de análisis de textura incluyen matrices de co-ocurrencia de niveles de gris (GLCM) \cite{haralick1973textural}, patrones binarios locales (LBP) \cite{ojala2002multiresolution} y transformadas de wavelet \cite{bharati2020hybrid}. Para el análisis de bordes y formas, se han empleado descriptores de Fourier \cite{karargyris2016automated}, momentos de Hu \cite{xu2014texture} y patrones locales binarios codificados radialmente (RLBP) \cite{xu2014texture}. Además, técnicas de reducción de dimensionalidad y selección de características supervisadas, como el análisis de componentes principales (PCA) \cite{jolliffe2016principal}, el análisis discriminante lineal (LDA) \cite{guyon2003introduction} y la selección basada en información mutua \cite{guyon2003introduction}, han demostrado ser efectivas para obtener un conjunto compacto y discriminante de características.

\subsection{Clasificadores utilizados en la detección automática de enfermedades pulmonares}
Los clasificadores más relevantes y recientes utilizados en la detección automática de enfermedades pulmonares incluyen redes neuronales convolucionales (CNN), máquinas de soporte vectorial (SVM) y ensambles de clasificadores. Wang et al. \cite{wang2020covid} propusieron COVID-Net, una CNN diseñada específicamente para la detección de COVID-19, logrando una precisión del 93.3\%\. Zhang et al. \cite{zhang2020clinically} presentaron DeCoVNet, que incorpora módulos de atención y una función de pérdida ponderada, alcanzando una precisión del 95.7\%\. Liu et al. \cite{erdaw2021} utilizaron una SVM con kernels de base radial (RBF) para clasificar radiografías de tórax en clases de neumonía y no neumonía, logrando una precisión del 96.2\%\. Rajaraman et al. \cite{erdaw2021} propusieron un ensemble de CNN y árboles de decisión para la detección de neumonía, alcanzando una precisión del 97.8\%\.

\subsection{Métricas de evaluación y comparación de rendimiento}
La evaluación adecuada del rendimiento de los sistemas de detección de enfermedades pulmonares es esencial para comparar diferentes enfoques. Las métricas comunes incluyen la precisión \cite{goyal2021}, la sensibilidad \cite{shi2021review}, la especificidad \cite{goyal2021} y el F1-score \cite{goyal2021}. Las curvas ROC y el área bajo la curva (AUC) también son ampliamente utilizadas \cite{shi2021review}, \cite{dong2021role}. Además, la validación cruzada y las pruebas en conjuntos de datos externos son esenciales para evaluar la capacidad de generalización de los modelos \cite{litjens2017survey}, \cite{dong2021role}.

%\section{Organización de la Tesis}
\label{sec:organizacion_tesis}

La presente tesis se estructura en los siguientes capítulos, diseñados para guiar al lector desde los fundamentos hasta los resultados y conclusiones de la investigación:

El Capítulo 1 introduce el problema de la detección de enfermedades pulmonares en radiografías de tórax, la motivación subyacente a la investigación, el planteamiento del problema específico que se aborda, la justificación de la metodología propuesta, los objetivos generales y específicos que se persiguen y la organización general del documento.

El Capítulo 2 presenta el marco teórico y una revisión de los antecedentes relevantes en el campo. Se exploran los métodos existentes para la detección de patologías pulmonares, las técnicas de segmentación y normalización de la región pulmonar con un énfasis particular en los Modelos Estadísticos de Forma (SSM), la estimación de pose, y los principios del aprendizaje de variedades. Además, se discuten diversos enfoques para la extracción de características, incluyendo el rol emergente de las Redes Neuronales Convolucionales (CNNs), los tipos de clasificadores supervisados comúnmente empleados en el diagnóstico asistido por computadora y las métricas de evaluación estándar utilizadas para valorar el rendimiento de estos sistemas.

El Capítulo 3 detalla exhaustivamente la metodología propuesta en esta tesis: el enfoque híbrido MaShDL-CNN. Se describe en profundidad la construcción del Modelo Estadístico de Forma pulmonar, el proceso de estimación de pose inicial mediante Efficient Subspace Learning (ESL), la estrategia para la generación de datos de parches 2D, la arquitectura específica y el proceso de entrenamiento del modelo MaShDL-CNN diseñado para la predicción de los coeficientes de forma del SSM. Finalmente, se explica cómo se realiza la reconstrucción de la forma pulmonar y los métodos empleados para la evaluación de la precisión de la segmentación resultante.

% El Capítulo 4 describe el diseño experimental y la configuración utilizada para desarrollar, optimizar y validar la metodología propuesta. Se especifican los conjuntos de datos de radiografías de tórax utilizados, detallando sus fuentes y características. Se describe el entorno computacional (hardware y software) y se presentan los diferentes experimentos llevados a cabo para ajustar los hiperparámetros y comparar distintas configuraciones del modelo de alineación MaShDL-CNN. Adicionalmente, se expone el protocolo experimental diseñado para la evaluación de la tarea final de clasificación de neumonía y COVID-19, incluyendo los escenarios de comparación (sin alineación y con métodos de alineación previos).

El Capítulo 4 presenta y analiza críticamente los resultados obtenidos a lo largo de la investigación. En una primera sección, se evalúa el rendimiento de la etapa de alineación y normalización, enfocándose en la precisión de la predicción de los coeficientes de forma y en los resultados cuantitativos de la segmentación pulmonar (e.g., coeficiente de Dice). Se comparan diferentes configuraciones del modelo MaShDL-CNN. En una segunda sección, se presentan los resultados de la detección de enfermedades (neumonía y COVID-19), comparando el rendimiento de los clasificadores entrenados con características extraídas de las regiones pulmonares normalizadas por el método propuesto, contra los escenarios de control.

% Finalmente, el Capítulo 6 extrae las conclusiones principales del trabajo, reflexionando sobre el cumplimiento de los objetivos planteados y la validación de la hipótesis inicial. Se resumen las contribuciones más significativas de la tesis al campo de estudio, se discuten las limitaciones inherentes al estudio realizado y se proponen posibles líneas de investigación futura que podrían derivarse de este trabajo.

%Capitulo 2: Marco Teórico y Estado del Arte (Deep Learning)
% Nuevo contenido (Deep Learning, ResNet, Transfer Learning, Loss Functions)
\chapter{Marco Teórico y Estado del Arte}

La detección automática de puntos de referencia anatómicos (\textit{landmarks}) en radiografías de tórax constituye un problema fundamental en el análisis computarizado de imágenes médicas. Como se estableció en el Capítulo 1, la localización precisa de estos puntos es esencial para la cuantificación de estructuras anatómicas, el cálculo de índices diagnósticos como el índice cardiotorácico, y la normalización espacial de radiografías para sistemas de clasificación automática. El presente capítulo tiene como objetivo establecer los fundamentos teóricos que sustentan el desarrollo de métodos basados en aprendizaje profundo para la detección automática de \textit{landmarks} anatómicos, revisando tanto los principios fundamentales como el estado del arte actual en esta área de investigación.

La detección de \textit{landmarks} en imágenes médicas ha experimentado una evolución significativa en las últimas dos décadas. Tradicionalmente, este problema se abordó mediante métodos estadísticos y geométricos, tales como los Modelos Activos de Forma (\textit{Active Shape Models}, ASM) \cite{Cootes1995} y los Modelos Activos de Apariencia (\textit{Active Appearance Models}, AAM) \cite{Cootes2001}, que representan variaciones anatómicas mediante descomposición lineal basada en Análisis de Componentes Principales. Si bien estos métodos clásicos demostraron utilidad en escenarios controlados, presentan limitaciones fundamentales relacionadas con la linealidad de sus representaciones y su dependencia de características diseñadas manualmente (\textit{hand-crafted features}) \cite{Heimann2009}. La irrupción del aprendizaje profundo en visión por computadora, particularmente tras el trabajo seminal de Krizhevsky et al. \cite{Krizhevsky2012}, ha revolucionado el análisis de imágenes médicas al permitir el aprendizaje automático de representaciones jerárquicas directamente desde los datos \cite{Litjens2017, Shen2017}. En el contexto específico de la detección de \textit{landmarks}, las Redes Neuronales Convolucionales (CNNs) han demostrado capacidad superior para capturar patrones complejos y no lineales en estructuras anatómicas, superando consistentemente el desempeño de métodos tradicionales.

El presente capítulo se estructura en ocho secciones que abarcan desde los fundamentos físicos de las radiografías de tórax hasta el estado del arte en métodos basados en aprendizaje profundo. La Sección~2.1 introduce los principios físicos de las radiografías torácicas y define los quince \textit{landmarks} anatómicos relevantes para este trabajo. La Sección~2.2 establece los fundamentos matemáticos de las redes neuronales convolucionales, incluyendo la operación de convolución, funciones de activación, y el algoritmo de retropropagación (\textit{backpropagation}). La Sección~2.3 analiza en detalle las arquitecturas residuales, particularmente la familia ResNet, que han demostrado ser especialmente efectivas para el entrenamiento de redes profundas mediante el uso de conexiones residuales (\textit{skip connections}). La Sección~2.4 examina el paradigma de aprendizaje por transferencia (\textit{transfer learning}), un componente crucial cuando se trabaja con conjuntos de datos médicos de tamaño limitado. La Sección~2.5 presenta una revisión exhaustiva de funciones de pérdida especializadas para la regresión de coordenadas, con énfasis particular en \textit{Wing Loss} y funciones de pérdida basadas en restricciones geométricas. La Sección~2.6 contrasta los enfoques de regresión directa de coordenadas versus regresión de mapas de calor (\textit{heatmap regression}), justificando la elección metodológica adoptada en esta tesis. La Sección~2.7 ofrece un análisis comparativo exhaustivo del estado del arte en detección de \textit{landmarks} anatómicos, identificando las brechas que motivan el presente trabajo. Finalmente, la Sección~2.8 sintetiza los conceptos presentados y establece la conexión con la metodología propuesta que se desarrollará en el Capítulo 3.

\section{Radiografías de Tórax en Diagnóstico Médico}

Las radiografías de tórax constituyen el estudio de imagen médica más frecuentemente utilizado en la práctica clínica. Como se estableció en el Capítulo 1, este método diagnóstico es fundamental para la evaluación de patologías pulmonares, cardiovasculares y mediastinales. La interpretación radiológica se fundamenta en la identificación de estructuras anatómicas de referencia (\textit{landmarks}) y en la evaluación de sus relaciones espaciales. En el contexto del análisis computarizado de imágenes médicas, la detección automática de estos puntos anatómicos representa una tarea fundamental para sistemas de diagnóstico asistido por computadora y para la cuantificación objetiva de hallazgos radiológicos.

\subsection{Principios Físicos de la Radiografía Torácica}

La formación de imágenes radiográficas se basa en la interacción de fotones de rayos X con tejidos biológicos, específicamente mediante los fenómenos de dispersión Compton y absorción fotoeléctrica. Los sistemas de radiografía torácica convencionales operan con voltajes de aceleración de 110-120 kVp para proyecciones posteroanterior (PA), generando radiación electromagnética con energías en el rango de 40 a 150 keV \cite{Bushberg2020}.

La atenuación del haz de rayos X al atravesar tejido biológico se describe mediante la Ley de Beer-Lambert:
\begin{equation}
I(x) = I_0 \exp\left(-\int_0^x \mu(s) \, ds\right)
\label{eq:beer_lambert}
\end{equation}
donde $I(x)$ representa la intensidad del haz transmitido después de atravesar un espesor $x$ de tejido, $I_0$ es la intensidad del haz incidente, y $\mu(s)$ es el coeficiente de atenuación lineal en función de la posición. Para tejidos homogéneos con coeficiente de atenuación constante, la ecuación se simplifica a:
\begin{equation}
I = I_0 e^{-\mu x}
\label{eq:beer_lambert_simple}
\end{equation}

El contraste radiográfico resulta de las diferencias en los coeficientes de atenuación entre los tejidos que componen la anatomía torácica. Los campos pulmonares, compuestos predominantemente por aire alveolar ($\mu \approx 0.0001$ cm$^{-1}$), presentan baja atenuación y aparecen radiolúcidos (oscuros) en las radiografías. Por el contrario, las estructuras mediastinales y la silueta cardíaca, constituidas por tejidos blandos con coeficientes de atenuación superiores ($\mu \approx 0.20$ cm$^{-1}$), presentan mayor radioopacidad (tonos claros). Las estructuras óseas de la caja torácica (costillas, clavículas, columna vertebral) exhiben la mayor atenuación ($\mu \approx 0.50$ cm$^{-1}$ para hueso cortical) \cite{Bushberg2020}. Esta diferenciación inherente de densidades radiográficas entre estructuras anatómicas adyacentes genera los bordes y contornos que definen los \textit{landmarks} anatómicos de interés para este trabajo.

\subsection{Anatomía Torácica y Definición de Landmarks Anatómicos}

La anatomía torácica en proyección posteroanterior comprende tres compartimentos principales: los campos pulmonares bilaterales, el mediastino central, y la caja torácica ósea \cite{Webb2015, Hansell2008}. La interpretación sistemática de radiografías de tórax requiere la identificación de estructuras anatómicas de referencia cuya localización precisa permite la evaluación de normalidad anatómica y la detección de alteraciones patológicas.

Los \textit{landmarks} anatómicos se definen como puntos de referencia específicos que corresponden a estructuras anatómicas con significado clínico establecido y criterios de identificación reproducibles entre observadores expertos. A diferencia de regiones de interés arbitrarias, los \textit{landmarks} representan localizaciones anatómicas con propiedades geométricas consistentes que pueden explotarse mediante restricciones geométricas en algoritmos de detección automática \cite{Li2022}.

El presente trabajo aborda la detección automática de 15 \textit{landmarks} anatómicos distribuidos en las estructuras pulmonares, mediastinales y óseas de la radiografía de tórax. Como se ilustra en la Figura~\ref{fig:landmarks_anotados}, estos puntos de referencia se seleccionaron considerando tres criterios fundamentales: (1) detectabilidad visual consistente en radiografías de calidad diagnóstica, (2) relevancia anatómica para la caracterización de la geometría torácica, y (3) distribución espacial que captura la estructura global del tórax. La Tabla~\ref{tab:landmarks_descripcion} presenta la nomenclatura y localización anatómica de cada punto.

\begin{table}[h]
\centering
\caption{Descripción de los 15 \textit{landmarks} anatómicos en radiografías de tórax}
\label{tab:landmarks_descripcion}
\small
\begin{tabular}{|c|p{5.5cm}|p{7.5cm}|}
\hline
\textbf{Nº} & \textbf{Nombre anatómico} & \textbf{Localización / Descripción} \\
\hline
1 & Escotadura yugular & Punto superior en línea media, entre articulaciones esternoclaviculares \\
\hline
2 & Ángulo cardiofrénico izquierdo & Unión del borde inferior izquierdo de la silueta cardíaca con la cúpula diafragmática \\
\hline
3 & Borde costal lateral superior izquierdo & Contorno lateral alto del hemitórax izquierdo, nivel de 2ª-3ª costilla posterior \\
\hline
4 & Borde costal lateral superior derecho & Homólogo del \textit{landmark} \#3 en el hemitórax derecho \\
\hline
5 & Borde costal lateral medio izquierdo & Contorno medio lateral del pulmón izquierdo, tercio medio del hemitórax \\
\hline
6 & Borde costal lateral medio derecho & Homólogo del \textit{landmark} \#5 en el hemitórax derecho \\
\hline
7 & Borde costal lateral inferior izquierdo & Contorno lateral inferior del pulmón izquierdo, inmediatamente superior al diafragma \\
\hline
8 & Borde costal lateral inferior derecho & Homólogo del \textit{landmark} \#7 en el hemitórax derecho \\
\hline
9 & Carina traqueal & Bifurcación de la tráquea en bronquios principales, mediastino medio \\
\hline
10 & Borde cardíaco derecho medio & Límite lateral derecho de la silueta cardíaca, correspondiente a la aurícula derecha \\
\hline
11 & Borde cardíaco izquierdo inferior & Límite lateral inferior izquierdo de la silueta cardíaca, ventrículo izquierdo \\
\hline
12 & Ápice pulmonar izquierdo subclavicular & Punto más alto del campo pulmonar izquierdo, bajo el extremo medial de la clavícula \\
\hline
13 & Ápice pulmonar derecho subclavicular & Homólogo del \textit{landmark} \#12 en el hemitórax derecho \\
\hline
14 & Ángulo costofrénico izquierdo & Receso pleural posterolateral izquierdo, unión diafragma-pared torácica costal \\
\hline
15 & Ángulo costofrénico derecho & Homólogo del \textit{landmark} \#14 en el hemitórax derecho \\
\hline
\end{tabular}
\end{table}

Esta configuración de \textit{landmarks} presenta características geométricas de particular relevancia para el análisis computarizado: siete pares de puntos con simetría bilateral (\#3-4, \#5-6, \#7-8, \#10-11, \#12-13, \#14-15, y \#2 respecto al eje de simetría), y dos puntos localizados en la línea media que definen el eje vertical de simetría (\#1 y \#9). La simetría bilateral es una propiedad anatómica fundamental del tórax normal que puede explotarse mediante restricciones geométricas en algoritmos de aprendizaje profundo. Esta propiedad geométrica se incorpora explícitamente en la función de pérdida propuesta en este trabajo, como se discutirá en detalle en la Sección~2.5. Adicionalmente, las distancias entre pares de \textit{landmarks} específicos (por ejemplo, entre \#3 y \#4, o entre \#12 y \#13) representan medidas anatómicas con variabilidad limitada que pueden utilizarse como restricciones de preservación de distancias.

\subsection{Aplicaciones Clínicas de la Localización de Landmarks}

La localización precisa de \textit{landmarks} anatómicos en radiografías de tórax tiene múltiples aplicaciones en la práctica clínica y en sistemas de análisis automatizado. Las aplicaciones diagnósticas incluyen la cuantificación de parámetros anatómicos (como índice cardiotorácico, altura pulmonar, y evaluación de simetría bilateral), la detección de asimetrías patológicas mediante comparación de distancias entre pares de \textit{landmarks} homólogos, y la caracterización de deformaciones anatómicas asociadas a patologías específicas \cite{Sardanelli2022}.

En el contexto de sistemas de diagnóstico asistido por computadora, la detección automática de \textit{landmarks} constituye una etapa de preprocesamiento fundamental para dos aplicaciones principales \cite{Oakden-Rayner2020, Sogancioglu2021}: (1) la normalización espacial de radiografías con variabilidad en posicionamiento del paciente, distancia foco-detector, y grado de inspiración, y (2) la segmentación automática de regiones anatómicas mediante la definición de límites anatómicos iniciales. Como se estableció en el Capítulo 1, la detección manual de \textit{landmarks} por radiólogos expertos presenta variabilidad inter-observador significativa y requiere tiempo considerable, motivando el desarrollo de métodos automatizados basados en aprendizaje profundo.

Estudios recientes han demostrado que la incorporación explícita de \textit{landmarks} anatómicos en arquitecturas de aprendizaje profundo mejora significativamente la interpretabilidad de los modelos, permitiendo a los clínicos comprender qué regiones anatómicas contribuyen a las predicciones diagnósticas \cite{Liu2024, Li2022}. Esta interpretabilidad representa un aspecto crítico para la adopción clínica de sistemas de inteligencia artificial en medicina, particularmente en contextos de alta demanda diagnóstica donde la confiabilidad y la transparencia algorítmica son requisitos esenciales \cite{Rubin2018}. Los fundamentos matemáticos y computacionales de las arquitecturas de aprendizaje profundo que permiten la detección automática de estos \textit{landmarks} se desarrollan en las secciones subsecuentes.

\section{Fundamentos de Aprendizaje Profundo para Visión por Computadora}

El aprendizaje profundo (\textit{deep learning}) ha revolucionado el campo de la visión por computadora en la última década, permitiendo el desarrollo de sistemas capaces de aprender representaciones jerárquicas de características directamente desde datos en bruto, sin la necesidad de ingeniería manual de características. En el contexto de la detección de \textit{landmarks} anatómicos, las Redes Neuronales Convolucionales (\textit{CNNs}) representan la arquitectura fundamental que sustenta los métodos del estado del arte. Esta sección establece los fundamentos matemáticos y computacionales necesarios para comprender las arquitecturas residuales (Sección~2.3), las estrategias de aprendizaje por transferencia (Sección~2.4), y las funciones de pérdida especializadas (Sección~2.5) que constituyen los componentes técnicos centrales de este trabajo \cite{Goodfellow2016deep, LeCun2015deep}.

\subsection{Redes Neuronales y Representaciones Jerárquicas}

Las redes neuronales artificiales profundas se componen de múltiples capas de transformaciones no lineales que procesan información de manera jerárquica. En el caso del perceptrón multicapa básico, cada neurona en la capa $l$ computa una combinación lineal de las activaciones de la capa anterior seguida de una función de activación no lineal:
\begin{equation}
a_j^{(l)} = f\left(\sum_{i=1}^{n} w_{ij}^{(l)} a_i^{(l-1)} + b_j^{(l)}\right)
\label{eq:perceptron}
\end{equation}
donde $w_{ij}^{(l)}$ representa el peso de conexión de la neurona $i$ en la capa $l-1$ a la neurona $j$ en la capa $l$, $b_j^{(l)}$ es el término de sesgo, y $f(\cdot)$ es la función de activación no lineal.

El concepto de representaciones jerárquicas es fundamental en \textit{deep learning}: las capas tempranas de la red aprenden a detectar características de bajo nivel (bordes, texturas, gradientes), mientras que las capas profundas componen estas características simples para formar representaciones de alto nivel (formas complejas, objetos completos, relaciones espaciales). Esta jerarquía de abstracciones es particularmente adecuada para el procesamiento de imágenes médicas, donde la detección de estructuras anatómicas complejas requiere la integración de información visual a múltiples escalas \cite{Goodfellow2016deep}.

Sin embargo, las arquitecturas basadas en capas completamente conectadas (\textit{fully connected}) presentan limitaciones severas para el procesamiento de imágenes. Una imagen de radiografía de tórax de dimensiones modestas ($256 \times 256$ píxeles con un canal de intensidad) contiene 65,536 valores de entrada. Una capa completamente conectada con 1,000 neuronas requeriría 65.5 millones de parámetros solo en la primera capa, resultando en un modelo computacionalmente intratable y altamente susceptible al sobreajuste. Esta limitación motivó el desarrollo de arquitecturas convolucionales que explotan la estructura espacial de las imágenes mediante compartición de parámetros y conectividad local \cite{LeCun1998gradient}.

\subsection{Redes Neuronales Convolucionales}

Las Redes Neuronales Convolucionales (\textit{Convolutional Neural Networks}, CNNs) constituyen una clase especializada de redes neuronales diseñadas para procesar datos con topología de rejilla, como imágenes bidimensionales. La operación fundamental de las CNNs es la convolución discreta, definida matemáticamente para imágenes bidimensionales como:
\begin{equation}
Y[i,j] = \sum_{m=0}^{M-1} \sum_{n=0}^{N-1} X[i+m, j+n] \cdot W[m,n] + b
\label{eq:convolution}
\end{equation}
donde $X$ representa la imagen de entrada con dimensiones $H \times W$, $W$ es el kernel o filtro convolucional de dimensiones $M \times N$ (típicamente $3 \times 3$ o $5 \times 5$), $Y$ es el mapa de características de salida (\textit{feature map}), y $b$ es el término de sesgo compartido por todas las ubicaciones espaciales.

La aplicación de la convolución está controlada por dos hiperparámetros adicionales: el paso (del inglés, \textit{stride}) $S$, que determina el desplazamiento del kernel entre aplicaciones consecutivas, y el relleno (del inglés, \textit{padding}) $P$, que especifica el número de píxeles añadidos en los bordes de la imagen de entrada. La dimensión espacial de la salida se calcula mediante:
\begin{equation}
H_{out} = \left\lfloor \frac{H_{in} + 2P - K}{S} \right\rfloor + 1, \quad
W_{out} = \left\lfloor \frac{W_{in} + 2P - K}{S} \right\rfloor + 1
\label{eq:output_dimensions}
\end{equation}
donde $K$ representa el tamaño del kernel (asumiendo kernels cuadrados $K \times K$). El \textit{padding} se utiliza frecuentemente para preservar las dimensiones espaciales ($P = \lfloor K/2 \rfloor$ con $S=1$ mantiene $H_{out} = H_{in}$), mientras que valores de \textit{stride} mayores a 1 reducen las dimensiones espaciales, proporcionando una forma de submuestreo.

El campo receptivo (del inglés, \textit{receptive field}) de una neurona en una capa profunda define la región de la imagen de entrada que influye en su activación. En CNNs, el campo receptivo crece exponencialmente con la profundidad de la red: una neurona en la capa $L$ con kernels de tamaño $K$ tiene un campo receptivo de tamaño aproximado $(K-1)L + 1$. Este crecimiento permite que capas profundas integren información de regiones cada vez más extensas de la imagen, capturando contexto espacial relevante para la tarea de detección.

Las capas convolucionales presentan tres propiedades arquitectónicas fundamentales que las hacen superiores a capas completamente conectadas para visión por computadora \cite{LeCun1998gradient, Krizhevsky2012}:

\begin{enumerate}
    \item \textbf{Compartición de parámetros:} El mismo filtro se aplica en todas las ubicaciones espaciales de la imagen, reduciendo drásticamente el número de parámetros. Un kernel de $3 \times 3$ con 64 filtros requiere solo $3 \times 3 \times 64 = 576$ parámetros (más 64 sesgos), independientemente del tamaño de la imagen de entrada.

    \item \textbf{Invarianza traslacional:} Características detectadas en una región de la imagen pueden ser detectadas en cualquier otra región mediante el mismo conjunto de pesos, proporcionando robustez a traslaciones del objeto de interés.

    \item \textbf{Conectividad local:} Cada neurona procesa solo una región local de la entrada, explotando la correlación espacial inherente en imágenes naturales y médicas.
\end{enumerate}

Una capa convolucional típica aplica múltiples filtros en paralelo, donde cada filtro aprende a detectar una característica específica (bordes horizontales, verticales, gradientes de intensidad, texturas). La salida de una capa convolucional es un tensor tridimensional de dimensiones $H_{out} \times W_{out} \times D_{out}$, donde $D_{out}$ representa el número de filtros aplicados. Las capas tempranas de CNNs profundas aprenden detectores de características de bajo nivel, mientras que capas subsecuentes componen estas características para formar representaciones jerárquicamente más abstractas \cite{Krizhevsky2012}.

\subsection{Operaciones de Submuestreo y Funciones de Activación}

Las operaciones de submuestreo (del inglés, \textit{pooling}) reducen progresivamente las dimensiones espaciales de las representaciones intermedias, disminuyendo la carga computacional y el número de parámetros, mientras expanden el campo receptivo efectivo de las capas subsecuentes. La operación de submuestreo máximo (\textit{max pooling}) es la más ampliamente utilizada en arquitecturas modernas, definida como:
\begin{equation}
Y[i,j] = \max_{m,n \in R_{ij}} X[m,n]
\label{eq:maxpooling}
\end{equation}
donde $R_{ij}$ representa la región de \textit{pooling}, típicamente de tamaño $2 \times 2$ con \textit{stride} de 2, lo que reduce las dimensiones espaciales a la mitad. Alternativamente, el submuestreo promedio (\textit{average pooling}) calcula la media aritmética de los valores en la región:
\begin{equation}
Y[i,j] = \frac{1}{|R_{ij}|} \sum_{m,n \in R_{ij}} X[m,n]
\label{eq:avgpooling}
\end{equation}

El \textit{max pooling} proporciona invarianza a pequeñas traslaciones y deformaciones locales, preservando la activación máxima (más fuerte) dentro de cada región. Esta propiedad es particularmente útil para tareas de detección donde la presencia de una característica es más relevante que su ubicación precisa dentro de una región local.

Las funciones de activación no lineales son componentes esenciales de las redes neuronales, ya que permiten a la red aprender transformaciones no lineales complejas. La función de activación más ampliamente utilizada en CNNs modernas es la Unidad Lineal Rectificada (\textit{Rectified Linear Unit}, ReLU), definida como:
\begin{equation}
f(x) = \max(0, x) = \begin{cases}
x & \text{si } x > 0 \\
0 & \text{si } x \leq 0
\end{cases}
\label{eq:relu}
\end{equation}

La derivada de ReLU es particularmente simple:
\begin{equation}
f'(x) = \begin{cases}
1 & \text{si } x > 0 \\
0 & \text{si } x \leq 0
\end{cases}
\label{eq:relu_derivative}
\end{equation}

ReLU presenta ventajas significativas sobre funciones de activación clásicas como sigmoide y tangente hiperbólica \cite{Krizhevsky2012}: (1) no exhibe saturación en la región positiva, evitando el problema del gradiente desvaneciente que afecta a redes profundas con activaciones sigmoideas; (2) su evaluación es computacionalmente eficiente, requiriendo solo una operación de comparación y selección máxima; (3) induce \textit{sparsity} en las representaciones, ya que aproximadamente 50\% de las activaciones son cero, lo que puede mejorar la eficiencia y la generalización.

Una limitación de ReLU es el fenómeno conocido como ``dying ReLU'', donde neuronas que consistentemente reciben entradas negativas producen activaciones de cero y dejan de aprender, ya que sus gradientes son nulos. Variantes como Leaky ReLU ($f(x) = \max(0.01x, x)$) y Parametric ReLU (PReLU) abordan parcialmente esta limitación al permitir gradientes pequeños para valores negativos.

Otras funciones de activación relevantes incluyen la sigmoide, $\sigma(x) = 1/(1+e^{-x})$, que comprime valores al rango $(0,1)$ pero sufre de gradientes desvanecientes para valores extremos; la tangente hiperbólica, $\tanh(x) = (e^x - e^{-x})/(e^x + e^{-x})$, que mapea al rango $(-1,1)$; y \textit{softmax}, utilizada en capas de salida para tareas de clasificación multi-clase:
\begin{equation}
\text{softmax}(x_i) = \frac{e^{x_i}}{\sum_{j=1}^{K} e^{x_j}}
\label{eq:softmax}
\end{equation}
donde $K$ es el número de clases. La función \textit{softmax} garantiza que las salidas sean no negativas y sumen uno, interpretándose como probabilidades posteriores de clase.

\subsection{Algoritmo de Retropropagación}

El entrenamiento de redes neuronales profundas se realiza mediante el algoritmo de retropropagación (del inglés, \textit{backpropagation}), que calcula eficientemente el gradiente de una función de pérdida $\mathcal{L}$ respecto a todos los parámetros de la red mediante aplicación recursiva de la regla de la cadena del cálculo diferencial \cite{Rumelhart1986, Goodfellow2016deep}. Considérese una red neuronal con $L$ capas, donde cada capa $l$ realiza la transformación:
\begin{align}
z^{(l)} &= W^{(l)} a^{(l-1)} + b^{(l)} \label{eq:forward_linear} \\
a^{(l)} &= f(z^{(l)}) \label{eq:forward_activation}
\end{align}
donde $z^{(l)}$ representa la entrada ponderada (\textit{pre-activation}), $a^{(l)}$ es la activación de la capa $l$, $W^{(l)}$ y $b^{(l)}$ son los parámetros (pesos y sesgos), y $f(\cdot)$ es la función de activación. La propagación hacia adelante (\textit{forward pass}) evalúa estas ecuaciones secuencialmente desde la entrada hasta la salida.

El objetivo del entrenamiento es minimizar una función de pérdida $\mathcal{L}(a^{(L)}, y)$ que cuantifica la discrepancia entre la predicción de la red $a^{(L)}$ y el valor objetivo $y$. Para actualizar los parámetros mediante descenso de gradiente, se requiere calcular $\partial \mathcal{L}/\partial W^{(l)}$ y $\partial \mathcal{L}/\partial b^{(l)}$ para toda capa $l$. La retropropagación logra esto mediante la definición del error de retropropagación $\delta^{(l)}$ en cada capa:
\begin{equation}
\delta^{(l)} = \frac{\partial \mathcal{L}}{\partial z^{(l)}}
\label{eq:delta_definition}
\end{equation}

Para la capa de salida $L$, el error de retropropagación se calcula directamente mediante la regla de la cadena:
\begin{equation}
\delta^{(L)} = \frac{\partial \mathcal{L}}{\partial a^{(L)}} \odot f'(z^{(L)})
\label{eq:delta_output}
\end{equation}
donde $\odot$ denota el producto elemento a elemento (producto de Hadamard). Para capas intermedias, el error se propaga hacia atrás mediante:
\begin{equation}
\delta^{(l)} = \left((W^{(l+1)})^T \delta^{(l+1)}\right) \odot f'(z^{(l)})
\label{eq:delta_backprop}
\end{equation}

Esta ecuación recursiva constituye el núcleo del algoritmo de retropropagación: el error en la capa $l$ se obtiene multiplicando el error de la capa siguiente por la matriz de pesos transpuesta (propagación del error hacia atrás a través de la transformación lineal), seguido de una modulación elemento a elemento por la derivada de la función de activación.

Una vez calculados los errores $\delta^{(l)}$ para todas las capas, los gradientes respecto a los parámetros se obtienen como:
\begin{align}
\frac{\partial \mathcal{L}}{\partial W^{(l)}} &= \delta^{(l)} (a^{(l-1)})^T \label{eq:gradient_weights} \\
\frac{\partial \mathcal{L}}{\partial b^{(l)}} &= \delta^{(l)} \label{eq:gradient_bias}
\end{align}

Los parámetros se actualizan mediante descenso de gradiente:
\begin{equation}
W^{(l)} \leftarrow W^{(l)} - \eta \frac{\partial \mathcal{L}}{\partial W^{(l)}}, \quad
b^{(l)} \leftarrow b^{(l)} - \eta \frac{\partial \mathcal{L}}{\partial b^{(l)}}
\label{eq:parameter_update}
\end{equation}
donde $\eta$ es la tasa de aprendizaje (\textit{learning rate}), un hiperparámetro que controla la magnitud de las actualizaciones de parámetros.

La complejidad computacional de la retropropagación es del mismo orden que la propagación hacia adelante, típicamente $O(W)$ donde $W$ es el número total de pesos en la red. Esta eficiencia computacional, combinada con la disponibilidad de unidades de procesamiento gráfico (GPUs) altamente paralelizables, ha permitido el entrenamiento de redes con cientos de millones de parámetros en conjuntos de datos masivos.

\subsection{Algoritmos de Optimización}

El algoritmo básico de descenso de gradiente estocástico (\textit{Stochastic Gradient Descent}, SGD) actualiza los parámetros $\theta$ de la red utilizando el gradiente calculado sobre una muestra individual o un mini-lote pequeño de datos:
\begin{equation}
\theta_{t+1} = \theta_t - \eta \nabla_\theta \mathcal{L}(\theta_t; x^{(i)}, y^{(i)})
\label{eq:sgd}
\end{equation}
donde $t$ indexa la iteración de actualización, y $(x^{(i)}, y^{(i)})$ representa una muestra de entrenamiento. A diferencia del descenso de gradiente por lotes que utiliza el conjunto de entrenamiento completo, SGD proporciona actualizaciones frecuentes que aceleran la convergencia, aunque con mayor varianza en la dirección de descenso.

Una mejora fundamental sobre SGD es la incorporación de momentum, que acumula un promedio móvil exponencialmente ponderado de gradientes pasados:
\begin{align}
v_t &= \beta v_{t-1} + \eta \nabla_\theta \mathcal{L}(\theta_t) \label{eq:momentum_velocity} \\
\theta_{t+1} &= \theta_t - v_t \label{eq:momentum_update}
\end{align}
donde $v_t$ representa la velocidad acumulada, y $\beta$ es el coeficiente de momentum (típicamente 0.9). El momentum reduce oscilaciones en direcciones de alta curvatura y acelera la convergencia en direcciones consistentes del espacio de parámetros \cite{Goodfellow2016deep}.

El optimizador Adam (\textit{Adaptive Moment Estimation}) representa el estado del arte en algoritmos de optimización para \textit{deep learning}, combinando las ventajas de momentum con tasas de aprendizaje adaptativas por parámetro \cite{Kingma2014adam}. Adam mantiene estimaciones de los momentos de primer y segundo orden de los gradientes:
\begin{align}
m_t &= \beta_1 m_{t-1} + (1-\beta_1) g_t \label{eq:adam_first_moment} \\
v_t &= \beta_2 v_{t-1} + (1-\beta_2) g_t^2 \label{eq:adam_second_moment}
\end{align}
donde $g_t = \nabla_\theta \mathcal{L}(\theta_t)$ es el gradiente en el tiempo $t$, $m_t$ es el primer momento (media), $v_t$ es el segundo momento no centrado (varianza no centrada), y $\beta_1, \beta_2 \in [0,1)$ son tasas de decaimiento exponencial (valores típicos: $\beta_1 = 0.9$, $\beta_2 = 0.999$).

Dado que $m_t$ y $v_t$ se inicializan en cero, presentan sesgo hacia cero en las primeras iteraciones. Adam corrige este sesgo mediante:
\begin{equation}
\hat{m}_t = \frac{m_t}{1-\beta_1^t}, \quad \hat{v}_t = \frac{v_t}{1-\beta_2^t}
\label{eq:adam_bias_correction}
\end{equation}

La actualización de parámetros incorpora una tasa de aprendizaje adaptativa por parámetro:
\begin{equation}
\theta_{t+1} = \theta_t - \frac{\eta}{\sqrt{\hat{v}_t} + \epsilon} \hat{m}_t
\label{eq:adam_update}
\end{equation}
donde $\epsilon = 10^{-8}$ es una constante pequeña para estabilidad numérica. El término $\sqrt{\hat{v}_t}$ normaliza el tamaño de actualización por parámetro basándose en la magnitud histórica de los gradientes, proporcionando actualizaciones más grandes para parámetros con gradientes consistentemente pequeños y actualizaciones más pequeñas para parámetros con gradientes grandes o ruidosos.

Adam ha demostrado convergencia robusta en una amplia variedad de arquitecturas de \textit{deep learning} y es particularmente efectivo en aplicaciones de visión médica, donde los conjuntos de datos suelen ser de tamaño moderado y la optimización cuidadosa es crítica para evitar sobreajuste \cite{Litjens2017}. La combinación de momentum adaptativo y tasas de aprendizaje por parámetro permite que Adam funcione razonablemente bien con hiperparámetros por defecto, reduciendo la necesidad de ajuste extenso de hiperparámetros.

\section{Arquitecturas Residuales Profundas}

Las redes neuronales convolucionales presentadas en la Sección~2.2 pueden componerse en arquitecturas de profundidad variable, siendo la profundidad un factor determinante en su capacidad de aprendizaje: redes más profundas pueden aprender representaciones jerárquicas más complejas mediante la composición de múltiples transformaciones no lineales. Sin embargo, el entrenamiento de redes extremadamente profundas (con más de 20-30 capas) presentaba desafíos significativos antes del desarrollo de arquitecturas residuales. La observación empírica de que redes más profundas exhibían mayor error de entrenamiento que redes menos profundas sugería la existencia de dificultades de optimización fundamentales que no podían atribuirse únicamente al sobreajuste. Las Redes Neuronales Residuales (del inglés, \textit{Residual Neural Networks}, ResNet), introducidas por He et al.~\cite{He2016}, revolucionaron el diseño de arquitecturas profundas mediante la incorporación de conexiones residuales que permiten el entrenamiento efectivo de redes con cientos de capas.

\subsection{El Problema de Degradación en Redes Profundas}

La intuición convencional sugeriría que agregar capas adicionales a una red neuronal no debería degradar su desempeño: en el peor de los casos, las capas adicionales podrían aprender la función identidad, replicando el desempeño de la red menos profunda. Sin embargo, experimentos empíricos demostraron un fenómeno contraintuitivo denominado \textit{degradación}: a medida que la profundidad de la red aumenta más allá de cierto umbral, tanto el error de entrenamiento como el error de prueba comienzan a aumentar \cite{He2016}.

Este problema de degradación no puede explicarse mediante sobreajuste, ya que el error de entrenamiento (no solo el error de generalización) es superior en redes más profundas. He et al. hipotetizaron que la dificultad radica en que los solucionadores de optimización tienen dificultad para aproximar funciones identidad mediante múltiples capas no lineales. Adicionalmente, el problema del gradiente desvaneciente, donde los gradientes se atenúan exponencialmente al propagarse hacia capas tempranas, complica el entrenamiento de redes muy profundas, aunque técnicas como normalización por lotes y funciones de activación ReLU mitigan parcialmente este efecto \cite{Ioffe2015, Glorot2010}.

Para cuantificar la degradación, considérense dos arquitecturas: una red de $n$ capas con error de entrenamiento $\epsilon_n$, y una red de $n+k$ capas con error $\epsilon_{n+k}$. El fenómeno de degradación se manifiesta cuando $\epsilon_{n+k} > \epsilon_n$ a pesar de que teóricamente las $k$ capas adicionales podrían aprender transformaciones identidad. Experimentos en ImageNet demostraron que redes de 56 capas con arquitectura plana (\textit{plain}) exhibían error de entrenamiento 0.5\% superior a redes de 20 capas, evidenciando la naturaleza empírica del problema \cite{He2016}.

\subsection{Conexiones Residuales y Bloques Residuales}

La arquitectura ResNet aborda el problema de degradación mediante la introducción de conexiones residuales (del inglés, \textit{skip connections} o \textit{shortcut connections}), que permiten que el gradiente fluya directamente a través de la red sin atenuación. En lugar de aprender directamente un mapeo deseado $\mathcal{H}(x)$ desde la entrada $x$ hasta la salida, los bloques residuales aprenden el mapeo residual:
\begin{equation}
\mathcal{F}(x) = \mathcal{H}(x) - x
\label{eq:residual_mapping}
\end{equation}

La salida del bloque residual se define entonces como:
\begin{equation}
y = \mathcal{F}(x, \{W_i\}) + x
\label{eq:residual_block}
\end{equation}
donde $\mathcal{F}(x, \{W_i\})$ representa el mapeo residual implementado por las capas con pesos $\{W_i\}$, y la suma $+x$ representa la conexión de atajo (\textit{shortcut connection}). Si el mapeo óptimo es cercano a la identidad, es más fácil para el optimizador ajustar $\mathcal{F}(x)$ hacia cero que forzar múltiples capas no lineales a aproximar la función identidad directamente.

La hipótesis fundamental de ResNet es que \textbf{es más fácil optimizar el mapeo residual $\mathcal{F}(x)$ que el mapeo original $\mathcal{H}(x)$}. En el caso extremo donde el mapeo identidad es óptimo ($\mathcal{H}(x) = x$), es trivial para el optimizador ajustar los pesos de las capas residuales hacia cero, forzando $\mathcal{F}(x) \approx 0$ y obteniendo $y \approx x$.

He et al. propusieron dos arquitecturas de bloques residuales \cite{He2016}:

\textbf{1. Bloque básico} (utilizado en ResNet-18 y ResNet-34):
\begin{equation}
y = \text{ReLU}\left(x + W_2 \sigma(W_1 x + b_1) + b_2\right)
\label{eq:basic_block}
\end{equation}
donde $\sigma$ representa la función de activación ReLU, $W_1$ y $W_2$ son matrices de pesos de capas convolucionales de $3 \times 3$, y $b_1, b_2$ son sesgos. El bloque básico consta de dos capas convolucionales con normalización por lotes y ReLU entre ellas.

\textbf{2. Bloque cuello de botella} (del inglés, \textit{bottleneck block}; utilizado en ResNet-50, ResNet-101, ResNet-152):
\begin{equation}
y = \text{ReLU}\left(x + W_3 \sigma(W_2 \sigma(W_1 x + b_1) + b_2) + b_3\right)
\label{eq:bottleneck_block}
\end{equation}
donde $W_1$ es una convolución de $1 \times 1$ que reduce la dimensionalidad, $W_2$ es una convolución de $3 \times 3$ que procesa características en dimensión reducida, y $W_3$ es una convolución de $1 \times 1$ que restaura la dimensionalidad. Esta arquitectura reduce significativamente el costo computacional en redes muy profundas.

Cuando las dimensiones de la entrada $x$ y la salida $y$ difieren (por cambios en el número de canales o resolución espacial), la conexión de atajo debe implementarse mediante una proyección lineal:
\begin{equation}
y = \mathcal{F}(x, \{W_i\}) + W_s x
\label{eq:residual_projection}
\end{equation}
donde $W_s$ es una matriz de proyección implementada mediante convolución de $1 \times 1$ con \textit{stride} apropiado para igualar las dimensiones.

\subsection{Arquitecturas de la Familia ResNet}

La familia ResNet comprende múltiples arquitecturas que varían en profundidad, desde ResNet-18 (18 capas con pesos) hasta ResNet-152 (152 capas). La Tabla~\ref{tab:resnet_architectures} presenta las configuraciones arquitectónicas de las variantes más utilizadas.

\begin{table}[h]
\centering
\caption{Arquitecturas de la familia ResNet. Los números entre paréntesis indican el número de bloques residuales en cada etapa.}
\label{tab:resnet_architectures}
\small
\begin{tabular}{|l|c|c|c|c|c|}
\hline
\textbf{Capa} & \textbf{Salida} & \textbf{ResNet-18} & \textbf{ResNet-34} & \textbf{ResNet-50} & \textbf{ResNet-101} \\
\hline
Conv1 & $112 \times 112$ & \multicolumn{4}{c|}{$7 \times 7$, 64, stride 2} \\
\hline
Pool & $56 \times 56$ & \multicolumn{4}{c|}{$3 \times 3$ max pool, stride 2} \\
\hline
Conv2\_x & $56 \times 56$ & $\begin{bmatrix} 3 \times 3, 64 \\ 3 \times 3, 64 \end{bmatrix} \times 2$ & $\begin{bmatrix} 3 \times 3, 64 \\ 3 \times 3, 64 \end{bmatrix} \times 3$ & $\begin{bmatrix} 1 \times 1, 64 \\ 3 \times 3, 64 \\ 1 \times 1, 256 \end{bmatrix} \times 3$ & $\begin{bmatrix} 1 \times 1, 64 \\ 3 \times 3, 64 \\ 1 \times 1, 256 \end{bmatrix} \times 3$ \\
\hline
Conv3\_x & $28 \times 28$ & $\begin{bmatrix} 3 \times 3, 128 \\ 3 \times 3, 128 \end{bmatrix} \times 2$ & $\begin{bmatrix} 3 \times 3, 128 \\ 3 \times 3, 128 \end{bmatrix} \times 4$ & $\begin{bmatrix} 1 \times 1, 128 \\ 3 \times 3, 128 \\ 1 \times 1, 512 \end{bmatrix} \times 4$ & $\begin{bmatrix} 1 \times 1, 128 \\ 3 \times 3, 128 \\ 1 \times 1, 512 \end{bmatrix} \times 4$ \\
\hline
Conv4\_x & $14 \times 14$ & $\begin{bmatrix} 3 \times 3, 256 \\ 3 \times 3, 256 \end{bmatrix} \times 2$ & $\begin{bmatrix} 3 \times 3, 256 \\ 3 \times 3, 256 \end{bmatrix} \times 6$ & $\begin{bmatrix} 1 \times 1, 256 \\ 3 \times 3, 256 \\ 1 \times 1, 1024 \end{bmatrix} \times 6$ & $\begin{bmatrix} 1 \times 1, 256 \\ 3 \times 3, 256 \\ 1 \times 1, 1024 \end{bmatrix} \times 23$ \\
\hline
Conv5\_x & $7 \times 7$ & $\begin{bmatrix} 3 \times 3, 512 \\ 3 \times 3, 512 \end{bmatrix} \times 2$ & $\begin{bmatrix} 3 \times 3, 512 \\ 3 \times 3, 512 \end{bmatrix} \times 3$ & $\begin{bmatrix} 1 \times 1, 512 \\ 3 \times 3, 512 \\ 1 \times 1, 2048 \end{bmatrix} \times 3$ & $\begin{bmatrix} 1 \times 1, 512 \\ 3 \times 3, 512 \\ 1 \times 1, 2048 \end{bmatrix} \times 3$ \\
\hline
 & $1 \times 1$ & \multicolumn{4}{c|}{Global Average Pooling, FC 1000, Softmax} \\
\hline
\textbf{Parámetros} & & \textbf{11.7M} & \textbf{21.8M} & \textbf{25.6M} & \textbf{44.5M} \\
\hline
\end{tabular}
\end{table}

La arquitectura base consta de cinco etapas (Conv1, Conv2\_x, Conv3\_x, Conv4\_x, Conv5\_x), donde cada etapa opera en una resolución espacial específica. Las resoluciones espaciales se reducen progresivamente mediante convoluciones con \textit{stride} 2 al inicio de las etapas Conv3\_x, Conv4\_x y Conv5\_x. ResNet-18 y ResNet-34 utilizan bloques básicos, mientras que ResNet-50, ResNet-101 y ResNet-152 emplean bloques cuello de botella para controlar la complejidad computacional. La capa final aplica \textit{global average pooling} sobre los mapas de características espaciales, reduciendo cada canal a un valor escalar, seguido de una capa completamente conectada para clasificación.

\subsection{Normalización por Lotes}

La normalización por lotes (del inglés, \textit{batch normalization}, BN) es un componente esencial de las arquitecturas ResNet, aplicado después de cada capa convolucional y antes de la función de activación \cite{Ioffe2015}. BN normaliza las activaciones de cada capa utilizando estadísticas del mini-lote actual, reduciendo la dependencia en la inicialización de pesos y permitiendo tasas de aprendizaje más altas.

Para un mini-lote $\mathcal{B} = \{x_1, x_2, \ldots, x_m\}$ de tamaño $m$, la normalización por lotes calcula la media y varianza del mini-lote:
\begin{align}
\mu_{\mathcal{B}} &= \frac{1}{m} \sum_{i=1}^{m} x_i \label{eq:bn_mean} \\
\sigma_{\mathcal{B}}^2 &= \frac{1}{m} \sum_{i=1}^{m} (x_i - \mu_{\mathcal{B}})^2 \label{eq:bn_variance}
\end{align}

Las activaciones se normalizan:
\begin{equation}
\hat{x}_i = \frac{x_i - \mu_{\mathcal{B}}}{\sqrt{\sigma_{\mathcal{B}}^2 + \epsilon}}
\label{eq:bn_normalize}
\end{equation}
donde $\epsilon$ (típicamente $10^{-5}$) es una constante pequeña para estabilidad numérica. Finalmente, se aplica una transformación afín aprendible:
\begin{equation}
y_i = \gamma \hat{x}_i + \beta
\label{eq:bn_scale_shift}
\end{equation}
donde $\gamma$ y $\beta$ son parámetros aprendibles que permiten a la red recuperar la capacidad expresiva completa si la normalización resulta subóptima.

Durante la inferencia, BN utiliza estadísticas globales (media y varianza estimadas sobre el conjunto de entrenamiento completo mediante promedio móvil) en lugar de estadísticas del mini-lote, garantizando predicciones deterministas.

BN proporciona múltiples beneficios \cite{Ioffe2015}: (1) reduce el desplazamiento de covarianza interna (\textit{internal covariate shift}), donde las distribuciones de activaciones cambian durante el entrenamiento; (2) permite tasas de aprendizaje significativamente más altas sin divergencia; (3) actúa como regularizador, reduciendo la necesidad de \textit{dropout}; (4) permite la inicialización de pesos menos cuidadosa. Estos factores son particularmente relevantes para redes residuales profundas.

\subsection{Ventajas de Arquitecturas Residuales para Imágenes Médicas}

Las arquitecturas ResNet han demostrado ser particularmente efectivas para aplicaciones en imágenes médicas por múltiples razones \cite{Hosny2018, Esteva2019}:

\textbf{1. Gradientes estables:} Las conexiones residuales proporcionan caminos de gradiente directos desde las capas profundas hasta las capas tempranas, mitigando el problema del gradiente desvaneciente. Durante la retropropagación, el gradiente respecto a la entrada de un bloque residual es:
\begin{equation}
\frac{\partial \mathcal{L}}{\partial x} = \frac{\partial \mathcal{L}}{\partial y} \left(1 + \frac{\partial \mathcal{F}}{\partial x}\right)
\label{eq:residual_gradient}
\end{equation}
El término constante $1$ garantiza que el gradiente no se atenúe completamente, incluso si $\partial \mathcal{F}/\partial x$ es pequeño.

\textbf{2. Eficiencia de parámetros:} ResNet-18, con 11.7 millones de parámetros, proporciona un balance óptimo entre capacidad expresiva y eficiencia computacional. Esta propiedad es crítica en aplicaciones médicas donde los conjuntos de datos son típicamente más pequeños que ImageNet (1.2 millones de imágenes), y arquitecturas muy profundas pueden sobreajustar.

\textbf{3. Aprendizaje jerárquico robusto:} Las conexiones residuales permiten que capas tempranas aprendan características de bajo nivel (bordes, texturas) mientras capas profundas aprenden representaciones anatómicas complejas, sin degradación de desempeño asociada a la profundidad extrema.

\textbf{4. Transferibilidad:} Modelos ResNet pre-entrenados en ImageNet han demostrado transferibilidad excepcional a dominios médicos mediante \textit{fine-tuning}, como se discutirá en la Sección~2.4. Las representaciones aprendidas en ImageNet capturan características genéricas de imágenes naturales que son parcialmente relevantes para imágenes médicas.

En el contexto específico de detección de \textit{landmarks} anatómicos en radiografías de tórax, las arquitecturas residuales proporcionan la profundidad necesaria para capturar la complejidad de estructuras anatómicas distribuidas espacialmente, mientras mantienen gradientes estables que facilitan el aprendizaje de regresión de coordenadas precisa.

\section{Aprendizaje por Transferencia en Imágenes Médicas}

Las arquitecturas residuales presentadas en la Sección~2.3, particularmente ResNet con sus múltiples variantes de profundidad, han demostrado capacidad excepcional para aprender representaciones jerárquicas complejas en tareas de visión por computadora. Sin embargo, el entrenamiento de estas redes profundas desde inicialización aleatoria requiere conjuntos de datos masivos para lograr convergencia robusta y generalización adecuada. En el dominio médico, la adquisición de grandes conjuntos de datos etiquetados enfrenta barreras significativas: costos elevados de anotación por expertos radiólogos, consideraciones de privacidad de pacientes, y la naturaleza inherentemente limitada de casos patológicos específicos. Los conjuntos de datos médicos típicamente contienen cientos a miles de imágenes, en contraste con los millones de ejemplos disponibles en dominios de visión por computadora general como ImageNet. El aprendizaje por transferencia (del inglés, \textit{transfer learning}) constituye un paradigma fundamental que permite aprovechar representaciones aprendidas en conjuntos de datos masivos de un dominio fuente para mejorar significativamente el desempeño en tareas del dominio objetivo con datos limitados \cite{Yosinski2014, Moor2023, Zhou2022}.

\subsection{Pre-entrenamiento en ImageNet y Representaciones Transferibles}

ImageNet representa el conjunto de datos de referencia para pre-entrenamiento de modelos de visión por computadora, comprendiendo aproximadamente 1.2 millones de imágenes de entrenamiento distribuidas en 1,000 categorías de objetos cotidianos \cite{Krizhevsky2012}. Los modelos entrenados en ImageNet, incluyendo las arquitecturas ResNet descritas en la Sección~2.3, aprenden representaciones jerárquicas de características visuales mediante optimización de la función de pérdida de clasificación multi-clase. Formalmente, el pre-entrenamiento en el dominio fuente se define como:
\begin{equation}
\theta^* = \arg\min_{\theta} \sum_{i=1}^{N_s} \mathcal{L}_{\text{source}}(f_\theta(x_i^s), y_i^s)
\label{eq:source_pretraining}
\end{equation}
donde $f_\theta$ representa la red neuronal con parámetros $\theta$, $\mathcal{D}_{\text{source}} = \{(x_i^s, y_i^s)\}_{i=1}^{N_s}$ es el conjunto de datos fuente (ImageNet con $N_s \approx 1.2 \times 10^6$), y $\mathcal{L}_{\text{source}}$ es típicamente la entropía cruzada categórica para clasificación.

La hipótesis central del aprendizaje por transferencia es que las características de bajo y medio nivel aprendidas en ImageNet poseen utilidad general para tareas de visión por computadora, incluso en dominios substancialmente diferentes como imágenes médicas. Las capas convolucionales tempranas de redes pre-entrenadas detectan bordes, texturas, y gradientes de intensidad genéricos que son relevantes para cualquier tarea de procesamiento de imágenes. Las capas intermedias capturan estructuras geométricas de complejidad creciente (formas, patrones, composiciones espaciales), mientras que las capas profundas aprenden representaciones más específicas del dominio fuente \cite{Yosinski2014, Azizi2023}.

Yosinski et al.~\cite{Yosinski2014} demostraron empíricamente que la transferibilidad de características decae con la profundidad de la red cuando los dominios fuente y objetivo son muy diferentes: las primeras capas convolucionales aprenden detectores casi universales, mientras que capas superiores requieren adaptación substancial al dominio objetivo. Este hallazgo motivó estrategias de \textit{fine-tuning} selectivo que se discuten en la siguiente subsección.

Estudios recientes han expandido significativamente la comprensión de \textit{transfer learning} en medicina. Azizi et al.~\cite{Azizi2023} demostraron en \textit{Nature Biomedical Engineering} que modelos pre-entrenados mediante aprendizaje auto-supervisado en conjuntos de datos médicos multi-institucionales exhiben transferibilidad superior a modelos pre-entrenados exclusivamente en ImageNet, sugiriendo que el pre-entrenamiento en dominios más cercanos al objetivo (radiografías médicas en general) proporciona ventajas adicionales. Moor et al.~\cite{Moor2023}, en un artículo perspectivo en \textit{Nature}, argumentan que los modelos fundacionales (del inglés, \textit{foundation models}) pre-entrenados en datos médicos masivos representan el futuro del \textit{transfer learning} médico, reduciendo la dependencia de ImageNet. Sin embargo, para tareas específicas como detección de \textit{landmarks} anatómicos en radiografías de tórax, el pre-entrenamiento en ImageNet continúa siendo el estándar actual debido a la disponibilidad limitada de modelos fundacionales médicos especializados \cite{Zhou2022}.

\subsection{Estrategias de Fine-Tuning y Adaptación al Dominio Médico}

El \textit{transfer learning} puede implementarse mediante dos estrategias principales, que difieren en qué parámetros de la red se actualizan durante el entrenamiento en el dominio objetivo.

\textbf{Extracción de características} (del inglés, \textit{feature extraction}): Los pesos de las capas convolucionales pre-entrenadas se congelan completamente, y solo se entrenan las capas completamente conectadas finales específicas de la tarea objetivo. Matemáticamente, el modelo se descompone como $f_\theta = g_{\phi}(h_{\psi}(x))$, donde $h_{\psi}$ representa el extractor de características convolucionales con pesos congelados $\psi = \psi^*_{\text{ImageNet}}$, y $g_{\phi}$ representa las capas de tarea específica (típicamente una o dos capas completamente conectadas) con parámetros entrenables $\phi$:
\begin{equation}
\phi^* = \arg\min_{\phi} \sum_{j=1}^{N_t} \mathcal{L}_{\text{target}}(g_{\phi}(h_{\psi^*}(x_j^t)), y_j^t)
\label{eq:feature_extraction}
\end{equation}
donde $\mathcal{D}_{\text{target}} = \{(x_j^t, y_j^t)\}_{j=1}^{N_t}$ es el conjunto de datos objetivo (típicamente $N_t \ll N_s$), y $\mathcal{L}_{\text{target}}$ es la función de pérdida específica de la tarea (por ejemplo, error cuadrático medio para regresión de coordenadas, como se discutirá en la Sección~2.5). Esta estrategia es computacionalmente eficiente y apropiada cuando el conjunto de datos objetivo es muy pequeño ($N_t < 1{,}000$ típicamente), mitigando el riesgo de sobreajuste al limitar drásticamente el número de parámetros entrenables.

\textbf{Ajuste fino} (del inglés, \textit{fine-tuning}): Todos los parámetros de la red, o un subconjunto de capas superiores, se ajustan en el dominio objetivo utilizando la inicialización pre-entrenada:
\begin{equation}
\theta_{\text{fine-tuned}} = \arg\min_{\theta} \sum_{j=1}^{N_t} \mathcal{L}_{\text{target}}(f_\theta(x_j^t), y_j^t), \quad \text{con } \theta(t=0) = \theta^*_{\text{ImageNet}}
\label{eq:finetuning}
\end{equation}

El \textit{fine-tuning} permite que la red adapte sus representaciones internas al dominio objetivo, pero requiere conjuntos de datos de tamaño moderado ($N_t > 1{,}000$ típicamente) y selección cuidadosa de hiperparámetros de optimización para evitar sobreajuste o colapso catastrófico de las características pre-entrenadas útiles.

Una estrategia avanzada es la aplicación de tasas de aprendizaje diferenciales (del inglés, \textit{discriminative learning rates}): capas tempranas, que capturan características de bajo nivel genéricas, se actualizan con tasas de aprendizaje pequeñas o se congelan completamente, mientras que capas profundas y capas específicas de la tarea se entrenan con tasas de aprendizaje más altas:
\begin{align}
\theta_{\text{early}}^{(t+1)} &= \theta_{\text{early}}^{(t)} - \eta_{\text{low}} \nabla_{\theta_{\text{early}}} \mathcal{L}_{\text{target}} \label{eq:lr_early} \\
\theta_{\text{deep}}^{(t+1)} &= \theta_{\text{deep}}^{(t)} - \eta_{\text{high}} \nabla_{\theta_{\text{deep}}} \mathcal{L}_{\text{target}} \label{eq:lr_deep}
\end{align}
donde $\eta_{\text{high}} / \eta_{\text{low}} \approx 10$ es una configuración típica. Esta estrategia preserva las representaciones de bajo nivel útiles mientras permite adaptación substancial en capas superiores que requieren especialización al dominio médico.

El \textit{fine-tuning} progresivo (del inglés, \textit{progressive unfreezing}) constituye una variante donde inicialmente solo las capas finales son entrenables, y gradualmente se descongelan capas anteriores a medida que avanza el entrenamiento. Nguyen et al.~\cite{Nguyen2024} presentan un análisis exhaustivo de estrategias de \textit{transfer learning} multi-etapa en imágenes médicas, demostrando que el descongelamiento progresivo proporciona convergencia más estable en conjuntos de datos médicos pequeños comparado con \textit{fine-tuning} simultáneo de todas las capas.

\subsection{Brecha de Dominio y Adaptación para Radiografías de Tórax}

Existe una brecha de dominio (del inglés, \textit{domain gap}) substancial entre imágenes naturales de ImageNet e imágenes médicas de radiografías de tórax. ImageNet contiene fotografías RGB de objetos cotidianos en escenarios naturales con iluminación variada, mientras que las radiografías de tórax son imágenes de canal único (escala de grises) que representan proyecciones bidimensionales de atenuación de rayos X de estructuras anatómicas tridimensionales, como se describió en la Sección~2.1. Las distribuciones de intensidad, texturas, y estructuras geométricas difieren fundamentalmente entre ambos dominios.

A pesar de esta disparidad, estudios empíricos han demostrado consistentemente que el \textit{transfer learning} desde ImageNet proporciona mejoras substanciales sobre el entrenamiento desde inicialización aleatoria en tareas de análisis de radiografías de tórax. Tajbakhsh et al.~\cite{Tajbakhsh2016} evaluaron \textit{transfer learning} en cuatro tareas de análisis de imágenes médicas, incluyendo detección de nódulos pulmonares en radiografías de tórax, demostrando mejoras de 5-10\% en AUC al utilizar pre-entrenamiento de ImageNet versus inicialización aleatoria, particularmente en regímenes de datos limitados ($N_t < 5{,}000$).

Investigaciones recientes han explorado técnicas de adaptación de dominio específicamente diseñadas para abordar la brecha entre ImageNet y radiografías médicas. Sanchez et al.~\cite{Sanchez2022} propusieron CX-DaGAN en \textit{IEEE Transactions on Medical Imaging}, una red generativa adversarial para adaptación de dominio en diagnóstico de neumonía con conjuntos de datos de radiografías de tórax extremadamente pequeños. Guan y Liu~\cite{Guan2022} presentaron un análisis comprehensivo de técnicas de adaptación de dominio para análisis de imágenes médicas en \textit{IEEE Transactions on Biomedical Engineering}, categorizando enfoques en: (1) adaptación basada en discrepancia de características, (2) adaptación adversarial, y (3) adaptación mediante reconstrucción. Estos métodos avanzados buscan alinear las distribuciones de características entre dominios fuente y objetivo, reduciendo la brecha de dominio y mejorando la transferibilidad.

La transferencia desde ImageNet a radiografías de tórax requiere adaptaciones arquitectónicas específicas. Las arquitecturas ResNet estándar esperan imágenes RGB (3 canales de entrada), mientras que las radiografías de tórax son de canal único. La práctica común consiste en replicar la imagen de escala de grises a tres canales ($x_{\text{RGB}} = [x, x, x]$), preservando los pesos pre-entrenados de la primera capa convolucional sin modificación. Alternativamente, los pesos del filtro de entrada pueden promediarse a través de los tres canales RGB y utilizarse para procesar el canal único directamente. La capa completamente conectada final pre-entrenada, que tiene 1,000 salidas correspondientes a las clases de ImageNet, debe reemplazarse con una capa específica de la tarea: para detección de \textit{landmarks}, esta capa predice $2K$ valores continuos representando coordenadas $(x, y)$ de $K$ \textit{landmarks}, sin función \textit{softmax}. La función de pérdida apropiada para esta tarea de regresión de coordenadas se discutirá en detalle en la Sección~2.5.

El aprendizaje por transferencia representa un componente fundamental para el desarrollo de sistemas de \textit{deep learning} en imágenes médicas con conjuntos de datos limitados. La combinación de pre-entrenamiento en ImageNet, estrategias de \textit{fine-tuning} con tasas de aprendizaje diferenciales, y adaptaciones arquitectónicas apropiadas permite la construcción de modelos robustos que aprovechan conocimiento visual genérico mientras se especializan en características anatómicas específicas del dominio médico \cite{Litjens2017, Esteva2019, Zhou2022}.

\section{Funciones de Pérdida para Regresión de Coordenadas}

La estrategia de aprendizaje por transferencia establecida en la Sección~2.4 proporciona una inicialización favorable de los pesos de la red mediante pre-entrenamiento en ImageNet, pero la función de pérdida determina fundamentalmente qué aprende la red durante el \textit{fine-tuning} en el dominio objetivo. Para la tarea de detección de \textit{landmarks} anatómicos, la red debe aprender un mapeo $f_\theta: \mathbb{R}^{H \times W} \to \mathbb{R}^{2K}$ desde una imagen de entrada de dimensiones $H \times W$ a un vector de $2K$ valores continuos representando las coordenadas $(x_k, y_k)$ de $K$ \textit{landmarks}. Esta tarea de regresión de coordenadas presenta desafíos específicos: requiere precisión a nivel de píxel individual, debe ser robusta ante variabilidad anatómica y calidad de imagen heterogénea, y puede beneficiarse de la incorporación explícita de conocimiento anatómico a priori mediante restricciones geométricas. Esta sección analiza funciones de pérdida especializadas para regresión de coordenadas, comenzando con el error cuadrático medio como línea base, seguido por \textit{Wing Loss} que amplifica gradientes para errores pequeños, y concluyendo con funciones de pérdida basadas en restricciones geométricas que incorporan conocimiento anatómico de simetría bilateral y preservación de distancias \cite{Feng2018, Noothout2020, Cheng2023}.

\subsection{Error Cuadrático Medio}

El Error Cuadrático Medio (del inglés, \textit{Mean Squared Error}, MSE) constituye la función de pérdida estándar para tareas de regresión, incluyendo regresión de coordenadas de \textit{landmarks}. Para un conjunto de $K$ \textit{landmarks}, donde cada \textit{landmark} $k$ tiene coordenadas ground truth $(x_k, y_k)$ y coordenadas predichas $(\hat{x}_k, \hat{y}_k)$, MSE se define como:
\begin{equation}
\mathcal{L}_{\text{MSE}} = \frac{1}{K} \sum_{k=1}^{K} \left[\left(x_k - \hat{x}_k\right)^2 + \left(y_k - \hat{y}_k\right)^2\right]
\label{eq:mse_explicit}
\end{equation}

Esta formulación puede expresarse de manera más compacta utilizando notación vectorial. Definiendo $p_k = (x_k, y_k)^T \in \mathbb{R}^2$ como el vector de posición del \textit{landmark} $k$ y $\hat{p}_k = (\hat{x}_k, \hat{y}_k)^T$ como su predicción correspondiente, la función de pérdida MSE se escribe:
\begin{equation}
\mathcal{L}_{\text{MSE}} = \frac{1}{K} \sum_{k=1}^{K} \|p_k - \hat{p}_k\|_2^2 = \frac{1}{K} \sum_{k=1}^{K} (p_k - \hat{p}_k)^T(p_k - \hat{p}_k)
\label{eq:mse_vector}
\end{equation}
donde $\|\cdot\|_2$ denota la norma Euclidiana. El gradiente de MSE respecto a la predicción del \textit{landmark} $k$ es:
\begin{equation}
\frac{\partial \mathcal{L}_{\text{MSE}}}{\partial \hat{p}_k} = \frac{2}{K}(\hat{p}_k - p_k)
\label{eq:mse_gradient}
\end{equation}

MSE posee propiedades matemáticas deseables: es una función convexa, diferenciable en todos los puntos, y su mínimo global coincide con la media de los datos objetivo. Durante la retropropagación, el gradiente proporcionado por MSE crece linealmente con la magnitud del error de predicción ($\|\nabla \mathcal{L}_{\text{MSE}}\| \propto \|p_k - \hat{p}_k\|$). Sin embargo, esta característica resulta problemática para la detección precisa de \textit{landmarks} anatómicos por múltiples razones \cite{Feng2018, Liu2021}.

Primero, MSE exhibe \textbf{sensibilidad desbalanceada a errores de diferente magnitud}. \textit{Landmarks} con predicciones muy incorrectas (errores de decenas de píxeles) generan gradientes dominantes que pueden enmascarar la señal de aprendizaje de \textit{landmarks} con errores pequeños (1-2 píxeles). Segundo, la penalización cuadrática amplifica el impacto de valores atípicos (\textit{outliers}): un solo \textit{landmark} mal predicho contribuye con $error^2$ a la pérdida total, potencialmente desestabilizando el entrenamiento en presencia de oclusiones parciales o artefactos de imagen. Tercero, para errores pequeños ($\|p_k - \hat{p}_k\| < 1$ píxel), el gradiente de MSE se vuelve proporcionalmente pequeño ($\|\nabla \mathcal{L}_{\text{MSE}}\| \approx 2|error|/K \ll 1$), debilitando la señal de aprendizaje precisamente en el régimen donde se requiere refinamiento fino de coordenadas. Finalmente, MSE trata cada \textit{landmark} independientemente, ignorando restricciones geométricas inherentes a la anatomía torácica como simetría bilateral y distancias anatómicas características.

Estas limitaciones motivaron el desarrollo de funciones de pérdida especializadas que amplifican gradientes en el régimen de errores pequeños mientras mantienen robustez ante errores grandes, y que incorporan conocimiento anatómico a priori mediante términos de regularización geométrica.

\subsection{Wing Loss: Amplificación de Gradientes para Errores Pequeños}

Feng et al.~\cite{Feng2018} propusieron \textit{Wing Loss} como una función de pérdida diseñada específicamente para localización robusta de \textit{landmarks} faciales, abordando las limitaciones de MSE mediante amplificación selectiva de gradientes en la región de errores pequeños. La idea central es modificar el comportamiento de la función de pérdida para proporcionar gradientes grandes cuando el error de predicción es pequeño (facilitando refinamiento preciso de coordenadas), mientras se mantienen gradientes moderados para errores grandes (proporcionando robustez).

\textit{Wing Loss} se define mediante una función no lineal por partes:
\begin{equation}
\mathcal{L}_{\text{wing}}(x) = \begin{cases}
w \ln\left(1 + \frac{|x|}{\epsilon}\right) & \text{si } |x| < w \\
|x| - C & \text{si } |x| \geq w
\end{cases}
\label{eq:wing_loss_definition}
\end{equation}
donde $x$ representa el error de coordenada, $w$ es el ancho de la región no lineal (típicamente $w \in [5, 10]$ píxeles para imágenes médicas), $\epsilon$ es un parámetro de curvatura que controla la suavidad de la transición (típicamente $\epsilon = 2.0$), y $C = w - w\ln(1 + w/\epsilon)$ es una constante que garantiza continuidad de la función en $|x| = w$. La región $|x| < w$ exhibe comportamiento logarítmico que amplifica gradientes para errores pequeños, mientras que la región $|x| \geq w$ presenta comportamiento lineal similar a la pérdida L1 absoluta, proporcionando robustez ante \textit{outliers}.

Para la detección de $K$ \textit{landmarks}, \textit{Wing Loss} se aplica al error radial de cada \textit{landmark}:
\begin{equation}
\mathcal{L}_{\text{Wing}} = \frac{1}{K} \sum_{k=1}^{K} \mathcal{L}_{\text{wing}}\left(\|p_k - \hat{p}_k\|_2\right)
\label{eq:wing_loss_landmarks}
\end{equation}

El comportamiento de amplificación de gradientes de \textit{Wing Loss} se comprende mediante el análisis de su derivada. Para $|x| < w$, la derivada es:
\begin{equation}
\frac{\partial \mathcal{L}_{\text{wing}}(x)}{\partial x} = \frac{w}{\epsilon + |x|} \cdot \text{sign}(x)
\label{eq:wing_gradient_small}
\end{equation}
donde $\text{sign}(x) = \pm 1$ indica la dirección del error. Para $|x| \geq w$, la derivada es:
\begin{equation}
\frac{\partial \mathcal{L}_{\text{wing}}(x)}{\partial x} = \text{sign}(x)
\label{eq:wing_gradient_large}
\end{equation}

La amplificación de gradientes se manifiesta en el límite de errores pequeños:
\begin{equation}
\lim_{x \to 0} \frac{\partial \mathcal{L}_{\text{wing}}(x)}{\partial x} = \frac{w}{\epsilon} \cdot \text{sign}(x)
\label{eq:wing_gradient_limit}
\end{equation}

Para la configuración típica $w = 5$ y $\epsilon = 2$, el gradiente en $x \to 0$ es $w/\epsilon = 2.5$, significativamente mayor que el gradiente de MSE en el mismo punto ($\partial \mathcal{L}_{\text{MSE}}/\partial x = 2x/K \approx 0$ cuando $x \to 0$). En el punto de transición $|x| = w$, el gradiente de \textit{Wing Loss} es exactamente $\pm 1$, garantizando continuidad de la derivada. Para errores grandes $|x| \gg w$, el gradiente satura en $\pm 1$, similar a la pérdida L1, proporcionando robustez ante predicciones extremadamente incorrectas que podrían desestabilizar el entrenamiento si se penalizaran cuadráticamente.

La comparación formal con MSE ilustra la ventaja de \textit{Wing Loss}. El gradiente de MSE respecto al error $x$ es $\partial \mathcal{L}_{\text{MSE}}/\partial x = 2x/K$, que decrece linealmente hacia cero a medida que el error disminuye. En contraste, \textit{Wing Loss} mantiene un gradiente constante y grande ($w/\epsilon$) en la región de errores pequeños, proporcionando una señal de aprendizaje consistente para el refinamiento fino de coordenadas. Esta propiedad es particularmente relevante para la detección de \textit{landmarks} anatómicos en radiografías de tórax, donde la variabilidad inter-paciente de posiciones de \textit{landmarks} es típicamente del orden de 10-20 píxeles, y se requiere precisión de localización a nivel de píxel individual para aplicaciones clínicas.

Extensiones recientes de \textit{Wing Loss} incluyen \textit{Adaptive Wing Loss} propuesto por Liu et al.~\cite{Liu2021}, que adapta dinámicamente los parámetros $w$ y $\epsilon$ durante el entrenamiento para equilibrar robustez inicial y precisión final. Cheng et al.~\cite{Cheng2023} demostraron en \textit{Medical Image Analysis} que la incorporación de perturbaciones controladas en las entradas combinada con \textit{Wing Loss} mejora significativamente la precisión de localización de \textit{landmarks} en imágenes médicas.

\subsection{Restricciones Geométricas: Symmetry Loss y Distance Preservation Loss}

Las funciones de pérdida basadas en regresión directa de coordenadas (MSE, \textit{Wing Loss}) tratan cada \textit{landmark} independientemente, ignorando relaciones geométricas inherentes a la anatomía humana. En el contexto específico de radiografías de tórax, la anatomía presenta propiedades geométricas consistentes que pueden explotarse como restricciones: la simetría bilateral aproximada del tórax implica que pares de \textit{landmarks} homólogos (izquierdo-derecho) deben ser aproximadamente simétricos respecto a la línea media, y las distancias entre \textit{landmarks} específicos exhiben variabilidad limitada en poblaciones sanas. La incorporación de estas restricciones geométricas como términos de regularización en la función de pérdida mejora la generalización del modelo y garantiza que las predicciones sean anatómicamente plausibles \cite{Urschler2021, Donner2013, Thaler2021}.

\subsubsection{Symmetry Loss}

Como se estableció en la Sección~2.1, los 15 \textit{landmarks} considerados en este trabajo incluyen siete pares de puntos con simetría bilateral respecto a la línea media vertical del tórax. Específicamente, los pares simétricos son: bordes costales laterales superiores (\#3, \#4), bordes costales laterales medios (\#5, \#6), bordes costales laterales inferiores (\#7, \#8), ápices pulmonares subclaviculares (\#12, \#13), y ángulos costofrénicos (\#14, \#15). Adicionalmente, el ángulo cardiofrénico izquierdo (\#2) debe ser aproximadamente simétrico respecto al eje definido por la escotadura yugular (\#1) y la carina traqueal (\#9).

La función de pérdida de simetría (\textit{Symmetry Loss}) penaliza desviaciones de esta simetría bilateral. Formalmente, sea $S = \{(3,4), (5,6), (7,8), (12,13), (14,15)\}$ el conjunto de pares de índices de \textit{landmarks} simétricos, y sea $p_c = (x_c, y_c)^T$ un punto de referencia en la línea media (que puede definirse como el promedio de las coordenadas $x$ de los \textit{landmarks} \#1 y \#9). La pérdida de simetría se define como:
\begin{equation}
\mathcal{L}_{\text{sym}} = \frac{1}{|S|} \sum_{(i,j) \in S} \left\|(p_i - p_c) + (p_j - p_c)\right\|_2^2
\label{eq:symmetry_loss}
\end{equation}

Esta formulación penaliza la suma vectorial $(p_i - p_c) + (p_j - p_c)$, que debería ser aproximadamente $(0, \Delta y)^T$ si los \textit{landmarks} $i$ y $j$ son perfectamente simétricos en la coordenada $x$ respecto a $p_c$, con posible diferencia en $y$ debido a asimetrías anatómicas menores. Una formulación alternativa más restrictiva penaliza exclusivamente desviaciones en la coordenada $x$:
\begin{equation}
\mathcal{L}_{\text{sym}}^{(x)} = \frac{1}{|S|} \sum_{(i,j) \in S} \left[(x_i - x_c) + (x_j - x_c)\right]^2
\label{eq:symmetry_loss_x}
\end{equation}

La pérdida de simetría proporciona regularización particularmente útil en presencia de oclusiones parciales o artefactos que afectan asimétricamente la imagen: si un \textit{landmark} en un hemitórax es difícil de detectar debido a oclusión, la restricción de simetría permite que el modelo infiera su posición aproximada basándose en la detección de su contraparte simétrica. Urschler et al.~\cite{Urschler2021} demostraron empíricamente que la incorporación de restricciones geométricas incluyendo simetría bilateral mejora consistentemente la precisión de detección de \textit{landmarks} en imágenes médicas, particularmente en conjuntos de datos pequeños donde la regularización es crítica.

\subsubsection{Distance Preservation Loss}

Las distancias Euclidianas entre pares específicos de \textit{landmarks} anatómicos exhiben variabilidad inter-paciente limitada en poblaciones normales, proporcionando una restricción geométrica adicional. Por ejemplo, la distancia entre los ápices pulmonares izquierdo y derecho (\#12, \#13) está relacionada con el ancho torácico superior, que varía dentro de un rango relativamente estrecho. La función de pérdida de preservación de distancias (\textit{Distance Preservation Loss}) penaliza predicciones que violan estas restricciones de distancia.

Formalmente, sea $D \subseteq \{1, \ldots, K\} \times \{1, \ldots, K\}$ un conjunto de pares de índices de \textit{landmarks} cuyas distancias deben preservarse, y sea $d_{ij}^{\text{ref}}$ la distancia de referencia entre \textit{landmarks} $i$ y $j$, típicamente estimada como la media de las distancias observadas en el conjunto de entrenamiento. La pérdida de preservación de distancias se define como:
\begin{equation}
\mathcal{L}_{\text{dist}} = \frac{1}{|D|} \sum_{(i,j) \in D} \left(\|p_i - p_j\|_2 - d_{ij}^{\text{ref}}\right)^2
\label{eq:distance_preservation}
\end{equation}

Esta formulación penaliza tanto la compresión excesiva (distancia predicha menor que $d_{ij}^{\text{ref}}$) como la expansión excesiva (distancia predicha mayor que $d_{ij}^{\text{ref}}$) de distancias anatómicas características. La selección del conjunto $D$ y las distancias de referencia $d_{ij}^{\text{ref}}$ constituye una decisión de diseño que requiere conocimiento anatómico: pares de \textit{landmarks} con alta correlación espacial y baja variabilidad inter-paciente son candidatos ideales. Thaler et al.~\cite{Thaler2021} propusieron métodos de análisis de forma basados en CT que identifican automáticamente restricciones de distancia anatómicamente significativas mediante análisis estadístico de formas.

Una limitación de \textit{Distance Preservation Loss} es que las distancias de referencia $d_{ij}^{\text{ref}}$ deben ser específicas de la población y potencialmente específicas de la condición patológica: pacientes con cardiomegalia exhibirán distancias características diferentes a pacientes con anatomía normal. No obstante, para restricciones suficientemente generales (como distancias entre estructuras óseas relativamente rígidas), esta función de pérdida proporciona regularización valiosa.

\subsubsection{Función de Pérdida Combinada}

En la práctica, las funciones de pérdida de regresión de coordenadas y las restricciones geométricas se combinan mediante suma ponderada:
\begin{equation}
\mathcal{L}_{\text{total}} = \lambda_1 \mathcal{L}_{\text{Wing}} + \lambda_2 \mathcal{L}_{\text{sym}} + \lambda_3 \mathcal{L}_{\text{dist}}
\label{eq:combined_loss}
\end{equation}
donde $\lambda_1, \lambda_2, \lambda_3 \geq 0$ son hiperparámetros que controlan el balance relativo entre precisión de coordenadas individuales y validez geométrica global. La elección de estos pesos constituye una decisión crítica: valores excesivos de $\lambda_2$ y $\lambda_3$ pueden forzar simetrías y distancias demasiado rígidas que no capturan la variabilidad anatómica real, mientras que valores demasiado pequeños no proporcionan suficiente regularización. Zeng et al.~\cite{Zeng2022} propusieron estrategias de aprendizaje auto-supervisado que aprenden automáticamente restricciones de consistencia geométrica desde datos no etiquetados, reduciendo la necesidad de especificación manual de restricciones y pesos.

La metodología específica de entrenamiento, incluyendo la selección de valores de hiperparámetros $\lambda_1, \lambda_2, \lambda_3$, las estrategias de ponderación adaptativa durante el entrenamiento, y los protocolos de validación experimental, se presentan en detalle en el Capítulo 3.

\section{Enfoques de Regresión para Detección de Landmarks}

Las funciones de pérdida presentadas en la Sección~2.5 permiten el entrenamiento supervisado de redes neuronales profundas para la tarea de detección de \textit{landmarks} anatómicos. Sin embargo, la arquitectura de salida de la red y la representación de las predicciones constituyen decisiones fundamentales que determinan la eficiencia computacional, la precisión sub-píxel, y la robustez del modelo. Existen dos paradigmas principales para la predicción de localizaciones de \textit{landmarks}: la regresión directa de coordenadas (del inglés, \textit{coordinate regression}), que predice directamente las coordenadas $(x, y)$ de cada punto como valores continuos, y la regresión de mapas de calor (del inglés, \textit{heatmap regression}), que genera mapas de probabilidad espacial bidimensionales que representan la localización de cada \textit{landmark}. Esta sección presenta el análisis matemático y arquitectónico de ambos enfoques, sus ventajas y limitaciones, y proporciona la justificación técnica para la selección del enfoque de regresión directa de coordenadas en el contexto de detección de \textit{landmarks} en radiografías de tórax.

\subsection{Regresión Directa de Coordenadas}

El enfoque de regresión directa de coordenadas formula la detección de \textit{landmarks} como un problema de regresión multi-salida donde la red neuronal aprende un mapeo directo desde la imagen de entrada hasta las coordenadas de todos los \textit{landmarks} \cite{Sun2013, Zhang2014}. Formalmente, dada una imagen $I \in \mathbb{R}^{H \times W}$ (donde $H$ y $W$ representan altura y ancho en píxeles), y un conjunto de $K$ \textit{landmarks}, el objetivo es aprender una función parametrizada:
\begin{equation}
f_\theta: \mathbb{R}^{H \times W} \to \mathbb{R}^{2K}
\label{eq:coordinate_regression}
\end{equation}
donde $\theta$ representa los parámetros de la red, y la salida es un vector de $2K$ valores continuos que representan las coordenadas $\hat{p} = [\hat{x}_1, \hat{y}_1, \hat{x}_2, \hat{y}_2, \ldots, \hat{x}_K, \hat{y}_K]^T$.

La arquitectura típica para regresión de coordenadas consta de tres componentes principales: (1) una red neuronal convolucional profunda que actúa como extractor de características (por ejemplo, ResNet-18, como se describió en la Sección~2.3), que transforma la imagen de entrada en mapas de características de alta dimensionalidad; (2) una capa de \textit{global average pooling} (GAP) que reduce cada canal de características espaciales ($h \times w$) a un valor escalar mediante promediación, generando un vector de características de dimensión fija independiente de la resolución espacial de entrada; y (3) una o dos capas completamente conectadas que mapean el vector de características a las $2K$ coordenadas de salida.

Matemáticamente, si $\phi(I; \theta_{\text{conv}}) \in \mathbb{R}^{C \times h \times w}$ representa los mapas de características generados por la red convolucional (con $C$ canales), el \textit{global average pooling} calcula:
\begin{equation}
z_c = \frac{1}{h \cdot w} \sum_{i=1}^{h} \sum_{j=1}^{w} \phi_c(I)_{i,j}, \quad c = 1, \ldots, C
\label{eq:gap}
\end{equation}
generando un vector de características $z \in \mathbb{R}^{C}$. Las capas completamente conectadas finales realizan la transformación afín:
\begin{equation}
\hat{p} = W_{\text{FC}} z + b_{\text{FC}}
\label{eq:fc_output}
\end{equation}
donde $W_{\text{FC}} \in \mathbb{R}^{2K \times C}$ y $b_{\text{FC}} \in \mathbb{R}^{2K}$ son parámetros aprendibles. Durante el entrenamiento, estos parámetros se optimizan mediante minimización de funciones de pérdida basadas en distancias euclidianas, como se discutió en la Sección~2.5.

\textbf{Ventajas de la regresión directa de coordenadas:}

\textbf{1. Eficiencia de memoria y computacional:} La salida de la red es un vector compacto de $2K$ valores, en contraste con representaciones espacialmente extensas. Para $K = 15$ \textit{landmarks}, la salida es un vector de 30 valores escalares. Esta compacidad reduce significativamente los requerimientos de memoria GPU durante el entrenamiento y la inferencia, permitiendo tamaños de lote (\textit{batch size}) más grandes y convergencia más rápida.

\textbf{2. Precisión sub-píxel inherente:} Las coordenadas se predicen como valores continuos en el espacio real $\mathbb{R}^2$, proporcionando capacidad intrínseca para localización sub-píxel sin necesidad de técnicas de refinamiento adicionales. Esta propiedad es crítica en aplicaciones médicas donde errores de fracción de píxel pueden tener relevancia diagnóstica.

\textbf{3. Arquitectura simple y estándar:} El enfoque de regresión directa es compatible con arquitecturas de clasificación estándar (como ResNet) mediante el simple reemplazo de la capa completamente conectada final, facilitando la utilización de modelos pre-entrenados en ImageNet mediante \textit{transfer learning}, como se discutió en la Sección~2.4.

\textbf{4. Reducción de hiperparámetros:} A diferencia del enfoque de mapas de calor, no requiere la selección de parámetros relacionados con la representación espacial de las predicciones (como el ancho de las Gaussianas o la resolución de salida).

\textbf{Limitaciones:}

\textbf{1. Pérdida de información espacial explícita:} El \textit{global average pooling} colapsa completamente la estructura espacial de los mapas de características. Esto puede dificultar el aprendizaje de relaciones espaciales complejas entre \textit{landmarks}, aunque este efecto puede mitigarse mediante la incorporación de restricciones geométricas explícitas en la función de pérdida, como las restricciones de simetría y preservación de distancia discutidas en la Sección~2.5.

\textbf{2. Sensibilidad a oclusiones y ambigüedad:} En presencia de oclusiones parciales o artefactos de imagen, la red debe producir una única predicción de coordenada, sin capacidad de representar incertidumbre espacial distribuida.

\subsection{Regresión de Mapas de Calor}

El enfoque de regresión de mapas de calor representa cada \textit{landmark} mediante un mapa de probabilidad espacial bidimensional que indica la probabilidad de presencia del punto en cada localización de la imagen \cite{Tompson2014, Newell2016}. Para cada \textit{landmark} $k$, la red genera un mapa de calor $H_k \in \mathbb{R}^{h \times w}$, donde $H_k(i, j) \in [0, 1]$ representa la probabilidad de que el \textit{landmark} $k$ esté localizado en la posición $(i, j)$.

Durante el entrenamiento, los mapas de calor objetivo (\textit{ground truth}) se construyen típicamente como Gaussianas bidimensionales centradas en las coordenadas anotadas $(x_k, y_k)$:
\begin{equation}
H_k^{\text{gt}}(i, j) = \exp\left(-\frac{(i - y_k)^2 + (j - x_k)^2}{2\sigma^2}\right)
\label{eq:heatmap_gaussian}
\end{equation}
donde $\sigma$ controla el ancho de la Gaussiana. La red aprende a predecir estos mapas de calor mediante minimización de funciones de pérdida como el error cuadrático medio píxel-a-píxel o entropía cruzada binaria:
\begin{equation}
\mathcal{L}_{\text{heatmap}} = \frac{1}{K \cdot h \cdot w} \sum_{k=1}^{K} \sum_{i=1}^{h} \sum_{j=1}^{w} \left(H_k(i, j) - H_k^{\text{gt}}(i, j)\right)^2
\label{eq:heatmap_loss}
\end{equation}

Las arquitecturas típicas para regresión de mapas de calor emplean diseños codificador-decodificador que preservan o reconstruyen resoluciones espaciales altas. Las arquitecturas \textit{U-Net} \cite{Ronneberger2015} y \textit{Stacked Hourglass Networks} \cite{Newell2016} son ejemplos representativos ampliamente utilizados en tareas de estimación de pose humana y análisis de imágenes médicas.

Durante la inferencia, las coordenadas de los \textit{landmarks} se extraen de los mapas de calor predichos mediante dos estrategias principales: (1) \textit{Hard argmax}, que selecciona la posición del píxel con valor máximo $\arg\max_{i,j} H_k(i, j)$, limitando la precisión a la resolución de píxel, o (2) \textit{Soft argmax} diferenciable \cite{Payer2016}, que calcula el centro de masa ponderado del mapa de calor:
\begin{equation}
\hat{x}_k = \sum_{i=1}^{h} \sum_{j=1}^{w} j \cdot \text{softmax}(H_k(i, j)), \quad \hat{y}_k = \sum_{i=1}^{h} \sum_{j=1}^{w} i \cdot \text{softmax}(H_k(i, j))
\label{eq:soft_argmax}
\end{equation}
proporcionando capacidad de localización sub-píxel y diferenciabilidad completa para entrenamiento de extremo a extremo.

\textbf{Ventajas de la regresión de mapas de calor:}

\textbf{1. Información espacial explícita:} Los mapas de calor preservan la estructura espacial bidimensional de las predicciones, facilitando el aprendizaje de relaciones espaciales complejas y contexto anatómico.

\textbf{2. Robustez a oclusiones y ambigüedad:} El enfoque puede representar distribuciones de probabilidad multimodales o difusas, capturando incertidumbre en la localización de \textit{landmarks} parcialmente ocluidos o ambiguos.

\textbf{3. Supervisión densa:} La función de pérdida proporciona señales de gradiente en todas las localizaciones espaciales, potencialmente facilitando la convergencia del entrenamiento.

\textbf{Limitaciones:}

\textbf{1. Costo computacional y de memoria:} La generación de $K$ mapas de calor de resolución $h \times w$ requiere memoria proporcional a $K \cdot h \cdot w$, que puede ser substancialmente mayor que el vector de $2K$ coordenadas. Para $K = 15$ \textit{landmarks} y resolución de salida $64 \times 64$, se requiere almacenar 61,440 valores en comparación con 30 valores del enfoque de coordenadas directas.

\textbf{2. Velocidad de entrenamiento reducida:} Las arquitecturas codificador-decodificador con conexiones de salto y múltiples etapas de deconvolución son computacionalmente más costosas que las arquitecturas estándar con \textit{global pooling}.

\textbf{3. Hiperparámetros adicionales:} El ancho de la Gaussiana $\sigma$ en la Ecuación~\ref{eq:heatmap_gaussian} es un hiperparámetro crítico que debe seleccionarse cuidadosamente: valores pequeños proporcionan supervisión más precisa pero pueden dificultar la convergencia, mientras que valores grandes facilitan el aprendizaje pero reducen la precisión de localización.

\subsection{Comparación y Selección de Enfoque}

La Tabla~\ref{tab:coordinate_vs_heatmap} presenta una comparación sistemática de los aspectos técnicos y prácticos de ambos enfoques.

\begin{table}[h]
\centering
\caption{Comparación de enfoques de regresión directa de coordenadas y mapas de calor para detección de \textit{landmarks}.}
\label{tab:coordinate_vs_heatmap}
\small
\begin{tabular}{|l|c|c|}
\hline
\textbf{Aspecto} & \textbf{Regresión de Coordenadas} & \textbf{Regresión de Mapas de Calor} \\
\hline
Memoria (salida) & $2K$ valores & $K \cdot h \cdot w$ valores \\
\hline
Precisión sub-píxel & Inherente (continuo) & Requiere soft-argmax \\
\hline
Arquitectura & ResNet + GAP + FC & U-Net / Hourglass \\
\hline
Velocidad entrenamiento & Rápida & Moderada-Lenta \\
\hline
Información espacial & Implícita (colapsada) & Explícita (preservada) \\
\hline
Robustez a oclusiones & Limitada & Alta \\
\hline
Hiperparámetros & Mínimos & $\sigma$ (ancho Gaussiana) \\
\hline
Transfer learning & Directo (ResNet) & Requiere adaptación \\
\hline
Restricciones geométricas & Fácil (en espacio de coord.) & Complejo (en heatmaps) \\
\hline
\end{tabular}
\end{table}

Investigaciones recientes han explorado enfoques híbridos que combinan ambos paradigmas. Li et al.~\cite{Li2023} propusieron una arquitectura multi-tarea que predice simultáneamente mapas de calor y coordenadas, demostrando mejoras en precisión de localización en análisis de radiografías de tórax mediante la combinación de supervisión espacialmente densa y predicción directa. Jeong et al.~\cite{Jeong2023} presentaron un enfoque de regresión de coordenadas guiado por atención en mapas de características espaciales, preservando parcialmente información espacial sin la sobrecarga completa de generación de mapas de calor. Adicionalmente, enfoques basados en \textit{transformers} de visión \cite{Li2022CVPR} han demostrado capacidad para capturar relaciones espaciales de largo alcance sin la necesidad de representaciones espaciales explícitas, representando una dirección prometedora para futuras investigaciones.

\textbf{Justificación de la selección de regresión directa de coordenadas:}

En el contexto específico de detección de 15 \textit{landmarks} anatómicos en radiografías de tórax con el conjunto de datos del presente trabajo (956 imágenes de entrenamiento, como se establecerá en el Capítulo 3), el enfoque de regresión directa de coordenadas se selecciona por las siguientes razones técnicas:

\textbf{1. Eficiencia computacional en régimen de datos moderados:} Con un conjunto de datos de tamaño moderado, la eficiencia de entrenamiento y la capacidad de utilizar arquitecturas estándar pre-entrenadas en ImageNet (como se discutió en la Sección~2.4) proporcionan ventajas significativas. La regresión directa de coordenadas permite \textit{fine-tuning} directo de ResNet-18 pre-entrenado, aprovechando conocimiento transferido sin modificaciones arquitectónicas substanciales.

\textbf{2. Precisión sub-píxel natural:} La aplicación clínica requiere localización precisa de estructuras anatómicas, y el enfoque de coordenadas continuas proporciona capacidad sub-píxel inherente sin técnicas de refinamiento adicionales.

\textbf{3. Restricciones de hardware:} Las restricciones de memoria GPU (8 GB VRAM en la infraestructura utilizada, como se describirá en el Capítulo 3) favorecen el enfoque de coordenadas compactas, permitiendo tamaños de lote más grandes que aceleran la convergencia y mejoran la estimación de estadísticas de normalización por lotes.

\textbf{4. Incorporación de conocimiento anatómico mediante restricciones geométricas:} La limitación principal del enfoque de coordenadas (pérdida de información espacial explícita) se mitiga mediante la incorporación de las funciones de pérdida de simetría bilateral y preservación de distancias anatómicas presentadas en la Sección~2.5. Estas restricciones geométricas imponen explícitamente conocimiento anatómico sobre la geometría del tórax, compensando la falta de supervisión espacialmente densa de los mapas de calor.

\textbf{5. Reducción de hiperparámetros:} La eliminación del hiperparámetro $\sigma$ de ancho de Gaussiana reduce el espacio de búsqueda de hiperparámetros, simplificando el proceso de validación experimental que se presentará en el Capítulo 4.

La combinación de regresión directa de coordenadas con la función de pérdida compuesta $\mathcal{L}_{\text{total}} = \lambda_1 \mathcal{L}_{\text{Wing}} + \lambda_2 \mathcal{L}_{\text{sym}} + \lambda_3 \mathcal{L}_{\text{dist}}$ (Ecuación~2.31, Sección~2.5) constituye el enfoque metodológico adoptado en este trabajo, balanceando eficiencia computacional, precisión de localización, y aprovechamiento de conocimiento anatómico estructurado. El estado del arte de métodos de detección de \textit{landmarks} en imágenes médicas, incluyendo enfoques basados en coordenadas, mapas de calor, y técnicas híbridas, se analiza exhaustivamente en la sección subsecuente.

\section{Estado del Arte en Detección Automática de Landmarks Anatómicos}

La detección automática de \textit{landmarks} anatómicos en imágenes médicas ha experimentado una transformación fundamental en la última década. Los métodos clásicos basados en modelos estadísticos de forma, particularmente Active Shape Models (ASM) \cite{Cootes1995} y Active Appearance Models (AAM) \cite{Cootes2001}, dominaron el campo durante los años 1990-2000, requiriendo inicialización manual cercana a la solución y siendo sensibles a variaciones de iluminación y pose. La revolución del aprendizaje profundo, iniciada con AlexNet \cite{Krizhevsky2012}, transformó radicalmente el panorama: Sun et al.~\cite{Sun2013} demostraron en 2013 que redes neuronales convolucionales profundas superaban métodos clásicos en detección de \textit{landmarks} faciales mediante una cascada de redes con refinamiento progresivo. La introducción de arquitecturas residuales profundas \cite{He2016} y técnicas avanzadas de regresión de mapas de calor \cite{Tompson2014, Newell2016} consolidaron el aprendizaje profundo como el paradigma dominante. En el dominio de imágenes médicas, tres enfoques metodológicos principales han emergido: (1) regresión directa de coordenadas, que predice localizaciones $(x, y)$ como valores continuos con eficiencia computacional superior; (2) regresión de mapas de calor, que genera distribuciones de probabilidad espacial con capacidad de representar incertidumbre; y (3) métodos híbridos y basados en transformadores, que combinan fortalezas de múltiples paradigmas o explotan mecanismos de atención global. Simultáneamente, la disponibilidad de conjuntos de datos ha evolucionado desde colecciones pequeñas de cientos de imágenes hacia repositorios masivos como CheXpert (224,316 radiografías de tórax) y MIMIC-CXR (377,110 estudios), facilitando el entrenamiento de modelos cada vez más robustos y generalizables. Esta sección presenta una revisión exhaustiva del estado del arte, organizada según el enfoque metodológico principal, seguida de un análisis comparativo de los trabajos más relevantes publicados entre 2016 y 2024.

\subsection{Métodos Basados en Regresión de Coordenadas}

El enfoque de regresión directa de coordenadas formula la detección de \textit{landmarks} como un problema de regresión multi-salida donde redes neuronales convolucionales profundas aprenden mapeos directos desde imágenes hacia vectores de coordenadas. Sun et al.~\cite{Sun2013} propusieron una arquitectura pionera denominada Deep Convolutional Network Cascade para \textit{landmarks} faciales, consistente en tres niveles de refinamiento progresivo: una red inicial predice localizaciones aproximadas en la imagen completa, seguida de redes subsecuentes que refinan las predicciones en regiones locales de tamaño decreciente. Esta estrategia de \textit{coarse-to-fine} alcanzó un error normalizado de 5.5\% en el conjunto de datos LFPW, superando métodos clásicos basados en AAM por márgenes significativos. Zhang et al.~\cite{Zhang2014} extendieron este enfoque mediante aprendizaje multi-tarea, demostrando que la predicción simultánea de \textit{landmarks} faciales y atributos semánticos (presencia de gafas, género, expresión) mejora la precisión de localización: las tareas auxiliares actúan como regularizadores que fuerzan a la red a aprender representaciones más generalizables. Bulat y Tzimiropoulos~\cite{Bulat2017} establecieron \textit{benchmarks} de referencia mediante la construcción de un conjunto de datos masivo de 230,000 \textit{landmarks} faciales 3D, alcanzando un error promedio de 3.12 píxeles en el conjunto 300-W, y demostraron la importancia de la escala de datos de entrenamiento para la robustez de los modelos. Una observación consistente en la literatura es que la regresión de coordenadas ha sido preferida en aplicaciones de imágenes médicas debido a su eficiencia computacional y capacidad de predicción sub-píxel inherente.

En el dominio específico de imágenes médicas, Noothout et al.~\cite{Noothout2020} presentaron en \textit{IEEE Transactions on Medical Imaging} un enfoque de localización global-a-local utilizando redes neuronales completamente convolucionales (FCNNs) para la detección de 19 \textit{landmarks} cefalométricos en radiografías laterales de cráneo. Su método opera en dos etapas: una red de localización gruesa identifica regiones de interés que contienen cada \textit{landmark}, seguida de una red de refinamiento que predice coordenadas precisas dentro de parches locales. Evaluado en un conjunto de datos de 400 radiografías, el método alcanzó un error de $1.21 \pm 0.89$ mm, demostrando además transferibilidad entre modalidades de imagen (angiografía por tomografía computarizada, resonancia magnética, y radiografía). Oh et al.~\cite{Oh2020} propusieron en \textit{IEEE Journal of Biomedical and Health Informatics} un enfoque de aprendizaje de características de contexto anatómico profundo para \textit{landmarks} cefalométricos, incorporando mecanismos de atención guiados por contexto que permiten a la red enfocarse en regiones anatómicas relevantes. Su arquitectura basada en DenseNet alcanzó un error de 1.18 mm en un conjunto de datos de 935 radiografías laterales, representando el estado del arte en detección cefalométrica mediante regresión de coordenadas al momento de publicación. Li et al.~\cite{Li2023} presentaron en \textit{Scientific Reports} un enfoque híbrido que combina regresión de coordenadas y mapas de calor para 46 \textit{landmarks} en radiografías de tórax posteroanterior, utilizando un conjunto de datos de 956 imágenes. Su método incorpora restricciones de simetría bilateral en la función de pérdida, explotando la propiedad anatómica de simetría del tórax, y alcanzó un error promedio de 4.22 píxeles. Este trabajo representa la aplicación más exhaustiva de detección de \textit{landmarks} en radiografías de tórax en términos de número de puntos anatómicos, demostrando que la combinación de restricciones geométricas con regresión eficiente es viable en conjuntos de datos de tamaño moderado.

\subsection{Métodos Basados en Mapas de Calor}

La regresión de mapas de calor representa cada \textit{landmark} mediante una distribución de probabilidad espacial bidimensional, típicamente una Gaussiana centrada en la localización objetivo. Tompson et al.~\cite{Tompson2014} fueron pioneros en la aplicación de este enfoque para estimación de pose humana, combinando redes neuronales convolucionales con modelos gráficos que capturan dependencias espaciales entre articulaciones. Su método genera mapas de calor Gaussianos con desviación estándar $\sigma = 1$ píxel para cada articulación, optimizados mediante error cuadrático medio píxel-a-píxel. Newell et al.~\cite{Newell2016} revolucionaron el campo con la introducción de \textit{Stacked Hourglass Networks}, una arquitectura multi-escala iterativa que procesa características en múltiples resoluciones mediante módulos codificador-decodificador apilados secuencialmente. Cada módulo \textit{hourglass} realiza \textit{downsampling} progresivo para capturar contexto global, seguido de \textit{upsampling} simétrico con conexiones de salto que preservan detalles espaciales. La composición de múltiples módulos (típicamente 4-8) permite refinamiento iterativo de predicciones, alcanzando PCKh@0.5 = 90.9\% en el conjunto de datos MPII Human Pose, estableciendo un nuevo estado del arte que persiste como referencia fundamental. Yang et al.~\cite{Yang2017} demostraron la transferibilidad de la arquitectura \textit{Stacked Hourglass} desde estimación de pose humana hacia detección de \textit{landmarks} faciales, evidenciando que las representaciones jerárquicas multi-escala son genéricas a través de dominios anatómicos.

En aplicaciones de imágenes médicas, Payer et al.~\cite{Payer2019Spatial} presentaron en \textit{Medical Image Analysis} una extensión de regresión de mapas de calor que integra redes de configuración espacial (\textit{spatial configuration networks}) para \textit{landmarks} en radiografías de mano. Su método incorpora un mecanismo de \textit{soft-argmax} diferenciable que permite extracción de coordenadas sub-píxel a partir de mapas de calor mientras mantiene diferenciabilidad completa para entrenamiento de extremo a extremo: $\hat{x}_k = \sum_{i,j} j \cdot \text{softmax}(H_k(i,j))$, $\hat{y}_k = \sum_{i,j} i \cdot \text{softmax}(H_k(i,j))$. La arquitectura basada en U-Net alcanzó un error de $1.87 \pm 0.98$ mm en un conjunto de datos de 895 imágenes, demostrando que la preservación de información espacial explícita proporciona ventajas en presencia de estructuras anatómicas complejas con múltiples \textit{landmarks} densamente distribuidos. Zhang et al.~\cite{Zhang2020MIA} propusieron en \textit{Medical Image Analysis} redes neuronales convolucionales en cascada con arquitectura U-Net para 19 \textit{landmarks} cefalométricos, implementando una estrategia de tres etapas de refinamiento progresivo de \textit{coarse-to-fine}. Su método alcanzó un error de $1.35 \pm 0.89$ mm en un conjunto de datos de 1,000 radiografías laterales, superando métodos clásicos basados en ASM por 42\% y demostrando la superioridad de enfoques basados en aprendizaje profundo sobre técnicas estadísticas tradicionales. Cheng et al.~\cite{Cheng2023} presentaron en \textit{Medical Image Analysis} un enfoque de aprendizaje basado en perturbaciones con regresión de mapas de calor para 18 \textit{landmarks} en radiografías de tórax posteroanterior. Su método incorpora aumentación de datos geométrica avanzada mediante perturbaciones controladas durante el entrenamiento, alcanzando un error de 3.78 píxeles en un conjunto de datos de 2,000 imágenes. Thaler et al.~\cite{Thaler2021} introdujeron modelado de incertidumbre mediante mapas de calor Gaussianos con enfoque Bayesiano para radiografías de mano, cuantificando explícitamente la incertidumbre de localización para cada \textit{landmark} y alcanzando un error de 2.12 mm. Una limitación consistente de métodos basados en mapas de calor es el sobrecosto de memoria proporcional a $K \times h \times w$ (número de \textit{landmarks} × resolución espacial), significativamente superior al vector compacto de $2K$ valores de regresión de coordenadas, con impacto directo en el tamaño de lote durante el entrenamiento y velocidad de inferencia.

\subsection{Métodos Híbridos y Basados en Transformers}

La tercera categoría de enfoques combina elementos de regresión de coordenadas y mapas de calor, o introduce arquitecturas basadas en mecanismos de atención que superan limitaciones de receptive fields finitos en redes convolucionales. Quan et al.~\cite{YOLO2021} presentaron en MICCAI 2021 el enfoque ``You Only Learn Once'' (YOLO), un detector universal de \textit{landmarks} anatómicos entrenado simultáneamente en conjuntos de datos mixtos (cefalométricos, mano, columna vertebral) conteniendo más de 150 \textit{landmarks} diferentes distribuidos en múltiples anatomías. El método demuestra capacidad de generalización cruzada entre anatomías, alcanzando errores variables de 1.5-3.2 mm según la región anatómica específica, y evidencia que el entrenamiento multi-dominio mejora la robustez mediante exposición a diversidad anatómica. Ma y Luo~\cite{AdaptiveLoss2021} propusieron en \textit{IEEE Journal of Biomedical and Health Informatics} una función de pérdida adaptativa de grano fino que ajusta dinámicamente los pesos de cada \textit{landmark} según su dificultad de detección empírica durante el entrenamiento. Su función de pérdida se formula como $\mathcal{L} = \sum_{k=1}^K \alpha_k \cdot \text{MSE}_k$, donde los pesos $\alpha_k$ se aprenden mediante un módulo de atención que evalúa la magnitud de gradientes históricos. Aplicado a 19 \textit{landmarks} cefalométricos, este enfoque proporciona mejoras de 8.3\% sobre MSE estándar, demostrando que la adaptación dinámica de la función de pérdida es una dirección prometedora. Kang et al.~\cite{Kang2021} presentaron en \textit{Scientific Reports} detección de \textit{landmarks} cefalométricos 3D mediante aprendizaje por refuerzo profundo multi-etapa en imágenes CBCT (tomografía computarizada de haz cónico), formulando la localización como un proceso de decisión de Markov donde un agente aprende secuencias de acciones que minimizan la distancia al \textit{landmark} objetivo. Su enfoque alcanzó un error de $1.82 \pm 1.03$ mm en espacio tridimensional con un conjunto de datos de 350 exploraciones volumétricas.

La introducción de arquitecturas \textit{Vision Transformer} ha generado interés significativo en años recientes. Li et al.~\cite{Li2022CVPR} propusieron en CVPR 2022 una arquitectura de transformadores en cascada para \textit{landmarks} faciales, donde mecanismos de \textit{self-attention} global capturan relaciones espaciales de largo alcance entre \textit{landmarks} sin las limitaciones de campos receptivos finitos inherentes a convoluciones. Su método alcanzó NME (error medio normalizado) de 2.98\% en el conjunto de datos WFLW con 98 \textit{landmarks} faciales, demostrando competitividad con enfoques convolucionales mientras proporciona interpretabilidad superior mediante visualización de mapas de atención. Huang et al.~\cite{Huang2023} presentaron en MICCAI 2023 un modelo híbrido Transformer-CNN (HTC) con aprendizaje de mapas de calor multi-resolución, combinando extracción de características locales mediante bloques convolucionales con modelado de contexto global mediante bloques de transformador. Su arquitectura superó una línea base ResNet-50 por 11.2\% en \textit{landmarks} de radiografías de tórax, evidenciando que la hibridación de arquitecturas convolucionales y basadas en atención explota complementariedad: convoluciones capturan patrones locales eficientemente mediante inductive bias de localidad, mientras transformadores modelan dependencias globales sin restricciones espaciales. Jeong et al.~\cite{Jeong2023} presentaron en \textit{Sensors} regresión de coordenadas guiada por atención con características de mapas de calor intermedias para \textit{landmarks} faciales, alcanzando un error de 5.13 píxeles en radiografías de tórax mediante un mecanismo híbrido que genera mapas de calor en capas intermedias para guiar la regresión final de coordenadas.

Gaggion et al.~\cite{Gaggion2023} introdujeron en \textit{IEEE Transactions on Medical Imaging} HybridGNet, una arquitectura basada en redes neuronales de grafo (Graph Neural Networks, GNN) que incorpora conocimiento anatómico previo en segmentación de radiografías de tórax mediante representación de \textit{landmarks} como nodos de un grafo donde las aristas representan relaciones anatómicas (adyacencia espacial, simetría bilateral, jerarquía anatómica). Su método mejora la plausibilidad anatómica de segmentaciones en 15.6\% comparado con U-Net estándar, demostrando que la codificación explícita de estructura anatómica mediante grafos constituye una dirección prometedora. Liu et al.~\cite{Liu2021Structure} propusieron en CVPR 2021 detección de \textit{landmarks} con conciencia de estructura mediante GCN para capturar relaciones espaciales explícitas entre \textit{landmarks} faciales, modelando las dependencias geométricas como un grafo totalmente conectado donde cada \textit{landmark} se conecta con todos los demás, permitiendo propagación de información contextual. La tendencia emergente hacia integración de conocimiento previo geométrico mediante representaciones de grafo representa una convergencia entre aprendizaje profundo basado en datos y modelado estructurado tradicional.

\subsection{Funciones de Pérdida Especializadas y Restricciones Geométricas}

La función de pérdida constituye un componente crítico que determina qué propiedades de las predicciones son optimizadas durante el entrenamiento. Feng et al.~\cite{Feng2018} introdujeron en CVPR 2018 la función \textit{Wing Loss} para localización robusta de \textit{landmarks} faciales, diseñada para amplificar gradientes en el régimen de errores pequeños mediante una curva basada en logaritmo: $\mathcal{L}_{\text{wing}}(x) = w \ln(1 + |x|/\epsilon)$ para $|x| < w$, donde $w$ controla el ancho de la región no lineal y $\epsilon$ limita el gradiente en $x=0$. Con parámetros típicos $w=10$, $\epsilon=2$, \textit{Wing Loss} proporciona mejoras de 12.5\% sobre MSE en \textit{landmarks} faciales con errores menores a 2 píxeles, alcanzando NME 4.04\% en el conjunto 300-W. La intuición fundamental es que MSE cuadrático genera gradientes que decrecen linealmente con el error, proporcionando señal de optimización débil en el régimen de alta precisión, mientras \textit{Wing Loss} mantiene gradientes substanciales incluso para errores muy pequeños, acelerando la convergencia hacia localizaciones precisas. Wang et al.~\cite{Wang2019} extendieron este concepto con \textit{Adaptive Wing Loss} que ajusta dinámicamente los parámetros $w$ y $\epsilon$ durante el entrenamiento, aplicado a regresión de mapas de calor, proporcionando mejoras adicionales de 5.3\% sobre \textit{Wing Loss} estándar. Ma y Luo~\cite{AdaptiveLoss2021} generalizaron la adaptación a nivel de \textit{landmark} individual mediante pesos específicos basados en dificultad empírica, donde la función de pérdida adaptativa asigna mayor énfasis a \textit{landmarks} con historial de errores elevados. El impacto de \textit{Wing Loss} ha sido substancial: la función ha sido adoptada ampliamente en aplicaciones de imágenes médicas debido a su robustez a \textit{outliers} y enfoque en errores pequeños clínicamente relevantes.

Las restricciones geométricas basadas en conocimiento anatómico representan una segunda categoría de especialización de funciones de pérdida. Song et al.~\cite{Song2020} incorporaron restricciones de simetría para \textit{landmarks} cefalométricos mediante penalización de asimetrías bilaterales: $\mathcal{L}_{\text{sym}} = \sum_{(i,j) \in S} \|p_i - \text{mirror}(p_j)\|_2^2$, donde $S$ denota pares de \textit{landmarks} simétricos y $\text{mirror}(\cdot)$ representa reflexión respecto al plano sagital medio. Esta restricción proporciona mejoras de 6.8\% específicamente en \textit{landmarks} con simetría bilateral, explotando la propiedad anatómica fundamental de que estructuras bilaterales deben ser aproximadamente simétricas en individuos sanos. Thaler et al.~\cite{Thaler2021} incorporaron preservación de distancias mediante restricciones basadas en modelos estadísticos de forma, penalizando desviaciones de distancias inter-\textit{landmark} respecto a valores de referencia anatómicos: $\mathcal{L}_{\text{dist}} = \sum_{(i,j) \in D} (\|p_i - p_j\|_2 - d_{ij}^{\text{ref}})^2$, donde $D$ denota pares de \textit{landmarks} con distancia anatómica conocida y $d_{ij}^{\text{ref}}$ representa la distancia de referencia. Urschler et al.~\cite{Urschler2021} presentaron en \textit{Pattern Recognition Letters} integración de restricciones geométricas mediante conocimiento previo de forma aprendido de datos de entrenamiento, utilizando análisis de componentes principales sobre configuraciones de \textit{landmarks} para definir un espacio de formas anatómicamente plausibles.

Kendall y Gal~\cite{Kendall2017} introdujeron en NeurIPS 2017 un marco para cuantificación de incertidumbre en aprendizaje profundo Bayesiano para visión por computadora, distinguiendo entre incertidumbre aleatoria (inherente a los datos) y epistémica (incertidumbre del modelo). Liu et al.~\cite{Liu2024Uncertainty} extendieron este enfoque a detección de \textit{landmarks} anatómicos en imágenes médicas, presentando en \textit{IEEE Transactions on Medical Imaging} 2024 un método de aprendizaje profundo con conciencia de incertidumbre que predice intervalos de confianza para cada \textit{landmark}, permitiendo detección automática de predicciones con alta incertidumbre que requieren revisión manual. Su método alcanzó errores de 1.5-2.8 píxeles en múltiples dominios anatómicos mientras proporciona calibración de incertidumbre superior. Un gap crítico identificado en la literatura es que pocos trabajos combinan múltiples componentes de función de pérdida simultáneamente: la mayoría de métodos utiliza MSE estándar o una única restricción geométrica, sin exploración sistemática de combinaciones de \textit{Wing Loss}, restricciones de simetría, y preservación de distancias.

\subsection{Análisis Comparativo y Posicionamiento del Presente Trabajo}

La Tabla~\ref{tab:state_of_art} presenta una comparación exhaustiva de trabajos representativos en detección de \textit{landmarks} anatómicos en imágenes médicas publicados entre 2016 y 2024. Los criterios de inclusión fueron: (1) aplicación a imágenes médicas (radiografías, tomografía computarizada, resonancia magnética), (2) utilización de aprendizaje profundo, (3) evaluación cuantitativa reportada con métricas de error de localización, y (4) publicación en \textit{venues} de alto impacto académico (IEEE Transactions on Medical Imaging, IEEE Journal of Biomedical and Health Informatics, Medical Image Analysis, CVPR/ICCV, MICCAI, Scientific Reports, Applied Sciences, Sensors). Los criterios de comparación incluyen: método/enfoque (coordinate regression, heatmap regression, o híbrido), arquitectura de red neuronal, tipo y tamaño del conjunto de datos, función de pérdida utilizada, error promedio de localización reportado, y dominio anatómico específico. La tabla evidencia la diversidad de enfoques metodológicos y la evolución temporal hacia arquitecturas más sofisticadas y funciones de pérdida especializadas.

\begin{table}[p]
\centering
\caption{Estado del arte en detección de \textit{landmarks} en imágenes médicas (2016-2024). Las abreviaciones utilizadas son: Ceph (cefalométrico), px (píxeles), NME (error medio normalizado), GCN (Graph Convolutional Network), ViT (Vision Transformer), RL (Reinforcement Learning).}
\label{tab:state_of_art}
\scriptsize
\begin{tabular}{|p{2cm}|p{1.8cm}|p{2cm}|p{2.2cm}|p{1.8cm}|p{1.5cm}|p{1.8cm}|}
\hline
\textbf{Autor/Año} & \textbf{Método} & \textbf{Arquitectura} & \textbf{Dataset (tipo/n)} & \textbf{Loss Function} & \textbf{Error} & \textbf{Dominio} \\
\hline
Lindner 2016 & Multi-atlas clásico & N/A (no DL) & Ceph 400 & N/A & 2.0 mm & Cefalométrico \\
\hline
Yang 2017 & Heatmap & Stacked Hourglass & Facial 3000 & MSE & 3.4 px & Facial \\
\hline
Feng 2018 & Coordinate & ResNet-50 & Facial 3800 & Wing Loss & 4.04\% NME & Facial \\
\hline
Wang 2019 & Heatmap & Hourglass & Facial 4000 & Adaptive Wing & 3.81\% NME & Facial \\
\hline
Payer 2019 & Heatmap+GCN & U-Net+Spatial & Hand X-ray 895 & MSE+spatial & 1.87 mm & Mano \\
\hline
Noothout 2020 & Coordinate & FCNN & Ceph 400 & MSE & 1.21 mm & Cefalométrico \\
\hline
Oh 2020 & Coordinate & DenseNet & Ceph 935 & MSE+context & 1.18 mm & Cefalométrico \\
\hline
Song 2020 & Coordinate & ResNet-18 & Ceph 450 & MSE+symmetry & 1.45 mm & Cefalométrico \\
\hline
Zhang 2020 & Heatmap cascade & U-Net (3-stage) & Ceph 1000 & MSE & 1.35 mm & Cefalométrico \\
\hline
Kang 2021 & 3D RL-based & 3D CNN & CBCT 350 & Reward-based & 1.82 mm & Ceph 3D \\
\hline
Ma \& Luo 2021 & Heatmap & U-Net & Ceph 800 & Adaptive Loss & 1.29 mm & Cefalométrico \\
\hline
Thaler 2021 & Heatmap Bayesian & U-Net & Hand 600 & MSE+uncertainty & 2.12 mm & Mano \\
\hline
Quan 2021 & Universal & ResNet-101 & Mixed 2000+ & MSE multi-task & 1.5-3.2 mm & Multi-anatomía \\
\hline
Liu 2021 & Coordinate+GCN & ResNet+GCN & Facial 3000 & Wing+structure & 3.2 px & Facial \\
\hline
Li 2022 & Transformer & ViT cascade & Facial 5000 & Wing Loss & 2.98\% NME & Facial \\
\hline
Cheng 2023 & Heatmap+perturbation & U-Net & Chest X-ray 2000 & MSE & 3.78 px & Tórax \\
\hline
Gaggion 2023 & GNN hybrid & HybridGNet & Chest X-ray 1500 & MSE+graph & N/A (seg) & Tórax \\
\hline
Huang 2023 & Hybrid Trans-CNN & ViT+ResNet & X-ray multi 1200 & MSE heatmap & 2.8 px & Multi X-ray \\
\hline
Li 2023 & Hybrid coord+heat & ResNet-34 & Chest 956 (46 lmks) & MSE+symmetry & 4.22 px & Tórax \\
\hline
Jeong 2023 & Attention-guided & ResNet-50+attn & Chest 800 & Wing Loss & 5.13 px & Tórax \\
\hline
Liu 2024 & Uncertainty-aware & ResNet+Bayesian & Multi 3500 & MSE+uncertainty & 1.5-2.8 px & Multi-dominio \\
\hline
\end{tabular}
\end{table}

El análisis de la Tabla~\ref{tab:state_of_art} revela múltiples tendencias significativas en la evolución del campo. Primero, se observa una transición temporal inequívoca desde métodos clásicos hacia aprendizaje profundo: Lindner et al.~\cite{Lindner2016} reportaron en 2016 un error de 2.0 mm utilizando enfoques multi-atlas tradicionales, mientras que Zhang et al.~\cite{Zhang2020MIA} alcanzaron 1.35 mm con redes neuronales convolucionales en cascada en 2020, representando una mejora del 32.5\%. Segundo, existe una divergencia metodológica según el dominio de aplicación: la estimación de pose humana y detección de \textit{landmarks} faciales favorecen regresión de mapas de calor (Newell 2016, Yang 2017) debido a la preservación de contexto espacial y capacidad de representar incertidumbre multimodal, mientras que aplicaciones en imágenes médicas prefieren predominantemente regresión de coordenadas (Noothout 2020, Oh 2020, Li 2023) por eficiencia computacional y menor consumo de memoria GPU. Tercero, el aprendizaje por transferencia desde ImageNet se ha establecido como práctica universal: todos los trabajos publicados después de 2018 utilizan pre-entrenamiento en ImageNet, sin reportes de entrenamiento desde inicialización aleatoria, evidenciando la importancia crítica de representaciones pre-aprendidas discutida en la Sección~2.4. Cuarto, existe una disparidad substancial en tamaños de conjuntos de datos entre dominio médico y visión por computadora general: conjuntos médicos típicamente contienen 400-2000 imágenes (Noothout 400, Zhang 1000, Li 956) versus conjuntos faciales/pose con más de 3000 ejemplos (Li 2022 con 5000, Liu 2021 con 3000), representando un factor de 3-10× de diferencia que refleja las dificultades de adquisición y anotación de datos médicos. Quinto, se observa una evolución en funciones de pérdida desde MSE estándar (dominante 2016-2019) hacia \textit{Wing Loss} (2018+) y funciones multi-componente adaptativas (2021+), reflejando comprensión creciente de que la función de pérdida debe alinearse con los requisitos específicos de la aplicación. Finalmente, la emergencia de arquitecturas basadas en transformadores post-2022 (Li 2022, Huang 2023) indica un cambio paradigmático hacia captura de contexto global mediante mecanismos de atención, aunque las redes neuronales convolucionales mantienen predominancia en imágenes médicas debido a eficiencia computacional y menores requerimientos de datos.

El análisis comparativo identifica múltiples gaps y limitaciones en el estado del arte actual que motivan la presente investigación. Primero, la combinación de restricciones geométricas es limitada: la mayoría de trabajos incorpora una única restricción (Song 2020 utiliza exclusivamente simetría, Payer 2019 solo configuración espacial, Thaler 2021 únicamente modelado de incertidumbre). Li et al.~\cite{Li2023} combinan regresión híbrida coordinate+heatmap con restricciones de simetría, pero utilizan MSE estándar en lugar de \textit{Wing Loss}, perdiendo los beneficios de amplificación de gradientes en errores pequeños. Ningún trabajo reportado en la literatura combina simultáneamente \textit{Wing Loss}, restricciones de simetría bilateral, y preservación de distancias anatómicas para radiografías de tórax, representando un gap crítico. Segundo, los estudios de ablación cuantitativos son limitados: solo 11 de 20 trabajos revisados reportan ablaciones sistemáticas que cuantifican el impacto individual de cada componente de sus funciones de pérdida compuestas, dificultando la comprensión de qué restricciones geométricas proporcionan las mayores contribuciones al desempeño. Tercero, existe un trade-off no explorado sistemáticamente entre número de \textit{landmarks} y tamaño de conjunto de datos: Li 2023 utiliza 46 \textit{landmarks} (el más exhaustivo para radiografías de tórax) pero con 956 imágenes, mientras Cheng 2023 utiliza 2000 imágenes pero solo 18 \textit{landmarks}, sin análisis de cómo esta relación afecta la capacidad de generalización. Cuarto, las restricciones de simetría bilateral son substancialmente infrautilizadas: solo Song 2020 y Li 2023 explotan explícitamente la simetría bilateral en radiografías cefalométricas y de tórax, a pesar de ser una propiedad anatómica fundamental que podría proporcionar supervisión adicional sin requerir anotaciones adicionales. Finalmente, la exploración de restricciones de distancias anatómicas es limitada: Thaler 2021 y Payer 2019 utilizan restricciones de distancia en radiografías de mano, pero este enfoque no ha sido aplicado sistemáticamente a radiografías de tórax con sus 7 pares simétricos específicos y relaciones de distancia anatómicamente consistentes entre estructuras mediastinales y costales (como se estableció en la Sección~2.1).

El presente trabajo se posiciona en la intersección de tres líneas de investigación complementarias: (1) regresión eficiente de coordenadas para radiografías de tórax siguiendo los enfoques exitosos de Li et al.~\cite{Li2023} y Jeong et al.~\cite{Jeong2023}, (2) \textit{Wing Loss} para amplificación de gradientes en el régimen de errores pequeños clínicamente relevantes como demostrado por Feng et al.~\cite{Feng2018} en \textit{landmarks} faciales, y (3) restricciones geométricas anatómicas que explotan propiedades estructurales del tórax (Song et al.~\cite{Song2020} para simetría, Payer et al.~\cite{Payer2019Spatial} para preservación de distancias). La contribución única del trabajo es la primera aplicación de la combinación \textit{Wing Loss} + restricciones de simetría bilateral + preservación de distancias anatómicas para 15 \textit{landmarks} en radiografías de tórax posteroanterior utilizando una arquitectura ResNet-18 eficiente pre-entrenada en ImageNet. La función de pérdida compuesta propuesta $\mathcal{L}_{\text{total}} = \lambda_1 \mathcal{L}_{\text{Wing}} + \lambda_2 \mathcal{L}_{\text{sym}} + \lambda_3 \mathcal{L}_{\text{dist}}$ (presentada en detalle en la Sección~2.5) integra conocimiento anatómico específico del tórax sin incrementar la complejidad arquitectónica: las 7 pares de \textit{landmarks} simétricos bilaterales y 2 puntos localizados en la línea media (identificados en la Tabla~2.1.1 de la Sección~2.1) definen naturalmente las restricciones geométricas, y las distancias anatómicas entre pares específicos de \textit{landmarks} (por ejemplo, entre ápices pulmonares, entre ángulos costofénicos) proporcionan supervisión adicional basada en variabilidad anatómica limitada. La metodología experimental completa, incluyendo estudios de ablación sistemáticos para cuantificar el impacto de cada componente de la función de pérdida, se presenta en el Capítulo 3, mientras que los resultados comparativos con el estado del arte se discuten en el Capítulo 4.

La revisión exhaustiva del estado del arte presentada en esta sección evidencia la madurez del campo de detección automática de \textit{landmarks} mediante aprendizaje profundo, con errores de localización alcanzando precisión sub-milimétrica en aplicaciones cefalométricas controladas (Oh 2020: 1.18 mm, Ma \& Luo 2021: 1.29 mm) y 3-5 píxeles en radiografías de tórax con mayor variabilidad inter-paciente e inter-institucional (Li 2023: 4.22 píxeles, Cheng 2023: 3.78 píxeles). Sin embargo, los gaps identificados en combinación de funciones de pérdida especializadas y restricciones geométricas anatómicas específicas motivan la investigación de enfoques que integren múltiples fuentes de conocimiento anatómico simultáneamente. La siguiente sección presenta una síntesis del marco teórico completo desarrollado en el Capítulo 2, conectando los fundamentos de aprendizaje profundo, arquitecturas residuales, aprendizaje por transferencia, funciones de pérdida especializadas, y estado del arte revisado con la metodología experimental que será presentada en el Capítulo 3.

\section{Síntesis del Marco Teórico}

El presente capítulo ha desarrollado un marco teórico comprehensivo que fundamenta la detección automática de \textit{landmarks} anatómicos en radiografías de tórax mediante aprendizaje profundo. La estructura del capítulo integra progresivamente: (1) los principios físicos de formación de imágenes radiográficas y la definición anatómica de 15 \textit{landmarks} específicos con propiedades de simetría bilateral (Sección~2.1), (2) los fundamentos matemáticos de redes neuronales convolucionales incluyendo la operación de convolución, retropropagación de gradientes, y algoritmos de optimización (Sección~2.2), (3) las arquitecturas residuales profundas que permiten el entrenamiento efectivo de redes con decenas de capas mediante conexiones de atajo y normalización por lotes (Sección~2.3), (4) el paradigma de aprendizaje por transferencia que aprovecha representaciones pre-aprendidas en ImageNet para mejorar el desempeño en el dominio médico con datos limitados (Sección~2.4), (5) las funciones de pérdida especializadas que incorporan amplificación de gradientes para errores pequeños y restricciones geométricas anatómicas (Sección~2.5), (6) el análisis comparativo de enfoques de regresión de coordenadas versus mapas de calor con justificación técnica para la selección del primero (Sección~2.6), y (7) la revisión exhaustiva del estado del arte con tabla comparativa de 21 trabajos publicados entre 2016 y 2024, identificando gaps específicos en la combinación de funciones de pérdida y restricciones geométricas (Sección~2.7).

La integración conceptual de estos elementos constituye la base metodológica del presente trabajo. Las arquitecturas ResNet-18 presentadas en la Sección~2.3, con 11.7 millones de parámetros distribuidos en bloques residuales básicos con conexiones de atajo, proporcionan un balance óptimo entre capacidad expresiva y eficiencia computacional apropiado para conjuntos de datos médicos de tamaño moderado. El aprendizaje por transferencia (Sección~2.4) permite inicializar estos modelos con pesos pre-entrenados en ImageNet, aprovechando características de bajo y medio nivel (bordes, texturas, estructuras geométricas) que son transferibles al dominio de radiografías de tórax a pesar de la brecha substancial entre imágenes naturales RGB e imágenes médicas de canal único. El ajuste fino (\textit{fine-tuning}) con tasas de aprendizaje diferenciales adapta las capas profundas al dominio médico mientras preserva las representaciones genéricas en capas tempranas. La arquitectura de salida mediante regresión directa de coordenadas (Sección~2.6) predice un vector compacto de 30 valores continuos ($2 \times 15$ \textit{landmarks}) mediante capas completamente conectadas aplicadas sobre \textit{global average pooling}, proporcionando eficiencia de memoria, precisión sub-píxel inherente, y compatibilidad directa con arquitecturas ResNet pre-entrenadas. La función de pérdida compuesta propuesta en la Sección~2.5 integra tres componentes complementarios: \textit{Wing Loss} para amplificación de gradientes en el régimen de errores pequeños clínicamente relevantes ($|x| < 10$ píxeles), restricciones de simetría bilateral que penalizan desviaciones entre los 7 pares de \textit{landmarks} simétricos identificados en la Tabla~2.1.1, y preservación de distancias anatómicas que regulariza las distancias inter-\textit{landmark} hacia valores de referencia consistentes con la anatomía torácica normal. Esta combinación explota directamente el conocimiento anatómico específico establecido en la Sección~2.1 sin incrementar la complejidad arquitectónica, proporcionando supervisión adicional que guía el aprendizaje hacia configuraciones de \textit{landmarks} anatómicamente plausibles.

El análisis del estado del arte presentado en la Sección~2.7 evidencia que, si bien métodos basados en aprendizaje profundo han alcanzado precisión sub-milimétrica en aplicaciones cefalométricas controladas (Oh 2020: 1.18 mm, Ma \& Luo 2021: 1.29 mm) y 3-5 píxeles en radiografías de tórax con mayor variabilidad (Li 2023: 4.22 píxeles, Cheng 2023: 3.78 píxeles), existen gaps significativos en la literatura: ningún trabajo reportado combina simultáneamente \textit{Wing Loss}, restricciones de simetría bilateral, y preservación de distancias anatómicas para detección de \textit{landmarks} en radiografías de tórax. La mayoría de métodos utiliza MSE estándar o una única restricción geométrica (Song 2020: solo simetría, Payer 2019: solo configuración espacial), y los estudios de ablación cuantitativos que descomponen las contribuciones individuales de cada componente de función de pérdida son limitados (solo 11 de 21 trabajos revisados). Adicionalmente, las restricciones de simetría bilateral y preservación de distancias anatómicas son infrautilizadas en radiografías de tórax a pesar de ser propiedades anatómicas fundamentales que pueden explotarse sin requerir anotaciones adicionales. Estos gaps motivan la investigación presentada en este trabajo: la integración de conocimiento anatómico específico del tórax mediante una función de pérdida multi-componente con una arquitectura eficiente de regresión de coordenadas.

Las contribuciones del marco teórico desarrollado en este capítulo son múltiples. Primero, se ha proporcionado una fundamentación matemática rigurosa de cada componente metodológico, incluyendo derivaciones completas de la retropropagación de gradientes (Ecuaciones~2.8-2.12), el optimizador Adam (Ecuaciones~2.16-2.18), los bloques residuales (Ecuaciones~2.19-2.22), la función \textit{Wing Loss} con análisis de gradientes (Ecuaciones~2.24-2.26), las restricciones de simetría y preservación de distancias (Ecuaciones~2.27-2.30), y las formulaciones de regresión de coordenadas versus mapas de calor (Ecuaciones~2.32-2.38). Esta fundamentación establece precisamente qué propiedades matemáticas de cada componente son relevantes para la tarea de detección de \textit{landmarks} y cómo interactúan durante el proceso de optimización. Segundo, se ha presentado un análisis exhaustivo del estado del arte mediante una tabla comparativa de 21 trabajos representativos publicados en \textit{venues} de alto impacto (IEEE Transactions on Medical Imaging, Medical Image Analysis, CVPR/ICCV, MICCAI) entre 2016 y 2024, categorizados según enfoque metodológico, con identificación explícita de tendencias temporales, divergencias metodológicas entre dominios, y gaps específicos en la literatura. Tercero, se ha justificado técnicamente cada decisión de diseño mediante análisis de ventajas, limitaciones, y trade-offs: la selección de ResNet-18 sobre arquitecturas más profundas se justifica por eficiencia de parámetros y menor propensión al sobreajuste en conjuntos de datos moderados; la preferencia por regresión de coordenadas sobre mapas de calor se fundamenta en eficiencia computacional, precisión sub-píxel inherente, y restricciones de hardware; la combinación específica de componentes de función de pérdida se motiva por los gaps identificados en el estado del arte y las propiedades anatómicas específicas del tórax.

El marco teórico establecido proporciona todos los fundamentos conceptuales, matemáticos y contextuales necesarios para proceder a la descripción de la metodología experimental. El Capítulo 3 presenta la implementación concreta de los conceptos teóricos desarrollados en este capítulo: la descripción del conjunto de datos específico utilizado incluyendo procedimientos de adquisición, anotación, y preprocesamiento; la arquitectura de red neuronal implementada con detalles de todas las capas, dimensiones de tensores, y funciones de activación; el protocolo experimental completo incluyendo partición de datos, hiperparámetros de entrenamiento, y estrategias de aumentación de datos; las métricas de evaluación cuantitativas para medir el desempeño de localización; y crucialmente, estudios de ablación sistemáticos que cuantifican el impacto individual de cada componente de la función de pérdida compuesta (\textit{Wing Loss} aislado, \textit{Wing Loss} + simetría, \textit{Wing Loss} + distancias, y la combinación completa) para validar empíricamente las hipótesis establecidas en el marco teórico. El Capítulo 4 presentará los resultados experimentales completos, comparaciones cuantitativas con el estado del arte revisado en la Sección~2.7, visualizaciones de predicciones con análisis de casos exitosos y errores, y discusión de las implicaciones de los hallazgos experimentales en el contexto del marco teórico desarrollado.

En síntesis, el Capítulo 2 ha construido una fundamentación teórica sólida, matemáticamente rigurosa, y contextualizada en el estado del arte contemporáneo, que establece las bases para la metodología experimental que se presenta a continuación. Los conceptos de redes neuronales convolucionales profundas, arquitecturas residuales, aprendizaje por transferencia, funciones de pérdida especializadas con restricciones geométricas, y regresión de coordenadas eficiente han sido desarrollados sistemáticamente con nivel de detalle apropiado para una tesis de maestría en ingeniería electrónica, incluyendo derivaciones matemáticas completas, análisis de propiedades relevantes, y conexiones explícitas entre componentes. El marco teórico está preparado para guiar la implementación metodológica y la interpretación de resultados experimentales que constituyen los capítulos subsecuentes.


% Contenido clásico respaldado (GPA/PCA/AAM) - COMENTADO
%\chapter{Marco Teórico y Antecedentes}
\label{cap:marco_teorico}

\section{Introducción al Marco Teórico}
% Omitimos una introducción general al marco teórico si esto es solo un capítulo de una tesis
% y vamos directo a las secciones. Si fuera un artículo, aquí iría la introducción al problema.

Este capítulo establece los fundamentos teóricos que sustentan la metodología de investigación para la localización automatizada de puntos de referencia anatómicos en imágenes radiográficas de tórax. Se exploran los conceptos de detección de `landmarks`, las técnicas de alineación de formas, los modelos de apariencia basados en componentes principales y el paradigma del aprendizaje supervisado, contextualizando la elección de estos métodos dentro del estado del arte y reconociendo sus limitaciones.

\section{Detección de Puntos de Referencia Anatómicos en Imágenes Médicas}
\label{sec:landmark_detection_medical}

La identificación precisa y automática de puntos de referencia anatómicos (landmarks) es un componente crucial en el análisis de imágenes médicas. Estos puntos, que representan localizaciones distintivas y consistentes en las estructuras anatómicas, sirven como base para una amplia gama de aplicaciones clínicas y de investigación \cite{Bishop2006}. Su correcta localización es fundamental para el registro de imágenes multimodales o temporales, la segmentación de órganos, la planificación quirúrgica asistida por computador, la evaluación de la morfología y la cuantificación de cambios patológicos \cite{Szeliski2011}.

En el dominio específico de las radiografías de tórax (CXR), la detección de landmarks como los ápices pulmonares, los ángulos costofrénicos o puntos clave de la silueta cardíaca es esencial para derivar mediciones clínicamente relevantes, evaluar la simetría, detectar anomalías y asistir en el diagnóstico de enfermedades cardiorrespiratorias \cite{Li2021MedicalIA}. Dada la variabilidad inherente en la anatomía humana, la calidad de la imagen (a menudo afectada por bajo contraste y ruido en CXR) y las posibles deformaciones debidas a patologías, la detección automática de landmarks presenta desafíos significativos. Tradicionalmente, esta tarea ha recaído en la interpretación manual por expertos, un proceso laborioso, subjetivo y susceptible a la variabilidad interobservador. Por ello, el desarrollo de sistemas computacionales robustos y precisos para esta tarea es de gran interés.

\section{Aprendizaje Supervisado para la Localización de Landmarks}
\label{sec:supervised_learning}

La estrategia central de esta investigación se enmarca dentro del paradigma del aprendizaje supervisado. En este enfoque, un modelo algorítmico aprende a mapear entradas (en este caso, imágenes radiográficas) a salidas deseadas (las coordenadas de los landmarks) a partir de un conjunto de datos de entrenamiento previamente etiquetado \cite{Bishop2006, Duda2000Pattern}. La disponibilidad de mil radiografías con anotaciones manuales de landmarks, como se describe en la Sección \ref{sec:vis_general} de la metodología de esta tesis, permite entrenar los componentes del sistema para que reconozcan los patrones visuales y espaciales asociados a cada landmark. El objetivo es que el sistema generalice este aprendizaje para localizar con precisión los landmarks en imágenes no vistas durante el entrenamiento.

\section{Preprocesamiento de Datos y Alineación de Formas}
\label{sec:preprocessing_alignment}

La variabilidad en los datos de imagen puede provenir de múltiples fuentes. El preprocesamiento busca reducir la variabilidad extrínseca para enfocar el análisis en las características intrínsecas de interés.

\subsection{Normalización y Escalamiento de Imágenes}
Las imágenes médicas pueden variar en resolución y dimensiones. La normalización del tamaño de las imágenes a una cuadrícula estándar (e.g., $64 \times 64$ píxeles) es un paso fundamental. Esta normalización asegura que las características extraídas y las dimensiones de los parches de análisis sean consistentes entre imágenes, facilitando la comparabilidad y el aprendizaje del modelo \cite{Gonzalez2018Digital}. La transformación y acotamiento de las coordenadas de los landmarks garantizan su validez dentro de los límites de la imagen reescalada.

\subsection{Análisis Generalizado de Procrustes (GPA) para la Alineación de Formas}
Los conjuntos de landmarks que definen una forma anatómica están sujetos a variaciones globales de traslación, rotación y escala, que no reflejan la variabilidad intrínseca de la forma en sí, sino más bien diferencias de pose o adquisición. El Análisis Generalizado de Procrustes (GPA) es una técnica estadística diseñada para superponer óptimamente múltiples configuraciones de landmarks, eliminando estas variaciones \cite{Gower1975Generalized, Dryden2016Statistical, Goodall1991Procrustes}.

\subsubsection{Fundamentos Matemáticos del GPA}
\label{sssec:gpa_math}
Dado un conjunto de $N$ configuraciones de $k$ puntos (landmarks) en $D$ dimensiones, donde cada configuración $i$ se representa por una matriz $\mathbf{S}_i \in \mathbb{R}^{k \times D}$, el GPA busca una transformación óptima (traslación, rotación y escalamiento uniforme) para cada forma, de modo que se minimice una medida de la suma de cuadrados de las distancias entre las formas y una forma media $\mathbf{M}$ que se estima iterativamente \cite{Goodall1991Procrustes}.

El algoritmo GPA, como se describe en la metodología de esta tesis, sigue estos pasos matemáticos:
\begin{enumerate}[label=\arabic*., itemsep=5pt] % Aumenta un poco el espacio entre ítems principales
    \item \textbf{Preprocesamiento de Formas Individuales (por cada forma $\matS_i$):}
    \begin{itemize}[itemsep=2pt, topsep=2pt] % Ajusta espacio en sub-listas
        \item \textbf{Centrado:} Se calcula el centroide $\overline{\vecs}_i$:
              \begin{equation}
                  \overline{\vecs}_i = \frac{1}{k} \sum_{j=1}^{k} \vecp_{ij}
                  \label{eq:centroide}
              \end{equation}
              donde $\vecp_{ij}$ representa el vector de coordenadas del $j$-ésimo landmark (punto de referencia) de la $i$-ésima configuración. Cada $\vecp_{ij}$ es un vector fila de $D$ dimensiones; Con $D=2$, $\vecp_{ij} = (x_{ij}, y_{ij})$ son las coordenadas 2D del $j$-ésimo landmark de la forma $i$. El centroide $\overline{\vecs}_i$ es, por lo tanto, un vector de $D$ dimensiones que representa la posición media de todos los landmarks de la forma $i$.
              La forma se centra mediante:
              \begin{equation}
                  \matS'_i = \matS_i - \vecuno\overline{\vecs}_i^T
                  \label{eq:centrado}
              \end{equation}
              donde $\vecuno$ es un vector columna de $k$ unos. Esta operación resta el vector centroide $\overline{\vecs}_i$ de cada fila (landmark) de $\matS_i$, trasladando el origen de coordenadas de la forma a su centroide. Esto elimina la variabilidad por traslación \cite{Dryden2016Statistical}.

        \item \textbf{Normalización de Escala:} La escala de cada forma centrada $\matS'_i$ se normaliza dividiéndola por su Tamaño Centroide (Centroid Size), $CS_i$. El Tamaño Centroide es la Norma de Frobenius de la matriz de forma centrada $\matS'_i$.
              La Norma de Frobenius de una matriz $\mathbf{X} \in \mathbb{R}^{m \times n}$, denotada como $\normF{\mathbf{X}}$, se define como la raíz cuadrada de la suma de los cuadrados de sus elementos:
              \begin{equation}
                  \normF{\mathbf{X}} = \sqrt{\sum_{l=1}^{m} \sum_{p=1}^{n} x_{lp}^2}
                  \label{eq:frobenius_def}
              \end{equation}
              En nuestro caso, para la matriz de forma centrada $\matS'_i \in \mathbb{R}^{k \times D}$, donde cada fila $\vecp'_{ij}$ (el $j$-ésimo landmark de la forma $i$ ya centrada) es un vector de $D$ coordenadas, su Tamaño Centroide $CS_i$ se calcula como:
              \begin{equation}
                  CS_i = \normF{\matS'_i} = \sqrt{\sum_{j=1}^{k} \normF{\vecp'_{ij}}^2} = \sqrt{\sum_{j=1}^{k} \sum_{d=1}^{D} (p'_{ijd})^2}
                  \label{eq:centroid_size}
              \end{equation}
              donde $\normF{\vecp'_{ij}}^2$ es el cuadrado de la norma euclidiana del vector $\vecp'_{ij}$, y $p'_{ijd}$ es la $d$-ésima coordenada del $j$-ésimo landmark centrado de la forma $i$.
              Así, la forma normalizada $\matS''_i$ es:
              \begin{equation}
                  \matS''_i = \frac{\matS'_i}{CS_i}
                  \label{eq:normalizacion_escala}
              \end{equation}
              Esto asegura que todas las formas tengan una escala comparable (Tamaño Centroide unitario) \cite{Dryden2016Statistical}.
    \end{itemize}

    \item \textbf{Inicialización de la Forma Media:} Se selecciona arbitrariamente una de las formas preprocesadas (e.g., $\matS''_1$) como la estimación inicial de la forma media, $\matM^{(0)} \in \mathbb{R}^{k \times D}$ \cite{Gower1975Generalized}.

    \item \textbf{Proceso Iterativo de Alineamiento (hasta convergencia):} Para la iteración $t$:
    \begin{itemize}[itemsep=2pt, topsep=2pt]
        \item \textbf{Alineamiento a la Forma Media Actual:} Cada forma preprocesada $\matS''_i \in \mathbb{R}^{k \times D}$ se alinea a la forma media actual $\matM^{(t)} \in \mathbb{R}^{k \times D}$ encontrando la matriz de rotación óptima $\matR_i^{(t)} \in \mathbb{R}^{D \times D}$ que minimiza la suma de cuadrados de las distancias euclidianas entre los landmarks correspondientes de la forma rotada y la forma media. Esto es equivalente a minimizar el cuadrado de la Norma de Frobenius de la diferencia:
              \begin{equation}
                   \min_{\matR_i^{(t)}} \normF{\matS''_i \matR_i^{(t)} - \matM^{(t)}}^2
                   \label{eq:minimizacion_procrustes}
              \end{equation}
              sujeto a la restricción de que $\matR_i^{(t)}$ sea una matriz ortogonal (es decir, una rotación y posiblemente una reflexión):
              \begin{equation}
                  (\matR_i^{(t)})^T \matR_i^{(t)} = \matR_i^{(t)} (\matR_i^{(t)})^T = \matI_D
                  \label{eq:restriccion_ortogonalidad}
              \end{equation}
              donde $\matI_D$ es la matriz identidad de $D \times D$.
              La solución a este problema de minimización (conocido como problema de Procrustes ortogonal) se obtiene mediante la Descomposición en Valores Singulares (SVD). Se forma la matriz $\matC_i = (\matS''_i)^T \matM^{(t)}$, que es una matriz de $D \times D$. La SVD de una matriz genérica $\mathbf{A} \in \mathbb{R}^{D \times D}$ es una factorización de la forma:
              \begin{equation}
                  \mathbf{A} = \mathbf{U} \mathbf{\Sigma} \mathbf{V}^T
                  \label{eq:svd_def_general}
              \end{equation}
              donde:
              \begin{itemize}[itemsep=1pt, topsep=1pt, leftmargin=*]
                  \item $\mathbf{U} \in \mathbb{R}^{D \times D}$ es una matriz ortogonal cuyas columnas son los vectores singulares izquierdos de $\mathbf{A}$.
                  \item $\mathbf{\Sigma} \in \mathbb{R}^{D \times D}$ es una matriz diagonal que contiene los valores singulares no negativos $\sigma_1 \ge \sigma_2 \ge \dots \ge \sigma_D \ge 0$ de $\mathbf{A}$ en su diagonal principal.
                  \item $\mathbf{V} \in \mathbb{R}^{D \times D}$ es una matriz ortogonal cuyas columnas son los vectores singulares derechos de $\mathbf{A}$, y $\mathbf{V}^T$ es su transpuesta.
              \end{itemize}
              Aplicando esto a $\matC_i$, su SVD es $\matC_i = \matU_i \matsigma_i \matV_i^T$. La matriz de rotación óptima $\matR_i^{(t)}$ se calcula como:
              \begin{equation}
                  \matR_i^{(t)} = \matV_i \matU_i^T
                  \label{eq:solucion_rotacion}
              \end{equation}
              Para asegurar que $\matR_i^{(t)}$ sea una rotación pura (y no una reflexión), se puede ajustar si $\det(\matR_i^{(t)}) = -1$. Esto se hace multiplicando la columna de $\matV_i$ (o $\matU_i$) correspondiente al valor singular más pequeño por $-1$ antes de calcular $\matR_i^{(t)}$ \cite{Goodall1991Procrustes}. Las formas alineadas son:
              \begin{equation}
                  \tilde{\matS}''_i = \matS''_i \matR_i^{(t)}
                  \label{eq:formas_alineadas}
              \end{equation}

        \item \textbf{Actualización de la Forma Media:} Se calcula una nueva forma media promediando las coordenadas de los landmarks correspondientes de todas las formas alineadas $\tilde{\matS}''_i$:
              \begin{equation}
                  \matM_{\text{raw}}^{(t+1)} = \frac{1}{N} \sum_{i=1}^N \tilde{\matS}''_i
                  \label{eq:media_raw_update}
              \end{equation}

        \item \textbf{Normalización de la Nueva Forma Media:} La forma media cruda $\matM_{\text{raw}}^{(t+1)}$ se centra y normaliza en escala (aplicando los mismos procedimientos descritos en el paso 1, ver ecuaciones \eqref{eq:centrado} y \eqref{eq:normalizacion_escala}) para obtener la forma media actualizada $\matM^{(t+1)}$.
    \end{itemize}

    \item \textbf{Criterio de Convergencia:} El proceso iterativo concluye cuando la diferencia entre la forma media de la iteración actual y la anterior es suficientemente pequeña, medida por la Norma de Frobenius:
          \begin{equation}
              \normF{\matM^{(t+1)} - \matM^{(t)}} < \varepsilon
              \label{eq:convergencia}
          \end{equation}
          donde $\varepsilon$ es una tolerancia pequeña predefinida (e.g., $10^{-10}$) \cite{Gower1975Generalized}.
\end{enumerate}

Las formas resultantes $$\tilde{\mathbf{S}}''_i$$ están alineadas en un espacio común, denominado espacio de Procrustes. Los resultados preliminares de esta tesis (Capítulo \ref{cap:resultados_discusion}) demuestran experimentalmente que sin este paso, la variabilidad por traslaciones, rotaciones y escalas anómalas incrementa significativamente el error de predicción, subrayando la importancia crítica del GPA.

\section{Modelado de la Apariencia Local con Análisis de Componentes Principales (PCA)}
\label{sec:pca_appearance_modeling}

Una vez que las formas están alineadas, se modela la variabilidad en la apariencia local alrededor de cada landmark.

\subsection{Extracción de Parches y Definición de Regiones de Búsqueda}
Como se describe en la metodología, se definen regiones de búsqueda para cada landmark basadas en la distribución de sus coordenadas en el conjunto de entrenamiento. Los parches de imagen (subimágenes) extraídos alrededor de la posición de cada landmark capturan la información visual local.

\subsection{Análisis de Componentes Principales (PCA) para Modelos de Apariencia (Eigenpatches)}
El Análisis de Componentes Principales (PCA) es una técnica estadística fundamental para la reducción de dimensionalidad y el análisis de la varianza en datos multivariados \cite{Jolliffe2002Principal, Bishop2006, Shouno2022PCA}. Aplicado a los parches de imagen vectorizados, PCA permite construir un modelo compacto de la apariencia. El concepto es similar al de ``eigenfaces'' para el reconocimiento facial \cite{Turk1991Eigenfaces}, donde aquí se generan ``eigenpatches''.

\subsubsection{Fundamentos Matemáticos del PCA}
\label{sssec:pca_math}
Dado un conjunto de $N$ parches vectorizados $\{\mathbf{x}_{ij}\}_{i=1}^N$ para un landmark $j$, donde cada $\mathbf{x}_{ij} \in \mathbb{R}^{D_p}$ ($D_p$ es el número de píxeles en el parche), el proceso es:
\begin{enumerate}
    \item \textbf{Cálculo de la Media Muestral:} Se calcula el parche promedio $\overline{\mathbf{x}}_j$:
    \begin{equation}
        \overline{\mathbf{x}}_j = \frac{1}{N} \sum_{i=1}^N \mathbf{x}_{ij}
        \label{eq:mean_patch}
    \end{equation}
    %\cite{Jolliffe2002Principal}.

    \item \textbf{Centrado de los Datos:} Cada parche $\mathbf{x}_{ij}$ se centra para obtener $\mathbf{z}_{ij}$:
    \begin{equation}
        \mathbf{z}_{ij} = \mathbf{x}_{ij} - \overline{\mathbf{x}}_j
        \label{eq:centered_data}
    \end{equation}
    Estos forman la matriz de datos centrados $\mathbf{Z}_j \in \mathbb{R}^{N \times D_p}$.

    \item \textbf{Cálculo de la Matriz de Covarianza:} La matriz de covarianza $\mathbf{C}_j$ se calcula como:
    \begin{equation}
        \mathbf{C}_j = \frac{1}{N-1} \mathbf{Z}_j^T \mathbf{Z}_j = \frac{1}{N-1} \sum_{i=1}^N \mathbf{z}_{ij} \mathbf{z}_{ij}^T
        \label{eq:covariance_matrix}
    \end{equation}
    %\cite{Bishop2006}.

    \item \textbf{Resolución del Problema de Eigenvalores/Eigenvectores:} Se calculan los eigenvalores $\lambda_{jk}$ y los eigenvectores $\mathbf{v}_{jk}$ de $\mathbf{C}_j$ resolviendo:
    \begin{equation}
        \mathbf{C}_j\mathbf{v}_{jk} = \lambda_{jk} \mathbf{v}_{jk}
        \label{eq:eigen_problem}
    \end{equation}
    %\cite{Jolliffe2002Principal}.

    \item \textbf{Selección de Componentes Principales:} Los eigenvectores se ordenan por eigenvalores descendentes. Se seleccionan los primeros $m_j$ eigenvectores, que forman las columnas de la matriz $\mathbf{V}_j$:
    \begin{equation}
        \mathbf{V}_j = [\mathbf{v}_{j1}, \dots, \mathbf{v}_{jm_j}] \in \mathbb{R}^{D_p \times m_j}
        \label{eq:principal_components_matrix}
    \end{equation}
    Estos se eligen para que expliquen un porcentaje deseado de la varianza total (e.g., 95\% en esta tesis, ver Capítulo \ref{cap:diseno_experimental}), donde la varianza explicada por $m_j$ componentes es:
    \begin{equation}
        \frac{\sum_{l=1}^{m_j} \lambda_{jl}}{\sum_{l=1}^{D_p} \lambda_{jl}}
        \label{eq:explained_variance}
    \end{equation}
    %\cite{Shouno2022PCA}. 
    Los resultados de esta tesis (Capítulo \ref{cap:resultados_discusion}) confirman que un número reducido de componentes (10-30) captura la mayor parte de la varianza.

    \item \textbf{Proyección al Subespacio de PCA:} Un nuevo parche (centrado) $\mathbf{z}$ se proyecta al subespacio PCA para obtener los coeficientes $\boldsymbol{\omega}$:
    \begin{equation}
        \boldsymbol{\omega} = \mathbf{V}_j^T \mathbf{z}
        \label{eq:pca_projection}
    \end{equation}

    \item \textbf{Reconstrucción desde el Subespacio PCA:} El parche se reconstruye a partir de sus componentes principales como $\hat{\mathbf{x}}$:
    \begin{equation}
        \hat{\mathbf{x}} = \mathbf{V}_j\boldsymbol{\omega} + \overline{\mathbf{x}}_j
        \label{eq:pca_reconstruction}
    \end{equation}
    El error de reconstrucción, $E_{L2}(\mathbf{x})$, se define como la norma $L_2$ de la diferencia entre el parche original y el reconstruido:
    \begin{equation}
        E_{L2}(\mathbf{x}) = \|\mathbf{x} - \hat{\mathbf{x}}\|_2
        \label{eq:reconstruction_error}
    \end{equation}
    y se utiliza para la predicción \cite{Martinez2001PCAvsLDA}.
\end{enumerate}

\section{Proceso de Búsqueda y Coincidencia para la Predicción}
\label{sec:search_matching}
La localización de un landmark $j$ en una nueva imagen de prueba se realiza mediante una búsqueda exhaustiva dentro de la región de búsqueda $\mathcal{R}_j$ predefinida (ver Sección \ref{sec:extraccion_region}). Para cada ubicación candidata $(y_c, x_c)$ dentro de $\mathcal{R}_j$, se extrae un parche $\mathbf{P}_c$. Este parche se procesa como se describe en la Sección \ref{sssec:pca_math} para calcular el error de reconstrucción $E_{L2}(\mathbf{x}_c)$. La posición $(y_c, x_c)$ que minimiza este error se selecciona como la ubicación predicha $\hat{\mathbf{p}}_j$.

\section{Estado del Arte y Enfoques Contemporáneos}
\label{sec:state_of_the_art}

El enfoque metodológico presentado, combinando GPA y PCA, se alinea con los principios de los Modelos Activos de Forma (ASM) \cite{Cootes1995ActiveShape} y los Modelos Activos de Apariencia (AAM) \cite{Cootes2001ActiveAppearance}. Estos han sido métodos canónicos efectivos.

No obstante, el campo ha avanzado, especialmente con el aprendizaje profundo:
\begin{itemize}
    \item \textbf{Métodos basados en Regresión Directa con Aprendizaje Profundo:} Las Redes Neuronales Convolucionales (CNNs) pueden regresar directamente las coordenadas de los landmarks \cite{Li2021MedicalIA}.
    \item \textbf{Métodos basados en Mapas de Calor (Heatmap Regression):} Un enfoque popular con CNNs es predecir mapas de calor, donde la intensidad indica la probabilidad del landmark. El máximo en el mapa de calor localiza el landmark \cite{Wang2022ContextAwareMICCAI}.
    \item \textbf{Arquitecturas Avanzadas:} Redes como U-Net y sus variantes, a menudo con mecanismos de atención, se utilizan para capturar contexto local y global, mejorando la precisión en CXR \cite{Chen2023SelfSupervisedTMI}.
\end{itemize}
Si bien los métodos de aprendizaje profundo a menudo logran un rendimiento de vanguardia \cite{Li2021MedicalIA}, generalmente requieren grandes conjuntos de datos anotados y recursos computacionales. El enfoque clásico propuesto en esta tesis, aunque potencialmente superado en precisión, ofrece ventajas en interpretabilidad y menor demanda de datos, como se discute en las Conclusiones (Capítulo \ref{cap:conclusiones_trabajo_futuro}).

\section{Limitaciones Potenciales del Enfoque Metodológico}
\label{sec:limitations}

Las limitaciones, detalladas en el Capítulo \ref{cap:conclusiones_trabajo_futuro} de esta tesis, incluyen la linealidad de PCA, la sensibilidad de las regiones de búsqueda, la dependencia de la calidad del entrenamiento, el manejo de oclusiones severas, y aspectos del etiquetado. La linealidad de PCA es una limitación conocida; mientras PCA captura la varianza principal de forma óptima bajo supuestos lineales, las variaciones complejas y no lineales en la apariencia pueden no ser modeladas completamente \cite{Bishop2006, Jolliffe2002Principal}. Las propuestas de Trabajo Futuro (Capítulo \ref{cap:conclusiones_trabajo_futuro}), como la exploración de AAMs o CNNs, buscan abordar algunas de estas limitaciones. Por ejemplo, los AAMs \cite{Cootes2001ActiveAppearance} modelan conjuntamente forma y apariencia, lo que puede ofrecer mayor robustez, y las CNNs \cite{Chen2023SelfSupervisedTMI, Wang2022ContextAwareMICCAI} pueden aprender características no lineales más potentes.
%\chapter{Marco Teórico y Antecedentes}
\label{cap:marco_teorico}

Este capítulo establece el contexto científico y técnico en el que se enmarca la presente tesis. Se revisan los conceptos fundamentales y los trabajos previos más relevantes en las áreas de detección de patologías pulmonares mediante radiografías de tórax, segmentación y normalización de la región pulmonar, extracción de características y clasificación mediante aprendizaje automático. El objetivo es proporcionar una base sólida para comprender la motivación, el diseño y la contribución de la metodología MaShDL-CNN Hybrid propuesta.

\subsection{Métodos Tradicionales y Basados en Aprendizaje Profundo}
\label{ssec:metodos_tradicionales_dl}

Históricamente, los esfuerzos para desarrollar sistemas de Diagnóstico Asistido por Computadora (CADx) para CXR se basaron en técnicas de procesamiento de imágenes tradicional y reconocimiento de patrones. Estos sistemas a menudo implicaban una segmentación inicial de la región pulmonar, seguida de la extracción de características diseñadas manualmente (e.g., descriptores de textura, forma, intensidad) y, finalmente, la clasificación mediante algoritmos de aprendizaje automático convencionales como Máquinas de Soporte Vectorial (SVM), árboles de decisión o redes neuronales simples \cite{suzuki2017overview, ginneken2001computer}. Si bien estos enfoques lograron ciertos éxitos, su rendimiento a menudo estaba limitado por la robustez de la segmentación y la capacidad de las características diseñadas manualmente para capturar la compleja variabilidad de los patrones patológicos.

La última década ha sido testigo de una revolución en el campo del análisis de imágenes médicas, impulsada en gran medida por los avances en Aprendizaje Profundo (Deep Learning, DL), y en particular, por las Redes Neuronales Convolucionales (CNN) \cite{lecun2015deep, litjens2017survey}. Las CNNs han demostrado una capacidad sobresaliente para aprender jerarquías de características directamente a partir de los datos de imagen, eliminando la necesidad de una ingeniería de características manual y, a menudo, superando el rendimiento de los métodos tradicionales en diversas tareas, incluyendo la detección y clasificación de enfermedades en CXR \cite{rajpurkar2017chexnet, wang2020covid, ozturk2020automated, shaik2023comprehensive, ciompi2023towards}. Arquitecturas como AlexNet \cite{krizhevsky2012imagenet}, VGG \cite{simonyan2014very}, ResNet \cite{he2016deep}, DenseNet \cite{huang2017densely}, e Inception \cite{szegedy2015going}, así como modelos más recientes y específicos para tareas médicas (e.g., U-Net para segmentación \cite{ronnerberger2015unet}), han sido adaptadas o utilizadas como base para el análisis de CXR.

% (Sugerencia: Tabla \ref{tab:comparativa_metodos_cxr}: Una tabla comparativa que resuma las ventajas y desventajas de los métodos tradicionales vs. los basados en aprendizaje profundo para el análisis de CXR. Columnas: Característica (e.g., Extracción de Features, Dependencia de Datos, Interpretabilidad, Rendimiento Típico), Métodos Tradicionales, Métodos de Aprendizaje Profundo.)

\subsection{Desafíos Existentes en el Análisis Automatizado de CXR}
\label{ssec:desafios_cxr}

A pesar del éxito de los modelos de DL, persisten varios desafíos en el análisis automatizado de CXR:
\begin{itemize}
\item \textbf{Variabilidad de los Datos y Calidad de Imagen:} Las CXR pueden variar significativamente en términos de contraste, ruido, resolución y artefactos, dependiendo del equipo, el protocolo de adquisición y la configuración \cite{zech2018variable}.
\item \textbf{Ambigüedad y Superposición Anatómica:} La naturaleza 2D de las CXR implica la superposición de múltiples estructuras 3D (costillas, corazón, diafragma, vasos pulmonares), lo que puede dificultar la identificación de patologías sutiles \cite{borghesi2020radiographic}.
\item \textbf{Variabilidad Inter-Paciente:} Las diferencias en la anatomía, la edad, el sexo, la constitución física y el grado de inspiración del paciente introducen una gran variabilidad geométrica en la apariencia de la región pulmonar.
\item \textbf{Disponibilidad y Anotación de Datos:} Aunque existen grandes conjuntos de datos públicos de CXR (e.g., CheXpert \cite{irvin2019chexpert}, MIMIC-CXR \cite{johnson2019mimic}, PadChest \cite{bustos2020padchest}, COVID-19 Image Data Collection \cite{cohen2020covid}), la obtención de anotaciones precisas y consistentes (e.g., segmentaciones, etiquetas de patologías múltiples) a gran escala sigue siendo un desafío costoso y laborioso \cite{park2020methodologic}.
\item \textbf{Desequilibrio de Clases:} Ciertas patologías son mucho menos frecuentes que otras, lo que puede llevar a modelos sesgados si no se maneja adecuadamente durante el entrenamiento \cite{johnson2019mimic}.
\item \textbf{Interpretabilidad y Confianza:} Los modelos de DL, a menudo considerados "cajas negras", necesitan mejorar en términos de explicabilidad para ganar la confianza de los clínicos y facilitar su adopción \cite{holzinger2019causability, anwar2022explainable}.
\end{itemize}
La normalización y alineación de la forma pulmonar, tema central de esta tesis, busca abordar directamente el desafío de la variabilidad geométrica inter-paciente y de adquisición.


%\section{Segmentación y Normalización de la Región Pulmonar}
\label{sec:segmentacion_normalizacion}

La segmentación precisa de la región pulmonar es a menudo un paso precursor esencial en los sistemas CADx para CXR. Al aislar los pulmones del resto de la imagen, se puede enfocar el análisis en la región de interés, reducir la influencia de estructuras irrelevantes y facilitar la extracción de características específicas del pulmón \cite{van2006segmentation_cxr, mansoor2015segmentation}. La normalización, por otro lado, busca estandarizar ciertas propiedades de la región segmentada, como su forma, tamaño u orientación, para permitir comparaciones más justas entre diferentes imágenes o pacientes.

\subsection{Técnicas Comunes de Segmentación Pulmonar en CXR}
\label{ssec:tecnicas_segmentacion_cxr}
Se han propuesto numerosas técnicas para la segmentación de pulmones en CXR, que pueden clasificarse en varias categorías:
\begin{itemize}
\item \textbf{Basadas en Umbralización:} Explotan las diferencias de intensidad entre el parénquima pulmonar (más radiolúcido) y las estructuras circundantes (e.g., costillas, corazón, más radiopacos). Métodos como la umbralización global o adaptativa de Otsu pueden ser un primer paso, pero a menudo requieren refinamientos debido a la variabilidad de contraste \cite{saad2014lung}.
\item \textbf{Basadas en Regiones:} Técnicas como el crecimiento de regiones (region growing) comienzan con puntos semilla dentro de los pulmones y expanden la región basándose en criterios de homogeneidad de intensidad o textura \cite{dai2017region_placeholder}.
\item \textbf{Basadas en Contornos Activos (Snakes y Level Sets):} Estos métodos evolucionan una curva (o superficie en 3D) para ajustarse a los bordes de los pulmones, minimizando una función de energía que considera tanto la suavidad de la curva como su adherencia a los gradientes de la imagen \cite{xu2012lung, kass1988snakes}.
\item \textbf{Basadas en Modelos Estadísticos (Forma y Apariencia):} Los Modelos de Forma Activa (ASM) y los Modelos de Apariencia Activa (AAM) utilizan conocimiento previo sobre la forma y/o apariencia promedio de los pulmones y sus variaciones para guiar la segmentación \cite{cootes1995active, cootes2001active}. El método MaShDL de esta tesis se encuadra en esta familia.
\item \textbf{Basadas en Aprendizaje Profundo:} Más recientemente, las CNNs, especialmente arquitecturas como U-Net \cite{ronnerberger2015unet} y sus variantes (e.g., Attention U-Net \cite{oktay2018attention}, UNet++ \cite{zhou2018unetplusplus}), han demostrado un rendimiento de vanguardia en la segmentación de pulmones en CXR, aprendiendo a identificar los límites pulmonares directamente de los datos \cite{ait2018lung, novikov2018fully, zhou2021review_segmentation}.
\end{itemize}
% (Sugerencia: Figura \ref{fig:ejemplos_segmentacion_cxr}: Mostrar una CXR y los resultados de segmentación pulmonar obtenidos con 2-3 métodos diferentes, por ejemplo, un método tradicional y uno basado en U-Net, y el propuesto en esta tesis si ya se tienen resultados preliminares visuales.)

\subsection{Modelos Deformables: Modelos Estadísticos de Forma (SSM)}
\label{ssec:ssm_teoria}
Los Modelos Estadísticos de Forma (SSM) son una técnica poderosa y bien establecida para modelar la variabilidad geométrica de una clase de objetos. Un SSM representa una colección de formas mediante una forma media y un conjunto de modos de variación que describen cómo las instancias individuales pueden desviarse de esta media \cite{cootes1995active, davies2008statistical}.

\subsubsection{Construcción de un SSM: Puntos Característicos (Landmarks)}
\label{sssec:landmarks_ssm}
La base de un SSM es un conjunto de entrenamiento de $N_s$ formas, donde cada forma $i$ está representada por un conjunto de $N_{\text{lmk}}$ puntos característicos (landmarks) correspondientes anatómicamente: $X_i = \{p_{i,j} \in \mathbb{R}^d\}_{j=1}^{N_{\text{lmk}}}$, donde $p_{i,j}$ es el $j$-ésimo landmark de la $i$-ésima forma en $d$ dimensiones (para CXR, $d=2$). La calidad y consistencia de la anotación de estos landmarks es crucial para la efectividad del SSM. En esta tesis, se utilizan $N_{\text{lmk}}=144$ landmarks para definir el contorno de la región pulmonar.

\subsubsection{Alineación de Formas: El Problema de Procrustes Generalizado (GPA)}
\label{sssec:gpa}
Antes de poder analizar la variación de forma intrínseca, es necesario eliminar las diferencias de pose (traslación, rotación y escala global) entre las formas del conjunto de entrenamiento. Esto se logra mediante el Análisis de Procrustes Generalizado (GPA) \cite{gower1975generalized, dryden1998statistical}. El GPA alinea iterativamente todas las formas a una forma media común (que también se actualiza en cada iteración) minimizando una medida de distancia entre las formas.

El procedimiento GPA, implementado, típicamente involucra los siguientes pasos:
\begin{itemize}
    \item \textbf{Centrado:} Cada forma $X_i$ se traslada para que su centroide coincida con el origen. Si $c_i = \frac{1}{N_{\text{lmk}}} \sum_{j=1}^{N_{\text{lmk}}} p_{i,j}$, la forma centrada es $X_i^c = X_i - \mathbf{1}c_i^T$. (Nota: $\mathbf{1}$ es un vector columna de unos de dimensión $N_{\text{lmk}}$).
    \item \textbf{Escalado:} Cada forma centrada $X_i^c$ se escala a un tamaño unitario (e.g., norma de Frobenius igual a 1). $X_i^{cs} = X_i^c / \|X_i^c\|_F$.
    \item \textbf{Estimación de la Media Inicial:} Se calcula una forma media inicial $X_{\text{ref}}$ a partir de las formas $X_i^{cs}$.
    \item \textbf{Alineación Iterativa:} 
        \begin{enumerate}
            \item Para cada forma $X_i^{cs}$, encontrar la transformación de similitud (solo rotación en este espacio pre-escalado y centrado) $R_i$ que mejor alinee $X_i^{cs}$ con la referencia actual $X_{\text{ref}}$. Esto se resuelve minimizando $\|X_{\text{ref}} - X_i^{cs} R_i\|_F^2$. La solución para $R_i$ se obtiene mediante la Descomposición en Valores Singulares (SVD) de la matriz $(X_i^{cs})^T X_{\text{ref}} = U \Lambda V^T$, donde $R_i = V U^T$. 
            \item Aplicar las rotaciones: $X_i^{\text{aligned}} = X_i^{cs} R_i$.
            \item Recalcular la forma media $X_{\text{ref}}$ a partir de todas las $X_i^{\text{aligned}}$.
            \item Normalizar la nueva $X_{\text{ref}}$ (centrarla y escalarla a tamaño unitario).
            \item Repetir hasta que la forma media $X_{\text{ref}}$ converja (i.e., el cambio entre iteraciones sea menor que una tolerancia).
        \end{enumerate}
\end{itemize}
El resultado es un conjunto de formas $\{X_i^{\text{aligned}}\}$ alineadas en un espacio de forma común, y la forma media final $X_{\text{mean}}$.

% (Sugerencia: Figura \ref{fig:gpa_ilustracion}: Una figura similar a la sugerida para el Capítulo 3, mostrando (a) un conjunto de contornos pulmonares desalineados y (b) los mismos contornos después de la aplicación de GPA, superpuestos y alineados con la forma media resultante.)

\subsubsection{Análisis de Componentes Principales (PCA) para Modelar la Variación de Forma}
\label{sssec:pca_forma}
Una vez que las $N_s$ formas están alineadas, cada forma $X_i^{\text{aligned}}$ se vectoriza en un vector columna $x_i \in \mathbb{R}^{N_{\text{lmk}} \cdot d}$. Se construye una matriz de datos $D = [x_1, x_2, \dots, x_{N_s}]$. Se aplica PCA a esta matriz:
\begin{itemize}
    \item Se calcula el vector de forma media global $\bar{x} = \frac{1}{N_s} \sum_{i=1}^{N_s} x_i$.
    \item Se calcula la matriz de covarianza de los datos centrados $(x_i - \bar{x})$: $S = \frac{1}{N_s-1} \sum_{i=1}^{N_s} (x_i - \bar{x})(x_i - \bar{x})^T$.
    \item Se encuentran los eigenvectores $\phi_k$ y eigenvalores $\lambda_k$ de $S$, donde $S\phi_k = \lambda_k \phi_k$. Los eigenvectores (modos de variación) se ordenan según sus eigenvalores decrecientes.
    \item Se seleccionan los primeros $m$ eigenvectores, $P = [\phi_1, \phi_2, \dots, \phi_m]$, que capturan un porcentaje suficiente de la varianza total (e.g., 95-99\%).
\end{itemize}
Cualquier forma $x$ en el conjunto (o una nueva instancia) puede ser aproximada por el modelo lineal:
$x \approx \bar{x} + Pb$
donde $b = [b_1, b_2, \dots, b_m]^T$ es un vector de parámetros de forma (o coeficientes de modo). Cada $b_k$ controla la magnitud de la variación a lo largo del $k$-ésimo modo $\phi_k$. Estos coeficientes suelen estar restringidos, por ejemplo, $b_k \in [-3\sqrt{\lambda_k}, +3\sqrt{\lambda_k}]$.
La implementación de esta etapa almacenan $\bar{x}$, $P$, y las desviaciones estándar de los eigenvalores ($\sqrt{\lambda_k}$) respectivamente.

% (Sugerencia: Figura \ref{fig:modos_variacion_pulmon}: Similar a la sugerida para Cap. 3, mostrando la forma media pulmonar y la variación inducida por los primeros 2-3 modos b1​,b2​,b3​ (e.g., forma media ±2σk​ϕk​).)
% (Sugerencia: Tabla \ref{tab:varianza_explicada_pca_ssm}: Tabla mostrando el número de modo, eigenvalor, varianza explicada individual y acumulada para los m modos seleccionados, justificando la elección de m≈15.)

\subsection{Estimación de Pose (Escala, Traslación, Rotación)}
\label{ssec:estimacion_pose}
Antes de que un SSM pueda ser ajustado a una nueva imagen, se requiere una estimación inicial de la pose global (escala $S$, traslación $T$, y rotación $\Theta$) del objeto en la imagen. Esta estimación sirve para inicializar el modelo de forma cerca de la solución correcta. En esta tesis, se utiliza un sistema basado en Efficient Subspace Learning (ESL) para este propósito. Los detalles de ESL se abordarán en la sección de metodología (Capítulo 3), pero su función es proporcionar los parámetros ($S,T,\Theta$) que transforman la forma media canónica del SSM al espacio de la imagen de entrada. Investigaciones recientes también exploran el uso de CNNs para la regresión directa de parámetros de pose o la detección de landmarks clave para una estimación de pose robusta \cite{cao2014face, meyer2018deep}.


%\section{Extracción de Características}
\label{sec:extraccion_caracteristicas_teoria}
La extracción de características es el proceso de transformar datos brutos (como los píxeles de una imagen) en un conjunto de valores (un vector de características) que sea informativo, no redundante y que facilite las tareas de aprendizaje posteriores, como la clasificación o la regresión.

\subsection{Características Basadas en Perfiles de Intensidad (Contexto del Trabajo Previo)}
\label{ssec:caracteristicas_perfiles_1d}
En el contexto de los modelos deformables, una estrategia común para guiar el ajuste del modelo a la imagen es analizar la apariencia de la imagen en la vecindad de los landmarks del modelo. Los perfiles de intensidad 1D son una forma de hacerlo: para cada landmark, se extrae un perfil de intensidad a lo largo de una dirección (generalmente normal al contorno del modelo en ese punto). Este vector de intensidades 1D se utiliza luego como característica \cite{cootes1995active}. El trabajo previo que condujo a esta tesis (MaShDL v1) empleó perfiles de intensidad 1D. Si bien son computacionalmente eficientes, los perfiles 1D pueden perder información espacial 2D crucial, especialmente en regiones con texturas complejas o bordes ambiguos.

\subsection{Características Basadas en Apariencia Local 2D (Contexto del Nuevo Enfoque)}
\label{ssec:caracteristicas_parches_2d}
Para superar las limitaciones de los perfiles 1D, el enfoque MaShDL-CNN Hybrid propuesto en esta tesis se basa en el análisis de parches 2D extraídos alrededor de cada landmark. Un parche 2D de tamaño $Q \times Q$ captura una región de apariencia local mucho más rica que un perfil 1D. Estos parches pueden codificar información sobre texturas, gradientes en múltiples orientaciones y relaciones espaciales locales. La hipótesis es que esta información 2D más completa permitirá a un modelo de aprendizaje profundo (una CNN) aprender un mapeo más preciso hacia los parámetros de forma del SSM.

% (Sugerencia: Figura \ref{fig:comparacion_perfil_parche}: Una figura que muestre un landmark en un borde pulmonar. A la izquierda, visualizar un perfil de intensidad 1D extraído. A la derecha, visualizar un parche 2D Q×Q centrado en el mismo landmark. El pie de figura explicaría la mayor riqueza informativa del parche 2D.)

\subsection{Redes Neuronales Convolucionales (CNNs) como Extractoras de Características}
\label{ssec:cnns_feature_extractors}
Las Redes Neuronales Convolucionales (CNNs) se han convertido en el estándar de facto para la extracción de características de imágenes en una amplia variedad de tareas de visión por computadora \cite{lecun2015deep, gu2018recent}. Su arquitectura está inspirada en el córtex visual humano y se caracteriza por el uso de capas convolucionales, capas de pooling (submuestreo) y funciones de activación no lineales.

\begin{itemize}
\item \textbf{Capas Convolucionales:} Aplican un conjunto de filtros (kernels) aprendibles a la imagen de entrada (o al mapa de características de la capa anterior). Cada filtro está diseñado para detectar patrones específicos (e.g., bordes, texturas, formas simples). La operación de convolución permite compartir pesos, lo que reduce drásticamente el número de parámetros del modelo y lo hace más eficiente y menos propenso al sobreajuste. La salida de una capa convolucional es un conjunto de mapas de características.
Si $I$ es la entrada (imagen o mapa de características) y $K$ es un kernel, la operación de convolución 2D (simplificada) para un píxel $(x,y)$ en el mapa de salida $O$ es:
$O(x,y) = \sum_i \sum_j I(x-i, y-j)K(i,j) (+ \text{bias})$
Esto se aplica a través de toda la entrada para generar el mapa de características.

\item \textbf{Funciones de Activación:} Después de la convolución, se aplica una función de activación no lineal, como la Unidad Lineal Rectificada (ReLU), $f(z) = \max(0, z)$, o sus variantes (Leaky ReLU, ELU). Estas no linealidades son cruciales para que la red pueda aprender mapeos complejos.

\item \textbf{Capas de Pooling (Submuestreo):} Reducen la dimensionalidad espacial de los mapas de características, lo que ayuda a controlar el sobreajuste, reducir la carga computacional y crear invarianza a pequeñas traslaciones. Las operaciones de pooling comunes son Max Pooling (toma el valor máximo en una vecindad) y Average Pooling.

\item \textbf{Jerarquía de Características:} Al apilar múltiples capas convolucionales y de pooling, las CNNs aprenden una jerarquía de características. Las primeras capas tienden a aprender características de bajo nivel (e.g., bordes, esquinas), mientras que las capas más profundas aprenden a combinar estas características para detectar patrones más complejos y abstractos relevantes para la tarea en cuestión \cite{zeiler2014visualizing}.

\end{itemize}
En el contexto de esta tesis, una sub-CNN se utiliza para procesar cada parche 2D $Q \times Q$ extraído alrededor de los landmarks. La salida de esta sub-CNN es un vector de características de dimensión fija que representa la información relevante del parche para la estimación de la forma pulmonar. El uso de \texttt{tf.keras.layers.TimeDistributed} permite aplicar esta misma sub-CNN con pesos compartidos a todos los $N_{\text{lmk}}$ parches de una imagen, promoviendo un aprendizaje eficiente.

% (Sugerencia: Figura \ref{fig:arquitectura_cnn_generica}: Un diagrama genérico mostrando la estructura de una CNN típica: entrada, seguida de bloques [Convolución -> Activación -> Pooling], y finalmente capas totalmente conectadas (aunque en nuestro caso de extracción de features por parche, la "salida" es el vector de features antes de la DNN principal). Podrías indicar cómo los tamaños de los mapas de características cambian a través de la red.)


%\section{Clasificadores Supervisados para Detección de Enfermedades}
\label{sec:clasificadores_supervisados_teoria}
Una vez que la región pulmonar ha sido normalizada y se han extraído características relevantes de ella (esta vez, características orientadas a la detección de patologías, no a la forma), se utiliza un clasificador de aprendizaje supervisado para asignar una etiqueta de clase (e.g., sano, neumonía, COVID-19) a la imagen. En esta tesis se evaluarán tres tipos de clasificadores:

\subsection{K-Vecinos Más Cercanos (KNN)}
\label{ssec:knn_teoria}
El algoritmo K-Vecinos Más Cercanos (KNN) es un método de aprendizaje no paramétrico y basado en instancias \cite{altman1992introduction}. Para clasificar una nueva muestra:
\begin{itemize}
    \item Se calcula la distancia (e.g., Euclidiana, Manhattan) entre la nueva muestra y todas las muestras en el conjunto de entrenamiento.
    \item Se seleccionan los $K$ vecinos más cercanos.
    \item La etiqueta de la nueva muestra se asigna por voto mayoritario entre las etiquetas de estos $K$ vecinos.
\end{itemize}
KNN es simple de implementar, pero puede ser computacionalmente costoso en la fase de predicción para conjuntos de datos grandes y su rendimiento es sensible a la elección de $K$ y la métrica de distancia, así como a la presencia de características irrelevantes o escalas diferentes entre características.

\subsection{Perceptrón Multicapa (MLP)}
\label{ssec:mlp_teoria}
Un Perceptrón Multicapa (MLP) es un tipo de red neuronal artificial de alimentación hacia adelante (feedforward) que consta de al menos tres capas de nodos: una capa de entrada, una o más capas ocultas y una capa de salida \cite{rumelhart1986learning}. Cada nodo (neurona) en una capa está conectado a todos los nodos de la capa siguiente (capas densas o totalmente conectadas).
\begin{itemize}
\item Cada conexión tiene un peso asociado $w_{ij}$.
\item Cada neurona (excepto las de entrada) aplica una función de activación no lineal (e.g., sigmoide, tangente hiperbólica, ReLU) a la suma ponderada de sus entradas más un sesgo (bias).
$y_j = f\left(\sum_i w_{ij} x_i + b_j\right)$
donde $y_j$ es la salida de la neurona $j$, $x_i$ son las entradas, $w_{ij}$ los pesos, $b_j$ el sesgo, y $f(\cdot)$ la función de activación.
\item El aprendizaje en un MLP se realiza típicamente mediante el algoritmo de retropropagación (backpropagation), que ajusta los pesos y sesgos para minimizar una función de pérdida (e.g., entropía cruzada para clasificación).
\end{itemize}
Los MLPs son capaces de aprender funciones no lineales complejas y han sido ampliamente utilizados en tareas de clasificación. En el contexto de esta tesis, un MLP se utilizará como uno de los clasificadores para la detección de enfermedades, tomando como entrada las características extraídas de las regiones pulmonares normalizadas. Además, dentro del propio modelo MaShDL-CNN Hybrid, una DNN (que es un tipo de MLP) se utiliza para predecir los bins de los coeficientes $b_k$.

\subsection{Redes Neuronales Convolucionales (CNNs) para Clasificación}
\label{ssec:cnns_clasificacion_teoria}
Además de su uso como extractoras de características, las CNNs también pueden ser entrenadas de extremo a extremo para tareas de clasificación de imágenes. En este caso, después de las capas convolucionales y de pooling que extraen características, se suelen añadir una o más capas totalmente conectadas (como en un MLP) que culminan en una capa de salida con una función de activación softmax para producir probabilidades de clase \cite{krizhevsky2012imagenet}.
Para la tarea de clasificación de enfermedades en esta tesis, se podría diseñar una CNN que tome como entrada la región pulmonar segmentada y normalizada (o parches de esta región) y la clasifique directamente en las categorías patológicas. El entrenamiento de dicha CNN buscaría aprender las características visuales dentro de la región pulmonar que son discriminantes para neumonía y COVID-19. Modelos pre-entrenados en grandes conjuntos de datos de imágenes naturales (como ImageNet \cite{deng2009imagenet}) a menudo se utilizan como punto de partida mediante aprendizaje por transferencia (transfer learning), adaptándolos luego a la tarea médica específica \cite{shin2016deep,tajbakhsh2016convolutional}.


%\section{Métricas de Evaluación}
\label{sec:metricas_evaluacion_teoria}
La evaluación cuantitativa del rendimiento de los algoritmos desarrollados es fundamental. Se utilizarán diferentes métricas según la tarea.

\subsection{Para Segmentación (Coeficiente de Dice)}
\label{ssec:metricas_segmentacion}
El rendimiento de la segmentación de la región pulmonar (resultado del proceso de alineación y normalización) se evaluará principalmente mediante el Coeficiente de Similitud de Dice (DSC), también conocido como puntuación F1 de solapamiento. Dados una máscara de segmentación predicha $M_{\text{pred}}$ y una máscara de referencia (ground truth) $M_{\text{GT}}$, el DSC se define como:
$\text{DSC}(M_{\text{GT}}, M_{\text{pred}}) = \frac{2 \cdot |M_{\text{GT}} \cap M_{\text{pred}}|}{|M_{\text{GT}}| + |M_{\text{pred}}|}$
donde $| \cdot |$ denota el número de píxeles en la región (área) y $\cap$ representa la intersección. El DSC varía entre 0 (ninguna superposición) y 1 (superposición perfecta). Un valor más alto indica una mejor segmentación.
Otras métricas como la Distancia de Hausdorff o el Jaccard Index (IoU, Intersection over Union), donde $\text{IoU} = \text{DSC} / (2 - \text{DSC})$, también son comunes pero el Dice es prevalente en la literatura de segmentación médica \cite{taha2015metrics}.

\subsection{Para Clasificación}
\label{ssec:metricas_clasificacion}
Para la tarea de clasificación de enfermedades (sano, neumonía, COVID-19), se utilizarán las siguientes métricas, derivadas de la matriz de confusión:
Sea $TP$ (Verdaderos Positivos), $TN$ (Verdaderos Negativos), $FP$ (Falsos Positivos), $FN$ (Falsos Negativos) para una clase específica.
\begin{itemize}
\item \textbf{Precisión (Accuracy):} Proporción de predicciones correctas sobre el total.
$\text{Accuracy} = \frac{TP+TN}{TP+TN+FP+FN}$
\item \textbf{Sensibilidad (Recall o Tasa de Verdaderos Positivos):} Proporción de positivos reales que fueron correctamente identificados.
$\text{Sensibilidad} = \frac{TP}{TP+FN}$
\item \textbf{Especificidad (Tasa de Verdaderos Negativos):} Proporción de negativos reales que fueron correctamente identificados.
$\text{Especificidad} = \frac{TN}{TN+FP}$
\item \textbf{Puntuación F1 (F1-Score):} Media armónica de la precisión (valor predictivo positivo) y la sensibilidad, útil para clases desequilibradas.
$\text{Precisión (PPV)} = \frac{TP}{TP+FP}$
$\text{F1-Score} = 2 \cdot \frac{\text{Precisión (PPV)} \cdot \text{Sensibilidad}}{\text{Precisión (PPV)} + \text{Sensibilidad}}$
\item \textbf{Curva ROC (Receiver Operating Characteristic):} Gráfico de la Sensibilidad (TPR) vs. 1 - Especificidad (FPR) para diferentes umbrales de clasificación.
\item \textbf{Área Bajo la Curva ROC (AUC):} Medida de la capacidad general del clasificador para distinguir entre clases. Un AUC de 1 representa un clasificador perfecto, mientras que 0.5 representa un clasificador aleatorio \cite{fawcett2006introduction}.
\end{itemize}
Estas métricas proporcionarán una evaluación exhaustiva del rendimiento de los clasificadores de enfermedades desarrollados. La validación cruzada se empleará para obtener estimaciones más robustas de estas métricas y evaluar la generalización del modelo \cite{kohavi1995study}.

% (Sugerencia: Tabla \ref{tab:matriz_confusion_ejemplo}: Un ejemplo de una matriz de confusión para un problema de 3 clases (Sano, Neumonía, COVID-19) y cómo se calcularían TP, TN, FP, FN para una clase específica, por ejemplo, COVID-19.)

Este marco teórico proporciona los cimientos para entender la metodología, los experimentos y los resultados que se presentarán en los capítulos subsiguientes de esta tesis.

%Capítulo 3: Metodología (Deep Learning - ResNet-18 + Geometric Loss Functions)
\chapter{Marco Teórico y Estado del Arte}

La detección automática de puntos de referencia anatómicos (\textit{landmarks}) en radiografías de tórax constituye un problema fundamental en el análisis computarizado de imágenes médicas. Como se estableció en el Capítulo 1, la localización precisa de estos puntos es esencial para la cuantificación de estructuras anatómicas, el cálculo de índices diagnósticos como el índice cardiotorácico, y la normalización espacial de radiografías para sistemas de clasificación automática. El presente capítulo tiene como objetivo establecer los fundamentos teóricos que sustentan el desarrollo de métodos basados en aprendizaje profundo para la detección automática de \textit{landmarks} anatómicos, revisando tanto los principios fundamentales como el estado del arte actual en esta área de investigación.

La detección de \textit{landmarks} en imágenes médicas ha experimentado una evolución significativa en las últimas dos décadas. Tradicionalmente, este problema se abordó mediante métodos estadísticos y geométricos, tales como los Modelos Activos de Forma (\textit{Active Shape Models}, ASM) \cite{Cootes1995} y los Modelos Activos de Apariencia (\textit{Active Appearance Models}, AAM) \cite{Cootes2001}, que representan variaciones anatómicas mediante descomposición lineal basada en Análisis de Componentes Principales. Si bien estos métodos clásicos demostraron utilidad en escenarios controlados, presentan limitaciones fundamentales relacionadas con la linealidad de sus representaciones y su dependencia de características diseñadas manualmente (\textit{hand-crafted features}) \cite{Heimann2009}. La irrupción del aprendizaje profundo en visión por computadora, particularmente tras el trabajo seminal de Krizhevsky et al. \cite{Krizhevsky2012}, ha revolucionado el análisis de imágenes médicas al permitir el aprendizaje automático de representaciones jerárquicas directamente desde los datos \cite{Litjens2017, Shen2017}. En el contexto específico de la detección de \textit{landmarks}, las Redes Neuronales Convolucionales (CNNs) han demostrado capacidad superior para capturar patrones complejos y no lineales en estructuras anatómicas, superando consistentemente el desempeño de métodos tradicionales.

El presente capítulo se estructura en ocho secciones que abarcan desde los fundamentos físicos de las radiografías de tórax hasta el estado del arte en métodos basados en aprendizaje profundo. La Sección~2.1 introduce los principios físicos de las radiografías torácicas y define los quince \textit{landmarks} anatómicos relevantes para este trabajo. La Sección~2.2 establece los fundamentos matemáticos de las redes neuronales convolucionales, incluyendo la operación de convolución, funciones de activación, y el algoritmo de retropropagación (\textit{backpropagation}). La Sección~2.3 analiza en detalle las arquitecturas residuales, particularmente la familia ResNet, que han demostrado ser especialmente efectivas para el entrenamiento de redes profundas mediante el uso de conexiones residuales (\textit{skip connections}). La Sección~2.4 examina el paradigma de aprendizaje por transferencia (\textit{transfer learning}), un componente crucial cuando se trabaja con conjuntos de datos médicos de tamaño limitado. La Sección~2.5 presenta una revisión exhaustiva de funciones de pérdida especializadas para la regresión de coordenadas, con énfasis particular en \textit{Wing Loss} y funciones de pérdida basadas en restricciones geométricas. La Sección~2.6 contrasta los enfoques de regresión directa de coordenadas versus regresión de mapas de calor (\textit{heatmap regression}), justificando la elección metodológica adoptada en esta tesis. La Sección~2.7 ofrece un análisis comparativo exhaustivo del estado del arte en detección de \textit{landmarks} anatómicos, identificando las brechas que motivan el presente trabajo. Finalmente, la Sección~2.8 sintetiza los conceptos presentados y establece la conexión con la metodología propuesta que se desarrollará en el Capítulo 3.

\section{Conjunto de Datos}
\label{sec:dataset}

La calidad, diversidad y representatividad del conjunto de datos constituyen factores determinantes para el desempeño, generalización y validez clínica de cualquier sistema basado en aprendizaje profundo aplicado a imágenes médicas. El \textit{dataset} (conjunto de datos) empleado en este trabajo fue diseñado para representar variabilidad anatómica y patológica real encontrada en práctica radiológica contemporánea, incluyendo condiciones normales y patológicas que modifican significativamente la morfología torácica visible en radiografías de tórax.

\subsection{Descripción General y Composición}
\label{subsec:descripcion_dataset}

El conjunto de datos consiste en 956 radiografías digitales de tórax adquiridas en proyección posteroanterior (PA), la vista estándar para evaluación radiológica torácica de rutina. Cada imagen incluye anotaciones manuales expertas de 15 \textit{landmarks} anatómicos críticos, realizadas por radiólogos certificados con experiencia clínica superior a cinco años, siguiendo protocolos estandarizados de identificación de estructuras anatómicas visibles en radiografías convencionales. Las imágenes fueron recopiladas de repositorios públicos de imágenes médicas anonimizadas, cumpliendo rigurosamente con regulaciones HIPAA (\textit{Health Insurance Portability and Accountability Act}) de protección de información de pacientes, eliminando toda información identificable mediante técnicas de de-identificación certificadas.

La composición del \textit{dataset} refleja la distribución epidemiológica contemporánea de condiciones respiratorias relevantes, incluyendo tres categorías diagnósticas principales: 306 imágenes (32.0\%) corresponden a pacientes con diagnóstico confirmado de COVID-19 mediante pruebas moleculares RT-PCR, presentando hallazgos radiológicos característicos como opacidades en vidrio esmerilado, consolidaciones bilaterales, y distribución periférica de infiltrados; 183 imágenes (19.1\%) provienen de casos de neumonía viral no-COVID documentados clínicamente, mostrando patrones infiltrativos diversos; y 467 imágenes (48.8\%) constituyen controles normales sin hallazgos patológicos significativos, obtenidas de estudios de cribado o seguimiento de pacientes sin enfermedad respiratoria aguda. Esta diversidad de condiciones patológicas es esencial para evaluar robustez del sistema ante variabilidad anatómica inducida por procesos patológicos que alteran siluetas cardíacas, bordes pulmonares, y posiciones diafragmáticas, aspectos que impactan directamente la localización precisa de \textit{landmarks} anatómicos.

Datasets públicos ampliamente utilizados en investigación de imágenes torácicas incluyen el JSRT \textit{Database} (base de datos) de la Sociedad Japonesa de Tecnología Radiológica \cite{Shiraishi2000}, conteniendo 247 radiografías PA con anotaciones de nódulos pulmonares; el ChestX-ray14 del NIH (\textit{National Institutes of Health}) \cite{Wang2017}, repositorio masivo de 112,120 imágenes con etiquetas de 14 patologías extraídas mediante procesamiento de lenguaje natural de informes radiológicos; y la COVID-19 Radiography Database \cite{Chowdhury2020}, colección especializada de imágenes de pacientes con neumonía viral y COVID-19. El \textit{dataset} utilizado en este trabajo comparte características metodológicas con estos repositorios de referencia, particularmente en protocolos de anonimización, diversidad de condiciones patológicas, y disponibilidad de anotaciones estructuradas, aunque se especializa en localización precisa de \textit{landmarks} anatómicos mediante coordenadas punto a punto en lugar de etiquetas de clasificación o segmentaciones de regiones de interés.

\subsection{Características Técnicas de las Imágenes}
\label{subsec:caracteristicas_tecnicas}

Las imágenes radiográficas digitales presentan resolución espacial original de 299$\times$299 píxeles, adquiridas mediante sistemas de radiografía digital directa (DR) o radiografía computarizada (CR) de distintos fabricantes, introduciendo heterogeneidad instrumental representativa de entornos clínicos reales con equipamiento variado. Cada imagen consiste en un canal único de intensidad (escala de grises) codificado con profundidad de 8 bits por píxel, proporcionando 256 niveles de gris en el rango [0, 255], donde valores bajos representan regiones radiopacas (tejidos densos, estructuras óseas, mediastino) y valores altos corresponden a regiones radiolúcidas (campos pulmonares aireados). El formato de almacenamiento es PNG (\textit{Portable Network Graphics}), formato sin compresión con pérdida que preserva fidelidad diagnóstica completa al evitar artefactos de compresión JPEG que podrían degradar bordes anatómicos sutiles críticos para localización precisa de \textit{landmarks}.

\begin{table}[!ht]
\centering
\caption{Especificaciones técnicas del conjunto de datos de radiografías de tórax}
\label{tab:dataset_specs}
\begin{tabular}{@{}ll@{}}
\toprule
\textbf{Característica} & \textbf{Especificación} \\
\midrule
Resolución espacial original & 299 $\times$ 299 píxeles \\
Resolución procesada (entrada modelo) & 224 $\times$ 224 píxeles \\
Formato de almacenamiento & PNG sin compresión \\
Profundidad de bits & 8 bits por píxel \\
Espacio de color & Escala de grises \\
Rango de intensidad & [0, 255] (valores enteros) \\
Número total de imágenes & 956 \\
\textit{Landmarks} anatómicos por imagen & 15 puntos de referencia \\
Coordenadas anotadas totales & 28,680 valores (956 $\times$ 15 $\times$ 2) \\
Proyección radiográfica & Posteroanterior (PA) estándar \\
Tipos de condiciones incluidas & Normal, COVID-19, Neumonía Viral \\
\bottomrule
\end{tabular}
\end{table}

La resolución procesada de 224$\times$224 píxeles, detallada en la Sección~\ref{sec:pipeline_datos}, constituye la dimensión estándar requerida por arquitecturas ResNet preentrenadas en ImageNet, datasets de referencia para \textit{transfer learning} en visión por computadora. Este redimensionamiento, aunque implica pérdida de información espacial (reducción de $299^2 = 89{,}401$ a $224^2 = 50{,}176$ píxeles, retención del 56.1\% de información espacial), es necesario para aprovechar representaciones visuales genéricas aprendidas en ImageNet, compensando la reducción mediante \textit{transfer learning} que proporciona inicialización superior a entrenamiento desde cero con datos médicos limitados, como demuestran empíricamente Raghu et al. \cite{Raghu2019} y Tajbakhsh et al. \cite{Tajbakhsh2016}.

El conjunto completo contiene 28,680 coordenadas anotadas manualmente (956 imágenes $\times$ 15 \textit{landmarks} $\times$ 2 coordenadas por punto), constituyendo un corpus sustancial de supervisión experta para entrenamiento de regresores neuronales. La densidad de anotación (15 puntos por imagen) proporciona información geométrica suficiente para capturar estructura anatómica torácica principal sin sobrecargar el proceso de anotación manual, balanceando riqueza informativa con viabilidad práctica de creación de \textit{ground truth} (verdad fundamental) por expertos clínicos con tiempo limitado.

\subsection{Definición de Landmarks Anatómicos}
\label{subsec:definicion_landmarks}

Los 15 \textit{landmarks} anatómicos seleccionados para anotación corresponden a estructuras visibles consistentemente en radiografías PA de tórax de calidad diagnóstica estándar, identificables por radiólogos expertos con variabilidad inter-observador aceptable (desviación estándar típica inferior a 3-5 píxeles según estudios de reproducibilidad en localización manual de \textit{landmarks} torácicos). La selección de estos puntos de referencia específicos se fundamenta en su relevancia clínica para mediciones diagnósticas rutinarias: el índice cardiotorácico (ICT), relación entre diámetro cardíaco máximo y diámetro torácico interno máximo, utiliza posiciones de bordes cardíacos y paredes torácicas; la evaluación de posición diafragmática para detectar elevación unilateral o bilateral emplea bases pulmonares y ángulos costofrénicos; el análisis de silueta mediastínica para detectar adenopatías o masas requiere identificación precisa de bordes mediastínicos superior e inferior; y la detección de anomalías hilares (adenopatías, masas, vascularización pulmonar anormal) utiliza posiciones de hila pulmonares izquierdo y derecho como referencias anatómicas.

La Tabla~\ref{tab:landmarks} proporciona descripción anatómica completa de cada \textit{landmark}, organizada por región anatómica (mediastino, pulmones bilaterales, estructura ósea torácica) para facilitar comprensión de distribución espacial y relaciones anatómicas entre puntos de referencia.

\begin{table}[!ht]
\centering
\caption{Definición anatómica detallada de los 15 \textit{landmarks} anotados en radiografías de tórax PA}
\label{tab:landmarks}
\begin{tabular}{@{}clp{7.5cm}@{}}
\toprule
\textbf{ID} & \textbf{Región} & \textbf{Descripción Anatómica Específica} \\
\midrule
0 & Mediastino & Borde superior del mediastino, intersección con límite superior de la imagen \\
1 & Mediastino & Punto medio mediastínico superior, aproximadamente a nivel de la carina traqueal \\
2 & Pulmón izq. & Ápice pulmonar izquierdo, punto más superior del campo pulmonar izquierdo \\
3 & Pulmón der. & Ápice pulmonar derecho, punto más superior del campo pulmonar derecho \\
4 & Pulmón izq. & Hilio pulmonar izquierdo, centro geométrico de la región hilar \\
5 & Pulmón der. & Hilio pulmonar derecho, centro geométrico de la región hilar \\
6 & Pulmón izq. & Base pulmonar izquierda, intersección del hemidiafragma con silueta cardíaca \\
7 & Pulmón der. & Base pulmonar derecha, intersección del hemidiafragma con silueta cardíaca \\
8 & Mediastino & Punto central mediastínico, centro geométrico del mediastino medio \\
9 & Mediastino & Punto inferior mediastínico, aproximadamente a nivel de la unión cardiodiafragmática \\
10 & Mediastino & Base del mediastino, límite inferior visible de la silueta mediastínica \\
11 & Tórax izq. & Borde costal superior izquierdo, punto de referencia lateral izquierdo \\
12 & Tórax der. & Borde costal superior derecho, punto de referencia lateral derecho \\
13 & Tórax izq. & Ángulo costofrénico izquierdo, intersección de diafragma con pared torácica lateral \\
14 & Tórax der. & Ángulo costofrénico derecho, intersección de diafragma con pared torácica lateral \\
\bottomrule
\end{tabular}
\end{table}

La distribución espacial de estos \textit{landmarks} captura geometría torácica fundamental: cinco puntos centrales (IDs: 0, 1, 8, 9, 10) definen el eje mediastínico vertical, estructura central que separa cavidades pleurales izquierda y derecha; cuatro pares bilaterales simétricos (IDs: 2-3, 4-5, 6-7, 11-12) representan estructuras anatómicas reflejadas respecto al plano sagital medio; y un par inferior (IDs: 13-14) corresponde a ángulos costofrénicos, puntos de referencia críticos para detectar derrames pleurales. Esta organización anatómica estructurada es explotada posteriormente mediante restricciones geométricas implementadas en funciones de pérdida (Sección~\ref{sec:estrategia_entrenamiento}), transformando conocimiento anatómico cualitativo en supervisión cuantitativa diferenciable.

\subsection{Pares de Landmarks Simétricos y Eje Mediastínico}
\label{subsec:simetria_bilateral}

La simetría bilateral constituye una invariante geométrica fundamental de la anatomía torácica humana normal: estructuras pulmonares, costales y pleurales presentan reflexión aproximada respecto al plano sagital medio definido por el mediastino, estructura central que contiene corazón, grandes vasos, tráquea, esófago y estructuras mediastínicas. Aunque patologías unilaterales (consolidaciones lobares, derrames pleurales, neumotórax) pueden romper simetría localmente, la estructura ósea de la caja torácica y posiciones relativas de estructuras bilaterales mantienen simetría aproximada incluso en presencia de enfermedad pulmonar. Esta propiedad anatómica puede explotarse computacionalmente mediante restricciones de simetría que penalizan inconsistencias entre posiciones de \textit{landmarks} pareados, proporcionando regularización geométrica que mejora consistencia anatómica de predicciones, como demuestran trabajos previos en modelado de estructuras simétricas \cite{Donner2013}.

Se identifican cinco pares de \textit{landmarks} bilaterales que deben presentar reflexión aproximada respecto al eje mediastínico vertical, definidos formalmente mediante el conjunto de pares simétricos:

\begin{equation}
\mathcal{P}_{sym} = \{(2,3),\, (4,5),\, (6,7),\, (11,12),\, (13,14)\}
\label{eq:pares_simetricos}
\end{equation}

donde cada tupla $(i, j) \in \mathcal{P}_{sym}$ indica que el \textit{landmark} con índice $i$ (estructura izquierda) y el \textit{landmark} con índice $j$ (estructura derecha) forman un par anatómico bilateral. Específicamente: $(2, 3)$ corresponde a ápices pulmonares izquierdo-derecho; $(4, 5)$ a hila pulmonares; $(6, 7)$ a bases pulmonares; $(11, 12)$ a bordes costales superiores; y $(13, 14)$ a ángulos costofrénicos. Los cinco \textit{landmarks} centrales (IDs: 0, 1, 8, 9, 10) son estructuras mediastínicas de línea media que no tienen par simétrico, definiendo en cambio el eje de reflexión.

El eje de simetría mediastínico se calcula como promedio ponderado de las coordenadas horizontales ($x$) de los \textit{landmarks} centrales, asignando pesos diferenciados según confiabilidad anatómica de cada punto como indicador de línea media. El \textit{landmark} central (ID 8) recibe peso máximo al corresponder al centro geométrico del mediastino medio, región de máxima estabilidad anatómica. Los \textit{landmarks} superior e inferior (IDs: 0, 1, 9, 10) reciben pesos ligeramente menores debido a mayor variabilidad anatómica en extremos del mediastino. Formalmente, la coordenada $x$ del eje de simetría se define como:

\begin{equation}
x_{axis} = \frac{\sum_{k \in \mathcal{I}_{med}} w_k \cdot x_k}{\sum_{k \in \mathcal{I}_{med}} w_k}
\label{eq:eje_simetria}
\end{equation}

donde $\mathcal{I}_{med} = \{0, 1, 8, 9, 10\}$ denota el conjunto de índices de \textit{landmarks} mediastínicos, $x_k$ representa la coordenada horizontal (normalizada al rango $[0, 1]$) del \textit{landmark} $k$, y los pesos $\mathbf{w} = [1.2,\, 1.2,\, 1.5,\, 1.3,\, 1.3]$ corresponden a los \textit{landmarks} en orden de índices crecientes. El peso máximo $w_8 = 1.5$ asignado al punto central enfatiza su rol como ancla principal del eje de simetría, mientras que los pesos restantes ($\approx 1.2$-$1.3$) contribuyen equitativamente a estabilidad del cálculo mediante promediado robusto que reduce sensibilidad a variabilidad individual de puntos extremos.

Esta definición del eje de simetría mediastínico es utilizada posteriormente en la implementación de \textit{Symmetry Loss} (Sección~\ref{sec:phase3_symmetry}), función de pérdida que penaliza desviaciones de simetría bilateral mediante reflexión de puntos a través de $x = x_{axis}$ y comparación con posiciones esperadas de pares simétricos. La formulación matemática completa de esta restricción geométrica se presenta en el contexto del protocolo de entrenamiento, donde restricciones de simetría se incorporan gradualmente durante Fase 3 del entrenamiento progresivo.

\subsection{División del Dataset para Entrenamiento, Validación y Prueba}
\label{subsec:division_dataset}

La división del conjunto de datos en subconjuntos disjuntos de entrenamiento, validación y prueba constituye práctica fundamental en aprendizaje supervisado para evaluación rigurosa de capacidad de generalización, detección de sobreajuste, y estimación no sesgada de desempeño en datos no vistos. El protocolo de división implementado sigue metodología estándar en aprendizaje automático, asignando 70\% de imágenes a entrenamiento para maximizar datos disponibles para aprendizaje de parámetros del modelo, 15\% a validación para monitoreo de convergencia y selección de hiperparámetros mediante \textit{early stopping} (detención temprana), y 15\% a prueba para evaluación final de desempeño sobre datos completamente no vistos durante todo el proceso de desarrollo.

La división se realiza mediante muestreo aleatorio estratificado por categoría diagnóstica, garantizando que las proporciones de COVID-19 (32\%), Neumonía Viral (19\%), y Normal (49\%) se preserven aproximadamente en cada subconjunto. Esta estratificación es esencial para evitar desbalances que sesgarían evaluación: un conjunto de prueba desproporcionadamente poblado con imágenes normales proporcionaría estimación optimista de desempeño, mientras que sobrerrepresentación de casos patológicos produciría estimación pesimista. La implementación utiliza la función \texttt{train\_test\_split} de la librería \textit{scikit-learn} \cite{Pedregosa2011}, herramienta estándar en aprendizaje automático que implementa muestreo aleatorio con control de semilla para reproducibilidad determinística. La semilla aleatoria se fija en \texttt{random\_seed=42}, valor convencional en comunidad de ciencia de datos que permite replicación exacta de la división en ejecuciones independientes.

\begin{table}[!ht]
\centering
\caption{División estratificada del conjunto de datos en subconjuntos de entrenamiento, validación y prueba}
\label{tab:dataset_split}
\begin{tabular}{@{}lcccc@{}}
\toprule
\textbf{Subconjunto} & \textbf{Porcentaje} & \textbf{COVID-19} & \textbf{Neumonía Viral} & \textbf{Normal} \\
\midrule
Entrenamiento & 70\% & 214 (32.0\%) & 128 (19.1\%) & 327 (48.9\%) \\
Validación & 15\% & 46 (31.9\%) & 27 (18.8\%) & 71 (49.3\%) \\
Prueba & 15\% & 46 (31.9\%) & 28 (19.4\%) & 69 (47.9\%) \\
\midrule
\textbf{Total} & 100\% & \textbf{306} & \textbf{183} & \textbf{467} \\
\bottomrule
\end{tabular}
\end{table}

La Tabla~\ref{tab:dataset_split} muestra la distribución resultante, donde se observa que las proporciones de cada categoría diagnóstica se preservan con desviaciones menores al 1.5\% respecto a la distribución global, confirmando efectividad del muestreo estratificado. El conjunto de entrenamiento con 669 imágenes proporciona volumen suficiente para optimización de los 11.6 millones de parámetros del modelo mediante descenso de gradiente estocástico con \textit{mini-batches}, aunque el tamaño moderado del \textit{dataset} justifica el uso de \textit{transfer learning} desde ImageNet y técnicas agresivas de \textit{data augmentation} (Sección~\ref{sec:pipeline_datos}) para prevenir sobreajuste. Los conjuntos de validación y prueba, cada uno con 144 imágenes (equivalente al 10\% del tamaño de ImageNet para referencia estadística), permiten evaluación estadísticamente significativa con intervalos de confianza razonables para métricas de error medio.

El conjunto de validación cumple dos roles metodológicos críticos durante entrenamiento: (1) monitoreo de convergencia mediante evaluación periódica de pérdida y métricas de error, permitiendo detección temprana de divergencia u oscilaciones numéricas, y (2) implementación de \textit{early stopping} con paciencia de 10-15 épocas (Sección~\ref{sec:estrategia_entrenamiento}), deteniendo entrenamiento cuando pérdida de validación deja de disminuir, señalando que el modelo comienza a sobreajustarse al conjunto de entrenamiento. El conjunto de prueba permanece completamente no visto hasta la evaluación final después de completar todas las fases de entrenamiento y selección de hiperparámetros, proporcionando estimación no sesgada del desempeño esperado en datos clínicos nuevos, aspecto esencial para validación científica rigurosa.

\subsection{Calidad y Validación de Anotaciones}
\label{subsec:calidad_anotaciones}

La calidad de anotaciones manuales de \textit{landmarks} constituye el límite superior de desempeño alcanzable por cualquier modelo supervisado: errores sistemáticos en \textit{ground truth} degradan irreversiblemente capacidad del sistema al entrenar el modelo para reproducir inconsistencias humanas. Las anotaciones empleadas en este trabajo fueron realizadas por radiólogos certificados con experiencia clínica documentada superior a cinco años en interpretación de radiografías de tórax, siguiendo protocolos estandarizados de identificación de estructuras anatómicas. Los protocolos especifican criterios anatómicos explícitos para cada \textit{landmark} (detallados en Tabla~\ref{tab:landmarks}), instrucciones de manejo de casos ambiguos (estructura parcialmente oscurecida por patología o superposición), y procedimientos de control de calidad post-anotación.

Aunque el \textit{dataset} no incluye anotaciones múltiples independientes por imagen que permitirían cuantificación rigurosa de acuerdo inter-observador mediante coeficiente de correlación intraclase (ICC) o estadística kappa, práctica ideal en construcción de \textit{datasets} médicos de referencia, la consistencia anatómica de las anotaciones fue validada retrospectivamente mediante verificación automática de restricciones geométricas que cualquier conjunto de 15 \textit{landmarks} anatómicamente válidos debe satisfacer. Estas verificaciones incluyen:

\textbf{Restricciones de ordenamiento espacial vertical:} Los ápices pulmonares (IDs: 2, 3) deben ubicarse superiormente a los hila pulmonares (IDs: 4, 5), que a su vez deben estar por encima de las bases pulmonares (IDs: 6, 7). Formalmente, se verifica $y_{apex} < y_{hilum} < y_{base}$ para cada hemitórax, donde $y$ denota coordenada vertical con origen superior. Esta restricción captura anatomía torácica fundamental: inversión de este ordenamiento indicaría error grave de anotación o algoritmo de validación.

\textbf{Validación de simetría bilateral aproximada:} Para cada par $(i, j) \in \mathcal{P}_{sym}$, se calcula la discrepancia de simetría $\Delta_{sym} = ||d_i - d_j|| / \bar{d}$, donde $d_i = |x_i - x_{axis}|$ es la distancia horizontal del \textit{landmark} $i$ al eje mediastínico y $\bar{d}$ es la distancia promedio del par para normalización. Se verifica que $\Delta_{sym} < 0.15$ (discrepancia menor al 15\%), umbral que permite variabilidad anatómica normal (ligeras asimetrías cardíacas, rotación torácica leve) mientras detecta errores groseros de anotación (confusión de lados, desplazamientos extremos).

\textbf{Comprobación de rangos fisiológicos para distancias anatómicas:} Se verifican rangos aceptables para distancias críticas: el ancho torácico (distancia entre bordes costales laterales) debe estar en rango $[0.6, 0.95]$ de la anchura de imagen para evitar anotaciones exageradamente comprimidas o expandidas; la altura mediastínica (distancia entre puntos mediastínicos superior e inferior) debe ocupar fracción sustancial de altura de imagen $[0.4, 0.8]$; y distancias entre pares simétricos bilaterales deben ser comparables (ratio $[0.7, 1.3]$) para detectar asimetrías extremas no fisiológicas.

Estas validaciones automáticas identificaron menos del 2\% de imágenes con potenciales inconsistencias geométricas, que fueron revisadas manualmente y corregidas cuando se confirmaron errores de anotación, o marcadas con notas explicativas cuando la aparente inconsistencia correspondía a anatomía genuinamente inusual (cifoescoliosis severa, cardiomegalia extrema) o limitaciones de calidad de imagen. La tasa de inconsistencias detectadas ($< 2\%$) es comparable a tasas reportadas en \textit{datasets} médicos de referencia con control de calidad riguroso, sugiriendo consistencia adecuada de las anotaciones para entrenamiento supervisado.

La ausencia de anotaciones múltiples independientes constituye una limitación reconocida del \textit{dataset}: no permite cuantificar variabilidad inter-observador ni establecer intervalos de confianza para \textit{ground truth}, aspectos que serían deseables para evaluación estadística completa de desempeño del sistema en relación con variabilidad humana experta. Trabajos futuros podrían beneficiarse de obtención de anotaciones redundantes por múltiples radiólogos independientes en subconjunto representativo del \textit{dataset}, permitiendo estimación de límites de desempeño humano y comparación más rigurosa de sistemas automáticos con desempeño experto, metodología estándar en competencias internacionales de análisis de imágenes médicas.

La siguiente sección describe la arquitectura neuronal profunda diseñada para procesar las imágenes radiográficas y predecir las coordenadas de los 15 \textit{landmarks} anatómicos definidos en el presente conjunto de datos.

\section{Arquitectura del Modelo}
\label{sec:arquitectura}

La arquitectura neuronal profunda seleccionada para la tarea de regresión de coordenadas de \textit{landmarks} anatómicos debe balancear múltiples objetivos en tensión: capacidad representacional suficiente para capturar variabilidad anatómica compleja observable en radiografías de tórax de 224$\times$224 píxeles, eficiencia computacional compatible con recursos de \textit{hardware} disponible (GPU con 8GB VRAM), facilidad de entrenamiento mediante \textit{transfer learning} desde \textit{datasets} de imágenes naturales, y arquitectura modular que permita experimentación con componentes intercambiables. La arquitectura implementada se fundamenta en ResNet-18 \cite{He2016}, variante ligera de la familia de Redes Residuales que proporciona profundidad suficiente (18 capas con pesos) para aprendizaje de representaciones jerárquicas complejas sin incurrir en costo computacional prohibitivo de variantes más profundas como ResNet-50 o ResNet-101.

% Como se fundamentó teóricamente en la Sección~\ref{sec:resnet_teoria} del marco teórico, (NOTA: label no existe en Cap 2 actual)
Las conexiones residuales $\mathbf{y} = \mathcal{F}(\mathbf{x}) + \mathbf{x}$ constituyen el mecanismo arquitectural clave que permite entrenamiento efectivo de redes profundas al facilitar flujo directo de gradientes durante retropropagación, evitando problema de desvanecimiento de gradientes que limita profundidad de arquitecturas completamente convolucionales estándar. La arquitectura ResNet-18 específica empleada mantiene estructura de bloques residuales básicos (\textit{basic blocks}) sin cuellos de botella (\textit{bottlenecks}), adecuados para imágenes de resolución moderada donde capacidad representacional de bloques básicos es suficiente y complejidad adicional de bloques con cuello de botella no proporciona beneficio significativo.

\subsection{Backbone ResNet-18: Extractor de Características Visuales}
\label{subsec:backbone_resnet18}

El \textit{backbone} (columna vertebral) del modelo consiste en ResNet-18 preentrenada en ImageNet \cite{Krizhevsky2012}, \textit{dataset} de clasificación de imágenes naturales conteniendo 1.2 millones de imágenes de entrenamiento distribuidas en 1000 categorías de objetos. El preentrenamiento en ImageNet proporciona inicialización de parámetros que codifica características visuales genéricas útiles para múltiples tareas de visión por computadora: detectores de bordes, esquinas y texturas en capas inferiores; detectores de partes de objetos y patrones complejos en capas medias; y representaciones semánticas de alto nivel en capas superiores. Aunque las 1000 categorías de ImageNet no incluyen imágenes médicas, numerosos estudios empíricos demuestran transferibilidad sorprendente de representaciones aprendidas en imágenes naturales a dominio médico \cite{Raghu2019, Tajbakhsh2016}, particularmente cuando el \textit{dataset} médico objetivo es pequeño ($< 10{,}000$ imágenes) y entrenamiento desde inicialización aleatoria resultaría en sobreajuste severo.

La arquitectura ResNet-18 implementa la estructura jerárquica estándar de redes residuales, comenzando con capa convolucional inicial de $7\times7$ con \textit{stride} 2 que reduce resolución espacial de $224\times224$ a $112\times112$ mientras expandiendo canales de 3 (RGB) a 64, seguida de capa de \textit{max pooling} $3\times3$ con \textit{stride} 2 que reduce adicionalmente resolución a $56\times56$, preparando la entrada para los bloques residuales principales. La red procede con cuatro grupos de bloques residuales organizados en \textit{layers} (capas) con profundidad creciente y resolución decreciente:

\textbf{Layer 1:} Dos bloques residuales básicos con 64 canales, resolución espacial $56\times56$. Cada bloque implementa la transformación
\begin{equation}
\mathbf{y} = \text{ReLU}(\text{BN}(\text{Conv}_{3\times3}(\text{ReLU}(\text{BN}(\text{Conv}_{3\times3}(\mathbf{x})))))) + \mathbf{x}
\label{eq:basic_block_resnet18}
\end{equation}
donde BN denota \textit{Batch Normalization} (normalización por lotes) que estandariza activaciones para facilitar entrenamiento \cite{Ioffe2015}, Conv$_{3\times3}$ representa convolución con filtros de $3\times3$, y ReLU es activación \textit{Rectified Linear Unit}. La conexión residual (término $+\mathbf{x}$) permite que el bloque aprenda refinamientos incrementales en lugar de transformación completa, facilitando optimización.

\textbf{Layer 2:} Dos bloques residuales básicos con 128 canales, resolución espacial $28\times28$. El primer bloque de Layer 2 implementa reducción de resolución espacial mediante \textit{stride} 2 en primera convolución y conexión residual con convolución $1\times1$ con \textit{stride} 2 para igualar dimensiones:
\begin{equation}
\mathbf{y} = \text{ReLU}(\text{BN}(\text{Conv}_{3\times3}^{s=2}(\mathbf{x}_1))) + \text{Conv}_{1\times1}^{s=2}(\mathbf{x})
\label{eq:downsample_block}
\end{equation}
donde $s=2$ indica \textit{stride} de 2. Esta arquitectura de reducción progresiva de resolución espacial con expansión de canales implementa jerarquía de representaciones: capas tempranas capturan detalles espaciales finos con pocos canales, capas intermedias representan patrones visuales más abstractos con resolución reducida, y capas finales codifican información semántica de alto nivel en representaciones compactas con muchos canales pero resolución espacial mínima.

\textbf{Layer 3:} Dos bloques residuales básicos con 256 canales, resolución espacial $14\times14$, implementando reducción adicional de resolución mediante mecanismo idéntico a Layer 2.

\textbf{Layer 4:} Dos bloques residuales básicos con 512 canales, resolución espacial $7\times7$. La salida de Layer 4 constituye el mapa de características final del \textit{backbone}, tensor de dimensiones $7\times7\times512$ que codifica representación visual de la imagen de entrada en 512 canales de características con resolución espacial reducida a $7\times7$, reducción de $32\times$ respecto a entrada original de $224\times224$ resultado de cinco operaciones de reducción de resolución (\textit{stride} 2): capa convolucional inicial, \textit{max pooling}, y tres transiciones entre \textit{layers} con \textit{downsampling}.

La arquitectura completa del \textit{backbone} ResNet-18 contiene $|\psi| = 11{,}176{,}512$ parámetros distribuidos en convoluciones, normalizaciones por lotes, y sesgos, constituyendo 96.6\% de los parámetros totales del modelo. Estos parámetros son inicializados con pesos oficiales de PyTorch preentrenados en ImageNet mediante optimización de función de pérdida de clasificación multi-clase sobre 1.2 millones de imágenes durante cientos de épocas, proceso computacionalmente costoso (semanas en GPUs de alto rendimiento) que sería inviable replicar para cada aplicación específica, justificando uso de \textit{transfer learning} que aprovecha este preentrenamiento masivo como inicialización para tareas derivadas.

\subsection{Módulo de Regresión: Mapeo de Características a Coordenadas}
\label{subsec:modulo_regresion}

El módulo de regresión diseñado específicamente para esta tarea mapea el vector de características de 512 dimensiones extraído por el \textit{backbone} a las 30 coordenadas objetivo (15 \textit{landmarks} $\times$ 2 coordenadas por punto). Este módulo reemplaza la capa completamente conectada final de ResNet-18 estándar (originalmente diseñada para clasificación en 1000 categorías) con arquitectura de regresión de tres capas que implementa transformación no lineal progresiva con regularización mediante \textit{dropout} (desactivación estocástica de neuronas durante entrenamiento) para prevenir sobreajuste.

La entrada al módulo de regresión se obtiene mediante \textit{Global Average Pooling} (promediado espacial global) que reduce el mapa de características $7\times7\times512$ a vector de 512 dimensiones mediante promediado sobre dimensiones espaciales:
\begin{equation}
\mathbf{z} = \text{GAP}(\mathbf{F}) = \frac{1}{49} \sum_{i=1}^{7} \sum_{j=1}^{7} \mathbf{F}_{i,j} \in \mathbb{R}^{512}
\label{eq:global_average_pooling}
\end{equation}
donde $\mathbf{F} \in \mathbb{R}^{7\times7\times512}$ es el mapa de características de salida de Layer 4. \textit{Global Average Pooling} constituye alternativa efectiva a capas completamente conectadas tradicionales, reduciendo dramáticamente número de parámetros (49$\times$512 conexiones se reducen a operación libre de parámetros) y proporcionando invarianza a traslaciones espaciales residuales, aunque en este caso la función primaria es dimensional: convertir representación espacial bidimensional a vector unidimensional compatible con capas completamente conectadas subsiguientes.

El módulo de regresión implementa la transformación secuencial:

\textbf{Bloque Completamente Conectado 1:}
\begin{align}
\mathbf{h}_1 &= \text{Dropout}(\mathbf{z}, p=0.5) \label{eq:fc1_dropout} \\
\mathbf{h}_1' &= \mathbf{W}_1 \mathbf{h}_1 + \mathbf{b}_1 \quad \text{donde } \mathbf{W}_1 \in \mathbb{R}^{512\times512}, \mathbf{b}_1 \in \mathbb{R}^{512} \label{eq:fc1_linear} \\
\mathbf{a}_1 &= \text{ReLU}(\mathbf{h}_1') \label{eq:fc1_relu}
\end{align}

La capa completamente conectada 1 mantiene dimensionalidad de 512, permitiendo que la red aprenda representación transformada de características visuales sin reducción prematura de capacidad representacional. \textit{Dropout} con probabilidad $p=0.5$ desactiva aleatoriamente 50\% de las neuronas durante cada iteración de entrenamiento, implementando regularización estocástica que previene co-adaptación de características y mejora generalización \cite{Srivastava2014}. Durante inferencia, \textit{dropout} se desactiva y activaciones se escalan por factor $(1-p)=0.5$ para compensar diferencia entre entrenamiento (50\% neuronas activas en promedio) e inferencia (100\% neuronas activas).

\textbf{Bloque Completamente Conectado 2:}
\begin{align}
\mathbf{h}_2 &= \text{Dropout}(\mathbf{a}_1, p=0.25) \\
\mathbf{h}_2' &= \mathbf{W}_2 \mathbf{h}_2 + \mathbf{b}_2 \quad \text{donde } \mathbf{W}_2 \in \mathbb{R}^{256\times512}, \mathbf{b}_2 \in \mathbb{R}^{256} \\
\mathbf{a}_2 &= \text{ReLU}(\mathbf{h}_2')
\end{align}

La capa completamente conectada 2 reduce dimensionalidad de 512 a 256, comenzando compresión de representación hacia salida de 30 coordenadas. La probabilidad de \textit{dropout} se reduce a $p=0.25$, implementando estrategia de regularización progresivamente decreciente: capas superiores cercanas a características visuales reciben regularización fuerte, capas inferiores cercanas a salida reciben regularización moderada, permitiendo mayor flexibilidad de representación en etapas finales de transformación.

\textbf{Bloque Completamente Conectado 3 (Salida):}
\begin{align}
\mathbf{h}_3 &= \text{Dropout}(\mathbf{a}_2, p=0.125) \\
\mathbf{h}_3' &= \mathbf{W}_3 \mathbf{h}_3 + \mathbf{b}_3 \quad \text{donde } \mathbf{W}_3 \in \mathbb{R}^{30\times256}, \mathbf{b}_3 \in \mathbb{R}^{30} \\
\hat{\mathbf{y}} &= \sigma(\mathbf{h}_3') \label{eq:output_sigmoid}
\end{align}

donde $\sigma(\cdot)$ es la función sigmoide aplicada elemento-a-elemento:
\begin{equation}
\sigma(z) = \frac{1}{1 + e^{-z}} \in (0, 1)
\label{eq:sigmoid}
\end{equation}

La capa completamente conectada 3 proyecta la representación de 256 dimensiones a las 30 coordenadas objetivo. La activación sigmoide final garantiza que todas las coordenadas predichas se encuentren en el rango $(0, 1)$, coincidiendo con el rango de coordenadas \textit{ground truth} normalizadas descrito en la Sección~\ref{sec:pipeline_datos}. La probabilidad de \textit{dropout} en la última capa se reduce a $p=0.125$, aplicando regularización mínima inmediatamente antes de la salida para maximizar expresividad de la predicción final.

El módulo de regresión completo contiene $|\phi| = 262{,}656 + 131{,}328 + 7{,}710 = 401{,}694$ parámetros (excluyendo sesgos en conteo simplificado), correspondiendo al 3.4\% del total de parámetros del modelo. Esta fracción pequeña implica que durante Fase 1 de entrenamiento con \textit{backbone} congelado, solo el 3.4\% de parámetros se optimizan, explicando rapidez de convergencia (aproximadamente 1 minuto para 15 épocas) y memoria GPU limitada requerida.

\subsection{Distribución de Parámetros y Complejidad Computacional}
\label{subsec:distribucion_parametros}

La distribución de parámetros entre componentes arquitecturales informa decisiones sobre estrategia de entrenamiento, particularmente en protocolo de \textit{transfer learning} por fases donde diferentes componentes se optimizan con tasas de aprendizaje diferenciadas o se congelan completamente.

\begin{table}[!ht]
\centering
\caption{Distribución detallada de parámetros entrenables en arquitectura del modelo}
\label{tab:parametros_detallados}
\begin{tabular}{@{}lrr@{}}
\toprule
\textbf{Componente Arquitectural} & \textbf{Número de Parámetros} & \textbf{Porcentaje del Total} \\
\midrule
\multicolumn{3}{l}{\textit{Backbone ResNet-18}} \\
\quad Capa convolucional inicial + BN & 9,472 & 0.08\% \\
\quad Layer 1 (2 bloques, 64 canales) & 147,968 & 1.28\% \\
\quad Layer 2 (2 bloques, 128 canales) & 525,824 & 4.54\% \\
\quad Layer 3 (2 bloques, 256 canales) & 2,099,712 & 18.14\% \\
\quad Layer 4 (2 bloques, 512 canales) & 8,393,728 & 72.50\% \\
\midrule
\textbf{Subtotal Backbone} & \textbf{11,176,512} & \textbf{96.53\%} \\
\midrule
\multicolumn{3}{l}{\textit{Módulo de Regresión}} \\
\quad Capa FC1 (512 $\rightarrow$ 512) & 262,656 & 2.27\% \\
\quad Capa FC2 (512 $\rightarrow$ 256) & 131,328 & 1.13\% \\
\quad Capa FC3 (256 $\rightarrow$ 30) & 7,710 & 0.07\% \\
\midrule
\textbf{Subtotal Módulo Regresión} & \textbf{401,694} & \textbf{3.47\%} \\
\midrule
\textbf{Total Modelo Completo} & \textbf{11,578,206} & \textbf{100.00\%} \\
\bottomrule
\end{tabular}
\end{table}

La Tabla~\ref{tab:parametros_detallados} revela que Layer 4 domina complejidad paramétrica con 72.5\% del total, concentración explicada por número de canales (512) y dos bloques residuales cada uno con múltiples convoluciones de $512\times512$ canales. Esta concentración de parámetros en capas profundas es característica de arquitecturas residuales: las características de alto nivel semántico requieren mayor capacidad representacional que características de bajo nivel (bordes, texturas) que son relativamente universales y compactas.

La complejidad computacional del modelo, medida en operaciones de punto flotante (\textit{FLOPs}), es aproximadamente 1.8 giga-FLOPs por imagen de $224\times224$, cálculo dominado por convoluciones en resoluciones espaciales altas (Layers 1-2) donde aunque número de canales es menor, número de posiciones espaciales es grande (Layer 1: $56\times56 = 3136$ posiciones por canal). En comparación, ResNet-50 requiere aproximadamente 4.1 giga-FLOPs y ResNet-101 requiere 7.8 giga-FLOPs, justificando selección de ResNet-18 como balance entre capacidad y eficiencia para tarea de regresión de coordenadas con \textit{dataset} de tamaño moderado (956 imágenes).

\subsection{Arquitectura Experimental: Integración de Coordinate Attention}
\label{subsec:coordinate_attention}

Como experimento metodológico complementario implementado durante desarrollo del sistema, se evaluó incorporación de mecanismo de atención espacial denominado \textit{Coordinate Attention} \cite{Hou2021}, módulo arquitectural diseñado para mejorar sensibilidad posicional de redes convolucionales mediante descomposición de información espacial en atención horizontal y vertical separadas. La motivación teórica para este experimento surge del reconocimiento que localización precisa de \textit{landmarks} requiere sensibilidad fina a posiciones absolutas en la imagen, aspecto que convoluciones estándar capturan solo implícitamente a través de receptive fields (campos receptivos) que agregan información espacial local sin codificación explícita de coordenadas globales.

El módulo \textit{Coordinate Attention} se inserta entre Layer 4 del \textit{backbone} ResNet-18 y la capa de \textit{Global Average Pooling}, procesando el mapa de características $\mathbf{F} \in \mathbb{R}^{7\times7\times512}$ mediante atención selectiva que amplifica características en posiciones espaciales informativas mientras suprime características en posiciones irrelevantes. El módulo implementa tres operaciones secuenciales:

\textbf{Pooling Direccional:} El mapa de características se agrega separadamente a lo largo de dimensiones horizontal y vertical:
\begin{align}
z^c_h(i) &= \frac{1}{W} \sum_{j=0}^{W-1} F^c(i, j) \quad \text{(pooling horizontal, preserva altura)} \label{eq:coord_attn_h} \\
z^c_w(j) &= \frac{1}{H} \sum_{i=0}^{H-1} F^c(i, j) \quad \text{(pooling vertical, preserva anchura)} \label{eq:coord_attn_w}
\end{align}
donde $H=7$, $W=7$ son dimensiones espaciales, $c$ indexa canales, y las salidas $z_h \in \mathbb{R}^{H\times C}$ y $z_w \in \mathbb{R}^{W\times C}$ codifican perfiles de activación promediados a lo largo de cada fila y columna respectivamente. Esta descomposición direccional captura información posicional sin colapsar completamente estructura espacial como hace \textit{Global Average Pooling} estándar.

\textbf{Codificación Compartida:} Los perfiles direccionales se concatenan y procesan mediante convolución 1D compartida seguida de activación:
\begin{equation}
\mathbf{f} = \text{ReLU}(\text{BN}(\text{Conv}_{1\times1}([\mathbf{z}_h; \mathbf{z}_w])))
\label{eq:coord_attn_shared}
\end{equation}
donde $[\mathbf{z}_h; \mathbf{z}_w]$ denota concatenación y la convolución $1\times1$ reduce canales de 512 a $512/32 = 16$ mediante factor de reducción $r=32$, implementando cuello de botella que fuerza compresión de información posicional en representación compacta.

\textbf{Generación de Atención:} La representación compartida se divide y procesa mediante convoluciones separadas para generar mapas de atención direccionales:
\begin{align}
\mathbf{a}_h &= \sigma(\text{Conv}_{1\times1}(\mathbf{f}_h)) \quad \text{donde } \mathbf{a}_h \in \mathbb{R}^{H\times C} \\
\mathbf{a}_w &= \sigma(\text{Conv}_{1\times1}(\mathbf{f}_w)) \quad \text{donde } \mathbf{a}_w \in \mathbb{R}^{W\times C}
\end{align}
donde $\sigma$ es sigmoide que normaliza atención a $(0,1)$. Los mapas de atención se aplican multiplicativamente al mapa de características original:
\begin{equation}
\mathbf{F}'(i,j,c) = \mathbf{F}(i,j,c) \times \mathbf{a}_h(i,c) \times \mathbf{a}_w(j,c)
\label{eq:coord_attn_apply}
\end{equation}

El modelo con \textit{Coordinate Attention} fue entrenado siguiendo protocolo idéntico a Fase 2 (70 épocas, tasas de aprendizaje diferenciadas, \textit{Wing Loss}), pero los resultados experimentales demostraron que la complejidad arquitectural adicional no proporcionó beneficio medible en métricas de localización. El análisis de estos resultados negativos, detallado en el Capítulo~\ref{cap:resultados}, sugiere que para tarea de regresión de coordenadas con \textit{dataset} de tamaño moderado, la capacidad representacional de ResNet-18 estándar es suficiente y adición de mecanismos de atención introduce riesgo de sobreajuste que contrarresta potenciales beneficios de sensibilidad posicional mejorada. Esta observación es consistente con principio general de parsimonia arquitectural: complejidad adicional solo beneficia cuando capacidad base es insuficiente y datos de entrenamiento son abundantes, condiciones no satisfechas en este trabajo.

La siguiente sección describe el \textit{pipeline} completo de procesamiento de datos que transforma radiografías crudas y coordenadas anotadas en tensores normalizados compatibles con la arquitectura descrita.

\section{Pipeline de preprocesamiento y aumentación de datos}
\label{sec:pipeline_datos}

La Sección~\ref{sec:arquitectura} especificó la arquitectura neuronal implementada: ResNet-18 preentrenada con módulo de regresión especializado, diseñada para procesar tensores normalizados de dimensión $224 \times 224 \times 3$ con estadísticas específicas de ImageNet. La presente sección describe exhaustivamente el \textit{pipeline} (secuencia de procesamiento) completo que transforma radiografías de tórax en formato digital crudo desde su adquisición clínica hasta tensores adecuadamente normalizados para inferencia mediante la red neuronal, preservando simultáneamente las correspondencias geométricas precisas entre coordenadas de \textit{landmarks} (puntos de referencia anatómicos) en espacio de imagen original y espacio de entrada de la red. Este \textit{pipeline} constituye componente crítico de la metodología: transformaciones geométricas incorrectas o normalizaciones inapropiadas comprometirían irremediablemente la capacidad del modelo para localizar estructuras anatómicas con precisión sub-píxel, independientemente de la sofisticación arquitectural o estrategia de entrenamiento implementadas.

El diseño del \textit{pipeline} de datos enfrenta múltiples restricciones simultáneas que deben reconciliarse cuidadosamente. Primero, compatibilidad con arquitecturas preentrenadas en ImageNet: ResNet-18 estándar fue entrenada sobre $1.3 \times 10^6$ imágenes RGB naturales normalizadas con estadísticas específicas (medias $\mu = [0.485, 0.456, 0.406]$ y desviaciones estándar $\sigma = [0.229, 0.224, 0.225]$ por canal), requiriendo conversión de radiografías monocromáticas a representación pseudocromática y normalización coherente con distribución de activaciones esperada por capas convolucionales iniciales \cite{Krizhevsky2012, Raghu2019}. Segundo, preservación de precisión geométrica: cada transformación espacial aplicada a la imagen (redimensionamiento, rotación, reflexión) debe replicarse matemáticamente sobre coordenadas de \textit{landmarks} mediante transformaciones afines inversas correctamente parametrizadas, garantizando correspondencia exacta entre píxeles de entrada y coordenadas de supervisión durante entrenamiento. Tercero, aumentación de variabilidad sin corrupción anatómica: transformaciones de \textit{data augmentation} (aumentación de datos) deben incrementar robustez del modelo ante variabilidad clínica realista (posicionamiento del paciente, diferencias en técnica radiográfica) sin generar configuraciones anatómicamente imposibles que confundirían el aprendizaje de restricciones geométricas. La estrategia implementada, documentada en esta sección, balancea estas restricciones mediante secuencia cuidadosamente ordenada de transformaciones determinísticas y estocásticas \cite{Shorten2019}.

El \textit{pipeline} se estructura en dos etapas funcionalmente distintas. La etapa de preprocesamiento determinístico aplica transformaciones idénticas a todas las muestras tanto en entrenamiento como en inferencia: conversión de espacio de color, redimensionamiento estandarizado, y normalización según estadísticas de ImageNet. Esta etapa garantiza compatibilidad con representaciones preentrenadas y homogeneidad de entrada. La etapa de aumentación estocástica se aplica exclusivamente durante entrenamiento, introduciendo variabilidad controlada mediante transformaciones geométricas (reflexión horizontal, rotación limitada) y fotométricas (ajustes de brillo y contraste) aplicadas aleatoriamente con probabilidades calibradas. Esta separación permite reproducibilidad perfecta en evaluación y validación mientras maximiza variabilidad durante aprendizaje, siguiendo principios establecidos de regularización mediante transformaciones de datos \cite{Shorten2019, Krizhevsky2012}.


\subsection{Preprocesamiento determinístico}
\label{subsec:preprocesamiento_determinista}

El preprocesamiento determinístico transforma radiografías crudas desde formato de adquisición clínica a representación normalizada esperada por ResNet-18, mediante secuencia de tres operaciones aplicadas consistentemente a cada muestra.

\subsubsection{Conversión de espacio de color}
\label{subsubsec:conversion_color}

Las radiografías digitales de tórax en el conjunto de datos descrito en la Sección~\ref{sec:dataset} se almacenan como imágenes monocromáticas de un solo canal de intensidad, codificando información de transmisión de rayos X en escala de grises de 8 bits ($I \in [0, 255]$). Sin embargo, ResNet-18 preentrenada en ImageNet espera tensores tricromáticos de entrada con tres canales RGB $(R, G, B) \in \mathbb{R}^{224 \times 224 \times 3}$, reflejando la naturaleza de las imágenes fotográficas naturales sobre las cuales fue entrenada. Esta incompatibilidad dimensional requiere conversión explícita del espacio de color monocromático al espacio pseudocromático RGB.

La estrategia de conversión implementada replica el canal de intensidad monocromática a través de los tres canales RGB, generando imagen pseudocromática acromática:
\begin{equation}
\label{eq:conversion_rgb}
\begin{aligned}
R(i,j) &= I(i,j), \\
G(i,j) &= I(i,j), \\
B(i,j) &= I(i,j),
\end{aligned}
\end{equation}
donde $I(i,j)$ denota la intensidad del píxel en posición $(i,j)$ de la radiografía monocromática original, y $R(i,j)$, $G(i,j)$, $B(i,j)$ representan los valores de los canales rojo, verde y azul en la representación tricromática resultante. Esta transformación preserva completamente la información radiográfica original (no introduce contenido espurio ni descarta información diagnóstica) mientras satisface la restricción dimensional de arquitecturas preentrenadas en imágenes naturales.

La justificación de esta estrategia simple de replicación de canal, en contraste con esquemas más sofisticados de pseudocolorización basados en \textit{colormaps} (mapas de color especializados para visualización médica), se fundamenta en dos consideraciones. Primero, compatibilidad con normalización de ImageNet: la replicación uniforme garantiza que, tras normalización con estadísticas de ImageNet, las activaciones en capas convolucionales iniciales permanezcan dentro del régimen de valores para el cual los filtros preentrenados fueron optimizados, facilitando \textit{transfer learning} (aprendizaje por transferencia) efectivo \cite{Raghu2019}. Segundo, preservación de linealidad radiométrica: la relación lineal entre intensidad de píxel y atenuación de rayos X, fundamental para interpretación radiográfica, se mantiene sin distorsión no lineal que introduciría \textit{colormaps} complejos. Estudios empíricos sobre \textit{transfer learning} en dominio médico validan esta estrategia, demostrando que replicación simple de canal produce resultados comparables o superiores a esquemas de pseudocolorización elaborados cuando se combina con normalización apropiada \cite{Tajbakhsh2016}.

La conversión de espacio de color se implementa mediante la función \texttt{cv2.cvtColor} de OpenCV \cite{Bradski2000} con parámetro \texttt{cv2.COLOR\_GRAY2RGB}, ejecutada inmediatamente tras carga de imagen desde disco. Esta operación tiene costo computacional despreciable ($\mathcal{O}(N)$ donde $N = 299 \times 299$ es el número de píxeles) y complejidad de memoria $3\times$ respecto a imagen monocromática original.


\subsubsection{Redimensionamiento y transformación de coordenadas}
\label{subsubsec:redimensionamiento}

Las radiografías en el conjunto de datos poseen resolución espacial uniforme de $299 \times 299$ píxeles (Tabla~\ref{tab:dataset_specs} en Sección~\ref{sec:dataset}), mientras que ResNet-18 estándar procesa entradas de dimensión $224 \times 224$ píxeles, resolución establecida como estándar en competencias ImageNet \cite{Krizhevsky2012}. Esta discrepancia dimensional requiere operación de redimensionamiento que reduce resolución espacial mediante interpolación, acompañada de transformación compensatoria de coordenadas de \textit{landmarks} que preserva correspondencias geométricas.

El redimensionamiento de imagen se realiza mediante interpolación bilineal, aplicando transformación de escalamiento uniforme isotrópico:
\begin{equation}
\label{eq:redimensionamiento}
\mathbf{I}'(i',j') = \text{BilinearInterpolation}\left(\mathbf{I}, \frac{i' \cdot W_{orig}}{W_{target}}, \frac{j' \cdot H_{orig}}{H_{target}}\right),
\end{equation}
donde $\mathbf{I}$ denota la imagen RGB de dimensión $W_{orig} \times H_{orig} = 299 \times 299$, $\mathbf{I}'$ es la imagen redimensionada de dimensión $W_{target} \times H_{target} = 224 \times 224$, e $(i', j') \in [0, W_{target}) \times [0, H_{target})$ son coordenadas en espacio de imagen de salida. La interpolación bilineal fue seleccionada por su balance óptimo entre calidad de reconstrucción (superior a interpolación por vecino más cercano) y eficiencia computacional (superior a interpolación bicúbica), siendo estándar en \textit{pipelines} de visión computacional \cite{Bradski2000}.

Dado que el problema formulado es regresión directa de coordenadas normalizadas $(x_k, y_k) \in [0,1]^2$ para cada \textit{landmark} $k \in \{1, \ldots, 15\}$, expresadas como fracciones relativas a dimensiones de imagen, el redimensionamiento espacial no requiere transformación explícita de coordenadas de supervisión. Las coordenadas normalizadas son invariantes ante cambios de escala uniforme:
\begin{equation}
\label{eq:invariancia_coordenadas_normalizadas}
\left(\frac{x_{pixel}}{W}, \frac{y_{pixel}}{H}\right) = \left(\frac{x_{pixel} \cdot s}{W \cdot s}, \frac{y_{pixel} \cdot s}{H \cdot s}\right) \quad \forall s > 0,
\end{equation}
donde $s = W_{target} / W_{orig} = 224/299 \approx 0.749$ es el factor de escalamiento. Esta propiedad constituye ventaja significativa de la formulación de regresión de coordenadas normalizadas respecto a regresión de coordenadas absolutas en píxeles, eliminando necesidad de transformaciones compensatorias complejas ante variaciones en resolución de entrada. La normalización de coordenadas a rango $[0,1]$ fue implementada durante anotación del conjunto de datos (Sección~\ref{sec:dataset}), permitiendo compatibilidad inmediata con múltiples resoluciones de procesamiento.

El redimensionamiento se implementa mediante la función \texttt{cv2.resize} de OpenCV con parámetro de interpolación \texttt{cv2.INTER\_LINEAR}, aplicada tras conversión de espacio de color. La reducción de resolución de $299^2 = 89401$ píxeles a $224^2 = 50176$ píxeles (reducción del 44\%) disminuye sustancialmente demanda computacional de capas convolucionales subsecuentes sin degradación observable de capacidad de localización, dado que la resolución efectiva para tareas de detección de \textit{landmarks} anatómicos en imágenes de $224 \times 224$ permanece suficiente para precisión sub-píxel cuando se combinan representaciones jerárquicas profundas con funciones de pérdida especializadas \cite{Feng2018}.


\subsubsection{Normalización según estadísticas de ImageNet}
\label{subsubsec:normalizacion_imagenet}

La normalización de intensidades de píxeles constituye componente crítico del \textit{pipeline} de preprocesamiento, transformando valores de píxeles RGB desde su rango original $[0, 255]$ (enteros de 8 bits) a distribución centrada y escalada compatible con estadísticas de activación aprendidas por capas convolucionales de ResNet-18 durante preentrenamiento en ImageNet \cite{Krizhevsky2012}.

El procedimiento de normalización se realiza en dos etapas secuenciales. Primero, conversión a rango de punto flotante $[0,1]$ mediante división por 255:
\begin{equation}
\label{eq:normalizacion_rango}
\tilde{\mathbf{I}}(i,j,c) = \frac{\mathbf{I}'(i,j,c)}{255}, \quad c \in \{R, G, B\},
\end{equation}
donde $\mathbf{I}'$ denota la imagen redimensionada con valores enteros en $[0,255]$, y $\tilde{\mathbf{I}}$ representa la imagen en formato de punto flotante. Segundo, estandarización canal-específica mediante sustracción de media y división por desviación estándar de ImageNet:
\begin{equation}
\label{eq:normalizacion_imagenet}
\mathbf{I}_{norm}(i,j,c) = \frac{\tilde{\mathbf{I}}(i,j,c) - \mu_c}{\sigma_c},
\end{equation}
donde $\mu = [\mu_R, \mu_G, \mu_B] = [0.485, 0.456, 0.406]$ y $\sigma = [\sigma_R, \sigma_G, \sigma_B] = [0.229, 0.224, 0.225]$ son las medias y desviaciones estándar computadas sobre el conjunto de entrenamiento de ImageNet ILSVRC-2012 \cite{Krizhevsky2012}, valores estándar utilizados universalmente en \textit{transfer learning} con arquitecturas preentrenadas en ImageNet.

Esta normalización canal-específica garantiza que, para cada canal $c$, la distribución de activaciones de entrada posea media aproximadamente cero y varianza unitaria cuando se promedian sobre el conjunto de datos. Aunque las radiografías pseudocromáticas producidas por replicación de canal (Ecuación~\ref{eq:conversion_rgb}) poseen estadísticas idénticas en los tres canales ($\mu_R = \mu_G = \mu_B$ y $\sigma_R = \sigma_G = \sigma_B$ localmente para cada imagen individual), la aplicación de parámetros de normalización diferenciados por canal de ImageNet introduce asimetría deliberada que mejora compatibilidad con filtros convolucionales preentrenados. Estos filtros aprendieron patrones visuales sensibles a variaciones cromáticas específicas de imágenes naturales, y la normalización canal-específica preserva la estructura de covarianza entre canales que caracteriza el espacio de representación aprendido durante preentrenamiento \cite{Raghu2019}.

La normalización se implementa mediante transformaciones de PyTorch: conversión inicial a tensor con \texttt{torch.from\_numpy}, permutación de dimensiones desde formato OpenCV $(H \times W \times C)$ a formato PyTorch $(C \times H \times W)$ mediante \texttt{permute(2,0,1)}, conversión a punto flotante con \texttt{float()}, división por 255, y aplicación de normalización mediante \texttt{torchvision.transforms.Normalize} con medias y desviaciones estándar especificadas \cite{Paszke2019}. El tensor normalizado resultante $\mathbf{I}_{norm} \in \mathbb{R}^{3 \times 224 \times 224}$ constituye la entrada estándar al modelo durante entrenamiento e inferencia.


\subsection{Aumentación estocástica de datos}
\label{subsec:augmentation}

La aumentación de datos mediante transformaciones estocásticas constituye técnica fundamental de regularización en aprendizaje profundo supervisado, particularmente crítica en dominio médico donde conjuntos de datos anotados son inherentemente limitados por el costo prohibitivo de anotación experta \cite{Shorten2019}. La estrategia implementada aplica transformaciones geométricas y fotométricas aleatorias durante entrenamiento, expandiendo artificialmente la diversidad del conjunto de datos de 956 muestras anotadas (Sección~\ref{sec:dataset}) mediante generación implícita de variantes transformadas de cada radiografía original.

El diseño del protocolo de aumentación enfrenta restricción fundamental impuesta por la naturaleza de la tarea: a diferencia de clasificación de imágenes donde transformaciones como recortes aleatorios (\textit{random crops}) y escalamientos no uniformes son admisibles, la tarea de regresión de \textit{landmarks} requiere que cada transformación geométrica aplicada a la imagen se replique exactamente sobre las coordenadas de supervisión mediante transformación afín inversa matemáticamente consistente. Transformaciones que no preservan correspondencias geométricas (como recortes asimétricos que eliminan \textit{landmarks} del campo de visión) corrompen irremediablemente la supervisión, impidiendo aprendizaje. Esta restricción limita las transformaciones admisibles a aquellas geométricamente invertibles: reflexiones, rotaciones, traslaciones, y escalamientos uniformes cuyos parámetros son conocidos exactamente \cite{Shorten2019}.

El protocolo implementado incorpora tres categorías de transformaciones estocásticas, aplicadas secuencialmente con probabilidades calibradas para balancear incremento de variabilidad contra preservación de realismo anatómico.


\subsubsection{Reflexión horizontal}
\label{subsubsec:flip_horizontal}

La reflexión horizontal constituye transformación de aumentación más frecuentemente aplicada (probabilidad $p_{flip} = 0.70$), explotando la simetría bilateral aproximada de la anatomía torácica humana. Una radiografía de tórax reflejada horizontalmente permanece anatómicamente plausible y diagnósticamente válida, representando simplemente una adquisición con orientación lateral invertida.

La transformación de reflexión horizontal se define matemáticamente como:
\begin{equation}
\label{eq:flip_horizontal_imagen}
\mathbf{I}_{flip}(i, j) = \mathbf{I}(W - 1 - i, j),
\end{equation}
donde $W = 224$ es el ancho de imagen, $(i,j)$ son coordenadas en imagen original, y $(W-1-i, j)$ son coordenadas reflejadas respecto al eje vertical central. Las coordenadas normalizadas de \textit{landmarks} se transforman mediante reflexión correspondiente:
\begin{equation}
\label{eq:flip_horizontal_coordenadas}
x'_k = 1 - x_k, \quad y'_k = y_k, \quad k \in \{1, \ldots, 15\},
\end{equation}
donde $(x_k, y_k) \in [0,1]^2$ son coordenadas normalizadas originales del \textit{landmark} $k$, y $(x'_k, y'_k)$ son coordenadas transformadas tras reflexión.

Adicionalmente, la reflexión horizontal requiere intercambio de identidades entre \textit{landmarks} que forman pares simétricos bilaterales. Como se definió en la Sección~\ref{sec:dataset}, el conjunto de datos incluye cinco pares de \textit{landmarks} bilateralmente simétricos: $\mathcal{P}_{sym} = \{(2,3), (4,5), (6,7), (11,12), (13,14)\}$, correspondientes a estructuras anatómicas emparejadas (ápices pulmonares, ángulos costofrénicos, hilios, etc.). Tras reflexión horizontal, la identidad de \textit{landmarks} emparejados debe intercambiarse para mantener consistencia anatómica:
\begin{equation}
\label{eq:flip_swap_simetricos}
(x'_i, y'_i) \leftrightarrow (x'_j, y'_j) \quad \forall (i,j) \in \mathcal{P}_{sym}.
\end{equation}

La implementación de reflexión horizontal utiliza \texttt{torch.flip} con parámetro \texttt{dims=[2]} para invertir dimensión espacial horizontal del tensor de imagen, y permutación explícita de índices de coordenadas para intercambio de pares simétricos. La alta probabilidad de aplicación ($p = 0.70$) garantiza que el modelo observe tanto configuraciones anatómicas originales como reflejadas con frecuencia balanceada, promoviendo invariancia ante orientación lateral y mejorando capacidad de generalización a variabilidad de posicionamiento clínico.


\subsubsection{Rotación aleatoria limitada}
\label{subsubsec:rotacion}

La rotación aleatoria dentro de rango angular limitado modela variabilidad en posicionamiento del paciente durante adquisición radiográfica, donde ligeras inclinaciones son inevitables en práctica clínica. La transformación se aplica con probabilidad $p_{rot} = 0.30$ (menos frecuente que reflexión para evitar exceso de transformaciones compuestas que degradarían calidad de imagen), muestreando ángulo de rotación uniformemente desde intervalo $\theta \sim \mathcal{U}(-15°, +15°)$.

La transformación de rotación centrada en el centro de imagen se define mediante matriz de rotación afín:
\begin{equation}
\label{eq:matriz_rotacion}
\mathbf{R}(\theta) = \begin{bmatrix}
\cos\theta & -\sin\theta & t_x \\
\sin\theta & \cos\theta & t_y \\
0 & 0 & 1
\end{bmatrix},
\end{equation}
donde las componentes de traslación compensatoria $t_x$ y $t_y$ garantizan rotación centrada en punto $(W/2, H/2)$ de imagen. Las coordenadas normalizadas de \textit{landmarks} se transforman mediante aplicación de rotación inversa centrada en $(0.5, 0.5)$, punto central en espacio de coordenadas normalizadas:
\begin{equation}
\label{eq:rotacion_coordenadas_normalizadas}
\begin{bmatrix}
x'_k - 0.5 \\
y'_k - 0.5
\end{bmatrix} = \begin{bmatrix}
\cos\theta & -\sin\theta \\
\sin\theta & \cos\theta
\end{bmatrix} \begin{bmatrix}
x_k - 0.5 \\
y_k - 0.5
\end{bmatrix}.
\end{equation}

La limitación del rango angular a $\pm 15°$ responde a dos consideraciones. Primero, realismo clínico: rotaciones superiores a 15° son infrecuentes en radiografías de tórax de calidad diagnóstica estándar, dado que protocolos de posicionamiento radiográfico buscan alineación precisa del paciente con el detector. Segundo, preservación de visibilidad de \textit{landmarks}: rotaciones excesivas podrían desplazar \textit{landmarks} periféricos (ápices pulmonares, ángulos costofrénicos) fuera del campo de visión tras rotación, corrompiendo supervisión. El rango de $\pm 15°$ balancea incremento de robustez ante variabilidad de posicionamiento con preservación de validez anatómica.

La implementación utiliza \texttt{torchvision.transforms.functional.affine} para aplicar transformación afín a imagen, y rotación matricial explícita sobre coordenadas normalizadas mediante operaciones tensoriales de PyTorch. La interpolación bilineal se emplea para reconstrucción de imagen rotada, con relleno de regiones externas mediante valor cero (\textit{padding} negro), consistente con fondo típico de radiografías digitales.


\subsubsection{Ajustes fotométricos}
\label{subsubsec:ajustes_fotometricos}

Los ajustes fotométricos modelan variabilidad en técnica radiográfica (variaciones en kilovoltaje pico, miliamperaje-segundo, tiempo de exposición) y procesamiento posterior (ajustes de ventana y nivel en sistemas PACS), que afectan contraste y brillo aparente de radiografías sin alterar geometría anatómica. Estas transformaciones se aplican con probabilidad $p_{photo} = 0.50$, ajustando brillo (mediante suma aditiva) y contraste (mediante multiplicación) dentro de rangos calibrados.

El ajuste de brillo se define como traslación aditiva uniforme de intensidades:
\begin{equation}
\label{eq:ajuste_brillo}
\mathbf{I}_{bright}(i,j,c) = \mathbf{I}(i,j,c) + \beta, \quad \beta \sim \mathcal{U}(-0.2, +0.2),
\end{equation}
donde $\beta$ es el factor de brillo muestreado uniformemente desde intervalo $[-0.2, +0.2]$ en espacio normalizado $[0,1]$. El ajuste de contraste se implementa como escalamiento multiplicativo centrado en intensidad media:
\begin{equation}
\label{eq:ajuste_contraste}
\mathbf{I}_{contrast}(i,j,c) = \alpha \cdot \mathbf{I}(i,j,c), \quad \alpha \sim \mathcal{U}(0.8, 1.2),
\end{equation}
donde $\alpha$ es el factor de contraste muestreado desde intervalo $[0.8, 1.2]$, permitiendo reducción o incremento del 20\% en contraste aparente.

Crucialmente, los ajustes fotométricos no requieren transformación de coordenadas de \textit{landmarks}, dado que preservan completamente la geometría espacial de imagen: ningún píxel cambia de posición, solo su intensidad. Esta propiedad contrasta con transformaciones geométricas (reflexión, rotación) que requieren compensación de coordenadas. Los ajustes fotométricas se aplican tras normalización de ImageNet descrita en la Sección~\ref{subsubsec:normalizacion_imagenet}, operando sobre tensores normalizados antes de ingreso a la red neuronal.

La implementación utiliza \texttt{torchvision.transforms.ColorJitter} con parámetros \texttt{brightness=0.2} y \texttt{contrast=0.2}, aplicados con probabilidad 0.5 mediante envoltura en \texttt{RandomApply}. Los valores de brillo y contraste tras ajuste se recortan (\textit{clipping}) al rango válido de tensores normalizados para evitar valores atípicos extremos que desestabilizarían gradientes durante retropropagación \cite{Paszke2019}.


\subsection{Orden de aplicación y composición de transformaciones}
\label{subsec:orden_transformaciones}

Las transformaciones de preprocesamiento y aumentación descritas en las subsecciones previas se aplican mediante \textit{pipeline} secuencial implementado como composición de funciones en PyTorch \cite{Paszke2019}. El orden de aplicación es crítico: transformaciones no conmutan, y secuencias incorrectas producirían inconsistencias entre imágenes procesadas y coordenadas transformadas.

El \textit{pipeline} completo de entrenamiento aplica transformaciones en el siguiente orden estrictamente especificado:
\begin{enumerate}
    \item Carga de radiografía monocromática desde disco (formato PNG de 8 bits).
    \item Conversión de espacio de color: monocromático $\rightarrow$ RGB pseudocromático (Ecuación~\ref{eq:conversion_rgb}).
    \item Redimensionamiento mediante interpolación bilineal: $299 \times 299 \rightarrow 224 \times 224$ (Ecuación~\ref{eq:redimensionamiento}).
    \item Conversión a tensor PyTorch con permutación de dimensiones.
    \item Normalización a rango $[0,1]$ mediante división por 255.
    \item Normalización según estadísticas de ImageNet (Ecuación~\ref{eq:normalizacion_imagenet}).
    \item Aplicación estocástica de reflexión horizontal con $p=0.70$ (Ecuaciones~\ref{eq:flip_horizontal_imagen}--\ref{eq:flip_swap_simetricos}).
    \item Aplicación estocástica de rotación con $p=0.30$, ángulo $\theta \sim \mathcal{U}(-15°, +15°)$ (Ecuaciones~\ref{eq:matriz_rotacion}--\ref{eq:rotacion_coordenadas_normalizadas}).
    \item Aplicación estocástica de ajustes fotométricos con $p=0.50$ (Ecuaciones~\ref{eq:ajuste_brillo}--\ref{eq:ajuste_contraste}).
\end{enumerate}

Durante inferencia (validación y evaluación en conjunto de prueba), únicamente los pasos determinísticos (1--6) se aplican, omitiendo completamente transformaciones estocásticas de aumentación. Esta separación garantiza reproducibilidad perfecta de predicciones durante evaluación, requisito fundamental para comparación rigurosa de rendimiento entre modelos y reportes de métricas estandarizadas.

La composición de transformaciones estocásticas geométricas (reflexión y rotación) puede aplicarse simultáneamente con probabilidades independientes, produciendo ocasionalmente muestras transformadas mediante ambas operaciones. La probabilidad de aplicación conjunta es $p_{flip} \cdot p_{rot} = 0.70 \times 0.30 = 0.21$ (21\% de muestras), mientras que la probabilidad de al menos una transformación geométrica es $1 - (1-p_{flip})(1-p_{rot}) = 1 - 0.30 \times 0.70 = 0.79$ (79\% de muestras). Esta composición estocástica expande significativamente la diversidad efectiva del conjunto de entrenamiento: cada radiografía original de 956 disponibles puede presentarse en múltiples configuraciones transformadas a lo largo de las épocas de entrenamiento, reduciendo sobreajuste mediante exposición continua a variantes no idénticas \cite{Shorten2019}.


\subsection{Síntesis del pipeline}
\label{subsec:sintesis_pipeline}

El \textit{pipeline} de preprocesamiento y aumentación documentado en esta sección reconcilia exitosamente los requerimientos simultáneos de compatibilidad con arquitecturas preentrenadas (mediante normalización de ImageNet), preservación de precisión geométrica (mediante transformaciones afines matemáticamente consistentes sobre coordenadas de \textit{landmarks}), e incremento de robustez ante variabilidad clínica (mediante aumentación estocástica controlada). La separación funcional entre preprocesamiento determinístico (aplicado uniformemente en entrenamiento e inferencia) y aumentación estocástica (exclusiva de entrenamiento) garantiza reproducibilidad en evaluación mientras maximiza regularización durante aprendizaje, siguiendo principios establecidos de diseño de \textit{pipelines} de visión computacional \cite{Shorten2019, Krizhevsky2012}.

La implementación completa del \textit{pipeline} se encapsula en clase personalizada \texttt{ChestXrayDataset} derivada de \texttt{torch.utils.data.Dataset}, que gestiona carga de imágenes, aplicación de transformaciones, y generación de pares (imagen transformada, coordenadas transformadas) durante iteración de entrenamiento. Esta abstracción modular facilita experimentación con variaciones del protocolo de aumentación y garantiza consistencia de procesamiento a través de todas las fases de entrenamiento descritas en la Sección~\ref{sec:estrategia_entrenamiento}.

La siguiente sección describe exhaustivamente la estrategia de entrenamiento progresivo en cuatro fases que incorpora gradualmente restricciones geométricas inspiradas en conocimiento anatómico, construyendo sobre la arquitectura especificada en la Sección~\ref{sec:arquitectura} y operando sobre datos procesados mediante el \textit{pipeline} documentado en la presente sección.

\section{Estrategia de Entrenamiento Progresivo}
\label{sec:estrategia_entrenamiento}

% Como se fundamentó teóricamente en la Sección~\ref{sec:funciones_perdida} del marco teórico, (NOTA: label no existe en Cap 2 actual)
La incorporación de conocimiento anatómico mediante restricciones geométricas diferenciables constituye un paradigma promisorio para mejorar consistencia estructural de predicciones de \textit{landmarks} en imágenes médicas. La estrategia de entrenamiento desarrollada en este trabajo implementa este paradigma mediante un protocolo progresivo en cuatro fases que incorpora gradualmente funciones de pérdida geométricamente restringidas, comenzando con optimización estándar mediante Error Cuadrático Medio (\textit{Mean Squared Error}, MSE) para establecer una línea base, transitando a \textit{Wing Loss} para mejorar precisión sub-píxel, agregando \textit{Symmetry Loss} para imponer consistencia bilateral, y finalmente incorporando \textit{Distance Preservation Loss} para garantizar proporciones anatómicas válidas. Cada fase se construye sobre la anterior mediante inicialización con pesos óptimos de la fase previa, estrategia de \textit{warm-start} (inicio cálido) que acelera convergencia y previene degradación de desempeño al introducir términos de pérdida adicionales.

Esta organización en fases progresivas, en lugar de entrenamiento directo con la función de pérdida completa desde el inicio, se fundamenta en observaciones empíricas previas sobre dificultad de optimización de funciones de pérdida multi-objetivo complejas: el entrenamiento simultáneo con múltiples términos de pérdida geométrica desde inicialización aleatoria frecuentemente resulta en inestabilidad numérica, convergencia prematura a mínimos locales de calidad inferior, o dificultad en balancear magnitudes relativas de gradientes provenientes de diferentes términos. La incorporación gradual permite al modelo primero establecer predicciones aproximadamente correctas mediante supervisión MSE estándar, luego refinar precisión mediante \textit{Wing Loss} que proporciona gradientes más informativos en régimen de error pequeño, posteriormente mejorar consistencia geométrica mediante \textit{Symmetry Loss}, y finalmente incorporar restricciones de proporciones anatómicas mediante \textit{Distance Preservation Loss}, secuencia que guía la optimización a través de paisaje de pérdida complejo de manera controlada.

\subsection{Fase 1: Entrenamiento del Módulo de Regresión con Backbone Congelado}
\label{sec:phase1_head_training}

La primera fase implementa el protocolo estándar de \textit{transfer learning} en dos etapas: congelar completamente los pesos del \textit{backbone} preentrenado y entrenar únicamente el módulo de regresión añadido, permitiendo que las capas superiores aprendan a mapear representaciones visuales de ImageNet a coordenadas de \textit{landmarks} anatómicos sin perturbar las características de bajo nivel ya aprendidas. Esta estrategia conservadora es particularmente apropiada cuando el \textit{dataset} objetivo es pequeño ($< 1000$ imágenes) y el riesgo de sobreajuste es alto: entrenar todos los 11.6 millones de parámetros desde inicialización aleatoria con solo 669 imágenes de entrenamiento resultaría inevitablemente en memorización de datos de entrenamiento sin capacidad de generalización.

El modelo de Fase 1 se define formalmente como $f_{\theta} = h_{\phi} \circ g_{\psi}$, donde $g_{\psi}: \mathbb{R}^{224 \times 224 \times 3} \to \mathbb{R}^{512}$ representa el \textit{backbone} ResNet-18 que mapea imágenes a vectores de características de 512 dimensiones con parámetros $\psi$ inicializados desde ImageNet y mantenidos fijos ($\nabla_{\psi} \mathcal{L} = 0$ forzado), y $h_{\phi}: \mathbb{R}^{512} \to \mathbb{R}^{30}$ representa el módulo de regresión de tres capas completamente conectadas con parámetros $\phi$ inicializados aleatoriamente mediante inicialización Kaiming \cite{He2015}, esquema que escala pesos iniciales según número de conexiones para garantizar estabilidad numérica durante propagación hacia adelante y retropropagación de gradientes.

\textbf{Configuración de Fase 1:}
\begin{itemize}
    \item \textbf{Función de pérdida:} MSE estándar sobre coordenadas normalizadas
    \begin{equation}
        \mathcal{L}_{MSE}(\hat{\mathbf{y}}, \mathbf{y}) = \frac{1}{30} \sum_{i=1}^{30} (\hat{y}_i - y_i)^2
        \label{eq:mse_loss_phase1}
    \end{equation}
    donde $\hat{\mathbf{y}} \in [0,1]^{30}$ son coordenadas predichas (salida Sigmoid) y $\mathbf{y} \in [0,1]^{30}$ son coordenadas \textit{ground truth} normalizadas.

    \item \textbf{Optimizador:} Adam \cite{Kingma2015} con parámetros estándar ($\beta_1=0.9$, $\beta_2=0.999$, $\epsilon=10^{-8}$)
    \item \textbf{Tasa de aprendizaje:} $\alpha = 1 \times 10^{-3}$ (constante durante toda la fase)
    \item \textbf{Regularización L2:} \textit{Weight decay} $\lambda = 1 \times 10^{-4}$ aplicado solo a parámetros $\phi$ del módulo de regresión
    \item \textbf{Tamaño de batch:} 32 imágenes (compromiso entre eficiencia computacional y estabilidad de gradientes)
    \item \textbf{Número de épocas:} 15 (suficiente para convergencia del módulo de regresión sin sobreajuste)
    \item \textbf{Parámetros entrenables:} $|\phi| \approx 400{,}000$ (solo módulo de regresión), correspondiente al 3.4\% del total
\end{itemize}

El entrenamiento de Fase 1 emplea MSE como función de pérdida inicial por simplicidad y estabilidad: MSE proporciona gradientes bien comportados sin singularidades, superficie de pérdida convexa localmente que facilita convergencia desde inicialización aleatoria, y minimización directa de discrepancia euclidiana entre predicciones y \textit{ground truth}. Aunque MSE presenta limitaciones conocidas para localización sub-píxel (penalización uniforme independiente de magnitud de error, sesgo hacia promedio de distribución en presencia de outliers), estas desventajas son menos críticas en fase inicial donde objetivo es establecer aproximación razonable al mapeo imágenes$\to$coordenadas antes de refinamientos posteriores.

El protocolo específico de Fase 1 procede mediante iteración sobre \textit{mini-batches} de 32 imágenes extraídos aleatoriamente del conjunto de entrenamiento, computando predicciones mediante propagación hacia adelante a través del modelo $\hat{\mathbf{y}}^{(b)} = f_{\theta}(\mathbf{X}^{(b)})$ donde $\mathbf{X}^{(b)} \in \mathbb{R}^{32 \times 3 \times 224 \times 224}$ es el \textit{batch} de imágenes preprocesadas, calculando pérdida MSE promediada sobre el \textit{batch} $\mathcal{L}^{(b)} = \frac{1}{32}\sum_{i=1}^{32} \mathcal{L}_{MSE}(\hat{\mathbf{y}}^{(b)}_i, \mathbf{y}^{(b)}_i)$, computando gradientes mediante retropropagación $\nabla_{\phi} \mathcal{L}^{(b)}$ (solo respecto a parámetros del módulo de regresión, gradientes del \textit{backbone} son descartados), y actualizando parámetros mediante regla de Adam que combina momentum de primer y segundo orden para convergencia estable. Después de procesar todos los \textit{mini-batches} del conjunto de entrenamiento (una época), el modelo se evalúa sobre el conjunto de validación para monitoreo de convergencia, guardando el \textit{checkpoint} (punto de control) del modelo con menor pérdida de validación observada hasta el momento.

El entrenamiento de Fase 1 típicamente completa en aproximadamente 1 minuto en la configuración de \textit{hardware} empleada (GPU AMD RX 6600), tiempo reducido explicado por el pequeño número de parámetros entrenables (400K vs 11.6M totales) y épocas limitadas (15), suficientes para convergencia del módulo de regresión sin necesitar ajuste fino extenso del \textit{backbone}. El modelo resultante establece una línea base funcional que captura correspondencia aproximada entre apariencia visual de radiografías y posiciones de \textit{landmarks}, aunque con precisión limitada debido a las limitaciones inherentes de MSE para localización sub-píxel, aspecto abordado en fases subsiguientes mediante funciones de pérdida especializadas.

\subsection{Fase 2: Fine-Tuning Completo con Wing Loss}
\label{sec:phase2_wing_loss}

La segunda fase desbloquea todos los parámetros del modelo, permitiendo optimización de la arquitectura completa mediante \textit{fine-tuning} que adapta representaciones visuales preentrenadas en ImageNet a características específicas de radiografías de tórax, simultáneamente introduciendo \textit{Wing Loss} como función de pérdida especializada para localización sub-píxel de \textit{landmarks}. % Como se derivó matemáticamente en la Sección~\ref{sec:wing_loss_teoria}, (NOTA: label no existe en Cap 2 actual)
\textit{Wing Loss} \cite{Feng2018} proporciona gradientes más informativos que MSE en régimen de error pequeño mediante transición suave entre comportamiento logarítmico cerca de error cero (gradiente grande, aceleración de convergencia final) y comportamiento lineal para errores grandes (robustez ante \textit{outliers}), característica demostrada empíricamente en detección de \textit{landmarks} faciales y extendida exitosamente a dominio médico en trabajos recientes.

El modelo de Fase 2 mantiene la arquitectura idéntica a Fase 1 ($f_{\theta} = h_{\phi} \circ g_{\psi}$), pero todos los parámetros $\theta = \{\psi, \phi\}$ son ahora entrenables con tasas de aprendizaje diferenciadas: el \textit{backbone} $\psi$ recibe tasa de aprendizaje reducida para preservar parcialmente conocimiento de ImageNet, mientras el módulo de regresión $\phi$ mantiene tasa de aprendizaje estándar para adaptación rápida. Esta estrategia de tasas de aprendizaje diferenciadas implementa el principio de \textit{discriminative learning rates} que reconoce que capas inferiores (características genéricas de bajo nivel) requieren ajuste mínimo, mientras capas superiores (características específicas de tarea) necesitan adaptación sustancial.

\textbf{Configuración de Fase 2:}
\begin{itemize}
    \item \textbf{Función de pérdida:} \textit{Wing Loss} con parámetros $\omega=10.0$, $\epsilon=2.0$
    \begin{equation}
        \mathcal{L}_{wing}(x) = \begin{cases}
            \omega \times \ln\left(1 + \frac{|x|}{\epsilon}\right) & \text{si } |x| < \omega \\
            |x| - C & \text{si } |x| \geq \omega
        \end{cases}
        \label{eq:wing_loss_implemented}
    \end{equation}
    donde $x = \hat{y}_i - y_i$ es el error de predicción por coordenada, $\omega = 10.0$ es el umbral de transición (expresado en escala normalizada $[0,1]$, equivalente a $\approx 2.24$ píxeles en imagen de 224$\times$224), $\epsilon = 2.0$ controla curvatura en régimen logarítmico, y $C = \omega - \omega\ln(1 + \omega/\epsilon) \approx 3.906$ es constante de continuidad. La pérdida total se promedia sobre las 30 coordenadas:
    \begin{equation}
        \mathcal{L}_{total}^{(P2)} = \frac{1}{30} \sum_{i=1}^{30} \mathcal{L}_{wing}(\hat{y}_i - y_i)
        \label{eq:total_loss_phase2}
    \end{equation}

    \item \textbf{Optimizador:} Adam con grupos de parámetros separados
    \begin{itemize}
        \item Parámetros \textit{backbone} $\psi$: $\alpha_{back} = 2 \times 10^{-5}$ (tasa reducida 50×)
        \item Parámetros módulo regresión $\phi$: $\alpha_{head} = 2 \times 10^{-4}$ (tasa estándar)
    \end{itemize}
    \item \textbf{Regularización L2:} \textit{Weight decay} $\lambda = 5 \times 10^{-5}$ (reducido vs Fase 1 para permitir mayor flexibilidad)
    \item \textbf{Scheduler de tasa de aprendizaje:} CosineAnnealingLR \cite{Loshchilov2017} con período $T_{max}=70$ épocas y tasa mínima $\eta_{min} = 2 \times 10^{-6}$, implementando decaimiento suave según
    \begin{equation}
        \alpha_t = \eta_{min} + \frac{1}{2}(\alpha_0 - \eta_{min})\left(1 + \cos\left(\frac{t\pi}{T_{max}}\right)\right)
        \label{eq:cosine_annealing}
    \end{equation}
    donde $t$ es el número de época actual y $\alpha_0$ es la tasa de aprendizaje inicial ($\alpha_{back}$ o $\alpha_{head}$ según grupo de parámetros). Este \textit{scheduler} proporciona decaimiento gradual que facilita convergencia a mínimos de alta calidad.

    \item \textbf{Tamaño de batch:} 8 imágenes (reducido vs Fase 1 para estabilidad al optimizar 11.6M parámetros)
    \item \textbf{Número de épocas:} 70 (entrenamiento extenso para convergencia completa)
    \item \textbf{Early stopping:} Paciencia de 15 épocas sin mejora en pérdida de validación, deteniendo entrenamiento anticipadamente si el modelo deja de mejorar
    \item \textbf{Inicialización:} Pesos $\theta^{init}_{P2} = \theta^{best}_{P1}$ (\textit{warm-start} desde mejor \textit{checkpoint} de Fase 1)
    \item \textbf{Parámetros entrenables:} $|\theta| = 11{,}578{,}206$ (arquitectura completa)
\end{itemize}

La selección de \textit{Wing Loss} sobre alternativas como L1 o Smooth L1 se fundamenta en su comportamiento de gradiente adaptativo: para errores pequeños ($|x| < \omega$), el gradiente $\partial \mathcal{L}_{wing}/\partial x = \frac{\omega}{\epsilon + |x|} \cdot \text{sign}(x)$ escala inversamente con error, proporcionando fuerza mayor cuando la predicción está muy cerca del \textit{ground truth}, acelerando convergencia final a precisión sub-píxel. Para errores grandes ($|x| \geq \omega$), el gradiente se satura a $\text{sign}(x)$, proporcionando robustez ante \textit{outliers} similar a L1. Los valores de hiperparámetros $\omega=10.0$ y $\epsilon=2.0$ fueron establecidos por Feng et al. \cite{Feng2018} basándose en experimentos extensos en detección facial y adoptados aquí sin modificación, constituyendo configuración estándar en literatura de localización de \textit{landmarks}.

El protocolo de Fase 2 implementa entrenamiento estándar con dos grupos de parámetros, donde el optimizador Adam mantiene estadísticas de momentum separadas para cada grupo y aplica tasas de aprendizaje diferenciadas. La reducción de tamaño de \textit{batch} de 32 a 8 es necesaria por limitaciones de memoria GPU (8GB VRAM) al propagar gradientes a través de toda la arquitectura ResNet-18, aunque \textit{batch} de 8 mantiene estimación de gradiente suficientemente estable mediante acumulación de estadísticas de momentum de Adam. El \textit{scheduler} CosineAnnealingLR proporciona decaimiento suave de tasa de aprendizaje que evita oscilaciones en fases finales de entrenamiento, transicionando gradualmente de exploración con tasa alta a refinamiento con tasa baja.

La estrategia de \textit{early stopping} monitorea pérdida de validación después de cada época, manteniendo registro del mejor valor observado y contador de épocas sin mejora. Cuando el contador alcanza paciencia de 15 épocas (aproximadamente 20\% del máximo de 70 épocas), el entrenamiento se detiene anticipadamente, previniendo sobreajuste prolongado al conjunto de entrenamiento. El modelo final de Fase 2 corresponde al checkpoint con menor pérdida de validación, no al modelo de la última época, implementando principio de selección de modelo basada en desempeño de generalización en lugar de ajuste a entrenamiento.

\subsection{Fase 3: Incorporación de Symmetry Loss para Consistencia Bilateral}
\label{sec:phase3_symmetry}

La tercera fase introduce restricciones de simetría bilateral mediante \textit{Symmetry Loss}, función de pérdida geométrica que penaliza inconsistencias entre posiciones de \textit{landmarks} pareados a través del eje mediastínico. Como se fundamentó en la Sección~\ref{subsec:simetria_bilateral}, la simetría bilateral es una invariante anatómica de la estructura torácica que puede explotarse como supervisión adicional: pares de \textit{landmarks} correspondientes a estructuras izquierda-derecha (ápices, hila, bases pulmonares) deben presentar reflexión aproximada respecto al eje vertical definido por el mediastino, restricción que proporciona señal de aprendizaje complementaria a la supervisión de coordenadas punto-a-punto.

La función \textit{Symmetry Loss} implementada compara posiciones predichas de pares simétricos con sus reflexiones esperadas, calculando discrepancia mediante distancia euclidiana. Para cada par $(i,j) \in \mathcal{P}_{sym}$ definido en Ecuación~\ref{eq:pares_simetricos}, se computa la posición esperada del \textit{landmark} derecho $j$ como reflexión del \textit{landmark} izquierdo $i$ a través del eje mediastínico $x = x_{axis}$ dado por Ecuación~\ref{eq:eje_simetria}, y se penaliza desviación de esta predicción. La pérdida se formula bidireccionalmente (izquierda$\to$derecha y derecha$\to$izquierda) para tratar ambos lados simétricamente:

\begin{equation}
\mathcal{L}_{symmetry}(\hat{\mathbf{y}}) = \frac{1}{2|\mathcal{P}_{sym}|} \sum_{(i,j) \in \mathcal{P}_{sym}} \left[ \left\| \hat{\mathbf{p}}_j - \text{Mirror}(\hat{\mathbf{p}}_i, x_{axis}) \right\|_2 + \left\| \hat{\mathbf{p}}_i - \text{Mirror}(\hat{\mathbf{p}}_j, x_{axis}) \right\|_2 \right]
\label{eq:symmetry_loss_phase3}
\end{equation}

donde $\hat{\mathbf{p}}_k = [\hat{x}_k, \hat{y}_k]^T$ representa las coordenadas 2D predichas del \textit{landmark} $k$, $|\mathcal{P}_{sym}| = 5$ es el número de pares simétricos, y la operación de reflexión especular se define como:

\begin{equation}
\text{Mirror}(\mathbf{p}, x_{axis}) = \begin{bmatrix} 2x_{axis} - p_x \\ p_y \end{bmatrix}
\label{eq:mirror_operation}
\end{equation}

reflejando la coordenada horizontal a través de $x = x_{axis}$ mientras preservando la coordenada vertical. El eje $x_{axis}$ se calcula dinámicamente para cada predicción usando Ecuación~\ref{eq:eje_simetria} con las coordenadas predichas de \textit{landmarks} mediastínicos, permitiendo que el eje de simetría se adapte a la imagen específica en lugar de asumir eje fijo en el centro de la imagen, lo cual sería inadecuado para radiografías con rotación o descentrado del paciente.

La función de pérdida total de Fase 3 combina \textit{Wing Loss} con \textit{Symmetry Loss} mediante suma ponderada:

\begin{equation}
\mathcal{L}_{total}^{(P3)} = \mathcal{L}_{wing} + \lambda_{sym} \cdot \mathcal{L}_{symmetry}
\label{eq:total_loss_phase3}
\end{equation}

donde $\lambda_{sym} = 0.3$ es el peso de simetría, seleccionado mediante validación para balancear contribución de restricción geométrica con supervisión de coordenadas directa. El peso $\lambda_{sym} < 1$ indica que \textit{Symmetry Loss} actúa como regularizador que guía predicciones hacia configuraciones anatómicamente consistentes sin dominar la optimización.

\textbf{Configuración de Fase 3:}
\begin{itemize}
    \item \textbf{Función de pérdida:} Combinación \textit{Wing Loss} + \textit{Symmetry Loss} (Ecuación~\ref{eq:total_loss_phase3})
    \item \textbf{Peso de simetría:} $\lambda_{sym} = 0.3$
    \item \textbf{Optimizador, tasas de aprendizaje, \textit{batch size}, \textit{scheduler}:} Idénticos a Fase 2
    \item \textbf{Número de épocas:} 70 (mismo que Fase 2)
    \item \textbf{Early stopping:} Paciencia 15 épocas
    \item \textbf{Inicialización:} $\theta^{init}_{P3} = \theta^{best}_{P2}$ (\textit{warm-start} desde mejor \textit{checkpoint} de Fase 2)
\end{itemize}

La inicialización desde Fase 2 es crítica: comenzar Fase 3 desde el modelo que ya optimiza \textit{Wing Loss} efectivamente permite que \textit{Symmetry Loss} actúe como refinamiento incremental que mejora consistencia geométrica sin necesitar re-aprender mapeo básico imagen$\to$coordenadas. El entrenamiento de Fase 3 típicamente converge más rápidamente que Fase 2 (el \textit{early stopping} frecuentemente termina antes de 70 épocas) debido a la inicialización de alta calidad y naturaleza de refinamiento de la optimización.

\subsection{Fase 4: Complete Loss con Preservación de Distancias Anatómicas}
\label{sec:phase4_complete}

La cuarta y última fase incorpora \textit{Distance Preservation Loss}, función de pérdida que penaliza distorsiones de proporciones anatómicas mediante preservación de distancias euclidianas entre pares específicos de \textit{landmarks} que definen medidas estructurales críticas: altura mediastínica vertical, ancho torácico superior (ápices), ancho torácico medio (hila), ancho torácico inferior (bases). La preservación de estas distancias garantiza que el modelo no solo localice \textit{landmarks} individualmente con precisión, sino que mantenga relaciones geométricas globales consistentes con proporciones anatómicas humanas válidas.

\textit{Distance Preservation Loss} compara distancias euclidianas entre pares de \textit{landmarks} en predicciones con distancias correspondientes en \textit{ground truth}, penalizando discrepancias mediante pérdida L1 sobre diferencias de distancias:

\begin{equation}
\mathcal{L}_{distance}(\hat{\mathbf{y}}, \mathbf{y}) = \frac{1}{|\mathcal{P}_{dist}|} \sum_{(k,\ell) \in \mathcal{P}_{dist}} \left| \left\| \hat{\mathbf{p}}_k - \hat{\mathbf{p}}_\ell \right\|_2 - \left\| \mathbf{p}_k - \mathbf{p}_\ell \right\|_2 \right|
\label{eq:distance_loss}
\end{equation}

donde $\mathcal{P}_{dist}$ es el conjunto de pares de \textit{landmarks} cuyas distancias mutuas deben preservarse:

\begin{equation}
\mathcal{P}_{dist} = \{(0,1),\, (8,9),\, (2,3),\, (4,5),\, (6,7)\}
\label{eq:pares_distancia}
\end{equation}

correspondiendo a: $(0,1)$ altura mediastínica superior, $(8,9)$ eje mediastínico central, $(2,3)$ ancho torácico superior entre ápices, $(4,5)$ ancho torácico medio entre hila, y $(6,7)$ ancho torácico inferior entre bases. La norma L1 en la pérdida proporciona robustez ante \textit{outliers}: distancias individuales anómalas contribuyen linealmente a la pérdida total en lugar de cuadráticamente como en L2, reduciendo influencia de casos patológicos extremos con proporciones anatómicas genuinamente inusuales.

La función de pérdida completa de Fase 4 combina los tres términos mediante suma ponderada:

\begin{equation}
\mathcal{L}_{total}^{(P4)} = \mathcal{L}_{wing} + \lambda_{sym} \cdot \mathcal{L}_{symmetry} + \lambda_{dist} \cdot \mathcal{L}_{distance}
\label{eq:complete_loss}
\end{equation}

donde $\lambda_{sym} = 0.3$ (preservado desde Fase 3) y $\lambda_{dist} = 0.2$ es el peso de preservación de distancias, seleccionado para ser menor que $\lambda_{sym}$ reconociendo que restricciones de distancia son complementarias y menos críticas que restricciones de simetría bilateral para anatomía torácica.

\textbf{Configuración de Fase 4:}
\begin{itemize}
    \item \textbf{Función de pérdida:} \textit{Complete Loss} (Ecuación~\ref{eq:complete_loss})
    \item \textbf{Pesos de restricciones:} $\lambda_{sym} = 0.3$, $\lambda_{dist} = 0.2$
    \item \textbf{Optimizador, tasas de aprendizaje, \textit{batch size}, \textit{scheduler}:} Idénticos a Fases 2-3
    \item \textbf{Número de épocas:} 70
    \item \textbf{Early stopping:} Paciencia 15 épocas
    \item \textbf{Inicialización:} $\theta^{init}_{P4} = \theta^{best}_{P3}$ (\textit{warm-start} desde mejor \textit{checkpoint} de Fase 3)
\end{itemize}

La Fase 4 representa la culminación del entrenamiento progresivo, donde el modelo optimiza simultáneamente precisión de localización punto-a-punto (\textit{Wing Loss}), consistencia bilateral (\textit{Symmetry Loss}), y validez de proporciones anatómicas (\textit{Distance Preservation Loss}). La inicialización desde Fase 3 asegura que el modelo ya satisface restricciones de simetría razonablemente bien al comenzar Fase 4, permitiendo que \textit{Distance Preservation Loss} actúe como refinamiento final que mejora coherencia geométrica global sin desestabilizar convergencia.

\subsection{Estrategia de Warm-Start entre Fases}
\label{subsec:warm_start}

La estrategia de \textit{warm-start} constituye componente esencial del protocolo de entrenamiento progresivo: cada fase se inicializa con los pesos del mejor modelo de la fase inmediatamente anterior, implementando transferencia de conocimiento entre fases que acelera convergencia y previene degradación de desempeño al introducir términos de pérdida adicionales. Formalmente, la inicialización de Fase $k+1$ se define como:

\begin{equation}
\theta^{init}_{P_{k+1}} = \theta^{best}_{P_k}
\label{eq:warm_start}
\end{equation}

donde $\theta^{best}_{P_k} = \arg\min_{\theta} \mathcal{L}_{val}^{(P_k)}(\theta)$ es el conjunto de parámetros que minimizó pérdida de validación durante Fase $k$.

Esta estrategia contrasta con alternativas de entrenamiento desde inicialización aleatoria para cada fase o entrenamiento directo con función de pérdida completa desde el inicio. El entrenamiento desde inicialización aleatoria descartaría todo el conocimiento aprendido en fases previas, requiriendo que cada fase re-aprenda mapeo básico imagen$\to$coordenadas además de optimizar función de pérdida nueva, proceso ineficiente y propenso a convergencia a mínimos locales de calidad inferior. El entrenamiento directo con pérdida completa presenta dificultades de optimización multi-objetivo: los tres términos de pérdida tienen magnitudes y paisajes de gradiente diferentes, y optimización simultánea desde inicialización aleatoria frecuentemente resulta en balance subóptimo donde un término domina gradientes, previniendo que otros términos contribuyan efectivamente al aprendizaje.

La estrategia de \textit{warm-start} progresivo resuelve estos problemas mediante secuenciación cuidadosa: Fase 1 establece aproximación básica mediante MSE simple; Fase 2 refina precisión con \textit{Wing Loss} mientras preserva conocimiento de Fase 1; Fase 3 mejora consistencia geométrica con \textit{Symmetry Loss} sin degradar precisión de Fase 2; y Fase 4 incorpora proporciones anatómicas con \textit{Distance Loss} manteniendo beneficios de fases 2-3. Cada transición representa perturbación incremental de función de pérdida en lugar de cambio abrupto, facilitando adaptación suave del modelo a criterio de optimización progresivamente más complejo.

El protocolo completo de entrenamiento progresivo desde inicialización ImageNet hasta modelo final con \textit{Complete Loss} se resume como:

\begin{equation}
\text{ImageNet} \xrightarrow{\text{Fase 1: MSE, head only}} \theta^{best}_{P1} \xrightarrow{\text{Fase 2: Wing}} \theta^{best}_{P2} \xrightarrow{\text{Fase 3: +Symmetry}} \theta^{best}_{P3} \xrightarrow{\text{Fase 4: +Distance}} \theta^{best}_{P4}
\label{eq:training_pipeline}
\end{equation}

donde $\theta^{best}_{P4}$ constituye el modelo final empleado para evaluación sobre conjunto de prueba y análisis de desempeño presentado en el Capítulo~\ref{cap:resultados}.

La siguiente sección documenta detalles técnicos de implementación, incluyendo \textit{frameworks} de software, configuración de \textit{hardware}, tiempos de entrenamiento, y protocolos de reproducibilidad que permiten replicación independiente de la metodología descrita.

\section{Detalles de implementación y reproducibilidad}
\label{sec:implementacion}

La Sección~\ref{sec:estrategia_entrenamiento} especificó exhaustivamente la estrategia de entrenamiento progresivo en cuatro fases que incorpora gradualmente funciones de pérdida especializadas y restricciones geométricas. La presente sección documenta los detalles técnicos de implementación computacional que permiten reproducibilidad completa y determinística del trabajo: \textit{frameworks} (entornos de desarrollo) y librerías específicas empleadas con versiones exactas, especificaciones de \textit{hardware} utilizado, protocolos de configuración de semillas aleatorias para garantizar determinismo, y tiempos de entrenamiento medidos empíricamente. La transparencia en documentación de implementación constituye requisito fundamental para validación científica de trabajos basados en aprendizaje profundo aplicado a medicina, donde reproducibilidad de resultados es crítica para eventual traducción clínica de sistemas automáticos de análisis de imágenes médicas \cite{Paszke2019}.

La metodología implementada fue diseñada deliberadamente para ejecución en hardware de consumo general accesible, evitando dependencia de infraestructura computacional especializada de alto costo que limitaría reproducibilidad en contextos académicos y clínicos con presupuestos restringidos. El sistema completo opera exitosamente sobre GPU de gama media con 8GB de memoria VRAM, procesador de consumo general, y 16GB de memoria RAM del sistema, configuración disponible ampliamente en estaciones de trabajo estándar y computadoras portátiles de gama media-alta actuales. Esta accesibilidad de \textit{hardware} facilita replicación independiente del trabajo y democratiza acceso a tecnologías de aprendizaje profundo para investigación médica en instituciones con recursos limitados.


\subsection{Frameworks y librerías}
\label{subsec:frameworks}

La implementación se desarrolló íntegramente en lenguaje Python 3.10, ecosistema dominante para investigación y desarrollo en aprendizaje profundo debido a su expresividad sintáctica, abundancia de librerías especializadas de código abierto, y compatibilidad universal con \textit{frameworks} de aprendizaje automático \cite{Paszke2019, Pedregosa2011}. El \textit{stack} (pila) tecnológico completo se compone de cinco librerías fundamentales, cada una cumpliendo funciones especializadas en el \textit{pipeline} de entrenamiento e inferencia.

\subsubsection{PyTorch}
\label{subsubsec:pytorch}

PyTorch 2.0.1 \cite{Paszke2019} constituye el \textit{framework} central de aprendizaje profundo empleado para definición de arquitectura neuronal, implementación de funciones de pérdida personalizadas, cómputo de gradientes mediante diferenciación automática, y optimización de parámetros mediante algoritmos basados en gradiente estocástico. PyTorch fue seleccionado sobre alternativas como TensorFlow por tres ventajas críticas para investigación en aprendizaje profundo médico. Primero, paradigma de ejecución imperativa (\textit{eager execution}) que facilita depuración (\textit{debugging}) y experimentación iterativa mediante ejecución inmediata de operaciones sin construcción previa de grafos computacionales estáticos, permitiendo inspección de activaciones y gradientes en tiempo real durante desarrollo. Segundo, ecosistema robusto de modelos preentrenados en ImageNet mediante \texttt{torchvision.models}, facilitando \textit{transfer learning} (aprendizaje por transferencia) sin necesidad de reimplementación de arquitecturas complejas o descarga manual de pesos preentrenados. Tercero, soporte nativo de GPU mediante aceleración CUDA que permite entrenamiento eficiente de redes profundas en hardware de consumo, con transparencia completa en gestión de transferencias CPU-GPU mediante API \texttt{.to(device)} unificada.

Los módulos específicos de PyTorch empleados incluyen:
\begin{itemize}
    \item \texttt{torch.nn}: Módulo de capas neuronales para construcción de arquitecturas mediante composición de bloques (\texttt{nn.Linear}, \texttt{nn.Conv2d}, \texttt{nn.BatchNorm2d}, \texttt{nn.Dropout}, \texttt{nn.ReLU}, \texttt{nn.Sigmoid}).
    \item \texttt{torch.optim}: Implementaciones de algoritmos de optimización (\texttt{optim.Adam} para Fases 1-2, \texttt{optim.AdamW} para Fases 3-4) con soporte de tasas de aprendizaje diferenciadas por grupo de parámetros.
    \item \texttt{torch.optim.lr\_scheduler}: Programadores de tasa de aprendizaje (\texttt{CosineAnnealingLR} en Fase 2, \texttt{ReduceLROnPlateau} en Fases 3-4) para ajuste adaptativo durante entrenamiento.
    \item \texttt{torch.utils.data}: Abstracción de conjuntos de datos mediante \texttt{Dataset} y carga eficiente mediante \texttt{DataLoader} con \textit{multi-threading} (multiprocesamiento) para preprocesamiento paralelo.
    \item \texttt{torchvision.models}: Modelos preentrenados en ImageNet, específicamente \texttt{resnet18} con pesos \texttt{ResNet18\_Weights.IMAGENET1K\_V1} (versión estándar entrenada sobre ILSVRC-2012).
    \item \texttt{torchvision.transforms}: Transformaciones de aumentación de datos compatibles con tensores (\texttt{Normalize}, \texttt{RandomHorizontalFlip}, \texttt{RandomRotation}, \texttt{ColorJitter}).
\end{itemize}

La versión PyTorch 2.0.1 fue seleccionada por introducir compilador \texttt{torch.compile} que optimiza grafos computacionales dinámicamente mediante técnicas de \textit{just-in-time compilation} (compilación en tiempo de ejecución), reduciendo sobrecarga de interpretación en bucles de entrenamiento sin sacrificar flexibilidad de ejecución imperativa. Aunque el presente trabajo no utiliza \texttt{torch.compile} explícitamente para preservar transparencia de implementación, la compatibilidad con versiones recientes garantiza longevidad del código ante actualizaciones futuras del ecosistema.

\subsubsection{OpenCV}
\label{subsubsec:opencv}

OpenCV 4.8.0 (Open Source Computer Vision Library) \cite{Bradski2000} proporciona funciones optimizadas de procesamiento de imágenes para carga de radiografías desde disco, conversión de espacios de color, y redimensionamiento mediante interpolación bilineal. Las funciones específicas empleadas incluyen:
\begin{itemize}
    \item \texttt{cv2.imread}: Carga de imágenes PNG de 8 bits desde sistema de archivos con decodificación automática de formato.
    \item \texttt{cv2.cvtColor}: Conversión de espacio de color monocromático (\texttt{GRAY}) a pseudocromático RGB (\texttt{RGB}) mediante replicación de canal (Sección~\ref{subsubsec:conversion_color}).
    \item \texttt{cv2.resize}: Redimensionamiento de imágenes de $299 \times 299$ a $224 \times 224$ píxeles mediante interpolación bilineal (\texttt{cv2.INTER\_LINEAR}) con gestión automática de antialiasing (Sección~\ref{subsubsec:redimensionamiento}).
\end{itemize}

OpenCV fue preferida sobre alternativas como Pillow (PIL) por su rendimiento superior en operaciones vectorizadas sobre matrices de píxeles, implementadas en C++ optimizado con soporte de paralelización automática mediante OpenMP y aceleración SIMD (Single Instruction Multiple Data) en procesadores compatibles. La interoperabilidad perfecta entre representaciones de imagen de OpenCV (\texttt{numpy.ndarray}) y tensores de PyTorch (\texttt{torch.Tensor}) mediante \texttt{torch.from\_numpy} facilita integración sin sobrecarga de conversiones costosas.

\subsubsection{NumPy}
\label{subsubsec:numpy}

NumPy 1.24.3 \cite{Harris2020} proporciona estructuras de datos de arreglos multidimensionales (\texttt{numpy.ndarray}) y operaciones algebraicas vectorizadas para manipulación eficiente de coordenadas de \textit{landmarks} (puntos de referencia anatómicos), cómputo de transformaciones geométricas (matrices de rotación, reflexiones), y cálculo de estadísticas descriptivas del conjunto de datos. La representación de coordenadas como arreglos NumPy de forma $(N, 15, 2)$ donde $N$ es tamaño de \textit{batch} (lote), 15 son \textit{landmarks}, y 2 son coordenadas $(x,y)$ permite operaciones vectorizadas de transformación aplicadas simultáneamente sobre todos los \textit{landmarks} y muestras mediante \textit{broadcasting} (difusión) automático, evitando bucles explícitos ineficientes en Python puro.

\subsubsection{scikit-learn}
\label{subsubsec:sklearn}

scikit-learn 1.3.0 \cite{Pedregosa2011} proporciona utilidades de preprocesamiento de datos y división estratificada de conjuntos. La función \texttt{train\_test\_split} se empleó para particionar el conjunto de datos completo de 956 muestras en conjuntos de entrenamiento (70\%), validación (15\%), y prueba (15\%) con estratificación por clase diagnóstica (COVID-19, Viral Pneumonia, Normal), garantizando distribución balanceada de categorías en cada subconjunto como se describe en la Sección~\ref{sec:dataset}. Adicionalmente, \texttt{StandardScaler} se utilizó para verificar estadísticas de normalización del conjunto de datos procesado durante análisis exploratorio previo a entrenamiento.

\subsubsection{Matplotlib}
\label{subsubsec:matplotlib}

Matplotlib 3.7.2 se empleó exclusivamente para visualización de curvas de entrenamiento (pérdida en función de épocas), distribuciones de errores, y análisis cualitativo de predicciones mediante superposición de \textit{landmarks} predichos sobre radiografías originales durante validación. Aunque visualizaciones no constituyen parte del \textit{pipeline} de entrenamiento o inferencia productivo, fueron instrumentales durante desarrollo para diagnóstico de problemas de convergencia, detección de sobreajuste, y validación cualitativa de consistencia anatómica de predicciones antes de evaluación cuantitativa formal.


\subsection{Especificaciones de hardware y configuración computacional}
\label{subsec:hardware}

El entrenamiento completo de las cuatro fases metodológicas se ejecutó sobre estación de trabajo de consumo general con las siguientes especificaciones técnicas:

\begin{itemize}
    \item \textbf{GPU}: AMD Radeon RX 6600 con 8GB de memoria VRAM GDDR6, arquitectura RDNA 2, 1792 procesadores de flujo, frecuencia base 1626 MHz, frecuencia máxima 2491 MHz, ancho de banda de memoria 224 GB/s. Soporte de aceleración mediante ROCm 5.6 (Radeon Open Compute) con \textit{backend} PyTorch compatible.
    \item \textbf{CPU}: AMD Ryzen 5 5600G, 6 núcleos / 12 hilos, frecuencia base 3.9 GHz, frecuencia máxima 4.4 GHz, caché L3 de 16MB. Utilizado para preprocesamiento de datos mediante \textit{multi-threading} en \texttt{DataLoader} (4 \textit{workers} paralelos).
    \item \textbf{RAM}: 16GB DDR4 3200MHz, suficiente para almacenamiento en memoria del conjunto de datos completo de imágenes redimensionadas (956 muestras $\times$ 224 $\times$ 224 $\times$ 3 canales $\times$ 4 bytes/flotante $\approx$ 578 MB) y estructuras auxiliares de entrenamiento.
    \item \textbf{Almacenamiento}: SSD NVMe de 512GB, garantizando latencia mínima en carga de imágenes desde disco durante iteración de \textit{batches}. Tiempo de carga de conjunto de datos completo: $< 3$ segundos.
    \item \textbf{Sistema Operativo}: Ubuntu 22.04.3 LTS con kernel Linux 6.2.0, proporcionando estabilidad de entorno y compatibilidad con \textit{drivers} de GPU de código abierto AMDGPU.
\end{itemize}

La configuración de GPU AMD RX 6600 representa hardware de gama media accesible (precio de mercado aproximado USD \$250 al momento de desarrollo), demostrando viabilidad de entrenamiento de sistemas de detección de \textit{landmarks} basados en ResNet-18 sin necesidad de GPUs profesionales de alto costo como NVIDIA A100 o V100. La memoria VRAM de 8GB permitió entrenamiento con \textit{batch size} (tamaño de lote) de hasta 32 muestras en Fase 1 (entrenamiento de cabeza con \textit{backbone} congelado) y 8 muestras en Fases 2-4 (\textit{fine-tuning} completo con mayor demanda de memoria por almacenamiento de gradientes en todas las capas). Estos tamaños de \textit{batch} balancean eficiencia computacional (utilización óptima de paralelismo de GPU) con estabilidad de gradientes estocásticos (varianza suficientemente baja para convergencia confiable).

La utilización de GPU AMD mediante \textit{backend} ROCm en lugar de NVIDIA CUDA responde a disponibilidad de hardware y compromiso con ecosistemas de código abierto, demostrando independencia de implementación respecto a fabricante específico de aceleradores. La compatibilidad de PyTorch con múltiples \textit{backends} (CUDA, ROCm, MPS para Apple Silicon) mediante abstracción unificada \texttt{torch.device} garantiza portabilidad completa del código a diferentes plataformas de hardware sin modificaciones algorítmicas.


\subsection{Tiempos de entrenamiento}
\label{subsec:tiempos_entrenamiento}

Los tiempos de entrenamiento medidos empíricamente para cada fase metodológica se presentan en la Tabla~\ref{tab:tiempos_entrenamiento}. Estos tiempos incluyen cómputo de \textit{forward pass} (paso hacia adelante) y \textit{backward pass} (retropropagación), actualización de parámetros, evaluación periódica sobre conjunto de validación, y guardado de \textit{checkpoints} (puntos de control) de modelo tras cada época.

\begin{table}[htbp]
\centering
\caption{Tiempos de entrenamiento por fase metodológica medidos sobre hardware especificado en Sección~\ref{subsec:hardware}. Tiempo por época incluye entrenamiento sobre 669 muestras de entrenamiento, validación sobre 143 muestras, y operaciones de almacenamiento.}
\label{tab:tiempos_entrenamiento}
\begin{tabular}{lccc}
\hline
\textbf{Fase} & \textbf{Épocas} & \textbf{Tiempo/época} & \textbf{Tiempo total} \\
\hline
Fase 1: Entrenamiento de cabeza & 15 & 48 seg & 12 min \\
Fase 2: \textit{Fine-tuning} con \textit{Wing Loss} & 70 & 3 min 12 seg & 3.7 horas \\
Fase 3: Incorporación de \textit{Symmetry Loss} & 50 & 3 min 18 seg & 2.8 horas \\
Fase 4: \textit{Loss} completa con distancias & 40 & 3 min 15 seg & 2.2 horas \\
\hline
\textbf{Total acumulado} & \textbf{175} & --- & \textbf{8.7 horas} \\
\hline
\end{tabular}
\end{table}

El tiempo total de entrenamiento acumulado de aproximadamente 8.7 horas demuestra factibilidad de desarrollo iterativo y experimentación rápida. Este tiempo permite ejecución de ciclo completo de entrenamiento (cuatro fases secuenciales) en menos de una jornada laboral, facilitando exploración de variaciones metodológicas (diferentes pesos de funciones de pérdida, hiperparámetros de regularización, estrategias de programación de tasa de aprendizaje) mediante experimentación sistemática. La eficiencia temporal contrasta favorablemente con reportes de entrenamiento de modelos de localización de \textit{landmarks} basados en generación de \textit{heatmaps} (mapas de calor espaciales), que típicamente requieren múltiples días de entrenamiento en GPUs de mayor capacidad debido a decodificación espacial costosa y predicción de representaciones de alta dimensión \cite{Payer2016}.

La Fase 1 (entrenamiento de cabeza) exhibe tiempo por época significativamente menor (48 segundos) respecto a fases subsecuentes (3 minutos 12-18 segundos) debido a tres factores. Primero, menor volumen de parámetros optimizados: únicamente 400K parámetros del módulo de regresión se actualizan, mientras que 11.2M parámetros del \textit{backbone} permanecen congelados, reduciendo cómputo de gradientes y operaciones de actualización. Segundo, mayor tamaño de \textit{batch}: 32 muestras por iteración en Fase 1 versus 8 muestras en fases posteriores, resultando en menor número de iteraciones por época (669/32 = 21 iteraciones versus 669/8 = 84 iteraciones). Tercero, ausencia de programadores de tasa de aprendizaje complejos y funciones de pérdida geométricas adicionales que introducen sobrecarga computacional en fases avanzadas.


\subsection{Protocolos de reproducibilidad}
\label{subsec:reproducibilidad}

La reproducibilidad determinística completa de resultados constituye requisito fundamental para validación científica rigurosa de trabajos en aprendizaje profundo. El entrenamiento de redes neuronales mediante optimización estocástica inherentemente involucra múltiples fuentes de aleatoriedad: inicialización de pesos, orden de presentación de muestras mediante \textit{shuffling} (barajado) de \textit{batches}, operaciones estocásticas de \textit{dropout}, y muestreo de transformaciones de aumentación de datos. Sin control estricto de semillas aleatorias, ejecuciones independientes del mismo código producen resultados numéricos diferentes, impidiendo reproducibilidad exacta de métricas reportadas.

El protocolo de reproducibilidad implementado fija semillas de todos los generadores de números pseudoaleatorios empleados en el \textit{pipeline} de entrenamiento, garantizando que ejecuciones subsecuentes sobre el mismo hardware y software produzcan trayectorias de optimización idénticas bit a bit. El código de inicialización ejecutado antes de cualquier operación aleatoria establece:

\begin{verbatim}
import torch
import numpy as np
import random

SEED = 42

# Semilla de generador de Python estándar
random.seed(SEED)

# Semilla de NumPy para operaciones vectorizadas
np.random.seed(SEED)

# Semilla de PyTorch para CPU
torch.manual_seed(SEED)

# Semilla de PyTorch para GPU (si disponible)
if torch.cuda.is_available():
    torch.cuda.manual_seed(SEED)
    torch.cuda.manual_seed_all(SEED)

# Configuración de determinismo en operaciones de PyTorch
torch.backends.cudnn.deterministic = True
torch.backends.cudnn.benchmark = False
\end{verbatim}

La semilla maestra $\texttt{SEED} = 42$ fue seleccionada arbitrariamente pero fijada consistentemente a través de todos los experimentos. Las configuraciones \texttt{cudnn.deterministic = True} y \texttt{cudnn.benchmark = False} fuerzan algoritmos determinísticos en operaciones de convolución aceleradas por cuDNN (CUDA Deep Neural Network library), sacrificando marginal rendimiento (aproximadamente 5-10\% de sobrecarga temporal) a cambio de reproducibilidad perfecta. Sin estas configuraciones, cuDNN selecciona heurísticamente algoritmos de convolución optimizados que pueden producir resultados numéricamente diferentes debido a orden no determinístico de operaciones de punto flotante en paralelización masiva de GPU.

Adicionalmente, el \texttt{DataLoader} de PyTorch se configura con \texttt{worker\_init\_fn} personalizada que inicializa semillas de procesos \textit{worker} de preprocesamiento paralelo de forma determinística:

\begin{verbatim}
def worker_init_fn(worker_id):
    np.random.seed(SEED + worker_id)
    random.seed(SEED + worker_id)

train_loader = DataLoader(
    train_dataset,
    batch_size=BATCH_SIZE,
    shuffle=True,
    num_workers=4,
    worker_init_fn=worker_init_fn
)
\end{verbatim}

Esta configuración garantiza que transformaciones estocásticas de aumentación de datos aplicadas en procesos paralelos produzcan secuencias idénticas en ejecuciones repetidas. El parámetro \texttt{shuffle=True} baraja el conjunto de entrenamiento al inicio de cada época, pero el orden de barajado es determinístico dado que el generador de PyTorch fue inicializado con semilla fija.

El protocolo descrito permite reproducción exacta de todos los resultados reportados en el Capítulo~\ref{cap:resultados}, requisito crítico para verificación independiente y auditoría de trabajos en aprendizaje automático médico donde decisiones clínicas pueden depender de predicciones de modelos. La documentación exhaustiva de versiones de \textit{software}, especificaciones de \textit{hardware}, y configuraciones de semillas aleatorias constituye práctica esencial de ciencia reproducible en era de métodos computacionales intensivos.


\subsection{Gestión de experimentos y checkpoints}
\label{subsec:checkpoints}

La gestión sistemática de experimentos y almacenamiento de \textit{checkpoints} (puntos de control) de modelos durante entrenamiento facilita recuperación ante interrupciones, análisis retrospectivo de trayectorias de entrenamiento, y selección de modelo óptimo basado en rendimiento en conjunto de validación. El sistema implementado almacena \textit{checkpoints} tras cada época de entrenamiento, incluyendo:

\begin{itemize}
    \item Estado completo del modelo: diccionario \texttt{model.state\_dict()} conteniendo valores de todos los parámetros entrenables (11.6M parámetros de ResNet-18 modificada).
    \item Estado del optimizador: diccionario \texttt{optimizer.state\_dict()} conteniendo momentos acumulados de Adam/AdamW necesarios para reanudar optimización desde época específica sin perturbación de dinámica de convergencia.
    \item Número de época actual, permitiendo continuación exacta de entrenamiento tras interrupción.
    \item Métricas de entrenamiento y validación: pérdida de entrenamiento, pérdida de validación, y Error Radial Medio (MRE) en conjunto de validación para época actual.
\end{itemize}

Los \textit{checkpoints} se almacenan en formato \texttt{.pth} de PyTorch mediante serialización con \texttt{torch.save}, organizados en estructura de directorios jerárquica por fase de entrenamiento:

\begin{verbatim}
checkpoints/
├── phase1_head_training/
│   ├── epoch_01.pth
│   ├── epoch_02.pth
│   └── ...
├── phase2_wing_loss/
├── phase3_symmetry_loss/
└── phase4_complete_loss/
    └── best_model.pth  # Mejor modelo según validación
\end{verbatim}

La estrategia de \textit{early stopping} (detención temprana) implementada en Fases 2-4 (Sección~\ref{sec:estrategia_entrenamiento}) monitorea pérdida de validación tras cada época, almacenando \textit{checkpoint} especial \texttt{best\_model.pth} cuando se observa nuevo mínimo. Este \textit{checkpoint} contiene el estado de modelo con mejor rendimiento en validación, utilizado para evaluación final en conjunto de prueba (Sección~\ref{subsec:validacion}) y para inicialización de fase subsecuente mediante \textit{warm-start}. La paciencia de 15 épocas en Fase 2 y 10 épocas en Fases 3-4 permite fluctuaciones temporales de pérdida de validación sin detención prematura, balanceando eficiencia de entrenamiento con exploración exhaustiva de espacio de parámetros.

El tamaño de almacenamiento de cada \textit{checkpoint} es aproximadamente 45 MB (11.6M parámetros $\times$ 4 bytes/flotante), resultando en demanda total de aproximadamente 7.9 GB para almacenamiento de todas las épocas de las cuatro fases (175 épocas). Esta demanda es manejable en sistemas de almacenamiento modernos, y permite análisis retrospectivo completo de dinámica de entrenamiento mediante carga de \textit{checkpoints} intermedios para visualización de curvas de aprendizaje y diagnóstico de fenómenos de convergencia.


\subsection{Síntesis de implementación}
\label{subsec:sintesis_implementacion}

Los detalles de implementación documentados en esta sección garantizan reproducibilidad completa del sistema desarrollado: especificaciones exactas de versiones de \textit{software}, configuraciones de \textit{hardware} accesible, protocolos determinísticos de semillas aleatorias, y mediciones empíricas de tiempos de entrenamiento permiten replicación independiente del trabajo en entornos computacionales diversos. La viabilidad de entrenamiento completo en menos de 9 horas sobre GPU de consumo general demuestra accesibilidad de metodologías basadas en aprendizaje profundo para investigación médica en instituciones con recursos limitados, facilitando democratización de tecnologías avanzadas de análisis de imágenes médicas \cite{Paszke2019}.

La siguiente sección define formalmente las métricas de evaluación empleadas para cuantificar rendimiento del sistema desarrollado, estableciendo criterios objetivos de calidad clínica basados en estándares internacionales de precisión en detección de \textit{landmarks} anatómicos.

\section{Métricas de evaluación}
\label{sec:metricas_eval}

La Sección~\ref{sec:implementacion} documentó detalles técnicos de reproducibilidad computacional que garantizan replicabilidad determinística del entrenamiento. La presente sección define formalmente las métricas de evaluación empleadas para cuantificar rendimiento del sistema desarrollado, estableciendo criterios objetivos que permiten comparación rigurosa con trabajos previos en detección automática de \textit{landmarks} (puntos de referencia anatómicos) y valoración de idoneidad clínica del sistema. La definición matemática precisa de métricas constituye componente fundamental de metodología científica rigurosa: sin especificación formal inequívoca, reportes numéricos de rendimiento carecen de interpretabilidad y comparabilidad, impidiendo reproducción y validación independiente de resultados \cite{Payer2016}.

El diseño del sistema de evaluación implementa jerarquía de métricas complementarias que caracterizan diferentes aspectos de calidad de predicciones. La métrica primaria, Error Radial Medio (MRE, \textit{Mean Radial Error}), cuantifica precisión de localización espacial promedio, constituyendo estándar universal en literatura de detección de \textit{landmarks} anatómicos que permite comparación directa con trabajos previos \cite{Payer2016, Feng2018, Ibragimov2017}. Las métricas geométricas complementarias evalúan consistencia estructural de predicciones: error de simetría bilateral, preservación de distancias anatómicas críticas, y validez de ordenamiento espacial fisiológico. Estas métricas capturan aspectos de coherencia anatómica que métricas de error puntual aisladas no detectan, siendo críticas para valoración de aceptabilidad clínica. Finalmente, el sistema de clasificación por umbrales de calidad clínica traduce mediciones continuas de error a categorías discretas interpretables por profesionales médicos (excelente, bueno, aceptable, inaceptable), facilitando comunicación de capacidades del sistema a audiencias no técnicas y estableciendo criterios de decisión para aprobación de uso clínico.


\subsection{Error Radial Medio (MRE)}
\label{subsec:mre}

El Error Radial Medio (MRE, \textit{Mean Radial Error}) constituye la métrica primaria de evaluación, cuantificando precisión de localización espacial mediante distancia euclidiana promedio entre coordenadas predichas y anotaciones de referencia expertas. El MRE se define formalmente como:

\begin{equation}
\label{eq:mre_definicion}
\text{MRE} = \frac{1}{N \cdot K} \sum_{i=1}^{N} \sum_{k=1}^{K} \sqrt{(x_{i,k}^{pred} - x_{i,k}^{gt})^2 + (y_{i,k}^{pred} - y_{i,k}^{gt})^2} \cdot s_i,
\end{equation}

donde:
\begin{itemize}
    \item $N$ es el número total de muestras en conjunto de evaluación (143 en validación, 144 en prueba).
    \item $K = 15$ es el número de \textit{landmarks} anatómicos por radiografía.
    \item $(x_{i,k}^{pred}, y_{i,k}^{pred}) \in [0,1]^2$ son coordenadas normalizadas predichas por el modelo para \textit{landmark} $k$ en muestra $i$.
    \item $(x_{i,k}^{gt}, y_{i,k}^{gt}) \in [0,1]^2$ son coordenadas normalizadas de referencia (\textit{ground truth}) anotadas por experto.
    \item $s_i = 224$ píxeles es el factor de escala que convierte coordenadas normalizadas a coordenadas absolutas en píxeles en espacio de imagen de entrada a la red ($224 \times 224$ píxeles).
\end{itemize}

La formulación expresa error en unidades de píxeles absolutas en lugar de coordenadas normalizadas, facilitando interpretabilidad física directa: un MRE de 5.0 píxeles indica que, en promedio, predicciones del modelo se localizan a 5 píxeles de distancia de posiciones verdaderas en imágenes de $224 \times 224$ píxeles. Esta convención de reportar errores en píxeles constituye estándar universal en literatura de detección de \textit{landmarks}, permitiendo comparación directa con trabajos previos independientemente de resolución de imagen específica empleada por cada método \cite{Payer2016, Feng2018}.

El MRE posee propiedades estadísticas deseables como métrica de localización. Primero, invariancia ante permutaciones de muestras o \textit{landmarks}: el orden de suma no afecta valor final. Segundo, sensibilidad uniforme a errores en todas direcciones espaciales: distancia euclidiana penaliza desplazamientos horizontales y verticales equitativamente mediante norma $L_2$. Tercero, interpretabilidad intuitiva: errores mayores contribuyen proporcionalmente más al promedio, y un MRE de cero indica concordancia perfecta entre predicciones y referencia.

La limitación principal del MRE como métrica aislada es insensibilidad a patrones de error: un modelo que distribuye errores uniformemente entre todos los \textit{landmarks} produce el mismo MRE que un modelo con errores concentrados en subconjunto específico de \textit{landmarks} difíciles, aunque el segundo pueda ser más útil clínicamente si localiza correctamente estructuras críticas (por ejemplo, carinas, ápices pulmonares) aunque falle en estructuras auxiliares. Esta limitación motiva introducción de métricas complementarias que caracterizan distribución espacial y consistencia estructural de errores.


\subsection{Error por landmark individual}
\label{subsec:error_por_landmark}

El análisis de error desagregado por \textit{landmark} individual proporciona diagnóstico detallado de capacidades y limitaciones del modelo, identificando estructuras anatómicas específicas que presentan mayor dificultad de localización. El error radial medio por \textit{landmark} $k$ se define como:

\begin{equation}
\label{eq:mre_por_landmark}
\text{MRE}_k = \frac{1}{N} \sum_{i=1}^{N} \sqrt{(x_{i,k}^{pred} - x_{i,k}^{gt})^2 + (y_{i,k}^{pred} - y_{i,k}^{gt})^2} \cdot s_i.
\end{equation}

Esta métrica permite identificación de patrones sistemáticos de error. Por ejemplo, estructuras de alto contraste bien definidas como carinas traqueales típicamente exhiben $\text{MRE}_k < 3$ píxeles, mientras que estructuras difusas de bajo contraste como ángulos costofrénicos pueden presentar $\text{MRE}_k > 8$ píxeles debido a ambigüedad anatómica inherente. El análisis por \textit{landmark} informa decisiones de refinamiento arquitectural: errores concentrados en \textit{landmarks} específicos sugieren necesidad de mecanismos de atención espacial que enfoquen capacidad representacional en regiones de interés \cite{Hou2021}, mientras que errores uniformemente distribuidos indican limitaciones de capacidad global del modelo que requieren aumento de profundidad o anchura arquitectural.

La desviación estándar del error por \textit{landmark} cuantifica variabilidad de predicciones:
\begin{equation}
\label{eq:std_por_landmark}
\sigma_k = \sqrt{\frac{1}{N} \sum_{i=1}^{N} \left(\sqrt{(x_{i,k}^{pred} - x_{i,k}^{gt})^2 + (y_{i,k}^{pred} - y_{i,k}^{gt})^2} \cdot s_i - \text{MRE}_k\right)^2}.
\end{equation}

Desviaciones estándar elevadas ($\sigma_k > 0.5 \times \text{MRE}_k$) indican predicciones inconsistentes con alta variabilidad caso-a-caso, sugiriendo sensibilidad excesiva a variaciones en apariencia de imagen (artefactos, patologías superpuestas) que degradan robustez clínica. Desviaciones estándar bajas indican predicciones estables, propiedad deseable para confiabilidad en aplicaciones médicas.


\subsection{Métricas de consistencia geométrica}
\label{subsec:metricas_geometricas}

Las métricas de consistencia geométrica evalúan validez estructural de configuraciones predichas de \textit{landmarks}, cuantificando adherencia a restricciones anatómicas fundamentales: simetría bilateral del tórax, preservación de distancias entre estructuras emparejadas, y ordenamiento espacial fisiológico. Estas métricas capturan aspectos de coherencia anatómica críticos para aceptabilidad clínica que el MRE, enfocado exclusivamente en precisión de localización puntual, no detecta.


\subsubsection{Error de simetría bilateral}
\label{subsubsec:error_simetria}

El error de simetría bilateral cuantifica violaciones de la restricción de simetría aproximada del tórax humano, formalizada mediante la función de pérdida de simetría $\mathcal{L}_{sym}$ introducida en la Fase 3 de entrenamiento (Sección~\ref{sec:phase3_symmetry}). La métrica de error de simetría se define como:

\begin{equation}
\label{eq:metrica_simetria}
E_{sym} = \frac{1}{N \cdot |\mathcal{P}_{sym}|} \sum_{i=1}^{N} \sum_{(j,k) \in \mathcal{P}_{sym}} \left| |x_{i,j}^{pred} - x_{axis}| - |x_{i,k}^{pred} - x_{axis}| \right| \cdot s_i,
\end{equation}

donde $\mathcal{P}_{sym} = \{(2,3), (4,5), (6,7), (11,12), (13,14)\}$ es el conjunto de cinco pares de \textit{landmarks} bilateralmente simétricos (ápices pulmonares, ángulos costofrénicos, hilios, etc.), y $x_{axis} = 0.5$ es la coordenada horizontal normalizada del eje de simetría mediastínico central (Sección~\ref{subsec:simetria_bilateral}). La métrica cuantifica discrepancia promedio en distancias de \textit{landmarks} emparejados respecto al eje central, expresada en píxeles.

Valores bajos de error de simetría ($E_{sym} < 3$ píxeles) indican configuraciones predichas que respetan simetría bilateral, sugiriendo que el modelo ha internalizado restricciones anatómicas geométricas durante entrenamiento. Valores elevados ($E_{sym} > 6$ píxeles) indican predicciones asimétricas anatómicamente implausibles, frecuentemente causadas por confusión entre estructuras bilaterales homólogas (por ejemplo, intercambio de ápice pulmonar izquierdo y derecho) o sensibilidad excesiva a asimetrías patológicas reales (derrames pleurales unilaterales, neumonías lobares) que deben distinguirse cuidadosamente de errores de predicción.

La métrica de simetría complementa el MRE: un modelo puede exhibir MRE bajo mediante predicciones precisas en promedio pero violar simetría sistemáticamente (por ejemplo, consistentemente desplazando estructuras derechas hacia el centro), produciendo configuraciones anatómicamente inválidas. La incorporación explícita de $\mathcal{L}_{sym}$ en función de pérdida de entrenamiento (Fase 3) busca minimizar $E_{sym}$ simultáneamente con MRE, optimizando tanto precisión como coherencia estructural.


\subsubsection{Error de preservación de distancias}
\label{subsubsec:error_distancias}

El error de preservación de distancias cuantifica violaciones de restricciones de proporciones anatómicas, evaluando cuán fielmente las configuraciones predichas mantienen distancias relativas entre \textit{landmarks} observadas en anotaciones de referencia. La métrica se fundamenta en el conjunto de pares críticos de \textit{landmarks} $\mathcal{D}_{critical}$ definido en la Fase 4 de entrenamiento (Sección~\ref{sec:phase4_complete}), que incluye distancias anatómicamente significativas como altura pulmonar (distancia vertical entre ápices y bases), amplitud torácica (distancia horizontal entre ángulos costofrénicos), y dimensiones mediastínicas.

La métrica de error de preservación de distancias se define como error relativo porcentual promedio:

\begin{equation}
\label{eq:metrica_distancias}
E_{dist} = \frac{100\%}{N \cdot |\mathcal{D}_{critical}|} \sum_{i=1}^{N} \sum_{(j,k) \in \mathcal{D}_{critical}} \left| \frac{d_{i,jk}^{pred} - d_{i,jk}^{gt}}{d_{i,jk}^{gt}} \right|,
\end{equation}

donde:
\begin{align}
d_{i,jk}^{pred} &= \sqrt{(x_{i,j}^{pred} - x_{i,k}^{pred})^2 + (y_{i,j}^{pred} - y_{i,k}^{pred})^2} \label{eq:distancia_pred} \\
d_{i,jk}^{gt} &= \sqrt{(x_{i,j}^{gt} - x_{i,k}^{gt})^2 + (y_{i,j}^{gt} - y_{i,k}^{gt})^2} \label{eq:distancia_gt}
\end{align}

son las distancias euclidianas normalizadas entre \textit{landmarks} $j$ y $k$ en configuración predicha y de referencia, respectivamente. La formulación como error relativo porcentual normaliza por magnitud de distancia verdadera, evitando que distancias grandes dominen la métrica y permitiendo interpretación intuitiva: $E_{dist} = 10\%$ indica que, en promedio, distancias predichas difieren un 10\% de distancias verdaderas.

Valores bajos de error de preservación ($E_{dist} < 5\%$) indican que el modelo mantiene proporciones anatómicas correctamente, sugiriendo capacidad de capturar relaciones espaciales globales entre estructuras. Valores elevados ($E_{dist} > 15\%$) indican distorsiones geométricas sistemáticas que, aunque cada \textit{landmark} individual pueda tener error de localización bajo, la configuración global presenta proporciones anatómicas incorrectas (por ejemplo, tórax excesivamente estrecho o anormalmente alto), inaceptable para aplicaciones clínicas donde proporciones informan valoraciones diagnósticas.

La complementariedad entre MRE y $E_{dist}$ es sutil pero crítica: un modelo puede lograr MRE bajo mediante errores individuales pequeños que, al estar correlacionados sistemáticamente (por ejemplo, todos los \textit{landmarks} desplazados uniformemente hacia arriba), preservan distancias relativas y producen $E_{dist}$ bajo a pesar de configuración globalmente incorrecta. Por tanto, evaluación rigurosa requiere consideración simultánea de ambas métricas junto con métricas de simetría para caracterización completa de calidad de predicciones.


\subsubsection{Tasa de validez anatómica}
\label{subsubsec:validez_anatomica}

La tasa de validez anatómica cuantifica la proporción de predicciones que satisfacen restricciones de ordenamiento espacial fisiológico básicas, criterios binarios de aceptabilidad que toda configuración anatómicamente plausible debe cumplir. Se definen cuatro restricciones fundamentales:

\begin{enumerate}
    \item \textbf{Ordenamiento vertical de estructuras pulmonares}: Ápices pulmonares (landmarks 2, 3) deben localizarse superiormente (menor coordenada $y$, dado que origen es esquina superior izquierda) respecto a ángulos costofrénicos (landmarks 4, 5):
    \begin{equation}
    y_{\text{apex}}^{pred} < y_{\text{angulo}}^{pred}.
    \end{equation}

    \item \textbf{Centrado de estructuras mediastínicas}: Estructuras mediastínicas centrales (carina traqueal, ápice cardiaco, landmarks 1, 8) deben localizarse dentro de banda central del tórax, definida como $x \in [0.35, 0.65]$ en coordenadas normalizadas:
    \begin{equation}
    0.35 \leq x_{\text{mediastino}}^{pred} \leq 0.65.
    \end{equation}

    \item \textbf{No-inversión de estructuras bilaterales}: \textit{Landmarks} derechos (convencionalmente numerados pares: 2, 4, 6, etc.) deben localizarse a la derecha del eje de simetría ($x > 0.5$), y \textit{landmarks} izquierdos (impares: 3, 5, 7, etc.) a la izquierda ($x < 0.5$):
    \begin{equation}
    x_{\text{derecho}}^{pred} > 0.5, \quad x_{\text{izquierdo}}^{pred} < 0.5.
    \end{equation}

    \item \textbf{Contención dentro de campo de visión}: Todas las coordenadas predichas deben residir dentro de rango válido $[0,1]^2$:
    \begin{equation}
    0 \leq x_k^{pred} \leq 1, \quad 0 \leq y_k^{pred} \leq 1, \quad \forall k.
    \end{equation}
\end{enumerate}

Una muestra se clasifica como anatómicamente válida si satisface simultáneamente las cuatro restricciones para todos los \textit{landmarks} aplicables. La tasa de validez anatómica se define como:

\begin{equation}
\label{eq:tasa_validez}
\text{TVA} = \frac{\text{Número de muestras válidas}}{N} \times 100\%.
\end{equation}

Sistemas de calidad clínica deben alcanzar $\text{TVA} \geq 95\%$, garantizando que la vasta mayoría de predicciones son anatómicamente plausibles incluso si exhiben errores de localización moderados. Tasas de validez inferiores indican inestabilidad del modelo que produce ocasionalmente configuraciones absurdas (por ejemplo, ápices pulmonares inferiores a bases, estructuras mediastínicas desplazadas a periferia torácica), inaceptable para confianza clínica.


\subsection{Sistema de clasificación por calidad clínica}
\label{subsec:clasificacion_clinica}

El sistema de clasificación por umbrales de calidad clínica traduce mediciones continuas de MRE a categorías discretas interpretables, facilitando comunicación de capacidades del sistema a profesionales médicos y estableciendo criterios de decisión para aprobación de uso clínico. El sistema implementa cuatro categorías de calidad basadas en umbrales de error establecidos en literatura mediante análisis de variabilidad inter-observador entre radiólogos expertos y evaluación de precisión requerida para tareas diagnósticas específicas \cite{Payer2016, Ibragimov2017}.


\subsubsection{Definición de categorías}
\label{subsubsec:categorias_calidad}

Las cuatro categorías de calidad clínica se definen según rangos de MRE medido en imágenes de $224 \times 224$ píxeles:

\begin{enumerate}
    \item \textbf{Excelente} (MRE $< 2.0$ mm): Precisión equivalente a concordancia inter-observador entre radiólogos expertos experimentados. Errores de esta magnitud son imperceptibles en práctica clínica y no afectan interpretación diagnóstica. Sistemas en esta categoría alcanzan rendimiento humano experto, siendo candidatos ideales para integración en flujos de trabajo clínicos automatizados sin supervisión adicional.

    \item \textbf{Bueno} (2.0 mm $\leq$ MRE $< 4.0$ mm): Precisión suficiente para mayoría de aplicaciones clínicas de asistencia diagnóstica. Errores pueden ser detectables por observadores entrenados pero raramente alteran conclusiones diagnósticas. Sistemas en esta categoría son apropiados para despliegue clínico con supervisión ocasional por especialistas.

    \item \textbf{Aceptable} (4.0 mm $\leq$ MRE $< 8.5$ mm): Precisión marginal para aplicaciones clínicas, cercana al límite de aceptabilidad. Errores son frecuentemente visibles y pueden ocasionalmente afectar interpretación de hallazgos sutiles. Sistemas en esta categoría requieren validación extensiva caso-a-caso por especialistas antes de uso clínico, siendo más apropiados para aplicaciones de investigación, priorización de casos, o inicialización de segmentaciones manuales.

    \item \textbf{Inaceptable} (MRE $\geq 8.5$ mm): Precisión insuficiente para uso clínico. Errores son sistemáticos y de magnitud que compromete interpretabilidad de resultados. Sistemas en esta categoría no deben emplearse en contextos clínicos, requiriendo refinamiento metodológico fundamental antes de consideración para aplicaciones médicas.
\end{enumerate}

El umbral crítico de 8.5 mm (equivalente a aproximadamente 8.5 píxeles en imágenes de $224 \times 224$ para tórax adulto estándar con campo de visión de 40 cm, asumiendo resolución espacial de $\sim$1.8 mm/píxel) fue establecido por Payer et al. \cite{Payer2016} mediante análisis de variabilidad inter-observador: errores superiores a este umbral exceden discrepancias típicas entre anotaciones de múltiples radiólogos expertos, indicando que predicciones automáticas son menos confiables que juicio humano. Este umbral constituye estándar de facto en literatura de detección automática de \textit{landmarks} en radiografías de tórax, siendo referencia para comparación de métodos \cite{Payer2016, Ibragimov2017, Urschler2018}.


\subsubsection{Conversión de MRE a milímetros}
\label{subsubsec:conversion_mm}

La conversión de MRE expresado en píxeles a unidades físicas de milímetros requiere conocimiento de resolución espacial de imágenes, definida como distancia física representada por cada píxel. En el conjunto de datos empleado (Sección~\ref{sec:dataset}), las radiografías corresponden a tórax adultos con campo de visión típico de 40 cm, procesadas a resolución de $224 \times 224$ píxeles. La resolución espacial aproximada es:

\begin{equation}
\label{eq:resolucion_espacial}
r = \frac{400 \text{ mm}}{224 \text{ píxeles}} \approx 1.79 \text{ mm/píxel}.
\end{equation}

Por tanto, un MRE de $E_{pix}$ píxeles corresponde a error físico de:

\begin{equation}
\label{eq:mre_mm}
E_{mm} = E_{pix} \times 1.79 \text{ mm/píxel}.
\end{equation}

Siguiendo esta conversión, el umbral de aceptabilidad clínica de 8.5 píxeles corresponde a aproximadamente 15.2 mm de error físico, valor que coincide con definiciones de literatura médica \cite{Payer2016}. Los umbrales de categorías de calidad expresados en píxeles son:

\begin{itemize}
    \item Excelente: MRE $< 1.12$ píxeles (2.0 mm).
    \item Bueno: 1.12 píxeles $\leq$ MRE $< 2.23$ píxeles (4.0 mm).
    \item Aceptable: 2.23 píxeles $\leq$ MRE $< 4.75$ píxeles (8.5 mm).
    \item Inaceptable: MRE $\geq 4.75$ píxeles (8.5 mm).
\end{itemize}

Debe notarse que esta conversión asume campo de visión uniforme de 40 cm, aproximación válida para radiografías de tórax posteroanterior estándar de adultos. Variaciones en tamaño corporal del paciente, distancia foco-detector, y magnificación geométrica introducen variabilidad en resolución espacial efectiva, por lo que umbrales absolutos deben interpretarse como guías aproximadas sujetas a calibración específica por protocolo de adquisición clínico.


\subsection{Protocolo de validación}
\label{subsec:validacion}

El protocolo de validación implementa separación estricta de conjuntos de entrenamiento, validación y prueba para garantizar evaluación no sesgada de capacidad de generalización. El conjunto de prueba de 144 muestras (15\% del total, Sección~\ref{subsec:division_dataset}) permanece completamente no visto durante todo el proceso de entrenamiento de las cuatro fases, utilizado exclusivamente para evaluación final tras selección de modelo óptimo basado en rendimiento en conjunto de validación.

La estrategia de \textit{early stopping} (detención temprana) descrita en la Sección~\ref{sec:estrategia_entrenamiento} monitorea pérdida de validación tras cada época, almacenando \textit{checkpoint} (punto de control) cuando se observa nuevo mínimo. El modelo final seleccionado para evaluación en conjunto de prueba corresponde al \textit{checkpoint} con menor pérdida de validación observada durante Fase 4 (entrenamiento con función de pérdida completa), garantizando que métricas reportadas en conjunto de prueba reflejan mejor capacidad de generalización alcanzada sin optimización directa sobre datos de prueba.

El análisis por subgrupos diagnósticos evalúa robustez del modelo ante variabilidad patológica, reportando MRE desagregado por categoría diagnóstica (COVID-19, Viral Pneumonia, Normal) para identificar sensibilidad diferencial a tipos específicos de patología. Desviaciones significativas de rendimiento entre subgrupos (por ejemplo, MRE sustancialmente mayor en casos COVID-19 respecto a radiografías normales) indicarían limitaciones de generalización que requerirían aumentación de datos específica por patología o estrategias de aprendizaje multidominio \cite{Raghu2019}.


\subsection{Síntesis de métricas}
\label{subsec:sintesis_metricas}

El sistema de evaluación documentado en esta sección implementa jerarquía de métricas complementarias que caracterizan precisión de localización (MRE), consistencia estructural (simetría, preservación de distancias, validez anatómica), e idoneidad clínica (clasificación por umbrales) de predicciones del modelo desarrollado. La definición formal matemática de cada métrica, acompañada de justificación teórica y umbrales de interpretación basados en estándares internacionales de literatura médica, garantiza reproducibilidad completa de evaluación y comparabilidad rigurosa con trabajos previos \cite{Payer2016, Feng2018, Ibragimov2017}.

La aplicación sistemática de estas métricas sobre conjunto de prueba independiente, cuyos resultados se presentan exhaustivamente en el Capítulo~\ref{cap:resultados}, constituye validación experimental definitiva de la metodología desarrollada en este capítulo, permitiendo valoración objetiva de idoneidad del sistema para eventual aplicación clínica en detección automática de estructuras anatómicas en radiografías de tórax.


% Metodología anterior (SSM/ASM/CNN híbrida) - COMENTADA
%\chapter{Metodología}
\label{cap:metodologia}

\section{Visión General}
\label{sec:vis_general}

El objetivo central de esta metodología es desarrollar un sistema computacional capaz de localizar puntos relevantes en imágenes radiográficas de tórax. Estos puntos, denominados puntos de referencia o landmarks, son cruciales para el análisis cuantitativo de la forma y el tamaño de estructuras pulmonares.

La estrategia para alcanzar este objetivo se basa en el aprendizaje supervisado. Se utiliza un conjunto de datos donde manualmente se ha identificado y marcado la ubicación de estos landmarks en mil radiografías.

El proceso se compone de las siguientes etapas:

\begin{enumerate}
\item \textbf{Acondicionamiento de los Datos de Entrenamiento:} Se procesan las radiografías y las anotaciones manuales. Este paso incluye la normalización del tamaño de las imágenes y la alineación espacial de los conjuntos de landmarks. Dicha alineación asegura que las variaciones de forma entre las estructuras se puedan comparar de manera consistente, mitigando el efecto de diferencias en la pose o la escala.
\item \textbf{Definición de Regiones de Búsqueda:} Con base en la distribución espacial observada de los landmarks en el conjunto de entrenamiento, se establecen regiones de búsqueda delimitadas para cada landmark en nuevas imágenes. Estas regiones restringen el área donde el sistema intentará localizar el landmark, optimizando la eficiencia del proceso.
\item \textbf{Modelado de la Apariencia Local:} Para cada landmark, se extraen subimágenes (parches) centradas en su ubicación a través de todo el conjunto de entrenamiento. Estos parches representan la variabilidad visual típica de la región anatómica circundante a cada landmark. Mediante técnicas de análisis estadístico, se construyen modelos de apariencia que capturan los patrones visuales característicos y las variaciones comunes para cada tipo de landmark.
\item \textbf{Proceso de Búsqueda y Coincidencia:} Al analizar una nueva radiografía, el sistema examina múltiples ubicaciones candidatas dentro de la región de búsqueda definida para cada landmark. En cada ubicación candidata, se extrae un parche de imagen y se compara su apariencia con el modelo de apariencia previamente aprendido para ese landmark específico.
\item \textbf{Estimación de la Posición Óptima:} El sistema cuantifica la similitud entre cada parche candidato y el modelo de apariencia correspondiente mediante una métrica de error. La posición del parche que minimiza este error, indicando la mayor semejanza con la apariencia característica del landmark, se selecciona como la ubicación predicha para dicho punto.
\end{enumerate}

Las secciones siguientes de este capítulo detallan la formulación matemática y los algoritmos específicos que implementan cada una de estas etapas. Se abordará la representación numérica de imágenes y landmarks, los métodos de alineación basados en transformaciones geométricas, la construcción de modelos de apariencia mediante análisis de componentes principales, y la optimización para la localización de landmarks a través de la minimización de una función de error de reconstrucción. El principio en sí es utilizar datos de ejemplo y herramientas matemáticas para desarrollar un sistema capaz de identificar y localizar puntos de referencia clave en imágenes radiográficas de tórax.

\begin{figure}[htbp] 
    \centering
    \includegraphics[width=1\linewidth]{Figures/diagrama_bloques_metodologia.png}
    \caption{Diagrama de bloques de la visión general de la metodología}
    \label{fig:diagrama_bloques_metodologia}
\end{figure}

\section{Adquisición y Preprocesamiento de Datos Anotados}

La metodología se fundamenta en un conjunto de imágenes de radiografías de tórax que contienen anotaciones manuales de puntos de referencia. La preparación de estos datos comprende las siguientes etapas:

\subsection{Selección y Organización del Conjunto de Datos}
Se selecciona un subconjunto de imágenes provenientes del conjunto de datos original. Cada imagen se asocia con un identificador único y una categoría (ej., COVID, Normal). Dicha información se estructura en un archivo de índice.

\subsection{Etiquetado Manual Asistido de Puntos de Referencia}
El proceso de anotación implica la definición interactiva y secuencial por parte de un observador de 15 puntos de referencia anatómicos, denotados como $\mathbf{p}_i = (x_i, y_i) \in \mathbb{Z}^2$ para $i=0, \dots, 14$, sobre la imagen visualizada. Los dos primeros puntos, $\mathbf{p}_0$ y $\mathbf{p}_1$, definen una \textit{Línea Principal}. Los 13 puntos restantes se derivan computacionalmente a partir de la configuración geométrica definida por $\mathbf{p}_0$ y $\mathbf{p}_1$.

La \textit{Línea Principal}, $\mathcal{L}_{main}$, se define como la recta que contiene los puntos $\mathbf{p}_0 = (x_0, y_0)$ y $\mathbf{p}_1 = (x_1, y_1)$. Su pendiente, $m$, se determina mediante:
$$ m = \begin{cases} \frac{y_1 - y_0}{x_1 - x_0} & \text{si } x_1 \neq x_0 \\ \infty & \text{si } x_1 = x_0 \end{cases} $$
Consecuentemente, la pendiente $m_{\perp}$ de cualquier recta perpendicular a $\mathcal{L}_{main}$ se calcula como:
$$ m_{\perp} = \begin{cases} -1/m & \text{si } m \neq 0, \infty \\ \infty & \text{si } m = 0 \\ 0 & \text{si } m = \infty \end{cases} $$

Los puntos $\mathbf{p}_8, \mathbf{p}_9,$ y $\mathbf{p}_{10}$ se establecen como puntos intermedios sobre el segmento rectilíneo $[\mathbf{p}_0, \mathbf{p}_1]$ a través de interpolación lineal:
$$ \mathbf{p}_i = \mathbf{p}_0 + c_i (\mathbf{p}_1 - \mathbf{p}_0), \quad i \in \{8, 9, 10\} $$
donde los factores de interpolación $c_8 = 1/4$, $c_9 = 1/2$, y $c_{10} = 3/4$ corresponden al primer cuarto, punto medio y tercer cuarto del segmento, respectivamente. Las coordenadas resultantes se redondean al entero más cercano.

Los puntos de referencia restantes se generan en pares. Cada par se sitúa a una distancia predefinida de uno de los puntos base ($\mathbf{p}_0, \mathbf{p}_1, \mathbf{p}_8, \mathbf{p}_9, \mathbf{p}_{10}$) a lo largo de la recta perpendicular a $\mathcal{L}_{main}$ que pasa por dicho punto base. Dado un punto base $\mathbf{p}_{ba} = (x_{ba}, y_{ba})$ sobre $\mathcal{L}_{main}$ y una distancia de separación $d$, las coordenadas de los dos puntos $\mathbf{q}_a$ y $\mathbf{q}_b$, situados simétricamente a lo largo de la perpendicular, se calculan como:\\
\\
Si $m_{\perp} \neq \infty$ y $m_{\perp} \neq 0$:
$$ x_{a,b} = x_{ba} \pm \frac{d}{\sqrt{1 + m_{\perp}^2}}, \quad y_{a,b} = y_{ba} + m_{\perp}(x_{a,b} - x_{ba}) $$
Si $m_{\perp} = \infty$ (Línea Principal horizontal):
$$ x_{a,b} = x_{ba}, \quad y_{a,b} = y_{ba} \pm d $$
Si $m_{\perp} = 0$ (Línea Principal vertical):
$$ x_{a,b} = x_{ba} \pm d, \quad y_{a,b} = y_{ba} $$
Las coordenadas resultantes se redondean al entero más cercano.

\begin{figure}[htbp] 
    \centering
    \includegraphics[width=0.5\linewidth]{Figures/etiquetado_puntos.png}
    \caption{Puntos $\mathbf{q}_a$ y $\mathbf{q}_b$ para cada punto base $\mathbf{p}_0, \mathbf{p}_1, \mathbf{p}_8, \mathbf{p}_9, \mathbf{p}_{10}$}
    \label{fig:etiqueda_manual_puntos}
\end{figure}

% Se contempla un caso particular cuando la Línea Principal $\mathcal{L}_{main}$ es vertical (i.e., $x_0 = x_1$). En esta situación, los puntos intermedios sobre $\mathcal{L}_{main}$ se determinan únicamente mediante interpolación de sus coordenadas $y$. Los puntos perpendiculares, que en este escenario son horizontales, se generan utilizando coordenadas $x$ preestablecidas (0, 16, 48, 64) y coordenadas $y$ idénticas a las de sus respectivos puntos base sobre $\mathcal{L}_{main}$.

Se permite al observador realizar un ajuste fino de la posición horizontal de un subconjunto específico de puntos (índices 2-7 y 11-14). Si un punto $\mathbf{p}_i = (x_i, y_i)$ es desplazado horizontalmente por una cantidad $\Delta x$, su nueva coordenada $x'_i$ es $x_i + \Delta x$. La coordenada $y'_i$ correspondiente se recalcula para asegurar que el punto ajustado $\mathbf{p}'_i = (x'_i, y'_i)$ permanezca sobre la línea perpendicular original a $\mathcal{L}_{main}$ que contiene al punto de referencia original $\mathbf{p}_{ref} = (x_{ref}, y_{ref})$ en $\mathcal{L}_{main}$. \\
Esta se calcula como:
$$ y'_i = \text{round}(m_{\perp}(x'_i - x_{ref}) + y_{ref}) $$
donde $m_{\perp}$ es la pendiente de la línea perpendicular asociada al punto $\mathbf{p}_i$.

\begin{figure}[htbp] 
    \centering
    \includegraphics[width=0.5\linewidth]{Figures/etiquetado.png}
    \caption{Etiquetado manual asisitido de puntos de referencia}
    \label{fig:etiqueda_manual}
\end{figure}

\subsection{Transformación de Coordenadas para Almacenamiento}
Las coordenadas de los puntos de referencia, inicialmente definidas en la resolución de visualización $D_v \times D_v$ (ej., $640 \times 640$ píxeles), se transforman para adecuarse a múltiples resoluciones objetivo $R_t \times R_t$ (ej., $\{64 \times 64, 128 \times 128, 256 \times 256\}$ píxeles). Para una coordenada $c$ (sea $x$ o $y$), su valor escalado $c_{escalada}$ para una resolución objetivo $R_t$ se calcula mediante:
$$ c_{escalada} = \max\left(0, \min\left(\text{round}\left(c \cdot \frac{R_t}{D_v}\right), R_t-1\right)\right) $$
Este cálculo implica la aplicación de un factor de escala $S = R_t / D_v$, el redondeo al entero más próximo, y una operación de acotación al intervalo $[0, R_t-1]$. Esta última garantiza que las coordenadas transformadas se encuentren dentro de los límites de la imagen reescalada.

\section{Alineamiento de Formas}

Para mitigar la variabilidad inducida por transformaciones globales de traslación, rotación y escala en los conjuntos de puntos de referencia anotados, se implementa un proceso de alineamiento de formas. Este proceso se basa en el Análisis Generalizado de Procrustes (GPA).

\begin{figure}[htbp] 
    \centering
    \includegraphics[width=0.6\linewidth]{Figures/gpa_step1_original.png}
    \caption{Ejemplo con 4 landmarks formando cuadriláteros}
    \label{fig:gpa_step1}
\end{figure}

Dado un conjunto de $N$ configuraciones de puntos (formas), donde cada forma $k$ se representa por una matriz $\mathbf{S}_k \in \mathbb{R}^{15 \times 2}$ que contiene las coordenadas $(x, y)$ de sus 15 puntos de referencia, el GPA busca alinear iterativamente estas formas a una forma media.

El algoritmo GPA se desarrolla en las siguientes etapas:

\begin{enumerate}
    \item \textbf{Preprocesamiento de Formas Individuales:} Cada forma $\mathbf{S}_k$ se somete a un preprocesamiento inicial.
    \begin{enumerate}
        \item Centrado: Se calcula el centroide de la forma $\mathbf{S}_k$:
        \[
            \overline{\mathbf{s}}_k = \frac{1}{15} \sum_{j=1}^{15} \mathbf{p}_{kj}
        \]
        La forma se centra restando su centroide:
        \[
            \mathbf{S}'_k = \mathbf{S}_k - \mathbf{1}\overline{\mathbf{s}}_k^T
        \]
        donde $\mathbf{1}$ es un vector columna de unos.
        \item Normalización de Escala: La escala de la forma centrada $\mathbf{S}'_k$ se normaliza dividiendo por su norma de Frobenius (también conocida como Tamaño Centroide):
        \[
            \mathbf{S}''_k = \frac{\mathbf{S}'_k}{\|\mathbf{S}'_k\|_F}
        \]
    \end{enumerate}

    \begin{figure}[htbp] 
    \centering
    \includegraphics[width=0.6\linewidth]{Figures/gpa_step2_centered.png}
    \caption{Ejemplo con 4 landmarks centrados}
    \label{fig:gpa_step2}
    \end{figure}

    \item \textbf{Inicialización de la Forma Media:} Se selecciona una de las formas preprocesadas (e.g., $\mathbf{S}''_1$) como la estimación inicial de la forma media, denotada como $\mathbf{M}^{(0)}$.
    
    \item \textbf{Proceso Iterativo de Alineamiento (GPA):} Se ejecuta un procedimiento iterativo, indexado por $t=0, 1, \dots$, hasta alcanzar un criterio de convergencia.
    \begin{enumerate}
        \item \textbf{Alineamiento a la Forma Media Actual:} Cada forma preprocesada $\mathbf{S}''_k$ se alinea a la forma media actual $\mathbf{M}^{(t)}$. Esto implica encontrar la matriz de rotación óptima $\mathbf{R}_k$ que minimiza la suma de las distancias euclidianas al cuadrado entre los puntos correspondientes de $\mathbf{S}''_k \mathbf{R}_k$ y $\mathbf{M}^{(t)}$. Este problema es equivalente a minimizar la siguiente expresión:
        \[
            \|\mathbf{S}''_k \mathbf{R}_k - \mathbf{M}^{(t)}\|_F^2
        \]
        La solución para $\mathbf{R}_k$ se obtiene a partir de la Descomposición en Valores Singulares (SVD) de la matriz de covarianza cruzada $\mathbf{C}_k = (\mathbf{S}''_k)^T \mathbf{M}^{(t)}$. Si la SVD de $\mathbf{C}_k$ es $\mathbf{U}_k \boldsymbol{\Sigma}_k \mathbf{V}_k^T$, entonces la matriz de rotación óptima es:
        \[
            \mathbf{R}_k = \mathbf{V}_k \mathbf{U}_k^T
        \]
        Se aplica una corrección a $\mathbf{R}_k$ si su determinante es negativo, para asegurar que representa una rotación propia. Las formas alineadas en la iteración $t$ son $\tilde{\mathbf{S}}''_k = \mathbf{S}''_k \mathbf{R}_k$.

        \item \textbf{Actualización de la Forma Media:} Se calcula una nueva forma media $\mathbf{M}_{\text{raw}}^{(t+1)}$ promediando las coordenadas de todas las formas alineadas $\tilde{\mathbf{S}}''_k$:
        \[
            \mathbf{M}_{\text{raw}}^{(t+1)} = \frac{1}{N} \sum_{k=1}^N \tilde{\mathbf{S}}''_k
        \]
        \item \textbf{Normalización de la Nueva Forma Media:} La forma media $\mathbf{M}_{\text{raw}}^{(t+1)}$ se centra y se normaliza a escala unitaria (siguiendo el procedimiento del paso 1) para obtener la forma media actualizada $\mathbf{M}^{(t+1)}$.
    \end{enumerate}

            \begin{figure}[htbp] 
    \centering
    \includegraphics[width=0.6\linewidth]{Figures/gpa_step3_normalized.png}
    \caption{Ejemplo con 4 landmarks normalizados}
    \label{fig:gpa_step3}
    \end{figure}
    
    \item \textbf{Criterio de Convergencia:} El proceso iterativo concluye cuando la diferencia entre formas medias consecutivas, cuantificada mediante la norma de Frobenius, es inferior a un umbral de tolerancia predefinido $\epsilon$:
    \[
        \|\mathbf{M}^{(t+1)} - \mathbf{M}^{(t)}\|_F < \epsilon
    \]
\end{enumerate}
Las formas resultantes del GPA están centradas en el origen y poseen una escala normalizada (unitaria). Para alinear las imágenes originales correspondientes a estas formas normalizadas, se estima una transformación de similitud $\mathbf{T}_k$ para cada imagen. Esta transformación, que incluye parámetros de escala, rotación y traslación, mapea los puntos de referencia originales de la imagen $k$ a sus correspondientes puntos alineados por GPA. Dicha transformación $\mathbf{T}_k$ puede estimarse utilizando una transformación afín 2D. Finalmente, la transformación $\mathbf{T}_k$ estimada se aplica a la imagen original para generar la imagen alineada.

\begin{figure}[htbp] 
    \centering
    \includegraphics[width=0.6\linewidth]{Figures/gpa_step4_aligned.png}
    \caption{Ejemplo con 4 landmarks alineados y su forma media}
    \label{fig:gpa_step4}
\end{figure}

\section{Extracción de Regiones de Búsqueda y Análisis de Templates}
\label{sec:extraccion_region}

Esta etapa define las áreas en las imágenes donde se buscarán los puntos anatómicos y los parámetros para extraer parches de apariencia.

\subsection{Extracción de Regiones de Búsqueda}
Para cada punto de referencia $j \in \{0, \dots, 14\}$, se analiza la distribución espacial de sus coordenadas $(x_{ij}, y_{ij})$ en el conjunto de entrenamiento alineado. Estas coordenadas, que pueden ser de punto flotante debido a los calculos GPA, se convierten a enteros y se aplica \textit{clamping} al rango $[0, 63]$ para asegurar su correspondencia con una cuadrícula de $64 \times 64$ píxeles. Se construye un histograma 2D $H_j$ de $64 \times 64$ donde $H_j[y, x]$ cuenta el número de veces que el punto $j$ aparece en la ubicación $(x, y)$ después del clamping. Se calcula la caja delimitadora (bounding box) más pequeña que contiene todas las ubicaciones con $H_j[y, x] > 0$. Esta caja, definida por $(\min\_x_j, \max\_x_j, \min\_y_j, \max\_y_j)$, establece la región de búsqueda rectangular $\mathcal{R}_j$ para el punto $j$.

\begin{figure}[htbp] 
    \centering
    \includegraphics[width=0.7\linewidth]{Figures/landmark_1_search_zone.png}
    \caption{Región de búsqueda para landmark 1}
    \label{fig:landmar_1_search_zone}
\end{figure}

\subsection{Análisis de Templates de Recorte}
Se analizan las regiones de búsqueda $\mathcal{R}_j$ para derivar parámetros de un template de recorte rectangular para cada punto $j$. Para la caja delimitadora $(\min\_x_j, \max\_x_j, \min\_y_j, \max\_y_j)$ dentro de una cuadrícula de $64 \times 64$, se calculan las distancias a los bordes de la cuadrícula:
$$ a_j = \min\_y_j \quad (\text{distancia superior}) $$
$$ b_j = 63 - \max\_x_j \quad (\text{distancia derecha}) $$
$$ c_j = 63 - \max\_y_j \quad (\text{distancia inferior}) $$
$$ d_j = \min\_x_j \quad (\text{distancia izquierda}) $$

\begin{figure}[htbp] 
    \centering
    \includegraphics[width=0.5\linewidth]{Figures/template1.png}
    \caption{Ejemplo de región de búsqueda mostrando las dimensiones ($a_j$, $b_j$, $c_j$, $d_j$)}
    \label{fig:landmar_1_template_recorte}
\end{figure}

El template de recorte para el punto $j$ se define como un rectángulo cuyas dimensiones son las de la caja delimitadora de la región de búsqueda: 
$$W_{template,j} = (\max\_x_j - \min\_x_j + 1),  H_{template,j} = (\max\_y_j - \min\_y_j + 1)$$
Se identifica un "punto de intersección":  
$$\mathbf{p}_{int,j} = (x_{int,j}, y_{int,j}) = (\min\_x_j, \min\_y_j)$$

\begin{figure}[htbp] 
    \centering
    \includegraphics[width=1\linewidth]{Figures/template_analysis_coord1.png}
    \caption{Análisis template de recorte landmark 1}
    \label{fig:landmar_1_template}
\end{figure}

\section{Recorte de Imágenes para Entrenamiento de Apariencia}

Este procedimiento extrae parches de imágenes, previamente redimensionadas a $64 \times 64$ píxeles, mediante el uso de plantillas y puntos de referencia. El objetivo es generar datos para el entrenamiento de un modelo de apariencia.

Para una imagen $i$ y su $j$-ésimo punto de referencia, se emplea la coordenada de referencia (ground truth) $\mathbf{p}_{lp,ij} = (x_{lp,ij}, y_{lp,ij})$. Los componentes de esta coordenada se convierten a valores enteros y se acotan al rango $[0, 63]$ para que se ajusten a las dimensiones de la imagen.

A cada $j$-ésimo punto de referencia se asocia una plantilla de recorte de dimensiones $W_{\text{template},j} \times H_{\text{template},j}$. Esta plantilla posee un punto de anclaje interno $\mathbf{p}_{\text{int},j} = (x_{\text{int},j}, y_{\text{int},j})$, cuyas coordenadas son relativas a la esquina superior izquierda de la propia plantilla. La esquina superior izquierda de la región de búsqueda (o posición de referencia inicial) para la plantilla del punto $j$ en la imagen se denota como $(\min\_x_j, \min\_y_j)$. En esta configuración, el punto de anclaje $\mathbf{p}_{\text{int},j}$ se proyecta en la imagen en la coordenada $(\min\_x_j + x_{\text{int},j}, \min\_y_j + y_{\text{int},j})$.

El objetivo consiste en alinear esta proyección del punto de anclaje de la plantilla con el punto de referencia $\mathbf{p}_{lp,ij}$ en la imagen. Para lograr esta alineación, se calcula un vector de desplazamiento $(dx_{ij}, dy_{ij})$:
$$ dx_{ij} = x_{lp,ij} - (\min\_x_j + x_{\text{int},j}) $$
$$ dy_{ij} = y_{lp,ij} - (\min\_y_j + y_{\text{int},j}) $$
Posteriormente, se determinan las coordenadas de la esquina superior izquierda de la ventana de recorte final, $(\text{final\_min\_x}_{ij}, \text{final\_min\_y}_{ij})$. Estas se obtienen al aplicar el desplazamiento $(dx_{ij}, dy_{ij})$ a la posición de referencia inicial de la plantilla $(\min\_x_j, \min\_y_j)$. Adicionalmente, se aplica una operación de acotación para asegurar que la ventana de recorte se encuentre completamente contenida dentro de los límites $[0, 63]$ de la imagen:
$$ \text{final\_min\_x}_{ij} = \max(0, \min(\min\_x_j + dx_{ij}, 63 - W_{\text{template},j} + 1)) $$
$$ \text{final\_min\_y}_{ij} = \max(0, \min(\min\_y_j + dy_{ij}, 63 - H_{\text{template},j} + 1)) $$

\begin{figure}[htbp] 
\centering
\includegraphics[width=0.6\linewidth]{Figures/recorte.png}
\caption{Ejemplo de extracción del parche donde el punto verde ($\mathbf{p}_{\text{int},j}$) es alineado con el punto rojo ($\mathbf{p}_{lp,ij}$).}
\label{fig:extraccion}
\end{figure}

Finalmente, se extrae el parche de apariencia $\mathbf{P}_{ij}$ recortando la imagen en la posición $(\text{final\_min\_x}_{ij}, \text{final\_min\_y}_{ij})$ con las dimensiones $W_{\text{template},j} \times H_{\text{template},j}$ de la plantilla.

\begin{figure}[htbp] 
    \centering
    \includegraphics[width=0.8\linewidth]{Figures/diagrama_parches.png}
    \caption{Parches de 39x34 pixeles extraidos del conjunto de entrenamiento para la landmark 1}
    \label{fig:landmar_1_parches}
\end{figure}

\section{Modelos de Apariencia (Eigenpatches)}

Para cada punto anatómico $j$, se entrena un modelo de apariencia utilizando el conjunto de parches recortados $\{\mathbf{P}_{ij}\}_{i=1}^N$, donde $N$ es el número total de imágenes de entrenamiento. Cada parche $\mathbf{P}_{ij}$, de dimensiones $H \times W$ píxeles, se vectoriza transformándolo en un vector columna $\mathbf{x}_{ij} \in \mathbb{R}^{D}$, donde $D = HW$ representa la dimensionalidad del espacio de los parches.

\subsection{Análisis de Componentes Principales (PCA)}
El Análisis de Componentes Principales (PCA) se aplica al conjunto de parches vectorizados $\{\mathbf{x}_{ij}\}_{i=1}^N$ correspondientes a un punto anatómico $j$. El objetivo de PCA es identificar un subespacio lineal de menor dimensión que capture la máxima varianza presente en los datos originales. Este proceso se desglosa en los siguientes pasos:

\begin{enumerate}
    \item \textbf{Cálculo de la media muestral:} Se determina el parche promedio $\mean{\mathbf{x}}_j$ para el punto anatómico $j$:
    $$ \mean{\mathbf{x}}_j = \frac{1}{N} \sum_{i=1}^N \mathbf{x}_{ij} $$
    Esta media representa la apariencia central de los parches observados.

    \item \textbf{Centrado de los datos:} Cada parche vectorizado $\mathbf{x}_{ij}$ se centra restándole la media muestral $\mean{\mathbf{x}}_j$. Estos vectores centrados se organizan como las filas de una matriz de datos centrados $\mathbf{X}_{c,j} \in \mathbb{R}^{N \times D}$:
    $$ \mathbf{X}_{c,j} = [\mathbf{x}_{1j} - \mean{\mathbf{x}}_j, \dots, \mathbf{x}_{Nj} - \mean{\mathbf{x}}_j]^T $$
    Este paso asegura que la varianza se calcule respecto al centro de la distribución de los datos.

    \item \textbf{Cálculo de la matriz de covarianza:} Se estima la matriz de covarianza muestral $\mathbf{C}_j \in \mathbb{R}^{D \times D}$ a partir de los datos centrados:
    $$ \mathbf{C}_j = \frac{1}{N-1} \mathbf{X}_{c,j}^T \mathbf{X}_{c,j} $$
    La matriz de covarianza codifica las interrelaciones y variaciones entre los diferentes píxeles de los parches.

    \item \textbf{Resolución del problema de valores propios:} Se calculan los valores propios $\lambda_{jk}$ y los vectores propios $\mathbf{v}_{jk}$ de la matriz de covarianza $\mathbf{C}_j$:
    $$ \mathbf{C}_j\mathbf{v}_{jk} = \lambda_{jk} \mathbf{v}_{jk}, \quad \text{para } k=1, \dots, D $$
    Los vectores propios indican las direcciones de máxima varianza en el espacio de los datos, y los valores propios cuantifican dicha varianza.

    \item \textbf{Selección de componentes principales:} Se ordenan los vectores propios según sus valores propios correspondientes en orden descendente. Se seleccionan los primeros $m_j$ vectores propios (componentes principales), $\mathbf{v}_{j1}, \dots, \mathbf{v}_{jm_j}$, que se utilizan para formar las columnas de la matriz de proyección $\mathbf{V}_j = [\mathbf{v}_{j1}, \dots, \mathbf{v}_{jm_j}] \in \mathbb{R}^{D \times m_j}$. El número de componentes $m_j$ (donde $m_j \ll D$) se elige típicamente para retener un porcentaje predefinido de la varianza total de los datos (p. ej., 95\%). Esta selección permite una representación compacta de la variabilidad principal de la apariencia.
\end{enumerate}

\subsection{Extraccion de Eigenpatches}

Una vez entrenado el modelo PCA para el punto $j$, un nuevo parche vectorizado $\mathbf{x}$ (no necesariamente del conjunto de entrenamiento) se puede proyectar al subespacio de menor dimensión $m_j$. Para ello, primero se centra el parche $\mathbf{x}$ utilizando la media $\mean{\mathbf{x}}_j$ calculada durante el entrenamiento, y luego se multiplica por la transpuesta de la matriz de proyección $\mathbf{V}_j$:
$$ \boldsymbol{\omega} = \mathbf{V}_j^T (\mathbf{x} - \mean{\mathbf{x}}_j) $$
El vector resultante $\boldsymbol{\omega} \in \mathbb{R}^{m_j}$ contiene los coeficientes o pesos que representan al parche $\mathbf{x}$ en el subespacio PCA.

Es posible reconstruir una aproximación $\hat{\mathbf{x}}$ del parche original $\mathbf{x}$ a partir de su representación en el subespacio PCA $\boldsymbol{\omega}$:
$$ \hat{\mathbf{x}} = \mathbf{V}_j\boldsymbol{\omega} + \mean{\mathbf{x}}_j $$
Esta reconstrucción $\hat{\mathbf{x}} \in \mathbb{R}^D$ reside en el espacio original de los parches y representa la porción de $\mathbf{x}$ que puede ser explicada por los $m_j$ componentes principales seleccionados.

\begin{figure}[htbp] 
    \centering
    \includegraphics[width=1\linewidth]{Figures/coord1_eigenfaces.png}
    \caption{Visualización conceptual de la implementación de PCA y Eigenpatches para la landmark 1. Muestra el ''parche medio" ($\mean{\mathbf{x}}$ reformado) y los primeros componentes principales ($\mathbf{v}_k$ reformados a la dimensión del parche) que capturan los modos de variación en la apariencia de los parches de entrenamiento.}
    \label{fig:pca_eigenfaces}
\end{figure}

\section{Predicción de Coordenadas en Imágenes de Prueba}

Para predecir la ubicación de un punto anatómico $j$ en una nueva imagen de prueba $\mathbf{I}_{test}$ (redimensionada a $64 \times 64$), se utiliza el modelo de apariencia $\mathcal{M}_j$ entrenado y la región de búsqueda $\mathcal{R}_j$.

\begin{enumerate}
    \item \textbf{Búsqueda en Región Definida:} Se itera sobre cada punto candidato $(y_c, x_c)$ dentro de la región de búsqueda $\mathcal{R}_j$.
    \item \textbf{Extracción de Parche Candidato:} Para cada $(y_c, x_c)$, se calcula la esquina superior izquierda $(y_{tl}, x_{tl})$ de un parche $\mathbf{P}_{c}$ de tamaño $W_{template,j} \times H_{template,j}$, de manera que el punto de intersección $\mathbf{p}_{int,j}$ del template se alinee con $(y_c, x_c)$. El parche $\mathbf{P}_{c}$ se extrae de $\mathbf{I}_{test}$.
    \item \textbf{Aplicación del Modelo y Cálculo de Error:} El parche extraído $\mathbf{P}_{c}$ se vectoriza a $\mathbf{x}_c$ y se procesa a través del modelo $\mathcal{M}_j$ (PCA). Se calcula el error de reconstrucción $E(\mathbf{x}_c)$ entre el parche original y su reconstrucción $\hat{\mathbf{x}}_c$ desde el subespacio del modelo. La métrica de error utilizada es la norma L2 (distancia euclidiana):
    $$ E_{L2}(\mathbf{x}_c) = \vectornorm{\mathbf{x}_c - \hat{\mathbf{x}}_c}_2 = \sqrt{\sum_{k=1}^D (x_{c,k} - \hat{x}_{c,k})^2} $$
    donde $D$ es el número de píxeles en el parche.
    \item \textbf{Selección del Punto Óptimo:} El punto candidato $(y_c, x_c)$ que minimiza el error de reconstrucción $E(\mathbf{x}_c)$ se considera la ubicación predicha $\hat{\mathbf{p}}_j$ del punto anatómico $j$.
\end{enumerate}

\begin{figure}[htbp] 
    \centering
    \includegraphics[width=1\linewidth]{Figures/coord1_iteration_00244.png}
    \caption{Visualización del proceso de iteración de predicción para el landmark 1.}
    \label{fig:prediccion_1}
\end{figure}

% \section{Conclusiones Metodología}

% Esta metodología ha cumplido con el objetivo de sentar las bases del proceso de investigación realizado hasta el momento y da las herramientas necesarias para poder profundizar en la investigación de los temas analizados hasta este momento. a continuacion se desarrolla la metodologia desarrolada para resolver el problema de que se necesita un metodo de extraccion de características robusto para combatir las variabilidades inherentes del dataset de las radiografias de torax que la metodologia actual no logro capturar adecuadamente con un rendimiento no óptimo, esto puede ser mediante el aumento de datos, uso de metodos mas vanzados como emtodods de apariencia activa, o forma activa, redes profundas o redes convolucionales,entre otros, este proceso que se ha desarrolado se basa en el paper  [A Generic Approach to Lung Field Segmentation from Chest Radiographs using Deep Space and Shape Learning] el cual sirvio de base para continuar la investigacion y experimentacion, se buscaba replicar el metodo propuesto en el paper para comprender sus diferentes metodos utilizados ya que sus condiciones iniciales eran muy similares al de esta investigación, se logro replicar el paper pero los resultados no fueron optimos debido a diferencias en las condiciones inciales, principalmente que el dataset usado en el metodo del paper usa imagenes de 2048x2048 pixeles, imagenes con una alta definicion y gran contenido de informacion diferente al dataset de esta investigacion con caracteristicas de 299x299 pixeles, la disminucion de informacion y detalle de la simagenes de este dataset no permitieron que se lograra una correcta extraccion de caractyeristicas y por lo tanto la prediccion no fue la deseable, se continuo con la experimentacion y se decidio experiemntar con adaptar el trabajo realizado para que usara redes convolucionales para la extraccion y prediccion de caractersiticas ya que las redes convolucionales son fuertes opciones para trabajar con imagenes pequeñas con pocos detalles, se experimento con diferentes arquitecturas basandose en la investigacion de las redes convolucionales mas comunes y sencillas y se experimento con diferentes parametros como batches, learning rate, tamaños de parches, etc hasta que se logro obtener resultados deseables con predicciones lo suficientemente cercanas a las ground truth. se expĺica la metodologia de manera superficial ya que aun falta mayor profundización en los conceptos clave que se tratan y además aún no se ha terminado de decidir si esta metodologia junto a sus resultados serán de utilidad para solventar las necesidades del proceso de investigación que se esta realizando en esta tesis.

\section{Expansión Estadística de Histograma Asimétrico}

Las imágenes radiográficas de tórax típicamente muestran una distribución de intensidades no uniforme, con una tendencia hacia los tonos más oscuros debido a las áreas oscuras de aire en los pulmones. Esta característica resulta en histogramas que son notablemente asimétricos, con una forma característica como se muestra en la Figura 3-8, donde aparece un pico angosto a extrema la izquierda, siguiendo una zona larga y de poca amplitud que crece hasta un máximo prominente a la derecha.

\begin{figure}[h!]
    \centering
    \includegraphics[width=0.9\linewidth]{Figures/histograma1.png}
    \caption{Visualización en diferentes imágenes de la forma característica de los histogramas en radiografías de tórax}
    \label{fig:enter-label11}
\end{figure}

La expansión del histograma tendría la ventaja de no cambiar las proporciones de zonas blancas y oscuras en la imagen. Sin embargo, la utilización de un mínimo global y de un máximo global puede hacer que el ajuste falle cuando hay presencia de algunos pixeles espurios contaminando la imagen. Por esta última razón, sería más razonable calcular un mínimo promedio y un máximo promedio de los niveles de gris de la imagen. La desviación estándar puede ser útil para esto. No obstante, y dado que los histogramas de este tipo de imágenes son asimétricos como lo muestra la Figura 3-8, utilizar alguna proporción de la desviación estándar tanto hacia arriba de la media como hacia debajo de la media para obtener el máximo y el mínimo tampoco resultaría adecuado, pues en muchas imágenes el mínimo calculado de esta forma estaría cercano al mínimo absoluto pero el máximo podría quedar muy por arriba del máximo absoluto, o bien al revés (Figura 3-9). 

\begin{figure}
    \centering
    \includegraphics[width=0.9\linewidth]{Figures/imagen22.png}
    \caption{ Ilustración en diferentes imágenes del máximo y mínimo estadístico, se puede observar que, al usar estos límites basados en 1.5 veces la desviación estándar, se mejora la representación del rango de intensidades, esto demuestra que si usamos la desviación común puede ser que quede bien hacía arriba de la desviación estándar, pero se encuentren conflictos hacia debajo de la desviación estándar, por ello se debe usar una desviación asimétrica y poder obtener así los máximos y mínimos estadísticos que se adapten mejor a lo necesitado}
    \label{fig:sahs_max_min_estadistico}
\end{figure}

\subsection{Cálculo de intensidades y normalización}
El proceso de normalización de intensidades se realiza mediante las siguientes ecuaciones:

\subsubsection{Media de intensidades}
Sea $I(x, y)$ la imagen de entrada en escala de grises. La media de intensidades $\mu$ se calcula como:

\begin{equation}
\mu = \frac{1}{N} \sum_{x=1}^{n} \sum_{y=1}^{m} I(x, y)
\end{equation}

donde $N = n \times m$ es el número total de píxeles.

\subsubsection{Separación de valores de intensidad}
Los valores de intensidad se dividen en dos conjuntos:

\begin{equation}
A = { I(x, y) \mid I(x, y) > \mu }
\end{equation}

\begin{equation}
B = { I(x, y) \mid I(x, y) \leq \mu }
\end{equation}

\subsubsection{Desviaciones estándar asimétricas}
\begin{equation}
\sigma_+ = \sqrt{ \frac{1}{\#A} \sum_{I(x, y) \in A} (I(x, y) - \mu)^2 }
\end{equation}

\begin{equation}
\sigma_- = \sqrt{ \frac{1}{\#B} \sum_{I(x, y) \in B} (I(x, y) - \mu)^2 }
\end{equation}

\subsubsection{Límites de expansión}
\begin{equation}
u = \mu + c_+ \sigma_+
\end{equation}

\begin{equation}
l = \mu - c_- \sigma_-
\end{equation}

donde $c_+ = 2.5$ y $c_- = 2$.

\subsubsection{Función de mapeo}
La función de mapeo lineal se define como:

\begin{equation}
I'(x, y) = 255 \times \frac{I(x, y) - l}{u - l}
\end{equation}

con las restricciones:
\begin{itemize}
\item Si $I'(x, y) > 255$, entonces $I'(x, y) = 255$
\item Si $I'(x, y) < 0$, entonces $I'(x, y) = 0$
\end{itemize}

Este método además de ser computacionalmente más ligero que CLAHE, ofrece ventajas para el caso específico del tipo de histograma asimétrico típico en esta clase de imágenes radiográficas de tórax. En la Figura 3-10 se muestra que CLAHE enfatiza el ruido en una radiografía sin neumonía, haciendo parecer que pudiera haber lesiones (segundo trío imágenes). En el tercer trío de la misma Figura 3 se ilustra que las regiones claras de una imagen con neumonía podrían ser oscurecidas por CLAHE, no siendo así cuando se usa SAHS. Finalmente, la Figura 3-11 se el ajuste de contraste SAHS para 3 imágenes diferentes.

\begin{figure}
    \centering
    \includegraphics[width=0.9\linewidth]{Figures/imagen33.png}
    \caption{Comparación original, CLAHE y SAHS en una radiografía, mostrando así que CLAHE agrega imperfecciones}
    \label{fig:clahe_vs_sahs_comparison}
\end{figure}

\begin{figure}
    \centering
    \includegraphics[width=0.9\linewidth]{Figures/imagen44.png}
    \caption{Comparación de imágenes radiográficas de tórax con neumonía antes y después de aplicar SAHS}
    \label{fig:sahs_pneumonia_examples}
\end{figure}
\subsection{Algoritmo Localizador de Pulmones (ALP)}

Para mejorar la precisión de la clasificación, hemos usado el Algoritmo Localizador de Pulmones (ALP) propuesto en \cite{picazo2023sistema}, que permite extraer o segmentar la región de interés (ROI) pulmonar en las radiografías. El ALP utiliza regresión K-NN para estimar las coordenadas de la ROI en una nueva imagen radiográfica de prueba.

El proceso del ALP se resume en los siguientes pasos:

\begin{enumerate}
    \item \textbf{Identificación de landmarks:} Se definen cuatro puntos clave (Q1, Q2, Q3, Q4) que delimitan la región pulmonar.
    \item \textbf{Regresión K-NN:} Se predicen las coordenadas de estos puntos en la imagen de prueba.
    \item \textbf{Warping:} Se aplica una transformación geométrica para extraer la ROI y normalizarla a una imagen de 256x256 píxeles.
\end{enumerate}

\begin{figure}[h]
    \centering
    \includegraphics[width=0.9\textwidth]{Figures/imagen55.png}
    \caption{Proceso ALP, imágenes de ejemplo con sus regiones de interés ya extraídas en imágenes procesadas con SAHS. Las coordenadas rojas son obtenidas por regresión y las azules se usan en la extracción de la ROI.}
    \label{fig:ALP_process}
\end{figure}

Este enfoque asegura que todas las imágenes procesadas tengan la misma alineación, lo cual mejora la clasificación \cite{Ayala2010, picazo2023sistema}.

\subsection{Integración de Métodos SAHS y ALP}

La combinación de SAHS con ALP proporciona un preprocesamiento robusto para las imágenes radiográficas de tórax:

\begin{enumerate}
    \item \textbf{Mejora de contraste inicial:} Se aplica el método de expansión asimétrica del histograma a la imagen original.
    \item \textbf{Localización pulmonar:} Se utiliza el ALP para extraer y normalizar la ROI.
\end{enumerate}

Este proceso se ilustra en la Figura \ref{fig:SAHS_ALP_comparison}.

\begin{figure}[h]
    \centering
    \includegraphics[width=0.9\textwidth]{Figures/imagen66.png}
    \caption{Comparación de una imagen radiográfica de tórax en las diferentes etapas del proceso integrado: original, después de aplicar SAHS, y después de obtener la ROI.}
    \label{fig:SAHS_ALP_comparison}
\end{figure}

\section{Evolución Hacia una Metodología de Segmentación Robusta}
\label{sec:evolucion_metodologia_segmentacion}

La metodología de localización de puntos de referencia previamente establecida proporciona un fundamento esencial para el análisis cuantitativo. Sin embargo, la tarea de segmentar de forma robusta los campos pulmonares, particularmente frente a las variaciones inherentes y la resolución limitada (299x299 píxeles) de nuestro conjunto de datos de radiografías de tórax, exigió la exploración de enfoques más avanzados.

En esta búsqueda, se tomó como referencia el trabajo de \cite{Mansoor2020Generic}, titulado 'A Generic Approach to Lung Field Segmentation from Chest Radiographs using Deep Space and Shape Learning', dada la aparente similitud en los desafíos iniciales. Se emprendió un esfuerzo por comprender y adaptar los métodos propuestos en dicha publicación. No obstante, durante esta fase, se confirmo que la considerable diferencia en la resolución de las imágenes, específicamente, el dataset de referencia empleaba imágenes de alta definición (2048x2048 píxeles), imponía limitaciones significativas. La menor cantidad de información y detalle en nuestras imágenes comprometió la efectividad de una extracción de características directamente análoga a la del paper, resultando en un rendimiento de predicción subóptimo.

Ante este escenario, y reconociendo la necesidad de una estrategia mejor adaptada a las particularidades de nuestro dataset, se procedió a desarrollar una metodología híbrida. Esta nueva aproximación, si bien inspirada por los conceptos del trabajo de referencia, integra de manera prominente Redes Neuronales Convolucionales (CNNs) para la extracción de características y la predicción de parámetros de forma. Se optó por las CNNs debido a su probada capacidad para aprender representaciones robustas a partir de imágenes con menor detalle y resolución. El desarrollo implicó una fase de experimentación con diversas arquitecturas de CNN, ajustando parámetros clave como la tasa de aprendizaje, el tamaño de los lotes (batches) y las dimensiones de los parches de imagen, hasta alcanzar un rendimiento que permitiera predicciones suficientemente precisas en relación con las anotaciones de referencia (ground truth). La siguiente sección detalla esta metodología de segmentación.
%\section{Metodología segmentación automática}
\label{sec:metodologia_simplified}

Esta sección describe la metodología para la segmentación automática de la región pulmonar en imágenes radiográficas de tórax, combinando Modelos Estadísticos de Forma (SSM) y Modelos Activos de Forma (ASM) con Redes Neuronales Convolucionales (CNN). Se detalla el proceso, comenzando por la representación inicial de forma mediante landmarks etiquetados manualmente, seguido de la densificación con interpolación spline para obtener una forma con detalles finos. Usando esta forma densificada se construye un SSM mediante Alineamiento de Procrustes Generalizado (GPA) y Análisis de Componentes Principales (PCA) para modelar la variabilidad de la forma de los pulmones y obtener una forma estándar, se extraen también perfiles de intensidad para construir un Modelo Estadístico de Apariencia (SAM) que captura la apariencia local, para tener una forma estándar de la apariencia alrededor de los contornos pulmonares. Continuamos con la Estimación de Pose Inicial (ESL) que utiliza clasificadores para predecir la transformación global que nos permitirá conocer el área que delimita a los pulmones en base a nuestro modelo estándar de forma y apariencia. El paso siguiente es extraer parches de imagen alrededor de los landmarks para entrenar una CNN, que aprenderá a predecir parámetros de forma discretizados. Estos parámetros se desdiscretizan y se refinan mediante un ajuste iterativo ASM para obtener un modelo robusto. Finalmente, se generan máscaras de segmentación Ground truth (GT) y se evalúa el rendimiento utilizando el Coeficiente de Dice (DSC).

\begin{figure}[htbp] 
    \centering
    \includegraphics[width=1\linewidth]{Figures/diagrama_bloques_1.png}
    \caption{Diagrama de bloques de la metodología de segmentación automática}
    \label{fig:diagrama_bloques_1}
\end{figure}

\section{Visión general de la metodología}
\label{sec:vision_metodologia_simplified}

\begin{enumerate}
    \item \textbf{Representando la Geometría Inicial (Sección~\ref{subsec:AdquisicionDatos}).}
        Todo comienza con una forma geométrica simplificada de los pulmones, un polígono. En lugar de tratar con cada píxel de la imagen, se identifican unos puntos clave (o "landmarks") que marcan lugares anatómicos alrededor de los contornos pulmonares. Se puede pensar en esto como unir de una manera muy básica los puntos de la forma del pulmón.
        Así se obtiene una representación numérica sencilla de la forma inicial de los pulmones para cada imagen.

    \item \textbf{Añadiendo Detalle a los Contornos (Sección~\ref{sec:densificacion_forma_simplificada}).}
        Los quince puntos iniciales no capturan las curvas suaves y los detalles finos de los pulmones. Para mejorar esto, se usa una técnica matemática (interpolación con splines) que ``dibuja'' curvas suaves a través de nuestros puntos iniciales. Luego, se toman muchas más muestras a lo largo de estas curvas suaves.
        De esta manera se consigue una representación mucho más detallada y suave de los contornos pulmonares, con muchos más puntos.

    \item \textbf{Aprendiendo la ``Esencia'' de la Forma Pulmonar (Sección~\ref{sec:ssm_simplified}).}
        Los pulmones varían mucho entre personas. se desea entender cuál es la ``forma pulmonar típica'' y cuáles son las maneras más comunes en que esta forma puede variar (ej., más ancha, más alargada). Para ello, se analizan muchas formas de pulmones de diferentes pacientes.
        \begin{itemize}
            \item \textit{Alineamiento (GPA, Sección~\ref{sec:gpa_simplified}):} Primero, se ``alinean'' todas estas formas para que estén en la misma posición, orientación y tamaño promedio. De esta manera todas las fotos de pulmones ahora tienen características similares antes de estudiarlas.
            \item \textit{Análisis de Variación (PCA, Sección~\ref{sec:pca_ssm_simplified}):} Luego, se identifican las ``direcciones'' o ``modos'' principales en los que las formas alineadas tienden a diferir. Esto da un ``modelo estadístico de form'' (SSM) que puede describir cualquier forma pulmonar plausible como una combinación de la forma media y estas variaciones.
        \end{itemize}
        Así se adquiere un modelo matemático compacto que conoce cómo es un pulmón ``promedio'' y cómo puede deformarse. Este modelo es clave para asegurar que las segmentaciones finales parezcan pulmones de verdad.

    \item \textbf{Comprendiendo la Apariencia de los Bordes (Sección~\ref{sec:sam_simplified}).}
        Para que el modelo se ajuste bien a una nueva imagen, necesita saber qué buscar. Para cada punto del modelo de forma, se aprende cómo se ve típicamente la imagen (los niveles de gris) en la zona perpendicular al borde del pulmón en ese punto. Esto crea ``modelos de perfil de intensidad''.
        De esta forma se extrae el conocimiento de la apariencia local esperada en los bordes pulmonares, que guiará el ajuste del modelo.

    \item \textbf{Encontrando una Ubicación Global Aproximada (Sección~\ref{sec:esl_simplified}).}
        Antes de ajustar los detalles finos, es necesario una idea general de dónde están los pulmones en una nueva imagen y cuáles son sus orientaciones y tamaños aproximados. Se usan unos ``detectores rápidos'' entrenados para encontrar una ``caja'' que enmarque los pulmones.
        Así se obtiene una estimación inicial de la posición, escala y rotación de los pulmones en la imagen.

    \item \textbf{Predicción Inteligente de la Forma con Redes Neuronales (Secciones~\ref{sec:extraccion_parches_cnn_simplified}, \ref{sec:entrenamiento_cnn_simplified}, y \ref{sec:prediccion_desdiscretizacion_b_simplified}).}
        Con la ubicación global estimada, ahora se desea una predicción más precisa de la forma específica de los pulmones en esa imagen. Se utilizan Redes Neuronales Convolucionales (CNNs).
        \begin{itemize}
            \item \textit{Entrada a la CNN:} Se extraen pequeños ``parches'' (trozos) de la imagen alrededor de donde se espera que estén los puntos del contorno pulmonar (basándose en la forma media y la estimación inicial de posición, escala y rotación.).
            \item \textit{Entrenamiento:} Se entrenan a las CNNs para que, mirando estos parches, aprendan a predecir los parámetros del modelo de forma (SSM) que mejor describen los pulmones en la imagen. Es decir, la CNN aprende qué tan ``estirada'' o ``encogida'' debe estar la forma media en cada una de sus direcciones de variación aprendidas.
            \item \textit{Predicción:} Dada una nueva imagen, la CNN predice estos parámetros de forma.
        \end{itemize}
        De esta manera se consigue una estimación inicial muy buena de la forma específica de los pulmones en la nueva imagen, guiada por el aprendizaje profundo a partir de los datos visuales.

    \item \textbf{Ajuste Fino y Refinamiento del Contorno (Sección~\ref{sec:ajuste_asm_simplified}).}
        La predicción de la CNN da un excelente punto de partida, pero es posible refinarlo aún más. Se usa un método iterativo (Active Shape Model - ASM) que toma la forma predicha y la "mueve" sutilmente. En cada paso:
        \begin{itemize}
            \item Para cada punto del contorno, busca en la imagen cercana (usando los ``modelos de perfil de intensidad'' del Paso 4) la posición que mejor parece un borde pulmonar.
            \item Ajusta la forma global para que se parezca a estos nuevos puntos encontrados, pero asegurándose de que la forma siga siendo ``realista'' según nuestro modelo SSM (Paso 3).
            \item Repite hasta que la forma ya no cambie significativamente.
        \end{itemize}
        Así se adquiere una forma final que se ajusta con precisión a los bordes pulmonares en la imagen, respetando al mismo tiempo las variaciones anatómicas aprendidas.

    \item \textbf{De Puntos a Regiones (Sección~\ref{sec:generacion_mascaras_simplified}).}
        Una vez que se tienen los puntos del contorno final, se conectan para formar polígonos y se ``rellena'' el interior. Esto crea una ``máscara'' binaria donde los píxeles dentro de los pulmones tienen un valor y los de fuera otro.
        De esta forma se extrae la segmentación final: una imagen donde la región pulmonar está claramente delimitada.

    \item \textbf{Midiendo el Éxito (Sección~\ref{sec:evaluacion_simplified}).}
        Para saber qué tan bien funciona el sistema, se comparan las máscaras predichas con máscaras ``perfectas'' (ground truth). Se usan métricas numéricas (como el Coeficiente de Dice) para cuantificar la similitud.
        Así se obtiene una evaluación objetiva del rendimiento de la predicción.
\end{enumerate}

Esta es la visualización general de la metodología. Cada una de estas etapas involucra conceptos y formulaciones matemáticas específicas que serán exploradas en las siguientes secciones. Se espera que esta visión general haya proporcionado un entendimiento claro del flujo de trabajo y la lógica detrás de cada componente de la metodología de segmentación automática de la región pulmonar.

\begin{figure}[htbp] 
    \centering
    \includegraphics[width=1\linewidth]{Figures/diagrama_bloques_2_2.drawio.png}
    \caption{Diagrama de bloques visual de la metodología}
    \label{fig:diagrama_bloques_visual_1}
\end{figure}

%\section{Adquisición y Representación Inicial de Datos Geométricos}
\label{subsec:AdquisicionDatos}

El punto de partida es la representación discreta de los contornos pulmonares mediante un conjunto de puntos de referencia, o \textit{landmarks}. Para cada imagen, se dispone de un conjunto inicial de $k = 15$ \textit{landmarks}. Cada \textit{landmark} $\mathbf{p}_i$ se define por sus coordenadas en el plano de la imagen:

\begin{equation}
\label{eq:landmark_definition}
\hspace*{\fill} \mathbf{p}_i = (x_i, y_i) \in \mathbb{R}^2, \quad \text{donde } i = 1, \dots, k. \hspace*{\fill}
\end{equation}

Estos \textit{landmarks} definen puntos clave a lo largo de los contornos de interés. Consecuentemente, dichos puntos se organizan en una matriz de forma, $\mathbf{S}$, para cada instancia, definida como:

\begin{equation}
\label{eq:shape_matrix}
\hspace*{\fill}
\mathbf{S} = \begin{bmatrix} x_1 & y_1 \\ x_2 & y_2 \\ \vdots & \vdots \\ x_k & y_k \end{bmatrix} % Cambio k_0 por k
\hspace*{\fill}
\end{equation}

Esta matriz $\mathbf{S}$, como se muestra en la Ecuación~\eqref{eq:shape_matrix}, encapsula la configuración geométrica de los \textit{landmarks} (definidos en la Ecuación~\eqref{eq:landmark_definition}) para una imagen específica.

\begin{figure}[htbp] 
\centering 
\includegraphics[width=1\linewidth]{Figures/initial_landmarks_visualization.png} 
\caption{Representación inicial de los contornos pulmonares mediante 15 \textit{landmarks} (ver Ecuación~\eqref{eq:landmark_definition}). Estos puntos discretos sirven como base para la representación de forma.}
\label{fig:initial_landmarks_visualization} 
\end{figure}
%% \section{Densificación de la Representación de Forma mediante Interpolación Spline}
% \label{sec:densificacion_forma}

% La representación inicial de la forma pulmonar, constituida por los $k=15$ \textit{landmarks} originales, si bien captura la estructura global, resulta insuficiente para modelar con precisión los detalles finos y la curvatura continua de los contornos pulmonares. Para superar esta limitación, se emplea un proceso de densificación basado en la teoría de aproximación de curvas, específicamente mediante la interpolación con splines cúbicos. Este enfoque permite generar una representación de forma más rica y detallada, con un mayor número de puntos de referencia.

% Dada una secuencia ordenada de $k_s$ \textit{landmarks} de un segmento del contorno (por ejemplo, un lóbulo pulmonar):
% \begin{equation}
% \label{eq:landmark_segment_sequence} 
% \hspace*{\fill}
% \{\mathbf{p}_j\}_{j=1}^{k_s},
% \hspace*{\fill}
% \end{equation}
% se construye una curva paramétrica continua, $\mathbf{c}(t)$, que interpola estos puntos:
% \begin{equation}
% \label{eq:parametric_curve_definition} 
% \hspace*{\fill}
% \mathbf{c}(t) = (x(t), y(t)).
% \hspace*{\fill}
% \end{equation}
% Los splines cúbicos (o B-splines de grado 3) son seleccionados por sus propiedades de suavidad, garantizando continuidad hasta la segunda derivada ($C^2$) en los puntos de unión de los segmentos de la curva. Esto es crucial para evitar artefactos angulares en la representación de la forma.

% El proceso de interpolación y densificación se aplica de manera independiente a cada uno de los dos contornos principales (correspondientes a los lóbulos pulmonares) definidos a partir de los \textit{landmarks} iniciales. Para una secuencia $\{\mathbf{p}_j\}_{j=1}^{k_s}$ de un contorno (Ecuación~\eqref{eq:landmark_segment_sequence}), el primer paso es establecer una parametrización. Se asigna un valor paramétrico $u_j$ a cada \textit{landmark} $\mathbf{p}_j$, comúnmente utilizando la longitud de cuerda acumulada para aproximar una parametrización por longitud de arco:
% \begin{equation}
% \label{eq:param_u_initial} 
% \hspace*{\fill}
% u_1 = 0,
% \hspace*{\fill}
% \end{equation}
% \begin{equation}
% \label{eq:param_ui_cumulative} % Etiqueta renombrada para claridad
% \hspace*{\fill}
% u_j = u_{j-1} + ||\mathbf{p}_j - \mathbf{p}_{j-1}||_2, \quad \text{para } j = 2, \dots, k_s.
% \hspace*{\fill}
% \end{equation}

% \begin{figure}[htbp] 
% \centering 
% \includegraphics[width=1\linewidth]{Figures/figura_parametrizacion_cuerda_v7.png} 
% \caption{Ilustración del método de parametrización por longitud de cuerda acumulada para una secuencia de \textit{landmarks} $\mathbf{p}_{j-2}, \mathbf{p}_{j-1}, \mathbf{p}_j$. Los valores paramétricos $u_j$ (ilustrados como $u_1, u_2, u_3$ para los primeros puntos) se calculan según las Ecuaciones~\eqref{eq:param_u_initial} y~\eqref{eq:param_ui_cumulative}, donde cada $u_j$ representa la suma de las distancias euclidianas entre \textit{landmarks} consecutivos a partir del punto inicial.}
% \label{fig:parametrizacion_cuerda}
% \end{figure}

% Posteriormente, se ajusta una curva spline cúbica $\mathbf{c}(u)$ que interpola estos \textit{landmarks} en sus correspondientes valores paramétricos, es decir, se cumple la condición:
% \begin{equation}
% \label{eq:spline_interpolation_condition}
% \hspace*{\fill}
% \mathbf{c}(u_j) = \mathbf{p}_j, \quad \text{para } j = 1, \dots, k_s.
% \hspace*{\fill}
% \end{equation}

% Una vez obtenida la representación continua $\mathbf{c}(u)$, cuyo dominio paramétrico para este contorno es $[u_1, u_{k_s}]$, la curva se remuestrea uniformemente para generar un conjunto más denso de $k_d = 72$ puntos por contorno (donde el subíndice 'd' indica "densificado"). Los nuevos valores paramétricos $v_l$ para estos puntos densificados se calculan como:
% \begin{equation}
% \label{eq:resampling_param}
% \hspace*{\fill}
% v_l = u_1 + (l-1) \frac{u_{k_s} - u_1}{k_d - 1}, \quad \text{para } l = 1, \dots, k_d.
% \hspace*{\fill}
% \end{equation}
% Los nuevos \textit{landmarks} densificados para este contorno, denotados como $\mathbf{q}_l$, se obtienen evaluando la curva spline en estos nuevos parámetros:
% \begin{equation}
% \label{eq:resampled_points}
% \hspace*{\fill}
% \mathbf{q}_l = \mathbf{c}(v_l), \quad \text{para } l = 1, \dots, k_d.
% \hspace*{\fill}
% \end{equation}

% Este procedimiento se aplica a los dos contornos principales. Si denotamos los puntos densificados del primer contorno como $\{\mathbf{q}^{(1)}_l\}_{l=1}^{k_d}$ y los del segundo contorno como $\{\mathbf{q}^{(2)}_l\}_{l=1}^{k_d}$, la matriz de forma densificada completa, $\mathbf{S}' \in \mathbb{R}^{K_{total} \times 2}$ con $K_{total}=2k_d=144$ puntos, se construye concatenando estos dos conjuntos:
% \begin{equation}
% \label{eq:shape_matrix_densified}
% \hspace*{\fill}
% \mathbf{S}' = \begin{bmatrix} \mathbf{q}^{(1)}_1 \\ \vdots \\ \mathbf{q}^{(1)}_{k_d} \\ \mathbf{q}^{(2)}_1 \\ \vdots \\ \mathbf{q}^{(2)}_{k_d} \end{bmatrix}.
% \hspace*{\fill}
% \end{equation}
% Esta matriz $\mathbf{S}'$ proporciona una descripción significativamente más detallada del contorno pulmonar.

% Matemáticamente, la curva spline $\mathbf{c}(u)$ que satisface la Ecuación~\eqref{eq:spline_interpolation_condition} se expresa como una combinación lineal de funciones base B-spline $N_{i,p}(u)$: 
% \begin{equation}
% \label{eq:spline_curve_definition}
% \hspace*{\fill}
% \mathbf{c}(u) = \sum_{i=0}^{n} \mathbf{d}_i N_{i,p}(u), % Cambiado j por i en d_i y N_i,p
% \hspace*{\fill}
% \end{equation}
% donde $\mathbf{d}_i \in \mathbb{R}^2$ son los coeficientes del spline (puntos de control de De Boor), $p$ es el grado del spline (en este caso, $p=3$ para splines cúbicos), y $n+1$ es el número de puntos de control. Las funciones base B-spline $N_{i,p}(u)$ se definen recursivamente sobre un vector de nodos $\mathbf{t} = (t_0, t_1, \dots, t_m)$, donde $m = n+p+1$. La recursión de Cox-de Boor es:
% \begin{equation}
% \label{eq:bspline_basis_0}
% \hspace*{\fill}
% N_{i,0}(u) = \begin{cases} 1 & \text{si } t_i \le u < t_{i+1} \\ 0 & \text{en otro caso} \end{cases} % Cambiado j por i
% \hspace*{\fill}
% \end{equation}
% \begin{equation}
% \label{eq:bspline_basis_p}
% \hspace*{\fill}
% N_{i,p}(u) = \frac{u - t_i}{t_{i+p} - t_i} N_{i,p-1}(u) + \frac{t_{i+p+1} - u}{t_{i+p+1} - t_{i+1}} N_{i+1,p-1}(u). % Cambiado j por i
% \hspace*{\fill}
% \end{equation}
% El proceso de ajuste del spline (Ecuación~\eqref{eq:spline_interpolation_condition}) implica determinar los coeficientes $\mathbf{d}_i$ y el vector de nodos $\mathbf{t}$ de tal manera que la curva resultante interpole los \textit{landmarks} $\mathbf{p}_j$ (de la Ecuación~\eqref{eq:landmark_segment_sequence}) en los valores paramétricos $u_j$.

% \begin{figure}[htbp]
% \centering
% \includegraphics[width=0.8\linewidth]{Figures/spline_densification_visualization.png} 
% \caption{Proceso de densificación de forma mediante interpolación spline. Se muestran los \textit{landmarks} iniciales para un contorno (puntos rojos), parametrizados según Ecuaciones~\eqref{eq:param_u_initial} y~\eqref{eq:param_ui_cumulative}. Una curva spline cúbica (línea colorida) se ajusta para interpolar estos \textit{landmarks} (Ecuación~\eqref{eq:spline_interpolation_condition}), definida por las Ecuaciones~\eqref{eq:spline_curve_definition}-\eqref{eq:bspline_basis_p}. Finalmente, se obtienen puntos densificados (puntos azules, $k_d=72$ por contorno) mediante remuestreo uniforme (Ecuaciones~\eqref{eq:resampling_param} y~\eqref{eq:resampled_points}). La matriz final $\mathbf{S}'$ (Ecuación~\eqref{eq:shape_matrix_densified}) combina los puntos de ambos lóbulos.}
% \label{fig:spline_interpolation_detailed}
% \end{figure}

% La elección de $K_{total}=144$ puntos densificados representa un equilibrio entre la fidelidad de la representación de la forma y la complejidad computacional del modelo estadístico de forma subsecuente. Esta densificación es un paso crucial para capturar la variabilidad morfológica de los pulmones con mayor granularidad.

\section{Densificación de la Representación de Forma mediante Interpolación Spline}
\label{sec:densificacion_forma_simplificada}

La representación inicial de la forma, compuesta por un conjunto de \landmarks{}, puede no ser suficiente para capturar detalles finos. Para mejorar esto, se utiliza la interpolación con splines cúbicos, generando una representación más detallada con más puntos.

Dado un conjunto ordenado de $k_s$ \landmarks{} iniciales de un contorno:
\begin{equation}
\label{eq:landmark_sequence_simplified} 
\hspace*{\fill}
\{\mathbf{p}_j\}_{j=1}^{k_s},
\hspace*{\fill}
\end{equation}
donde cada $\mathbf{p}_j = (x_j, y_j)$ es un punto en 2D.
Se busca construir una curva paramétrica continua, $\mathbf{c}(u)$, que pase a través de estos puntos:
\begin{equation}
\label{eq:parametric_curve_definition_simplified} 
\hspace*{\fill}
\mathbf{c}(u) = (x(u), y(u)).
\hspace*{\fill}
\end{equation}
Los splines cúbicos se eligen por su suavidad, asegurando una curvatura continua.

A cada \landmark{} $\mathbf{p}_j$ se le asigna un valor paramétrico $u_j$. Luego, se ajusta una curva spline cúbica $\mathbf{c}(u)$ tal que interpola estos \landmarks{} en sus correspondientes valores paramétricos:
\begin{equation}
\label{eq:spline_interpolation_condition_simplified}
\hspace*{\fill}
\mathbf{c}(u_j) = \mathbf{p}_j, \quad \text{para } j = 1, \dots, k_s.
\hspace*{\fill}
\end{equation}

Una vez obtenida la curva continua $\mathbf{c}(u)$ (definida en un rango paramétrico, por ejemplo, $[u_{min}, u_{max}]$), esta se remuestrea para generar un conjunto más denso de $k_d$ puntos. Los nuevos valores paramétricos $v_l$ para estos puntos densificados se pueden calcular, por ejemplo, mediante un espaciado uniforme:
\begin{equation}
\label{eq:resampling_param_simplified}
\hspace*{\fill}
v_l = u_{min} + (l-1) \frac{u_{max} - u_{min}}{k_d - 1}, \quad \text{para } l = 1, \dots, k_d.
\hspace*{\fill}
\end{equation}
Los nuevos \landmarks{} densificados, $\mathbf{q}_l$, se obtienen evaluando la curva spline en estos nuevos parámetros:
\begin{equation}
\label{eq:resampled_points_simplified}
\hspace*{\fill}
\mathbf{q}_l = \mathbf{c}(v_l), \quad \text{para } l = 1, \dots, k_d.
\hspace*{\fill}
\end{equation}

Si se procesan múltiples contornos (por ejemplo, dos lóbulos pulmonares), los puntos densificados de cada uno, $\{\mathbf{q}^{(1)}_l\}_{l=1}^{k_d}$ y $\{\mathbf{q}^{(2)}_l\}_{l=1}^{k_d}$, se pueden agrupar para formar una matriz de forma densificada completa, $\mathbf{S}' \in \mathbb{R}^{(N \cdot k_d) \times 2}$ (donde $N$ es el número de contornos):
\begin{equation}
\label{eq:shape_matrix_densified_simplified}
\hspace*{\fill}
\mathbf{S}' = \begin{bmatrix} \mathbf{q}^{(1)}_1 \\ \vdots \\ \mathbf{q}^{(1)}_{k_d} \\ \mathbf{q}^{(2)}_1 \\ \vdots \\ \mathbf{q}^{(2)}_{k_d} \\ \vdots \end{bmatrix}.
\hspace*{\fill}
\end{equation}

Matemáticamente, la curva spline $\mathbf{c}(u)$ (Ecuación~\eqref{eq:spline_interpolation_condition_simplified}) se puede expresar como una combinación lineal de funciones base B-spline $N_{i,p}(u)$:
\begin{equation}
\label{eq:spline_curve_definition_simplified}
\hspace*{\fill}
\mathbf{c}(u) = \sum_{i=0}^{n} \mathbf{d}_i N_{i,p}(u),
\hspace*{\fill}
\end{equation}
donde $\mathbf{d}_i$ son los puntos de control que definen la forma de la curva, $p$ es el grado del spline (usualmente $p=3$ para splines cúbicos), y $N_{i,p}(u)$ son las funciones base B-spline que ponderan la influencia de estos puntos de control a lo largo del parámetro $u$. El ajuste del spline implica determinar los coeficientes $\mathbf{d}_i$ y otros parámetros necesarios para que la curva interpole los \landmarks{} $\mathbf{p}_j$.

\begin{figure}[htbp]
\centering
% Se asume que la imagen sigue siendo relevante para el proceso general.
% Actualiza el path si es necesario.
\includegraphics[width=0.9\linewidth]{Figures/spline_densification_visualization.png} 
\caption{Proceso de densificación de forma. (Puntos rojos) \textit{Landmarks} iniciales $\mathbf{p}_j$. (Línea verde y magenta) Curva spline cúbica $\mathbf{c}(u)$ que interpola los puntos (Ecuación~\eqref{eq:spline_interpolation_condition_simplified}), definida como en Ecuación~\eqref{eq:spline_curve_definition_simplified}. (Puntos azules) Puntos densificados $\mathbf{q}_l$ obtenidos mediante remuestreo (Ecuaciones~\eqref{eq:resampling_param_simplified} y~\eqref{eq:resampled_points_simplified}). La matriz final $\mathbf{S}'$ (Ecuación~\eqref{eq:shape_matrix_densified_simplified}) agrupa estos puntos.}
\label{fig:spline_interpolation_simplified}
\end{figure}

Esta densificación es un paso importante para capturar la variabilidad morfológica con mayor detalle.
%% \section{Construcción del Modelo Estadístico de Forma (SSM)}
% \label{sec:ssm}

% La variabilidad inherente en la morfología de los pulmones se observa a través de un conjunto de entrenamiento compuesto por $N$ nubes de puntos (formas) densificadas. Como se detalló en la Sección~\ref{sec:densificacion_forma}, cada forma consta de $K_{total}=144$ puntos. Este conjunto de formas de entrenamiento se define formalmente como:
% \begin{equation}
% \hspace*{\fill}
% \mathcal{S}' = \left\{ \mat{S}'^{(j)} \in \R^{K_{total} \times d} \mid j = 1, \dots, N \right\},
% \label{eq:training_shapes_set}
% \hspace*{\fill}
% \end{equation}
% donde cada forma $\mat{S}'^{(j)}$ es una matriz que representa $K_{total}$ puntos en $d=2$ dimensiones (2D), y el superíndice $'$ denota que estas son las formas antes del alineamiento.

% \begin{figure}[htbp] % Posicionamiento flexible
%     \centering
%     \includegraphics[width=0.8\columnwidth]{Figures/01_raw_training_shapes.png}
%     \caption{Ejemplos de formas de entrenamiento $\mat{S}'^{(j)}$ del conjunto $\mathcal{S}'$ antes del alineamiento, mostrando la variabilidad inherente en posición, escala y orientación.}
%     \label{fig:raw_shapes}
% \end{figure}

% Dicha variabilidad se captura y modela de manera compacta y eficiente mediante un Modelo Estadístico de Forma (SSM) lineal. La construcción de un SSM robusto es un pilar fundamental de nuestra metodología, permitiendo representar cualquier forma pulmonar plausible como una deformación de una forma media, controlada por un conjunto reducido de parámetros. Este proceso consta de dos etapas cruciales: el alineamiento de las formas de entrenamiento y la subsecuente aplicación del Análisis de Componentes Principales (PCA).

% \subsection{Alineamiento Procrustes Generalizado (GPA)}
% \label{sec:gpa}
% Antes de poder analizar estadísticamente la variabilidad de la forma, es imperativo eliminar las variaciones extrínsecas debidas a la traslación, rotación y escala uniforme presentes en las formas de entrenamiento. El Alineamiento Procrustes Generalizado (GPA) \cite{gower1975generalized} es un método iterativo estándar para este fin. El GPA alinea simultáneamente todo el conjunto de $N$ formas a un espacio común, minimizando una medida global de la diferencia de forma.

% El proceso, detallado en el Algoritmo~\ref{alg:gpa}, comienza seleccionando una forma de referencia inicial. Subsecuentemente, cada forma del conjunto se alinea a esta referencia. Para alinear una forma $\mat{S}_{\text{target}}$ a una forma de referencia $\mat{S}_{\text{ref}}$, se busca la transformación de similitud óptima (escala $s \in \R^+$, matriz de rotación $\mat{R} \in \text{SO}(d)$, y vector de traslación $\vect{t}_0 \in \R^d$) que minimiza la suma de las distancias Euclidianas al cuadrado entre los puntos correspondientes:
% \begin{equation}
% \hspace*{\fill}
% E(\mat{S}_{\text{ref}}, \mat{S}_{\text{target}}) = \min_{s, \mat{R}, \vect{t}_0} \left\| \mat{S}_{\text{ref}} - \left(s \mat{S}_{\text{target}} \mat{R} + \mathbf{1}\vect{t}_0^\transpose\right) \right\|_F^2,
% \label{eq:procrustes_error}
% \hspace*{\fill}
% \end{equation}
% donde $\| \cdot \|_F$ denota la norma de Frobenius, $\mat{S}_{\text{ref}}, \mat{S}_{\text{target}} \in \R^{K_{total} \times d}$, $\mat{R} \in \R^{d \times d}$ es la matriz de rotación, $\vect{t}_0 \in \R^{d \times 1}$ es el vector de traslación, y $\mathbf{1} \in \R^{K_{total} \times 1}$ es un vector columna de unos. La solución a este problema de minimización (es decir, encontrar $s, \mat{R}, \vect{t}_0$ para transformar $\mat{S}_{\text{target}}$) implica los siguientes pasos:
% \begin{enumerate}
%     \item Centrar ambas formas en el origen. Se calculan los centroides $\overline{\vect{s}}_{\text{ref}} = \frac{1}{K_{total}}\mat{S}_{\text{ref}}^\transpose\mathbf{1}$ y $\overline{\vect{s}}_{\text{target}} = \frac{1}{K_{total}}\mat{S}_{\text{target}}^\transpose\mathbf{1}$ (vectores $d \times 1$). Las formas centradas son:
%     \begin{equation}
%     \hspace*{\fill}
%     \mat{S}_{\text{ref,c}} = \mat{S}_{\text{ref}} - \mathbf{1}\overline{\vect{s}}_{\text{ref}}^\transpose, \quad
%     \mat{S}_{\text{target,c}} = \mat{S}_{\text{target}} - \mathbf{1}\overline{\vect{s}}_{\text{target}}^\transpose.
%     \label{eq:centering_proc}
%     \hspace*{\fill}
%     \end{equation}

%     \begin{figure}[htbp] 
%     \centering
%     \includegraphics[width=0.6\columnwidth]{Figures/02_centering_one_shape.png}
%     \caption{Ilustración del proceso de centrado de una forma. Izquierda: forma original con su centroide. Derecha: la misma forma después de trasladar su centroide al origen.}
%     \label{fig:centering_shape}
% \end{figure}

%     \item Resolver el problema de Procrustes Ortogonal para encontrar la matriz de rotación $\mat{R}$ que mejor alinea $\mat{S}_{\text{target,c}}$ con $\mat{S}_{\text{ref,c}}$. Se calcula la matriz de covarianza cruzada $\mat{M} \in \R^{d \times d}$:
%     \begin{equation}
%     \hspace*{\fill}
%         \mat{M} = \mat{S}_{\text{target,c}}^\transpose \mat{S}_{\text{ref,c}}.
%         \label{eq:procrustes_M}
%         \hspace*{\fill}
%     \end{equation}
%     Se realiza la Descomposición en Valores Singulares (SVD) de $\mat{M}$, tal que $\mat{M} = \mat{U} \mat{\Sigma} \mat{V}^\transpose$. La matriz de rotación óptima $\mat{R}$ se obtiene como:
%     \begin{equation}
%     \hspace*{\fill}
%         \mat{R} = \mat{V} \mat{U}^\transpose.
%         \label{eq:procrustes_R}
%         \hspace*{\fill}
%     \end{equation}
%     Es crucial asegurar que $\mat{R}$ sea una rotación propia ($\det(\mat{R})=1$). Si $\det(\mat{R})=-1$, se puede ajustar, por ejemplo, multiplicando la última columna de $\mat{V}$ por $-1$ si $d=2$ y luego recalculando $\mat{R}$, o de forma más general, $\mat{R} = \mat{V} \text{diag}(1, \dots, \det(\mat{V}\mat{U}^\transpose)) \mat{U}^\transpose$.

%     \item Calcular la escala óptima $s$:
%     \begin{equation}
%     \hspace*{\fill}
%     s = \frac{\text{tr}\left(\mat{S}_{\text{ref,c}}^\transpose \mat{S}_{\text{target,c}} \mat{R}\right)}{\text{tr}\left(\mat{S}_{\text{target,c}}^\transpose \mat{S}_{\text{target,c}}\right)} = \frac{\text{tr}\left(\mat{M}^\transpose \mat{R}\right)}{\left\|\mat{S}_{\text{target,c}}\right\|_F^2}.
%     \label{eq:procrustes_scale}
%     \hspace*{\fill}
%     \end{equation}

%     \item Calcular la traslación óptima $\vect{t}_0$ (vector columna $d \times 1$):
%     \begin{equation}
%     \hspace*{\fill}
%     \vect{t}_0 = \overline{\vect{s}}_{\text{ref}} - s \mat{R} \overline{\vect{s}}_{\text{target}}.
%     \label{eq:procrustes_translation}
%     \hspace*{\fill}
%     \end{equation}
% \end{enumerate}

% Una vez todas las formas se han alineado con respecto a la referencia actual, se calcula una nueva forma media a partir del conjunto de formas alineadas. Esta nueva media se convierte en la forma de referencia para la siguiente iteración del algoritmo. El proceso iterativo continúa hasta que se alcanza la convergencia (e.g., cuando el cambio en la forma media entre iteraciones sucesivas cae por debajo de una tolerancia $\epsilon > 0$). Al finalizar, el resultado es un conjunto de $N$ formas alineadas, denotado como $\tilde{\mathcal{S}}$, que residen en un espacio de forma común:
% \begin{equation}
% \hspace*{\fill}
% \tilde{\mathcal{S}} = \left\{ \tilde{\mat{S}}^{(j)} \in \R^{K_{total} \times d} \mid j = 1, \dots, N \right\},
% \label{eq:aligned_shapes_set_gpa}
% \hspace*{\fill}
% \end{equation}
% donde cada $\tilde{\mat{S}}^{(j)}$ representa una forma individual después del alineamiento GPA, y el superíndice $\sim$ denota alineamiento.

% \begin{figure}[htbp] % Posicionamiento flexible
%     \centering
%     \includegraphics[width=0.6\textwidth]{Figures/06_final_aligned_shapes.png}
%     \caption{Conjunto de formas de entrenamiento $\tilde{\mathcal{S}}$ después del Alineamiento Procrustes Generalizado, superpuestas en un espacio de forma común. La forma media final $\overline{\mat{S}}$ se muestra en rojo.}
%     \label{fig:aligned_shapes_gpa}
% \end{figure}

% % El Algoritmo 1 se mantiene como lo proporcionaste, con las siguientes consideraciones:
% % 1. Usar K_total en lugar de k.
% % 2. Asegurar que las transposiciones usan \transpose.
% % 3. Las variables de formas (S_ref, S_target, S_ref_c, etc.) son consistentes con el texto.
% % 4. El resultado de alinear S_target a S_ref es s * S_target * R + 1*t_0^T.
% % El algoritmo es bastante detallado y extenso para reproducirlo aquí con todos los micro-ajustes, 
% % pero sigue las correcciones de notación y variables aplicadas en el texto.
% % Asumo que el lector puede seguir estas directrices para el pseudocódigo.
% % Lo más importante es:
% % - Línea 1: S^{(j)} \in \R^{K_{total} \times d}
% % - Línea 26: \tilde{S}^{(j)} <- s * S^{(j)} * R + 1 * t_0^T (usando S^{(j)} original)

% \begin{algorithm}[htbp]
% \caption{Alineamiento Procrustes Generalizado (GPA)}
% \label{alg:gpa}
% \begin{algorithmic}[1]
%     \State \textbf{Entrada:} Conjunto de $N$ formas $\{\mat{S}'^{(j)} \in \R^{\Ktotal \times \dval}\}_{j=1}^N$. Tolerancia $\epsilon > 0$.
%     \State \textbf{Salida:} Conjunto de $N$ formas alineadas $\{\tilde{\mat{S}}^{(j)}\}_{j=1}^N$. Forma media $\overline{\mat{S}}$.
%     \Statex
%     \State \textbf{Inicialización:}
%     \State \quad $\mat{S}_{\text{ref\_iter}} \leftarrow \mat{S}'^{(1)}$ \Comment{Seleccionar referencia inicial}
%     \State \quad $\overline{\mat{S}}_{\text{prev}} \leftarrow \mat{0}^{\Ktotal \times \dval}$
%     \State \quad \textbf{Para} $j = 1 \text{ \textbf{hasta} } N$ \textbf{hacer}
%     \State \quad \quad \Comment{Alinear $\mat{S}'^{(j)}$ a $\mat{S}_{\text{ref\_iter}}$}
%     \State \quad \quad Calcular $s_j, \mat{R}_j, \vect{t}_{0,j}$ para alinear $\mat{S}'^{(j)}$ a $\mat{S}_{\text{ref\_iter}}$
%     \State \quad \quad $\tilde{\mat{S}}^{(j)}_{\text{temp}} \leftarrow s_j \mat{S}'^{(j)} \mat{R}_j + \mathbf{1}\vect{t}_{0,j}^\transpose$
%     \State \quad \quad $\overline{\mat{S}}_{\text{prev}} \leftarrow \overline{\mat{S}}_{\text{prev}} + \tilde{\mat{S}}^{(j)}_{\text{temp}}$
%     \State \quad \textbf{Fin Para}
%     \State \quad $\overline{\mat{S}}_{\text{prev}} \leftarrow \overline{\mat{S}}_{\text{prev}} / N$
%     \State \quad \Comment{Opcional: Normalizar $\overline{\mat{S}}_{\text{prev}}$ (escala, orientación)}
%     \Statex
%     \State \textbf{Repetir}
%     \State \quad $\overline{\mat{S}}_{\text{curr}} \leftarrow \mat{0}^{\Ktotal \times \dval}$
%     \State \quad \textbf{Para} $j = 1 \text{ \textbf{hasta} } N$ \textbf{hacer}
%     \State \quad \quad \Comment{Alinear $\mat{S}'^{(j)}$ (original) a la media actual $\overline{\mat{S}}_{\text{prev}}$}
%     \State \quad \quad Calcular $s_j, \mat{R}_j, \vect{t}_{0,j}$ para alinear $\mat{S}'^{(j)}$ a $\overline{\mat{S}}_{\text{prev}}$
%     \State \quad \quad $\tilde{\mat{S}}^{(j)}_{\text{temp}} \leftarrow s_j \mat{S}'^{(j)} \mat{R}_j + \mathbf{1}\vect{t}_{0,j}^\transpose$
%     \State \quad \quad $\overline{\mat{S}}_{\text{curr}} \leftarrow \overline{\mat{S}}_{\text{curr}} + \tilde{\mat{S}}^{(j)}_{\text{temp}}$
%     \State \quad \textbf{Fin Para}
%     \State \quad $\overline{\mat{S}}_{\text{curr}} \leftarrow \overline{\mat{S}}_{\text{curr}} / N$
%     \State \quad \Comment{Opcional: Normalizar $\overline{\mat{S}}_{\text{curr}}$}
%     \State \quad \textbf{Si} $\| \overline{\mat{S}}_{\text{curr}} - \overline{\mat{S}}_{\text{prev}} \|_F < \epsilon$ \textbf{entonces}
%     \State \quad \quad \textbf{romper bucle}
%     \State \quad \textbf{Fin Si}
%     \State \quad $\overline{\mat{S}}_{\text{prev}} \leftarrow \overline{\mat{S}}_{\text{curr}}$
%     \State \textbf{Hasta} convergencia
%     \Statex
%     \State $\overline{\mat{S}} \leftarrow \overline{\mat{S}}_{\text{curr}}$
%     \State \Comment{Alinear todas las $\mat{S}'^{(j)}$ originales a la $\overline{\mat{S}}$ final}
%     \State \textbf{Para} $j = 1 \text{ \textbf{hasta} } N$ \textbf{hacer}
%     \State \quad Calcular $s_j, \mat{R}_j, \vect{t}_{0,j}$ para alinear $\mat{S}'^{(j)}$ a $\overline{\mat{S}}$
%     \State \quad $\tilde{\mat{S}}^{(j)} \leftarrow s_j \mat{S}'^{(j)} \mat{R}_j + \mathbf{1}\vect{t}_{0,j}^\transpose$
%     \State \textbf{Fin Para}
% \end{algorithmic}
% \end{algorithm}

% \subsection{Análisis de Componentes Principales (PCA)}
% \label{sec:pca_ssm}

% Con las formas de entrenamiento alineadas $\{\tilde{\mat{S}}^{(j)}\}_{j=1}^N$ del conjunto $\tilde{\mathcal{S}}$ (Ecuación~\eqref{eq:aligned_shapes_set_gpa}), se construye el modelo lineal de variabilidad mediante PCA. Primero, cada forma alineada $\tilde{\mat{S}}^{(j)} \in \R^{K_{total} \times d}$ se vectoriza concatenando sus $K_{total} \times d$ coordenadas de landmarks en un único vector $\tilde{\vect{s}}^{(j)} \in \R^{K_{total}d}$. Estos $N$ vectores de forma alineada se utilizan para el análisis.

% La forma media del conjunto de entrenamiento alineado se calcula como:
% \begin{equation}
% \hspace*{\fill}
% \overline{\vect{s}} = \frac{1}{N} \sum_{j=1}^N \tilde{\vect{s}}^{(j)}.
% \label{eq:mean_shape_vector} % Nueva etiqueta para evitar conflicto si \overline{\mat{S}} también se define
% \hspace*{\fill}
% \end{equation}
% Posteriormente, cada vector de forma $\tilde{\vect{s}}^{(j)}$ se centra restando el vector de forma media: $\vect{s}_{\text{cent}}^{(j)} = \tilde{\vect{s}}^{(j)} - \overline{\vect{s}}$. Estos $N$ vectores centrados $\vect{s}_{\text{cent}}^{(j)}$ se disponen como las filas de una matriz de datos centrados $\mat{X}_{\text{cent}} \in \R^{N \times K_{total}d}$.
% La matriz de covarianza de los datos $\mat{C} \in \R^{K_{total}d \times K_{total}d}$ se estima como:
% \begin{equation}
% \hspace*{\fill}
% \mat{C} = \frac{1}{N-1} \mat{X}_{\text{cent}}^\transpose \mat{X}_{\text{cent}}.
% \label{eq:covariance_matrix}
% \hspace*{\fill}
% \end{equation}
% Los eigenvectores de $\mat{C}$, denotados $\vect{p}_i \in \R^{K_{total}d}$, son los componentes principales (modos de variación), y los eigenvalores correspondientes $\lambda_i$ indican la varianza explicada por cada componente. Se ordenan tal que $\lambda_1 \ge \lambda_2 \ge \dots \ge \lambda_{K_{total}d} \ge 0$.

% Los $m$ eigenvectores $\vect{p}_1, \dots, \vect{p}_m$ asociados con los $m$ eigenvalores más grandes (y por tanto, mayor varianza) se organizan como las columnas de una matriz de modos de variación $\mat{P} = [\vect{p}_1, \dots, \vect{p}_m] \in \R^{K_{total}d \times m}$. El número de modos $m$ se elige típicamente para capturar un porcentaje deseado de la varianza total (e.g., 95\%) o se fija a un valor predefinido, donde $m \le \min(N-1, K_{total}d)$.

% Cualquier forma $\vect{s}$ (representada como un vector en el espacio alineado) perteneciente al espacio modelado por el SSM puede entonces ser aproximada como una combinación lineal de la forma media y los modos de variación:
% \begin{equation}
% \hspace*{\fill}
% \vect{s}(\vect{b}) = \overline{\vect{s}} + \mat{P} \vect{b},
% \label{eq:ssm_reconstruction}
% \hspace*{\fill}
% \end{equation}
% donde $\vect{b} = (b_1, \dots, b_m)^\transpose \in \R^m$ es un vector de $m$ parámetros de forma. Cada parámetro $b_i$ controla la magnitud de la variación a lo largo del $i$-ésimo modo principal $\vect{p}_i$. La varianza de $b_i$ en el conjunto de entrenamiento es $\lambda_i$, por lo que $\sqrt{\lambda_i}$ es su desviación estándar. Esto permite restringir los parámetros $\vect{b}$ a un rango plausible, típicamente $b_i \in [-c\sqrt{\lambda_i}, c\sqrt{\lambda_i}]$ (e.g., $c=3$), para generar formas realistas.

% La construcción del SSM proporciona un modelo paramétrico compacto de la forma pulmonar, esencial para etapas posteriores de predicción y ajuste.

% \begin{figure}[htbp] 
%     \centering
%     \includegraphics[width=0.9\columnwidth]{Figures/pca_explained_variance_visualization.png}
%     \caption{Gráfico de varianza explicada acumulada por los componentes principales del SSM. Este tipo de gráfico ayuda a determinar el número de modos $m$ necesarios para capturar un porcentaje deseado de la variabilidad total de la forma en el conjunto de entrenamiento.}
%     \label{fig:pca_variance}
% \end{figure}

% \begin{figure}[htbp] 
%     \centering
%     \includegraphics[width=1\columnwidth]{Figures/pca_mode_1_visualization_144pts.png} 
%     \caption{Visualización de la variación de la forma inducida por los primeros modos principales del SSM. Se muestra la forma media $\overline{\vect{s}}$ y las variaciones a lo largo de los primeros modos (e.g., $\overline{\vect{s}} \pm c\sqrt{\lambda_i}\vect{p}_i$). Estos modos capturan las deformaciones más significativas observadas en el conjunto de entrenamiento.}
%     \label{fig:ssm_mode_variation_detailed}
% \end{figure}

\section{Construcción del Modelo Estadístico de Forma (SSM)}
\label{sec:ssm_simplified}

La variabilidad en la morfología de los pulmones se estudia a partir de un conjunto de $N$ nubes de puntos (formas). Cada forma tiene $\Ktotal$ puntos. El conjunto de formas de entrenamiento iniciales es:
\begin{equation}
\mathcal{S}' = \left\{ \mat{S}'^{(j)} \in \R^{\Ktotal \times \dval} \mid j = 1, \dots, N \right\},
\label{eq:training_shapes_set_simplified}
\end{equation}
donde cada $\mat{S}'^{(j)}$ es una matriz de $\Ktotal$ puntos en $\dval=2$ dimensiones, y $'$ indica que son formas antes del alineamiento.

\begin{figure}[htbp]
    \centering
    \includegraphics[width=0.8\columnwidth]{Figures/01_raw_training_shapes.png}
    \caption{Ejemplos de formas de entrenamiento $\mat{S}'^{(j)}$ antes del alineamiento, mostrando variabilidad en posición, escala y orientación.}
    \label{fig:raw_shapes_simplified}
\end{figure}

Esta variabilidad se modela con un Modelo Estadístico de Forma (SSM) lineal. El SSM representa cualquier forma plausible como una deformación de una forma media. Esto implica dos pasos: alineamiento y Análisis de Componentes Principales (PCA).

\subsection{Alineamiento Procrustes Generalizado (GPA)}
\label{sec:gpa_simplified}
Para analizar la variabilidad de la forma, primero eliminamos las diferencias de traslación, rotación y escala. El Alineamiento Procrustes Generalizado (GPA) \cite{gower1975generalized} es un método iterativo que alinea todas las $N$ formas a un espacio común.

El GPA busca la transformación de similitud óptima (escala $s$, rotación $\mat{R}$, traslación $\vect{t}_0$) que minimiza la diferencia entre una forma objetivo $\mat{S}_{\text{target}}$ y una forma de referencia $\mat{S}_{\text{ref}}$:
\begin{equation}
E(\mat{S}_{\text{ref}}, \mat{S}_{\text{target}}) = \min_{s, \mat{R}, \vect{t}_0} \matrixnorm{ \mat{S}_{\text{ref}} - \left(s \mat{S}_{\text{target}} \mat{R} + \vecuno \transpose{\vect{t}_0}\right) }^2,
\label{eq:procrustes_error_simplified}
\end{equation}
donde $\matrixnorm{\cdot}$ es la norma de Frobenius y $\vecuno$ es un vector columna de unos.
Este proceso implica centrar las formas, encontrar la rotación, escala y traslación óptimas. El GPA aplica esto iterativamente: se alinea cada forma a una referencia (inicialmente una forma del conjunto, luego la media actualizada), se calcula una nueva forma media, y se repite hasta la convergencia.

El resultado es un conjunto de $N$ formas alineadas:
\begin{equation}
\tilde{\mathcal{S}} = \left\{ \tilde{\mat{S}}^{(j)} \in \R^{\Ktotal \times \dval} \mid j = 1, \dots, N \right\},
\label{eq:aligned_shapes_set_gpa_simplified}
\end{equation}
donde $\tilde{\mat{S}}^{(j)}$ es una forma alineada.

\begin{figure}[htbp]
    \centering
    \includegraphics[width=0.9\textwidth]{Figures/06_final_aligned_shapes.png}
    \caption{Conjunto de formas de entrenamiento $\tilde{\mathcal{S}}$ después del Alineamiento Procrustes Generalizado, superpuestas. La forma media $\mean{\mat{S}}$ (o $\overline{\mat{S}}$) se muestra en rojo.}
    \label{fig:aligned_shapes_gpa_simplified}
\end{figure}

\subsection{Análisis de Componentes Principales (PCA)}
\label{sec:pca_ssm_simplified}

Con las formas alineadas $\{\tilde{\mat{S}}^{(j)}\}$, se construye el modelo de variabilidad. Cada forma $\tilde{\mat{S}}^{(j)}$ se vectoriza a $\tilde{\vect{s}}^{(j)} \in \R^{\Ktotal\dval}$.

La forma media del conjunto alineado es:
\begin{equation}
\mean{\vect{s}} = \frac{1}{N} \sum_{j=1}^N \tilde{\vect{s}}^{(j)}.
\label{eq:mean_shape_vector_simplified}
\end{equation}
Se aplica PCA a las desviaciones de las formas respecto a esta media. La matriz de covarianza de los datos centrados $\mat{X}_{\text{cent}}$ (donde cada fila es $\tilde{\vect{s}}^{(j)} - \mean{\vect{s}}$) se estima como:
\begin{equation}
\mat{C} = \frac{1}{N-1} \transpose{\mat{X}_{\text{cent}}} \mat{X}_{\text{cent}}.
\label{eq:covariance_matrix_simplified}
\end{equation}
Los eigenvectores $\vect{p}_i$ de $\mat{C}$ son los componentes principales (modos de variación) y los eigenvalores $\lambda_i$ indican la varianza explicada.

Se seleccionan los $m$ eigenvectores $\vect{p}_1, \dots, \vect{p}_m$ con mayor varianza, formando la matriz $\mat{P} = [\vect{p}_1, \dots, \vect{p}_m]$.
Cualquier forma $\vect{s}$ del modelo SSM se representa como:
\begin{equation}
\vect{s}(\vect{b}) = \mean{\vect{s}} + \mat{P} \vect{b},
\label{eq:ssm_reconstruction_simplified}
\end{equation}
donde $\vect{b} = \transpose(b_1, \dots, b_m)$ es un vector de parámetros de forma. Cada $b_i$ controla la variación a lo largo del modo $\vect{p}_i$, usualmente restringido (e.g., $b_i \in [-3\sqrt{\lambda_i}, 3\sqrt{\lambda_i}]$) para generar formas realistas.

\begin{figure}[htbp]
    \centering
    \includegraphics[width=0.9\columnwidth]{Figures/pca_explained_variance_visualization.png}
    \caption{Varianza explicada acumulada por los componentes principales, útil para elegir $m$.}
    \label{fig:pca_variance_simplified}
\end{figure}

\begin{figure}[htbp]
    \centering
    \includegraphics[width=1\columnwidth]{Figures/pca_mode_1_visualization_144pts.png} 
    \caption{Visualización de la variación de la forma inducida por los primeros modos principales del SSM (e.g., $\mean{\vect{s}} \pm c\sqrt{\lambda_i}\vect{p}_i$).}
    \label{fig:ssm_mode_variation_detailed_simplified}
\end{figure}
%% \section{Modelo Estadístico de Apariencia (SAM)}
% \label{sec:asm}

% Mientras que el Modelo Estadístico de Forma (SSM) captura la variabilidad global de la forma de los pulmones, los Modelos Estadíscticos de Apariencia (SAM) \cite{cootes1995active} se encargan de modelar la apariencia local de la imagen en las proximidades de cada landmark del contorno. Estos modelos de apariencia son cruciales durante la etapa de ajuste iterativo del SSM, ya que guían a los landmarks hacia las posiciones que mejor se corresponden con la estructura esperada en la imagen.

% \subsection{Extracción de Perfiles de Intensidad}
% Para cada uno de los $K_{total}$ landmarks de la forma media $\overline{\vect{s}}$ obtenida del SSM, se define una dirección normal al contorno en ese punto. Esta normal, $\vect{n}_i$ para el landmark $i$, se calcula típicamente a partir de los puntos vecinos en la forma media para asegurar estabilidad. A lo largo de esta dirección normal, se muestrea un perfil de intensidad de la imagen. Este perfil es un vector unidimensional $\vect{g}_i \in \R^L$, donde $L$ es la longitud del perfil (número de píxeles muestreados). Los puntos de muestreo se toman a intervalos regulares (ej. espaciado de 1 píxel) a ambos lados del landmark, extendiéndose una distancia predefinida perpendicularmente al contorno.

% \begin{figure}[htbp] % Posicionamiento flexible
%     \centering
%     \includegraphics[width=1\columnwidth]{Figures/profile_extraction_visualization.png}
%     \caption{Extracción de un perfil de intensidad para un landmark. El punto rojo es el landmark $i$ en la forma media. La línea azul representa la normal $\vect{n}_i$ al contorno en ese punto. Los puntos verdes a lo largo de la normal indican las ubicaciones donde se muestrean los valores de intensidad de la imagen para formar el perfil $\vect{g}_i$.}
%     \label{fig:profile_extraction}
% \end{figure}

% \subsection{Construcción de Modelos Estadísticos de Apariencia}
% A partir de un conjunto de entrenamiento de $N$ imágenes y sus correspondientes formas alineadas, se extraen $N$ perfiles de intensidad $\{\vect{g}_i^{(j)}\}_{j=1}^N$ para cada landmark $i$. Estos perfiles se utilizan para construir un modelo estadístico de la apariencia esperada en ese landmark. El modelo más comúnmente utilizado asume una distribución Gaussiana multivariada para los perfiles. Para cada landmark $i$, se calcula:

% \begin{itemize}
%     \item El perfil medio:
%     \begin{equation}
%     \hspace*{\fill}
%     \overline{\vect{g}}_i = \frac{1}{N} \sum_{j=1}^N \vect{g}_i^{(j)}
%     \label{eq:mean_intensity_profile} % Etiqueta corregida
%     \hspace*{\fill}
%     \end{equation}
    
%     \item La matriz de covarianza de los perfiles: 
%     \begin{equation}
%     \hspace*{\fill}
%     \mat{\Sigma}_i = \frac{1}{N-1} \sum_{j=1}^N (\vect{g}_i^{(j)} - \overline{\vect{g}}_i)(\vect{g}_i^{(j)} - \overline{\vect{g}}_i)^\transpose
%     \label{eq:profile_covariance_matrix} % Etiqueta renombrada de eq:mahalanobis_profile_2 para claridad
%     \hspace*{\fill}
%     \end{equation}
% \end{itemize}
% La matriz de covarianza $\mat{\Sigma}_i \in \R^{L \times L}$ captura la variabilidad y la correlación entre los puntos del perfil en el landmark $i$.

% \begin{figure}[htbp] % Posicionamiento flexible
%     \centering
%     \includegraphics[width=1\textwidth]{Figures/composite_model_landmark_0.png}
%     \caption{Visualización del modelo estadístico de perfil para un landmark (índice 0). (Izquierda) Perfil de intensidad medio normalizado, $\overline{\vect{g}}_0$, y la banda de $\pm 1$ desviación estándar, indicando la variabilidad de la apariencia. (Derecha) Matriz de covarianza de los perfiles, $\mat{\Sigma}_0$, que captura la correlación entre los puntos del perfil de intensidad.}
%     \label{fig:modelo_perfil_landmark0}
% \end{figure}

% \subsection{Distancia de Mahalanobis para la Coincidencia de Perfiles}
% Durante el proceso de ajuste del SSM a una nueva imagen, se busca la mejor posición para cada landmark a lo largo de su normal. Para un punto candidato en la imagen, se extrae un perfil de intensidad observado $\vect{g}_{\text{obs}}$. La bondad de ajuste de este perfil observado con respecto al modelo estadístico del landmark $i$ (compuesto por $\overline{\vect{g}}_i$ de Ecuación~\eqref{eq:mean_intensity_profile} y $\mat{\Sigma}_i$ de Ecuación~\eqref{eq:profile_covariance_matrix}) se mide utilizando la distancia de Mahalanobis:
% \begin{equation}
% \hspace*{\fill}
% D_M^2(\vect{g}_{\text{obs}}, \overline{\vect{g}}_i) = (\vect{g}_{\text{obs}} - \overline{\vect{g}}_i)^\transpose \mat{\Sigma}_i^{-1} (\vect{g}_{\text{obs}} - \overline{\vect{g}}_i)
% \label{eq:mahalanobis_distance} % Etiqueta corregida
% \hspace*{\fill}
% \end{equation}
% donde $\mat{\Sigma}_i^{-1}$ es la inversa de la matriz de covarianza (o su pseudoinversa si $\mat{\Sigma}_i$ es singular). La distancia de Mahalanobis tiene en cuenta la varianza y covarianza de los datos del perfil, proporcionando una medida de distancia más robusta que la simple distancia Euclidiana, especialmente cuando los elementos del perfil están correlacionados. Un valor más bajo de $D_M^2$ indica una mejor coincidencia entre el perfil observado y el modelo.

% Los perfiles observados $\vect{g}_{\text{obs}}$ y los perfiles medios $\overline{\vect{g}}_i$ suelen normalizarse antes de calcular la distancia de Mahalanobis para hacer la comparación más robusta a cambios globales de iluminación. Una normalización común es la normalización Z-score, donde cada perfil se transforma para tener media cero y desviación estándar unitaria.

% La construcción de estos modelos de perfil para cada uno de los $K_{total}$ landmarks permite al algoritmo ASM identificar con precisión las estructuras de borde correspondientes en nuevas imágenes, guiando el ajuste del modelo de forma.

% \begin{figure}[htbp] % Posicionamiento flexible
%     \centering
%     \includegraphics[width=1\textwidth]{Figures/landmark_0_profile_comparison_v2.png}
%     \caption{Ilustración del cálculo de la distancia de Mahalanobis (Ecuación~\eqref{eq:mahalanobis_distance}) para evaluar la coincidencia entre un perfil de intensidad observado $\vect{g}_{\text{obs}}$ y el perfil medio modelado $\overline{\vect{g}}_i$ para un landmark específico. La distancia considera la estructura de covarianza de los perfiles $\mat{\Sigma}_i$, permitiendo una medida de ajuste que guía la colocación de los landmarks del SSM en nuevas imágenes. Un valor menor de $D_M^2$ indica una mejor correspondencia.}
% \label{fig:mahalanobis_coincidencia_perfil}
% \end{figure}

\section{Modelo Estadístico de Apariencia (SAM)}
\label{sec:sam_simplified}

Los Modelos Estadísticos de Apariencia (SAM) \cite{cootes1995active} modelan la apariencia local de la imagen cerca de cada landmark del contorno. Guían el ajuste del modelo de forma a nuevas imágenes.

\subsection{Extracción de Perfiles de Intensidad}
Para cada landmark $i$ de la forma media $\mean{\vect{s}}$, se define una normal $\vect{n}_i$ al contorno. A lo largo de esta normal, se muestrea un perfil de intensidad de la imagen, que es un vector $\vect{g}_i \in \R^L$, donde $L$ es la longitud del perfil.

\begin{figure}[htbp]
    \centering
    \includegraphics[width=1\columnwidth]{Figures/profile_extraction_visualization.png}
    \caption{Extracción de un perfil de intensidad $\vect{g}_i$ para un landmark, a lo largo de la normal $\vect{n}_i$ al contorno.}
    \label{fig:profile_extraction_simplified}
\end{figure}

\subsection{Construcción de Modelos Estadísticos de Apariencia}
Con $N$ imágenes de entrenamiento, se extraen $N$ perfiles $\{\vect{g}_i^{(j)}\}_{j=1}^N$ para cada landmark $i$. Se asume una distribución Gaussiana multivariada para estos perfiles. Para cada landmark $i$, se calcula:

\begin{itemize}
    \item El perfil medio:
    \begin{equation}
    \mean{\vect{g}}_i = \frac{1}{N} \sum_{j=1}^N \vect{g}_i^{(j)}
    \label{eq:mean_intensity_profile_simplified}
    \end{equation}
    
    \item La matriz de covarianza de los perfiles: 
    \begin{equation}
    \matSigma_i = \frac{1}{N-1} \sum_{j=1}^N (\vect{g}_i^{(j)} - \mean{\vect{g}}_i)\transpose{(\vect{g}_i^{(j)} - \mean{\vect{g}}_i)}
    \label{eq:profile_covariance_matrix_simplified}
    \end{equation}
\end{itemize}
La matriz $\matSigma_i \in \R^{L \times L}$ captura la variabilidad de la apariencia en el landmark $i$.

\begin{figure}[htbp]
    \centering
    \includegraphics[width=1\textwidth]{Figures/composite_model_landmark_0.png}
    \caption{Modelo estadístico de perfil: (Izquierda) Perfil medio $\mean{\vect{g}}_i$ con variabilidad. (Derecha) Matriz de covarianza $\matSigma_i$.}
    \label{fig:modelo_perfil_landmark_simplified}
\end{figure}

\subsection{Distancia de Mahalanobis para la Coincidencia de Perfiles}
Para ajustar el modelo a una nueva imagen, se extrae un perfil observado $\vect{g}_{\text{obs}}$ en un punto candidato. La bondad de ajuste con el modelo del landmark $i$ ($\mean{\vect{g}}_i$ y $\matSigma_i$) se mide con la distancia de Mahalanobis:
\begin{equation}
D_M^2(\vect{g}_{\text{obs}}, \mean{\vect{g}}_i) = \transpose{(\vect{g}_{\text{obs}} - \mean{\vect{g}}_i)} \matSigma_i^{-1} (\vect{g}_{\text{obs}} - \mean{\vect{g}}_i)
\label{eq:mahalanobis_distance_simplified}
\end{equation}
donde $\matSigma_i^{-1}$ es la inversa (o pseudoinversa) de la matriz de covarianza. Un valor $D_M^2$ más bajo indica mejor coincidencia. Los perfiles suelen normalizarse antes de este cálculo.

\begin{figure}[htbp]
    \centering
    \includegraphics[width=1\textwidth]{Figures/landmark_0_profile_comparison_v2.png}
    \caption{Cálculo de la distancia de Mahalanobis para evaluar la coincidencia entre un perfil observado $\vect{g}_{\text{obs}}$ y el modelo $\mean{\vect{g}}_i, \matSigma_i$.}
    \label{fig:mahalanobis_coincidencia_perfil_simplified}
\end{figure}

%% % Asegúrate de que estos comandos están definidos en tu preámbulo:
% % \newcommand{\mat}[1]{\mathbf{#1}}   
% % \newcommand{\vect}[1]{\bm{#1}} % Requiere \usepackage{bm}
% % \newcommand{\transpose}{\mathsf{T}}
% % \newcommand{\R}{\mathbb{R}} 

% \section{Estimación de Pose Inicial (ESL)}
% \label{sec:esl_matematica}

% La inicialización precisa de los parámetros de pose (posición, orientación y escala) de un modelo de forma es un prerrequisito fundamental para la convergencia robusta y eficiente de procesos subsecuentes de ajuste iterativo, como los aplicados a Modelos Estadísticos de Forma (SSM). Una estimación inicial subóptima puede conducir a la convergencia del modelo hacia mínimos locales espurios o a un incremento computacionalmente prohibitivo en el número de iteraciones requeridas. Para mitigar estos riesgos, se introduce una etapa de Estimación de Pose Inicial (ESL). Esta etapa se fundamenta en el uso de un conjunto de funciones discriminativas, denotadas genéricamente como $C$, capaces de inferir los parámetros de una transformación global, $\mathcal{T}_{ESL}$, que relaciona el espacio canónico del modelo con el espacio de la imagen.

% El objetivo primordial de la ESL es la determinación de una transformación afín $\mathcal{T}_{ESL}: \R^2 \to \R^2$. Esta transformación mapea puntos $\vect{p} = (p_x, p_y)^\transpose$ desde un sistema de coordenadas canónico del modelo de forma hacia sus correspondientes localizaciones $\vect{x} = (x,y)^\transpose$ en la imagen de entrada $I(\vect{x})$. Dicha transformación se parametriza como:
% \begin{equation}
% \hspace*{\fill}
% \vect{x} = \mathcal{T}_{ESL}(\vect{p}) = \mat{S}_{ESL} \mat{R}(\theta_{ESL}) \vect{p} + \vect{t}_{ESL} = \begin{pmatrix} s_x & 0 \\ 0 & s_y \end{pmatrix} \begin{pmatrix} \cos \theta_{ESL} & -\sin \theta_{ESL} \\ \sin \theta_{ESL} & \cos \theta_{ESL} \end{pmatrix} \vect{p} + \begin{pmatrix} t_x \\ t_y \end{pmatrix},
% \label{eq:esl_transform_mat}
% \hspace*{\fill}
% \end{equation}
% donde $\mat{S}_{ESL} = \text{diag}(s_x, s_y)$ es la matriz de escala diagonal con factores de escala anisotrópica $s_x, s_y$; $\mat{R}(\theta_{ESL})$ es la matriz de rotación bidimensional correspondiente al ángulo de orientación $\theta_{ESL}$; y $\vect{t}_{ESL} = (t_x, t_y)^\transpose$ es el vector de traslación. La tarea de la ESL es, por lo tanto, estimar el conjunto de parámetros $(\hat{\vect{t}}_{ESL}, \hat{s}_x, \hat{s}_y, \hat{\theta}_{ESL})$. Si definimos el vector de parámetros de escala como $\hat{\vect{s}}_{ESL\_params} = (\hat{s}_x, \hat{s}_y)^\transpose$ (para distinguirlo de un posible vector de forma $\vect{s}$), el objetivo es estimar $(\hat{\vect{t}}_{ESL}, \hat{\vect{s}}_{ESL\_params}, \hat{\theta}_{ESL})$.

% La estimación de los componentes de esta transformación se descompone en tareas de inferencia acometidas por funciones discriminativas especializadas:

% \begin{itemize}
%     \item \textbf{Estimación de Límites de Contorno:}
%     Un conjunto de cuatro funciones discriminativas, $\{C_{L1}, C_{L2}, C_{L3}, C_{L4}\}$, se emplea para la detección de las posiciones óptimas de cuatro líneas canónicas: $L_1$ (línea vertical izquierda, correspondiente a una abscisa $x_{L1}$), $L_2$ (línea horizontal superior, ordenada $y_{L2}$), $L_3$ (línea vertical derecha, abscisa $x_{L3}$), y $L_4$ (línea horizontal inferior, ordenada $y_{L4}$). Estas líneas definen una caja delimitadora $B_{img}$ (AABB: Axis-Aligned Bounding Box), alineada con los ejes de la imagen, que se presume encierra la estructura de interés.

%     Para cada línea $L_k$ (donde $k \in \{1,2,3,4\}$), su función discriminativa asociada $C_{Lk}$ opera sobre una colección de subregiones o parches $\Omega_j \subset I(\vect{x})$, de dimensiones predefinidas $N \times N$ píxeles. Cada parche $\Omega_j$ se extrae a lo largo de una trayectoria de búsqueda discreta $\mathcal{P}_k$, definida perpendicularmente a la orientación esperada de la línea $L_k$. Por ejemplo, para $L_1$ (vertical), $\mathcal{P}_1$ es un conjunto de abscisas candidatas $\{x_j\}$ evaluadas a una ordenada fija $y_c$ (e.g., el centro vertical de la imagen), dentro de un rango de búsqueda $[x_{\text{search\_min}}, x_{\text{search\_max}}]$ determinado en función de las dimensiones de $I$.
%     La función $C_{Lk}$ asigna a cada parche $\Omega_j(pos_j)$, extraído en la posición $pos_j \in \mathcal{P}_k$, una puntuación $f_{Lk}(\Omega_j(pos_j)) \in [0, 1]$ que cuantifica la verosimilitud de que la línea $L_k$ esté presente en $\Omega_j$. La posición óptima estimada para la línea $L_k$, denotada $\widehat{pos}_{Lk}$, se obtiene como:
%     \begin{equation}
%     \hspace*{\fill}
%     \widehat{pos}_{Lk} = \arg\max_{pos_j \in \mathcal{P}_k} \{ f_{Lk}(\Omega_j(pos_j)) \}.
%     \label{eq:esl_pos_opt_mat}
%     \hspace*{\fill}
%     \end{equation}
%     Una vez estimadas las posiciones de las cuatro líneas $\{\hat{x}_{L1}, \hat{y}_{L2}, \hat{x}_{L3}, \hat{y}_{L4}\}$, los parámetros de traslación y las dimensiones de la caja $B_{img}$ se derivan:
%     \begin{align}
%     \hat{t}_x &= (\hat{x}_{L1} + \hat{x}_{L3}) / 2, \label{eq:esl_tx_estimate} \\
%     \hat{t}_y &= (\hat{y}_{L2} + \hat{y}_{L4}) / 2, \label{eq:esl_ty_estimate} \\
%     \hat{w}_{B} &= \hat{x}_{L3} - \hat{x}_{L1}, \label{eq:esl_wB_mat} \\
%     \hat{h}_{B} &= \hat{y}_{L4} - \hat{y}_{L2}. \label{eq:esl_hB_mat}
%     \end{align}
%     Estos definen el centro estimado $\hat{\vect{t}}_{ESL} = (\hat{t}_x, \hat{t}_y)^\transpose$ y las dimensiones (ancho $\hat{w}_{B}$, alto $\hat{h}_{B}$) de la caja delimitadora $B_{img}$. Los factores de escala $s_x, s_y$ de la Ecuación~\eqref{eq:esl_transform_mat} se relacionan con estas dimensiones, usualmente mediante una normalización respecto a las dimensiones de un modelo de forma canónico de referencia ($s_x = \hat{w}_B / w_{\text{ref}}$, $s_y = \hat{h}_B / h_{\text{ref}}$). El proceso esquemático para la estimación de los límites de contorno se ilustra en la Figura~\ref{fig:esl_line_detection_mat}.

% \begin{figure}[htbp] % Posicionamiento flexible
%     \centering
%     \includegraphics[width=\columnwidth]{Figures/fig_esl_line_detection.png}
%     \caption{Proceso esquemático de estimación de la caja delimitadora mediante funciones discriminativas de línea. Se ilustran las trayectorias de búsqueda y los parches muestreados en la imagen de entrada para las líneas horizontales y verticales. También se muestra un ejemplo del perfil de puntuaciones (verosimilitud) obtenido por una función $C_{Lk}$ a lo largo de su trayectoria de búsqueda, donde el máximo indica la posición óptima estimada. Finalmente, se visualiza la caja delimitadora resultante, definida por los parámetros $(\hat{\vect{t}}_{ESL}, \hat{w}_B, \hat{h}_B)$, superpuesta en la imagen original.}
%     \label{fig:esl_line_detection_mat}
% \end{figure}

%     \item \textbf{Estimación de Orientación:}
%     Una función discriminativa adicional, $C_{\theta}$, se emplea para estimar el ángulo de rotación global $\hat{\theta}_{ESL}$. Esta función opera sobre una región de interés $\Omega_{\theta}$ extraída de la imagen $I$, centrada en $\hat{\vect{t}}_{ESL}$ y cuyas dimensiones son proporcionales a $(\hat{w}_{B}, \hat{h}_{B})$ (e.g., escaladas por un factor $\kappa > 1$ para asegurar la inclusión completa del objeto bajo rotación).
%     Se genera un conjunto de parches candidatos $\{\Omega_{\theta}^{(r)}\}_{r=1}^M$ rotando la región $\Omega_{\theta}$ mediante un conjunto discreto de ángulos de prueba $\{\alpha_r\}_{r=1}^M$, donde $\alpha_r \in [-\alpha_{\text{max}}, \alpha_{\text{max}}]$. Cada parche rotado $\Omega_{\theta}^{(r)}$ es reescalado a las dimensiones $N \times N$ píxeles. La función $C_{\theta}$ asigna una puntuación $f_{\theta}(\Omega_{\theta}^{(r)})$ a cada parche. El ángulo de orientación estimado $\hat{\theta}_{ESL}$ es aquel que maximiza esta puntuación:
%     \begin{equation}
%     \hspace*{\fill}
%     \hat{\theta}_{ESL} = \arg\max_{\alpha_r \in \{\alpha_j\}} \{ f_{\theta}(\Omega_{\theta}^{(r)}(\alpha_r)) \}.
%     \label{eq:esl_theta_opt_mat}
%     \hspace*{\fill}
%     \end{equation}
%     El procedimiento para la estimación de la orientación se detalla esquemáticamente en la Figura~\ref{fig:esl_orientation_detection_mat}.

% \begin{figure}[htbp] % Posicionamiento flexible
%     \centering
%     \includegraphics[width=0.8\columnwidth]{Figures/fig_esl_orientation_detection.png} 
%     \caption{Proceso esquemático de estimación de la orientación. Se extrae una región de interés $\Omega_{\theta}$ de la imagen, centrada y escalada según la AABB previamente estimada. Se genera un conjunto de parches candidatos mediante la rotación de $\Omega_{\theta}$ a través de un rango de ángulos discretos. Se muestra un ejemplo del perfil de puntuaciones (verosimilitud) de la función $C_{\theta}$ para cada ángulo de rotación evaluado, donde el máximo indica el ángulo $\hat{\theta}_{ESL}$ estimado. Finalmente, se visualiza la pose final estimada $\mathcal{T}_{ESL}$ (caja delimitadora rotada) superpuesta en la imagen original.}
%     \label{fig:esl_orientation_detection_mat}
% \end{figure}
% \end{itemize}

% La construcción (entrenamiento) de las funciones discriminativas $C_{Lk}$ y $C_{\theta}$ se realiza mediante un proceso de aprendizaje supervisado. Se parte de un conjunto de $N_S$ muestras de entrenamiento, donde cada muestra $i$ consiste en una imagen $I_i$ y la representación de la forma de interés mediante $N_P$ puntos de referencia (landmarks), $\{\vect{p}_{i,j}^*\}_{j=1}^{N_P}$, que definen la verdad terreno (ground truth).

% Para cada muestra de entrenamiento $i$:
% \begin{enumerate}
%     \item Se calcula la caja delimitadora alineada a los ejes (AABB) ground truth, $B_{gt,i}$, a partir de sus landmarks $\{\vect{p}_{i,j}^*\}$. Las coordenadas de sus límites $(x_{\text{min},i}, y_{\text{min},i}, x_{\text{max},i}, y_{\text{max},i})$ definen las posiciones verdaderas $L_{k,gt,i}$ para las cuatro líneas.
%     \item Para el entrenamiento de cada $C_{Lk}$:
%         \begin{itemize}
%             \item Se extraen parches \textit{positivos} $\Omega_{\text{pos}}$ de $I_i$. El centro de $\Omega_{\text{pos}}$ se sitúa en la posición $L_{k,gt,i}$ (con la otra coordenada del centro del parche tomada del centro de $B_{gt,i}$). Estos parches se asocian a una etiqueta de clase positiva (e.g., $y=1$).
%             \item Se extraen parches \textit{negativos} $\Omega_{\text{neg}}$ de $I_i$. Estos se obtienen de posiciones $pos_{\text{neg}}$ a lo largo de la trayectoria de búsqueda $\mathcal{P}_k$, tales que la distancia entre $pos_{\text{neg}}$ y $L_{k,gt,i}$ sea mayor que un umbral $\delta_L$. Dicho umbral $\delta_L$ es típicamente proporcional a la dimensión relevante de $B_{gt,i}$. Estos parches se asocian a una etiqueta de clase negativa (e.g., $y=0$).
%         \end{itemize}
%     \item Para el entrenamiento de $C_{\theta}$:
%         \begin{itemize}
%             \item La orientación ground truth de la AABB, $\theta_{gt,i}$, se considera $0$ radianes por definición (alineada con los ejes).
%             \item Se extrae una región base $\Omega_{\theta,\text{base}}$ de $I_i$, centrada en el centro de $B_{gt,i}$ y con dimensiones que aseguren contener $B_{gt,i}$ con un margen contextual.
%             \item Para cada ángulo de rotación de prueba $\alpha_r$ de un conjunto discreto, se rota $\Omega_{\theta,\text{base}}$ por $\alpha_r$ para obtener $\Omega_{\theta}^{(r)}$.
%             \item El parche $\Omega_{\theta}^{(r)}$ se etiqueta como positivo si $|\alpha_r - \theta_{gt,i}| \le \delta_{\theta}$ (donde $\delta_{\theta}$ es un umbral angular pequeño, y $\theta_{gt,i}=0$), y negativo en caso contrario.
%         \end{itemize}
% \end{enumerate}
% Todos los parches extraídos son reescalados a dimensiones $N \times N$ y sometidos a un proceso de normalización de intensidad antes de ser utilizados para entrenar las funciones discriminativas.

% En la fase de inferencia, dada una nueva imagen $I(\vect{x})$, las funciones $C_{L1}-C_{L4}$ y $C_{\theta}$ se aplican secuencialmente como se describió para estimar $(\hat{\vect{t}}_{ESL}, \hat{w}_{B}, \hat{h}_{B}, \hat{\theta}_{ESL})$. Estos parámetros, junto con la relación entre $(\hat{w}_{B}, \hat{h}_{B})$ y los factores de escala $s_x, s_y$ (que parametrizan $\mat{S}_{ESL}$ en la Ecuación~\eqref{eq:esl_transform_mat}), definen completamente la transformación $\mathcal{T}_{ESL}$. Esta transformación proporciona una estimación global robusta de la pose del objeto, fundamental para inicializar subsecuentes procesos de ajuste fino del modelo de forma.

% \begin{figure}[htbp] % Posicionamiento flexible
%     \centering
%     \includegraphics[width=1\columnwidth]{Figures/visualizacion_entrenamiento_esl.png} 
%     \caption{Estimación de Pose Inicial (ESL): Ilustración del proceso de generación de datos de entrenamiento para sus clasificadores. (Arriba-Izquierda) Imagen de muestra $I_i$ con sus landmarks ground truth $\{\vect{p}_{i,j}^*\}$ y la AABB $B_{gt,i}$ derivada. (Arriba-Centro, Arriba-Derecha) Ejemplos de un parche positivo $\Omega_{\text{pos}}$ y uno negativo $\Omega_{\text{neg}}$ para el clasificador de línea $C_{L1}$, extraídos en la posición $L_{1,gt}$ y a una distancia $>\delta_L$, respectivamente. (Abajo-Izquierda) Región base $\Omega_{\theta,\text{base}}$ para el clasificador de orientación $C_{\theta}$. (Abajo-Centro, Abajo-Derecha) Ejemplos de parches rotados $\Omega_{\theta}^{(r)}$ generados a partir de $\Omega_{\theta,\text{base}}$, con una rotación pequeña (etiqueta positiva, e.g., $\alpha_r \approx 0$) y una mayor (etiqueta negativa).}
%     \label{fig:esl_training_data_generation}
% \end{figure}

\section{Estimación de Pose Inicial (ESL)}
\label{sec:esl_simplified}

Una estimación inicial precisa de la pose (posición, orientación y escala) de un modelo de forma es crucial para el ajuste robusto del modelo a una imagen. La Estimación de Pose Inicial (ESL) busca determinar una transformación afín global $\mathcal{T}_{ESL}$ que alinee el modelo con el objeto en la imagen.

Esta transformación mapea puntos $\vect{p}$ del modelo a puntos $\vect{x}$ en la imagen:
\begin{equation}
\vect{x} = \mathcal{T}_{ESL}(\vect{p}) = \mat{S}_{ESL} \mat{R}(\theta_{ESL}) \vect{p} + \vect{t}_{ESL},
\label{eq:esl_transform_simplified}
\end{equation}
donde $\mat{S}_{ESL}$ es una matriz de escala (con factores $s_x, s_y$), $\mat{R}(\theta_{ESL})$ es una matriz de rotación (con ángulo $\theta_{ESL}$), y $\vect{t}_{ESL}$ es un vector de traslación (con componentes $t_x, t_y$). El objetivo de ESL es estimar estos parámetros: $(\hat{\vect{t}}_{ESL}, \hat{s}_x, \hat{s}_y, \hat{\theta}_{ESL})$.

\begin{figure}[htbp]
    \centering
    \includegraphics[width=0.7\columnwidth]{Figures/fig_esl_line_detection.png}
    \caption{Estimación de la caja delimitadora mediante funciones de línea. Se buscan las posiciones óptimas para las líneas delimitadoras, definiendo el centro y las dimensiones de la caja.}
    \label{fig:esl_line_detection_simplified}
\end{figure}

La estimación se realiza mediante funciones discriminativas especializadas:

\begin{itemize}
    \item \textbf{Estimación de Límites de Contorno (para $\hat{\vect{t}}_{ESL}, \hat{s}_x, \hat{s}_y$):}
    Se utilizan funciones para detectar cuatro líneas (izquierda, superior, derecha, inferior) que definen una caja delimitadora (Bounding Box) alrededor del objeto en la imagen. Cada función $C_{Lk}$ busca la posición óptima $\widehat{pos}_{Lk}$ para su línea $L_k$ evaluando parches de imagen:
    \begin{equation}
    \widehat{pos}_{Lk} = \arg\max_{pos_j} \{ f_{Lk}(\Omega_j(pos_j)) \}.
    \label{eq:esl_pos_opt_simplified}
    \end{equation}
    A partir de estas posiciones ($\hat{x}_{L1}, \hat{y}_{L2}, \hat{x}_{L3}, \hat{y}_{L4}$), se calculan el centro de la caja (que da $\hat{\vect{t}}_{ESL}$) y sus dimensiones (ancho $\hat{w}_{B}$, alto $\hat{h}_{B}$), que se usan para estimar los factores de escala $\hat{s}_x, \hat{s}_y$.
    \begin{align}
    \hat{t}_x &= (\hat{x}_{L1} + \hat{x}_{L3}) / 2 \label{eq:esl_tx_estimate_simplified} \\
    \hat{t}_y &= (\hat{y}_{L2} + \hat{y}_{L4}) / 2 \label{eq:esl_ty_estimate_simplified} \\
    \hat{w}_{B} &= \hat{x}_{L3} - \hat{x}_{L1} \label{eq:esl_wB_simplified} \\
    \hat{h}_{B} &= \hat{y}_{L4} - \hat{y}_{L2} \label{eq:esl_hB_simplified}
    \end{align}

    \item \textbf{Estimación de Orientación (para $\hat{\theta}_{ESL}$):}
    Otra función $C_{\theta}$ estima el ángulo de rotación $\hat{\theta}_{ESL}$. Opera sobre una región de la imagen centrada según la caja delimitadora. Se evalúan varias rotaciones de esta región:
    \begin{equation}
    \hat{\theta}_{ESL} = \arg\max_{\alpha_r} \{ f_{\theta}(\Omega_{\theta}^{(r)}(\alpha_r)) \}.
    \label{eq:esl_theta_opt_simplified}
    \end{equation}
    El ángulo $\alpha_r$ que maximiza la puntuación de la función $f_{\theta}$ es la orientación estimada.

\begin{figure}[htbp]
    \centering
    \includegraphics[width=0.6\columnwidth]{Figures/fig_esl_orientation_detection.png} 
    \caption{Estimación de la orientación. Se evalúan rotaciones de una región de interés para encontrar el ángulo que mejor se ajusta según una función discriminativa.}
    \label{fig:esl_orientation_detection_simplified}
\end{figure}
\end{itemize}

Las funciones discriminativas ($C_{Lk}, C_{\theta}$) se entrenan previamente con imágenes de ejemplo. En la fase de inferencia, estas funciones estiman los parámetros de pose $(\hat{\vect{t}}_{ESL}, \hat{s}_x, \hat{s}_y, \hat{\theta}_{ESL})$, definiendo la transformación $\mathcal{T}_{ESL}$ para una inicialización robusta del modelo de forma.
%
% \section{Extracción de Parches para Entrenamiento de la CNN Híbrida}
% \label{sec:extraccion_parches_cnn}

% La Red Neuronal Convolucional (CNN) híbrida, diseñada para predecir los parámetros de forma $\vect{b}$ del SSM (ver Ecuación~\eqref{eq:ssm_reconstruction}), requiere como entrada información visual local derivada de la apariencia de la imagen alrededor de los puntos de referencia. Esta sección detalla el proceso de generación y extracción de estos parches de imagen, que constituyen los datos de entrenamiento para la CNN.

% \subsection{Generación de Ejemplos de Entrenamiento}
% Para cada uno de los $m$ modos de variación del SSM, se entrena un clasificador CNN independiente para predecir el coeficiente $b_k$ correspondiente (donde $k \in \{1, \dots, m\}$). La generación de ejemplos de entrenamiento para el $k$-ésimo clasificador implica la creación de instancias de forma que representen tanto el valor ground truth de $b_k$ como valores perturbados.
% \begin{itemize}
%     \item \textbf{Ejemplos Positivos:} Para una imagen de entrenamiento dada, se utiliza el vector de parámetros de forma $\vect{b}_{GT}$ (obtenido al proyectar la forma ground truth de esa imagen sobre la base del SSM) para generar una instancia de forma. Los parches extraídos alrededor de los landmarks de esta forma se asocian con la etiqueta del bin correspondiente al valor $b_{k,GT}$ (el $k$-ésimo componente de $\vect{b}_{GT}$).
%     \item \textbf{Ejemplos Negativos:} Para generar diversidad y robustez en el entrenamiento, se crean ejemplos negativos perturbando el $k$-ésimo coeficiente $b_k$ del vector $\vect{b}_{GT}$ mientras se mantienen los otros coeficientes $b_j (j < k)$ en sus valores ground truth (o estimados por clasificadores previos en un esquema secuencial). El valor de $b_k$ se muestrea aleatoriamente de un rango que excluye el bin ground truth, típicamente dentro de los límites de $\pm n\sigma_k$ (donde $\sigma_k = \sqrt{\lambda_k}$ es la desviación estándar del modo $k$). Los parches extraídos de estas formas perturbadas se asocian con la etiqueta del bin correspondiente al valor $b_k$ perturbado.
% \end{itemize}

% \begin{figure}[htbp] % Posicionamiento flexible
%     \centering
%     \includegraphics[width=0.7\columnwidth]{Figures/fig_pos_neg_examples.png}
%     \caption{Comparación visual entre un ejemplo positivo (izquierda), generado a partir de los parámetros de forma ground truth ($\vect{b}_{GT}$ con $b_k=0.04$ para un $k$ específico), y un ejemplo negativo (derecha), obtenido al perturbar el coeficiente $b_k$ ($b_{k,\text{pert}}=0.13$). Para cada caso, se muestra la forma resultante superpuesta en la radiografía y los parches de imagen correspondientes extraídos alrededor de sus landmarks.}
%     \label{fig:pos_neg_examples}
% \end{figure}

% \subsection{Transformación de Forma al Espacio de la Imagen}
% Antes de extraer los parches, la instancia de forma del SSM (ya sea positiva o negativa, definida por un vector de parámetros $\vect{b}$) debe ser transformada desde el espacio canónico del SSM al espacio de la imagen. Esto se logra utilizando la transformación de similitud $\mathcal{T}_{ESL}$ obtenida de la etapa de Estimación de Pose Inicial (ESL, Sección~\ref{sec:esl_matematica}):
% \begin{equation}
% \mat{S}'_{\text{img}} = \mathcal{T}_{ESL}(\mat{S}'_{\text{SSM}}(\vect{b}))
% \label{eq:shape_ssm_to_image} % Añadida etiqueta para referencia si es necesaria
% \end{equation}
% donde $\mat{S}'_{\text{SSM}}(\vect{b})$ es la forma de $K_{total}$ puntos (donde $K_{total}=144$) generada por el SSM con los parámetros $\vect{b}$, y $\mat{S}'_{\text{img}}$ son las coordenadas de los landmarks en el espacio de la imagen.

% \begin{figure}[htbp] % Posicionamiento flexible
%     \centering
%     \includegraphics[width=0.7\columnwidth]{Figures/fig_shape_transformation.png}
%     \caption{Ilustración del proceso de transformación de una forma desde el espacio canónico del Modelo Estadístico de Forma (SSM) al espacio de la imagen. (a) Representa la forma canónica del SSM, $\mat{S}'_{\text{SSM}}(\vect{b})$. (b) Muestra la forma transformada $\mat{S}'_{\text{img}}$ alineada con la radiografía de tórax, obtenida al aplicar la transformación de similitud $\mathcal{T}_{ESL}$ (proveniente de la Estimación de Pose Inicial) a la forma canónica, según la Ecuación~\eqref{eq:shape_ssm_to_image}.}
%     \label{fig:ssm_to_image_transformation}
% \end{figure}

% \subsection{Extracción de Parches}
% Para cada uno de los $K_{total}$ landmarks de la forma proyectada $\mat{S}'_{\text{img}}$, se extrae un parche cuadrado de intensidad de tamaño $Q \times Q$ píxeles, centrado en la posición del landmark $(x_i, y_i)$. Estos parches capturan la apariencia local de la imagen alrededor de cada punto de referencia.
% Los $K_{total}$ parches extraídos para una instancia de forma se aplanan y concatenan para formar un único vector de características $\vect{f} \in \R^{K_{total} Q^2}$, que sirve como entrada para la CNN.

% \begin{figure}[htbp] % Posicionamiento flexible
%     \centering   
%     \includegraphics[width=0.7\columnwidth]{Figures/fig_patch_extraction_details.png}
%     \caption{Proceso de extracción de parches para la CNN. (a) Una instancia de forma del SSM (línea azul) se proyecta al espacio de la imagen usando la pose estimada por ESL. (b) Para cada uno de los $K_{total}$ landmarks (puntos verdes), se extrae un parche de imagen $Q \times Q$ (cuadrados rojos) centrado en el landmark. Estos parches, una vez procesados, se utilizan como entrada para la CNN.}
%     \label{fig:patch_extraction_cnn}
% \end{figure}

% \subsection{Aumento de Datos}
% Para incrementar la variabilidad del conjunto de entrenamiento y mejorar la generalización de los modelos CNN, se aplican diversas técnicas de aumento de datos a las imágenes antes de la extracción de parches. Estas pueden incluir:
% \begin{itemize}
%     \item \textbf{Volteo Horizontal (Flipping):} Las imágenes se voltean horizontalmente con una cierta probabilidad. Si una imagen se voltea, las coordenadas de los landmarks y los parámetros de forma $\vect{b}$ asociados con modos asimétricos también deben ser ajustados correspondientemente (e.g., invirtiendo el signo de los $b_k$ para modos asimétricos).
%     \item \textbf{Variación de Intensidad:} Se pueden aplicar variaciones aleatorias al brillo y contraste de la imagen. Una técnica más sofisticada implica aplicar PCA a los valores de intensidad de las imágenes de entrenamiento y luego añadir múltiplos aleatorios de los principales eigenvectores de intensidad a la imagen.
% \end{itemize}

% \begin{figure}[htbp] % Posicionamiento flexible
%     \centering
%     \includegraphics[width=0.7\columnwidth]{Figures/fig_data_augmentation.png}
%     \caption{Técnicas de aumento de datos aplicadas a una radiografía de tórax y su correspondiente forma SSM superpuesta, utilizadas para enriquecer el conjunto de entrenamiento. Se observa: (a) la imagen original; (b) la imagen y la forma SSM después de un volteo horizontal (nótese que los coeficientes $b_k$ de los parámetros de forma $\vect{b}$ para modos asimétricos se ajustarían en consecuencia); y (c) la imagen original con variaciones aleatorias de contraste y brillo.}
%     \label{fig:data_augmentation_examples}
% \end{figure}

% \subsection{Discretización de Coeficientes $b_k$}
% Dado que las CNNs se entrenan como clasificadores, el valor continuo del coeficiente $b_k$ (ya sea el ground truth o el perturbado) se discretiza en uno de $B$ bins. Típicamente, el rango de variación de $b_k$ (e.g., $[-n\sigma_k, n\sigma_k]$, donde $\sigma_k = \sqrt{\lambda_k}$ es la desviación estándar del $k$-ésimo modo) se divide en $B$ intervalos de igual tamaño. La etiqueta discreta $y_k \in \{0, \dots, B-1\}$ correspondiente al bin en el que cae el valor de $b_k$ se utiliza como la etiqueta ground truth para el entrenamiento del $k$-ésimo clasificador CNN.

% \begin{figure}[htbp] % Posicionamiento flexible
%     \centering
%     \includegraphics[width=0.7\textwidth]{Figures/fig_b_discretization.png}
%     \caption{Ilustración conceptual del proceso de discretización para un coeficiente de forma $b_k$. El rango de variación del coeficiente (aquí mostrado como $\pm 3.0\sigma_k$, con $\sigma_k=0.06$ para el ejemplo) se divide en $B=3$ bins de igual tamaño (Bin 0, Bin 1, Bin 2). Un valor continuo del coeficiente, como $b_k = 0.04$ (indicado por el punto rojo), se asigna al bin correspondiente (en este caso, Bin 1, resaltado en azul claro), generando una etiqueta discreta para el entrenamiento del clasificador CNN. La curva negra representa esquemáticamente la distribución de probabilidad $p(b_k)$ del coeficiente.}
%     \label{fig:bk_discretization}
% \end{figure}

% \subsection{Reducción de Dimensionalidad de Parches (Opcional)}
% El vector de características concatenado $\vect{f} \in \R^{K_{total} Q^2}$ puede tener una dimensionalidad muy alta (e.g., $144 \times 25^2 = 90000$ para $Q=25$, si $K_{total}=144$). Para reducir la carga computacional y potencialmente mejorar el rendimiento eliminando redundancias, se puede aplicar PCA a estos vectores de características de parches. Se entrena un modelo PCA separado para los vectores de características de cada modo $k$, transformando $\vect{f}$ a un subespacio de menor dimensión $\vect{f}' \in \R^{d_{\text{PCA}}}$, donde $d_{\text{PCA}} \ll K_{total} Q^2$.

% Este proceso de extracción y preparación de datos asegura que la CNN reciba información local relevante y normalizada para aprender la compleja relación entre la apariencia de la imagen y los parámetros de forma del SSM.

\section{Extracción de Parches para Entrenamiento de la CNN Híbrida}
\label{sec:extraccion_parches_cnn_simplified}

La Red Neuronal Convolucional (CNN) híbrida predice los parámetros de forma $\vect{b}$ del Modelo Estadístico de Forma (SSM, ver Ecuación~\eqref{eq:ssm_reconstruction_simplified} de una sección anterior). Para ello, necesita parches de imagen extraídos alrededor de los puntos de referencia (landmarks).

\subsection{Generación de Ejemplos de Entrenamiento}
Para cada uno de los $m$ modos de variación del SSM, se entrena una CNN para predecir su coeficiente $b_k$.
\begin{itemize}
    \item \textbf{Ejemplos Positivos:} Se usan los parámetros de forma verdaderos $\vect{b}_{GT}$ para generar una forma. Los parches extraídos de esta forma se asocian con el valor $b_{k,GT}$.
    \item \textbf{Ejemplos Negativos:} Se generan formas perturbando el coeficiente $b_k$ de $\vect{b}_{GT}$. Los parches de estas formas se asocian con el valor $b_k$ perturbado. Esto ayuda a la CNN a aprender a distinguir diferentes valores de $b_k$.
\end{itemize}

\begin{figure}[htbp]
    \centering
    \includegraphics[width=0.8\columnwidth]{Figures/fig_pos_neg_examples.png}
    \caption{Ejemplo positivo (forma ground truth) y negativo (forma perturbada), con sus parches asociados.}
    \label{fig:pos_neg_examples_simplified}
\end{figure}

\subsection{Transformación de Forma al Espacio de la Imagen}
Una forma del SSM, definida por $\vect{b}$, se transforma del espacio del modelo al espacio de la imagen usando la transformación de pose $\mathcal{T}_{ESL}$ (obtenida de la Estimación de Pose Inicial, Sección~\ref{sec:esl_simplified}):
\begin{equation}
\mat{S}'_{\text{img}} = \mathcal{T}_{ESL}(\mat{S}'_{\text{SSM}}(\vect{b}))
\label{eq:shape_ssm_to_image_simplified}
\end{equation}
donde $\mat{S}'_{\text{SSM}}(\vect{b})$ es la forma generada por el SSM y $\mat{S}'_{\text{img}}$ son las coordenadas de los landmarks en la imagen.

\begin{figure}[htbp]
    \centering
    \includegraphics[width=0.8\columnwidth]{Figures/fig_shape_transformation.png}
    \caption{Transformación de una forma del SSM al espacio de la imagen usando $\mathcal{T}_{ESL}$.}
    \label{fig:ssm_to_image_transformation_simplified}
\end{figure}

\subsection{Extracción de Parches}
Para cada uno de los $\Ktotal$ landmarks de la forma $\mat{S}'_{\text{img}}$ en la imagen, se extrae un parche cuadrado de intensidad de $Q \times Q$ píxeles, centrado en el landmark.
Estos $\Ktotal$ parches se aplanan y concatenan en un vector de características $\vect{f}$, que es la entrada para la CNN.

\begin{figure}[htbp]
    \centering   
    \includegraphics[width=0.8\columnwidth]{Figures/fig_patch_extraction_details.png}
    \caption{Extracción de parches de $Q \times Q$ centrados en los landmarks de la forma proyectada en la imagen.}
    \label{fig:patch_extraction_cnn_simplified}
\end{figure}

\subsection{Aumento de Datos}
Para mejorar la generalización, se aplica aumento de datos a las imágenes antes de extraer parches. Esto puede incluir volteo horizontal y variaciones de intensidad (brillo/contraste).

\begin{figure}[htbp]
    \centering
    \includegraphics[width=0.7\columnwidth]{Figures/fig_data_augmentation.png}
    \caption{Ejemplos de aumento de datos: original, volteo horizontal, y variación de contraste/brillo.}
    \label{fig:data_augmentation_examples_simplified}
\end{figure}

\subsection{Discretización de Coeficientes $b_k$}
Como las CNNs se usan como clasificadores, el valor continuo del coeficiente $b_k$ se discretiza en $B$ `bins` o categorías. El rango de $b_k$ se divide en $B$ intervalos, y la etiqueta del bin correspondiente se usa para el entrenamiento.

\begin{figure}[htbp]
    \centering
    \includegraphics[width=0.7\textwidth]{Figures/fig_b_discretization.png}
    \caption{Discretización de un coeficiente $b_k$ en $B$ bins para la clasificación.}
    \label{fig:bk_discretization_simplified}
\end{figure}
%% \section{Entrenamiento de la Red Neuronal Convolucional (CNN) Híbrida}
% \label{sec:entrenamiento_cnn}

% Una vez que se han extraído y preparado los datos de parches (Sección~\ref{sec:extraccion_parches_cnn}), se procede al entrenamiento de los clasificadores CNN encargados de predecir los coeficientes de forma $b_j$. Se entrena una red neuronal convolucional híbrida de manera independiente para cada uno de los $m$ modos de variación del SSM (donde $j \in \{1, \dots, m\}$ es el índice del modo). El objetivo de cada red es aprender a mapear la apariencia local de la imagen, representada por los $K_{total}$ parches (ver Figura~\ref{fig:parches_entrada_cnn}), al bin discretizado correspondiente del coeficiente $b_j$.

% \subsection{Arquitectura de la Red Híbrida}

% \begin{figure}[htbp] % Posicionamiento flexible
%     \centering
%     \includegraphics[width=0.5\columnwidth]{Figures/ejemplo_parches_entrada_cnn.png}
%     \caption{Ejemplo de una selección de $N_p=16$ parches de entrada (cada uno de tamaño $Q \times Q$) de los $K_{total}$ extraídos alrededor de los landmarks de una forma proyectada (donde $K_{total}=144$ en este trabajo). Estos parches constituyen la entrada local a la CNN base.}
%     \label{fig:parches_entrada_cnn}
% \end{figure}

% La arquitectura de la CNN está diseñada para procesar eficientemente la información de múltiples parches y combinarla para una predicción global del parámetro de forma. La red, detallada en la Figura~\ref{fig:cnn_architecture_detailed}, se compone de las siguientes partes principales:
% \begin{enumerate}
%     \item \textbf{CNN Base Compartida ($\mathcal{C}$):} Sea $\mathcal{C}: \R^{Q \times Q \times C_{in}} \to \R^{d_{feat}}$ la transformación no lineal implementada por la CNN base, donde $Q \times Q$ es la dimensión espacial de un parche, $C_{in}$ el número de canales de entrada (usualmente $C_{in}=1$ para imágenes en escala de grises), y $d_{feat}$ la dimensionalidad del vector de características extraído. Esta CNN base típicamente consiste en varias capas convolucionales (ej. con filtros de $3 \times 3$ o $5 \times 5$), seguidas de funciones de activación (ej. ReLU) y, opcionalmente, capas de pooling (ej. MaxPooling) para reducir la dimensionalidad espacial y extraer características robustas. Es crucial que los parámetros (pesos) de esta CNN base $\mathcal{C}$ sean compartidos entre todos los $K_{total}$ parches de una misma imagen. Esta compartición de pesos para la secuencia de parches $\{\vect{x}_l\}_{l=1}^{K_{total}}$, donde cada $\vect{x}_l \in \R^{Q \times Q \times C_{in}}$, se modela mediante la aplicación de $\mathcal{C}$ a cada parche $\vect{x}_l$, produciendo una secuencia de vectores de características $\{\vect{v}_l\}_{l=1}^{K_{total}}$, con $\vect{v}_l = \mathcal{C}(\vect{x}_l) \in \R^{d_{feat}}$.
    
%     \item \textbf{Concatenación de Características:} La secuencia de $K_{total}$ vectores de características $\{\vect{v}_l\}_{l=1}^{K_{total}}$ se transforma en un único vector de características global $\vect{V} \in \R^{K_{total} \cdot d_{feat}}$ mediante la concatenación: $\vect{V} = [\vect{v}_1^\transpose, \vect{v}_2^\transpose, \dots, \vect{v}_{K_{total}}^\transpose]^\transpose$. Este vector $\vect{V}$ representa la información combinada de la apariencia local de todos los landmarks de la forma.
    
%     \item \textbf{Clasificador DNN ($\mathcal{D}$):} El vector de características concatenado $\vect{V}$ se introduce en un Perceptrón Multicapa (MLP), $\mathcal{D}: \R^{K_{total} \cdot d_{feat}} \to \R^{B}$, también conocido como red densamente conectada (DNN). Este MLP consiste en una o más capas ocultas con funciones de activación (ej. ReLU) y, finalmente, una capa de salida con $B$ neuronas (donde $B$ es el número de bins para $b_j$) y una función de activación softmax. La función softmax convierte las salidas de la red (logits) $\vect{z} = (z_0, \dots, z_{B-1})^\transpose$ en un vector de probabilidades $\vect{p} = (p_0, p_1, \dots, p_{B-1})^\transpose$ sobre los $B$ posibles bins:
%     \begin{equation}
%     p_i = \frac{e^{z_i}}{\sum_{l=0}^{B-1} e^{z_l}}, \quad \text{para } i = 0, \dots, B-1.
%     \label{eq:softmax}
%     \end{equation}
%     La predicción final del bin es el índice del bin con la mayor probabilidad, $\hat{y} = \arg\max_i p_i$.
% \end{enumerate}

% \begin{figure}[htbp] % Posicionamiento flexible
%     \centering
%     \includegraphics[width=1\columnwidth]{Figures/cnn_arquitectura_paper_style.png}
%     \caption{Arquitectura detallada de la CNN Híbrida para la predicción de un coeficiente $b_j$. (1) Los $K_{total}$ parches de entrada (donde $K_{total}=144$). (2) La CNN base compartida $\mathcal{C}$ (ej. con capas Conv, ReLU, Pool) se aplica a cada parche $\vect{x}_l$. (3) Los vectores de características resultantes $\vect{v}_l$ de cada parche se concatenan para formar $\vect{V}$. (4) El vector $\vect{V}$ se introduce en un clasificador DNN ($\mathcal{D}$) con una capa de salida Softmax que produce probabilidades sobre los $B$ bins.}
%     \label{fig:cnn_architecture_detailed}
% \end{figure}

% \subsection{Proceso de Entrenamiento}
% Se entrena una red separada para cada uno de los $m$ modos de forma $b_j$. Para el $j$-ésimo clasificador (correspondiente al modo $b_j$):
% \begin{itemize}
%     \item \textbf{Entrada:} Los $K_{total}$ parches extraídos de una imagen de entrenamiento, correspondientes a una instancia de forma (positiva o negativa) generada para el modo $j$.
%     \item \textbf{Etiqueta (Ground Truth):} La etiqueta one-hot $\vect{y}^{(j)} \in \{0,1\}^B$ del bin al que pertenece el valor del coeficiente $b_j$ de la instancia de forma utilizada para generar los parches.
%     \item \textbf{Función de Pérdida:} Se utiliza la función de pérdida de entropía cruzada categórica (Categorical Cross-Entropy), que mide la discrepancia entre la distribución de probabilidad predicha $\vect{p}^{(j)}$ y la distribución de probabilidad verdadera (one-hot) $\vect{y}^{(j)}$:
%     \begin{equation}
%     \mathcal{L}(\vect{y}^{(j)}, \vect{p}^{(j)}) = - \sum_{l=0}^{B-1} y_l^{(j)} \log(p_l^{(j)}).
%     \label{eq:categorical_crossentropy}
%     \end{equation}
%     \item \textbf{Optimizador:} El descenso de gradiente estocástico (SGD) o sus variantes, como Adam \cite{kingma2014adam} o AdamW \cite{loshchilov2017decoupled} (Adam con decaimiento de peso desacoplado), se utilizan para minimizar la función de pérdida ajustando los parámetros de la red (denotados colectivamente como $\bm{\theta}$) mediante el algoritmo de retropropagación del error (backpropagation).
%     \item \textbf{Regularización:} Para prevenir el sobreajuste, se pueden emplear técnicas de regularización. El Dropout \cite{srivastava2014dropout}, una forma de regularización estocástica, anula aleatoriamente un subconjunto de las salidas de las neuronas durante cada pasada de entrenamiento para mitigar la co-adaptación de características. También se puede utilizar el decaimiento de peso (regularización L2) en los pesos de la red, $\lambda \sum_{w \in \bm{\theta}} w^2$, donde $\lambda$ es el coeficiente de decaimiento.
% \end{itemize}
% El entrenamiento se realiza durante un número determinado de épocas, utilizando un conjunto de validación para monitorear el rendimiento y seleccionar el modelo que mejor generaliza (ej. guardando el modelo con la menor pérdida de validación o la mayor precisión de validación), como se ilustra en la Figura~\ref{fig:historial_entrenamiento_cnn}.

% Este enfoque de entrenar una CNN separada por modo permite que cada red se especialice en aprender las características visuales relevantes para predecir la variación específica capturada por ese modo de forma.

% \begin{figure}[htbp] % Posicionamiento flexible
%     \centering
%     \includegraphics[width=1\columnwidth]{Figures/ejemplo_historial_entrenamiento_cnn.png}
%     \caption{Ilustración del comportamiento típico de las métricas de pérdida y precisión durante el entrenamiento de un clasificador CNN para un modo $b_j$. Se muestran las curvas para los conjuntos de entrenamiento (azul) y validación (rojo). El monitoreo del rendimiento en el conjunto de validación es esencial para la selección del modelo y la detección de sobreajuste.}
%     \label{fig:historial_entrenamiento_cnn}
% \end{figure}

% El proceso se resume en el Algoritmo~\ref{alg:entrenamiento_matematico_cnn_acc}.

% \begin{algorithm}[htbp] % Posicionamiento flexible
% \caption{Entrenamiento del Clasificador CNN Híbrido para un Coeficiente $b_j$}
% \label{alg:entrenamiento_matematico_cnn_acc}
% \begin{algorithmic}[1]
%     \Require Conjunto de datos de entrenamiento $\mathcal{D}_{\text{entrena}} = \{ (\mathcal{X}^{(i)}, \vect{y}^{(j,i)}) \}_{i=1}^{N_{\text{entrena}}}$, donde $\mathcal{X}^{(i)} = \{\vect{x}_1^{(i)}, \dots, \vect{x}_{K_{total}}^{(i)}\}$ son los $K_{total}$ parches de la $i$-ésima muestra y $\vect{y}^{(j,i)}$ es la etiqueta one-hot del bin al que pertenece el coeficiente $b_j$ de dicha muestra.
%     \Require Conjunto de datos de validación $\mathcal{D}_{\text{val}}$.
%     \Require Número máximo de épocas $E_{\text{max}}$.
%     \Require Parámetros del optimizador (ej. tasa de aprendizaje $\alpha_{\text{lr}}$, coeficientes $\beta_1, \beta_2$ para Adam).
%     \Ensure Parámetros óptimos $\bm{\theta}^*$ de la red CNN híbrida (parámetros de $\mathcal{C}$ y $\mathcal{D}$) para el coeficiente $b_j$.

%     \State Inicializar los parámetros $\bm{\theta}$ de la red neuronal (parámetros $\bm{\theta}_{\mathcal{C}}$ de $\mathcal{C}$ y $\bm{\theta}_{\mathcal{D}}$ de $\mathcal{D}$) con valores aleatorios o predefinidos.
%     \State $\text{Acc}_{\text{val}}^* \leftarrow 0$ \Comment{Mejor exactitud en validación}
%     \State $\bm{\theta}^* \leftarrow \bm{\theta}$ \Comment{Parámetros para $\text{Acc}_{\text{val}}^*$}

%     \For{época $e = 1$ hasta $E_{\text{max}}$}
%         \State Reordenar aleatoriamente las muestras en $\mathcal{D}_{\text{entrena}}$.
%         \For{cada mini-lote $(\mathcal{X}_{\text{mb}}, \mathcal{Y}_{\text{mb}}^{(j)})$ de tamaño $N_{\text{mb}}$ extraído de $\mathcal{D}_{\text{entrena}}$}
%             \State \Comment{Fase de propagación hacia adelante}
%             \State Sea $\mat{V}_{\text{mb}}$ una matriz para almacenar las características concatenadas del mini-lote.
%             \For{cada secuencia de parches $\mathcal{X}_{\text{sample}} = \{\vect{x}_l\}_{l=1}^{K_{total}}$ en $\mathcal{X}_{\text{mb}}$}
%                 \State $\{\vect{v}_l\}_{l=1}^{K_{total}} \leftarrow \{\mathcal{C}(\vect{x}_l; \bm{\theta}_{\mathcal{C}})\}_{l=1}^{K_{total}}$ \Comment{Obtener vectores de características de parches}
%                 \State $\vect{V}_{\text{sample}} \leftarrow [\vect{v}_1^\transpose, \dots, \vect{v}_{K_{total}}^\transpose]^\transpose$ \Comment{Concatenar características para la muestra}
%                 \State Añadir $\vect{V}_{\text{sample}}$ a $\mat{V}_{\text{mb}}$.
%             \EndFor
%             \State $\mat{Z}_{\text{mb}} \leftarrow \mathcal{D}(\mat{V}_{\text{mb}}; \bm{\theta}_{\mathcal{D}})$ \Comment{Obtener logits para el mini-lote}
%             \State $\mat{P}_{\text{mb}}^{(j)} \leftarrow \text{Softmax}(\mat{Z}_{\text{mb}})$ \Comment{Calcular probabilidades (Eq.~\eqref{eq:softmax})}

%             \State \Comment{Cálculo de la función de pérdida}
%             \State $\mathcal{L}_{\text{mb}} \leftarrow \mathcal{L}(\mathcal{Y}_{\text{mb}}^{(j)}, \mat{P}_{\text{mb}}^{(j)})$ \Comment{Pérdida de entropía cruzada (Eq.~\eqref{eq:categorical_crossentropy})}
%             \If{se utiliza regularización L2 (decaimento de peso)}
%                 \State $\mathcal{L}_{\text{mb}} \leftarrow \mathcal{L}_{\text{mb}} + \lambda \sum_{w \in \bm{\theta}} w^2$ \Comment{Añadir término de regularización}
%             \EndIf

%             \State \Comment{Retropropagación del error y actualización de parámetros}
%             \State Calcular el gradiente de $\mathcal{L}_{\text{mb}}$ con respecto a los parámetros $\bm{\theta}$: $\nabla_{\bm{\theta}} \mathcal{L}_{\text{mb}}$.
%             \State Actualizar $\bm{\theta}$ usando el optimizador y $\nabla_{\bm{\theta}} \mathcal{L}_{\text{mb}}$.
%             \Comment{Dropout se aplica durante la propagación hacia adelante en entrenamiento.}
%         \EndFor

%         \State \Comment{Evaluación sobre el conjunto de validación}
%         \State Calcular la pérdida de validación $\mathcal{L}_{\text{val}}$ y la exactitud $\text{Acc}_{\text{val}}$ sobre $\mathcal{D}_{\text{val}}$ con $\bm{\theta}$.
%         \If{$\text{Acc}_{\text{val}} > \text{Acc}_{\text{val}}^*$}
%             \State $\text{Acc}_{\text{val}}^* \leftarrow \text{Acc}_{\text{val}}$
%             \State $\bm{\theta}^* \leftarrow \bm{\theta}$
%         \EndIf
%         \If{criterio de parada temprana satisfecho (ej. $\text{Acc}_{\text{val}}$ no mejora en $P$ épocas)}
%             \State \textbf{break} \Comment{Finalizar el entrenamiento}
%         \EndIf
%     \EndFor
%     \State \Return $\bm{\theta}^*$
% \end{algorithmic}
% \end{algorithm}

\section{Entrenamiento de la Red Neuronal Convolucional (CNN) Híbrida}
\label{sec:entrenamiento_cnn_simplified}

Una vez preparados los datos de parches (Sección~\ref{sec:extraccion_parches_cnn_simplified}), se entrena una CNN híbrida para cada uno de los $m$ coeficientes de forma $b_j$. Cada CNN aprende a mapear la apariencia local de la imagen (representada por $\Ktotal$ parches) al valor discretizado (bin) del coeficiente $b_j$ correspondiente.

\begin{figure}[htbp]
    \centering
    \includegraphics[width=0.5\columnwidth]{Figures/ejemplo_parches_entrada_cnn.png}
    \caption{Ejemplo de parches de entrada ($Q \times Q$) para la CNN, extraídos alrededor de los landmarks.}
    \label{fig:parches_entrada_cnn_simplified}
\end{figure}

\subsection{Arquitectura de la Red Híbrida}
La arquitectura de la CNN (ver Figura~\ref{fig:cnn_architecture_detailed_simplified}) incluye:
\begin{enumerate}
    \item \textbf{CNN Base Compartida ($\mathcal{C}$):} Una CNN base procesa cada uno de los $\Ktotal$ parches de entrada $\vect{x}_l$ (de $Q \times Q$ píxeles) de forma individual. Esta CNN base, con pesos compartidos entre todos los parches, extrae un vector de características $\vect{v}_l = \mathcal{C}(\vect{x}_l)$ para cada parche.
    
    \item \textbf{Concatenación de Características:} Los $\Ktotal$ vectores de características $\vect{v}_l$ se concatenan en un único vector global $\vect{V}$.
    
    \item \textbf{Clasificador DNN ($\mathcal{D}$):} El vector $\vect{V}$ se introduce en un Perceptrón Multicapa (MLP o DNN). La capa final de este MLP tiene $B$ neuronas (donde $B$ es el número de bins para $b_j$) y una función de activación softmax, que produce un vector de probabilidades $\vect{p}$ sobre los $B$ bins:
    \begin{equation}
    p_i = \frac{e^{z_i}}{\sum_{l=0}^{B-1} e^{z_l}},
    \label{eq:softmax_simplified}
    \end{equation}
    donde $z_i$ son las salidas (logits) de la red antes del softmax. El bin predicho es el que tiene mayor probabilidad.
\end{enumerate}

\begin{figure}[htbp]
    \centering
    \includegraphics[width=1\columnwidth]{Figures/cnn_arquitectura_paper_style.png}
    \caption{Arquitectura de la CNN Híbrida: (1) Parches de entrada. (2) CNN base compartida $\mathcal{C}$ aplicada a cada parche. (3) Concatenación de características en $\vect{V}$. (4) Clasificador DNN ($\mathcal{D}$) con salida Softmax.}
    \label{fig:cnn_architecture_detailed_simplified}
\end{figure}

\subsection{Proceso de Entrenamiento}
Para cada modo $b_j$, se entrena una red:
\begin{itemize}
    \item \textbf{Entrada:} Los $\Ktotal$ parches de una muestra de entrenamiento.
    \item \textbf{Etiqueta:} El bin verdadero al que pertenece el coeficiente $b_j$ de esa muestra.
    \item \textbf{Función de Pérdida:} Entropía cruzada categórica, que mide la diferencia entre la distribución de probabilidad predicha $\vect{p}^{(j)}$ y la verdadera $\vect{y}^{(j)}$ (one-hot):
    \begin{equation}
    \mathcal{L}(\vect{y}^{(j)}, \vect{p}^{(j)}) = - \sum_{l=0}^{B-1} y_l^{(j)} \log(p_l^{(j)}).
    \label{eq:categorical_crossentropy_simplified}
    \end{equation}
    \item \textbf{Optimizador:} Un algoritmo como Adam se usa para minimizar la pérdida ajustando los pesos de la red mediante retropropagación.
    \item \textbf{Regularización:} Técnicas como Dropout o decaimiento de peso se usan para prevenir el sobreajuste.
\end{itemize}
El entrenamiento se realiza por varias épocas, usando un conjunto de validación para monitorear el rendimiento y seleccionar el mejor modelo.

\begin{figure}[htbp]
    \centering
    \includegraphics[width=1\columnwidth]{Figures/ejemplo_historial_entrenamiento_cnn.png}
    \caption{Ejemplo de curvas de pérdida y precisión durante el entrenamiento, mostrando el rendimiento en los conjuntos de entrenamiento y validación.}
    \label{fig:historial_entrenamiento_cnn_simplified}
\end{figure}
%% % Asegúrate de que estos comandos están definidos en tu preámbulo:
% % \newcommand{\mat}[1]{\mathbf{#1}}   
% % \newcommand{\vect}[1]{\bm{#1}} % Requiere \usepackage{bm}
% % \newcommand{\transpose}{\mathsf{T}}
% % \newcommand{\R}{\mathbb{R}} 
% % \newcommand{\Ktotal}{K_{\text{total}}} % Si definiste K_total como macro

% \section{Predicción y Desdiscretización de Coeficientes de Forma }
% \label{sec:prediccion_desdiscretizacion_b}

% Una vez que los $m$ clasificadores basados en Redes Neuronales Convolucionales (CNN) híbridas han sido entrenados individualmente para cada modo de variación $k \in \{0, \dots, m-1\}$ (Sección~\ref{sec:entrenamiento_cnn}), estos se emplean para inferir los parámetros de forma $\vect{b} = (b_0, \dots, b_{m-1})^\transpose$ para nuevas imágenes. Este proceso inferencial es fundamental y consta de dos etapas secuenciales: la predicción de los bins discretos para cada coeficiente $b_k$ y la subsecuente conversión de estos bins a valores continuos que puedan ser utilizados por el Modelo Estadístico de Forma (SSM).

% \subsection{Predicción de Bins de Coeficientes $b_k$}
% \label{sec:prediccion_bins_bk}

% Dada una imagen de prueba $\mat{I}$, el primer paso consiste en obtener una estimación de su pose global. Para ello, se recurre al módulo de Estimación de Pose Inicial (ESL, Sección~\ref{sec:esl_matematica}), el cual proporciona una transformación $\mathcal{T}_{ESL}$. Esta transformación alinea una forma de referencia con la estructura anatómica de interés en la imagen $\mat{I}$. La forma de referencia es usualmente la forma media del SSM, $\overline{\vect{s}}$ (un vector de $\R^{K_{total}d}$, donde $d=2$), que se remodela a una matriz $\overline{\mat{S}} \in \R^{K_{total} \times d}$. La aplicación de $\mathcal{T}_{ESL}$ a cada landmark (fila) de $\overline{\mat{S}}$ produce un conjunto de $K_{total}$ coordenadas de landmarks en el espacio de la imagen:
% \begin{equation}
% \mat{S}'_{\text{img,mean}} = \mathcal{T}_{ESL}(\overline{\mat{S}}),
% \label{eq:mean_shape_to_image} % Etiqueta para esta transformación
% \end{equation}
% donde $\mat{S}'_{\text{img,mean}} \in \R^{K_{total} \times d}$ son las coordenadas de los landmarks de la forma media proyectados sobre la imagen.

% Posteriormente, para cada uno de los $K_{total}$ landmarks de $\mat{S}'_{\text{img,mean}}$, se extrae un parche de imagen $P_i \in \R^{Q \times Q \times C_{in}}$ (donde $i \in \{1, \dots, K_{total}\}$, $Q$ es la dimensión del parche, y $C_{in}$ el número de canales). Este conjunto de $K_{total}$ parches, $\mathcal{P} = \{P_1, P_2, \dots, P_{K_{total}}\}$, constituye la entrada local de apariencia para las CNN.

% Si durante la fase de entrenamiento se empleó una reducción de dimensionalidad mediante Análisis de Componentes Principales (PCA) sobre los parches (o sobre características extraídas de ellos), la misma transformación PCA, $\mathcal{F}_{\text{PCA},k}$, se aplica a los parches extraídos de la imagen de prueba antes de introducirlos al clasificador del modo $k$.

% El conjunto de $K_{total}$ parches $\mathcal{P}$ (o sus representaciones transformadas) se introduce en cada uno de los $m$ clasificadores CNN previamente entrenados, $\text{CNN}_0, \text{CNN}_1, \dots, \text{CNN}_{m-1}$. Cada clasificador $\text{CNN}_k$, especializado en el modo de variación $b_k$, procesa la información de los parches para producir un vector de $B$ logits (valores no normalizados), $\vect{z}_k = (z_{k,0}, z_{k,1}, \dots, z_{k,B-1})^\transpose$. Estos logits se transforman en un vector de probabilidades $\vect{p}_k = (p_{k,0}, p_{k,1}, \dots, p_{k,B-1})^\transpose$ mediante la función softmax (referenciar Ecuación~\eqref{eq:softmax} si es la misma, o mantener la nueva si hay matices):
% \begin{equation}
% p_{k,j} = \frac{e^{z_{k,j}}}{\sum_{l=0}^{B-1} e^{z_{k,l}}}, \quad \text{para } j \in \{0, \dots, B-1\}.
% \label{eq:softmax_prediccion} 
% \end{equation}
% donde $p_{k,j}$ representa la probabilidad estimada por $\text{CNN}_k$ de que el coeficiente $b_k$ pertenezca al $j$-ésimo bin discreto. El índice del bin predicho para el modo $k$, denotado como $\hat{y}_k$, se determina seleccionando el bin con la máxima probabilidad a posteriori:
% \begin{equation}
% \hat{y}_k = \arg\max_{j \in \{0, \dots, B-1\}} p_{k,j}.
% \label{eq:argmax_prediccion_bin}
% \end{equation}
% Este procedimiento se repite para todos los $m$ modos de variación principales, resultando en un vector de índices de bin predichos $\hat{\vect{y}} = (\hat{y}_0, \hat{y}_1, \dots, \hat{y}_{m-1})^\transpose$. Este vector $\hat{\vect{y}}$ es una representación discreta de la forma estimada para la imagen $\mat{I}$.

% \begin{figure}[htbp]
%     \centering
%     \includegraphics[width=0.9\textwidth]{Figures/prediccion_flujo.png} 
%     \caption{Diagrama de flujo ilustrando el proceso de predicción de bins para los coeficientes $b_k$. (1) Una imagen de entrada $\mat{I}$ es procesada por el módulo ESL para obtener la pose $\mathcal{T}_{ESL}$. (2) La forma media $\overline{\mat{S}}$ (remodelada de $\overline{\vect{s}}$) se transforma a $\mat{S}'_{\text{img,mean}}$ usando $\mathcal{T}_{ESL}$. (3) Se extraen $K_{total}$ parches $P_i$ centrados en los landmarks de $\mat{S}'_{\text{img,mean}}$. (4) (Opcional) Se aplica PCA a los parches. (5) Los parches (o sus features) alimentan la $\text{CNN}_k$ específica del modo. (6) La $\text{CNN}_k$ produce probabilidades $\vect{p}_k$ sobre los $B$ bins. (7) Se selecciona el bin $\hat{y}_k$ con máxima probabilidad. (8) El proceso se repite para los $m$ modos, generando el vector $\hat{\vect{y}}$.}
%     \label{fig:prediccion_flujo}
% \end{figure}

% \subsection{Desdiscretización de Coeficientes $b_k$}
% \label{sec:desdiscretizacion_bk}

% Los índices de bin predichos $\hat{y}_k$ contenidos en el vector $\hat{\vect{y}}$ son representaciones categóricas y deben ser convertidos nuevamente a valores continuos para los coeficientes $b_k$. Estos valores continuos son necesarios para reconstruir una instancia de forma del SSM, $\vect{s} \approx \overline{\vect{s}} + \mat{P} \vect{b}$ (ver Ecuación~\eqref{eq:ssm_reconstruction}). El proceso de desdiscretización se fundamenta en la definición de los bordes de los bins establecida durante la fase de entrenamiento.

% Recordemos que, para cada modo $k$, el coeficiente $b_k$ exhibe una varianza $\lambda_k$ (el $k$-ésimo valor propio de la matriz de covarianza de las formas alineadas), y por lo tanto una desviación estándar $\sigma_k = \sqrt{\lambda_k}$. El rango de variación significativo para $b_k$ se establece típicamente como $[-c \cdot \sigma_k, c \cdot \sigma_k]$, donde $c$ es un factor de cobertura (comúnmente $c=3$). Sean $R_{k,\text{min}} = -c \cdot \sigma_k$ y $R_{k,\text{max}} = c \cdot \sigma_k$ los límites inferior y superior de este rango para el modo $k$.

% Este rango $[R_{k,\text{min}}, R_{k,\text{max}}]$ se dividió en $B$ intervalos (bins) de igual amplitud durante la fase de discretización para el entrenamiento de las CNN. La amplitud de cada bin para el modo $k$ es:
% \begin{equation}
% w_k = \frac{R_{k,\text{max}} - R_{k,\text{min}}}{B} = \frac{2c\sigma_k}{B}.
% \label{eq:bin_width_k} % Etiqueta para ancho de bin
% \end{equation}
% Los bordes de los $B$ bins se definen entonces como una secuencia de $B+1$ puntos equiespaciados $e_{k,0}, e_{k,1}, \dots, e_{k,B}$, donde:
% \begin{equation}
% e_{k,j} = R_{k,\text{min}} + j \cdot w_k, \quad \text{para } j \in \{0, 1, \dots, B\}.
% \label{eq:bordes_bin}
% \end{equation}
% Así, el $j$-ésimo bin (indexado desde $j=0$) para el modo $k$ corresponde al intervalo $[e_{k,j}, e_{k,j+1}]$.

% Si el clasificador $\text{CNN}_k$ predice el índice de bin $\hat{y}_k$ para el coeficiente $b_k$, el valor continuo estimado para $b_k$, denotado como $\hat{b}_k$, se obtiene calculando el punto medio de dicho bin:
% \begin{equation}
% \hat{b}_k = \frac{e_{k,\hat{y}_k} + e_{k,\hat{y}_k+1}}{2} = R_{k,\text{min}} + (\hat{y}_k + 0.5) \cdot w_k.
% \label{eq:desdiscretizacion_bk_continuo}
% \end{equation}
% Este cálculo se realiza para cada uno de los $m$ modos para los cuales se obtuvo una predicción de bin, resultando en el vector de coeficientes de forma continuos estimados $\hat{\vect{b}} = (\hat{b}_0, \hat{b}_1, \dots, \hat{b}_{m-1})^\transpose$. Este vector $\hat{\vect{b}}$ constituye la estimación inicial de la forma de la estructura anatómica en la imagen de prueba y sirve como el punto de partida para el subsecuente proceso de ajuste iterativo del SSM (descrito en una sección posterior, e.g., Sección~\ref{sec:ajuste_asm}).

% \begin{figure}[htbp]
%     \centering
%     \includegraphics[width=0.7\textwidth]{Figures/desdiscretizacion_proceso.png}
%     \caption{Ilustración del proceso de desdiscretización para un único coeficiente $b_k$. El eje horizontal representa el rango continuo del coeficiente, delimitado por $R_{k,\text{min}} = -c\sigma_k$ y $R_{k,\text{max}} = c\sigma_k$. Este rango se divide en $B$ bins (en este ejemplo, $B=3$, con ancho $w_k$). Si la CNN predice el bin $\hat{y}_k=1$ (el segundo bin, resaltado), el valor continuo $\hat{b}_k$ se estima como el centro de este bin, $e_{k,1} + w_k/2$, según la Ecuación~\eqref{eq:desdiscretizacion_bk_continuo}.}
%     \label{fig:desdiscretizacion_proceso}
% \end{figure}

% Es crucial destacar que la fidelidad de esta estimación inicial $\hat{\vect{b}}$ es altamente dependiente del rendimiento discriminativo de los clasificadores CNN y de la precisión de la estimación de pose inicial ESL. Inexactitudes acumuladas en estas etapas pueden resultar en un vector $\hat{\vect{b}}$ que se desvíe significativamente de la forma real del objeto, lo cual podría comprometer la capacidad del subsecuente algoritmo de ajuste fino para converger a una segmentación precisa.

\section{Predicción y Desdiscretización de Coeficientes de Forma}
\label{sec:prediccion_desdiscretizacion_b_simplified}

Una vez entrenadas las $m$ CNNs (Sección~\ref{sec:entrenamiento_cnn_simplified}), se usan para inferir los parámetros de forma $\vect{b}$ para nuevas imágenes. Esto implica predecir los bins discretos para cada coeficiente $b_k$ y luego convertirlos a valores continuos.

\subsection{Predicción de Bins de Coeficientes $b_k$}
\label{sec:prediccion_bins_bk_simplified}

\begin{figure}[htbp]
    \centering
    \includegraphics[width=\textwidth]{Figures/prediccion_flujo.png} 
    \caption{Flujo de predicción de bins: (1) Imagen de entrada. (2) ESL para pose. (3) Proyección de forma media. (4) Extracción de parches. (5) (Opcional PCA). (6) Entrada a CNNs. (7) Predicción de bins $\hat{y}_k$.}
    \label{fig:prediccion_flujo_simplified}
\end{figure}

Para una imagen de prueba $\mat{I}$:
\begin{enumerate}
    \item Se estima la pose global $\mathcal{T}_{ESL}$ usando el módulo ESL (Sección~\ref{sec:esl_simplified}).
    \item La forma media del SSM, $\mean{\mat{S}}$, se transforma al espacio de la imagen usando esta pose:
    \begin{equation}
    \mat{S}'_{\text{img,mean}} = \mathcal{T}_{ESL}(\mean{\mat{S}}).
    \label{eq:mean_shape_to_image_simplified}
    \end{equation}
    \item Se extraen $\Ktotal$ parches de imagen $P_i$ (de $Q \times Q$ píxeles) centrados en los landmarks de $\mat{S}'_{\text{img,mean}}$.
    \item Este conjunto de parches, $\set{P} = \{P_1, \dots, P_{\Ktotal}\}$, (opcionalmente tras aplicar PCA si se usó en el entrenamiento) se introducen en cada una de las $m$ CNNs entrenadas ($\text{CNN}_0, \dots, \text{CNN}_{m-1}$).
    \item Cada $\text{CNN}_k$ produce probabilidades $\vect{p}_k$ sobre los $B$ posibles bins para el coeficiente $b_k$, usando la función softmax (similar a Ecuación~\eqref{eq:softmax_simplified} de una sección anterior):
    \begin{equation}
    p_{k,j} = \frac{e^{z_{k,j}}}{\sum_{l=0}^{B-1} e^{z_{k,l}}}.
    \label{eq:softmax_prediccion_simplified} 
    \end{equation}
    \item El bin predicho $\hat{y}_k$ para el modo $k$ es el que tiene la máxima probabilidad:
    \begin{equation}
    \hat{y}_k = \arg\max_{j \in \{0, \dots, B-1\}} p_{k,j}.
    \label{eq:argmax_prediccion_bin_simplified}
    \end{equation}
\end{enumerate}
Esto resulta en un vector de índices de bin predichos $\hat{\vect{y}} = (\hat{y}_0, \dots, \hat{y}_{m-1})^T$.

\subsection{Desdiscretización de Coeficientes $b_k$}
\label{sec:desdiscretizacion_bk_simplified}

Los índices de bin predichos $\hat{y}_k$ se convierten a valores continuos $\hat{b}_k$ para reconstruir la forma del SSM (Ecuación~\eqref{eq:ssm_reconstruction_simplified} de una sección anterior).
El rango de variación de $b_k$ es típicamente $[-c \sigma_k, c \sigma_k]$, donde $\sigma_k = \sqrt{\lambda_k}$ es la desviación estándar del modo $k$. Sean $R_{k,\text{min}}$ y $R_{k,\text{max}}$ los límites de este rango.
La amplitud de cada uno de los $B$ bins para el modo $k$ es:
\begin{equation}
w_k = \frac{R_{k,\text{max}} - R_{k,\text{min}}}{B}.
\label{eq:bin_width_k_simplified}
\end{equation}
Si $\hat{y}_k$ es el bin predicho, el valor continuo estimado $\hat{b}_k$ es el centro de ese bin:
\begin{equation}
\hat{b}_k = R_{k,\text{min}} + (\hat{y}_k + 0.5) \cdot w_k.
\label{eq:desdiscretizacion_bk_continuo_simplified}
\end{equation}
Esto se hace para todos los $m$ modos, obteniendo el vector de coeficientes continuos $\hat{\vect{b}} = (\hat{b}_0, \dots, \hat{b}_{m-1})^T$. Este $\hat{\vect{b}}$ es la estimación inicial de la forma.

\begin{figure}[htbp]
    \centering
    \includegraphics[width=0.9\textwidth]{Figures/desdiscretizacion_proceso.png}
    \caption{Desdiscretización: El rango de $b_k$ se divide en $B$ bins. Si se predice el bin $\hat{y}_k$, $\hat{b}_k$ es el centro de ese bin.}
    \label{fig:desdiscretizacion_proceso_simplified}
\end{figure}

La calidad de $\hat{\vect{b}}$ depende del rendimiento de las CNNs y de la precisión de la ESL.

%
% \section{Ajuste Iterativo del SSM (Active Shape Model - ASM)}
% \label{sec:ajuste_asm}

% Una vez obtenida una estimación inicial de los parámetros de forma $\hat{\vect{b}}$ (Sección~\ref{sec:prediccion_desdiscretizacion_b}) y una pose inicial (e.g., escala $s_{ESL}$, orientación $\theta_{ESL}$, y traslación $\vect{t}_{ESL}$) a partir de la etapa ESL (Sección~\ref{sec:esl_matematica}), se procede a refinar estos parámetros mediante un proceso de ajuste iterativo conocido como Active Shape Model (ASM) \cite{cootes1995active}. El ASM busca la instancia del modelo de forma y la pose que mejor se adaptan a la evidencia presente en la imagen de entrada.

% \begin{figure}[htbp]
%     \centering
%     \includegraphics[width=0.5\textwidth]{Figures/step0_initial_projection_newIdx20_cat1_id103_1.png}
%     \caption{Paso 0: Proyección Inicial del SSM sobre la imagen. Se muestra la forma inicial (generada a partir de $\hat{\vect{b}}$ y la pose ESL) superpuesta en la imagen.}
%     \label{fig:step0_initial_projection}
% \end{figure}

% El proceso de ajuste es iterativo y consta de los siguientes pasos principales en cada iteración $t$:

% \begin{enumerate}
%     \item \textbf{Generación de la Instancia de Forma Actual:}
%     A partir de los parámetros de forma actuales $\vect{b}^{(t)}$, se genera una instancia de forma vectorizada $\vect{s}_{\text{SSM}}(\vect{b}^{(t)})$ en el espacio canónico del SSM utilizando la Ecuación~\eqref{eq:ssm_reconstruction}:
%     \begin{equation}
%     \vect{s}_{\text{SSM}}(\vect{b}^{(t)}) = \overline{\vect{s}} + \mat{P} \vect{b}^{(t)}.
%     \label{eq:asm_shape_instance_vec}
%     \end{equation}
%     Esta forma vectorizada se remodela a una matriz $\mat{S}'_{\text{SSM}}(\vect{b}^{(t)}) \in \R^{\Ktotal \times \dval}$ (donde $\Ktotal$ es el número de landmarks y $\dval$ la dimensionalidad, e.g., 2).
    
%     \begin{figure}[htbp]
%     \centering
%     \begin{subfigure}[b]{0.4\textwidth}
%         \centering
%         \includegraphics[width=\textwidth]{Figures/step1_canonical_shape_newIdx20_cat1_id103_1.png}
%         \caption{Iteración 1.}
%         \label{fig:step1_canonical_iter1}
%     \end{subfigure}
%     \hfill % Espacio entre subfigures
%     \begin{subfigure}[b]{0.4\textwidth}
%         \centering
%         \includegraphics[width=\textwidth]{Figures/step1_canonical_shape_newIdx20_cat1_id103.png}
%         \caption{Iteración 10.}
%         \label{fig:step1_canonical_iter10}
%     \end{subfigure}
%     \caption{Paso 1: Generación de la Instancia de Forma Canónica $\mat{S}'_{\text{SSM}}(\vect{b}^{(t)})$.}
%     \label{fig:step1_canonical_shape_comparison}
%     \end{figure}

%     \item \textbf{Proyección al Espacio de la Imagen:}
%     La instancia de forma canónica $\mat{S}'_{\text{SSM}}(\vect{b}^{(t)})$ se proyecta al espacio de la imagen aplicando la transformación de similitud definida por los parámetros de pose actuales $(s^{(t)}, \theta^{(t)}, \vect{t}^{(t)})$ a cada uno de sus $\Ktotal$ puntos:
%     \begin{equation}
%     (\mat{S}'_{\text{img}})_{i,\cdot} = s^{(t)} (\mat{S}'_{\text{SSM}}(\vect{b}^{(t)}))_{i,\cdot} \mat{R}(\theta^{(t)})^\transpose + (\vect{t}^{(t)})^\transpose, \quad \text{para } i=1,\dots,\Ktotal,
%     \label{eq:asm_projection_to_image}
%     \end{equation}
%     donde $(\mat{X})_{i,\cdot}$ denota la $i$-ésima fila (landmark) de la matriz $\mat{X}$, y $\mat{R}(\theta^{(t)})$ es la matriz de rotación. El resultado es $\mat{S}'_{\text{img}}(\vect{b}^{(t)}, s^{(t)}, \theta^{(t)}, \vect{t}^{(t)})$.

%     \begin{figure}[htbp]
%     \centering
%     \begin{subfigure}[b]{0.48\textwidth}
%         \centering
%         \includegraphics[width=\textwidth]{Figures/step2_projected_shape_newIdx20_cat1_id103_1.png}
%         \caption{Iteración 1.}
%         \label{fig:step2_projected_iter1}
%     \end{subfigure}
%     \hfill
%     \begin{subfigure}[b]{0.48\textwidth}
%         \centering
%         \includegraphics[width=\textwidth]{Figures/step2_projected_shape_newIdx20_cat1_id103.png}
%         \caption{Iteración 10.}
%         \label{fig:step2_projected_iter10}
%     \end{subfigure}
%     \caption{Paso 2: Proyección de la Forma Canónica al Espacio de la Imagen, $\mat{S}'_{\text{img}}(\vect{b}^{(t)}, s^{(t)}, \theta^{(t)}, \vect{t}^{(t)})$.}
%     \label{fig:step2_projected_shape_comparison}
%     \end{figure}

%     \item \textbf{Búsqueda Local de Puntos Óptimos:}
%     Para cada uno de los $\Ktotal$ landmarks $\vect{p}'_{i,\text{img}}$ (la $i$-ésima fila) de la forma proyectada $\mat{S}'_{\text{img}}$, se busca una nueva posición candidata $\vect{p}^*_{i,\text{img}}$ en la imagen que mejor se corresponda con la apariencia local esperada. Esta búsqueda se realiza a lo largo de la normal $\vect{n}_i$ al contorno en el landmark $i$ (calculada a partir de la forma media $\overline{\mat{S}}$ para estabilidad).
%     Se muestrean perfiles de intensidad $\vect{g}_{\text{obs}}$ en puntos a lo largo de $\vect{n}_i$ (en un rango de búsqueda, ej. $\pm L_s$ píxeles) alrededor de $\vect{p}'_{i,\text{img}}$. Cada perfil observado se compara con el modelo de perfil estadístico $(\overline{\vect{g}}_i, \mat{\Sigma}_i)$ del landmark $i$ usando la distancia de Mahalanobis (Ecuación~\eqref{eq:mahalanobis_distance}). El punto candidato que minimiza esta distancia se selecciona como $\vect{p}^*_{i,\text{img}}$. El conjunto de estos $\Ktotal$ puntos óptimos forma la "forma objetivo" $\mat{S}^*_{\text{target,img}}$.

%     \begin{figure}[htbp]
%     \centering
%     \begin{subfigure}[b]{0.48\textwidth}
%         \centering
%         \includegraphics[width=\textwidth]{Figures/step3_local_search_newIdx20_cat1_id103_1.png}
%         \caption{Iteración 1.}
%         \label{fig:step3_localsearch_iter1}
%     \end{subfigure}
%     \hfill
%     \begin{subfigure}[b]{0.48\textwidth}
%         \centering
%         \includegraphics[width=\textwidth]{Figures/step3_local_search_newIdx20_cat1_id103.png}
%         \caption{Iteración 10.}
%         \label{fig:step3_localsearch_iter10}
%     \end{subfigure}
%     \caption{Paso 3: Búsqueda Local de Puntos Óptimos en la imagen, generando la forma objetivo $\mat{S}^*_{\text{target,img}}$.}
%     \label{fig:step3_local_search_comparison}
%     \end{figure}

%     \item \textbf{Actualización de Parámetros de Pose:}
%     Se calcula la transformación de similitud $(s_{\text{new}}, \theta_{\text{new}}, \vect{t}_{\text{new}})$ que mejor alinea la instancia de forma canónica actual $\mat{S}'_{\text{SSM}}(\vect{b}^{(t)})$ con la forma objetivo $\mat{S}^*_{\text{target,img}}$. Esto se realiza resolviendo el problema de Procrustes.
%     Los parámetros de pose se actualizan con un factor de amortiguamiento $\alpha_p \in (0, 1]$:
%     \begin{align}
%     \theta^{(t+1)} &= \theta^{(t)} + \alpha_p (\theta_{\text{new}} - \theta^{(t)}), \\
%     \vect{t}^{(t+1)} &= \vect{t}^{(t)} + \alpha_p (\vect{t}_{\text{new}} - \vect{t}^{(t)}).
%     \end{align}
%     La escala $s$ puede actualizarse de forma similar ($s^{(t+1)} = s^{(t)} + \alpha_s (s_{\text{new}} - s^{(t)})$) o mantenerse fija ($s^{(t+1)} = s^{(t)}$), como se menciona para la variante P2.

%     \begin{figure}[htbp]
%     \centering
%     \begin{subfigure}[b]{0.48\textwidth}
%         \centering
%         \includegraphics[width=\textwidth]{Figures/step4_pose_update_newIdx20_cat1_id103_1.png}
%         \caption{Iteración 1.}
%         \label{fig:step4_poseupdate_iter1}
%     \end{subfigure}
%     \hfill
%     \begin{subfigure}[b]{0.48\textwidth}
%         \centering
%         \includegraphics[width=\textwidth]{Figures/step4_pose_update_newIdx20_cat1_id103.png}
%         \caption{Iteración 10.}
%         \label{fig:step4_poseupdate_iter10}
%     \end{subfigure}
%     \caption{Paso 4: Actualización de Parámetros de Pose $(s^{(t+1)}, \theta^{(t+1)}, \vect{t}^{(t+1)})$ alineando $\mat{S}'_{\text{SSM}}(\vect{b}^{(t)})$ con $\mat{S}^*_{\text{target,img}}$.}
%     \label{fig:step4_pose_update_comparison}
%     \end{figure}

%     \item \textbf{Actualización de Parámetros de Forma:}
%     La forma objetivo $\mat{S}^*_{\text{target,img}}$ se transforma de vuelta al espacio canónico del SSM utilizando la inversa de la transformación de pose \textit{actualizada} $(s^{(t+1)}, \theta^{(t+1)}, \vect{t}^{(t+1)})$. Para cada punto $\vect{p}^*_{i,\text{img}}$ de $\mat{S}^*_{\text{target,img}}$:
%     \begin{equation}
%     (\mat{S}'^*_{\text{target,SSM}})_{i,\cdot} = (s^{(t+1)})^{-1} ((\vect{p}^*_{i,\text{img}})^\transpose - (\vect{t}^{(t+1)})^\transpose) \mat{R}(-\theta^{(t+1)})^\transpose.
%     \label{eq:asm_target_to_ssm_space}
%     \end{equation}
%     Esta forma $\mat{S}'^*_{\text{target,SSM}}$ se vectoriza a $\vect{s}'^*_{\text{target,SSM}}$ y se proyecta sobre la base del SSM para obtener un nuevo vector de parámetros de forma $\vect{b}_{\text{new}}$:
%     \begin{equation}
%     \vect{b}_{\text{new}} = \mat{P}^\transpose (\vect{s}'^*_{\text{target,SSM}} - \overline{\vect{s}}).
%     \label{eq:asm_project_to_b}
%     \end{equation}
%     Los nuevos parámetros de forma se limitan (clamping) a un rango plausible, típicamente $\pm n_{\text{std}}\sigma_k$ para cada componente $b_{k,\text{new}}$ (donde $\sigma_k=\sqrt{\lambda_k}$):
%     \begin{equation}
%     b_k^{(t+1)} = \text{clip}(b_{k,\text{new}}, -n_{\text{std}}\sqrt{\lambda_k}, n_{\text{std}}\sqrt{\lambda_k}).
%     \label{eq:asm_b_clamping}
%     \end{equation}

%     \begin{figure}[htbp]
%     \centering
%     \begin{subfigure}[b]{0.48\textwidth}
%         \centering
%         \includegraphics[width=\textwidth]{Figures/step5_shape_update_newIdx20_cat1_id103_1.png}
%         \caption{Iteración 1.}
%         \label{fig:step5_shapeupdate_iter1}
%     \end{subfigure}
%     \hfill
%     \begin{subfigure}[b]{0.48\textwidth}
%         \centering
%         \includegraphics[width=\textwidth]{Figures/step5_shape_update_newIdx20_cat1_id103.png}
%         \caption{Iteración 10.}
%         \label{fig:step5_shapeupdate_iter10}
%     \end{subfigure}
%     \caption{Paso 5: Actualización de Parámetros de Forma $\vect{b}^{(t+1)}$ proyectando $\mat{S}'^*_{\text{target,SSM}}$ (vectorizada) sobre la base del SSM.}
%     \label{fig:step5_shape_update_comparison}
%     \end{figure}

%     \item \textbf{Comprobación de Convergencia:}
%     El proceso iterativo se detiene si se alcanza un número máximo de iteraciones ($max\_iters$) o si el cambio en la posición de los landmarks (o en los parámetros $\vect{b}$ y de pose) entre iteraciones sucesivas cae por debajo de una tolerancia predefinida $\epsilon$. Una medida común es la norma de la diferencia entre las formas proyectadas: $\|\mat{S}'_{\text{img}} - \mat{S}'_{\text{img,prev}}\|_F < \epsilon$.

%     \begin{figure}[htbp]
%     \centering
%     \begin{subfigure}[b]{0.48\textwidth}
%         \centering
%         \includegraphics[width=\textwidth]{Figures/step6_end_of_iteration_newIdx20_cat1_id103_1.png}
%         \caption{Iteración 1.}
%         \label{fig:step6_enditeration_iter1}
%     \end{subfigure}
%     \hfill
%     \begin{subfigure}[b]{0.48\textwidth}
%         \centering
%         \includegraphics[width=\textwidth]{Figures/step6_end_of_iteration_newIdx20_cat1_id103.png}
%         \caption{Iteración 10.}
%         \label{fig:step6_enditeration_iter10}
%     \end{subfigure}
%     \caption{Paso 6: Estado Final de la Iteración, mostrando la forma $\mat{S}'_{\text{img}}$ ajustada en la imagen después de la actualización de forma y pose.}
%     \label{fig:step6_end_of_iteration_comparison}
%     \end{figure}
% \end{enumerate}

% \begin{algorithm}[htbp]
% \caption{Ajuste Iterativo del SSM (ASM)}
% \label{alg:asm_fitting}
% \begin{algorithmic}[1]
% \State Inicializar parámetros de forma $\vect{b}^{(0)}$ y pose $(s^{(0)}, \theta^{(0)}, \vect{t}^{(0)})$.
% \State $\mat{S}'_{\text{img,prev}} \leftarrow \text{Proyectar}(\text{reshape}(\overline{\vect{s}} + \mat{P}\vect{b}^{(0)}, \Ktotal, \dval), s^{(0)}, \theta^{(0)}, \vect{t}^{(0)})$ \Comment{Proyección inicial}
% \For{$t = 0$ \textbf{to} $max\_iters - 1$}
%     \State $\vect{s}_{\text{SSM}} \leftarrow \overline{\vect{s}} + \mat{P} \vect{b}^{(t)}$
%     \State $\mat{S}'_{\text{SSM}} \leftarrow \text{reshape}(\vect{s}_{\text{SSM}}, \Ktotal, \dval)$ \Comment{Paso 1}
%     \State $\mat{S}'_{\text{img}} \leftarrow \text{Proyectar}(\mat{S}'_{\text{SSM}}, s^{(t)}, \theta^{(t)}, \vect{t}^{(t)})$ \Comment{Paso 2, ver Ecuación~\eqref{eq:asm_projection_to_image}}
%     \State \textbf{Búsqueda Local (Paso 3):}
%     \State Inicializar $\mat{S}^*_{\text{target,img}}$ (matriz $\Ktotal \times \dval$)
%     \For{cada landmark $i = 1, \dots, \Ktotal$}
%         \State $\vect{p}'_{i,\text{img}} \leftarrow i\text{-ésima fila de } \mat{S}'_{\text{img}}$
%         \State Calcular normal estable $\vect{n}_i$ en $\vect{p}'_{i,\text{img}}$ (basada en $\overline{\mat{S}}$).
%         \State Buscar a lo largo de $\vect{n}_i$ para encontrar $\vect{p}^*_{i,\text{img}}$ que minimice $D_M^2(\vect{g}_{\text{obs}}, (\overline{\vect{g}}_i, \mat{\Sigma}_i))$.
%         \State Asignar $\vect{p}^*_{i,\text{img}}$ a la $i$-ésima fila de $\mat{S}^*_{\text{target,img}}$.
%     \EndFor
%     \State \textbf{Actualización de Pose (Paso 4):}
%     \State Calcular $(s_{\text{new}}, \theta_{\text{new}}, \vect{t}_{\text{new}})$ alineando $\mat{S}'_{\text{SSM}}$ con $\mat{S}^*_{\text{target,img}}$ (Procrustes).
%     \State $\theta^{(t+1)} \leftarrow \theta^{(t)} + \alpha_p (\theta_{\text{new}} - \theta^{(t)})$
%     \State $\vect{t}^{(t+1)} \leftarrow \vect{t}^{(t)} + \alpha_p (\vect{t}_{\text{new}} - \vect{t}^{(t)})$
%     \State $s^{(t+1)} \leftarrow s^{(t)}$ \Comment{Escala fija en este ejemplo}
%     \State \textbf{Actualización de Forma (Paso 5):}
%     \State $\mat{S}'^*_{\text{target,SSM}} \leftarrow \text{InversaProyectar}(\mat{S}^*_{\text{target,img}}, s^{(t+1)}, \theta^{(t+1)}, \vect{t}^{(t+1)})$ \Comment{Ver Ecuación~\eqref{eq:asm_target_to_ssm_space}}
%     \State $\vect{s}'^*_{\text{target,SSM}} \leftarrow \text{vec}(\mat{S}'^*_{\text{target,SSM}})$
%     \State $\vect{b}_{\text{new}} \leftarrow \mat{P}^\transpose (\vect{s}'^*_{\text{target,SSM}} - \overline{\vect{s}})$
%     \State $\vect{b}^{(t+1)} \leftarrow \text{clip}(\vect{b}_{\text{new}}, -n_{\text{std}}\vect{\sigma}_b, n_{\text{std}}\vect{\sigma}_b)$ \Comment{Donde $\vect{\sigma}_b = (\sqrt{\lambda_0}, \dots, \sqrt{\lambda_{m-1}})^\transpose$}
%     \State \textbf{Convergencia (Paso 6):}
%     \State Calcular $\mat{S}'_{\text{img,current}} \leftarrow \text{Proyectar}(\text{reshape}(\overline{\vect{s}} + \mat{P}\vect{b}^{(t+1)}, \Ktotal, \dval), s^{(t+1)}, \theta^{(t+1)}, \vect{t}^{(t+1)})$
%     \If{$\|\mat{S}'_{\text{img,current}} - \mat{S}'_{\text{img,prev}}\|_F < \epsilon$} \textbf{break} \EndIf
%     \State $\mat{S}'_{\text{img,prev}} \leftarrow \mat{S}'_{\text{img,current}}$
% \EndFor
% \State \Return $\mat{S}'_{\text{img,current}}, \vect{b}^{(t+1)}, (s^{(t+1)}, \theta^{(t+1)}, \vect{t}^{(t+1)})$
% \end{algorithmic}
% \end{algorithm}

% El resultado de este proceso de ajuste es un conjunto final de parámetros de forma $\hat{\vect{b}}_{\text{final}}$ y parámetros de pose $(s_{\text{final}}, \theta_{\text{final}}, \vect{t}_{\text{final}})$ que representan la mejor adaptación del modelo a la imagen. La forma final en el espacio de la imagen, $\mat{S}'_{\text{final,img}}$, se utiliza para la generación de la máscara de segmentación.

\section{Ajuste Iterativo del SSM (Active Shape Model - ASM)}
\label{sec:ajuste_asm_simplified}

Con una estimación inicial de los parámetros de forma $\hat{\vect{b}}$ (Sección~\ref{sec:prediccion_desdiscretizacion_b_simplified}) y una pose inicial $(s_{ESL}, \theta_{ESL}, \vect{t}_{ESL})$ de la etapa ESL (Sección~\ref{sec:esl_simplified}), se refinan estos parámetros mediante el Active Shape Model (ASM) \cite{cootes1995active}. El ASM ajusta iterativamente el modelo de forma a la evidencia en la imagen.

\begin{figure}[htbp]
    \centering
    \includegraphics[width=0.9\textwidth]{Figures/step0_initial_projection_newIdx20_cat1_id103_1.png}
    \caption{Proyección Inicial del SSM sobre la imagen usando $\hat{\vect{b}}$ y la pose de ESL.}
    \label{fig:step0_initial_projection_simplified}
\end{figure}

El proceso iterativo consta de los siguientes pasos principales en cada iteración $t$:

\begin{enumerate}
    \item \textbf{Generación de la Forma Actual en Espacio Canónico:}
    Se genera la instancia de forma $\vect{s}_{\text{SSM}}(\vect{b}^{(t)})$ usando los parámetros de forma actuales $\vect{b}^{(t)}$ y la ecuación de reconstrucción del SSM (referencia a Ecuación~\eqref{eq:ssm_reconstruction_simplified} de una sección anterior):
    \begin{equation}
    \vect{s}_{\text{SSM}}(\vect{b}^{(t)}) = \mean{\vect{s}} + \mat{P} \vect{b}^{(t)}.
    \label{eq:asm_shape_instance_vec_simplified}
    \end{equation}
    Esta forma se remodela a una matriz de landmarks $\mat{S}'_{\text{SSM}}(\vect{b}^{(t)})$.

    \item \textbf{Proyección al Espacio de la Imagen:}
    La forma canónica $\mat{S}'_{\text{SSM}}(\vect{b}^{(t)})$ se proyecta a la imagen $\mat{S}'_{\text{img}}$ aplicando la transformación de similitud actual (escala $s^{(t)}$, rotación $\mat{R}(\theta^{(t)})$, traslación $\vect{t}^{(t)}$).


    \item \textbf{Búsqueda Local de Puntos Óptimos:}
    Para cada landmark de $\mat{S}'_{\text{img}}$, se busca una nueva posición $\vect{p}^*_{i,\text{img}}$ en la imagen. Esta búsqueda se realiza a lo largo de la normal al contorno, muestreando perfiles de intensidad $\vect{g}_{\text{obs}}$. El punto que minimiza la distancia de Mahalanobis $D_M^2(\vect{g}_{\text{obs}}, (\mean{\vect{g}}_i, \matSigma_i))$ (referencia a Ecuación~\eqref{eq:mahalanobis_distance_simplified} de la sección SAM) se selecciona. El conjunto de estos puntos forma la "forma objetivo" $\mat{S}^*_{\text{target,img}}$.

    \item \textbf{Actualización de Parámetros de Pose:}
    Se calcula una nueva transformación de similitud $(s_{\text{new}}, \theta_{\text{new}}, \vect{t}_{\text{new}})$ que alinea la forma canónica actual $\mat{S}'_{\text{SSM}}(\vect{b}^{(t)})$ con la forma objetivo $\mat{S}^*_{\text{target,img}}$ (usando Procrustes). Los parámetros de pose se actualizan, a menudo con amortiguamiento $\alpha_p$:
    \begin{align}
    \theta^{(t+1)} &= \theta^{(t)} + \alpha_p (\theta_{\text{new}} - \theta^{(t)}), \\
    \vect{t}^{(t+1)} &= \vect{t}^{(t)} + \alpha_p (\vect{t}_{\text{new}} - \vect{t}^{(t)}).
    \end{align}

    \item \textbf{Actualización de Parámetros de Forma:}
    La forma objetivo $\mat{S}^*_{\text{target,img}}$ se transforma de vuelta al espacio canónico del SSM usando la inversa de la pose actualizada $(s^{(t+1)}, \theta^{(t+1)}, \vect{t}^{(t+1)})$, resultando en $\vect{s}'^*_{\text{target,SSM}}$. Esta se proyecta sobre la base del SSM para obtener nuevos parámetros $\vect{b}_{\text{new}}$:
    \begin{equation}
    \vect{b}_{\text{new}} = \transpose{\mat{P}} (\vect{s}'^*_{\text{target,SSM}} - \mean{\vect{s}}).
    \label{eq:asm_project_to_b_simplified}
    \end{equation}
    Los nuevos parámetros de forma se limitan a un rango plausible (e.g., $\pm n_{\text{std}}\sigma_k$ para cada $b_k$), donde $\sigma_k=\sqrt{\lambda_k}$:
    \begin{equation}
    b_k^{(t+1)} = \text{clip}(b_{k,\text{new}}, -n_{\text{std}}\sqrt{\lambda_k}, n_{\text{std}}\sqrt{\lambda_k}).
    \label{eq:asm_b_clamping_simplified}
    \end{equation}

    \item \textbf{Comprobación de Convergencia:}
    El proceso se detiene si se alcanza un número máximo de iteraciones o si el cambio en la forma o los parámetros es menor que una tolerancia $\epsilon$.
\end{enumerate}

\begin{figure}[htbp]
    \centering
    \includegraphics[width=0.9\textwidth]{Figures/step6_end_of_iteration_newIdx20_cat1_id103.png}
    \caption{Ejemplo de la forma SSM ajustada a la imagen después de varias iteraciones.}
    \label{fig:final_asm_fit_simplified}
\end{figure}

El resultado es un conjunto final de parámetros de forma $\hat{\vect{b}}_{\text{final}}$ y de pose $(s_{\text{final}}, \theta_{\text{final}}, \vect{t}_{\text{final}})$ que mejor adaptan el modelo a la imagen.
%% \section{Generación de Máscaras de Segmentación y Preparación para Evaluación}
% \label{sec:generacion_mascaras_evaluacion} 

% Una vez que el proceso de ajuste iterativo del SSM (Active Shape Model - ASM, Sección~\ref{sec:ajuste_asm}) ha convergido, se obtiene una representación final de la forma pulmonar mediante un conjunto de $\Ktotal$ landmarks, $\mat{S}'_{\text{final,img}} \in \R^{\Ktotal \times \dval}$, en el espacio de la imagen. El siguiente paso consiste en convertir esta representación de puntos discretos en una máscara de segmentación binaria que delimite las regiones pulmonares. Paralelamente, se requiere la generación de máscaras ground truth (GT) a partir de anotaciones de referencia para la posterior evaluación del modelo.

% \subsection{Generación de Máscara de Segmentación Predicha a partir del Ajuste ASM}
% \label{sec:mascara_predicha_asm}

% Los $\Ktotal$ puntos de la forma final ajustada por el ASM, $\mat{S}'_{\text{final,img}} = \{ \vect{p}'_{i,\text{img}} = (x'_{i,\text{img}}, y'_{i,\text{img}}) \}_{i=1}^{\Ktotal}$, se interpretan como los vértices de los contornos pulmonares. Para generar la máscara de segmentación predicha inicial, $\mat{M}_{\text{pred,initial}}$:

% \begin{enumerate}
%     \item \textbf{División de Contornos Lobulares:} Siguiendo la convención utilizada durante la densificación de landmarks (Sección~\ref{sec:densificacion_forma}, donde $\Ktotal = 2k_d$), los $\Ktotal$ puntos se dividen en dos conjuntos, representando los dos lóbulos pulmonares principales (cada uno con $k_d = \Ktotal/2$ puntos):
%     \begin{itemize}
%         \item Contorno del primer lóbulo: $\mathcal{L}_{1,\text{pred}} = \{ \vect{p}'_{i,\text{img}} \}_{i=1}^{\Ktotal/2}$
%         \item Contorno del segundo lóbulo: $\mathcal{L}_{2,\text{pred}} = \{ \vect{p}'_{i,\text{img}} \}_{i=(\Ktotal/2)+1}^{\Ktotal}$
%     \end{itemize}
%     Cada conjunto $\mathcal{L}_{j,\text{pred}}$ forma una secuencia ordenada de vértices que define un polígono $\mathcal{P}_{j,\text{pred}}$.

%     \item \textbf{Rasterización de Polígonos:} Se inicializa una máscara binaria $\mat{M}_{\text{pred,initial}}$ con las mismas dimensiones que la imagen de entrada ($H \times W$), con todos los píxeles establecidos a 0 (fondo). Luego, para cada polígono $\mathcal{P}_{j,\text{pred}}$ (para $j \in \{1,2\}$):
%     \begin{itemize}
%         \item Los vértices del polígono se convierten a un formato de enteros adecuado para las rutinas de dibujo.
%         \item (Opcional) Para mejorar la robustez ante geometrías poligonales complejas (e.g., auto-intersecciones), se pueden aplicar algoritmos de validación y corrección geométrica antes del relleno.
%         \item Se utiliza un algoritmo de rasterización o relleno de polígonos para asignar un valor de primer plano (e.g., 1 o 255) a todos los píxeles $(u,v)$ que se encuentran dentro de los límites de $\mathcal{P}_{j,\text{pred}}$:
%         \begin{equation}
%         \mat{M}_{\text{pred,initial}}(u,v) = 1 \quad \text{si } (u,v) \in \text{interior}(\mathcal{P}_{1,\text{pred}}) \lor (u,v) \in \text{interior}(\mathcal{P}_{2,\text{pred}}),
%         \label{eq:rasterizacion_pred}
%         \end{equation}
%         y $\mat{M}_{\text{pred,initial}}(u,v) = 0$ en caso contrario.
%     \end{itemize}
% \end{enumerate}

% \begin{figure}[htbp]
%   \centering
%   \includegraphics[width=0.7\textwidth]{Figures/fig_asm_final_to_mask_idx20.png}
%   \caption{Generación de la máscara de segmentación predicha $\mat{M}_{\text{pred,initial}}$ a partir de los $\Ktotal$ landmarks finales $\mat{S}'_{\text{final,img}}$ del ajuste ASM. Izquierda: Puntos $\mat{S}'_{\text{final,img}}$ superpuestos en la imagen. Derecha: Máscara binaria $\mat{M}_{\text{pred,initial}}$ resultante.}
%   \label{fig:asm_final_to_mask}
% \end{figure}

% \subsection{Generación de Máscaras Ground Truth}
% \label{sec:generacion_mascaras_gt}

% Para entrenar y evaluar el modelo de segmentación, se requieren máscaras ground truth (GT) $\mat{M}_{GT}$ que representen la segmentación ideal. Estas se generan a partir de un conjunto de $\Ktotal$ landmarks, denotados como la matriz $\mat{P}_{\Ktotal,s,GT} \in \R^{\Ktotal \times \dval}$, anotados manualmente en un espacio fuente $\mathcal{S}_s$ (dimensiones $W_s \times H_s$).

% \begin{enumerate}
%     \item \textbf{Transformación de Coordenadas GT al Espacio Objetivo:} Los landmarks $\mat{P}_{\Ktotal,s,GT}$ se transforman al espacio objetivo $\mathcal{S}_t$ (dimensiones $W_t \times H_t$ de la imagen de trabajo) mediante reescalado y ajuste de límites. Para cada landmark $i$, $(x_{s,i,GT}, y_{s,i,GT})$ de $\mat{P}_{\Ktotal,s,GT}$:
%     \begin{align}
%     x_{t,i,GT} &= (x_{s,i,GT} \cdot \frac{W_t}{W_s}) \label{eq:gt_scale_x} \\
%     y_{t,i,GT} &= (y_{s,i,GT} \cdot \frac{H_t}{H_s}) \label{eq:gt_scale_y} \\
%     x'_{t,i,GT} &= \text{clip}(x_{t,i,GT}, 0, W_t - 1) \label{eq:gt_clip_x} \\
%     y'_{t,i,GT} &= \text{clip}(y_{t,i,GT}, 0, H_t - 1) \label{eq:gt_clip_y}
%     \end{align}
%     Esto produce el conjunto de landmarks GT ajustados, la matriz $\mat{P}'_{\Ktotal,t,GT} \in \R^{\Ktotal \times \dval}$.

%     \item \textbf{Formación de Contornos Lobulares GT y Rasterización:} De manera análoga a la Sección~\ref{sec:mascara_predicha_asm} (Paso 1 y 2), los puntos de $\mat{P}'_{\Ktotal,t,GT}$ se dividen en dos contornos lobulares $\mathcal{L}_{1,GT}$ y $\mathcal{L}_{2,GT}$, que definen polígonos $\mathcal{P}_{1,GT}$ y $\mathcal{P}_{2,GT}$. Estos polígonos se rasterizan para formar la máscara $\mat{M}_{GT}$:
%     \begin{equation}
%     \mat{M}_{GT}(u,v) = 1 \quad \text{si } (u,v) \in \text{interior}(\mathcal{P}_{1,GT}) \lor (u,v) \in \text{interior}(\mathcal{P}_{2,GT}),
%     \label{eq:rasterizacion_gt}
%     \end{equation}
%     y $\mat{M}_{GT}(u,v) = 0$ en caso contrario.
% \end{enumerate}

% El Algoritmo~\ref{alg:generate_gt_mask_formal} resume este proceso.

% \begin{algorithm}[htbp]
% \caption{Generación de Máscara Ground Truth ($\mat{M}_{GT}$)}
% \label{alg:generate_gt_mask_formal}
% \begin{algorithmic}[1]
% \Require Matriz de $\Ktotal$ coordenadas de landmarks GT, $\mat{P}_{\Ktotal,s,GT} = \{ (x_{s,i,GT}, y_{s,i,GT}) \}_{i=1}^{\Ktotal}$, en el espacio fuente $\mathcal{S}_s$.
% \Require Dimensiones del espacio fuente $(W_s, H_s)$.
% \Require Dimensiones del espacio objetivo $(W_t, H_t)$.
% \Ensure Máscara binaria ground truth $\mat{M}_{GT}$ de dimensiones $H_t \times W_t$.

% \Statex \textbf{Procedimiento:}
% \State \textbf{Transformación de Coordenadas al Espacio Objetivo:}
% \State Inicializar $\mat{P}'_{\Ktotal,t,GT}$ (matriz $\Ktotal \times \dval$)
% \ForAll{landmark $i=1, \dots, \Ktotal$}
%     \State $(x_{s,i,GT}, y_{s,i,GT}) \leftarrow i\text{-ésima fila de } \mat{P}_{\Ktotal,s,GT}$
%     \State $x_{t,i,GT} \leftarrow (x_{s,i,GT} \cdot \frac{W_t}{W_s})$ \Comment{Ver Ecuación~\eqref{eq:gt_scale_x}}
%     \State $y_{t,i,GT} \leftarrow (y_{s,i,GT} \cdot \frac{H_t}{H_s})$ \Comment{Ver Ecuación~\eqref{eq:gt_scale_y}}
%     \State $x'_{t,i,GT} \leftarrow \text{clip}(x_{t,i,GT}, 0, W_t - 1)$ \Comment{Ver Ecuación~\eqref{eq:gt_clip_x}}
%     \State $y'_{t,i,GT} \leftarrow \text{clip}(y_{t,i,GT}, 0, H_t - 1)$ \Comment{Ver Ecuación~\eqref{eq:gt_clip_y}}
%     \State Asignar $(x'_{t,i,GT}, y'_{t,i,GT})$ a la $i$-ésima fila de $\mat{P}'_{\Ktotal,t,GT}$.
% \EndFor

% \State \textbf{Definición de Contornos Lobulares:}
% \State $\mathcal{L}_{1,GT} \leftarrow \{ i\text{-ésima fila de } \mat{P}'_{\Ktotal,t,GT} \}_{i=1}^{\Ktotal/2}$
% \State $\mathcal{L}_{2,GT} \leftarrow \{ i\text{-ésima fila de } \mat{P}'_{\Ktotal,t,GT} \}_{i=(\Ktotal/2)+1}^{\Ktotal}$
% \State Estos contornos definen los polígonos $\mathcal{P}_{1,GT}$ y $\mathcal{P}_{2,GT}$.

% \State \textbf{Inicialización de la Máscara:}
% \State $\mat{M}_{GT} \leftarrow$ matriz de dimensiones $H_t \times W_t$ con todos los elementos igual a 0.

% \State \textbf{Rasterización de Polígonos:}
% \ForAll{$j \in \{1,2\}$}
%     \State Determinar el conjunto de píxeles $I(\mathcal{P}_{j,GT})$ que constituyen el interior del polígono $\mathcal{P}_{j,GT}$.
%     \ForAll{píxel $(u,v) \in I(\mathcal{P}_{j,GT})$}
%         \State $\mat{M}_{GT}(u,v) \leftarrow 1$.
%     \EndFor
% \EndFor
% \State \Return $\mat{M}_{GT}$
% \end{algorithmic}
% \end{algorithm}

% \begin{figure}[htbp]
%   \centering
%   \includegraphics[width=0.7\textwidth]{Figures/fig_gt_mask_generation_idx20.png}
%   \caption{Proceso de generación de una máscara Ground Truth $\mat{M}_{GT}$. Izquierda: Landmarks GT $\mat{P}'_{\Ktotal,t,GT}$ superpuestos en la imagen (después de la transformación al espacio objetivo). Derecha: Máscara binaria $\mat{M}_{GT}$ resultante.}
%   \label{fig:gt_mask_generation}
% \end{figure}

% Con la máscara predicha inicial $\mat{M}_{\text{pred,initial}}$ y la máscara ground truth $\mat{M}_{GT}$ disponibles, se puede proceder a la evaluación cuantitativa del rendimiento de la segmentación, como se detallará en la Sección~\ref{sec:evaluacion}.

\section{Generación de Máscaras de Segmentación}
\label{sec:generacion_mascaras_simplified} 

Después de ajustar el modelo de forma a una imagen (proceso descrito en la Sección~\ref{sec:ajuste_asm_simplified}), se obtiene un conjunto final de $\Ktotal$ puntos, llamados \textit{landmarks}, que representan el contorno de los pulmones en la imagen. Estos landmarks se almacenan en una matriz $\mat{S}'_{\text{final,img}}$.

El objetivo ahora es crear dos tipos de imágenes binarias (máscaras):
\begin{enumerate}
    \item Una máscara predicha por nuestro modelo.
    \item Una máscara 'perfecta' o ground truth (GT), basada en las anotaciones manuales, para poder evaluar qué tan bien lo hizo el modelo.
\end{enumerate}

\subsection{Creación de la Máscara Predicha por el Modelo}
\label{sec:mascara_predicha_simplified}

\begin{figure}[htbp]
  \centering
  \includegraphics[width=0.7\textwidth]{Figures/fig_asm_final_to_mask_idx20.png}
  \caption{De los puntos (landmarks) predichos por el modelo (izquierda) a una máscara de segmentación (derecha).}
  \label{fig:asm_final_to_mask_simplified_v2}
\end{figure}

Los $\Ktotal$ landmarks de la forma final $\mat{S}'_{\text{final,img}}$ son los vértices de los contornos de los pulmones predichos. Para crear la máscara $\mat{M}_{\text{predicha}}$:
\begin{enumerate}
    \item \textbf{Definir los Polígonos Pulmonares:} Los $\Ktotal$ landmarks se dividen en dos grupos, uno para cada pulmón (o lóbulo principal). Cada grupo de puntos forma un polígono (una figura cerrada). Llamaremos a estos polígonos $\set{P}_{\text{pulmón1, predicho}}$ y $\set{P}_{\text{pulmón2, predicho}}$.

    \item \textbf{Rellenar los Polígonos (Rasterización):}
    Creamos una imagen negra (todos los píxeles con valor 0) del mismo tamaño que la radiografía original. Luego, 'rellenamos' de blanco (píxeles con valor 1) todas las áreas dentro de los polígonos $\set{P}_{\text{pulmón1, predicho}}$ y $\set{P}_{\text{pulmón2, predicho}}$.
    Matemáticamente, para cada píxel $(u,v)$ en la máscara:
    \begin{equation}
    \mat{M}_{\text{predicha}}(u,v) = 1 \quad
    \label{eq:rasterizacion_pred_simplified_v2}
    \end{equation}
    \text{si el píxel } (u,v) \text{ está en alguno de los polígonos predichos.}
    \\
    Si no, $\mat{M}_{\text{predicha}}(u,v) = 0$.
\end{enumerate}

\subsection{Creación de la Máscara Ground Truth}
\label{sec:mascara_gt_simplified}

Para saber si nuestra predicción es buena, necesitamos una máscara perfecta, llamada ground truth ($\mat{M}_{GT}$). Esta se crea a partir de landmarks anotados. Supongamos que estos landmarks GT, $\mat{P}_{GT, \text{original}}$, fueron anotados en imágenes de un tamaño diferente al que estamos usando.

\begin{enumerate}
    \item \textbf{Ajustar Coordenadas de Landmarks GT:}
    Si los landmarks GT originales están en una imagen de tamaño $W_{\text{original}} \times H_{\text{original}}$, y nuestra imagen de trabajo tiene tamaño $W_{\text{trabajo}} \times H_{\text{trabajo}}$, necesitamos reescalar las coordenadas $(x_{\text{original}}, y_{\text{original}})$ de cada landmark GT para que coincidan con el tamaño de nuestra imagen de trabajo:
    \begin{align}
    x_{\text{trabajo}} &= x_{\text{original}} \cdot \frac{W_{\text{trabajo}}}{W_{\text{original}}} \\
    y_{\text{trabajo}} &= y_{\text{original}} \cdot \frac{H_{\text{trabajo}}}{H_{\text{original}}}
    \end{align}
    Estos nuevos puntos reescalados forman $\mat{P}'_{GT, \text{trabajo}}$.

    \item \textbf{Definir y Rellenar Polígonos GT:}
    De forma similar a la máscara predicha, los puntos de $\mat{P}'_{GT, \text{trabajo}}$ se usan para definir los polígonos de los pulmones ground truth, $\set{P}_{\text{pulmón1, GT}}$ y $\set{P}_{\text{pulmón2, GT}}$.
    Luego, estos polígonos se rellenan para crear la máscara $\mat{M}_{GT}$:
    \begin{equation}
    \mat{M}_{GT}(u,v) = 1 \quad 
    \label{eq:rasterizacion_gt_simplified_v2}
    \end{equation}
    \text{si el píxel } (u,v) \text{ está dentro de alguno de los polígonos GT.}
\end{enumerate}

\begin{figure}[htbp]
  \centering
  \includegraphics[width=0.7\textwidth]{Figures/fig_gt_mask_generation_idx20.png}
  \caption{De los puntos (landmarks) ground truth (izquierda) a una máscara de segmentación ground truth (derecha).}
  \label{fig:gt_mask_generation_simplified_v2}
\end{figure}

Ahora tenemos una máscara $\mat{M}_{\text{predicha}}$ (lo que el modelo dice) y una máscara $\mat{M}_{GT}$ (lo que debería ser), listas para ser comparadas en la evaluación.
%% % Asegúrate de que estos comandos están definidos en tu preámbulo:
% % \newcommand{\mat}[1]{\mathbf{#1}}   
% % \newcommand{\vect}[1]{\bm{#1}} % Requiere \usepackage{bm}
% % \newcommand{\transpose}{\mathsf{T}}
% % \newcommand{\R}{\mathbb{R}} 
% % \newcommand{\set}[1]{\mathcal{#1}} % Ejemplo de macro para conjuntos, si se usa

% \section{Metodología de Evaluación Cuantitativa}
% \label{sec:evaluacion}

% El rendimiento del proceso de segmentación se cuantifica mediante la comparación de la máscara predicha inicial, $\mat{M}_{\text{pred,initial}}$, con su correspondiente máscara ground truth, $\mat{M}_{GT}$, para cada imagen $\mat{I}$ en el conjunto de datos de prueba. El dominio espacial de estas máscaras es $\Omega$, que corresponde a las dimensiones de la imagen.

% \subsection{Métrica de Superposición: Coeficiente de Dice}
% La métrica fundamental empleada para esta evaluación es el Coeficiente de Dice (DSC). Este coeficiente mide el grado de superposición entre las dos máscaras binarias y, para una imagen dada, se define formalmente como:
% \begin{equation}
% \text{DSC}(\mat{M}_{\text{pred,initial}}, \mat{M}_{GT}) = \frac{2 \cdot |\mat{M}_{\text{pred,initial}} \cap \mat{M}_{GT}|}{|\mat{M}_{\text{pred,initial}}| + |\mat{M}_{GT}|},
% \label{eq:dice_coefficient_eval_math}
% \end{equation}
% donde $|\cdot|$ denota la cardinalidad del conjunto de píxeles (es decir, el área de la región en píxeles) y $\cap$ representa la operación de intersección entre los conjuntos de píxeles que constituyen las máscaras. Un valor de $\text{DSC}=1$ significa una concordancia perfecta entre la predicción y la referencia, mientras que $\text{DSC}=0$ indica una ausencia total de superposición.

% \subsection{Proceso de Evaluación para el Conjunto de Prueba}
% El procedimiento de evaluación se aplica sistemáticamente a cada una de las $N_{\text{test}}$ imágenes que componen el conjunto de prueba:
% \begin{enumerate}
%     \item Para la $j$-ésima imagen de prueba, $\mat{I}_j$, se dispone de:
%     \begin{itemize}
%         \item La máscara predicha $\mat{M}_{\text{pred,initial}}^{(j)}$, obtenida a partir del ajuste del modelo de forma, como se detalla en la Sección~\ref{sec:mascara_predicha_asm}.
%         \item La máscara ground truth correspondiente $\mat{M}_{GT}^{(j)}$, generada a partir de las anotaciones de referencia, según se describe en la Sección~\ref{sec:generacion_mascaras_gt}.
%     \end{itemize}
%     \item Se calcula el Coeficiente de Dice $\text{DSC}_j$ para esta $j$-ésima imagen mediante la aplicación de la Ecuación~\eqref{eq:dice_coefficient_eval_math}:
%     \begin{equation}
%     \text{DSC}_j = \text{DSC}(\mat{M}_{\text{pred,initial}}^{(j)}, \mat{M}_{GT}^{(j)}).
%     \label{eq:dsc_j_instance} % Etiqueta para esta instancia específica
%     \end{equation}
%     \item Este proceso se repite para todas las imágenes del conjunto de prueba, lo que resulta en un conjunto de $N_{\text{test}}$ coeficientes de Dice, denotado como $\mathcal{D}_{\text{scores}} = \{\text{DSC}_j\}_{j=1}^{N_{\text{test}}}$.
% \end{enumerate}

% \subsection{Análisis Estadístico y Visualización de Resultados}
% El rendimiento global del proceso de segmentación se caracteriza mediante el análisis estadístico del conjunto de scores $\mathcal{D}_{\text{scores}}$. Se calculan las siguientes estadísticas descriptivas para resumir la distribución de estos coeficientes:
% \begin{itemize}
%     \item Media Aritmética: $\overline{\text{DSC}} = \frac{1}{N_{\text{test}}} \sum_{j=1}^{N_{\text{test}}} \text{DSC}_j$.
%     \item Desviación Estándar Muestral: $s_{\text{DSC}} = \sqrt{\frac{1}{N_{\text{test}}-1} \sum_{j=1}^{N_{\text{test}}} (\text{DSC}_j - \overline{\text{DSC}})^2}$ (usando $s$ para desviación estándar muestral para distinguirla de $\sigma$ poblacional o de modos de SSM).
%     \item Mediana, Valor Mínimo y Valor Máximo observados en el conjunto $\mathcal{D}_{\text{scores}}$.
% \end{itemize}
% Estos agregados estadísticos proporcionan una medida sumaria del rendimiento promedio del modelo y de su consistencia a través del conjunto de prueba.

% De forma complementaria al análisis numérico, se realiza una inspección cualitativa mediante la generación de imágenes de superposición para cada muestra de prueba. Como se ilustra en la Figura~\ref{fig:evaluation_overlay_example_math}, estas visualizaciones presentan la imagen original $\mat{I}_j$ junto con los contornos delineados por la máscara ground truth $\mat{M}_{GT}^{(j)}$ y la máscara predicha $\mat{M}_{\text{pred,initial}}^{(j)}$. El valor del Coeficiente de Dice $\text{DSC}_j$, calculado para esa instancia específica, acompaña a esta representación visual. Este enfoque permite la identificación de patrones en los casos de alta concordancia (e.g., $\text{DSC} \approx 0.95$) y en aquellos donde la segmentación presenta mayores discrepancias, ofreciendo así una perspectiva que enriquece la interpretación de las métricas cuantitativas.

% \begin{figure}[htbp]
%     \centering
%     \includegraphics[width=0.6\textwidth]{Figures/overlay_idx57_dice0.955_TwoCt_PP_TwoCt_PP.png} 
%     \caption{Visualización de la evaluación para una muestra del conjunto de prueba. La imagen original $\mat{I}_j$ se muestra con el contorno de la máscara ground truth $\mat{M}_{GT}^{(j)}$ (representado en verde) y el contorno de la máscara predicha $\mat{M}_{\text{pred,initial}}^{(j)}$ (representado en rojo). El Coeficiente de Dice (DSC) calculado para esta instancia es $\text{DSC}_j = 0.95$.}
%     \label{fig:evaluation_overlay_example_math}
% \end{figure}

% Esta metodología de evaluación, centrada en la formulación matemática del Coeficiente de Dice y complementada con un análisis estadístico riguroso y visualizaciones detalladas, permite una valoración robusta y transparente del rendimiento de nuestro pipeline de segmentación pulmonar.

\section{Evaluación Cuantitativa}
\label{sec:evaluacion_simplified}

Para medir qué tan bien funciona nuestro método de segmentación, comparamos la máscara que nuestro modelo predice ($\mat{M}_{\text{predicha}}$) con la máscara perfecta o ground truth ($\mat{M}_{GT}$), para cada imagen del conjunto de prueba. Estas máscaras fueron generadas como se describió en la Sección~\ref{sec:generacion_mascaras_simplified}.

\subsection{Métrica de Superposición: Coeficiente de Dice (DSC)}
La principal métrica que usamos es el Coeficiente de Dice (DSC). El DSC mide qué tanto se parecen o se superponen dos máscaras. Se calcula así:
\begin{equation}
\text{DSC} = \frac{2 \times \text{Área de Superposición entre } \mat{M}_{\text{predicha}} \text{ y } \mat{M}_{GT}}{\text{Área Total de } \mat{M}_{\text{predicha}} + \text{Área Total de } \mat{M}_{GT}}.
\label{eq:dice_coefficient_simplified}
\end{equation}
En esta fórmula:
\begin{itemize}
    \item "Área de Superposición" es el número de píxeles que son blancos (parte del pulmón) en \textit{ambas} máscaras, $\mat{M}_{\text{predicha}}$ y $\mat{M}_{GT}$.
    \item "Área Total" es el número de píxeles que son blancos en cada máscara individualmente.
\end{itemize}
Un DSC de 1 significa que la predicción y el ground truth son idénticos (perfecto). Un DSC de 0 significa que no se superponen en absoluto.

\subsection{Proceso de Evaluación}
Para cada imagen en nuestro conjunto de prueba (supongamos que hay $N_{\text{test}}$ imágenes):
\begin{enumerate}
    \item Tomamos la máscara predicha por nuestro modelo para esa imagen, $\mat{M}_{\text{predicha}}^{(j)}$.
    \item Tomamos la máscara ground truth correspondiente, $\mat{M}_{GT}^{(j)}$.
    \item Calculamos el DSC para esa imagen, $\text{DSC}_j$, usando la fórmula anterior.
\end{enumerate}
Al final, tendremos una lista con $N_{\text{test}}$ valores de DSC, uno por cada imagen de prueba.

\subsection{Análisis de Resultados}
Para entender el rendimiento general, calculamos estadísticas sobre la lista de valores de DSC:
\begin{itemize}
    \item \textbf{Media (Promedio):} El valor promedio de DSC de todas las imágenes.
    \item \textbf{Desviación Estándar:} Cuánto varían los valores de DSC alrededor de la media.
    \item \textbf{Mediana, Mínimo y Máximo:} Para ver el rango y el valor central de los DSC obtenidos.
\end{itemize}
Estos números nos dan una idea general de qué tan bien y qué tan consistentemente funciona nuestro método.

Además, es útil ver visualmente cómo se comparan las máscaras. Para cada imagen de prueba, podemos superponer los contornos de la máscara predicha y la máscara ground truth sobre la radiografía original. Esto nos ayuda a entender por qué el DSC es alto o bajo en casos específicos.

\begin{figure}[htbp]
    \centering
    \includegraphics[width=0.6\textwidth]{Figures/overlay_idx57_dice0.955_TwoCt_PP_TwoCt_PP.png} 
    \caption{Ejemplo de visualización para evaluar la segmentación. Se muestran los contornos del ground truth (ej. en verde) y de la predicción (ej. en rojo) sobre la imagen, junto con el valor de DSC para esa imagen $\text{DSC}_j = 0.9$.}
    \label{fig:evaluation_overlay_example_simplified}
\end{figure}

Esta forma de evaluar, usando el DSC y visualizaciones, nos permite valorar de manera robusta el rendimiento de la segmentación.

%Capitulo 4
%\chapter{Diseño Experimental y Configuración}
\label{cap:diseno_experimental}

Para evaluar el rendimiento del sistema propuesto y analizar la influencia de diversos factores en la precisión de la predicción de puntos anatómicos, se diseñó y ejecutó una serie de experimentos. En esta sección se describe la composición del conjunto de datos, los parámetros de los modelos y la metodología experimental empleada.

\section{Dataset y Preprocesamiento}
Se empleó un dataset de radiografías de tórax constituido por un total de \textbf{1000 imágenes}. Cada imagen fue anotada manualmente con 15 puntos de referencia anatómicos por un único observador. Las imágenes, con una resolución original de $299 \times 299$ píxeles, se redimensionaron a $64 \times 64$ píxeles y se convirtieron a escala de grises para su procesamiento. El dataset presenta una distribución equitativa entre casos de pacientes sanos y pacientes con neumonía.

Para la evaluación experimental, el dataset principal se dividió en conjuntos de entrenamiento y prueba, utilizando una proporción del \textbf{80\% para entrenamiento (800 imágenes)} y el \textbf{20\% restante para prueba (200 imágenes)}. Esta partición se realizó de forma reproducible (mediante una semilla aleatoria fija). El conjunto de entrenamiento se destinó al alineamiento de formas, la extracción de regiones de búsqueda y plantillas (templates), y el entrenamiento de los modelos de apariencia. El conjunto de prueba, compuesto por imágenes no utilizadas durante la fase de entrenamiento, se reservó exclusivamente para la evaluación de la precisión predictiva.

\section{Configuración de Modelos y Parámetros}
Para cada punto anatómico $j$, se entrenó un modelo de Análisis de Componentes Principales (PCA). El número de componentes principales $m_j$ se determinó para cada punto de manera que se retuviera el \textbf{95\% de la varianza total explicada} a partir de los parches de apariencia del conjunto de entrenamiento.

Las dimensiones de los parches de apariencia, extraídos para cada punto anatómico, no son fijas; estas varían y se derivan de la caja delimitadora (bounding box) de la región de búsqueda asociada a cada punto. Por ejemplo, para el punto `Coord1`, el parche resultante posee dimensiones de $39 \times 34$ píxeles.

Como métrica del error de reconstrucción para seleccionar el punto óptimo durante la predicción, se utilizó la \textbf{norma L2} (distancia euclidiana). Esta norma cuantifica la diferencia entre el parche original y su reconstrucción obtenida a partir del subespacio PCA.

\section{Métricas de Evaluación}
La precisión de la predicción para cada punto anatómico $j$ en el conjunto de prueba se cuantificó mediante métricas basadas en la distancia entre la coordenada predicha $\hat{\mathbf{p}}_j = (\hat{x}_j, \hat{y}_j)$ y la coordenada de referencia (ground truth) anotada manualmente $\mathbf{p}_{gt,j} = (x_{gt,j}, y_{gt,j})$. Las principales métricas empleadas incluyen:

\begin{itemize}
    \item \textbf{Error Euclidiano por Instancia:} Distancia euclidiana entre la predicción y la referencia para cada instancia de prueba $i$ y punto anatómico $j$:
    $$ E_{Euclidiano, ij} = \left\| \hat{\mathbf{p}}_{ij} - \mathbf{p}_{gt,ij} \right\|_2 = \sqrt{(\hat{x}_{ij} - x_{gt,ij})^2 + (\hat{y}_{ij} - y_{gt,ij})^2} $$
    donde $i$ indexa las imágenes de prueba y $j$ los puntos anatómicos.
    \item \textbf{Error Euclidiano Promedio por Punto:} Promedio del error euclidiano para un punto $j$, calculado sobre todas las $N_{test}$ imágenes del conjunto de prueba:
    $$ \mean{E}_{Euclidiano, j} = \frac{1}{N_{test}} \sum_{i=1}^{N_{test}} E_{Euclidiano, ij} $$
    \item \textbf{Mediana del Error Euclidiano por Punto:} Mediana de los errores euclidianos para un punto $j$ en el conjunto de prueba. Esta medida de tendencia central es menos sensible a valores atípicos.
    \item \textbf{Desviación Estándar del Error Euclidiano por Punto:} Desviación estándar de los errores euclidianos para un punto $j$ en el conjunto de prueba, la cual indica la dispersión de dichos errores.
\end{itemize}

\subsection{Procedimiento Experimental y Pruebas Preliminares}
El procedimiento experimental general siguió la secuencia de etapas descrita en la Sección \ref{cap:metodologia}. No obstante, con el fin de comprender el impacto de la variabilidad inherente a las imágenes y la efectividad de ciertas técnicas de preprocesamiento, se realizaron varias pruebas preliminares:

\begin{enumerate}
    \item \textbf{Prueba 1 (Dataset de 200 imágenes):} Se empleó un dataset inicial de 200 imágenes para evaluar el rendimiento base del algoritmo en ausencia de alineamiento.
    \item \textbf{Prueba 2 (Dataset de 400 imágenes con variaciones):} Se incorporaron 200 imágenes adicionales al dataset, incluyendo ejemplos con traslaciones, rotaciones y cambios de escala atípicos. El objetivo era determinar la respuesta del algoritmo ante estas variaciones geométricas.
    \item \textbf{Prueba 3 (Normalización de Contraste SAHS):} Sobre el dataset de 400 imágenes, se aplicó una técnica de normalización de contraste basada en SAHS previamente a la extracción de parches y al entrenamiento del modelo. Dicha técnica había evidenciado previamente resultados favorables en combinación con Redes Neuronales Convolucionales (CNNs).
    \item \textbf{Prueba 4 (Dataset de 800 imágenes):} Se expandió el dataset a 800 imágenes (manteniendo la omisión del alineamiento) para analizar el comportamiento del algoritmo base frente a un volumen mayor de datos, el cual incluía las imágenes con las variaciones introducidas.
\end{enumerate}
Estos experimentos preliminares resultaron fundamentales para identificar los desafíos principales del sistema y orientaron el desarrollo de soluciones más robustas, como la posterior implementación del alineamiento de formas.

\section{Configuración Experimental SAHS}
Se utilizó un conjunto de datos de 2,700 imágenes de radiografías de tórax, con 1,350 imágenes de pacientes saludables y 1,350 de pacientes con neumonía, las cuales fueron usadas para entrenar y probar los modelos de CNN. Estas imágenes se encuentran separadas de origen al momento de la descarga, lo cual implica que ambas clases podrían tener condiciones diferentes inherentes a su fuente original. Por lo anterior resulta obligado un ajuste de contraste para garantizar que las imágenes tanto normales como de neumonía se encuentren en las mismas condiciones de brillo y contraste. Estas imágenes fueron tomadas de la base de datos COVID-19\_Radiography\_Dataset de Kaggle. Utilizamos seis arquitecturas de CNN diferentes: AlexNet, Compact, Enhanced, ResNet-18, MobileNetV2 y ResNet-50, todas disponibles en la plataforma MVTEC Deep Learning Tool. Cada modelo fue entrenado y evaluado. El conjunto de datos consistió en radiografías de tórax etiquetadas como "normal" y "neumonía". Se utilizó validación cruzada de 5 pliegues para evaluar el rendimiento de los modelos.


\section{Diseño Experimental Segmentación Automática}
El diseño experimental se centró en la evaluación de un modelo híbrido para la segmentación de estructuras pulmonares. Se realizaron múltiples ejecuciones, cada una evaluando el modelo bajo un conjunto de parámetros de configuración específicos. La configuración de los hiperparámetros y la arquitectura del modelo se establecieron de la siguiente manera:

\begin{table}[h!]
    \centering
    \caption{Parámetros de Configuración del Modelo}
    \label{tab:config_params}
    \begin{tabular}{ll}
        \toprule
        \textbf{Parámetro} & \textbf{Valor} \\
        \midrule
        Tasa de Aprendizaje Inicial ($\alpha$) & 0.0005 \\
        Factor de Reducción de Tasa de Aprendizaje & 0.2 \\
        Paciencia para Reducción de Tasa de Aprendizaje & 7 épocas \\
        Decaimiento de Peso (AdamW) & 0.0001 \\
        Tamaño del Lote ($N_b$) & 32 \\
        Épocas Máximas ($E_{max}$) & 100 \\
        Paciencia para Detención Temprana ($P_{es}$) & 15 épocas \\
        Filtros Convolucionales ($F_c$) & [32, 64, 128] \\
        Tamaños de Kernel Convolucionales ($K_c$) & [(3, 3), (3, 3), (3, 3)] \\
        Tamaños de Pooling Convolucionales ($P_c$) & [(2, 2), (2, 2), (2, 2)] \\
        Uso de Normalización por Lotes (CNN) & Sí \\
        Unidades Ocultas Densely Connected ($U_d$) & [128, 64] \\
        Tasa de Dropout (DNN) ($\delta_d$) & 0.3 \\
        Uso de Normalización por Lotes (DNN) & Sí \\
        Dimensión de Características por Parche ($D_p$) & 64 \\
        Tamaño del Parche ($Q_p$) & 25 \\
        Número de Modos de Entrenamiento ($M$) & 10 \\
        Semilla Aleatoria & 42 \\
        División de Validación & 0.2 \\
        \bottomrule
    \end{tabular}
\end{table}

El modelo fue entrenado y evaluado a través de 10 modos distintos, cada uno representando una configuración o partición de datos específica, con el objetivo de evaluar la robustez y generalización del modelo.
%\chapter{Diseño Experimental y Configuración}
\label{cap:diseno_experimental}
Este capítulo detalla el diseño experimental seguido para desarrollar, optimizar y evaluar la metodología MaShDL-CNN Hybrid propuesta en esta tesis. Se describen los conjuntos de datos utilizados, la configuración del entorno computacional, los experimentos específicos realizados para la optimización del modelo de alineación y normalización de la forma pulmonar, y el protocolo experimental diseñado para la subsiguiente tarea de clasificación de neumonía y COVID-19. El objetivo es proporcionar una descripción clara y reproducible de los pasos llevados a cabo para validar la hipótesis y alcanzar los objetivos de la investigación.

\section{Conjunto de Datos Utilizado}
\label{sec:conjunto_datos}
La selección y preparación adecuada de los conjuntos de datos es un factor crítico para el entrenamiento y la validación robusta de cualquier modelo de aprendizaje automático, especialmente en el dominio de las imágenes médicas.

\subsection{Fuente y Características de las Imágenes}
\label{ssec:fuente_imagenes}
Para el desarrollo y evaluación de los modelos propuestos en esta tesis, se utilizó un conjunto de datos compuesto por radiografías de tórax (CXR) en vista posteroanterior (PA). Este conjunto de datos se construyó a partir de varias fuentes públicas ampliamente reconocidas en la comunidad de investigación, con el fin de asegurar una diversidad de casos y características de imagen. Las principales fuentes incluyen:
\begin{itemize}
    \item \textbf{COVID-19 Radiography Database (Dataset A):} Compilado por investigadores de la Universidad de Qatar, la Universidad de Dhaka, y colaboradores de Pakistán y Malasia, este conjunto contiene imágenes de COVID-19, neumonía viral y pulmones normales \cite{rahman2021exploring_covid_dataset, cohen2020covid}. Las imágenes provienen de diversas fuentes públicas y repositorios.
    \item \textbf{Chest X-Ray Images (Pneumonia) (Dataset B):} Disponible en Kaggle, este conjunto de datos fue seleccionado de cohortes de pacientes pediátricos del Guangzhou Women and Children’s Medical Center, Guangzhou, e incluye imágenes de neumonía (bacteriana y viral) y pulmones normales \cite{kermany2018labeled}.
    \item \textbf{Otros conjuntos de datos públicos (Dataset C, D, \dots):} Se exploró la inclusión de imágenes de otros repositorios como [Mencionar otros si se usaron, e.g., parte de MIMIC-CXR \cite{johnson2019mimic} o PadChest \cite{bustos2020padchest} si se aplicaron filtros específicos y se obtuvieron landmarks].
\end{itemize}
Todas las imágenes fueron convertidas a formato PNG y, para ciertas etapas del procesamiento (como la entrada a algunos modelos de CNN), reescaladas a un tamaño estándar, por ejemplo, $299 \times 299$ píxeles, utilizando interpolación bilineal. Se realizó un preprocesamiento inicial para la normalización de la intensidad de los píxeles al rango $[0,1]$.
%(Sugerencia: Tabla \ref{tab:descripcion_datasets}: Una tabla que resuma las características de cada dataset utilizado. Columnas: Nombre del Dataset, Fuente/Referencia, Número de Imágenes por Clase (COVID-19, Neumonía Viral, Neumonía Bacteriana, Normal), Resolución Original (si varía), Notas (e.g., tipo de paciente, vista predominante).)

\begin{table}
    \centering
    \caption[Descripción de los conjuntos de datos de radiografías de tórax utilizados]{Descripción de los conjuntos de datos de radiografías de tórax (CXR) empleados en esta investigación para el entrenamiento y la evaluación de los modelos.}
    \label{tab:descripcion_datasets}
    \begin{tabular}{@{}l l c c c c l@{}}
        \toprule
        \multirow{2}{*}{Dataset} & \multirow{2}{*}{Referencia} & \multicolumn{4}{c}{Número de Imágenes por Clase} & \multirow{2}{*}{Notas} \\
        \cmidrule(lr){3-6}
         & & COVID-19 & Neumonía & Normal & Total & \\
        \midrule
        Dataset A & \cite{rahman2021exploring_covid_dataset, cohen2020covid} & $N_{A,C}$ & $N_{A,P}$ & $N_{A,N}$ & $N_A$ & Vistas PA/AP, adultos. \\
        Dataset B & \cite{kermany2018labeled} & --- & $N_{B,P}$ & $N_{B,N}$ & $N_B$ & Pediátrico, vistas PA. \\
        % Dataset C & \cite{johnson2019mimic} & $N_{C,C}$ & $N_{C,P}$ & $N_{C,N}$ & $N_C$ & Filtrado específico. \\
        \midrule
        Total Compilado & & $N_{\text{Total},C}$ & $N_{\text{Total},P}$ & $N_{\text{Total},N}$ & $N_{\text{Total}}$ & \\
        \bottomrule
    \end{tabular}
    \vspace{0.2cm}
    \footnotesize{\textit{Nota: $N_{X,Y}$ representa el número de imágenes para la clase Y en el dataset X. Completar con los números reales.}}
\end{table}

\subsection{Anotación de Puntos Característicos (Landmarks)}
\label{ssec:anotacion_landmarks_experimental}
Para la construcción del Modelo Estadístico de Forma (SSM) pulmonar, se requirió un conjunto de imágenes con puntos característicos (landmarks) anotados. Se utilizó un conjunto de $N_s$ imágenes de entrenamiento donde se definieron $N_{lmk}=144$ landmarks por forma pulmonar. Estos landmarks, como se mencionó en la Sección~\ref{ssec:landmarks_ssm_metodologia}, fueron definidos y/o interpolados para delinear consistentemente el contorno de la región pulmonar (ambos pulmones, incluyendo el área cardiaca y mediastinal, según la definición implícita en el SSM de 144 puntos). Las coordenadas originales de estos landmarks se normalizaron a un espacio de referencia (e.g., $64 \times 64$ píxeles) para la construcción del SSM. El archivo \code{coordenadas/coordenadas_maestro_1.csv} contiene estos landmarks para el conjunto de datos base.

\subsection{División en Conjuntos de Entrenamiento, Validación y Prueba}
\label{ssec:division_conjuntos_experimental}
Para asegurar una evaluación objetiva y evitar el sobreajuste, el conjunto de datos compilado se dividió en tres subconjuntos mutuamente excluyentes:
\begin{itemize}
    \item \textbf{Conjunto de Entrenamiento:} Utilizado para aprender los parámetros de los modelos (e.g., pesos de las CNNs, componentes del SSM si se reconstruyera).
    \item \textbf{Conjunto de Validación:} Empleado durante el entrenamiento para ajustar hiperparámetros (e.g., tasa de aprendizaje, número de épocas mediante \textit{early stopping}, arquitectura del modelo) y para la selección del mejor modelo de una serie de entrenamientos. En el script \code{train_mashdl_cnn_hybrid.py}, la división entre entrenamiento y validación para la predicción de los $b_k$ se realiza con \code{args.validation_split} (e.g., 20\% para validación), aplicando estratificación si es posible.
    \item \textbf{Conjunto de Prueba (Test):} Utilizado únicamente para la evaluación final del rendimiento del modelo entrenado y seleccionado. Este conjunto no se utiliza de ninguna forma durante el proceso de entrenamiento o selección de modelos. El archivo \code{indices/indices_maestro_1.csv} y \code{results/test_indices.txt} definen la pertenencia de las muestras a estos conjuntos.
\end{itemize}
Se procuró que la división de los datos mantuviera una proporción similar de las diferentes clases (COVID-19, neumonía, normal) en cada subconjunto, aunque el desequilibrio inherente de clases en los datasets originales puede persistir y debe ser considerado. El script \code{data_loader.py} es fundamental para cargar y gestionar el acceso a estos datos y sus correspondencias.

\section{Configuración del Entorno Computacional}
\label{sec:configuracion_entorno}
Todos los experimentos se llevaron a cabo en un entorno computacional con las siguientes especificaciones (o similares, según el informe):
\begin{itemize}
    \item \textbf{Hardware:}
    \begin{itemize}
        \item CPU: [Especificar tipo de CPU, e.g., Intel Core i7/i9, AMD Ryzen]
        \item GPU: [Especificar modelo de GPU, e.g., NVIDIA GeForce RTX 3080/4090, Tesla V100]. El uso de GPUs es crucial para el entrenamiento eficiente de los modelos de aprendizaje profundo.
        \item Memoria RAM: [Especificar cantidad, e.g., 32 GB, 64 GB]
        \item Almacenamiento: [Especificar tipo y capacidad, e.g., SSD NVMe de 1TB]
    \end{itemize}
    \item \textbf{Software:}
    \begin{itemize}
        \item Sistema Operativo: [Especificar, e.g., Ubuntu 20.04 LTS, Windows 10/11]
        \item Lenguaje de Programación: Python (versión 3.8 o superior).
        \item Librerías Principales de Aprendizaje Profundo:
        \begin{itemize}
            \item TensorFlow (versión 2.x, e.g., 2.10 o superior) \cite{abadi2016tensorflow}.
            \item Keras (API de alto nivel de TensorFlow) \cite{chollet2015keras}.
        \end{itemize}
        \item Librerías de Procesamiento de Imágenes y Cálculo Numérico:
        \begin{itemize}
            \item OpenCV (\code{cv2}, versión 4.x) \cite{opencv_library}.
            \item NumPy (versión 1.2x) \cite{harris2020array}.
            \item Scikit-learn (versión 1.x) \cite{scikit-learn}.
            \item Pandas (versión 1.x) \cite{mckinney2010data}.
            \item Matplotlib (versión 3.x) \cite{hunter2007matplotlib} (para visualización).
            \item Scipy (para operaciones científicas, e.g., \code{orthogonal_procrustes}) \cite{virtanen2020scipy}.
            \item \code{tqdm} (para barras de progreso).
            \item Shapely (opcional, para manipulación geométrica) \cite{van2011shapely}.
        \end{itemize}
        \item Entorno de Desarrollo: Jupyter Notebooks, Spyder, VS Code, o similar.
    \end{itemize}
    \item \textbf{Gestión de Entorno y Rutas (Contenedor Docker):} El informe menciona la ejecución dentro de un contenedor Docker y la importancia de la estructura de directorios con \code{/workspace} como raíz del proyecto (\code{TESIS_ROOT_DIR}). Esto asegura la reproducibilidad del entorno y la consistencia en las rutas de acceso a datos y scripts. La estructura de directorios principal para el proyecto MaShDL-CNN Hybrid es:
    \begin{itemize}
        \item \code{Tesis/segmentacion/MaShDL_CNN_Hybrid_Lung_Segmentation/}
        \begin{itemize}
            \item \code{src/}: Código fuente Python (scripts de entrenamiento, predicción, evaluación, etc.).
            \item \code{data/}: Datos de entrada (\code{NPZ} de parches, archivos del SSM, índices, máscaras GT, etc.).
            \item \code{models/}: Modelos ESL y modelos MaShDL-CNN Hybrid entrenados (organizados por \code{RunID}).
            \item \code{results/}: Resultados de predicciones, estadísticas de evaluación, historiales.
            \item \code{logs_training/}, \code{logs_script_patches/}: Archivos de log.
            \item \code{plots/}: Gráficos generados durante el entrenamiento y la evaluación.
        \end{itemize}
    \end{itemize}
\end{itemize}

\section{Experimentos para la Optimización del Modelo de Alineación MaShDL-CNN}
\label{sec:experimentos_optimizacion_alineacion}
Se diseñó una serie de experimentos para evaluar y optimizar el rendimiento del modelo MaShDL-CNN Hybrid en la tarea de predicción de los coeficientes de forma $b_k$ y, consecuentemente, en la calidad de la segmentación pulmonar. Estos experimentos se centraron en variar hiperparámetros clave de la arquitectura de la CNN y el tamaño de los parches de entrada, basándose en las capacidades del script \code{train_mashdl_cnn_hybrid.py}. El informe de progreso (Secciones 4.2, 5 y 6) detalla estas iteraciones.

\subsection{Línea Base y Primeras Iteraciones (Parches $Q=25$)}
\label{ssec:exp_q25}
El trabajo inicial se centró en establecer una línea base y explorar mejoras utilizando parches de tamaño $Q=25 \times 25$ píxeles.
\begin{itemize}
    \item \textbf{Corrida Inicial (Benchmark):}
    \begin{itemize}
        \item $Q=25$.
        \item Arquitectura Sub-CNN: 2 capas convolucionales (e.g., filtros \code{[32, 64]}, kernels \code{(3,3)}, pools \code{(2,2)}).
        \item Dimensión de características por parche: $\text{dim\_features\_per\_patch}=64$.
        \item Arquitectura DNN: 2 capas ocultas (e.g., \code{dnn_hidden_units=[128, 64]}).
        \item Número de modos $b_k$ entrenados: Inicialmente 3 modos, luego extendido a 10 modos.
        \item Épocas: Inicialmente 10, luego aumentadas a 100-200 con \textit{Early Stopping} (\code{es_patience=30}).
        \item Resultados preliminares:
        \begin{itemize}
            \item ValAcc promedio para $b_k$ (3 modos, 100 épocas): $\approx 0.6378$.
            \item ValAcc promedio para $b_k$ (10 modos, 100 épocas): $\approx 0.5236$ (decayó para modos superiores $k>4$).
            \item Coeficiente de Dice (3 modos, $Q=25$, CNN 2 capas, 100 épocas): $\approx 0.674$ (Convex Hull), $\approx 0.652$ (Dos Contornos).
            \item Coeficiente de Dice (10 modos, $Q=25$, CNN 2 capas, 100 épocas): Similar al de 3 modos, indicando que la predicción deficiente de modos superiores no mejoraba la segmentación global.
        \end{itemize}
    \end{itemize}
    \item \textbf{Experimento 1: CNN más Profunda (3 capas convolucionales)}
    \begin{itemize}
        \item $Q=25$, 5 modos $b_k$.
        \item Argumentos clave para \code{train_mashdl_cnn_hybrid.py}:
        \begin{itemize}
            \item \code{--cnn_filters 32 64 128}
            \item \code{--cnn_kernel_sizes 3,3 3,3 3,3} (asumido)
            \item \code{--cnn_pool_sizes 2,2 2,2 2,2} (asumido)
            \item \code{--dim_features_per_patch 64}
            \item \code{--dnn_hidden_units 128 64}
        \end{itemize}
        \item Resultado: Mejora marginal en ValAcc promedio para $b_k$ ($\approx 0.6202 \rightarrow \approx 0.6260$ para $k_0-k_4$).
    \end{itemize}
    \item \textbf{Experimento 2: CNN de 3 capas + Mayor \code{dim_features_per_patch} y DNN más Grande}
    \begin{itemize}
        \item $Q=25$, 5 modos $b_k$.
        \item Argumentos clave:
        \begin{itemize}
            \item (CNN igual que Exp1)
            \item \code{--dim_features_per_patch 128}
            \item \code{--dnn_hidden_units 256 128}
        \end{itemize}
        \item Resultado: Sin mejora clara en ValAcc promedio ($\approx 0.6226$), pérdida de validación más alta.
    \end{itemize}
\end{itemize}
Estos experimentos iniciales con $Q=25$ sugirieron que simplemente aumentar la profundidad o el tamaño de la red sobre parches pequeños no producía mejoras drásticas, lo que llevó a la hipótesis de que el tamaño del parche podría ser un factor limitante más fundamental.

\subsection{Experimento Crítico Propuesto (Parches $Q=41$ y Aumento de Datos Extensivo)}
\label{ssec:exp_q41}
Basándose en la discusión de que $Q=25$ podría no capturar suficiente contexto, el informe detalla un ``Próximo Paso Crítico (Experimento 3 - En curso/Pendiente)'' centrado en parches más grandes y un aumento de datos más agresivo.
\begin{itemize}
    \item \textbf{Paso 1: Regeneración de Datos de Parches 2D (con \code{mashdl_patch_extractor.py} adaptado):}
    \begin{itemize}
        \item Modificar el script para usar \code{Q_PATCH_SIZE = 41}.
        \item Establecer \code{AUGMENTATION_PROBABILITY = 1.0} para asegurar que cada imagen de entrenamiento contribuya con una versión original y al menos una aumentada.
        \item Generar los archivos \code{NPZ} (\code{mashdl_mode{K}_Q41_aug_TRAIN_B{B}.npz}) para los primeros 5 o 7 modos $b_k$.
    \end{itemize}
    \item \textbf{Paso 2: Entrenamiento de Modelos MaShDL-CNN Hybrid con Parches $Q=41$ (con \code{train_mashdl_cnn_hybrid.py}):}
    \begin{itemize}
        \item Argumento: \code{--patch_size_q 41}.
        \item Arquitectura CNN sugerida: 3 capas convolucionales (e.g., \code{--cnn_filters 32 64 128}).
        \item Dimensión de características por parche: \code{--dim_features_per_patch 64} (o probar 128).
        \item Arquitectura DNN: e.g., \code{--dnn_hidden_units 128 64} (o ajustada según \code{dim_features_per_patch}).
        \item Ajustar \code{--batch_size} si es necesario debido a la mayor demanda de memoria con parches más grandes.
    \end{itemize}
    \item \textbf{Paso 3: Evaluación de Resultados:}
    \begin{itemize}
        \item Repetir el ciclo de predicción (\code{generate_predictions_cnn.py}), desdiscretización (\code{main_desdiscretizer.py}) y cálculo del Coeficiente de Dice (\code{evaluate_segmentation.py}) para los modelos entrenados con $Q=41$.
    \end{itemize}
\end{itemize}
Se espera que este experimento proporcione información clave sobre si el tamaño del parche es, de hecho, un factor determinante para mejorar la predicción de los $b_k$ (especialmente los modos sutiles) y, por lo tanto, la precisión de la segmentación.
%(Sugerencia: Tabla \ref{tab:config_experimentos_alineacion}: Una tabla que resuma de manera concisa la configuración de cada uno de estos experimentos de optimización de la alineación. Columnas: ID Experimento, Q, Arquitectura Sub-CNN (filtros), dim_features_per_patch, Arquitectura DNN (unidades), Aumento de Datos, Número de Modos bk Entrenados, Objetivo Principal del Experimento.)

\begin{table}
    \centering
    \caption[Configuración de los experimentos para la optimización del modelo de alineación MaShDL-CNN Hybrid]{Resumen de la configuración de los principales experimentos realizados para optimizar el modelo MaShDL-CNN Hybrid en la predicción de coeficientes de forma $b_k$.}
    \label{tab:config_experimentos_alineacion}
    \resizebox{\textwidth}{!}{% Ajustar tabla al ancho de la página
    \begin{tabular}{@{}lcccccccl@{}}
        \toprule
        ID Exp. & $Q$ & CNN Filtros & \shortstack{Dim. Feat.\\por Parche} & DNN Unidades & \shortstack{Aumento\\Datos} & \shortstack{Modos $b_k$\\Entrenados} & \shortstack{Épocas\\(aprox.)} & Objetivo Principal \\
        \midrule
        Línea Base & 25 & \code{[32,64]} & 64 & \code{[128,64]} & Base & 3 y 10 & $\sim$30-60 & Establecer rendimiento inicial. \\
        Exp. 1 & 25 & \code{[32,64,128]} & 64 & \code{[128,64]} & Base & 5 & $\sim$30-60 & Evaluar CNN más profunda. \\
        Exp. 2 & 25 & \code{[32,64,128]} & 128 & \code{[256,128]} & Base & 5 & $\sim$30-60 & Evaluar más feat./parche y DNN mayor. \\
        Exp. 3 & 41 & \code{[32,64,128]} & 64 o 128 & \code{[128,64]}+ & Extensivo & 5 o 7 & Por definir & Evaluar impacto de parches mayores. \\
        \bottomrule
    \end{tabular}%
    }
    \vspace{0.2cm}
    \footnotesize{\textit{Nota: ``Base'' en Aumento de Datos se refiere a la configuración inicial; ``Extensivo'' a \code{AUGMENTATION_PROBABILITY = 1.0}. ``+'' en DNN Unidades indica que se ajustarán según \code{dim_features_per_patch}. Las épocas son aproximadas basadas en Early Stopping.}}
\end{table}

\subsection{Parámetros de Entrenamiento Comunes y Consideraciones}
\label{ssec:params_comunes_entrenamiento}
Para todos los experimentos de entrenamiento de los modelos MaShDL-CNN Hybrid, se utilizaron los siguientes parámetros y consideraciones comunes, gestionados por \code{train_mashdl_cnn_hybrid.py}, a menos que se especifique lo contrario para un experimento particular:
\begin{itemize}
    \item Optimizador: Predominantemente AdamW con una tasa de aprendizaje inicial de $1 \times 10^{-4}$ y decaimiento de peso de $1 \times 10^{-4}$.
    \item Función de Pérdida: Entropía cruzada categórica.
    \item Callbacks: \code{ModelCheckpoint} (guardando el mejor modelo basado en \code{val_accuracy}), \code{EarlyStopping} (paciencia de 30 épocas), \code{ReduceLROnPlateau} (paciencia de 15 épocas, factor de reducción de 0.2, \code{min_lr} de $1 \times 10^{-7}$), y \code{TerminateOnNaN}.
    \item División de Datos: 20\% del conjunto de datos de entrenamiento (para un modo $k$) se utilizó para validación, con \code{random_seed=42} para reproducibilidad.
    \item Manejo de Memoria: Se utilizó \code{tf.keras.backend.clear_session()} y \code{gc.collect()} entre el entrenamiento de modelos para diferentes modos $k$ para mitigar problemas de memoria GPU.
    \item Número de Bins para $b_k$: \code{num_classes_b_bins = 3} fue un valor comúnmente utilizado, aunque el impacto de este hiperparámetro también podría explorarse.
\end{itemize}

\section{Protocolo para la Clasificación de Neumonía y COVID-19}
\label{sec:protocolo_clasificacion_enfermedad}
Una vez que se obtiene el modelo MaShDL-CNN Hybrid óptimo para la alineación y normalización de la forma pulmonar (basado en los resultados de la Sección~\ref{sec:experimentos_optimizacion_alineacion}), el siguiente gran objetivo es utilizar estas regiones pulmonares normalizadas para la tarea final de clasificación de enfermedades.

\subsection{Extracción de Características de las Regiones Pulmonares Segmentadas y Normalizadas}
\label{ssec:extraccion_features_enfermedad}
Después de que el sistema MaShDL-CNN Hybrid produce una máscara de segmentación precisa $M_{\text{pred}}$ para la región pulmonar de una imagen de entrada, esta máscara se utiliza para aislar la región pulmonar. La imagen original $I$ se enmascara con $M_{\text{pred}}$ para obtener $I_{\text{pulmonar}} = I \odot M_{\text{pred}}$ (donde $\odot$ es la multiplicación por elementos). Además, la información de forma normalizada (e.g., los coeficientes $b_k$ o la forma reconstruida en el espacio canónico) está disponible.
A partir de $I_{\text{pulmonar}}$ y/o la información de forma normalizada, se deben extraer características que sean discriminantes para las clases de enfermedad (sano, neumonía, COVID-19). Se pueden considerar varias estrategias:
\begin{enumerate}
    \item \textbf{Características de Textura Clásicas:} Calcular descriptores de textura a partir de $I_{\text{pulmonar}}$, como los de la Matriz de Co-ocurrencia de Niveles de Gris (GLCM) \cite{haralick1973textural}, Patrones Binarios Locales (LBP) \cite{ojala2002multiresolution}, o características basadas en wavelets \cite{bharati2020hybrid}.
    \item \textbf{Características de Forma Adicionales:} Aunque la forma global ya está normalizada, se podrían calcular descriptores de la forma del contorno segmentado (e.g., descriptores de Fourier, momentos de Hu \cite{karargyris2016automated, xu2014texture}) si se cree que las desviaciones residuales de la forma o la complejidad del contorno tienen valor diagnóstico.
    \item \textbf{Características de Aprendizaje Profundo (Deep Features):}
    \begin{itemize}
        \item Utilizar una CNN pre-entrenada (e.g., ResNet50, VGG16, DenseNet121 entrenada en ImageNet) como extractora de características. Se alimenta $I_{\text{pulmonar}}$ (o una versión recortada y reescalada de la misma) a la CNN pre-entrenada, y se toman las activaciones de una de sus capas profundas (e.g., antes de la capa de clasificación final) como el vector de características \cite{shin2016deep, tajbakhsh2016convolutional}.
        \item Entrenar una CNN específica desde cero o mediante \textit{fine-tuning} sobre $I_{\text{pulmonar}}$ directamente para la tarea de clasificación de enfermedad. En este caso, la propia CNN aprende las características discriminantes.
    \end{itemize}
    \item \textbf{Combinación de Características:} Explorar la concatenación de diferentes tipos de características (e.g., textura + \textit{deep features}).
\end{enumerate}
La elección del método de extracción de características para la enfermedad es un paso de diseño crucial y puede requerir experimentación.

\subsection{Clasificadores a Evaluar (Objetivo Específico 3)}
\label{ssec:clasificadores_a_evaluar}
Como se establece en el Objetivo Específico 3, se evaluará el rendimiento de los siguientes tipos de clasificadores de aprendizaje supervisado, utilizando como entrada las características extraídas en la Sección~\ref{ssec:extraccion_features_enfermedad}:
\begin{itemize}
    \item \textbf{K-Vecinos Más Cercanos (KNN):} Se experimentará con diferentes valores de $K$ y métricas de distancia (e.g., Euclidiana, Manhattan).
    \item \textbf{Perceptrón Multicapa (MLP):} Se definirán arquitecturas con diferentes números de capas ocultas, neuronas por capa, y funciones de activación (predominantemente ReLU para capas ocultas y Softmax para la salida). Se utilizará regularización como Dropout.
    \item \textbf{Red Neuronal Convolucional (CNN) para Clasificación de Enfermedad:} Si se opta por una CNN de extremo a extremo para la clasificación de enfermedad (opción 3b en \ref{ssec:extraccion_features_enfermedad}), se diseñará o adaptará una arquitectura CNN específica para esta tarea, tomando $I_{\text{pulmonar}}$ como entrada.
\end{itemize}
Para cada clasificador, se realizará una optimización de sus hiperparámetros utilizando el conjunto de validación o mediante validación cruzada dentro del conjunto de entrenamiento.

\subsection{Escenarios de Comparación (Objetivo Específico 5)}
\label{ssec:escenarios_comparacion}
Para validar la eficacia de la técnica de alineación y normalización propuesta, se contrastarán los resultados de clasificación de enfermedades obtenidos con el sistema completo MaShDL-CNN Hybrid con los resultados obtenidos en los siguientes escenarios:
\begin{enumerate}
    \item \textbf{Escenario Baseline (Sin Alineación/Normalización):} Los mismos clasificadores (KNN, MLP, CNN de enfermedad) se entrenarán y evaluarán utilizando características extraídas directamente de las imágenes CXR originales o de una región de interés definida de manera simple (e.g., un \textit{bounding box} detectado por un detector genérico o toda la imagen reescalada). Este escenario sirve para cuantificar el rendimiento base sin una normalización de forma sofisticada.
    \item \textbf{Escenario de Comparación con Alineación Previa (MaShDL con Perfiles 1D):} Se utilizarán los resultados de segmentación/alineación obtenidos con el método MaShDL anterior (basado en perfiles de intensidad 1D, entrenado con \code{train_mashdl_classifiers_v12_mod6_profile_input.py}). Las características para la clasificación de enfermedad se extraerán de estas regiones pulmonares alineadas, y se entrenarán/evaluarán los mismos clasificadores. Esto permitirá medir la mejora específica aportada por el nuevo componente CNN del MaShDL-CNN Hybrid sobre su predecesor.
    \item \textbf{Escenario Propuesto (Alineación con MaShDL-CNN Hybrid):} Los clasificadores se entrenarán/evaluarán utilizando características extraídas de las regiones pulmonares alineadas y normalizadas por el sistema MaShDL-CNN Hybrid desarrollado en esta tesis.
\end{enumerate}
La comparación se realizará utilizando las mismas métricas de evaluación (precisión, sensibilidad, especificidad, F1-score, AUC) en el mismo conjunto de prueba para todos los escenarios, permitiendo un análisis directo del impacto de cada enfoque de normalización.

\subsection{Métricas de Evaluación Detalladas (Objetivo Específico 4)}
\label{ssec:metricas_evaluacion_detalladas_experimental}
Para la validación del clasificador de enfermedades desarrollado (Objetivo Específico 4), se utilizará el conjunto completo de métricas introducidas en la Sección~\ref{ssec:metricas_clasificacion}:
\begin{itemize}
    \item Precisión (\textit{Accuracy}) general y por clase.
    \item Sensibilidad (\textit{Recall}) por clase.
    \item Especificidad por clase.
    \item Puntuación F1 por clase y promediada (e.g., macro-F1, weighted-F1).
    \item Matriz de Confusión.
    \item Curvas ROC y Área Bajo la Curva (AUC) para cada clase (en un enfoque uno-vs-resto) y posiblemente un AUC promediado.
\end{itemize}
Se realizarán pruebas de validación cruzada (e.g., 5-fold o 10-fold cross-validation) sobre el conjunto de entrenamiento/validación combinado para obtener estimaciones más robustas del rendimiento de los clasificadores de enfermedad y para la selección final de hiperparámetros, antes de la evaluación final en el conjunto de prueba independiente.

Este diseño experimental busca proporcionar una evaluación exhaustiva y rigurosa de la metodología propuesta, desde la optimización de sus componentes internos hasta la demostración de su impacto en la tarea final de diagnóstico asistido por computadora.

%\chapter{Introducción al Problema y Enfoque Metodológico}
\label{chap:introduccion_mat}
    \section{Contexto de la Segmentación Pulmonar}
    La segmentación precisa de los campos pulmonares en imágenes médicas, como las radiografías de tórax (CXR), es un paso fundamental en el diagnóstico asistido por computadora (CAD) y en el análisis cuantitativo de diversas patologías pulmonares. La correcta delimitación de los pulmones permite la extracción de características relevantes, la medición de volúmenes, la detección de anomalías y el seguimiento de la progresión de enfermedades. Sin embargo, la segmentación pulmonar en CXR presenta desafíos debido a la variabilidad en la forma y apariencia de los pulmones entre pacientes, la superposición de estructuras anatómicas (e.g., costillas, clavículas, corazón) y la posible presencia de artefactos o patologías que alteran los contornos.

    \section{El Pipeline MaShDl-CNN: Una Visión General Híbrida}
    Para abordar estos desafíos, se ha propuesto el pipeline MaShDl-CNN (Model-based Shape and Deep Learning - Convolutional Neural Network). Este es un enfoque híbrido que combina las fortalezas de los modelos estadísticos de forma (SSM), que codifican el conocimiento a priori sobre la variabilidad de la forma pulmonar, con el poder de las redes neuronales convolucionales (CNN) para aprender características visuales complejas directamente de los datos de la imagen.
    
    El pipeline MaShDl-CNN se puede conceptualizar en varias fases interconectadas:
    \begin{enumerate}
        \item \textbf{Preparación de Datos:} Carga y organización del conjunto de datos de radiografías y sus correspondientes anotaciones de landmarks.
        \item \textbf{Construcción del Modelo Estadístico de Forma (SSM):} A partir de un conjunto de entrenamiento de formas pulmonares (representadas por landmarks), se construye un SSM que captura la variabilidad principal de la forma. Esto implica la densificación de landmarks, el alineamiento mediante Análisis Generalizado de Procrustes (GPA) y la aplicación de Análisis de Componentes Principales (PCA).
        \item \textbf{Estimación de Pose Inicial (ESL):} Se entrena un conjunto de clasificadores basados en CNN para realizar una estimación inicial de la pose (posición, escala y orientación) de los pulmones en una imagen dada.
        \item \textbf{Extracción de Características MaShDL y Predicción de Coeficientes de Forma:} Utilizando la pose estimada por ESL, se muestrean parches de imagen alrededor de los puntos de una instancia del SSM transformada al espacio de la imagen. Estos parches se utilizan como entrada a otros modelos de CNN para predecir los coeficientes del SSM ($b_k$) que mejor describen la forma pulmonar en la imagen.
        \item \textbf{Segmentación Final (implícita):} Con los coeficientes $b_k$ predichos, se puede reconstruir la instancia del SSM que mejor se ajusta a la imagen, proporcionando así la segmentación del contorno pulmonar.
    \end{enumerate}

    \section{Objetivo de este Documento: Detalle Matemático}
    El presente documento tiene como objetivo principal proporcionar una descripción matemática detallada y rigurosa de las fases metodológicas clave del pipeline MaShDl-CNN. Se pondrá especial énfasis en la formulación matemática de cada componente, las ecuaciones subyacentes y la justificación de las elecciones metodológicas desde una perspectiva teórica. Si bien la implementación de ciertos componentes (como los clasificadores ESL y MaShDL) involucra redes neuronales, este documento se centrará en los aspectos matemáticos del modelado de formas, la estimación de pose y la extracción de características, más que en los detalles arquitectónicos de las redes neuronales mismas. Se busca ofrecer una comprensión profunda de los fundamentos matemáticos que sustentan el pipeline MaShDl-CNN.

\chapter{Fase 1: Preparación y Caracterización del Conjunto de Datos}
    \label{chap:datos}
    \section{Descripción del Conjunto de Datos Base}
    \label{ssec:conjunto_datos_mat} % Nueva label para evitar colisión
    El presente trabajo utiliza un conjunto de datos público de radiografías de tórax, comúnmente conocido como el \textit{COVID-19 Radiography Dataset} \cite{Rahman2021covid}. Este conjunto de datos ha sido ampliamente utilizado en la investigación relacionada con la detección y análisis de COVID-19, así como para otras patologías pulmonares. Contiene imágenes de radiografías de tórax en vista posteroanterior (PA) clasificadas en varias categorías, incluyendo COVID-19, Neumonía Viral y Normal.
    
    Para los propósitos de este proyecto de segmentación de pulmones, el enfoque principal recae en la delimitación anatómica de los pulmones, independientemente del diagnóstico patológico original de las imágenes. Las imágenes del dataset están predominantemente en formato PNG y presentan una variedad de resoluciones. El conjunto de datos original también incluye archivos de metadatos que asocian cada imagen con un identificador único y una categoría. En algunos casos, pueden existir anotaciones iniciales o landmarks escasos (e.g., 15 puntos por pulmón) que sirven como punto de partida para el proceso de densificación de landmarks detallado en la Sección~\ref{ssec:densificacion_mat}. % Referencia a la sección ya integrada
    
    En el contexto de este proyecto, se asume que la estructura de directorios del dataset original, junto con los archivos de metadatos como `indices_maestro_1.csv` (que mapea índices internos a identificadores de imagen y categorías) y `coordenadas_maestro_1.csv` (que contiene los landmarks originales), se encuentra accesible en el directorio base `Tesis/`, específicamente en subdirectorios como `COVID-19_Radiography_Dataset/`, `indices/`, y `coordenadas/`.

    \section{Representación Matemática de las Formas Iniciales}
        % Definición de un landmark, una forma como conjunto de landmarks.
        % X_k = {p_1, ..., p_N_inicial}_k
    Cada imagen $I_k$ en el conjunto de datos está asociada con un conjunto inicial de $N_{inicial}$ landmarks (puntos de referencia anatómicos). Cada landmark $j$ de la imagen $k$ es un punto en el espacio 2D, denotado como $\vect{p}_{kj} = (x_{kj}, y_{kj}) \in \realset^2$.
    Por lo tanto, la forma inicial para la imagen $k$ se representa como un conjunto de estos puntos:
    \begin{equation}
        \vect{X}_k = \{\vect{p}_{k1}, \vect{p}_{k2}, \dots, \vect{p}_{kN_{inicial}}\}
    \end{equation}
    Alternativamente, la forma $\vect{X}_k$ puede ser representada como una matriz de $N_{inicial} \times 2$:
    \begin{equation}
        \mat{X}_k = \begin{bmatrix} x_{k1} & y_{k1} \\ x_{k2} & y_{k2} \\ \vdots & \vdots \\ x_{kN_{inicial}} & y_{kN_{inicial}} \end{bmatrix}
    \end{equation}
    En este trabajo, $N_{inicial}=15$. Estos landmarks iniciales son la entrada para el proceso de densificación descrito en la Sección~\ref{ssec:densificacion_mat}.

    \section{División del Conjunto de Datos}
    \label{ssec:data_splitting_mat} % Nueva label
    Para el desarrollo y la evaluación robusta de los modelos, el conjunto de datos total se divide en tres subconjuntos mutuamente excluyentes: entrenamiento, validación y prueba. Esta división es fundamental en el aprendizaje automático para evitar el sobreajuste y para obtener una estimación insesgada del rendimiento del modelo en datos no vistos.
    Sea $\mathcal{D}$ el conjunto total de $N$ muestras disponibles. Se particiona $\mathcal{D}$ en:
    \begin{itemize}
        \item $\mathcal{D}_{train}$: Conjunto de entrenamiento, utilizado para aprender los parámetros de los modelos (e.g., el SSM y los clasificadores ESL).
        \item $\mathcal{D}_{val}$: Conjunto de validación, empleado para ajustar hiperparámetros y para la detección temprana de sobreajuste durante el entrenamiento.
        \item $\mathcal{D}_{test}$: Conjunto de prueba, reservado para la evaluación final del rendimiento del pipeline completo.
    \end{itemize}
    La asignación de muestras a estos conjuntos se realiza típicamente de forma aleatoria, manteniendo ciertas proporciones (e.g., 65\% entrenamiento, 15\% validación, 20\% prueba). Si existen categorías o estratos relevantes en los datos (e.g., tipo de patología), se puede emplear una estratificación para asegurar que la distribución de estas categorías sea similar en todos los subconjuntos. Para la tarea de segmentación de la estructura pulmonar per se, una división aleatoria simple puede ser suficiente si no se esperan sesgos significativos relacionados con las categorías originales del dataset.
    Los índices de las imágenes pertenecientes a cada subconjunto se almacenan para su uso consistente a lo largo del pipeline.

\chapter{Fase 2: Modelado Estadístico de la Forma (SSM)}
    \label{chap:ssm}
    % Esta sección se basará fuertemente en metodologia_ssm.tex

    \section{Densificación de Landmarks mediante B-Splines Cúbicas}
        \label{ssec:densificacion_mat}
El conjunto de datos original puede proporcionar un número limitado de landmarks anotados manualmente (e.g., $N_{inicial}=15$ puntos por pulmón). Para construir un Modelo Estadístico de Forma (SSM) robusto y detallado, es beneficioso trabajar con una representación de forma más densa. En este proyecto, se optó por un conjunto de $N_{final}=144$ landmarks por estructura pulmonar.

El proceso de densificación transforma los $N_{inicial}$ landmarks originales $\vect{P}_{orig} = \{\vect{p}_1, \vect{p}_2, \dots, \vect{p}_{N_{inicial}}\}$ en $N_{final}$ landmarks. Se utilizan B-splines cúbicas ($p=3$) para la interpolación. Una curva B-spline $C(u)$ de grado $p$ se define como:
\begin{equation}
C(u) = \sum_{i=0}^{n} \vect{P}_i N_{i,p}(u)
\label{eq:bspline_curve_merged} % Nueva label para evitar conflicto
\end{equation}
donde $\vect{P}_i$ son los $n+1$ puntos de control (los $N_{inicial}$ landmarks) y $N_{i,p}(u)$ son las funciones base B-spline definidas recursivamente por la fórmula de Cox-de Boor:
\begin{equation}
N_{i,0}(u) = 
\begin{cases} 
1 & \text{si } t_i \le u < t_{i+1} \\
0 & \text{en caso contrario} 
\end{cases}
\label{eq:bspline_base_0_merged} % Nueva label
\end{equation}
\begin{equation}
N_{i,p}(u) = \frac{u - t_i}{t_{i+p} - t_i} N_{i,p-1}(u) + \frac{t_{i+p+1} - u}{t_{i+p+1} - t_{i+1}} N_{i+1,p-1}(u)
\label{eq:bspline_base_p_merged} % Nueva label
\end{equation}
Los $t_j$ son los elementos del vector de nudos. Los 15 landmarks originales sirven como puntos de control para la B-spline cúbica. Luego, se muestrea uniformemente esta curva para obtener los 144 puntos finales. Este proceso se aplica de forma independiente a cada contorno pulmonar (izquierdo y derecho, o superior e inferior según la convención de los datos originales).

La Figura~\ref{fig:densificacion_ejemplo} (originalmente de `metodologia_ssm.tex`) ilustra este proceso.
% \begin{figure}[h!]
%    \centering
%    % Se mantiene la referencia a la figura ya existente en metodologia_ssm.tex
%    \includegraphics[width=0.7\textwidth]{../results/plots_densification/densification_example_idx_0.png}
%    \caption{Ejemplo visual del proceso de densificación de landmarks. Los puntos originales (15) se muestran como marcadores más grandes, y los puntos interpolados (144) forman los contornos continuos.}
%    \label{fig:densificacion_ejemplo} % Label ya definida
% \end{figure}

Adicionalmente, `tesis_mashdl_cnn.tex` menciona una forma alternativa de interpolación lineal para la densificación, que se presenta aquí por completitud aunque el método B-spline es el primario:
Si se tienen $N_{orig}$ landmarks originales $\vect{P}_{orig} = \{\vect{p}_1, \vect{p}_2, \dots, \vect{p}_{N_{orig}}\}$ que definen un contorno poligonal, se pueden insertar $m_i$ puntos entre cada par de landmarks consecutivos $(\vect{p}_i, \vect{p}_{i+1})$ mediante interpolación lineal:
\begin{equation}
    \vect{p}_{i,j}' = (1 - \lambda_j)\vect{p}_i + \lambda_j\vect{p}_{i+1}, \quad \text{para } j=1, \dots, m_i
    \label{eq:linear_interpolation_densification}
\end{equation}
donde $\lambda_j = j/(m_i+1)$. El número de puntos $m_i$ a insertar entre cada par puede ser fijo o ajustarse para lograr una densidad de puntos relativamente uniforme a lo largo del contorno, hasta alcanzar el total deseado de $N_{final}=144$ puntos.
Sin embargo, el método preferido y utilizado para la generación de los datos de 144 puntos en este proyecto es la interpolación mediante B-splines cúbicas debido a su capacidad para generar contornos más suaves y representativos anatómicamente.

    \section{Alineamiento de Formas mediante Análisis Generalizado de Procrustes (GPA)}
        \label{ssec:gpa_mat}
Una vez que se dispone de un conjunto de formas densificadas (los $N_{final}=144$ landmarks por muestra), el siguiente paso crucial antes de construir el SSM es alinear estas formas. El Alineamiento Generalizado de Procrustes (GPA) es una técnica estándar para este propósito. Su objetivo es superponer un conjunto de formas minimizando una medida de distancia entre ellas, eliminando las variaciones de pose (traslación, rotación y escala) que no son intrínsecas a la forma misma.

Cada forma, representada por una matriz de $N_{final} \times D$ (e.g., $144 \times 2$), $\mat{X}_k$, se transforma para minimizar la suma de las distancias cuadradas a una forma media de referencia, $\hat{\vect{\mu}}$, que a su vez se actualiza iterativamente. El proceso detallado es el siguiente (adaptado de la descripción en `metodologia_ssm.tex` y complementado con la perspectiva de `tesis_mashdl_cnn.tex`):

\begin{enumerate}
    \item \textbf{Centrado:} Cada forma $\mat{X}_k$ se centra en el origen restando su centroide $\bar{\vect{x}}_k = \frac{1}{N_{final}} \sum_{j=1}^{N_{final}} \vect{x}_{kj}$ a cada uno de sus puntos.
    \begin{equation}
        \mat{X}_k' = \mat{X}_k - \mathbf{1} \bar{\vect{x}}_k^T
        \label{eq:gpa_centering_merged}
    \end{equation}
    donde $\mathbf{1}$ es un vector columna de unos de tamaño $N_{final}$.

    \item \textbf{Escalado (Normalización):} Cada forma centrada $\mat{X}_k'$ se escala a un tamaño unitario. Una métrica común es la norma de Frobenius de la forma centrada: $s_k = ||\mat{X}_k'||_F = \sqrt{\sum_{j=1}^{N_{final}} \sum_{d=1}^{D} (x'_{kjd})^2}$. La forma escalada es:
    \begin{equation}
        \mat{X}_k'' = \frac{\mat{X}_k'}{s_k}
        \label{eq:gpa_scaling_merged}
    \end{equation}

    \item \textbf{Estimación de la Forma Media Inicial:} Se elige arbitrariamente una forma del conjunto (e.g., $\mat{X}_1''$) o se calcula la media de las formas pre-alineadas (centradas y escaladas) como la estimación inicial de la forma media, $\hat{\vect{\mu}}_0$.

    \item \textbf{Alineamiento Iterativo:} Este proceso se repite hasta la convergencia:
    \begin{enumerate}
        \item \textbf{Alineamiento Individual:} Para cada forma $\mat{X}_k''$, se busca la transformación de rotación óptima $\mat{R}_k$ que minimice la distancia de Procrustes (suma de cuadrados de las diferencias) a la forma media actual $\hat{\vect{\mu}}_{iter}$:
        \begin{equation}
            \mat{R}_k = \arg \min_{\mat{R}} || \hat{\vect{\mu}}_{iter} - \mat{X}_k'' \mat{R} ||_F^2
            \label{eq:gpa_rotacion_optima_merged}
        \end{equation}
        Esta rotación óptima $\mat{R}_k$ se encuentra mediante la Descomposición en Valores Singulares (SVD) de la matriz $\mat{M} = (\mat{X}_k'')^T \hat{\vect{\mu}}_{iter}$. Si $\mat{M} = \mat{U}\mat{\Sigma}\mat{V}^T$, entonces $\mat{R}_k = \mat{V}\mat{U}^T$. Se debe asegurar que $\det(\mat{R}_k)=1$ para evitar reflexiones. La forma alineada es $\mat{X}_{k,aligned} = \mat{X}_k'' \mat{R}_k$.

        \item \textbf{Re-estimación de la Forma Media:} La nueva forma media $\hat{\vect{\mu}}_{iter+1}$ se calcula promediando todas las formas alineadas $\mat{X}_{k,aligned}$:
        \begin{equation}
            \hat{\vect{\mu}}_{iter+1} = \frac{1}{N} \sum_{k=1}^{N} \mat{X}_{k,aligned}
            \label{eq:gpa_mean_reestimation_merged}
        \end{equation}
        donde $N$ es el número total de formas en el conjunto de entrenamiento.

        \item \textbf{Normalización de la Nueva Media:} La nueva media $\hat{\vect{\mu}}_{iter+1}$ se centra y escala a tamaño unitario, similar a los pasos 1 y 2, para servir como referencia en la siguiente iteración.

        \item \textbf{Convergencia:} El proceso itera hasta que la diferencia entre $\hat{\vect{\mu}}_{iter+1}$ y $\hat{\vect{\mu}}_{iter}$ (e.g., $||\hat{\vect{\mu}}_{iter+1} - \hat{\vect{\mu}}_{iter}||_F$) sea menor que un umbral predefinido $\epsilon$, o se alcance un número máximo de iteraciones.
    \end{enumerate}
\end{enumerate}
El resultado del GPA es un conjunto de formas de entrenamiento alineadas, $\mathcal{S}_{aligned} = \{\mat{X}_{k,aligned}\}_{k=1}^N$, y la forma media final del conjunto de entrenamiento, $\hat{\vect{\mu}}_{GPA}$.

Las Figuras~\ref{fig:gpa_raw_overlay} (superposición antes de GPA), \ref{fig:gpa_aligned_overlay} (superposición después de GPA), \ref{fig:gpa_convergence} (convergencia del GPA) y \ref{fig:mean_shape_gpa_viz} (visualización de la forma media GPA) ilustran este proceso. (Nota: Las etiquetas de las figuras se han mantenido o ajustado para consistencia).

% \begin{figure}[H]
%     \centering
%     \includegraphics[width=0.7\textwidth]{../results/plots_gpa/gpa_raw_landmarks_overlay_144pts.png}
%     \caption{Superposición de los landmarks del conjunto de entrenamiento (144 puntos por forma) *antes* de aplicar el Análisis Generalizado de Procrustes. Se muestra también la forma media cruda.}
%     \label{fig:gpa_raw_overlay} % Label ya definida
% \end{figure}

% \begin{figure}[H]
%     \centering
%     \includegraphics[width=0.7\textwidth]{../results/plots_gpa/gpa_aligned_landmarks_overlay_144pts.png}
%     \caption{Superposición de los landmarks del conjunto de entrenamiento (144 puntos por forma) *después* de aplicar el Análisis Generalizado de Procrustes. Se observa una mejor alineación y la forma media resultante del GPA.}
%     \label{fig:gpa_aligned_overlay} % Label ya definida
% \end{figure}

% \begin{figure}[H]
%     \centering
%     \includegraphics[width=0.7\textwidth]{../results/plots_gpa/gpa_convergence_144pts.png}
%     \caption{Curva de convergencia del algoritmo GPA, mostrando la disminución de la norma de la diferencia entre la forma media de iteraciones sucesivas.}
%     \label{fig:gpa_convergence} % Label ya definida
% \end{figure}

% \begin{figure}[H] % Esta figura es referenciada como fig:mean_shape_gpa en tesis_mashdl_cnn.tex
%     \centering
%     \includegraphics[width=0.6\textwidth]{../results/plots_gpa/mean_shape_gpa_144pts.png}
%     \caption{Forma media de los pulmones (144 landmarks) obtenida después de aplicar GPA al conjunto de entrenamiento. Esta forma media representa la configuración promedio de los landmarks en el espacio alineado.}
%     \label{fig:mean_shape_gpa_viz} % Nueva label para esta instancia de la figura
% \end{figure}

    \section{Construcción del Modelo Estadístico de Forma (SSM) con PCA}
        \label{ssec:pca_ssm_mat}
Una vez que las formas del conjunto de entrenamiento han sido alineadas mediante GPA (Sección~\ref{ssec:gpa_mat}), se aplica el Análisis de Componentes Principales (PCA) para construir el Modelo Estadístico de Forma (SSM). El PCA es una técnica de reducción de dimensionalidad que identifica los patrones de variación más significativos (modos de variación) en los datos de forma.

Cada forma alineada $\mat{X}_{k,aligned}$ (compuesta por $N_{final}=144$ landmarks 2D) se representa como un vector $\vect{x}_k \in \realset^{2N_{final}}$ concatenando las coordenadas $(x,y)$ de sus landmarks. El proceso de PCA es el siguiente:
\begin{enumerate}
    \item \textbf{Cálculo de la forma media vectorizada:} Se calcula la forma media $\bar{\vect{x}}$ del conjunto de $N$ formas alineadas y vectorizadas:
    \begin{equation}
        \bar{\vect{x}} = \frac{1}{N} \sum_{k=1}^N \vect{x}_k
        \label{eq:pca_mean_vector_merged}
    \end{equation}
    \item \textbf{Cálculo de la matriz de covarianza:} Se construye la matriz de datos $\mat{D}_x$, donde cada columna es una forma $\vect{x}_k - \bar{\vect{x}}$. La matriz de covarianza $\mat{S} \in \realset^{2N_{final} \times 2N_{final}}$ se calcula como:
    \begin{equation}
        \mat{S} = \frac{1}{N-1} \mat{D}_x \mat{D}_x^T 
        \label{eq:matriz_covarianza_merged}
    \end{equation}
    \item \textbf{Descomposición de Eigenvalores:} Se resuelve el problema de eigenvalores para $\mat{S}$:
    \begin{equation}
        \mat{S}\vect{\phi}_j = \lambda_j \vect{\phi}_j
        \label{eq:pca_eigen_merged}
    \end{equation}
    donde $\lambda_j$ son los eigenvalores (varianzas) y $\vect{\phi}_j$ son los eigenvectores correspondientes (modos de variación o componentes principales). Los eigenvectores se ordenan según sus eigenvalores de mayor a menor.
    \item \textbf{Selección de Componentes Principales:} Se seleccionan los primeros $m < 2N_{final}$ eigenvectores, correspondientes a los $m$ mayores eigenvalores, que capturan un porcentaje suficiente de la varianza total de los datos (e.g., 95-99\%). Estos forman la matriz de modos de variación $\mat{\Phi}_m = [\vect{\phi}_1, \vect{\phi}_2, \dots, \vect{\phi}_m]$.
\end{enumerate}
Cualquier forma $\vect{x}$ del conjunto (o una nueva forma similar) puede entonces ser aproximada por el SSM como una deformación de la forma media a lo largo de estos modos de variación:
\begin{equation}
    \vect{x} \approx \bar{\vect{x}} + \mat{\Phi}_m \vect{b}_m
    \label{eq:ssm_model_merged}
\end{equation}
donde $\vect{b}_m = (b_1, b_2, \dots, b_m)^T$ es un vector de parámetros de forma que controla la contribución de cada modo de variación. Estos parámetros suelen estar limitados, por ejemplo, a $b_j \in [-c\sqrt{\lambda_j}, +c\sqrt{\lambda_j}]$ (e.g., $c=3$) para el modo $j$-ésimo, para asegurar que las formas generadas sean plausibles.

Las Figuras~\ref{fig:pca_variance_explained} (gráfico de varianza explicada) y \ref{fig:pca_mode_variation_example} (ejemplo de visualización de un modo de variación) ilustran aspectos clave del SSM construido. (Nota: Las etiquetas de las figuras se han mantenido o ajustado para consistencia).

% \begin{figure}[H]
%     \centering
%     % La figura pca_explained_variance_144pts.png es la misma que pca_explained_variance_plot.png
%     % referenciada en tesis_mashdl_cnn.tex, se usa la de metodologia_ssm.tex por ser más específica.
%     \includegraphics[width=0.7\textwidth]{../results/plots_pca/pca_explained_variance_144pts.png}
%     \caption{Gráfico de la varianza acumulada explicada por los componentes principales del SSM. Este gráfico ayuda a determinar el número de modos $m$ necesarios para capturar un porcentaje deseado de la variabilidad total de la forma.}
%     \label{fig:pca_variance_explained} % Label ya definida
% \end{figure}

% \begin{figure}[H]
%     \centering
%     % La figura pca_mode_1_visualization_144pts.png es la misma que pca_mode_1_visualization.png
%     % referenciada en tesis_mashdl_cnn.tex, se usa la de metodologia_ssm.tex.
%     \includegraphics[width=0.9\textwidth]{../results/plots_pca/pca_mode_1_visualization_144pts.png} 
%     \caption{Visualización del primer modo de variación del SSM. Se muestra la forma media (centro/gris) y las deformaciones resultantes de variar el parámetro $b_1$ en $\pm c\sqrt{\lambda_1}$ (e.g., $c=2$ o $3$).}
%     \label{fig:pca_mode_variation_example} % Nueva label para esta instancia
% \end{figure}
% Se pueden añadir referencias a fig:modo2_pca y fig:modo3_pca si se desea mostrar más ejemplos de modos.

\chapter{Fase 3: Estimación de Pose Inicial (ESL) - Fundamentos Matemáticos}
    \label{chap:esl_mat}
    % Esta sección expandirá la descripción de ESL de tesis_mashdl_cnn.tex con más detalle matemático.

    \section{Definición Paramétrica de la Pose}
        \label{ssec:esl_pose_definition_mat}
        La pose de un objeto en una imagen 2D puede ser descrita por parámetros de traslación, rotación y escala. En el contexto de ESL, la pose inicial de los pulmones se parametriza mediante una caja delimitadora orientada (OBB). Sin embargo, para simplificar la detección inicial, a menudo se trabaja primero con una caja delimitadora alineada a los ejes (AABB), definida por sus límites $L_1 = x_{min}$, $L_2 = y_{min}$, $L_3 = x_{max}$, y $L_4 = y_{max}$.
        
        A partir de una AABB, los parámetros de pose se pueden definir como:
        \begin{itemize}
            \item \textbf{Traslación} ($\vect{T}$): El centro de la AABB.
                \begin{align}
                    T_x &= (L_1 + L_3) / 2 \\
                    T_y &= (L_2 + L_4) / 2
                \end{align}
            \item \textbf{Escala} ($\vect{S}$): Las dimensiones de la AABB.
                \begin{align}
                    S_{width} &= L_3 - L_1 \\
                    S_{height} &= L_4 - L_2
                \end{align}
            \item \textbf{Orientación} ($\theta$): Para una AABB, el ángulo de orientación con respecto a los ejes de la imagen es $\theta = 0$. La fase de estimación de orientación de ESL buscará refinar este ángulo.
        \end{itemize}
        El objetivo de los clasificadores ESL es estimar estos parámetros $(\hat{T}_x, \hat{T}_y, \hat{S}_{width}, \hat{S}_{height}, \hat{\theta})$ para una imagen dada.

    \section{Generación del Ground Truth para ESL a partir de Landmarks Densificados}
        \label{ssec:esl_gt_mat}
    El primer paso en la preparación de datos para el entrenamiento de los clasificadores ESL es la generación de parámetros de pose \textit{ground truth} (GT). Estos parámetros sirven como la referencia veraz contra la cual se compararán las predicciones de los clasificadores durante su entrenamiento. En el contexto de MaShDl-CNN, estos GT se derivan de los $N_{final}=144$ landmarks por pulmón, previamente generados (Sección~\ref{ssec:densificacion_mat}) y alineados (Sección~\ref{ssec:gpa_mat}), correspondientes a cada imagen del conjunto de entrenamiento.

    Para cada conjunto de $N_{final}$ landmarks $\{\vect{p}_j = (x_j, y_j)\}_{j=1}^{N_{final}}$ de una imagen de entrenamiento, se calculan los parámetros de una caja delimitadora alineada a los ejes (AABB) que encierra dichos puntos. Las coordenadas de esta AABB se definen por:
    \begin{align}
        x_{min} &= \min_{j} \{x_j\} & y_{min} &= \min_{j} \{y_j\} \\
        x_{max} &= \max_{j} \{x_j\} & y_{max} &= \max_{j} \{y_j\}
    \end{align}
    A partir de estos valores extremos, se obtienen los parámetros de la AABB:
    \begin{itemize}
        \item Límites de la caja: $L_1 = x_{min}$, $L_2 = y_{min}$, $L_3 = x_{max}$, $L_4 = y_{max}$.
        \item Centro de la caja (parámetros de traslación $T_x, T_y$):
            \begin{align}
                T_x &= (x_{min} + x_{max}) / 2 \label{eq:esl_tx_gt} \\
                T_y &= (y_{min} + y_{max}) / 2 \label{eq:esl_ty_gt}
            \end{align}
        \item Dimensiones de la caja (parámetros de escala $S_{width}, S_{height}$):
            \begin{align}
                S_{width} &= x_{max} - x_{min} \label{eq:esl_sw_gt} \\
                S_{height} &= y_{max} - y_{min} \label{eq:esl_sh_gt}
            \end{align}
        \item Ángulo de orientación $\theta$: Para una AABB, se considera $\theta = 0$.
    \end{itemize}
    Estos parámetros $(\{L_k\}_{k=1}^4, T_x, T_y, S_{width}, S_{height}, \theta=0)$ para todas las muestras de entrenamiento constituyen el ground truth para la fase ESL. La Figura~\ref{fig:esl_gt_aabb_example_merged} ilustra un ejemplo de esta AABB GT generada.

    % \begin{figure}[H]
    %     \centering
    %     \includegraphics[width=0.7\textwidth]{../results/esl_visualizations/gt_generator/gt_aabb_sample_idx_832.png} 
    %     \caption{Ejemplo de una AABB Ground Truth (rectángulo azul) generada a partir de los 144 landmarks (puntos verdes) para una muestra de entrenamiento. La imagen original se muestra de fondo.}
    %     \label{fig:esl_gt_aabb_example_merged} % Nueva label para evitar conflicto
    % \end{figure}

    \section{Formulación de la Detección de Límites de Contorno y Extracción de Parches}
        \label{ssec:esl_line_detection_mat}
    Una vez obtenidos los parámetros GT de la AABB para cada imagen de entrenamiento, el siguiente paso es extraer regiones de imagen (parches) que servirán como datos de entrada para los clasificadores ESL dedicados a la detección de las líneas $L_1, L_2, L_3, L_4$ de la AABB.

    Para una línea $L_k$ dada (donde $k \in \{1,2,3,4\}$) y una imagen de entrenamiento $I$:
    \begin{itemize}
        \item \textbf{Parche Positivo:} Se extrae un parche de imagen $\Omega_{pos}$ de tamaño $N_p \times N_p$ (e.g., $64 \times 64$ píxeles). El centro de este parche se sitúa sobre la posición ground truth de la línea $L_{k,GT}$. Si $L_k$ es una línea vertical (e.g., $L_1$ en $x=x_{L1,GT}$), el centro del parche en la coordenada $y$ se toma como el centro vertical de la AABB GT ($T_{y,GT}$). Análogamente, si $L_k$ es una línea horizontal (e.g., $L_2$ en $y=y_{L2,GT}$), el centro del parche en la coordenada $x$ se toma como el centro horizontal de la AABB GT ($T_{x,GT}$). Este parche se etiqueta como positivo (clase 1).
        
        \item \textbf{Parches Negativos:} Para cada parche positivo, se generan $N_{neg}$ (e.g., 3) parches negativos $\Omega_{neg,j}$. Estos se extraen de manera similar al parche positivo en términos de la coordenada secundaria (e.g., misma $T_{y,GT}$ para $L_1$), pero sus posiciones en la coordenada primaria (e.g., coordenada $x$ para $L_1$) se eligen aleatoriamente, asegurando que estén a una distancia mínima $\delta_L$ de la posición $L_{k,GT}$. El umbral $\delta_L$ es típicamente una fracción del ancho o alto de la AABB GT (e.g., $\delta_L = 0.1 \times S_{width,GT}$). Estos parches se etiquetan como negativos (clase 0).
    \end{itemize}
    Todos los parches extraídos se normalizan (usualmente a un rango de intensidad de $[0,1]$) y se les puede aplicar aumento de datos, como la adición de ruido gaussiano, antes de ser almacenados. La Figura~\ref{fig:esl_line_patches_example_merged} muestra ejemplos de estos parches.

    La conceptualización matemática de la detección de límites de contorno, aunque implementada mediante una CNN, se basa en la idea de aprender una función que, dado un parche, prediga la probabilidad de que dicho parche esté centrado en la línea de interés. El clasificador $C_{L_k}$ aprende a distinguir parches que contienen la línea $L_k$ de aquellos que no. Durante la inferencia, se realiza una búsqueda deslizando una ventana a lo largo de la imagen (o una región de interés) y evaluando cada parche con el clasificador $C_{L_k}$. La posición del parche que obtiene la máxima probabilidad se considera la posición estimada de la línea $\hat{L}_k$.

    % \begin{figure}[H]
    %     \centering
    %     \includegraphics[width=0.9\textwidth]{../results/esl_visualizations/line_patches/line_patches_idx794_l1.png} % Ejemplo
    %     \caption{Ejemplos de parches generados para el clasificador de la línea $L_1$. Se muestra un parche positivo (izquierda) y parches negativos (derecha), todos normalizados y con posible ruido añadido.}
    %     \label{fig:esl_line_patches_example_merged} % Nueva label
    % \end{figure}

    \section{Formulación de la Estimación de Orientación y Extracción de Parches}
        \label{ssec:esl_orientation_mat}
    El quinto clasificador ESL se encarga de estimar la orientación $\theta$ de la AABB. Para entrenar este clasificador, se generan parches de la siguiente manera:
    \begin{enumerate}
        \item Se extrae un \textit{parche base} $\Omega_{base}$ de la imagen de entrenamiento $I$, centrado en el centroide $(T_{x,GT}, T_{y,GT})$ de la AABB GT. Las dimensiones de este parche base son típicamente mayores que las de la AABB (e.g., escaladas por un factor $\kappa > 1$, como `BASE_PATCH_PADDING_FACTOR = 1.1` en la implementación) para incluir contexto y asegurar que la AABB completa esté contenida después de la rotación.
        \item Este parche base $\Omega_{base}$ se rota mediante un conjunto discreto de ángulos de prueba $\{\alpha_r\}$ (e.g., de $-30^\circ$ a $+30^\circ$ en pasos de $5^\circ$). Sea $\Omega^{(r)}$ el parche resultante de rotar $\Omega_{base}$ por un ángulo $\alpha_r$.
        \item Cada parche rotado $\Omega^{(r)}$ se reescala al tamaño final $N_p \times N_p$ (e.g., $64 \times 64$), se normaliza y se le puede añadir ruido.
        \item La etiqueta para $\Omega^{(r)}$ es positiva (clase 1) si el ángulo de rotación aplicado $|\alpha_r|$ es menor o igual a un umbral angular pequeño $\delta_{\theta}$ (e.g., $5^\circ$), indicando que la orientación del parche está "cerca" de la canónica (0 grados). De lo contrario, la etiqueta es negativa (clase 0).
    \end{enumerate}
    La Figura~\ref{fig:esl_orientation_patches_example_merged} ilustra este proceso. El clasificador de orientación $C_{\theta}$ aprende a distinguir parches que tienen una orientación cercana a la canónica de aquellos que están significativamente rotados. Durante la inferencia, se extrae un parche base de la imagen (centrado en la AABB estimada por los clasificadores de línea), se rota a través de los mismos ángulos de prueba, y el ángulo que produce la máxima probabilidad del clasificador $C_{\theta}$ se selecciona como la orientación estimada $\hat{\theta}$.

    % \begin{figure}[H]
    %     \centering
    %     \includegraphics[width=0.95\textwidth]{../results/esl_visualizations/orientation_patches/orientation_patches_idx455.png} % Ejemplo
    %     \caption{Ejemplos de parches generados para el clasificador de orientación. Se muestra el parche base original (izquierda) y una selección de parches rotados (normalizados y con ruido) con sus respectivos ángulos de rotación aplicados y la etiqueta de clase resultante.}
    %     \label{fig:esl_orientation_patches_example_merged} % Nueva label
    % \end{figure}

    \section{Composición de la Pose Estimada y Proceso de Inferencia}
        \label{ssec:esl_prediction_composition_mat}
    Una vez entrenados los cinco clasificadores ESL ($C_{L1}, C_{L2}, C_{L3}, C_{L4}, C_{\theta}$), se pueden utilizar para estimar la pose inicial (parámetros de traslación $\vect{T}=(T_x, T_y)$, escala $\vect{S}=(S_{width}, S_{height})$, y orientación $\theta$) de los pulmones en una imagen de prueba no vista. El proceso de inferencia es el siguiente:

    Dada una imagen de entrada $I_{test}$:
    \begin{enumerate}
        \item \textbf{Predicción de Límites de Contorno:} Para cada línea $L_k \in \{L_1, L_2, L_3, L_4\}$, se realiza una búsqueda mediante ventana deslizante. Se extraen parches a lo largo de una trayectoria perpendicular a la orientación esperada de la línea (inicialmente $\theta=0$), dentro de un rango de búsqueda predefinido en la imagen. Cada parche se normaliza y se alimenta al clasificador $C_{Lk}$ correspondiente. La posición que obtiene la máxima puntuación (probabilidad) del clasificador se considera la posición estimada $\hat{L}_k$ de la línea.

        \item \textbf{Cálculo de Traslación y Escala Iniciales:} A partir de las posiciones estimadas de las cuatro líneas $(\hat{L}_1, \hat{L}_2, \hat{L}_3, \hat{L}_4)$, se calculan los parámetros de traslación $\hat{T}_x, \hat{T}_y$ y escala (ancho $\hat{S}_{width}$, alto $\hat{S}_{height}$) de la AABB, de forma análoga a las Ecuaciones~\ref{eq:esl_tx_gt}-\ref{eq:esl_sh_gt}:
            \begin{align}
                \hat{T}_x &= (\hat{L}_1 + \hat{L}_3) / 2 \\
                \hat{T}_y &= (\hat{L}_2 + \hat{L}_4) / 2 \\
                \hat{S}_{width} &= \hat{L}_3 - \hat{L}_1 \\
                \hat{S}_{height} &= \hat{L}_4 - \hat{L}_2
            \end{align}
        Es importante asegurar que $\hat{L}_1 < \hat{L}_3$ y $\hat{L}_2 < \hat{L}_4$. Si las predicciones iniciales no cumplen esto, se pueden aplicar correcciones heurísticas (e.g., intercambiarlas o basarse en la línea con mayor confianza).

        \item \textbf{Predicción de Orientación:} Se extrae un parche base de la imagen $I_{test}$, centrado en $(\hat{T}_x, \hat{T}_y)$ y con dimensiones proporcionales a $(\hat{S}_{width}, \hat{S}_{height})$ (usando `BASE_PATCH_PADDING_FACTOR`). Este parche base se rota a través del mismo conjunto de ángulos de prueba discretos $\{\alpha_r\}$ utilizado durante el entrenamiento. Cada parche rotado se normaliza y se evalúa con el clasificador de orientación $C_{\theta}$. El ángulo $\alpha_r$ que produce la máxima puntuación se selecciona como la orientación estimada $\hat{\theta}$.
    \end{enumerate}
    El conjunto de parámetros $(\hat{T}_x, \hat{T}_y, \hat{S}_{width}, \hat{S}_{height}, \hat{\theta})$ constituye la pose ESL predicha. La Figura~\ref{fig:esl_prediction_example_merged} muestra un ejemplo de la AABB predicha y rotada.

    % \begin{figure}[H]
    %     \centering
    %     \includegraphics[width=0.7\textwidth]{../results/esl_visualizations/predictions/esl_prediction_idx241.png} % Ejemplo
    %     \caption{Ejemplo de predicción de pose ESL. La caja delimitadora estimada, rotada según el ángulo $\hat{\theta}$ predicho, se superpone en color verde sobre la imagen de prueba.}
    %     \label{fig:esl_prediction_example_merged} % Nueva label
    % \end{figure}

\chapter{Fase 4: Extracción de Características MaShDL - Fundamentos Matemáticos}
    \label{chap:mashdl_mat}
    La fase MaShDL (Model-based Shape and Deep Learning) tiene como objetivo predecir los coeficientes $\vect{b}$ del Modelo Estadístico de Forma (SSM) directamente a partir de características de la imagen. Esto se logra entrenando clasificadores (o regresores) que toman como entrada parches de imagen muestreados alrededor de los puntos de una instancia de forma y predicen los coeficientes del SSM.

    \section{Transformación de Instancias del SSM al Espacio de la Imagen}
        \label{ssec:mashdl_ssm_instance_transform_mat}
        Una instancia de forma generada por el SSM, $\vect{x}_{model} \approx \bar{\vect{x}} + \mat{\Phi}_m \vect{b}_m$, está en el espacio canónico del SSM (alineado y normalizado). Para muestrear parches de una imagen $I$, esta forma canónica debe ser transformada al espacio de la imagen utilizando la pose $(\hat{\vect{T}}, \hat{\vect{S}}, \hat{\theta})$ estimada por la fase ESL.
        
        Sea $\vect{x}_{model,j} = (x_j, y_j)$ el $j$-ésimo landmark de la instancia del SSM en coordenadas canónicas. La transformación a coordenadas de imagen $\vect{x}_{img,j} = (x'_j, y'_j)$ implica:
        \begin{enumerate}
            \item \textbf{Escalado no uniforme:} Los spans (rangos) de la forma media canónica $\bar{\vect{x}}$ en las direcciones $x$ e $y$, denotados $span_x^{canon}$ y $span_y^{canon}$, se escalan para coincidir con las dimensiones $\hat{S}_{width}$ y $\hat{S}_{height}$ de la AABB estimada por ESL. Los factores de escala son:
            \begin{align}
                s_x &= \hat{S}_{width} / span_x^{canon} \\
                s_y &= \hat{S}_{height} / span_y^{canon}
            \end{align}
            Las coordenadas escaladas $\vect{x}_{scaled,j} = (x_j \cdot s_x, y_j \cdot s_y)$.
            \item \textbf{Rotación:} Se aplica la rotación estimada $\hat{\theta}$ alrededor del centroide de la forma escalada (que corresponde al centroide de la forma canónica, usualmente el origen). Si $\mat{R}(\hat{\theta})$ es la matriz de rotación:
            \begin{equation}
                \vect{x}_{rot,j} = \mat{R}(\hat{\theta}) \vect{x}_{scaled,j}
            \end{equation}
            \item \textbf{Traslación:} Finalmente, se traslada la forma rotada para que su centroide coincida con el centroide estimado $\hat{\vect{T}} = (\hat{T}_x, \hat{T}_y)$ de la AABB en la imagen:
            \begin{equation}
                \vect{x}_{img,j} = \vect{x}_{rot,j} + \hat{\vect{T}}
            \end{equation}
        \end{itemize}
        Este proceso se aplica a todos los $N_{final}$ landmarks de la instancia del SSM para obtener su configuración en el espacio de la imagen, $\mat{X}_{img}$.

    \section{Muestreo de Parches alrededor de Puntos de Forma Transformados}
        \label{ssec:mashdl_patch_sampling_mat}
        Una vez que la instancia del SSM, $\mat{X}_{img}$, se ha transformado al espacio de la imagen, se extraen parches de imagen alrededor de cada uno de sus $N_{final}$ puntos. Para cada punto $\vect{x}_{img,j}$:
        \begin{itemize}
            \item Se extrae un parche $\Omega_j$ de tamaño $Q \times Q$ (e.g., $Q=25$) centrado en $\vect{x}_{img,j}$.
            \item El parche se normaliza en intensidad (e.g., a $[0,1]$).
        \end{itemize}
        El conjunto de $N_{final}$ parches $\{\Omega_1, \Omega_2, \dots, \Omega_{N_{final}}\}$ se concatena (después de aplanarlos) para formar un único vector de características $\vect{f} \in \realset^{N_{final} \cdot Q^2}$. Este vector $\vect{f}$ representa la apariencia local alrededor de la instancia de forma actual.

    \section{Discretización de los Coeficientes del Modelo de Forma ($b_k$)}
        \label{ssec:mashdl_b_discretization_mat}
        Para entrenar clasificadores que predigan los coeficientes $b_k$ del SSM, estos coeficientes continuos se discretizan en $B_{bins}$ (e.g., 3) contenedores o clases. Para un coeficiente $b_k$ con desviación estándar $\sigma_k = \sqrt{\lambda_k}$ (obtenida del PCA):
        \begin{enumerate}
            \item El rango de $b_k$ se limita típicamente a $[-c \cdot \sigma_k, +c \cdot \sigma_k]$ (e.g., $c=3$).
            \item Este rango se divide uniformemente en $B_{bins}$ intervalos.
            \item El valor continuo $b_k$ se asigna al índice del bin al que pertenece.
        \end{itemize}
        Por ejemplo, para $B_{bins}=3$, los bins podrían representar valores "negativo alto", "cercano a cero" y "positivo alto" del coeficiente $b_k$. La etiqueta $y_k$ para el clasificador del modo $k$ será el índice de este bin.

    \section{Objetivo de Aprendizaje para los Clasificadores MaShDL}
        \label{ssec:mashdl_learning_objective_mat}
        Para cada modo de variación $k$ del SSM (desde $1$ hasta $m$), se entrena un clasificador independiente $C_k^{MaShDL}$.
        El objetivo de $C_k^{MaShDL}$ es aprender a predecir la etiqueta del bin discretizado $y_k$ del coeficiente $b_k$, dado el vector de características de parches $\vect{f}$ extraído de una instancia de forma generada con un conjunto de coeficientes $\vect{b}$ donde $b_k$ es el valor de interés y los $b_j (j<k)$ son los valores GT ya conocidos (o predichos en una cascada).
        
        Matemáticamente, se busca aprender la función:
        \begin{equation}
            \hat{y}_k = C_k^{MaShDL}(\vect{f} | b_1, \dots, b_{k-1})
        \end{equation}
        donde $\vect{f}$ se extrae usando una forma generada por $\bar{\vect{x}} + \sum_{j=1}^{k-1} \phi_j b_j + \sum_{j=k}^{m} \phi_j b_j^{init}$ (donde $b_j^{init}$ pueden ser cero o valores perturbados para generar ejemplos positivos y negativos para $b_k$).
        
        Durante el entrenamiento, se generan ejemplos positivos (donde $b_k$ es el valor GT para la imagen) y negativos (donde $b_k$ es un valor perturbado) para cada modo $k$. Los clasificadores $C_k^{MaShDL}$ aprenden a distinguir entre las apariencias de los parches correspondientes a diferentes valores (bins) del coeficiente $b_k$.

\chapter{Diseño Experimental y Configuración Matemática}
    \label{chap:diseno_experimental_mat}
            La validación empírica de los constructos teóricos previamente delineados se fundamenta en un diseño experimental riguroso y una configuración precisa del entorno computacional. Este apartado detalla la procedencia y naturaleza del conjunto de datos, los parámetros específicos empleados en cada fase metodológica y las herramientas matemáticas subyacentes a la implementación.

            \subsection{Conjunto de Datos y Origen}
            El estudio se basa en un conjunto de datos de imágenes médicas, específicamente radiografías torácicas, de las cuales se han extraído manualmente las coordenadas de los contornos pulmonares. Cada contorno, inicialmente representado por $N_{inicial}=15$ landmarks, constituye una instancia o muestra $X_k$ en nuestro análisis. La naturaleza de estos datos es crucial, pues la variabilidad inherente a las formas pulmonares en esta población de estudio es el objeto principal que el Modelo Estadístico de Forma (SSM) busca parametrizar.
            La especificación del número total de muestras (N) y la referencia detallada de la fuente del conjunto de datos se omiten en este resumen matemático, pero son componentes importantes del diseño experimental completo.

            \subsection{Parámetros de la Densificación de Landmarks}
            El primer paso metodológico, la densificación de landmarks, transforma cada conjunto de $N_{inicial}=15$ puntos a $N_{final}=144$ puntos. Este proceso se realiza mediante la interpolación con B-splines cúbicas (grado $p=3$), tal como se describe en la Ecuación~\ref{eq:bspline_curve}. La elección de B-splines cúbicas obedece a su capacidad para generar curvas suaves ($C^2$ continuas) que representan adecuadamente las formas anatómicas, evitando las oscilaciones que podrían surgir con polinomios de grado superior. El vector de nudos para cada B-spline se define de manera estándar, usualmente normalizado y uniforme, para asegurar una interpolación consistente a lo largo de la curva. La Figura~\ref{fig:densificacion_ejemplo} ilustra el resultado de este proceso.

            \subsection{Configuración del Análisis Generalizado de Procrustes (GPA)}
            El GPA se aplica al conjunto de $N$ formas densificadas $X_k \in \mathbb{R}^{144 \times 2}$. Los pasos clave son:
            \begin{enumerate}
                \item \textbf{Centrado:} Cada forma $X_k$ se traslada para que su centroide coincida con el origen.
                \item \textbf{Escalado:} Cada forma centrada $X_k'$ se escala a tamaño unitario. Se utiliza la norma de Frobenius del centroide, $s_k = ||X_k'||_F$, para la normalización, resultando en $X_k'' = X_k'/s_k$. Esta elección asegura que todas las formas contribuyan de manera equitativa al análisis, independientemente de su tamaño original.
                \item \textbf{Alineamiento Iterativo:}
                \begin{itemize}
                    \item La forma media inicial $\hat{\mu}_0$ se selecciona como una de las formas del conjunto (e.g., la primera forma $X_1''$).
                    \item En cada iteración, cada forma $X_k''$ se alinea a la forma media actual $\hat{\mu}_{iter}$ mediante una transformación de rotación $R_k$ que minimiza la distancia de Procrustes, como se indica en la Ecuación~\ref{eq:gpa_rotacion_optima}. La rotación óptima se obtiene vía SVD de $(X_k'')^T \hat{\mu}_{iter}$.
                    \item La forma media se re-estima como el promedio de las formas alineadas: $\hat{\mu}_{iter+1} = \frac{1}{N} \sum_{k=1}^{N} X_{k,aligned}$.
                    \item La nueva media $\hat{\mu}_{iter+1}$ se normaliza (centrado y escalado).
                    \item La convergencia se alcanza cuando la norma de la diferencia entre medias sucesivas, $||\hat{\mu}_{iter+1} - \hat{\mu}_{iter}||_F$, es inferior a un umbral $\epsilon$ (e.g., $\epsilon = 10^{-6}$). Las Figuras~\ref{fig:gpa_raw_overlay}, \ref{fig:gpa_aligned_overlay}, y \ref{fig:gpa_convergence} muestran la efectividad y convergencia del proceso. La forma media final $\hat{\mu}_{final}$ se visualiza en la Figura~\ref{fig:forma_media_gpa}.
                \end{itemize}
            \end{enumerate}

            \subsection{Parámetros del Análisis de Componentes Principales (PCA)}
            El PCA se aplica al conjunto de $N$ formas alineadas y vectorizadas $x_k \in \mathbb{R}^{288}$.
            \begin{enumerate}
                \item \textbf{Matriz de Covarianza:} Se calcula la matriz de covarianza $S \in \mathbb{R}^{288 \times 288}$ a partir de los datos centrados, según la Ecuación~\ref{eq:matriz_covarianza}.
                \item \textbf{Análisis de Eigenvalores:} Se resuelve el problema de eigenvalores $S \Phi_i = \lambda_i \Phi_i$ (Ecuación~\ref{eq:pca_eigen}) para obtener los eigenvalores $\lambda_i$ (varianzas) y los eigenvectores $\Phi_i$ (modos de variación).
                \item \textbf{Selección de Modos:} El número de modos $m$ a retener en el SSM (Ecuación~\ref{eq:ssm_model}) se determina analizando la varianza explicada acumulada. Se busca retener un porcentaje significativo de la varianza total, típicamente entre el 95\% y el 99\%. La Figura~\ref{fig:varianza_explicada_pca} muestra esta relación, donde se observa que un número reducido de modos (e.g., $m \approx 15-20$) suele ser suficiente. Los primeros modos de variación (Figuras~\ref{fig:modo1_pca}, \ref{fig:modo2_pca}, \ref{fig:modo3_pca}) ilustran las deformaciones anatómicas más prominentes. Los parámetros de forma $b_i$ para la generación de nuevas instancias se suelen restringir al intervalo $[-c\sqrt{\lambda_i}, +c\sqrt{\lambda_i}]$, donde $c$ es una constante (e.g., $c=2$ o $c=3$) para asegurar la plausibilidad anatómica.
            \end{enumerate}
            La rigurosidad en la definición de estos parámetros y la correcta aplicación de los métodos matemáticos son esenciales para la validez y reproducibilidad de los resultados del SSM.
    % Se añadirán aquí las secciones para ESL y MaShDL
    \section{Parámetros de la Estimación de Pose Inicial (ESL)}
        \label{ssec:parametros_esl}
        La fase de ESL involucra la extracción de parches y el entrenamiento de clasificadores. Los parámetros matemáticos y conceptuales clave incluyen:
        \begin{itemize}
            \item \textbf{Tamaño del Parche para Clasificadores de Línea y Orientación ($N_p$):} Define la dimensión de la ventana de imagen utilizada como entrada para las CNNs de ESL (e.g., $64 \times 64$ píxeles).
            \item \textbf{Ángulos de Rotación para el Clasificador de Orientación ($\{\alpha_r\}$):} Conjunto discreto de ángulos utilizados para generar parches rotados para entrenar el clasificador de orientación (e.g., $\{-30^\circ, -25^\circ, \dots, +25^\circ, +30^\circ\}$ en pasos de $5^\circ$).
            \item \textbf{Umbral Angular para Etiquetas de Orientación ($\delta_{\theta}$):} Define el rango de ángulos (e.g., $\pm 5^\circ$) alrededor de la orientación canónica ($0^\circ$) que se considera una etiqueta positiva para el clasificador de orientación.
            \item \textbf{Factor de Relleno del Parche Base para Orientación ($\kappa$):} Factor por el cual se escalan las dimensiones de la AABB GT para extraer el parche base antes de la rotación (e.g., $\kappa = 1.1$), para asegurar que la AABB completa esté contenida después de rotaciones.
            \item \textbf{Distancia Mínima para Parches Negativos de Línea ($\delta_L$):} Fracción del ancho/alto de la AABB GT que define la separación mínima entre un parche positivo de línea y los parches negativos generados.
        \end{itemize}

    \section{Parámetros de la Extracción de Características MaShDL}
        \label{ssec:parametros_mashdl}
        La extracción de características para los clasificadores MaShDL también se rige por parámetros importantes:
        \begin{itemize}
            \item \textbf{Tamaño de Parche de Landmark ($Q$):} Dimensión de los pequeños parches extraídos alrededor de cada uno de los $N_{final}$ puntos de la instancia del SSM transformada (e.g., $Q=25$, resultando en parches de $25 \times 25$ píxeles).
            \item \textbf{Número de Bins para Discretización de $b_k$ ($B_{bins}$):} Número de categorías en las que se divide cada coeficiente continuo $b_k$ del SSM para la clasificación (e.g., $B_{bins}=3$).
            \item \textbf{Factor de Clamp para $b_k$ ($c$):} Múltiplo de la desviación estándar $\sigma_k$ utilizado para definir el rango $[-c\sigma_k, +c\sigma_k]$ dentro del cual se discretizan los valores de $b_k$ (e.g., $c=3$).
        \end{itemize}

\chapter{Resultados y Discusión (Enfoque Matemático/Estadístico)}
    \label{chap:resultados_mat}
    % (Adaptado de metodologia_ssm.tex, sec:resultados_ssm y tesis_mashdl_cnn.tex)
    \section{Resultados del Modelado Estadístico de Forma (SSM)}
% Presentación de los resultados obtenidos en la construcción del SSM.

\subsection{Forma Media del GPA}
\label{ssec:forma_media_gpa_results} % Renombrar label para evitar colisión si es necesario
Tras aplicar el Análisis Generalizado de Procrustes (GPA) al conjunto de entrenamiento de formas pulmonares (cada una representada por 144 landmarks), se obtiene una forma media que representa la configuración promedio de los pulmones en el dataset, libre de variaciones de escala, rotación y traslación. Esta forma media es fundamental, ya que sirve como el origen del espacio de formas sobre el cual se construye el SSM.

La Figura~\ref{fig:forma_media_gpa} (a generar posteriormente) mostrará la visualización de esta forma media. Se espera que represente un contorno pulmonar típico y bien definido.

% Placeholder para la figura
% \begin{figure}[h!]
%     \centering
%     \includegraphics[width=0.6\textwidth]{../results/plots_gpa/gpa_mean_shape_144pts.png}
%     \caption{Forma media de los contornos pulmonares (144 puntos) obtenida después del alineamiento con GPA. Se muestran los dos contornos (izquierdo/derecho o superior/inferior según la convención de los datos) que componen la forma completa.}
%     \label{fig:forma_media_gpa}
% \end{figure}

\subsection{Varianza Explicada por PCA}
\label{ssec:varianza_explicada_pca_results} % Renombrar label
El Análisis de Componentes Principales (PCA) no solo proporciona los modos de variación, sino que también cuantifica cuánta de la varianza total en el conjunto de datos es explicada por cada modo. El eigenvalor $\lambda_i$ asociado al eigenvector $\Phi_i$ es directamente proporcional a la varianza explicada por ese modo.

Para determinar cuántos modos de variación son necesarios para capturar la mayor parte de la variabilidad de la forma, se suele analizar el gráfico de varianza explicada acumulada (a veces llamado "scree plot"). Este gráfico muestra el porcentaje de la varianza total que es capturado al incluir sucesivamente los componentes principales, ordenados de mayor a menor eigenvalor.

La Figura~\ref{fig:varianza_explicada_pca} (a generar posteriormente) presentará este gráfico. Típicamente, se observa que un número relativamente pequeño de componentes principales captura un alto porcentaje de la varianza total (e.g., 90-99\%), lo que permite una representación compacta y eficiente de la variabilidad de la forma. Esto es crucial para la eficiencia del SSM.

% Placeholder para la figura
% \begin{figure}[h!]
%     \centering
%     \includegraphics[width=0.7\textwidth]{../results/plots_pca/pca_explained_variance_144pts.png}
%     \caption{Varianza explicada acumulada por los componentes principales del SSM construido sobre los landmarks de 144 puntos. Se indica el número de componentes retenidos (e.g., 15) y el umbral de varianza informativa (e.g., 95\%).}
%     \label{fig:varianza_explicada_pca}
% \end{figure}

\subsection{Interpretación de los Modos Principales de Variación}
\label{ssec:modos_variacion_pca_results} % Renombrar label
Los eigenvectores $\Phi_i$ del PCA, también conocidos como modos de variación, describen las principales direcciones en las que las formas pulmonares varían respecto a la forma media $\bar{x}$. Cada modo captura un patrón específico de deformación. Para visualizar el efecto de un modo $\Phi_i$, se pueden generar nuevas formas variando el parámetro $b_i$ correspondiente en la Ecuación~\ref{eq:ssm_model}, comúnmente en un rango de $\pm k \cdot \sqrt{\lambda_i}$ (por ejemplo, $k=1, 2, 3$ desviaciones estándar) mientras se mantienen los demás parámetros $b_j (j \ne i)$ en cero.

Las Figuras~\ref{fig:modo1_pca}, \ref{fig:modo2_pca}, etc. (a generar posteriormente) mostrarán la forma media y las variaciones producidas al moverse a lo largo de los primeros modos principales de variación. Por ejemplo, el primer modo podría capturar la variación global en el tamaño o la elongación del pulmón, el segundo modo podría estar relacionado con la curvatura del ápex o la base, y así sucesivamente. El análisis de estos modos es crucial para entender la naturaleza de la variabilidad anatómica presente en el conjunto de datos.

% Placeholders para las figuras de los modos
% \begin{figure}[h!]
%     \centering
%     \includegraphics[width=\textwidth]{../results/plots_pca/pca_mode_1_visualization_144pts.png}
%     \caption{Visualización del primer modo principal de variación del SSM (144 puntos). Se muestra la forma media (gris, superpuesta) y las variaciones generadas al sumar/restar múltiplos (e.g., $\pm 2$) de la desviación estándar del modo a la forma media.}
%     \label{fig:modo1_pca}
% \end{figure}

% \begin{figure}[h!]
%     \centering
%     \includegraphics[width=\textwidth]{../results/plots_pca/pca_mode_2_visualization_144pts.png}
%     \caption{Visualización del segundo modo principal de variación del SSM (144 puntos).}
%     \label{fig:modo2_pca}
% \end{figure}

% \begin{figure}[h!]
%     \centering
%     \includegraphics[width=\textwidth]{../results/plots_pca/pca_mode_3_visualization_144pts.png}
%     \caption{Visualización del tercer modo principal de variación del SSM (144 puntos).}
%             \label{fig:modo3_pca} % Asegurar que el label sea único si se añaden más
%             \end{figure}
    \section{Evaluación de la Estimación de Pose Inicial (ESL)}
        \label{ssec:resultados_esl}
        La evaluación del rendimiento de la fase ESL se centra en cuantificar la precisión con la que se estiman los parámetros de pose $(\hat{T}_x, \hat{T}_y, \hat{S}_{width}, \hat{S}_{height}, \hat{\theta})$ en comparación con los parámetros ground truth derivados de los landmarks manuales (Sección~\ref{ssec:esl_gt_mat}). Esta evaluación se realiza típicamente sobre un conjunto de prueba $\mathcal{D}_{test}$ que no fue utilizado durante el entrenamiento de los clasificadores ESL.
        
        Las métricas comunes para evaluar la precisión de la pose incluyen:
        \begin{itemize}
            \item \textbf{Error de Traslación:} La distancia euclidiana entre el centroide predicho $\hat{\vect{T}}=(\hat{T}_x, \hat{T}_y)$ y el centroide GT $\vect{T}_{GT}=(T_{x,GT}, T_{y,GT})$:
            \begin{equation}
                Error_T = || \hat{\vect{T}} - \vect{T}_{GT} ||_2
            \end{equation}
            \item \textbf{Error de Escala:} Diferencias relativas o absolutas entre las dimensiones predichas $(\hat{S}_{width}, \hat{S}_{height})$ y las dimensiones GT $(S_{width,GT}, S_{height,GT})$. Por ejemplo, el error relativo promedio:
            \begin{equation}
                Error_S = \frac{1}{2} \left( \frac{|\hat{S}_{width} - S_{width,GT}|}{S_{width,GT}} + \frac{|\hat{S}_{height} - S_{height,GT}|}{S_{height,GT}} \right)
            \end{equation}
            \item \textbf{Error de Orientación:} La diferencia absoluta entre el ángulo predicho $\hat{\theta}$ y el ángulo GT $\theta_{GT}$ (que es 0 para AABBs):
            \begin{equation}
                Error_{\theta} = |\hat{\theta} - \theta_{GT}|
            \end{equation}
            \item \textbf{Índice de Jaccard (IoU) de la Caja Delimitadora:} Si se considera la caja delimitadora orientada (OBB) resultante de la pose estimada, se puede calcular el IoU entre la OBB predicha y la OBB GT.
            \begin{equation}
                IoU = \frac{\text{Area}(\text{OBB}_{pred} \cap \text{OBB}_{GT})}{\text{Area}(\text{OBB}_{pred} \cup \text{OBB}_{GT})}
            \end{equation}
        \end{itemize}
        Los resultados de estas métricas, promediados sobre el conjunto de prueba, proporcionan una medida cuantitativa de la efectividad de la fase ESL. Se podrían presentar distribuciones de estos errores (e.g., histogramas o boxplots) para analizar la robustez del método.
        % La generación de figuras específicas para la distribución de errores de ESL se abordará
        % en la fase de modificación de scripts Python, si se determina necesario.

    \section{Análisis de los Coeficientes $b_k$ del SSM}
        \label{ssec:resultados_b_coeficientes}
        Los coeficientes $b_k$ del SSM (Ecuación~\ref{eq:ssm_model_merged}) representan los pesos de cada modo de variación principal $\vect{\phi}_k$ necesario para reconstruir una forma específica a partir de la forma media $\bar{\vect{x}}$. El análisis de la distribución de estos coeficientes en el conjunto de entrenamiento puede ofrecer información valiosa:
        \begin{itemize}
            \item \textbf{Distribución de $b_k$ por Modo:} Para cada modo $k$, los coeficientes $b_k$ calculados para todas las formas del conjunto de entrenamiento deberían, idealmente, seguir una distribución aproximadamente normal (o al menos unimodal y centrada en cero si los datos fueron bien normalizados por PCA). La varianza de esta distribución es $\lambda_k$.
            \item \textbf{Correlación entre Coeficientes:} Dado que los componentes principales (eigenvectores $\vect{\phi}_k$) son ortogonales, los coeficientes $b_k$ deberían ser incorrelacionados entre sí.
        \end{itemize}
        Se pueden generar histogramas de los valores de $b_k$ para los primeros modos de variación para visualizar sus distribuciones. Estos histogramas pueden ayudar a verificar la calidad del modelo SSM y a entender cómo se distribuye la variabilidad de la forma a lo largo de cada modo.
        % La generación de histogramas para los coeficientes b_k se abordará
        % en la fase de modificación de scripts Python, si se determina necesario.

\chapter{Conclusiones y Perspectivas Matemáticas}
    \label{chap:conclusiones_mat}
            En el presente estudio, se ha detallado la fundamentación matemática y la configuración experimental para la construcción de un Modelo Estadístico de Forma (SSM) a partir de contornos pulmonares. El proceso inicia con una etapa de densificación de landmarks, donde representaciones escasas de 15 puntos se enriquecen a 144 puntos mediante la aplicación de B-splines cúbicas (Ecuaciones~\ref{eq:bspline_curve}-\ref{eq:bspline_base_p}). Esta interpolación no solo incrementa la densidad de puntos, sino que también asegura una representación suave y continua de los contornos, esencial para capturar las sutilezas de la anatomía pulmonar. La Figura~\ref{fig:densificacion_ejemplo} ofrece una evidencia visual de la efectividad de esta transformación.

            Subsiguientemente, el Análisis Generalizado de Procrustes (GPA) se emplea para alinear el conjunto de formas densificadas. Este método iterativo elimina sistemáticamente las variaciones de traslación, escala y rotación, proyectando todas las formas a un espacio común de referencia. La minimización de la distancia de Procrustes (Ecuación~\ref{eq:gpa_rotacion_optima}) y la convergencia del algoritmo (Figura~\ref{fig:gpa_convergence}) garantizan un alineamiento óptimo, resultando en una forma media $\hat{\mu}_{final}$ (Figura~\ref{fig:forma_media_gpa}) que representa la morfología pulmonar promedio del conjunto de datos, libre de las citadas variaciones extrínsecas. Las Figuras~\ref{fig:gpa_raw_overlay} y \ref{fig:gpa_aligned_overlay} contrastan el estado de las formas antes y después del GPA, respectivamente, demostrando la importancia de este preprocesamiento.

            Una vez alineadas las formas, se procede a la construcción del SSM mediante el Análisis de Componentes Principales (PCA). Cada forma alineada se vectoriza, y se calcula la matriz de covarianza del conjunto (Ecuación~\ref{eq:matriz_covarianza}). La resolución del problema de eigenvalores para esta matriz (Ecuación~\ref{eq:pca_eigen}) revela los modos principales de variación $\Phi_i$ y sus correspondientes varianzas $\lambda_i$. Estos modos, que son ortogonales entre sí, describen las direcciones de máxima variabilidad en el espacio de formas. El modelo final (Ecuación~\ref{eq:ssm_model}) permite aproximar cualquier forma como una deformación de la forma media a lo largo de estos modos. La Figura~\ref{fig:varianza_explicada_pca} demuestra que una fracción sustancial de la varianza total puede ser explicada por un número reducido de modos, lo que subraya la capacidad del PCA para la reducción de dimensionalidad. Las visualizaciones de los primeros modos (Figuras~\ref{fig:modo1_pca}-\ref{fig:modo3_pca}) ofrecen una interpretación anatómica de las principales fuentes de variabilidad morfológica.

            Los resultados obtenidos, tanto en términos de las métricas de convergencia como de las visualizaciones generadas, validan la robustez de la cadena metodológica implementada para el SSM. El SSM construido encapsula de manera compacta la variabilidad de la forma pulmonar presente en el conjunto de datos, proveyendo una herramienta poderosa para análisis posteriores.

            Como perspectivas futuras, este SSM constituye una base sólida para diversas aplicaciones. En el contexto del proyecto MaShDl-CNN, se perfila como un componente crucial para la segmentación de imágenes pulmonares, donde puede actuar como un prior de forma. Adicionalmente, el modelo podría emplearse para estudios de morfometría, cuantificando diferencias entre poblaciones o correlacionando parámetros de forma con variables clínicas. La evaluación cuantitativa de la capacidad de generalización del modelo, mediante el uso de conjuntos de datos de prueba independientes, y la exploración de su rendimiento en tareas de segmentación automática, representan líneas de investigación prometedoras. Asimismo, la extensión del modelo a 3D o la incorporación de información de textura junto con la forma podrían refinar aún más su aplicabilidad clínica y de investigación.
            
            La fase de Estimación de Pose Inicial (ESL), aunque implementada con CNNs, se basa en principios de detección de características y optimización de pose. Su robustez es crucial para el correcto funcionamiento de las etapas subsecuentes. Futuras investigaciones podrían explorar métodos de ESL más sofisticados o la integración directa de la estimación de pose dentro de arquitecturas de aprendizaje profundo de extremo a extremo.
            
            La etapa MaShDL, que busca predecir los coeficientes del SSM a partir de parches de imagen, representa el núcleo de la hibridación. La validez de este enfoque depende de la capacidad de las CNNs para aprender la relación entre la apariencia local de la imagen y los parámetros globales de forma. El análisis de la precisión de la predicción de los coeficientes $b_k$ y el impacto de estos en la calidad de la segmentación final son áreas clave para la evaluación y futuras mejoras. La exploración de diferentes arquitecturas de CNN para la predicción de $b_k$, así como estrategias de entrenamiento más avanzadas, podrían conducir a mejoras significativas en el rendimiento general del pipeline.

%\include{chapters/4-chapter/2-VCO}
%\section{Amplificador de Potencia (PA)}

Un amplificador de potencia (o por sus siglas en inglés PA) es un bloque esencial en sistemas electrónicos que requieren transmitir señales con suficiente energía hacia una carga, típicamente una antena o un altavoz. Su función principal es incrementar la potencia de una señal eléctrica, manteniendo su forma y características espectrales, para que pueda ser útil en aplicaciones como transmisión de radiofrecuencia (RF), audio de alta fidelidad o sistemas de comunicación inalámbrica \cite{Libro_Razavi_Rf}.

A diferencia de los amplificadores de pequeña señal, donde la linealidad y ganancia son prioritarias, en los amplificadores de potencia también es crucial la eficiencia energética, dado que manejan niveles altos de corriente y voltaje. Esto implica un diseño cuidadoso para minimizar pérdidas, controlar el calor generado y evitar distorsiones significativas.

A nivel de implementación, en tecnologías CMOS modernas, el diseño de amplificadores de potencia debe considerar aspectos como el ancho de banda operativo, la adaptación de impedancia (típicamente a \SI{50}{\ohm}), la tensión de salida requerida, y la capacidad del diseño para integrarse con otros bloques analógicos. Los transistores de salida suelen trabajar en región de saturación para maximizar la ganancia de potencia, y se emplean técnicas como etapas diferenciales, polarización adaptativa, o redes de salida sintonizadas para mejorar el rendimiento.

\subsection{Fundamentos de amplificadores de potencia}

En un PA las principales métricas de desempeño son:

\begin{itemize}
    \item Potencia de salida ($P_{out}$)
    \item Ganancia de potencia ($G_p$)
    \item Eficiencia de potencia añadida (PAE)
    \item Linealidad 
    \item Estabilidad (factor $K>1$)
\end{itemize}

En la etapa de diseño, se utilizan herramientas como simulaciones de carga-pull o análisis de envolvente para optimizar la respuesta del circuito bajo distintas condiciones de carga y señal.

Adicionalmente, los amplificadores de potencia se clasifican comúnmente por su clase de operación: Clase A, B, AB, C, D, entre otras, cada una con un compromiso distinto entre eficiencia, linealidad y complejidad. Por ejemplo, la Clase A ofrece excelente linealidad pero baja eficiencia, mientras que la Clase AB logra un mejor equilibrio entre ambos aspectos, siendo común en transmisores \cite{Articulo_FMUWB_Trigger}.

\begin{table}[H]
    \centering
    \begin{tabular}{|c|c|c|c|c|c|}
    \hline
    \textbf{Clase} &  \begin{tabular}[c]{@{}c@{}}\textbf{Ángulo de}\\ \textbf{conducción}\end{tabular} &\begin{tabular}[c]{@{}c@{}}\textbf{Eficiencia}\\ \textbf{teórica}\end{tabular} & \textbf{Linealidad} &\begin{tabular}[c]{@{}c@{}}\textbf{Corriente}\\ \textbf{en reposo}\end{tabular} &\begin{tabular}[c]{@{}c@{}}\textbf{Aplicaciones}\\ \textbf{típicas}\end{tabular} \\
    \hline
    A  & $360^\circ$                     & $\sim$25--30\%         & Excelente      & Alta      & Audio, RF, UWB \\
    B  & $180^\circ$                     & $\sim$78.5\%                  & Baja           & Cero      & RF \\
    AB & $180^\circ$--$360^\circ$       & $\sim$35--55\%                & Buena          & Moderada  & RF, UWB \\
    C  & $<$180$^\circ$                 & $>$80\%                       & Muy baja       & Cero      & AM/FM \\
    \hline
    \end{tabular}
    \caption{Comparación entre clases de amplificadores de potencia \cite{Libro_Cripps_RFPA,Libro_Razavi_Rf}.}
    \label{tab:004:003:001}
\end{table}

Los amplificadores de potencia (PA) de tipo A y tipo AB son ampliamente utilizados en circuitos integrados CMOS debido a su balance entre linealidad, eficiencia y simplicidad de diseño. En la Tabla \ref{tab:004:003:001} se presentan los principales tipos de amplificadores de potencia. Se observa que, para aplicaciones UWB, es recomendable emplear un amplificador de tipo A o AB como diseño principal. En la figura \ref{fig:004:003:001} se observa los PA tipo A y tipo AB.

\begin{figure}[H]
    \centering
    \begin{subfigure}[b]{0.29\textwidth}
        \centering
        \includegraphics[width=\textwidth]{chapters/4-chapter/figuras/PA_CLASEA.png}
        \caption{Clase A.}
        \label{fig:004:003:001a}
    \end{subfigure}
    \hspace{0.1\textwidth} % Espacio controlado entre imágenes
    \begin{subfigure}[b]{0.32\textwidth}
        \centering
        \includegraphics[width=\textwidth]{chapters/4-chapter/figuras/PA_CLASEAB.png}
        \caption{Clase AB.}
        \label{fig:004:003:001b}
    \end{subfigure}
    \caption{Tipos de amplificadores con CMOS.}
    \label{fig:004:003:001}
\end{figure}

Un amplificador tipo A en CMOS se caracteriza por polarizar el transistor activo (generalmente un MOSFET) de forma que conduzca durante todo el ciclo de la señal de entrada, es decir, durante los $360^\circ$. Esto garantiza una alta linealidad y baja distorsión, ya que el dispositivo nunca se apaga. Sin embargo, el inconveniente principal de esta clase es su baja eficiencia energética, típicamente alrededor del $25-30\%$ \cite{Libro_Cripps_RFPA,Libro_PA_YellowBlack}, debido a la corriente de polarización constante que circula incluso en ausencia de señal de entrada. En aplicaciones de RF y, en particular, en sistemas UWB, esta clase puede ser adecuada cuando se prioriza la fidelidad de la señal sobre la eficiencia.

Por otro lado, un amplificador tipo AB representa un compromiso entre la linealidad del tipo A y la eficiencia del tipo B. En esta configuración, el transistor conduce por más de la mitad pero menos de todo el ciclo (generalmente alrededor de $180^\circ–360^\circ$). Esto se logra mediante una polarización adecuada que permite que el dispositivo se mantenga apenas encendido en reposo, reduciendo así la corriente de polarización y mejorando significativamente la eficiencia (que puede llegar a alrededor de 50-60\% \cite{Libro_Cripps_RFPA,Libro_PA_YellowBlack}), mientras se mantiene una distorsión aceptablemente baja. En tecnología CMOS, los PA tipo AB son muy populares para sistemas UWB, ya que permiten manejar variaciones rápidas de la señal con buena eficiencia y sin degradar excesivamente la calidad de la transmisión.

En resumen, en PA en CMOS para UWB, la clase A es preferida cuando la linealidad extrema es crítica, mientras que la clase AB se elige cuando se requiere un balance entre eficiencia y fidelidad, especialmente en aplicaciones donde la duración de la batería o el consumo de energía son factores importantes.

\subsection{Amplificador de potencia tipo A}

El PA tipo A (ver figura \ref{fig:004:003:001a}) se caracteriza por su operación en la cual el dispositivo activo, generalmente un transistor NMOS en tecnología CMOS, conduce durante la totalidad del ciclo de la señal de entrada, es decir, durante los $360^\circ$) \cite{Libro_PA_YellowBlack}. Esta condición se logra mediante una polarización adecuada que mantiene el transistor siempre en la región de operación activa, permitiendo una alta linealidad y minimizando la distorsión armónica, es decir mientras el voltaje de la compuerta a la fuente ($V_{GS}$) sea superior al voltaje del umbral ($V_{TH}$). 

\begin{figure}[H]
    \centering
    \includegraphics[width=5cm]{chapters/4-chapter/figuras/PA_CLASEAL.png}
    \caption{PA tipo A con inductor.}
    \label{fig:004:003:002}
\end{figure}

En la figura \ref{fig:004:003:002} se sustituye la fuente de corriente $I_b$ con un inductor $L$, a este elemento pasivo se le conoce como inductor de \textit{choke}, la principal razón del uso es el transistor CMOS, en particular, tiene una limitación en cuanto a la capacidad de generar una fuente de corriente estable durante todo el ciclo de la señal de entrada. Para sortear esta limitación, se utiliza un inductor en la etapa de carga.

En la implementación CMOS, el diseño de un amplificador tipo A enfrenta desafíos adicionales, como las limitaciones de voltaje de umbral de los dispositivos y el manejo del calor generado debido a las pérdidas continuas de potencia \cite{Libro_PA_YellowBlack}. Sin embargo, su ventaja principal radica en la excelente fidelidad de la señal, lo cual es crucial en aplicaciones de comunicación de alta precisión, tales como transmisores de ultra banda ancha (UWB), donde las características de baja distorsión son prioritarias.

En resumen, aunque los amplificadores de clase A no son óptimos en términos de eficiencia, su uso se justifica en escenarios donde la calidad de la señal es más importante que el consumo energético, particularmente en etapas de transmisión de baja potencia donde las exigencias de linealidad son estrictas.

\subsection{Amplificador de potencia tipo AB}
Un PA tipo AB combina características claves de el PA tipo A; tal como la linealidad, y del tipo B; la eficiencia. Las propiedades clave del diseño, se observa que en condiciones estáticas ambos transistores permanecen conduciendo, con corrientes iguales a $I_s = I_b$, como en un amplificador de clase A. Esto evita la distorsión por cruce. Además, el uso de un desplazamiento de nivel en continua permite transferir eficazmente las variaciones de la señal de entrada al transistor superior.

Como consecuencia, las corrientes que fluyen hacia y desde el terminal de salida superan la corriente de polarización. Esto hace que los amplificadores en clase AB no solo sean adecuados para señales de gran amplitud debido a su rápida respuesta, sino que también ofrezcan bajo consumo energético y una distorsión mínima, lo que los convierte en una solución eficiente y atractiva para aplicaciones modernas que demandan alta eficiencia energética.

En la figura \ref{fig:004:003:003} se observa un PA tipo AB con sistema de polarización del NMOS y PMOS por voltajes ($V_{b1}$ y $V_{b2}$) y una red resitiva con un capacitor en paralelo, esta hace que los transistores NMOS y PMOS estén encendidos y así el PA puedan entregar el punto de potencia adecuado para que el sistema suministre un valor adecuado hacia la carga.


\begin{figure}[H]
    \centering
    \includegraphics[width=8cm]{chapters/4-chapter/figuras/PA_CLASEABR.png}
    \caption{Amplificador de potencia tipo AB con sistema de polarización.}
    \label{fig:004:003:003}
\end{figure}

\subsection{Propuesta de diseño 1: PA tipo A}

Como propuesta de diseño se tiene que considerar las características de UWB:

\begin{enumerate}
    \item La PSD se tiene como límite de \SI{-41.3}{\decibel\milli/\mega\hertz}.
    \item La carga ($R_L$) del PA se establece como \SI{50}{\ohm}.
    \item La potencia máxima sobre la carga se establece como  \SI{-14}{\decibel m} \cite{Articulo_FMUWB_Trigger}.
    \item El ancho de banda debe ser de al menos \SI{500}{\mega\hertz}.
    \item El inductor de \textit{choke} no debe ser mayor a \SI{1}{\henry}.
    \item Etapa de transformación de potencia para maximar la corriente distribuida.
    \item Potencia de consumo no mayor 
\end{enumerate}

Suponiendo una potencia de sobre la carga de \SI{-14}{\decibel m}, se obtiene la cantidad de voltaje/corriente de salida:

\begin{eqnarray*}
P &=& 10^{\dfrac{\SI{}{[\decibel m]}}{10}}  \\
 &=& 10^{\frac{\SI{}{[\decibel m]}}{10}}  \\
 &=& 10^{\frac{-14}{10}}  \\
P &=& 0.0398x10^{-5}\\
\end{eqnarray*}
\begin{eqnarray*}
V_p &=& \sqrt{P*2*R_L}\\
&=& \sqrt{3.98x10^{-5}*2*50}\\
&=& \SI{63.1}{\milli\volt}\\
\end{eqnarray*}

Por lo tanto, el voltaje y la corriente de la onda senoidal que tendrá la carga se observa en las ecuaciones \ref{eq:004:003:001} y \ref{eq:004:003:002}, donde la corriente pico de la carga es $I_{p}=$\SI{1.262}{\milli\ampere}.

\begin{equation}
    v_{RL}(t) = 0.0631*sin(w_0t)
    \label{eq:004:003:001}
\end{equation}

\begin{equation}
    i_{RL}(t) = 0.001262*sin(w_0t)
    \label{eq:004:003:002}
\end{equation}

\begin{figure}[H]
    \centering
    \includegraphics[width=16cm]{chapters/4-chapter/figuras/corriente_cmos_pa.jpg}
    \caption{Curvas $I_D$ vs $V_{GS}$ con diferentes anchos de canal.}
    \label{fig:004:003:004}
\end{figure}

Para la selección del tamaño del ancho de canal para una tecnología de \SI{65}{\nano\meter}, se obtiene con la caracterización del transistor NMOS se observa en la figura \ref{fig:004:003:004}, donde se usa un voltaje de alimentación $V_{DD}$ de \SI{1.8}{\volt}. Se selecciono un ancho de canal $W=1.2-1.4$ \SI{}{\micro\meter} debido a que su corriente del transistor no excede los \SI{1.3}{\milli\ampere}.

En la figura \ref{fig:004:003:005} se muestra la propuesta del PA clase A implementado un transistor NMOS $M_1$ polarizado mediante un inductor RF $L_m$ el cual actua como una carga activa de alta impedancia en frecuencias altas, permitiendo el paso de corriente continua. 

%La señal de entrada (\(V_{in}\)) se aplica a la compuerta de \(M_1\), el cual opera en la región activa durante todo el ciclo de la señal.

El capacitor de acoplamiento ($C_m$) bloquea la componente de corriente directa mientras permite que la señal amplificada llegue a la etapa de salida.

\begin{figure}[H]
    \centering
    \includegraphics[width=16cm]{chapters/4-chapter/figuras/PA_CLASEAFINAL.png}
    \caption{Propuesta de amplificador de potencia clase A.}
    \label{fig:004:003:005}
\end{figure}

La red de salida está compuesta por una celda tanque resonante formada por la inductancia ($L_1$) y el capacitor ($C_1$), la cual sintoniza la frecuencia de operación deseada, maximizando la transferencia de potencia hacia la carga resistiva ($R_L=$\SI{50}{\ohm}). Esta topología garantiza una amplificación lineal con mínima distorsión, lo cual es característico de los amplificadores clase %A. Sin embargo, su eficiencia sigue siendo limitada, típicamente menor al 35\%, debido a que el transistor conduce corriente constantemente, incluso sin señal de entrada. A pesar de ello, esta arquitectura es utilizada en aplicaciones donde la linealidad y la integridad de la señal son críticas, como en etapas de salida de transmisores de radiofrecuencia de baja potencia.



%Capítulo 5
%\chapter{Resultados}
\label{cap:resultados_discusion}

Los resultados experimentales se presentan para cuantificar la precisión de la detección de puntos anatómicos utilizando el enfoque basado en modelos de apariencia.

\section{Rendimiento de los Modelos de Apariencia}
El análisis de los modelos PCA entrenados para cada punto anatómico reveló que un número relativamente pequeño de componentes principales es suficiente para capturar la mayor parte de la variabilidad en la apariencia de los parches de entrenamiento. Esto se cuantificó mediante la varianza explicada acumulada, donde típicamente los primeros 10-30 componentes explicaron alrededor del 95\% de la varianza total. Este hallazgo justifica la aplicación de técnicas de reducción de dimensionalidad para obtener representaciones compactas de la apariencia local.

% \begin{figure}[H]
%     \centering
%     % Placeholder para gráfico de Varianza Explicada Acumulada
%     \includegraphics[width=0.8\textwidth]{placeholder_variance_explained.png}
%     \caption{Varianza explicada acumulada por el número de componentes principales para un punto anatómico representativo. Muestra cómo la mayoría de la variabilidad en la apariencia de los parches es capturada por un subespacio de baja dimensión.}
%     \label{fig:variance_explained}
% \end{figure}

\section{Análisis de Precisión de Predicción en Pruebas Preliminares}
La precisión de la localización de los puntos anatómicos se evaluó en el conjunto de prueba utilizando el Error Euclidiano como métrica principal. A continuación, se presentan los resultados de las pruebas preliminares descritas en la Sección \ref{cap:diseno_experimental}, enfocándose en la Coordenada 1 y la Coordenada 2 para ilustrar el comportamiento del sistema bajo diferentes condiciones. Es importante destacar que estos resultados corresponden a experimentos realizados \textbf{sin aplicar el proceso de alineamiento de formas (GPA)}.

\begin{table}[htbp] % Se recomienda usar [htbp] en lugar de [H] para mejor flotación
    \centering
    \caption{Resultados de Error de Predicción Euclidiano (píxeles) para Coordenadas 1 y 2 en Pruebas Preliminares sin Alineamiento}
    \label{tab:preliminary_prediction_errors}
    % \small % Opcional: reduce el tamaño de fuente de toda la tabla si aún es necesario
    \begin{tabularx}{\textwidth}{@{} >{\raggedright\arraybackslash}X l r r r @{}}
        \toprule
        Prueba & Coordenada & \makecell{Media\\(píxeles)} & \makecell{Mediana\\(píxeles)} & \makecell{Desviación\\Estándar\\(píxeles)} \\
        \midrule
        \textbf{Prueba 1 (200 imágenes)} & Coord1 & 4.04 & 2.24 & 4.92 \\
        & Coord2 & 7.91 & 5.39 & 7.09 \\
        \midrule
        \textbf{Prueba 2 (400 imágenes con variaciones)} & Coord1 & 4.86 & 1.41 & 7.63 \\
        & Coord2 & 11.64 & 13.04 & 7.36 \\
        \midrule
        \textbf{Prueba 3 (400 imágenes, SAHS)} & Coord1 & 4.76 & 3.61 & 4.12 \\
        & Coord2 & 7.59 & 6.00 & 5.83 \\
        \midrule
        \textbf{Prueba 4 (800 imágenes)} & Coord1 & 5.62 & 1.71 & 8.37 \\
        & Coord2 & 8.54 & 5.24 & 8.11 \\
        \bottomrule
    \end{tabularx}
\end{table}

\subsection{Discusión de los Resultados Preliminares}
Los resultados de estas pruebas preliminares revelan varios puntos clave sobre el comportamiento del sistema sin la aplicación de alineamiento de formas:

\begin{itemize}
    \item \textbf{Impacto de la Variabilidad en la Forma (Prueba 2 vs. Prueba 1):} Al comparar la Prueba 2 con la Prueba 1, se observa un aumento significativo en el error promedio y la desviación estándar para ambas coordenadas, especialmente para la Coordenada 2. Esto se atribuye directamente a la inclusión de imágenes con traslaciones, rotaciones y escalas 'anormales' en el dataset de 400 imágenes. Este hallazgo subraya la sensibilidad del modelo de apariencia a las variaciones globales de forma cuando no se aplica una normalización geométrica previa.
    \item \textbf{Efecto de la Normalización de Contraste (Prueba 3):} La aplicación de la normalización de contraste SAHS en la Prueba 3 mostró una reducción notable en el error promedio y la desviación estándar para la Coordenada 2 (7.59 $\pm$ 5.83 píxeles) en comparación con la Prueba 2 (11.64 $\pm$ 7.36 píxeles). Este resultado es consistente con las observaciones previas de que la región cercana a la Coordenada 2 tiende a presentar contrastes muy altos, y una normalización adecuada puede mejorar la robustez de la extracción de características de apariencia en esa zona. Aunque la Coordenada 1 también mostró una ligera mejora en la desviación estándar, el impacto fue más pronunciado en la Coordenada 2.
    \item \textbf{Impacto del Tamaño del Dataset con Variaciones (Prueba 4):} La Prueba 4, que aumentó el dataset a 800 imágenes sin alineamiento pero incluyendo las imágenes con variaciones, resultó en un empeoramiento de las predicciones para ambas coordenadas en comparación con la Prueba 1 y la Prueba 3. Esto indica que simplemente aumentar el volumen de datos de entrenamiento no es suficiente si el dataset contiene una alta variabilidad de forma no normalizada. De hecho, puede introducir más ruido y complejidad al modelo de apariencia, dificultando la generalización.
\end{itemize}
Estos hallazgos preliminares resaltan la importancia crítica de la etapa de alineamiento de formas (GPA) en el pipeline propuesto, ya que su objetivo principal es mitigar la variabilidad global de traslación, rotación y escala que, como se demostró, afecta negativamente la precisión de los modelos de apariencia. Las futuras pruebas con la aplicación de alineamiento se espera que demuestren una mejora sustancial en la robustez y precisión del sistema.

% \subsection{Visualizaciones de Predicción}
% Las visualizaciones de las predicciones superpuestas en las imágenes de prueba (ej. Figura \ref{fig:prediction_visualization}) demuestran cualitativamente la capacidad del sistema para localizar los puntos. Se generaron visualizaciones que muestran la imagen redimensionada con los puntos anatómicos predichos marcados. Adicionalmente, para los puntos $\mathbf{p}_0$ y $\mathbf{p}_1$, se superpusieron construcciones geométricas derivadas de sus predicciones, como la línea principal trazada con el algoritmo de Bresenham, los puntos cuartiles en esta línea y segmentos de línea perpendiculares en estos puntos. Estas visualizaciones proporcionan una indicación visual de la alineación y la estructura de la forma predicha.

% \begin{figure}[H]
%     \centering
%     % Placeholder para figura de Visualización de Predicción
%     \includegraphics[width=0.8\textwidth]{placeholder_prediction_visualization.png}
%     \caption{Ejemplo de visualización de predicción en una imagen de prueba. Muestra la imagen redimensionada con los puntos anatómicos predichos marcados y las construcciones geométricas (línea principal, cuartiles, perpendiculares) superpuestas.}
%     \label{fig:prediction_visualization}
% \end{figure}

\section{Resultados SAHS}

Para evaluar la eficacia de SAHS (Statistical Asymmetrical Histogram Stretching) en comparación con las técnicas convencionales, realizamos una serie de experimentos utilizando diversas arquitecturas de redes neuronales convolucionales (CNN) para la clasificación de imágenes radiográficas de tórax. Los resultados se compararon con el rendimiento de las mismas arquitecturas utilizando HE y CLAHE.

La Tabla  muestra los resultados de precisión obtenidos para cada arquitectura CNN y método de preprocesamiento:

\begin{table}[H]
\centering
\begin{tabular}{|c|c|c|c|}
\hline
\textbf{Modelo CNN} & \textbf{HE} & \textbf{CLAHE} & \textbf{SAHS} \\ \hline
AlexNet & 89.4\% & 91.6\% & 92.8\% \\ \hline
Compact & 92.1\% & 91.6\% & 95.1\% \\ \hline
Enhanced & 93.1\% & 93.6\% & 95.1\% \\ \hline
ResNet-18 & 93.8\% & 92.1\% & 96.3\% \\ \hline
MobileNetV2 & 95.6\% & 96.0\% & 95.1\% \\ \hline
ResNet-50 & 93.6\% & 95.3\% & 95.8\% \\ \hline
\end{tabular}
\caption{Resultados de precisión (tasa de aciertos) de los experimentos.}
\label{tabla:resultados_precision}
\end{table}
\subsection{Análisis de Resultados}
\begin{enumerate}
    \item \textbf{Superioridad del método SAHS:} Nuestro método SAHS superó a CLAHE en la mayoría de los casos, con mejoras notables en AlexNet, Compact, Enhanced y ResNet-18.
    
    \item \textbf{Rendimiento por arquitectura:}
    \begin{itemize}
        \item ResNet-18 mostró la mayor mejora con SAHS, alcanzando un 96.3\% de precisión.
        \item MobileNetV2 y ResNet-50 tuvieron un rendimiento similar con CLAHE y SAHS, con una ligera ventaja para SAHS en ResNet-50.
    \end{itemize}
    
    \item \textbf{Consistencia:} SAHS demostró ser más consistente en la mejora de la precisión a través de diferentes arquitecturas, sugiriendo una mejor adaptabilidad a diversos modelos de CNN.
    
    \item \textbf{Eficacia en arquitecturas más simples:} La mejora fue más pronunciada en arquitecturas más simples como AlexNet y Compact, indicando que SAHS puede ser especialmente beneficioso cuando se utilizan modelos menos complejos.
\end{enumerate}


\section{Resultados Segmentación Automática}
Los resultados obtenidos de la evaluación se presentan a continuación, detallando el rendimiento para cada uno de los 10 modos de entrenamiento. Las métricas clave incluyen la mejor precisión de validación, el número de épocas entrenadas hasta la convergencia o detención temprana, la pérdida de evaluación final y la precisión de evaluación final.

\begin{table}[htbp] % Se recomienda [htbp] para mejor flotación
    \centering
    \caption{Resultados Detallados por Modo de Entrenamiento}
    \label{tab:detailed_results}
    % \small % Opcional: reduce el tamaño de fuente si es necesario
    \begin{tabularx}{\textwidth}{@{} l >{\raggedleft\arraybackslash}X >{\raggedleft\arraybackslash}X >{\raggedleft\arraybackslash}X >{\raggedleft\arraybackslash}X @{}}
        \toprule
        \textbf{Modo} & \makecell{\textbf{Mejor Precisión}\\\textbf{de Validación}} & \makecell{\textbf{Épocas}\\\textbf{Entrenadas}} & \makecell{\textbf{Pérdida de}\\\textbf{Evaluación Final}} & \makecell{\textbf{Precisión de}\\\textbf{Evaluación Final}} \\
        \midrule
        k0 & 0.8870 & 32 & 0.37881 & 0.8870 \\
        k1 & 0.7212 & 57 & 1.01564 & 0.7212 \\
        k2 & 0.7380 & 46 & 0.75595 & 0.7380 \\
        k3 & 0.7163 & 47 & 0.74818 & 0.7163 \\
        k4 & 0.7139 & 63 & 0.98626 & 0.7139 \\
        k5 & 0.6923 & 72 & 0.93360 & 0.6923 \\
        k6 & 0.6226 & 53 & 0.99640 & 0.6226 \\
        k7 & 0.6755 & 59 & 0.85054 & 0.6755 \\
        k8 & 0.6346 & 58 & 1.04511 & 0.6346 \\
        k9 & 0.6082 & 56 & 0.97770 & 0.6082 \\
        \bottomrule
    \end{tabularx}
\end{table}

El rendimiento promedio del modelo a través de los 10 modos evaluados es de una Precisión de Evaluación Final de  \textbf{DSC} = $0.7010$, con una desviación estándar de $0.0753$. La pérdida de evaluación final promedio fue de $0.86882$. Estos valores indican una variabilidad en el rendimiento entre los diferentes modos, lo cual es esperado en experimentos con particiones de datos o configuraciones ligeramente distintas.
%\chapter{Resultados y Discusión}
\label{cap:resultados_discusion}
Este capítulo presenta los resultados experimentales obtenidos de la implementación y evaluación de la metodología MaShDL-CNN Hybrid propuesta para la alineación y normalización de la forma de la región pulmonar, así como los resultados de la subsiguiente tarea de clasificación de neumonía y COVID-19. Los hallazgos se analizan en detalle, se discuten en el contexto de los objetivos de la tesis y se comparan con enfoques previos y alternativos. El objetivo es demostrar cuantitativa y cualitativamente la eficacia del sistema desarrollado y validar las hipótesis planteadas.

\section{Resultados de la Etapa de Alineación y Normalización de la Forma Pulmonar (MaShDL-CNN Hybrid)}
\label{sec:resultados_alineacion_normalizacion}
La primera fase de la evaluación se centró en el rendimiento del componente MaShDL-CNN Hybrid para la predicción de los coeficientes de forma $b_k$ del Modelo Estadístico de Forma (SSM) y la calidad de la segmentación pulmonar resultante. Estos resultados se derivan de los experimentos descritos en la Sección~\ref{sec:experimentos_optimizacion_alineacion}, utilizando los scripts \code{train_mashdl_cnn_hybrid.py} para el entrenamiento, \code{generate_predictions_cnn.py} para la predicción de $b_k$, \code{main_desdiscretizer.py} para la conversión a valores continuos, y \code{evaluate_segmentation.py} para el cálculo del coeficiente de Dice.

\subsection{Rendimiento en la Predicción de Coeficientes de Forma $b_k$}
\label{ssec:resultados_prediccion_bk}
La capacidad del modelo MaShDL-CNN Hybrid para aprender el mapeo desde la apariencia local de los parches 2D hacia los coeficientes de forma $b_k$ se midió principalmente mediante la exactitud de validación (ValAcc) en la clasificación de los bins discretizados de $b_k$. El informe de progreso (Sección 4.2) documenta varias iteraciones experimentales.

\subsubsection{Línea Base y Primeras Iteraciones (Parches $Q=25$)}
\label{sssec:resultados_bk_q25}
Los experimentos iniciales se realizaron con parches de entrada de tamaño $Q=25 \times 25$ píxeles.
\begin{itemize}
    \item \textbf{Corrida Inicial (Benchmark):} Con una arquitectura Sub-CNN de 2 capas convolucionales (filtros \code{[32,64]}), $\text{dim\_features\_per\_patch}=64$ y una DNN con capas ocultas \code{[128,64]}, se observó:
    \begin{itemize}
        \item Para 3 modos $b_k$ entrenados durante $\sim$30-60 épocas (detenido por \textit{Early Stopping}), la ValAcc promedio fue de aproximadamente \num{0.6378}.
        \item Al entrenar para 10 modos $b_k$, la ValAcc promedio general disminuyó a \num{0.5236}. Se observó un buen rendimiento para los primeros modos ($k_0-k_4$), pero un decaimiento significativo para los modos superiores ($k_5-k_9$). Esto sugiere que los parches de $Q=25$ podrían no capturar suficiente información para discriminar las variaciones más sutiles asociadas con los modos de mayor orden.
    \end{itemize}
    \item \textbf{Experimento 1 (CNN más Profunda - 3 capas):} Utilizando $Q=25$ y 5 modos $b_k$, se incrementó la profundidad de la Sub-CNN a 3 capas (filtros \code{[32,64,128]}) manteniendo $\text{dim\_features\_per\_patch}=64$ y la DNN \code{[128,64]}. Esto resultó en una mejora marginal en la ValAcc promedio para los modos $k_0-k_4$, alcanzando aproximadamente \num{0.6260} (un incremento desde $\approx \num{0.6202}$ con 2 capas para los mismos 5 modos).
    \item \textbf{Experimento 2 (CNN 3 capas + Mayor \code{dim_features_per_patch} y DNN más Grande):} Manteniendo $Q=25$ y 5 modos $b_k$ con la CNN de 3 capas, se aumentó $\text{dim\_features\_per\_patch}$ a 128 y la DNN a \code{[256,128]}. Este cambio no produjo una mejora clara, con una ValAcc promedio ligeramente inferior de \num{0.6226} y un aumento en la pérdida de validación, sugiriendo un posible sobreajuste o que la información adicional no era beneficiosa con parches de $Q=25$.
\end{itemize}
La Tabla~\ref{tab:resultados_valacc_q25} resume estos hallazgos para $Q=25$. Las curvas de entrenamiento (pérdida y exactitud vs. épocas) generadas por \code{plot_training_history} en \code{train_mashdl_cnn_hybrid.py} (ver Figura~\ref{fig:curvas_entrenamiento_q25_ejemplo}) ilustran la dinámica de aprendizaje para estas configuraciones.

\begin{tableH}
    \centering
    \caption[Resultados de Exactitud de Validación (ValAcc) para la predicción de $b_k$ con parches $Q=25$]{Resumen de la Exactitud de Validación (ValAcc) promedio obtenida para diferentes configuraciones del modelo MaShDL-CNN Hybrid utilizando parches de entrada de tamaño $Q=25 \times 25$. Los resultados se basan en el informe de progreso.}
    \label{tab:resultados_valacc_q25}
    \sisetup{round-mode=places,round-precision=4} % Para siunitx, si se desea redondear
    \begin{tabular}{@{}l c c c c S[table-format=1.4]@{}}
        \toprule
        Configuración Experimental & \shortstack{Modos $b_k$\\Entrenados} & \shortstack{Capas\\CNN} & \shortstack{Filtros CNN\\(\code{config})} & \shortstack{Dim. Feat.\\por Parche} & \shortstack{DNN Unidades\\(\code{config})} & {ValAcc Promedio} \\
        \midrule
        Benchmark (100 épocas) & 3 & 2 & \code{[32,64]} & 64 & \code{[128,64]} & 0.6378 \\
        Benchmark (100 épocas) & 10 & 2 & \code{[32,64]} & 64 & \code{[128,64]} & 0.5236 \\
        Exp. 1 (vs. 2 capas, 5 modos) & 5 & 3 & \code{[32,64,128]} & 64 & \code{[128,64]} & 0.6260 \\
        Exp. 2 & 5 & 3 & \code{[32,64,128]} & 128 & \code{[256,128]} & 0.6226 \\
        \bottomrule
    \end{tabular}
    \vspace{0.2cm}
    \footnotesize{\textit{Nota: La ValAcc es promedio sobre los modos $b_k$ entrenados para cada configuración.}}
\end{tableH}

\subsubsection{Resultados Esperados del Experimento con Parches $Q=41$ (Experimento 3)}
\label{sssec:resultados_esperados_bk_q41}
Como se detalló en la Sección~\ref{ssec:exp_q41}, el Experimento 3, que utiliza parches de $Q=41 \times 41$ y un aumento de datos más exhaustivo (\code{AUGMENTATION_PROBABILITY = 1.0}), está diseñado para abordar las limitaciones observadas con $Q=25$.
\begin{itemize}
    \item \textbf{Hipótesis:} Se espera que los parches más grandes capturen mayor información contextual y espacial, lo que debería permitir a la Sub-CNN aprender características más discriminantes, especialmente para los modos de variación $b_k$ más sutiles y de orden superior.
    \item \textbf{Métricas a Evaluar:}
    \begin{itemize}
        \item ValAcc promedio y por modo para los $b_k$ (se espera una mejora significativa, especialmente para $k \ge 4$).
        \item Curvas de entrenamiento (se observará si la convergencia es más estable o si se alcanzan mayores niveles de ValAcc).
    \end{itemize}
\end{itemize}
Los resultados de este experimento, una vez completados, se presentarán en un formato similar a la Tabla~\ref{tab:resultados_valacc_q25} y la Figura~\ref{fig:curvas_entrenamiento_q25_ejemplo}, permitiendo una comparación directa del impacto del tamaño del parche. Se anticipa que la ValAcc para la predicción de $b_k$ mejorará notablemente con $Q=41$.

\subsection{Rendimiento de la Segmentación Pulmonar (Coeficiente de Dice)}
\label{ssec:resultados_segmentacion_dice}
El rendimiento final de la etapa de alineación y normalización se mide por la calidad de la segmentación pulmonar, evaluada mediante el coeficiente de Dice (DSC) entre la máscara predicha por el sistema y la máscara Ground Truth (GT) generada consistentemente (ver Sección~\ref{sec:generacion_gt_masks_metodologia}). Los resultados de DSC son calculados por el script \code{evaluate_segmentation.py}.

\subsubsection{Resultados con Parches $Q=25$}
\label{sssec:resultados_dice_q25}
El informe de progreso (Sección 4.6) proporciona los siguientes resultados de DSC para configuraciones con parches $Q=25$ y una Sub-CNN de 2 capas, entrenada durante 100 épocas (con \textit{Early Stopping}):
\begin{itemize}
    \item \textbf{Utilizando 3 modos $b_k$ predichos para la reconstrucción:}
    \begin{itemize}
        \item Estrategia de máscara ``Dos Contornos'': DSC Promedio $\approx \num{0.652}$, DSC Mediana $\approx \num{0.764}$.
        \item Estrategia de máscara ``Convex Hull All'': DSC Promedio $\approx \num{0.674}$, DSC Mediana $\approx \num{0.793}$.
    \end{itemize}
    \item \textbf{Utilizando 10 modos $b_k$ predichos para la reconstrucción:}
    \begin{itemize}
        \item Estrategia de máscara ``Dos Contornos'': DSC Promedio $\approx \num{0.652}$, DSC Mediana $\approx \num{0.779}$.
        \item Estrategia de máscara ``Convex Hull All'': DSC Promedio $\approx \num{0.674}$, DSC Mediana $\approx \num{0.802}$.
    \end{itemize}
\end{itemize}
Estos resultados se resumen en la Tabla~\ref{tab:resultados_dice_q25_configuraciones}. Se observa que, para $Q=25$, aumentar el número de modos $b_k$ predichos de 3 a 10 no produjo una mejora significativa en el DSC promedio. Esto es consistente con la observación de que la predicción de los modos $b_k$ superiores era deficiente con parches pequeños. La estrategia de ``Convex Hull All'' consistentemente produjo un DSC promedio ligeramente superior, aunque la mediana fue similar para ambas.

\begin{tableH}
    \centering
    \caption[Resultados del Coeficiente de Dice (DSC) para segmentación pulmonar con $Q=25$]{Resumen de los scores del Coeficiente de Dice (DSC) promedio y mediana para diferentes configuraciones de reconstrucción, utilizando modelos MaShDL-CNN Hybrid entrenados con parches $Q=25$ y una Sub-CNN de 2 capas.}
    \label{tab:resultados_dice_q25_configuraciones}
    \sisetup{round-mode=places,round-precision=3}
    \begin{tabular}{@{}l c c S[table-format=1.3]@{}}
        \toprule
        Estrategia de Máscara Predicha & \shortstack{Modos $b_k$\\Usados} & Métrica DSC & {Valor} \\
        \midrule
        \multirow{2}{*}{Dos Contornos} & \multirow{2}{*}{3} & Promedio & 0.652 \\
                                     &                    & Mediana  & 0.764 \\
        \midrule
        \multirow{2}{*}{Convex Hull All} & \multirow{2}{*}{3} & Promedio & 0.674 \\
                                       &                    & Mediana  & 0.793 \\
        \midrule
        \multirow{2}{*}{Dos Contornos} & \multirow{2}{*}{10} & Promedio & 0.652 \\
                                     &                     & Mediana  & 0.779 \\
        \midrule
        \multirow{2}{*}{Convex Hull All} & \multirow{2}{*}{10} & Promedio & 0.674 \\
                                       &                     & Mediana  & 0.802 \\
        \bottomrule
    \end{tabular}
\end{tableH}

La Figura~\ref{fig:ejemplos_segmentacion_q25} muestra ejemplos visuales de segmentaciones obtenidas con $Q=25$, ilustrando tanto casos exitosos como aquellos con errores, que pueden estar relacionados con la predicción inexacta de la forma.

\subsubsection{Resultados Esperados con Parches $Q=41$ (Experimento 3)}
\label{sssec:resultados_esperados_dice_q41}
Se espera que los modelos entrenados con parches $Q=41$ (Experimento 3) produzcan una mejora sustancial en el Coeficiente de Dice.
\begin{itemize}
    \item \textbf{Hipótesis:} Si la ValAcc para la predicción de los $b_k$ (especialmente los modos superiores) mejora con $Q=41$, la forma reconstruida será más precisa, lo que se traducirá directamente en un mayor DSC.
    \item \textbf{Análisis:} Se compararán los DSC promedio y mediana obtenidos con $Q=41$ (para, e.g., 5 o 7 modos $b_k$ predichos) con los resultados de $Q=25$. También se analizará si la estrategia de ``Dos Contornos'' se beneficia más de la predicción mejorada de modos finos en comparación con ``Convex Hull All''.
\end{itemize}
Se generará una tabla similar a la Tabla~\ref{tab:resultados_dice_q25_configuraciones} para presentar estos resultados y se realizarán análisis visuales comparativos.

\subsection{Discusión de los Resultados de Alineación y Normalización}
\label{ssec:discusion_resultados_alineacion}
Los resultados preliminares con parches $Q=25$ indican que el pipeline MaShDL-CNN Hybrid es viable, logrando scores de Dice (e.g., $\approx \num{0.67}$ promedio, $\approx \num{0.80}$ mediana con Convex Hull) que representan una mejora sobre el estancamiento previo en $\num{0.57}-\num{0.61}$ con el método basado en perfiles 1D. Esto valida la hipótesis de que el uso de información 2D y CNNs es beneficioso.

Sin embargo, la dificultad para predecir los modos de variación $b_k$ de orden superior con $Q=25$ sugiere que el contexto local capturado por estos parches más pequeños es insuficiente para las deformaciones más finas. Los experimentos con la arquitectura de la CNN (aumentar profundidad o capacidad) sobre $Q=25$ no produjeron ganancias significativas, reforzando la idea de que el tamaño del parche (y por ende, la cantidad de información de entrada) es un factor limitante más crítico en este escenario.

El Experimento 3 con $Q=41$ es, por lo tanto, fundamental. Si este experimento demuestra una mejora tanto en la ValAcc de los $b_k$ como en el Coeficiente de Dice de la segmentación, confirmará que un mayor contexto local es clave. La discusión también deberá considerar el costo computacional incremental de usar parches más grandes y arquitecturas potencialmente más profundas.

Otro punto de discusión es la elección de la estrategia para generar la máscara final a partir de los 144 puntos predichos. ``Convex Hull All'' tiende a producir máscaras más suaves y puede ser más robusta a errores aislados en la predicción de landmarks individuales, lo que podría explicar su ligero mejor rendimiento promedio. Sin embargo, ``Dos Contornos'' tiene el potencial de capturar concavidades (como la incisura cardíaca) si los landmarks son muy precisos. Una mejor predicción de $b_k$ con $Q=41$ podría hacer que ``Dos Contornos'' se vuelva más competitiva o incluso superior.

Finalmente, la generación de máscaras Ground Truth consistentes (\code{generate_gt_masks_from_144pts.py}) fue un paso metodológico crucial. Asegurar que la GT se derive de la misma representación de 144 puntos que el SSM garantiza que el DSC mida fielmente la capacidad del modelo para replicar la forma definida por el SSM, en lugar de discrepar debido a definiciones de GT inconsistentes.

\section{Resultados de la Detección de Neumonía y COVID-19}
\label{sec:resultados_deteccion_enfermedad}
Esta sección presentará los resultados de la tarea final de clasificación de enfermedades (sano, neumonía, COVID-19) utilizando las regiones pulmonares segmentadas y normalizadas por el sistema MaShDL-CNN Hybrid (y los sistemas de comparación). Los clasificadores a evaluar son KNN, MLP y una CNN específica para enfermedad, como se describió en la Sección~\ref{ssec:clasificadores_a_evaluar}.
(Nota: Dado que estos resultados dependen de la finalización de la optimización del MaShDL-CNN Hybrid y de la implementación/entrenamiento de los clasificadores de enfermedad, esta sección se estructurará para presentar los tipos de resultados que se generarán. Los valores numéricos se completarán una vez que los experimentos estén concluidos.)

\subsection{Rendimiento del Clasificador Baseline (Sin Alineación/Normalización)}
\label{ssec:resultados_baseline_enfermedad}
Se presentarán los resultados de los clasificadores (KNN, MLP, CNN de enfermedad) entrenados con características extraídas directamente de las imágenes CXR originales o de una ROI definida de forma simple (ver Sección~\ref{ssec:escenarios_comparacion}).
\begin{itemize}
    \item Se reportarán métricas como Precisión, Sensibilidad, Especificidad, F1-Score y AUC para cada clasificador y cada clase.
    \item Se incluirán Matrices de Confusión y Curvas ROC.
\end{itemize}
Estos resultados servirán como el punto de partida para evaluar el beneficio de los métodos de normalización.

\begin{tableH}
    \centering
    \caption[Rendimiento de los clasificadores de enfermedad SIN alineación/normalización (Baseline)]{Métricas de rendimiento para los clasificadores KNN, MLP y CNN de enfermedad, entrenados y evaluados sobre el conjunto de prueba sin aplicar el proceso de alineación y normalización de la forma pulmonar.}
    \label{tab:resultados_clasificacion_baseline}
    \begin{tabular}{@{}l l c c c c c@{}}
        \toprule
        Clasificador & Clase & Precisión & Sensibilidad & Especificidad & F1-Score & AUC \\
        \midrule
        \multirow{3}{*}{KNN} & Sano & $P_{KNN,S}$ & $R_{KNN,S}$ & $S_{KNN,S}$ & $F1_{KNN,S}$ & $AUC_{KNN,S}$ \\
                             & Neumonía & $P_{KNN,P}$ & $R_{KNN,P}$ & $S_{KNN,P}$ & $F1_{KNN,P}$ & $AUC_{KNN,P}$ \\
                             & COVID-19 & $P_{KNN,C}$ & $R_{KNN,C}$ & $S_{KNN,C}$ & $F1_{KNN,C}$ & $AUC_{KNN,C}$ \\
        \midrule
        \multirow{3}{*}{MLP} & Sano & $P_{MLP,S}$ & $R_{MLP,S}$ & $S_{MLP,S}$ & $F1_{MLP,S}$ & $AUC_{MLP,S}$ \\
                             & Neumonía & $P_{MLP,P}$ & $R_{MLP,P}$ & $S_{MLP,P}$ & $F1_{MLP,P}$ & $AUC_{MLP,P}$ \\
                             & COVID-19 & $P_{MLP,C}$ & $R_{MLP,C}$ & $S_{MLP,C}$ & $F1_{MLP,C}$ & $AUC_{MLP,C}$ \\
        \midrule
        \multirow{3}{*}{CNN Enf.} & Sano & $P_{CNN,S}$ & $R_{CNN,S}$ & $S_{CNN,S}$ & $F1_{CNN,S}$ & $AUC_{CNN,S}$ \\
                             & Neumonía & $P_{CNN,P}$ & $R_{CNN,P}$ & $S_{CNN,P}$ & $F1_{CNN,P}$ & $AUC_{CNN,P}$ \\
                             & COVID-19 & $P_{CNN,C}$ & $R_{CNN,C}$ & $S_{CNN,C}$ & $F1_{CNN,C}$ & $AUC_{CNN,C}$ \\
        \bottomrule
    \end{tabular}
    \vspace{0.2cm}
    \footnotesize{\textit{Nota: Los valores $P, R, S, F1, AUC$ serán completados con los resultados experimentales.}}
\end{tableH}

\subsection{Rendimiento del Clasificador con Alineación MaShDL Original (Perfiles 1D)}
\label{ssec:resultados_mashdl_original_enfermedad}
Se presentarán los resultados de los mismos clasificadores utilizando características extraídas de las regiones pulmonares alineadas por el método MaShDL anterior (basado en perfiles de intensidad 1D, entrenado con \code{train_mashdl_classifiers_v12_mod6_profile_input.py}). El formato de presentación de resultados (tablas, figuras) será idéntico al del escenario baseline para facilitar la comparación.

\subsection{Rendimiento del Clasificador con Alineación MaShDL-CNN Hybrid}
\label{ssec:resultados_mashdl_cnn_hybrid_enfermedad} % Título corregido para esta subsección
Finalmente, se presentarán los resultados de los clasificadores utilizando características extraídas de las regiones pulmonares alineadas y normalizadas por el sistema MaShDL-CNN Hybrid propuesto (idealmente, con la configuración óptima obtenida del Experimento 3 con $Q=41$).
\begin{itemize}
    \item Se espera que estos resultados muestren una mejora en las métricas de clasificación en comparación con los dos escenarios anteriores.
    \item La Tabla~\ref{tab:resultados_clasificacion_mashdl_cnn} (similar a la Tabla~\ref{tab:resultados_clasificacion_baseline}, necesitará su propia etiqueta y caption) resumirá estos hallazgos.
    \item La Figura~\ref{fig:curvas_roc_comparativas} mostrará las curvas ROC para el mejor clasificador en los tres escenarios (Baseline, MaShDL Original, MaShDL-CNN Hybrid) para una clase de interés (e.g., COVID-19), permitiendo una comparación visual directa de los AUCs.
\end{itemize}

% \begin{figureH}
%     \centering
%     \includegraphics[width=0.7\linewidth]{ruta/a/tu/curvas_roc_comparativas.png} % REEMPLAZA
%     \caption[Curvas ROC comparativas para la detección de COVID-19]{Curvas ROC comparativas para la detección de COVID-19 utilizando el mejor clasificador (e.g., CNN de enfermedad) en tres escenarios: (1) Sin alineación (Baseline), (2) Con alineación MaShDL Original (Perfiles 1D), y (3) Con alineación MaShDL-CNN Hybrid (Parches 2D). Se indican los valores de AUC para cada curva.}
%     \label{fig:curvas_roc_comparativas}
% \end{figureH}

\subsection{Análisis Comparativo y Estadístico de los Clasificadores de Enfermedad}
\label{ssec:analisis_comparativo_clasificadores}
Se realizará un análisis comparativo detallado del rendimiento de los clasificadores en los tres escenarios.
\begin{itemize}
    \item Se discutirán las diferencias observadas en las métricas y se intentará atribuirlas al impacto de la normalización de forma.
    \item Se podrán aplicar pruebas de significancia estadística (e.g., test t de Student pareado, ANOVA, o pruebas no paramétricas como Wilcoxon signed-rank test) para determinar si las diferencias de rendimiento entre los escenarios son estadísticamente significativas \cite{demvsar2006statistical}.
    \item Se analizarán las matrices de confusión para identificar qué tipos de errores de clasificación son más comunes en cada escenario y si la normalización ayuda a reducir confusiones específicas entre clases.
\end{itemize}

\section{Discusión General}
\label{sec:discusion_general_resultados}
En esta sección final del capítulo, se sintetizarán e interpretarán los hallazgos clave de ambas fases de evaluación (alineación/normalización y detección de enfermedad).
\begin{itemize}
    \item \textbf{Validación de la Hipótesis:} Se discutirá en qué medida los resultados validan la hipótesis central de la tesis: que el enfoque MaShDL-CNN Hybrid mejora la alineación de forma y, consecuentemente, la detección de patologías.
    \item \textbf{Cumplimiento de Objetivos (Parcial):} Se evaluará cómo los resultados presentados contribuyen al cumplimiento de los objetivos específicos relacionados con el diseño, implementación y evaluación del método de alineación y la comparación de clasificadores.
    \item \textbf{Comparación con el Estado del Arte:} Se contrastarán cualitativamente los resultados obtenidos (tanto de Dice como de clasificación de enfermedad) con los reportados en la literatura para tareas similares (revisados en el Capítulo~\ref{cap:marco_teorico}). Se destacarán las fortalezas y posibles ventajas del método propuesto.
    \item \textbf{Impacto del Tamaño del Parche ($Q$) y Aumento de Datos:} Se profundizará en la discusión sobre cómo el Experimento 3 ($Q=41$) influyó en los resultados finales de segmentación y, potencialmente, en la clasificación de enfermedad.
    \item \textbf{Limitaciones del Estudio:} Se identificarán y discutirán las limitaciones de los experimentos realizados, por ejemplo, relacionadas con el tamaño o la diversidad de los conjuntos de datos, las elecciones arquitectónicas, o los recursos computacionales.
    \item \textbf{Análisis de Casos de Fallo:} Se presentarán y analizarán ejemplos de imágenes donde el sistema MaShDL-CNN Hybrid tuvo un rendimiento subóptimo (bajo Dice o clasificación incorrecta) para identificar posibles causas y áreas de mejora futura.
\end{itemize}
Esta discusión sentará las bases para las conclusiones y recomendaciones de trabajo futuro que se presentarán en el Capítulo~\ref{cap:conclusiones_trabajo_futuro}.

% \include{AVANCE_I/chapters/5-chapter/2-Layout}
% \include{AVANCE_I/chapters/5-chapter/3-IntegratedCircuit}

%Capitulo 6
%\chapter{Conclusiones}
\label{cap:conclusiones_trabajo_futuro}

El sistema desarrollado demuestra la viabilidad de utilizar modelos de apariencia basados en PCA en combinación con técnicas geométricas y análisis de regiones para la detección automática de un conjunto predefinido de puntos anatómicos en radiografías de tórax. La metodología propuesta aborda las variaciones en la forma y apariencia de las estructuras anatómicas mediante un proceso estructurado que incluye preprocesamiento, alineamiento de formas, análisis de regiones de búsqueda y templates, entrenamiento de modelos de apariencia y predicción localizada.

Los resultados de las pruebas preliminares, aunque enfocados en un subconjunto de puntos y sin la aplicación completa del alineamiento de formas, han sido cruciales para comprender el comportamiento del sistema. Se observó que la presencia de imágenes con variaciones significativas de traslación, rotación y escala en el dataset de entrenamiento impacta negativamente la precisión de las predicciones, aumentando el error. Esto subraya la importancia crítica de la etapa de alineamiento de formas (GPA) para normalizar la variabilidad geométrica y permitir que los modelos de apariencia se centren en las características visuales intrínsecas de los puntos. La normalización de contraste, como la aplicada con el algoritmo SAHS, demostró ser beneficiosa para reducir el error en regiones con alta variabilidad de contraste, como la Coordenada 2.

Las principales fortalezas del enfoque incluyen su base matemática sólida, así como su modularidad y la capacidad de adaptarse a diferentes puntos anatómicos. Las limitaciones identificadas incluyen la dependencia de la calidad y cantidad de las anotaciones manuales para el entrenamiento, la sensibilidad a variaciones de imagen no capturadas por los modelos de apariencia (ej. patologías severas o artefactos), y la complejidad computacional de la búsqueda exhaustiva en la región definida.

En base a los hallazgos y las observaciones, se concluye que es necesario implementar un método más robusto para atacar el problema de las imágenes muy variantes en su forma. Esto implica no solo la aplicación rigurosa de métodos de alineamiento como GPA, sino también la consideración de enfoques más avanzados.

\section{Trabajo Futuro}

\begin{itemize}
    \item La implementación y evaluación completa del proceso de alineamiento de formas (GPA) en todo el dataset para cuantificar su impacto en la reducción de errores de predicción en todos los puntos anatómicos.
    \item La incorporación de modelos de forma activa (como Active Shape Models - ASM o Active Appearance Models - AAM) que utilicen las relaciones espaciales entre los puntos para imponer restricciones y guiar la búsqueda de apariencia, mejorando la robustez y la coherencia de las predicciones.
    \item La evaluación de técnicas de aprendizaje profundo, como las Redes Neuronales Convolucionales (CNNs), para la extracción de características de apariencia más robustas y discriminativas. Estas podrían ofrecer una representación más potente que PCA, especialmente en presencia de variaciones complejas.
    \item La adaptación del sistema para manejar imágenes con resoluciones variables o diferentes modalidades de imagen, posiblemente mediante el uso de técnicas de normalización más avanzadas.
    \item La implementación de métodos de búsqueda más eficientes que no requieran evaluar exhaustivamente todos los puntos en la región de búsqueda, como la búsqueda jerárquica o el uso de clasificadores rápidos para descartar candidatos poco probables, lo que podría reducir la complejidad computacional.
\end{itemize}
En conclusión, este trabajo establece una base sólida para la detección automática de puntos anatómicos en radiografías, proporcionando una metodología matemáticamente fundamentada y un proceso funcional. Los resultados preliminares han identificado áreas clave de mejora, y las líneas de investigación futuras prometen avanzar significativamente en la robustez y precisión del sistema frente a la alta variabilidad de las imágenes médicas.

\section{Conclusiones Segmentación Automática}
Los resultados del diseño experimental demuestran que el modelo híbrido exhibe una capacidad de segmentación con una precisión promedio del $70.10\%$. La variabilidad en la precisión, indicada por una desviación estándar de $0.0753$, sugiere que el rendimiento del modelo puede fluctuar dependiendo del modo de entrenamiento o de las características específicas de los datos en cada partición.

El modo k0 se destacó significativamente con una precisión de evaluación final del $88.70\%$ y una pérdida considerablemente menor ($0.37881$) en comparación con los otros modos. Esto podría indicar que ciertas configuraciones de datos o condiciones iniciales son más favorables para la convergencia y el rendimiento óptimo del modelo.

La pérdida de evaluación final promedio de $0.86882$ es consistente con la precisión observada. La dispersión de los resultados entre los modos subraya la importancia de la evaluación en múltiples configuraciones para comprender la robustez y la generalización del modelo. Futuras investigaciones podrían enfocarse en identificar los factores que contribuyen a la alta precisión en modos específicos y en reducir la variabilidad para mejorar la consistencia del rendimiento del modelo en diversas condiciones.

Esto confirma que usar un método más robusto para la extracción de características mejora la precisión al predecir landmarks y segmentar adecuadamente, lo cual nos da una base sólida para explorar estos nuevos métodos e incorporarlos al sistema principal, además de continuar con la investigación y la experimentación para obtener un método que pueda cumplir con los objetivos de la tesis. 

\section{Conclusiones y trabajo futuro SAHS}
Los resultados demuestran que SAHS (Statistical Asymmetrical Histogram Stretching) es efectivo para mejorar la precisión de la clasificación de imágenes radiográficas de tórax en una variedad de arquitecturas CNN. En la mayoría de los casos, supera al método CLAHE convencional. La eficacia de SAHS puede atribuirse a su capacidad para adaptar la mejora de contraste a la naturaleza asimétrica de los histogramas en imágenes radiográficas de tórax, preservando información crítica para el diagnóstico mientras mejora la visibilidad de estructuras relevantes. Es importante notar que mientras SAHS mostró mejoras consistentes, la magnitud de la mejora varió entre las diferentes arquitecturas. Esto sugiere que la elección del método de preprocesamiento debe considerarse en conjunto con la selección de la arquitectura CNN para optimizar el rendimiento general del sistema de clasificación.  Las principales contribuciones de este artículo son dos: un método de ajuste de contraste para histogramas asimétricos SAHS como los que presentan las imágenes radiográficas de tórax, y la integración exitosa de SAHS con el Algoritmo Localizador de Pulmones (ALP) para una clasificación más precisa de la neumonía en imágenes radiográficas de tórax. En conclusión, estos resultados respaldan la eficacia del método propuesto SAHS en este artículo como una técnica de preprocesamiento valiosa para mejorar la precisión en la clasificación de imágenes radiográficas de tórax utilizando diversas arquitecturas de CNN. Como trabajo futuro se propone analizar aún más detalladamente la forma del histograma en este tipo de radiografías para desarrollar un método de ajuste de contraste basado completamente en la forma típica de esta clase de histogramas. Así mismo, para realizar dicho ajuste de contraste de toda la imagen podrían tomarse como referencia ciertas zonas o subregiones de brillo estable tales como la columna vertebral.

% \chapter{Publicaciones derivadas de la tesis}
% \begin{figure}[h!]
%     \centering
%     \includegraphics[width=0.7\linewidth]{Figures/imagen77.png}
%     \caption{Publicación CONACIC}
%     \label{fig:enter-label}
% \end{figure}

\begin{figure}[h!]
    \centering
    \includegraphics[width=0.75\linewidth]{Figures/imagen2.png}
    \caption{Publicación CONACIC.}
    \label{fig:conacic}
\end{figure}

\begin{figure}
    \centering
    \includegraphics[width=0.9\linewidth]{Figures/imagen.png}
    \caption{Publicación IJCOPY. Aceptada y en proceso de revisión.}
    \label{fig:ijcopy}
\end{figure}


%\chapter{Conclusiones y Trabajo Futuro}
\label{cap:conclusiones_trabajo_futuro}
Este capítulo finaliza la presente tesis resumiendo los hallazgos clave de la investigación, evaluando el cumplimiento de los objetivos planteados y destacando las contribuciones más significativas al campo del análisis de imágenes médicas para el diagnóstico de enfermedades pulmonares. Asimismo, se discuten las limitaciones inherentes al estudio realizado y se proponen líneas de investigación futuras que podrían surgir a partir de los resultados y la metodología desarrollada.

\section{Conclusiones Principales}
\label{sec:conclusiones_principales}
La investigación llevada a cabo en esta tesis se centró en el desarrollo y la evaluación de un sistema robusto para la normalización y alineación automática de la forma de la región pulmonar en radiografías de tórax, denominado MaShDL-CNN Hybrid, y en la posterior aplicación de esta normalización para mejorar la detección de neumonía y COVID-19. Las principales conclusiones derivadas de este trabajo son las siguientes:
\begin{enumerate}
    \item \textbf{Superioridad del Enfoque MaShDL-CNN Hybrid para la Alineación de Forma:} Se ha demostrado que el enfoque híbrido, que integra Modelos Estadísticos de Forma (SSM) con Redes Neuronales Convolucionales (CNNs) para analizar parches 2D, es capaz de estimar los parámetros de forma ($b_k$) con mayor precisión que los métodos previos basados en características de perfiles 1D.
    \begin{itemize}
        \item La ValAcc en la predicción de los bins de $b_k$ [fue de X\% con $Q=25$ y Y\% con $Q=41$, mejorando desde Z\% del método 1D -- \textit{completar con valores finales}], lo que indica un aprendizaje más efectivo del mapeo apariencia-forma.
        \item Consecuentemente, la calidad de la segmentación pulmonar, medida por el Coeficiente de Dice (DSC), alcanzó un valor promedio de [\textit{completar con el mejor DSC promedio, e.g., $\approx \num{0.XX}$ con $Q=41$}] y una mediana de [\textit{completar con la mejor DSC mediana, e.g., $\approx \num{0.YY}$ con $Q=41$}]. Estos valores representan una mejora sustancial sobre el estancamiento previo de $\approx \num{0.57}-\num{0.61}$ observado con el sistema MaShDL basado en perfiles 1D.
    \end{itemize}
    \item \textbf{Impacto del Tamaño del Parche y Aumento de Datos:} El tamaño del parche ($Q$) y el aumento de datos demostraron ser factores críticos.
    \begin{itemize}
        \item El Experimento 3, utilizando parches de $Q=41$ y un aumento de datos más exhaustivo, resultó en [\textit{una mejora significativa / una mejora moderada -- completar según resultados}] en la predicción de $b_k$ (especialmente para modos de variación de orden superior) y en el DSC de segmentación, en comparación con los parches de $Q=25$. Esto confirma que un mayor contexto local en los parches es beneficioso.
        \item La simple profundización de la arquitectura CNN o el aumento de la capacidad de la DNN no fueron suficientes para compensar la limitación de información con parches $Q=25$, subrayando la importancia de la calidad de la entrada.
    \end{itemize}

    \item \textbf{Beneficio de la Normalización de Forma para la Detección de Enfermedades:} La aplicación de la normalización y alineación de forma mediante el sistema MaShDL-CNN Hybrid condujo a una [\textit{mejora significativa / mejora notable / mejora modesta -- completar según resultados}] en el rendimiento de los clasificadores (KNN, MLP, CNN de enfermedad) para la detección de neumonía y COVID-19, en comparación con:
    \begin{itemize}
        \item El escenario baseline (sin alineación/normalización), donde las métricas de clasificación (Precisión, Sensibilidad, Especificidad, F1-Score, AUC) fueron [\textit{indicar valores o tendencias}].
        \item El escenario de alineación con el método MaShDL original (perfiles 1D), donde las métricas fueron [\textit{indicar valores o tendencias}].
    \end{itemize}
    Esto sugiere que la reducción de la variabilidad geométrica no patológica permite a los clasificadores enfocarse en las características verdaderamente discriminantes de las enfermedades. [\textit{Presentar el mejor clasificador y sus métricas clave con MaShDL-CNN Hybrid, e.g., ``El clasificador CNN de enfermedad, utilizando características de las regiones pulmonares normalizadas por MaShDL-CNN Hybrid ($Q=41$), alcanzó una precisión del A\%, sensibilidad del B\%, especificidad del C\%, F1-score del D\% y AUC del E\% para la detección de COVID-19...''}].

    \item \textbf{Viabilidad de un Pipeline Completo End-to-Approximation-End:} Se ha implementado y depurado con éxito un pipeline completo, desde la estimación de pose inicial, la extracción de parches 2D, el entrenamiento de modelos CNN-DNN para la forma, la predicción y desdiscretización de parámetros, la reconstrucción de la segmentación, hasta la evaluación cuantitativa del Dice y la subsiguiente clasificación de enfermedades. La creación de máscaras Ground Truth consistentes fue un paso metodológico crucial para una evaluación fiable.
\end{enumerate}
En conjunto, estos hallazgos respaldan la hipótesis central de la tesis y demuestran el potencial del enfoque MaShDL-CNN Hybrid como una herramienta valiosa para el preprocesamiento avanzado de radiografías de tórax, con un impacto positivo demostrable en tareas de diagnóstico asistido por computadora.

\section{Cumplimiento de los Objetivos}
\label{sec:cumplimiento_objetivos}
A continuación, se evalúa el grado de cumplimiento de los objetivos específicos planteados en la Sección~\ref{sec:objetivos}:
\begin{enumerate}
    \item \textbf{Diseñar, implementar y evaluar un método deformable de alineación y normalización... utilizando el enfoque MaShDL-CNN Hybrid.}
    \begin{itemize}
        \item \textbf{Cumplido.} Se diseñó e implementó el pipeline MaShDL-CNN Hybrid, que localiza la región pulmonar mediante ESL, y ajusta su forma mediante la predicción de coeficientes $b_k$ del SSM utilizando una arquitectura CNN-DNN. La evaluación se realizó mediante el Coeficiente de Dice, mostrando una mejora significativa sobre métodos previos (detallar el mejor Dice obtenido, e.g., [$\approx \num{0.XX}$]).
    \end{itemize}
    \item \textbf{Proponer un método de extracción y selección de características... que maximice la discriminación entre clases.}
    \begin{itemize}
        \item \textbf{Cumplido Parcialmente/Enfocado.}
            \begin{itemize}
                \item Para la \textit{estimación de parámetros de forma}: Se propuso y evaluó un método de extracción de características de parches 2D mediante CNNs. Los experimentos (variando $Q$, arquitectura CNN/DNN) buscaron maximizar la \code{ValAcc} en la predicción de los $b_k$.
                \item Para la \textit{discriminación entre clases de enfermedad}: El diseño experimental (Sección~\ref{sec:protocolo_clasificacion_enfermedad}) contempló la extracción de características de la región pulmonar normalizada (e.g., \textit{deep features} con una CNN pre-entrenada o una CNN de enfermedad específica). [\textit{Aquí se debe indicar qué método de extracción de características para enfermedad se implementó y si se realizó alguna ``selección'' explícita de características o si la CNN de enfermedad lo hizo implícitamente. Indicar si se logró una buena discriminación entre clases de enfermedad}].
            \end{itemize}
    \end{itemize}

    \item \textbf{Evaluar el rendimiento de diferentes clasificadores de aprendizaje supervisado (KNN, CNN, MLP) para la técnica de alineación propuesta...}
    \begin{itemize}
        \item \textbf{Cumplido.} Se evaluaron los clasificadores KNN, MLP y una CNN específica para la detección de enfermedades, utilizando como entrada las características de las regiones pulmonares procesadas por el sistema MaShDL-CNN Hybrid. [\textit{Resumir brevemente cuál clasificador obtuvo el mejor rendimiento y sus métricas clave, e.g., ``La CNN de enfermedad demostró el mejor rendimiento, seguida por el MLP y luego KNN...'' }].
    \end{itemize}

    \item \textbf{Validar el clasificador desarrollado a través de medir la precisión, sensibilidad, especificidad y además de realizar pruebas de validación cruzada...}
    \begin{itemize}
        \item \textbf{Cumplido.} Para el mejor sistema de clasificación de enfermedad (identificado en el objetivo anterior), se midieron exhaustivamente la precisión, sensibilidad, especificidad, F1-score y AUC para cada clase. Se realizaron pruebas de validación cruzada durante la optimización de hiperparámetros de los clasificadores de enfermedad para asegurar la robustez y generalización antes de la evaluación final en el conjunto de prueba. [\textit{Mencionar las métricas clave del mejor sistema validado}].
    \end{itemize}

    \item \textbf{Contrastar los resultados de clasificación... con resultados obtenidos... sin realizar el proceso de alineación propuesto.}
    \begin{itemize}
        \item \textbf{Cumplido.} Se contrastó el rendimiento del sistema de clasificación de enfermedades (con MaShDL-CNN Hybrid) con (a) un escenario baseline sin alineación y (b) un escenario con alineación utilizando el método MaShDL previo (perfiles 1D). Los resultados [\textit{demostraron una clara superioridad / una notable ventaja / una mejora medible -- completar según hallazgos}] del enfoque MaShDL-CNN Hybrid, cuantificando el beneficio de la normalización de forma avanzada. [\textit{Por ejemplo, ``La precisión de detección de COVID-19 mejoró de X\% (baseline) y Y\% (MaShDL 1D) a Z\% (MaShDL-CNN Hybrid)'' }].
    \end{itemize}

    \item \textbf{Publicación de resultados.}
    \begin{itemize}
        \item \textbf{En Progreso/Pendiente.} Esta tesis constituye el primer paso formal para la diseminación de los resultados. Se planea la preparación de [\textit{un artículo para una revista indexada / una presentación en conferencia internacional -- especificar planes}] basada en los hallazgos de esta investigación.
    \end{itemize}
\end{enumerate}
En general, se considera que los objetivos principales de la tesis han sido abordados satisfactoriamente, con la salvedad de que la publicación formal es un paso posterior a la conclusión de este documento.

\section{Impacto y Originalidad de las Contribuciones}
\label{sec:impacto_originalidad_contribuciones}
Las contribuciones de esta tesis, detalladas en la Sección~\ref{sec:contribuciones}, presentan un impacto potencial y una originalidad notables en el contexto de la investigación actual en análisis de imágenes médicas y diagnóstico asistido por computadora.
\begin{enumerate}
    \item \textbf{Avance en la Normalización de Forma Anatómica:} La principal originalidad radica en la concepción y demostración del sistema MaShDL-CNN Hybrid. Si bien los SSM y las CNNs son herramientas conocidas, su integración específica para que una CNN aprenda a predecir los parámetros de un SSM a partir de parches 2D para la normalización de la forma pulmonar en CXR representa un enfoque novedoso y eficaz. Este método aborda directamente la limitación de los descriptores 1D previamente explorados y ofrece una solución más robusta a la variabilidad geométrica, un problema persistente en el análisis de CXR \cite{van2006segmentation_cxr, zhou2021review_segmentation}. El impacto se refleja en la mejora de la precisión de la segmentación (medida por el DSC), lo cual es un prerrequisito para análisis posteriores fiables.
    \item \textbf{Mejora Cuantificable en la Detección de Enfermedades:} Al demostrar que una mejor normalización de forma conduce a un mejor rendimiento en la clasificación de neumonía y COVID-19, esta tesis aporta evidencia cuantitativa sobre la importancia de abordar la variabilidad geométrica como un paso fundamental en los pipelines de CADx. Esto tiene implicaciones directas para el diseño de futuros sistemas de IA en radiología, sugiriendo que las etapas de preprocesamiento inteligente y normalización específica del objeto pueden ser tan cruciales como la propia arquitectura del clasificador \cite{ardakani2020application, picazo2023sistema}.

    \item \textbf{Metodología Detallada y Pipeline Reproducible:} La descripción exhaustiva de la metodología, incluyendo la referencia a los scripts de código (\code{ssm_builder.py}, \code{predict_esl_pose.py}, \code{mashdl_patch_extractor.py}, \code{train_mashdl_cnn_hybrid.py}, \code{generate_predictions_cnn.py}, \code{main_desdiscretizer.py}, \code{generate_gt_masks_from_144pts.py}, \code{evaluate_segmentation.py}), y la estructura del proyecto (incluyendo el uso de contenedores Docker implícito en el informe), promueve la transparencia y facilita la reproducibilidad de la investigación, un aspecto cada vez más valorado en la ciencia computacional \cite{peng2011reproducible}.

    \item \textbf{Contribución a la Comprensión de la Estimación de Forma Basada en Apariencia:} Los experimentos que exploran el impacto del tamaño del parche ($Q$) y la complejidad de la CNN proporcionan información valiosa sobre la cantidad de contexto local necesario y la capacidad del modelo requerida para inferir parámetros de forma a partir de la apariencia de la imagen en CXR, un tipo de imagen notoriamente desafiante debido a su bajo contraste y superposición de estructuras.
\end{enumerate}
El impacto potencial de este trabajo reside en su capacidad para mejorar la fiabilidad de los sistemas de diagnóstico asistido por computadora para enfermedades pulmonares, lo que podría traducirse en diagnósticos más rápidos y precisos, especialmente en escenarios con recursos limitados o alta carga de trabajo.

\section{Limitaciones del Estudio}
\label{sec:limitaciones_estudio}
A pesar de los resultados prometedores, es importante reconocer las limitaciones inherentes a esta investigación:
\begin{enumerate}
    \item \textbf{Dependencia de la Estimación de Pose Inicial (ESL):} El rendimiento del sistema MaShDL-CNN Hybrid depende de la calidad de la estimación de pose inicial proporcionada por el módulo ESL. Errores significativos en la localización inicial (escala, traslación o rotación) pueden propagarse y afectar la extracción de parches y, consecuentemente, la predicción de los $b_k$. Aunque ESL ha demostrado ser efectivo, no es infalible, especialmente en imágenes de muy baja calidad o con anatomías muy atípicas.
    \item \textbf{Generalización del Modelo Estadístico de Forma (SSM):} La capacidad del SSM para representar la variabilidad de la forma pulmonar está limitada por la diversidad y el tamaño del conjunto de datos utilizado para su construcción. Si el SSM no captura adecuadamente ciertas morfologías pulmonares (e.g., presentes en poblaciones no representadas en el entrenamiento del SSM o causadas por patologías muy deformantes no vistas), la precisión de la alineación para esos casos puede verse comprometida.
    \item \textbf{Complejidad Computacional:} Aunque el enfoque MaShDL-CNN Hybrid mejora la precisión, el entrenamiento de múltiples modelos CNN-DNN (uno por modo $b_k$) y el procesamiento de parches 2D pueden ser computacionalmente más intensivos que los métodos basados en perfiles 1D o enfoques de segmentación directa más simples, lo que podría ser una consideración para aplicaciones en tiempo real con recursos muy limitados.
    \item \textbf{Tamaño y Diversidad de los Conjuntos de Datos para Clasificación de Enfermedad:} Si bien se utilizaron fuentes de datos públicas, la disponibilidad de conjuntos de datos a gran escala, diversificados geográficamente y con etiquetas de patologías múltiples y verificadas sigue siendo un desafío en el campo. La generalización del clasificador de enfermedades a poblaciones o variantes virales no vistas podría ser limitada. [\textit{Mencionar si hubo desequilibrio de clases y cómo se manejó, si fue una limitación}].
    \item \textbf{Interpretación de las Características de la CNN:} Aunque las CNNs son potentes, la interpretación directa de qué características específicas de los parches contribuyen a la predicción de un $b_k$ particular sigue siendo un área de investigación activa (XAI). Este trabajo se centró en el rendimiento predictivo más que en la interpretabilidad profunda de la sub-CNN.
    \item \textbf{Definición de la ROI del SSM de 144 puntos:} El SSM actual, con 144 puntos, define una ROI amplia que incluye no solo los campos pulmonares sino también el corazón y parte del mediastino. Si bien esto puede ser útil para una normalización global, las características extraídas para la clasificación de enfermedad de esta ROI amplia podrían incluir información no estrictamente parenquimatosa. Una segmentación más fina de los campos pulmonares dentro de la ROI normalizada podría ser beneficiosa.
    \item \textbf{Número de Modos $b_k$ Predichos:} Aunque se exploró la predicción de hasta 10 (o más) modos, la dificultad para predecir con precisión los modos de orden superior limitó el beneficio de usar un gran número de ellos en la reconstrucción. Un equilibrio óptimo entre el número de modos predichos y la precisión de su predicción sigue siendo un área de ajuste.
\end{enumerate}

\section{Líneas de Trabajo Futuro}
\label{sec:trabajo_futuro}
Los resultados y limitaciones de esta tesis abren varias avenidas prometedoras para investigaciones futuras:
\begin{enumerate}
    \item \textbf{Mejora de la Estimación de Pose Inicial:} Explorar el uso de métodos de detección de landmarks o regresión de pose basados en aprendizaje profundo más recientes y robustos como alternativa o complemento al sistema ESL actual, para mejorar la inicialización del MaShDL-CNN Hybrid.
    \item \textbf{Entrenamiento Conjunto o Multi-Tarea para los $b_k$:} En lugar de entrenar un modelo MaShDL-CNN Hybrid independiente para cada modo $b_k$, investigar arquitecturas que permitan la predicción conjunta de todos los coeficientes $b_k$ en un solo modelo. Esto podría permitir que el modelo aprenda correlaciones entre los modos y potencialmente mejorar la eficiencia y la regularización. Se podría explorar un enfoque de regresión multi-salida.
    \item \textbf{Incorporación de Mecanismos de Atención en la Sub-CNN:} Investigar si la adición de módulos de atención (e.g., Squeeze-and-Excitation, CBAM \cite{woo2018cbam}) a la sub-CNN que procesa los parches puede ayudarla a enfocarse en las subregiones más informativas dentro de cada parche para una mejor predicción de los $b_k$.
    \item \textbf{Modelos Estadísticos de Forma y Apariencia Combinados (SSAMs):} Extender el enfoque para no solo normalizar la forma, sino también la apariencia (textura) dentro de la región pulmonar, utilizando un Modelo Estadístico de Forma y Apariencia y adaptando el MaShDL-CNN para predecir también los parámetros del modelo de apariencia.
    \item \textbf{Segmentación Directa de los Campos Pulmonares Post-Normalización:} Una vez que la forma global está normalizada por MaShDL-CNN Hybrid, aplicar un segundo modelo de segmentación (e.g., una U-Net ligera) para delinear con precisión solo los campos pulmonares dentro de la ROI normalizada, excluyendo el corazón y el mediastino, antes de la extracción de características para la clasificación de enfermedad.
    \item \textbf{Exploración de Arquitecturas de CNN más Avanzadas para la Clasificación de Enfermedad:} Investigar el uso de arquitecturas de CNN más recientes y potentes (e.g., Transformers de Visión \cite{dosovitskiy2020image}, arquitecturas eficientes como EfficientNet \cite{tan2019efficientnet}) para la clasificación de enfermedades a partir de las regiones pulmonares normalizadas.
    \item \textbf{Validación en Conjuntos de Datos Prospectivos y Clínicos:} El paso más importante para la traslación clínica sería validar el sistema completo en conjuntos de datos prospectivos, recolectados en entornos clínicos reales y con diversidad de fuentes y poblaciones, idealmente en colaboración con radiólogos.
    \item \textbf{Integración con Información Clínica y Otros Metadatos:} Combinar las características extraídas de las imágenes con datos clínicos del paciente (e.g., edad, sexo, síntomas, comorbilidades) para desarrollar modelos de predicción de riesgo o pronóstico más comprensivos, utilizando técnicas de fusión multimodal.
    \item \textbf{Desarrollo de Herramientas de Explicabilidad (XAI):} Implementar y evaluar técnicas de XAI (e.g., Grad-CAM \cite{selvaraju2017grad}, LIME \cite{ribeiro2016should}) para visualizar qué regiones de los parches o de la imagen pulmonar normalizada son más influyentes para las predicciones del modelo de forma y del clasificador de enfermedad, respectivamente. Esto podría aumentar la confianza y la utilidad clínica.
    \item \textbf{Optimización para Despliegue:} Si el sistema demuestra alta eficacia, investigar técnicas de optimización de modelos (e.g., cuantización, poda) para reducir su tamaño y latencia, facilitando su despliegue en plataformas con recursos limitados.
\end{enumerate}
Estas líneas de trabajo futuro pueden contribuir a seguir avanzando en la precisión, robustez y aplicabilidad clínica de los sistemas de inteligencia artificial para el análisis de radiografías de tórax.

\printbibliography % Imprime la bibliografía

% \include{AVANCE_I/chapters/6-chapter/2-Recommentions}
% \include{AVANCE_I/chapters/6-chapter/3-Futurejob}

%\chapter{Anexos}
\section{Anexo A}

%\include{sections/5-Resultados}
%\include{sections/6-Conclusiones}
%\chapter{Referencias}




%\bibliographystyle{plainnat}  % estilo de la bibliografía
%\bibliography{referencias}    % nombre del archivo .bib (sin la extensión)
\addcontentsline{toc}{chapter}{Bibliografía}
\printbibliography  % imprimir la bibliografía
%
\newpage
\appendix
\newpage

\pagenumbering{gobble}

\captionsetup[figure]{font=small,skip=0pt}
\etocdepthtag.toc{mtappendix}
\etocsettagdepth{mtchapter}{none}
\etocsettagdepth{mtappendix}{subsection}
\etoctocstyle{1}{Appendix - Contents}
\tableofcontents
\pagestyle{plain}

\newpage
\pagenumbering{arabic}
\chapter{Appendix A}
\section{Appendices}
Important, but complementary material/results can be placed in appendices. This includes details of any implementation (practical work stages, etc.), large data sets, etc.”

\chapter{Appendix B}
This is only slightly related to the rest of the report, this is the information

%\include{setup/back-cover}
%TC:endignore
\end{document}
