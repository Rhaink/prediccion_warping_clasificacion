% ==============================================================================
% DOCUMENTO 03: FUNCIONES DE PÉRDIDA ESPECIALIZADAS
% Sesiones cubiertas: 2
% Proyecto: Detección de COVID-19 mediante Landmarks Anatómicos
% ==============================================================================

\documentclass[12pt,a4paper]{article}
% ==============================================================================
% PREÁMBULO LATEX PARA DOCUMENTACIÓN CIENTÍFICA DE NIVEL DOCTORAL
% Proyecto: Detección de COVID-19 mediante Landmarks Anatómicos y Warping Geométrico
% ==============================================================================

% --- Codificación y idioma ---
\usepackage[utf8]{inputenc}
\usepackage[T1]{fontenc}
\usepackage[spanish,es-tabla]{babel}

% --- Matemáticas ---
\usepackage{amsmath,amssymb,amsfonts,amsthm}
\usepackage{mathtools}
\usepackage{bm}  % Negritas en matemáticas

% --- Tablas profesionales ---
\usepackage{booktabs}
\usepackage{multirow}
\usepackage{array}
\usepackage{longtable}
\usepackage{tabularx}

% --- Figuras y gráficos ---
\usepackage{graphicx}
\usepackage{subfig}
\usepackage{float}
\usepackage{caption}

% --- Algoritmos y pseudocódigo ---
\usepackage{algorithm}
\usepackage{algorithmic}

% --- Colores (debe cargarse antes de listings) ---
\usepackage{xcolor}
\definecolor{covidred}{RGB}{220,53,69}
\definecolor{normalgreen}{RGB}{40,167,69}
\definecolor{viralblue}{RGB}{0,123,255}
\definecolor{codeblue}{RGB}{0,0,180}
\definecolor{codegray}{RGB}{128,128,128}

% --- Código fuente ---
\usepackage{listings}
\lstset{
    language=Python,
    basicstyle=\ttfamily\small,
    keywordstyle=\color{blue}\bfseries,
    commentstyle=\color{gray}\itshape,
    stringstyle=\color{red},
    numbers=left,
    numberstyle=\tiny\color{gray},
    stepnumber=1,
    numbersep=5pt,
    backgroundcolor=\color{white},
    frame=single,
    rulecolor=\color{black},
    tabsize=4,
    captionpos=b,
    breaklines=true,
    breakatwhitespace=false,
    showspaces=false,
    showstringspaces=false,
    showtabs=false,
    literate={á}{{\'a}}1 {é}{{\'e}}1 {í}{{\'i}}1 {ó}{{\'o}}1 {ú}{{\'u}}1
             {Á}{{\'A}}1 {É}{{\'E}}1 {Í}{{\'I}}1 {Ó}{{\'O}}1 {Ú}{{\'U}}1
             {ñ}{{\~n}}1 {Ñ}{{\~N}}1
             {ü}{{\"u}}1 {Ü}{{\"U}}1
             {¿}{{?`}}1 {¡}{{!`}}1
}

% --- Referencias y enlaces ---
\usepackage{hyperref}
\hypersetup{
    colorlinks=true,
    linkcolor=blue,
    filecolor=magenta,
    urlcolor=cyan,
    citecolor=blue,
    pdftitle={Documentación del Proyecto de Tesis},
    pdfauthor={},
}
\usepackage{cleveref}

% --- Bibliografía ---
\usepackage{natbib}
\bibliographystyle{plainnat}

% --- Geometría de página ---
\usepackage[a4paper,margin=2.5cm]{geometry}

% --- Espaciado ---
\usepackage{setspace}
\onehalfspacing

% --- Encabezados y pies ---
\usepackage{fancyhdr}
\pagestyle{fancy}
\fancyhf{}
\rhead{\rightmark}
\lhead{\leftmark}
\cfoot{\thepage}

% --- Entornos personalizados ---
\theoremstyle{definition}
\newtheorem{definicion}{Definición}[section]
\newtheorem{proposicion}{Proposición}[section]
\newtheorem{teorema}{Teorema}[section]
\newtheorem{lema}{Lema}[section]
\newtheorem{corolario}{Corolario}[section]

\theoremstyle{remark}
\newtheorem{observacion}{Observación}[section]
\newtheorem{nota}{Nota}[section]
\newtheorem{hipotesis}{Hipótesis}[section]

% --- Comandos personalizados ---
% Vectores y matrices
\newcommand{\vect}[1]{\mathbf{#1}}
\newcommand{\mat}[1]{\mathbf{#1}}

% Operadores
\DeclareMathOperator*{\argmin}{arg\,min}
\DeclareMathOperator*{\argmax}{arg\,max}
\DeclareMathOperator{\sgn}{sgn}
\DeclareMathOperator{\diag}{diag}
\DeclareMathOperator{\trace}{tr}

% Normas y productos
\newcommand{\norm}[1]{\left\|#1\right\|}
\newcommand{\abs}[1]{\left|#1\right|}
\newcommand{\inner}[2]{\langle #1, #2 \rangle}

% Conjuntos
\newcommand{\R}{\mathbb{R}}
\newcommand{\N}{\mathbb{N}}
\newcommand{\Z}{\mathbb{Z}}

% Espacios de imágenes y landmarks
\newcommand{\imgspace}{\mathcal{I}}
\newcommand{\landmarkspace}{\mathcal{L}}
\newcommand{\classspace}{\mathcal{Y}}

% Landmarks específicos
\newcommand{\landmark}[1]{L_{#1}}
\newcommand{\landmarkpair}[2]{(L_{#1}, L_{#2})}

% Funciones de pérdida
\newcommand{\loss}{\mathcal{L}}
\newcommand{\wingloss}{\mathcal{L}_{\text{wing}}}
\newcommand{\mseloss}{\mathcal{L}_{\text{MSE}}}

% Métricas
\newcommand{\accuracy}{\text{Acc}}
\newcommand{\precision}{\text{Prec}}
\newcommand{\recall}{\text{Rec}}
\newcommand{\fscore}{F_1}

% Abreviaciones
\newcommand{\ie}{\textit{i.e.}}
\newcommand{\eg}{\textit{e.g.}}
\newcommand{\etal}{\textit{et al.}}
\newcommand{\etc}{\textit{etc.}}

% Referencias a archivos del proyecto
\newcommand{\archivo}[1]{\texttt{#1}}
\newcommand{\funcion}[1]{\texttt{#1()}}
\newcommand{\clase}[1]{\texttt{#1}}
\newcommand{\parametro}[1]{\texttt{#1}}

% Unidades
\newcommand{\px}{\,\text{px}}
\newcommand{\epoch}{\,\text{época}}
\newcommand{\epochs}{\,\text{épocas}}

% --- Información del documento ---
\newcommand{\proyectotitulo}{Detección de COVID-19 en Radiografías de Tórax mediante Landmarks Anatómicos y Normalización Geométrica}
\newcommand{\proyectosubtitulo}{Documentación Científica del Proceso de Desarrollo}

% --- Formato de secciones ---
\usepackage{titlesec}
\titleformat{\section}
  {\normalfont\Large\bfseries}{\thesection}{1em}{}
\titleformat{\subsection}
  {\normalfont\large\bfseries}{\thesubsection}{1em}{}
\titleformat{\subsubsection}
  {\normalfont\normalsize\bfseries}{\thesubsubsection}{1em}{}

% --- Notas al margen para figuras sugeridas ---
\usepackage{marginnote}
\newcommand{\figurasugerida}[1]{\marginnote{\scriptsize\textcolor{blue}{[Figura: #1]}}}

% --- Cajas para resultados importantes ---
\usepackage{tcolorbox}
\newtcolorbox{resultadoimportante}[1][]{
    colback=green!5!white,
    colframe=green!75!black,
    fonttitle=\bfseries,
    title=Resultado Importante,
    #1
}

\newtcolorbox{hallazgo}[1][]{
    colback=blue!5!white,
    colframe=blue!75!black,
    fonttitle=\bfseries,
    title=Hallazgo,
    #1
}

\newtcolorbox{metodologia}[1][]{
    colback=yellow!5!white,
    colframe=yellow!75!black,
    fonttitle=\bfseries,
    title=Metodología,
    #1
}

\newtcolorbox{figuradescripcion}[1][]{
    colback=gray!5!white,
    colframe=gray!75!black,
    fonttitle=\bfseries,
    title=Descripción de Figura,
    #1
}

% ==============================================================================
% FIN DEL PREÁMBULO
% ==============================================================================


\title{\textbf{Funciones de Pérdida Especializadas para\\Localización de Landmarks}\\[0.5em]
\large Documentación del Proceso de Desarrollo - Sesión 2}
\author{Proyecto de Tesis Doctoral\\Detección de COVID-19 mediante Landmarks Anatómicos}
\date{Noviembre 2024}

\begin{document}
\maketitle

\begin{abstract}
Este documento presenta el diseño e implementación de funciones de pérdida especializadas para la tarea de regresión de landmarks anatómicos. Se analiza en detalle Wing Loss y su comportamiento adaptativo, junto con restricciones geométricas adicionales: Central Alignment Loss para alineación del eje central y Soft Symmetry Loss para simetría bilateral con margen de tolerancia. Se incluyen análisis de gradientes, justificación de parámetros, y comparaciones experimentales que fundamentan la selección final de Wing Loss como función principal.
\end{abstract}

\tableofcontents
\newpage

% ==============================================================================
\section{Introducción}
% ==============================================================================

\subsection{Motivación}

La selección de la función de pérdida es crítica en problemas de regresión de landmarks. A diferencia de la clasificación, donde la cross-entropy es el estándar, la regresión de coordenadas espaciales presenta desafíos específicos:

\begin{enumerate}
    \item \textbf{Sensibilidad a errores pequeños}: Errores de pocos píxeles son significativos clínicamente
    \item \textbf{Robustez a outliers}: Algunas anotaciones pueden contener errores
    \item \textbf{Gradientes informativos}: El entrenamiento debe progresar incluso cerca del óptimo
\end{enumerate}

\subsection{Funciones Candidatas}

Se evaluaron las siguientes funciones de pérdida:

\begin{itemize}
    \item Mean Squared Error (MSE / L2 Loss)
    \item Mean Absolute Error (MAE / L1 Loss)
    \item Smooth L1 Loss (Huber Loss)
    \item \textbf{Wing Loss} (propuesta por Feng \etal, CVPR 2018)
\end{itemize}

% ==============================================================================
\section{Análisis de Funciones Estándar}
% ==============================================================================

\subsection{Mean Squared Error (MSE)}

\begin{definicion}[MSE Loss]
Para predicciones $\hat{\vect{p}} \in \R^{K \times 2}$ y ground truth $\vect{p} \in \R^{K \times 2}$:
\begin{equation}
\loss_{\text{MSE}} = \frac{1}{2K} \sum_{k=1}^{K} \norm{\hat{\vect{p}}_k - \vect{p}_k}_2^2
\label{eq:mse}
\end{equation}
\end{definicion}

\subsubsection{Propiedades de MSE}

\begin{itemize}
    \item \textbf{Gradiente}: $\nabla_{\hat{p}} \loss_{\text{MSE}} = \hat{\vect{p}} - \vect{p}$
    \item \textbf{Comportamiento}: Cuadrático - penaliza fuertemente errores grandes
    \item \textbf{Problema}: Gradientes pequeños cerca del óptimo dificultan la convergencia fina
\end{itemize}

\begin{equation}
\frac{\partial \loss_{\text{MSE}}}{\partial \hat{p}} = \hat{p} - p \xrightarrow[\hat{p} \to p]{} 0
\end{equation}

\begin{observacion}
Cuando las predicciones se acercan al ground truth, los gradientes de MSE se vuelven muy pequeños, ralentizando la convergencia en las etapas finales del entrenamiento.
\end{observacion}

\subsection{Mean Absolute Error (MAE / L1)}

\begin{definicion}[L1 Loss]
\begin{equation}
\loss_{\text{L1}} = \frac{1}{K} \sum_{k=1}^{K} \norm{\hat{\vect{p}}_k - \vect{p}_k}_1
\label{eq:l1}
\end{equation}
\end{definicion}

\subsubsection{Propiedades de L1}

\begin{itemize}
    \item \textbf{Gradiente}: $\nabla_{\hat{p}} \loss_{\text{L1}} = \text{sgn}(\hat{\vect{p}} - \vect{p})$
    \item \textbf{Comportamiento}: Lineal - gradiente constante independiente del error
    \item \textbf{Problema}: No diferenciable en el origen, gradiente constante no adaptativo
\end{itemize}

\subsection{Smooth L1 Loss (Huber)}

\begin{definicion}[Smooth L1 Loss]
\begin{equation}
\loss_{\text{SmoothL1}}(x) =
\begin{cases}
\frac{x^2}{2\beta} & \text{si } |x| < \beta \\
|x| - \frac{\beta}{2} & \text{en otro caso}
\end{cases}
\label{eq:smooth_l1}
\end{equation}
donde $\beta$ es el umbral de transición (típicamente $\beta = 1$).
\end{definicion}

\subsubsection{Propiedades de Smooth L1}

\begin{itemize}
    \item Cuadrático para errores pequeños (como MSE)
    \item Lineal para errores grandes (como L1)
    \item Diferenciable en todo el dominio
    \item \textbf{Problema}: No optimizado para localización de landmarks
\end{itemize}

% ==============================================================================
\section{Wing Loss}
% ==============================================================================

\subsection{Motivación Original}

Wing Loss fue propuesta por Feng \etal\ (CVPR 2018) específicamente para detección de landmarks faciales. La idea clave es proporcionar:

\begin{enumerate}
    \item Mayor sensibilidad a errores pequeños (comportamiento logarítmico)
    \item Robustez a errores grandes (comportamiento lineal)
    \item Gradientes no decrecientes cerca del óptimo
\end{enumerate}

\subsection{Definición Matemática}

\begin{definicion}[Wing Loss]
Para un error escalar $x = |\hat{p} - p|$:
\begin{equation}
\wingloss(x) =
\begin{cases}
\omega \ln\left(1 + \frac{|x|}{\epsilon}\right) & \text{si } |x| < \omega \\
|x| - C & \text{en otro caso}
\end{cases}
\label{eq:wing_loss}
\end{equation}
donde:
\begin{itemize}
    \item $\omega$: umbral que define el límite de la región no lineal
    \item $\epsilon$: parámetro de curvatura de la región logarítmica
    \item $C$: constante de continuidad
\end{itemize}
\end{definicion}

\subsection{Constante de Continuidad}

Para garantizar continuidad en $x = \omega$, la constante $C$ se calcula como:

\begin{equation}
C = \omega - \omega \ln\left(1 + \frac{\omega}{\epsilon}\right)
\label{eq:wing_C}
\end{equation}

\begin{proof}
La continuidad requiere que ambas ramas sean iguales en $x = \omega$:
\begin{align}
\omega \ln\left(1 + \frac{\omega}{\epsilon}\right) &= \omega - C \\
C &= \omega - \omega \ln\left(1 + \frac{\omega}{\epsilon}\right)
\end{align}
\end{proof}

\subsection{Análisis de Gradientes}

El gradiente de Wing Loss:

\begin{equation}
\frac{\partial \wingloss}{\partial x} =
\begin{cases}
\frac{\omega}{\epsilon + |x|} \cdot \text{sgn}(x) & \text{si } |x| < \omega \\
\text{sgn}(x) & \text{en otro caso}
\end{cases}
\label{eq:wing_gradient}
\end{equation}

\begin{hallazgo}
En la región cercana al óptimo ($|x| \to 0$):
\begin{equation}
\lim_{x \to 0} \frac{\partial \wingloss}{\partial x} = \frac{\omega}{\epsilon} \neq 0
\end{equation}
A diferencia de MSE ($\lim_{x \to 0} \nabla = 0$), Wing Loss mantiene gradientes informativos incluso para errores muy pequeños.
\end{hallazgo}

\begin{figure}[H]
\centering
% [Figura sugerida: Comparación de Wing Loss vs MSE vs L1]
\caption{Comparación de funciones de pérdida. Wing Loss (azul) presenta comportamiento logarítmico para errores pequeños y lineal para errores grandes, combinando las ventajas de MSE y L1.}
\label{fig:loss_comparison}
\end{figure}

\subsection{Parámetros para Coordenadas Normalizadas}

En este proyecto, las coordenadas están normalizadas al rango $[0, 1]$. Los parámetros originales de Wing Loss (diseñados para píxeles) deben escalarse:

\begin{table}[H]
\centering
\caption{Parámetros de Wing Loss}
\label{tab:wing_params}
\begin{tabular}{lccc}
\toprule
\textbf{Parámetro} & \textbf{Original (px)} & \textbf{Normalizado} & \textbf{Fórmula} \\
\midrule
$\omega$ & 10 & 0.0446 & $10 / 224$ \\
$\epsilon$ & 2 & 0.0089 & $2 / 224$ \\
$C$ & - & 0.0297 & Calculado \\
\bottomrule
\end{tabular}
\end{table}

\subsection{Implementación}

\begin{lstlisting}[caption={Implementación de Wing Loss}]
class WingLoss(nn.Module):
    def __init__(self, omega=10.0, epsilon=2.0, image_size=224):
        super().__init__()
        # Normalizar parametros para coordenadas [0, 1]
        self.omega = omega / image_size
        self.epsilon = epsilon / image_size
        # Calcular constante de continuidad
        self.C = self.omega - self.omega * math.log(
            1 + self.omega / self.epsilon
        )

    def forward(self, pred, target):
        # pred, target: (batch, 30) coordenadas normalizadas
        diff = torch.abs(pred - target)

        # Mascara para region logaritmica
        mask = diff < self.omega

        # Calcular loss por region
        loss_log = self.omega * torch.log(1 + diff / self.epsilon)
        loss_lin = diff - self.C

        # Combinar con mascara
        loss = torch.where(mask, loss_log, loss_lin)

        return loss.mean()
\end{lstlisting}

% ==============================================================================
\section{Restricciones Geométricas}
% ==============================================================================

Además de Wing Loss, se implementaron restricciones geométricas basadas en el análisis exploratorio del dataset.

\subsection{Central Alignment Loss}

\subsubsection{Motivación}

El análisis del Ground Truth reveló que los puntos centrales $L_9$, $L_{10}$, $L_{11}$ están casi exactamente sobre el eje $L_1$-$L_2$ (error promedio: 1.37 px). Esta restricción puede incorporarse explícitamente.

\begin{definicion}[Central Alignment Loss]
Penaliza la distancia perpendicular de los puntos centrales al eje:
\begin{equation}
\loss_{\text{central}} = \frac{1}{3} \sum_{k \in \{9, 10, 11\}} d_\perp(\hat{\vect{p}}_k, \overrightarrow{L_1 L_2})
\label{eq:central_loss}
\end{equation}
\end{definicion}

\subsubsection{Cálculo de Distancia Perpendicular}

Para un punto $\vect{p}$ y una línea definida por $\vect{a}$ (punto) y $\vect{u}$ (dirección unitaria):

\begin{equation}
d_\perp(\vect{p}, \text{línea}) = \norm{\vect{p} - \vect{a} - \text{proj}_{\vect{u}}(\vect{p} - \vect{a})}_2
\end{equation}

donde la proyección es:

\begin{equation}
\text{proj}_{\vect{u}}(\vect{v}) = (\vect{v} \cdot \vect{u}) \vect{u}
\end{equation}

\begin{algorithm}[H]
\caption{Central Alignment Loss}
\label{alg:central_loss}
\begin{algorithmic}[1]
\REQUIRE Landmarks predichos $\hat{\vect{L}} = \{\hat{\vect{p}}_k\}_{k=1}^{15}$
\ENSURE Pérdida de alineación central
\STATE $\vect{a} \leftarrow \hat{\vect{p}}_1$ \COMMENT{Punto del eje}
\STATE $\vect{u} \leftarrow (\hat{\vect{p}}_2 - \hat{\vect{p}}_1) / \norm{\hat{\vect{p}}_2 - \hat{\vect{p}}_1}$ \COMMENT{Dirección unitaria}
\STATE $\loss \leftarrow 0$
\FOR{$k \in \{9, 10, 11\}$}
    \STATE $\vect{v} \leftarrow \hat{\vect{p}}_k - \vect{a}$
    \STATE $\text{proj} \leftarrow (\vect{v} \cdot \vect{u}) \cdot \vect{u}$
    \STATE $d_\perp \leftarrow \norm{\vect{v} - \text{proj}}$
    \STATE $\loss \leftarrow \loss + d_\perp$
\ENDFOR
\RETURN $\loss / 3$
\end{algorithmic}
\end{algorithm}

\subsection{Soft Symmetry Loss}

\subsubsection{Motivación}

Los pares bilaterales presentan simetría aproximada respecto al eje central. Sin embargo, el Ground Truth revela asimetría natural de 5.5-7.9 px. Por tanto, no se debe forzar simetría perfecta.

\begin{definicion}[Soft Symmetry Loss]
Penaliza \textbf{solo} asimetrías que excedan un margen $m$:
\begin{equation}
\loss_{\text{sym}} = \frac{1}{|\mathcal{P}|} \sum_{(l, r) \in \mathcal{P}} \max\left(0, |d_l - d_r| - m\right)^2
\label{eq:soft_sym}
\end{equation}
donde:
\begin{itemize}
    \item $\mathcal{P} = \{(3,4), (5,6), (7,8), (12,13), (14,15)\}$ son los pares bilaterales
    \item $d_l$, $d_r$ son las distancias perpendiculares al eje
    \item $m$ es el margen de tolerancia
\end{itemize}
\end{definicion}

\subsubsection{Selección del Margen}

Basado en el análisis del GT:

\begin{table}[H]
\centering
\caption{Asimetría en Ground Truth y selección de margen}
\label{tab:margin_selection}
\begin{tabular}{lcc}
\toprule
\textbf{Par} & \textbf{Asimetría GT (px)} & \textbf{Percentil 75} \\
\midrule
$(L_3, L_4)$ & $5.51 \pm 4.58$ & 7.8 \\
$(L_5, L_6)$ & $5.55 \pm 5.20$ & 8.2 \\
$(L_7, L_8)$ & $6.82 \pm 5.85$ & 9.1 \\
$(L_{12}, L_{13})$ & $6.15 \pm 5.42$ & 8.7 \\
$(L_{14}, L_{15})$ & $7.89 \pm 6.84$ & 11.2 \\
\midrule
\textbf{Margen seleccionado} & \multicolumn{2}{c}{$m = 6$ px $\approx 0.027$ normalizado} \\
\bottomrule
\end{tabular}
\end{table}

\begin{observacion}
El margen $m = 6$ px permite que el modelo replique la asimetría natural del GT sin penalizar predicciones anatómicamente correctas.
\end{observacion}

\subsection{Combined Loss}

\begin{definicion}[Combined Landmark Loss]
La pérdida combinada pondera los tres términos:
\begin{equation}
\loss_{\text{total}} = \wingloss + \alpha \cdot \loss_{\text{central}} + \beta \cdot \loss_{\text{sym}}
\label{eq:combined_loss}
\end{equation}
donde $\alpha$ y $\beta$ son hiperparámetros de ponderación.
\end{definicion}

\subsubsection{Análisis de Ponderaciones}

\begin{table}[H]
\centering
\caption{Experimentos de ponderación de pérdidas}
\label{tab:loss_weights}
\begin{tabular}{cccc}
\toprule
$\alpha$ & $\beta$ & \textbf{Error Val (px)} & \textbf{Observación} \\
\midrule
0 & 0 & 9.08 & Solo Wing Loss \\
1.0 & 0 & 9.12 & Central alignment \\
0 & 0.5 & 9.21 & Soft symmetry \\
1.0 & 0.5 & 9.34 & Combinado \\
\bottomrule
\end{tabular}
\end{table}

\begin{hallazgo}
Las restricciones geométricas adicionales \textbf{no mejoraron} el rendimiento. El modelo con Wing Loss solo logra el mejor resultado. Esto sugiere que las restricciones geométricas son \textbf{aprendidas implícitamente} por el modelo sin necesidad de pérdidas explícitas.
\end{hallazgo}

% ==============================================================================
\section{Weighted Wing Loss}
% ==============================================================================

\subsection{Motivación}

Los landmarks tienen diferentes niveles de dificultad. Los puntos costofrénicos ($L_{14}$, $L_{15}$) son más difíciles que los puntos centrales. Una estrategia es ponderar la pérdida por landmark.

\begin{definicion}[Weighted Wing Loss]
\begin{equation}
\loss_{\text{weighted}} = \frac{1}{\sum_k w_k} \sum_{k=1}^{K} w_k \cdot \wingloss(\hat{\vect{p}}_k - \vect{p}_k)
\label{eq:weighted_wing}
\end{equation}
donde $w_k$ es el peso del landmark $k$.
\end{definicion}

\subsection{Estrategias de Ponderación}

\begin{table}[H]
\centering
\caption{Estrategias de ponderación evaluadas}
\label{tab:weighting_strategies}
\begin{tabular}{lcc}
\toprule
\textbf{Estrategia} & \textbf{Error Val (px)} & \textbf{Descripción} \\
\midrule
Uniforme ($w_k = 1$) & \textbf{9.08} & Todos igual peso \\
Inverso al error & 9.15 & $w_k \propto 1/\epsilon_k$ \\
Prioridad a difíciles & 9.22 & Más peso a $L_{14}$, $L_{15}$ \\
Prioridad a eje & 9.11 & Más peso a $L_1$, $L_2$ \\
\bottomrule
\end{tabular}
\end{table}

\begin{resultadoimportante}
La ponderación uniforme resultó óptima. Las estrategias de ponderación no mejoraron el rendimiento, posiblemente porque:
\begin{enumerate}
    \item El modelo ya aprende a balancear la atención entre landmarks
    \item Ponderar ciertos landmarks puede desbalancear el aprendizaje global
    \item La dificultad de un landmark no implica que requiera más énfasis
\end{enumerate}
\end{resultadoimportante}

% ==============================================================================
\section{Comparación Experimental}
% ==============================================================================

\subsection{Configuración Experimental}

Todas las funciones de pérdida se evaluaron con:
\begin{itemize}
    \item Misma arquitectura: ResNet18Landmarks (sin CoordAttention ni DeepHead)
    \item Mismo optimizer: Adam, lr=1e-3 (fase 1), 2e-4 (fase 2)
    \item Misma semilla: random\_seed=42
    \item 50 épocas de entrenamiento
\end{itemize}

\subsection{Resultados}

\begin{table}[H]
\centering
\caption{Comparación de funciones de pérdida}
\label{tab:loss_comparison_results}
\begin{tabular}{lcccc}
\toprule
\textbf{Función} & \textbf{Error Train (px)} & \textbf{Error Val (px)} & \textbf{Error Test (px)} & $\Delta$ \textbf{vs MSE} \\
\midrule
MSE (L2) & 10.21 & 11.34 & 11.52 & baseline \\
MAE (L1) & 9.87 & 10.87 & 11.01 & -4.4\% \\
Smooth L1 & 9.42 & 10.12 & 10.34 & -10.2\% \\
\textbf{Wing Loss} & \textbf{8.45} & \textbf{9.08} & \textbf{9.21} & \textbf{-20.0\%} \\
\bottomrule
\end{tabular}
\end{table}

\begin{figure}[H]
\centering
% [Figura sugerida: Curvas de convergencia por función de pérdida]
\caption{Evolución del error de validación durante el entrenamiento para diferentes funciones de pérdida. Wing Loss converge más rápido y alcanza mejor resultado final.}
\label{fig:loss_convergence}
\end{figure}

\subsection{Análisis por Landmark}

\begin{table}[H]
\centering
\caption{Error por landmark con Wing Loss vs MSE}
\label{tab:error_by_landmark}
\begin{tabular}{lccc}
\toprule
\textbf{Landmark} & \textbf{MSE (px)} & \textbf{Wing (px)} & \textbf{Mejora} \\
\midrule
$L_1$ (Apex traqueal) & 9.24 & 7.12 & -22.9\% \\
$L_2$ (Base diafrag.) & 12.87 & 10.34 & -19.7\% \\
$L_9$ (Central sup.) & 8.12 & 6.45 & -20.6\% \\
$L_{10}$ (Central med.) & 7.89 & 6.21 & -21.3\% \\
$L_{11}$ (Central inf.) & 8.45 & 6.78 & -19.8\% \\
$L_{14}$ (Costofrénico izq.) & 15.67 & 12.89 & -17.7\% \\
$L_{15}$ (Costofrénico der.) & 15.23 & 12.54 & -17.7\% \\
\bottomrule
\end{tabular}
\end{table}

\begin{hallazgo}
Wing Loss mejora consistentemente todos los landmarks, con mayor impacto relativo en los landmarks centrales (20-23\%) que en los costofrénicos (17-18\%). Esto sugiere que Wing Loss es particularmente efectivo cuando los errores son pequeños.
\end{hallazgo}

% ==============================================================================
\section{Análisis Teórico de Gradientes}
% ==============================================================================

\subsection{Magnitud de Gradientes}

Para entender por qué Wing Loss es superior, analizamos la magnitud de gradientes en función del error:

\begin{table}[H]
\centering
\caption{Magnitud de gradientes por función de pérdida}
\label{tab:gradient_magnitude}
\begin{tabular}{lccc}
\toprule
\textbf{Error (px)} & $|\nabla \text{MSE}|$ & $|\nabla \text{L1}|$ & $|\nabla \text{Wing}|$ \\
\midrule
0.1 & 0.00045 & 1.0 & 0.89 \\
1.0 & 0.0045 & 1.0 & 0.83 \\
5.0 & 0.022 & 1.0 & 0.69 \\
10.0 & 0.045 & 1.0 & 1.0 \\
20.0 & 0.089 & 1.0 & 1.0 \\
\bottomrule
\end{tabular}
\end{table}

\begin{observacion}
Para errores pequeños (< 5 px), Wing Loss mantiene gradientes significativos ($|\nabla| > 0.69$), mientras que MSE tiene gradientes muy pequeños ($|\nabla| < 0.022$). Esto explica la convergencia más rápida y precisa de Wing Loss.
\end{observacion}

\subsection{Ratio de Gradientes}

El ratio $|\nabla \text{Wing}| / |\nabla \text{MSE}|$ crece dramáticamente para errores pequeños:

\begin{equation}
\frac{|\nabla \wingloss|}{|\nabla \loss_{\text{MSE}}|} = \frac{\omega / (\epsilon + |x|)}{|x|} \approx \frac{\omega}{\epsilon \cdot |x|} \quad \text{para } |x| \ll \epsilon
\end{equation}

Para $|x| = 0.1$ px (error muy pequeño):
\begin{equation}
\frac{|\nabla \wingloss|}{|\nabla \loss_{\text{MSE}}|} \approx \frac{10}{2 \cdot 0.1} = 50
\end{equation}

\begin{resultadoimportante}
Wing Loss proporciona gradientes hasta 50 veces mayores que MSE para errores muy pequeños, permitiendo refinamiento continuo incluso cuando el modelo está cerca del óptimo.
\end{resultadoimportante}

% ==============================================================================
\section{Figuras Sugeridas}
% ==============================================================================

\subsection{Figura 3.1: Comparación Wing Loss vs L2 vs L1}

\textbf{Descripción}: Gráfico mostrando las tres funciones de pérdida en función del error.

\textbf{Elementos}:
\begin{itemize}
    \item Eje X: Error absoluto (0 a 20 px)
    \item Eje Y: Valor de pérdida
    \item Tres curvas: MSE (roja, cuadrática), L1 (verde, lineal), Wing (azul, logarítmica-lineal)
    \item Línea vertical en $\omega = 10$ px marcando transición de Wing
    \item Región sombreada para errores típicos (0-8 px)
\end{itemize}

\subsection{Figura 3.2: Gradientes de Wing Loss}

\textbf{Descripción}: Gráfico de la magnitud del gradiente en función del error.

\textbf{Elementos}:
\begin{itemize}
    \item Eje X: Error absoluto (log scale, 0.01 a 50 px)
    \item Eje Y: Magnitud del gradiente
    \item Comparación MSE vs Wing
    \item Anotación del ratio de gradientes
\end{itemize}

\subsection{Figura 3.3: Ilustración de Restricciones Geométricas}

\textbf{Descripción}: Diagrama mostrando Central Alignment y Soft Symmetry.

\textbf{Elementos}:
\begin{itemize}
    \item Landmarks sobre radiografía esquemática
    \item Eje central $L_1$-$L_2$ destacado
    \item Distancias perpendiculares de $L_9$, $L_{10}$, $L_{11}$
    \item Pares bilaterales con distancias al eje
    \item Margen de tolerancia ilustrado
\end{itemize}

\subsection{Figura 3.4: Distribución de Asimetría en GT}

\textbf{Descripción}: Histogramas de asimetría por par bilateral.

\textbf{Elementos}:
\begin{itemize}
    \item 5 histogramas (uno por par)
    \item Línea vertical en margen $m = 6$ px
    \item Media y desviación estándar anotadas
\end{itemize}

% ==============================================================================
\section{Conclusiones}
% ==============================================================================

El análisis comparativo de funciones de pérdida condujo a las siguientes conclusiones:

\begin{enumerate}
    \item \textbf{Wing Loss es superior}: Mejora 20\% sobre MSE gracias a gradientes informativos para errores pequeños

    \item \textbf{Restricciones geométricas no son necesarias}: El modelo aprende implícitamente la estructura del eje central y simetría bilateral

    \item \textbf{Ponderación uniforme es óptima}: Estrategias de ponderación por dificultad no mejoran el rendimiento

    \item \textbf{Parámetros normalizados}: $\omega$ y $\epsilon$ deben escalarse para coordenadas en $[0, 1]$
\end{enumerate}

\begin{metodologia}
La función de pérdida final seleccionada es \textbf{Wing Loss} con parámetros:
\begin{itemize}
    \item $\omega = 10$ px (0.0446 normalizado)
    \item $\epsilon = 2$ px (0.0089 normalizado)
    \item Sin restricciones geométricas adicionales
    \item Ponderación uniforme para todos los landmarks
\end{itemize}
\end{metodologia}

% ==============================================================================
\section{Archivos de Referencia}
% ==============================================================================

\begin{table}[H]
\centering
\caption{Archivos fuente relacionados con este documento}
\begin{tabular}{ll}
\toprule
\textbf{Archivo} & \textbf{Descripción} \\
\midrule
\archivo{src\_v2/models/losses.py} & Implementación de Wing Loss y restricciones \\
\archivo{tests/test\_losses.py} & Tests unitarios de funciones de pérdida \\
\archivo{SESSION\_LOG.md} & Registro de sesión 2 \\
\bottomrule
\end{tabular}
\end{table}

% ==============================================================================
\bibliographystyle{ieeetr}
% \bibliography{referencias}
% ==============================================================================

\end{document}
