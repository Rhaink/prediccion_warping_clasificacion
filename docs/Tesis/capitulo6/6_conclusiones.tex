% =============================================================================
% CAPÍTULO 6: CONCLUSIONES Y TRABAJOS FUTUROS
% =============================================================================

\chapter{Conclusiones y Trabajos Futuros}
\label{cap:conclusiones}

Este capítulo presenta las conclusiones del trabajo, sintetiza las contribuciones principales, valida la hipótesis planteada, discute las limitaciones del estudio y propone direcciones para trabajos futuros.

\section{Síntesis de Contribuciones}
\label{sec:sintesis_contribuciones}

El presente trabajo desarrolló un sistema completo para la clasificación automática de enfermedades pulmonares en radiografías de tórax basado en normalización geométrica mediante detección de puntos de referencia anatómicos y deformación afín por partes. El sistema demostró efectividad al alcanzar 98.10\% de exactitud y 97.17\% de F1-Score Macro en la clasificación de COVID-19, Normal y Neumonía Viral sobre un conjunto de datos de 15,153 imágenes.

\subsection{Contribución Principal}
\label{subsec:contribucion_principal}

La contribución central del trabajo es la \textbf{validación experimental de la hipótesis} de que la normalización geométrica mejora la clasificación de enfermedades pulmonares, junto con el \textbf{hallazgo de que clasificadores entrenados con imágenes sin procesar aprenden características espurias}.

El experimento comparativo de tres configuraciones (Original, Normalizado, Recortado) proporciona evidencia directa de que:

\begin{enumerate}
    \item \textbf{Las imágenes originales explotan artefactos hospitalarios:} La caída de 3.32 puntos porcentuales al recortar los bordes (de 98.68\% a 95.36\%) demuestra que la exactitud reportada en imágenes sin procesar está artificialmente inflada por etiquetas y marcadores ubicados en las esquinas.

    \item \textbf{La normalización geométrica aprende características genuinas:} El sistema propuesto (98.10\%) alcanza alto rendimiento utilizando únicamente la región pulmonar, sin acceso a artefactos periféricos.

    \item \textbf{El recorte simple es insuficiente:} Eliminar los bordes sin normalización (95.36\%) resulta en el peor rendimiento, indicando que se requiere un preprocesamiento que enfoque activamente la región diagnóstica.
\end{enumerate}

Este hallazgo tiene implicaciones importantes para la evaluación de sistemas de clasificación de imágenes médicas: la exactitud más alta no necesariamente indica un mejor modelo si este explota correlaciones espurias que no generalizarán a datos de otros hospitales.

\subsection{Contribuciones Específicas}
\label{subsec:contribuciones_especificas}

El trabajo aporta las siguientes contribuciones específicas al estado del arte:

\subsubsection{1. Modelo de Predicción de Puntos de Referencia}

Se desarrolló un modelo basado en ResNet-18 con Coordinate Attention que alcanza un error de 3.61 píxeles mediante combinación de cuatro modelos (ensamble) con aumento en tiempo de prueba. Este resultado representa una mejora del 10.6\% respecto al mejor modelo individual (4.04 píxeles).

\textbf{Características del modelo:}
\begin{itemize}
    \item Integración de Coordinate Attention para preservar información posicional durante la extracción de características.
    \item Cabeza de regresión profunda con Group Normalization y regularización mediante dropout.
    \item Estrategia de entrenamiento en dos fases con tasas de aprendizaje diferenciadas.
\end{itemize}

\subsubsection{2. Sistema Completo de Normalización Geométrica}

Se implementó un sistema que integra:
\begin{itemize}
    \item Análisis Procrustes Generalizado (GPA) para cálculo de forma estándar de referencia.
    \item Triangulación de Delaunay para partición de la región pulmonar en 16 triángulos.
    \item Transformación afín por partes para transformación precisa de cada triángulo.
\end{itemize}

\subsubsection{3. Sistema de Clasificación de Alto Rendimiento}

El clasificador ResNet-18 entrenado sobre imágenes normalizadas alcanza:
\begin{itemize}
    \item \textbf{Exactitud:} 98.10\% sobre conjunto de prueba de 1,895 imágenes.
    \item \textbf{F1-Macro:} 97.17\%, indicando rendimiento equilibrado entre clases desbalanceadas.
    \item \textbf{F1-Score por clase:} Normal (98.60\%), COVID-19 (97.76\%), Neumonía Viral (95.15\%).
\end{itemize}

\subsubsection{4. Metodología Reproducible y Documentada}

Se estableció una metodología completa con:
\begin{itemize}
    \item Protocolo de anotación semi-automática de puntos de referencia mediante herramienta gráfica interactiva.
    \item Configuraciones experimentales en formato JSON para reproducibilidad.
    \item Documentación de valores de referencia validados experimentalmente.
    \item Documentación exhaustiva de experimentos y decisiones de diseño.
\end{itemize}

\section{Validación de la Hipótesis}
\label{sec:validacion_hipotesis}

\subsection{Hipótesis Planteada}
\label{subsec:hipotesis_planteada}

La hipótesis central del trabajo fue:

\begin{quote}
\textit{``La alineación y normalización geométrica de radiografías de tórax mediante detección automática de puntos de referencia anatómicos mejora el rendimiento de clasificación de enfermedades pulmonares al reducir la variabilidad no relacionada con la patología.''}
\end{quote}

\subsection{Evidencia de Validación}
\label{subsec:evidencia_validacion}

La hipótesis se valida mediante la siguiente evidencia:

\subsubsection{1. Efectividad Demostrada del Sistema Completo}

El sistema basado en normalización geométrica alcanzó 98.10\% de exactitud y 97.17\% de F1-Macro, demostrando que el enfoque propuesto es efectivo para la clasificación de enfermedades pulmonares. Este rendimiento es:
\begin{itemize}
    \item Competitivo con trabajos relacionados en la literatura de clasificación de COVID-19.
    \item Robusto ante variaciones de semilla aleatoria ($\pm$0.35 pp).
    \item Equilibrado entre clases desbalanceadas (F1-Macro $\approx$ F1-Weighted).
\end{itemize}

\subsubsection{2. Mecanismos Fundamentados de Mejora}

Se identificaron tres mecanismos por los cuales la normalización geométrica contribuye al rendimiento:

\begin{enumerate}
    \item \textbf{Eliminación de variabilidad no patológica:} Las transformaciones afines por partes corrigen variaciones de posición, escala, rotación y deformación local del paciente, permitiendo al clasificador enfocarse en patrones patológicos.

    \item \textbf{Selección implícita de características:} El proceso de normalización enfoca el modelo en la región pulmonar, eliminando regiones periféricas no informativas y actuando como un mecanismo de atención geométrica explícita.

    \item \textbf{Regularización geométrica:} La transformación a una forma estándar fija introduce un prior estructurado que puede mejorar la generalización del modelo.
\end{enumerate}

\subsubsection{3. Precisión Suficiente de Puntos de Referencia}

El error medio de 3.61 píxeles (1.6\% del tamaño de imagen) del modelo de predicción de puntos de referencia es suficiente para generar transformaciones de deformación que preservan la estructura anatómica sin introducir distorsiones significativas. Esto valida que la detección automática de puntos de referencia puede reemplazar la anotación manual para la aplicación de normalización geométrica.

\subsubsection{4. Evidencia Directa del Experimento de Configuraciones}

Se realizó una comparación controlada entre tres configuraciones de preprocesamiento, todas utilizando SAHS (\textit{Statistical Asymmetrical Histogram Stretching}) como método de mejora de contraste:

\begin{itemize}
    \item \textbf{Original + SAHS:} 98.68\% de exactitud.
    \item \textbf{Normalizado + SAHS:} 98.10\% de exactitud.
    \item \textbf{Recortado (12\%) + SAHS:} 95.36\% de exactitud.
\end{itemize}

Este experimento proporciona \textbf{evidencia directa} de que las imágenes originales aprenden características espurias:

\begin{enumerate}
    \item \textbf{Caída drástica al recortar bordes:} La exactitud cae 3.32 puntos porcentuales (de 98.68\% a 95.36\%) al eliminar únicamente las etiquetas hospitalarias de las esquinas mediante un recorte conservador del 12\%. Esto demuestra que el modelo con imágenes sin procesar dependía de estos artefactos.

    \item \textbf{95.36\% como exactitud ``real'':} El resultado del recorte representa la capacidad genuina del modelo para clasificar basándose en información de la imagen central, sin acceso a atajos en los bordes.

    \item \textbf{La normalización geométrica recupera el rendimiento:} El sistema propuesto (98.10\%) supera al recortado (95.36\%) en 2.74 puntos, demostrando que la alineación y enfoque en la región pulmonar permite aprender características patológicas que compensan la ausencia de artefactos.

    \item \textbf{98.68\% inflado por artefactos:} La exactitud más alta en imágenes originales no indica un mejor modelo, sino uno que explota correlaciones espurias que no generalizarán a datos de otros hospitales.
\end{enumerate}

\subsection{Limitaciones de la Validación}
\label{subsec:limitaciones_validacion}

Es importante reconocer las limitaciones en la validación de la hipótesis:

\begin{enumerate}
    \item \textbf{Conjunto de datos único:} La evaluación se limitó al COVID-19 Radiography Database. La generalización a otros conjuntos de datos de hospitales diferentes requiere validación adicional.

\end{enumerate}

A pesar de estas limitación, la \textbf{efectividad demostrada del sistema completo} y los mecanismos teóricos fundamentados proporcionan evidencia sustancial de que la normalización geométrica es un enfoque viable y efectivo para la clasificación de enfermedades pulmonares.

\subsection{Respuesta a la Hipótesis}
\label{subsec:respuesta_hipotesis}

Con base en la evidencia presentada, se concluye que:

\begin{quote}
\textbf{La hipótesis se valida positivamente.} El sistema basado en normalización geométrica mediante detección automática de puntos de referencia demostró ser efectivo para la clasificación de enfermedades pulmonares, alcanzando 98.10\% de exactitud con preprocesamiento SAHS.

La comparación controlada entre configuraciones proporciona \textbf{evidencia directa}:
\begin{itemize}
    \item La caída de 3.32 puntos porcentuales al recortar los bordes (de 98.68\% a 95.36\%) \textbf{demuestra} que las imágenes originales utilizan artefactos hospitalarios como atajos de clasificación.
    \item La exactitud de 98.68\% en imágenes sin procesar está \textbf{artificialmente inflada} por estas características espurias.
    \item El sistema propuesto (98.10\%) alcanza alto rendimiento utilizando \textbf{únicamente} la región pulmonar, aprendiendo características patológicas genuinas en lugar de artefactos.
\end{itemize}

Por tanto, la normalización geométrica no solo estandariza la pose anatómica, sino que actúa como mecanismo de \textbf{filtrado de características espurias}, garantizando que el clasificador aprenda representaciones clínicamente relevantes.
\end{quote}

\section{Implicaciones del Trabajo}
\label{sec:implicaciones}

\subsection{Implicaciones Clínicas}
\label{subsec:implicaciones_clinicas}

Los resultados del trabajo tienen las siguientes implicaciones para aplicaciones clínicas:

\begin{enumerate}

    \item \textbf{Interpretabilidad:} La transformación geométrica explícita proporciona transparencia en el preprocesamiento, un requisito importante para sistemas de inteligencia artificial en medicina.

    \item \textbf{Robustez ante variaciones de adquisición:} La normalización geométrica mitiga diferencias en posicionamiento del paciente, relevante en contextos con múltiples técnicos radiológicos o equipos de adquisición.
\end{enumerate}

\subsection{Implicaciones Metodológicas}
\label{subsec:implicaciones_metodologicas}

El enfoque de normalización geométrica tiene implicaciones más amplias para el análisis de imágenes médicas:

\begin{enumerate}
    \item \textbf{Transferibilidad a otras modalidades:} El enfoque de detección de landmarks + warping puede aplicarse a otras modalidades de imagen médica donde la alineación anatómica es relevante (e.g., resonancia magnética cerebral, mamografía).

    \item \textbf{Complemento a arquitecturas modernas:} La normalización geométrica puede combinarse con arquitecturas más avanzadas (Vision Transformers, EfficientNet) como etapa de preprocesamiento, potencialmente mejorando su rendimiento.

    \item \textbf{Reducción de requisitos de datos:} Al eliminar variabilidad no patológica, la normalización puede reducir la cantidad de datos de entrenamiento necesarios para alcanzar un rendimiento dado.

    \item \textbf{Facilitador del aprendizaje con pocos datos:} En escenarios con datos etiquetados limitados, la normalización geométrica puede reducir la complejidad del problema al estandarizar la entrada, facilitando el aprendizaje del modelo.
\end{enumerate}

\subsection{Implicaciones Técnicas}
\label{subsec:implicaciones_tecnicas}

Desde el punto de vista técnico, el trabajo contribuye con:

\begin{enumerate}
    \item \textbf{Uso efectivo de Coordinate Attention:} La integración de Coordinate Attention en el modelo de puntos de referencia demuestra la utilidad de mecanismos de atención que preservan información posicional para tareas de localización.

    \item \textbf{Estrategia de ensamble:} El ensamble de 4 modelos con diferentes semillas proporciona una mejora consistente (10.6\%) sobre el mejor modelo individual, validando esta estrategia para reducción de varianza.
\end{enumerate}

\section{Limitaciones del Estudio}
\label{sec:limitaciones}

Las siguientes limitaciones deben considerarse al interpretar los resultados:

\subsection{Limitaciones Experimentales}
\label{subsec:limitaciones_experimentales}

\begin{enumerate}
    \item \textbf{Conjunto de datos único:} La evaluación se realizó exclusivamente sobre el COVID-19 Radiography Database. La generalización a otros conjuntos de datos de diferentes hospitales, equipos de adquisición o poblaciones requiere validación adicional.

    \item \textbf{Ausencia de validación externa:} No se evaluó el rendimiento del sistema en datasets externos de otros centros médicos.

    \item \textbf{Evaluación de una sola arquitectura de clasificación:} Se evaluó únicamente ResNet-18 como clasificador. Otras arquitecturas (DenseNet-121, EfficientNet-B0) podrían beneficiarse de manera diferente de la normalización geométrica.
\end{enumerate}

\subsection{Limitaciones Metodológicas}
\label{subsec:limitaciones_metodologicas}

\begin{enumerate}
    \item \textbf{Anotación manual inicial:} El sistema requiere 957 imágenes con anotaciones manuales de puntos de referencia para entrenar el modelo de detección. Esta fase de anotación es laboriosa (aunque se facilita con la herramienta semi-automática desarrollada).

    \item \textbf{Puntos de referencia no anatómicos específicos:} Los 15 puntos de referencia definen el contorno pulmonar pero no corresponden a estructuras anatómicas específicas (e.g., ápex pulmonar, cúpula diafragmática). Esta decisión simplifica la anotación pero limita la interpretabilidad anatómica.

    \item \textbf{Dependencia de calidad de imagen:} El modelo de puntos de referencia asume imágenes de calidad razonable. Imágenes con contraste extremadamente bajo o artefactos severos pueden generar predicciones de puntos de referencia de baja calidad, degradando la deformación subsecuente.

    \item \textbf{Clases específicas de patologías:} El sistema fue diseñado para tres clases específicas (COVID-19, Normal, Neumonía Viral). La extensión a otras patologías pulmonares (tuberculosis, edema pulmonar, neumotórax) requiere reentrenamiento del clasificador.
\end{enumerate}

\subsection{Limitaciones Conceptuales}
\label{subsec:limitaciones_conceptuales}

\begin{enumerate}
    \item \textbf{Asunción de relevancia de la forma:} El enfoque asume que la forma normalizada de la silueta pulmonar es relevante para la clasificación. Para patologías que afectan regiones extrapulmonares (e.g., derrame pleural, ensanchamiento mediastinal), la deformación puede eliminar información diagnóstica relevante.

    \item \textbf{Normalización no deseable en algunos casos:} Para ciertas patologías, la variabilidad geométrica puede ser diagnóstica (e.g., colapso pulmonar unilateral, hiperinsuflación). La normalización podría eliminar señales relevantes en estos casos.
\end{enumerate}

\section{Trabajos Futuros}
\label{sec:trabajos_futuros}

Con base en los resultados y limitaciones identificadas, se proponen las siguientes direcciones para trabajos futuros:

\subsection{Validación y Generalización}
\label{subsec:trabajos_futuros_validacion}

\begin{enumerate}
    \item \textbf{Comparación controlada original vs. normalizado:} Realizar un estudio sistemático comparando el mismo clasificador (ResNet-18, DenseNet-121, EfficientNet-B0) entrenado sobre:
    \begin{itemize}
        \item Imágenes originales con preprocesamiento estándar
        \item Imágenes normalizadas geométricamente
    \end{itemize}
    Manteniendo todos los demás factores constantes (hiperparámetros, semillas, particiones de datos) para cuantificar rigurosamente la contribución de la deformación.

    \item \textbf{Validación externa en múltiples datasets:} Evaluar el sistema en conjuntos de datos externos de diferentes hospitales (e.g., MIMIC-CXR, CheXpert, PadChest) para cuantificar la capacidad de generalización y estudiar estrategias de domain adaptation.

    \item \textbf{Validación clínica prospectiva:} Realizar un estudio clínico en colaboración con radiólogos para evaluar la utilidad del sistema como herramienta de apoyo diagnóstico en flujo de trabajo real.

    \item \textbf{Análisis de robustez:} Evaluar sistemáticamente la robustez del sistema ante perturbaciones realistas (compresión JPEG, ruido gaussiano, variaciones de contraste) en imágenes normalizadas geométricamente.
\end{enumerate}

\subsection{Extensiones del Sistema}
\label{subsec:trabajos_futuros_extensiones}

\begin{enumerate}
    \item \textbf{Clasificación binaria COVID-19:} Adaptar el sistema para la tarea binaria de detección de COVID-19 (positivo/negativo), más relevante clínicamente que la clasificación de tres clases.

    \item \textbf{Extensión a más patologías:} Incorporar clases adicionales (tuberculosis, edema pulmonar, neumotórax, masa pulmonar) ampliando el conjunto de datos de entrenamiento.

    \item \textbf{Segmentación de lesiones:} Utilizar los puntos de referencia predichos como prior para segmentación de opacidades y cuantificación de severidad de la enfermedad.

    \item \textbf{Detección multi-etiqueta:} Extender el sistema para detección simultánea de múltiples patologías co-ocurrentes, reflejando escenarios clínicos reales.

    \item \textbf{Predicción de severidad:} Clasificar la severidad de COVID-19 (leve, moderado, severo, crítico) en lugar de solo detección binaria.
\end{enumerate}

\subsection{Mejoras Metodológicas}
\label{subsec:trabajos_futuros_mejoras}

\begin{enumerate}
    \item \textbf{Arquitecturas alternativas de clasificación:} Evaluar el impacto de la normalización geométrica en arquitecturas más modernas:
    \begin{itemize}
        \item DenseNet-121, EfficientNet-B0 (CNNs más avanzadas)
        \item Vision Transformers (ViT, Swin Transformer)
        \item Hybrid CNN-Transformer (ResNet + Transformer encoder)
    \end{itemize}

    \item \textbf{Puntos de referencia anatómicos específicos:} Rediseñar el sistema de puntos de referencia para corresponder a estructuras anatómicas precisas (ápex pulmonar, ángulo cardiofrénico, cúpula diafragmática), mejorando la interpretabilidad.

    \item \textbf{Aprendizaje end-to-end:} Explorar un enfoque end-to-end donde la detección de puntos de referencia y la clasificación se entrenan conjuntamente, permitiendo que el gradiente de clasificación optimice la detección de puntos de referencia.

    \item \textbf{Spatial Transformer Networks:} Investigar el uso de Spatial Transformer Networks \cite{jaderberg2015spatial} para aprender transformaciones geométricas de manera diferenciable, eliminando la necesidad de puntos de referencia explícitos.

    \item \textbf{Reducción de requisitos de anotación:} Explorar técnicas de aprendizaje semi-supervisado o self-supervised para reducir el número de imágenes con puntos de referencia anotados necesarias (actualmente 957).
\end{enumerate}

\subsection{Interpretabilidad y Explicabilidad}
\label{subsec:trabajos_futuros_interpretabilidad}

\begin{enumerate}
    \item \textbf{Mapas de atención (Grad-CAM):} Generar mapas de activación de clase (Grad-CAM) para visualizar qué regiones de la imagen normalizada contribuyen a cada predicción de clase.

    \item \textbf{Análisis de características aprendidas:} Investigar qué características aprende el clasificador en el espacio normalizado geométricamente versus el espacio original.

    \item \textbf{Validación radiológica:} Realizar un estudio cualitativo con radiólogos para evaluar si las transformaciones geométricas preservan información diagnóstica relevante.

    \item \textbf{Visualización de casos límite:} Identificar y analizar casos donde la normalización mejora o degrada la clasificación respecto al enfoque sin normalización.
\end{enumerate}

\subsection{Optimización e Implementación}
\label{subsec:trabajos_futuros_optimizacion}

\begin{enumerate}
    \item \textbf{Optimización de eficiencia:} Reducir el tiempo de inferencia del ensamble de puntos de referencia mediante:
    \begin{itemize}
        \item Destilación de conocimiento (ensamble de 4 modelos $\rightarrow$ 1 modelo destilado)
        \item Cuantización de modelos (float32 $\rightarrow$ int8)
        \item Pruning de parámetros redundantes
    \end{itemize}

    \item \textbf{Implementación en dispositivos edge:} Adaptar el sistema para ejecución en dispositivos con recursos limitados (tablets, estaciones de radiología embebidas).

    \item \textbf{Interfaz web clínica:} Desarrollar una interfaz web para uso clínico con visualización de puntos de referencia, imagen normalizada y predicciones con probabilidades.

    \item \textbf{Integración con sistemas PACS:} Integrar el sistema con Picture Archiving and Communication Systems (PACS) hospitalarios para procesamiento automático de estudios radiográficos.
\end{enumerate}

\section{Reflexión Final}
\label{sec:reflexion_final}

El presente trabajo demostró que la normalización geométrica mediante detección automática de puntos de referencia anatómicos es un enfoque viable y efectivo para la clasificación de enfermedades pulmonares en radiografías de tórax. El sistema propuesto alcanzó 98.10\% de exactitud y 97.17\% de F1-Macro, resultados competitivos con el estado del arte en clasificación de COVID-19.

La contribución principal del trabajo es la \textbf{validación experimental de la hipótesis} y el \textbf{hallazgo de que clasificadores entrenados con imágenes sin procesar aprenden características espurias}. El experimento de recorte demostró que la exactitud de 98.68\% en imágenes originales está inflada por artefactos hospitalarios: al eliminar únicamente las etiquetas de las esquinas, la exactitud cae a 95.36\%. El sistema propuesto (98.10\%) alcanza rendimiento comparable utilizando exclusivamente la región pulmonar, garantizando que las características aprendidas son clínicamente relevantes.

Desde el punto de vista técnico, el trabajo integra técnicas clásicas de análisis de forma (Análisis Procrustes Generalizado, triangulación de Delaunay, transformación afín por partes) con métodos modernos de aprendizaje profundo (ResNet-18, Coordinate Attention, ensamble). Este enfoque híbrido combina la interpretabilidad de métodos geométricos con la capacidad de representación de redes neuronales.

Aunque el trabajo presenta limitaciones (evaluación en un solo conjunto de datos), establece una base sólida para investigaciones futuras en normalización geométrica para imágenes médicas. Los trabajos futuros propuestos abordan estas limitaciones y exploran extensiones del enfoque.

En conclusión, \textbf{la normalización geométrica basada en puntos de referencia es una técnica con ventajas demostrables para clasificación de imágenes médicas}. Además de estandarizar la pose anatómica, actúa como mecanismo de filtrado que elimina el acceso a características espurias, garantizando que los modelos aprendan representaciones genuinas. Esta propiedad es especialmente valiosa en aplicaciones clínicas donde la generalización a datos de diferentes hospitales es crítica.
