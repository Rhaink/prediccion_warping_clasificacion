% =============================================================================
% OBJETIVOS DE LA TESIS - VERSIÓN AJUSTADA
% =============================================================================
% Este archivo contiene los objetivos ajustados que reflejan lo realmente
% implementado en el proyecto.
%
% Fecha de creación: 16 Diciembre 2025
% Basado en: Análisis de Fase 1 de la tesis
% =============================================================================

\section*{Objetivos}

\subsection*{Objetivo General}

Desarrollar e implementar algoritmos de visión por computadora para la detección, alineación y normalización de la forma de la región pulmonar en imágenes radiográficas de tórax, utilizando además un método eficaz para la selección de características discriminantes, con el fin de mejorar la precisión en la detección automática de neumonía y COVID-19.

\subsection*{Objetivos Específicos}

\begin{enumerate}[leftmargin=*]
    \item Diseñar e implementar un modelo de predicción de landmarks anatómicos basado en ResNet-18 con módulo de Coordinate Attention para localizar 15 puntos en el contorno pulmonar de radiografías de tórax.

    \item Desarrollar un método de normalización geométrica mediante warping afín por partes (\textit{piecewise affine warping}) que transforme las imágenes a una forma canónica utilizando Análisis Procrustes Generalizado (GPA).

    \item Evaluar múltiples arquitecturas de redes neuronales convolucionales (ResNet-18, ResNet-50, VGG-16, DenseNet-121, EfficientNet-B0, MobileNetV2, AlexNet) para la clasificación de COVID-19 y neumonía viral en imágenes normalizadas.

    \item Validar el sistema mediante métricas de clasificación (accuracy, F1-score, precisión, sensibilidad, especificidad) y métricas de robustez (degradación bajo compresión JPEG y desenfoque gaussiano).

    \item Cuantificar la contribución de la normalización geométrica a la robustez del sistema mediante experimentos de control que separen el efecto de reducción de información del efecto de normalización geométrica.

    \item Evaluar la capacidad de generalización del sistema mediante validación cruzada intra-dominio (\textit{cross-evaluation}) y validación externa en un dataset independiente.
\end{enumerate}

% =============================================================================
% NOTA SOBRE "SELECCIÓN DE CARACTERÍSTICAS DISCRIMINANTES"
% =============================================================================
% El título de la tesis menciona "selección de características discriminantes".
% En este trabajo, este concepto se implementa de manera implícita:
%
% La normalización geométrica mediante landmarks actúa como un mecanismo de
% selección de características a nivel de imagen, eliminando información no
% discriminante (background, artefactos hospitalarios, variaciones de pose)
% y preservando únicamente la región pulmonar relevante para la clasificación.
%
% Experimentalmente se demostró que:
% - ~75% de la mejora en robustez proviene de la reducción de información
% - ~25% adicional proviene de la normalización geométrica propiamente dicha
%
% Este enfoque difiere de métodos tradicionales de selección de características
% (PCA, LDA, etc.) pero logra el mismo objetivo: retener únicamente la
% información discriminante para la tarea de clasificación.
% =============================================================================
