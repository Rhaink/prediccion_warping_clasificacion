% Tabla 3.3: Comparación de métodos de normalización geométrica para clasificación
% Uso: % Tabla 3.3: Comparación de métodos de normalización geométrica para clasificación
% Uso: % Tabla 3.3: Comparación de métodos de normalización geométrica para clasificación
% Uso: % Tabla 3.3: Comparación de métodos de normalización geométrica para clasificación
% Uso: \input{capitulo3/tabla_3_3_normalizacion_geometrica}

\begin{table}[htbp]
\centering
\begin{threeparttable}
\caption{Comparación de métodos de normalización geométrica aplicados a clasificación de imágenes médicas}
\label{tab:geometric_normalization_comparison}
\small
\setlength{\tabcolsep}{2pt}
\begin{tabular}{@{}p{3.2cm}p{2.6cm}p{2.4cm}p{1.5cm}p{2.2cm}@{}}
\toprule
\textbf{Trabajo} & \textbf{Método} & \textbf{Conjunto de datos} & \textbf{Mejora Exact. (\%)} & \textbf{Transform. Tipo} \\
\midrule
Jaderberg et al. (2015) \cite{jaderberg2015spatial} & STN & MNIST, etc. & Variable & Affine global \\
Rocha et al. (2024) \cite{rocha2024stern} & STERN & ChestX-ray14 & +2.1\tnote{a} & STN + Attention \\
\midrule
\multicolumn{5}{c}{\textit{Trabajos del grupo de investigación}} \\
\midrule
Picazo-Castillo (2024) \cite{picazo2024comparative} & Normalización + & Neumonía & -- & Representaciones \\
 et al. & Feature selection & (chest X-rays) & & comparativas \\
Ayala-Raggi (2023) \cite{ayala2023synergizing} & Normalización + & COVID-19 & +1.5\tnote{b} & Cropping + \\
 et al. & PCA features & (chest X-rays) & & PCA \\
\midrule
\multicolumn{5}{c}{\textit{Deformación afín por partes para clasificación: Brecha en literatura}} \\
\midrule
\textbf{Este trabajo (2026)} & \textbf{GPA + Deformación} & \textbf{COVID-19} & \textbf{98.10} & \textbf{Piecewise affine} \\
 & \textbf{afín por partes} & \textbf{Radiography} & \textbf{(norm.)\tnote{c}} & \textbf{(local)} \\
\cmidrule{4-4}
 & & & 98.68\tnote{d} (orig.) & 15 puntos ref. + Delaunay \\
\bottomrule
\end{tabular}
\begin{tablenotes}
\footnotesize
\item[a] Mejora de +2.1\% en AUC (\textit{Area Under Curve}), no exactitud directa
\item[b] Mejora estimada de +1.5\% aproximadamente (reportada sin números exactos en el paper)
\item[c] Exactitud en imágenes normalizadas geométricamente: 98.10\%
\item[d] Exactitud en imágenes originales (sin normalización): 98.68\%
\item norm. = Imágenes normalizadas geométricamente, orig. = Imágenes originales
\item Validación cruzada (5-fold): 98.60\% $\pm$ 0.26\%
\item STN = Red de Transformación Espacial (\textit{Spatial Transformer Network}), GPA = Análisis de Procrustes Generalizado (\textit{Generalized Procrustes Analysis})
\item[--] No reportado cuantitativamente en el trabajo original
\item \textbf{Brecha identificada:} Escasa literatura sobre deformación afín por partes aplicada a clasificación médica
\item \textbf{Compromiso observado:} Normalización reduce exactitud absoluta (-0.58\%) pero aprende características genuinas sin artefactos hospitalarios
\item \textit{Fill rate}: 47\% (conservador, preserva valores originales) vs 96-99\% con técnicas de relleno
\item \textit{Margin scale} óptimo: 1.05 (5\% expansión desde centroide de puntos de referencia)
\end{tablenotes}
\end{threeparttable}
\end{table}


\begin{table}[htbp]
\centering
\begin{threeparttable}
\caption{Comparación de métodos de normalización geométrica aplicados a clasificación de imágenes médicas}
\label{tab:geometric_normalization_comparison}
\small
\setlength{\tabcolsep}{2pt}
\begin{tabular}{@{}p{3.2cm}p{2.6cm}p{2.4cm}p{1.5cm}p{2.2cm}@{}}
\toprule
\textbf{Trabajo} & \textbf{Método} & \textbf{Conjunto de datos} & \textbf{Mejora Exact. (\%)} & \textbf{Transform. Tipo} \\
\midrule
Jaderberg et al. (2015) \cite{jaderberg2015spatial} & STN & MNIST, etc. & Variable & Affine global \\
Rocha et al. (2024) \cite{rocha2024stern} & STERN & ChestX-ray14 & +2.1\tnote{a} & STN + Attention \\
\midrule
\multicolumn{5}{c}{\textit{Trabajos del grupo de investigación}} \\
\midrule
Picazo-Castillo (2024) \cite{picazo2024comparative} & Normalización + & Neumonía & -- & Representaciones \\
 et al. & Feature selection & (chest X-rays) & & comparativas \\
Ayala-Raggi (2023) \cite{ayala2023synergizing} & Normalización + & COVID-19 & +1.5\tnote{b} & Cropping + \\
 et al. & PCA features & (chest X-rays) & & PCA \\
\midrule
\multicolumn{5}{c}{\textit{Deformación afín por partes para clasificación: Brecha en literatura}} \\
\midrule
\textbf{Este trabajo (2026)} & \textbf{GPA + Deformación} & \textbf{COVID-19} & \textbf{98.10} & \textbf{Piecewise affine} \\
 & \textbf{afín por partes} & \textbf{Radiography} & \textbf{(norm.)\tnote{c}} & \textbf{(local)} \\
\cmidrule{4-4}
 & & & 98.68\tnote{d} (orig.) & 15 puntos ref. + Delaunay \\
\bottomrule
\end{tabular}
\begin{tablenotes}
\footnotesize
\item[a] Mejora de +2.1\% en AUC (\textit{Area Under Curve}), no exactitud directa
\item[b] Mejora estimada de +1.5\% aproximadamente (reportada sin números exactos en el paper)
\item[c] Exactitud en imágenes normalizadas geométricamente: 98.10\%
\item[d] Exactitud en imágenes originales (sin normalización): 98.68\%
\item norm. = Imágenes normalizadas geométricamente, orig. = Imágenes originales
\item Validación cruzada (5-fold): 98.60\% $\pm$ 0.26\%
\item STN = Red de Transformación Espacial (\textit{Spatial Transformer Network}), GPA = Análisis de Procrustes Generalizado (\textit{Generalized Procrustes Analysis})
\item[--] No reportado cuantitativamente en el trabajo original
\item \textbf{Brecha identificada:} Escasa literatura sobre deformación afín por partes aplicada a clasificación médica
\item \textbf{Compromiso observado:} Normalización reduce exactitud absoluta (-0.58\%) pero aprende características genuinas sin artefactos hospitalarios
\item \textit{Fill rate}: 47\% (conservador, preserva valores originales) vs 96-99\% con técnicas de relleno
\item \textit{Margin scale} óptimo: 1.05 (5\% expansión desde centroide de puntos de referencia)
\end{tablenotes}
\end{threeparttable}
\end{table}


\begin{table}[htbp]
\centering
\begin{threeparttable}
\caption{Comparación de métodos de normalización geométrica aplicados a clasificación de imágenes médicas}
\label{tab:geometric_normalization_comparison}
\small
\setlength{\tabcolsep}{2pt}
\begin{tabular}{@{}p{3.2cm}p{2.6cm}p{2.4cm}p{1.5cm}p{2.2cm}@{}}
\toprule
\textbf{Trabajo} & \textbf{Método} & \textbf{Conjunto de datos} & \textbf{Mejora Exact. (\%)} & \textbf{Transform. Tipo} \\
\midrule
Jaderberg et al. (2015) \cite{jaderberg2015spatial} & STN & MNIST, etc. & Variable & Affine global \\
Rocha et al. (2024) \cite{rocha2024stern} & STERN & ChestX-ray14 & +2.1\tnote{a} & STN + Attention \\
\midrule
\multicolumn{5}{c}{\textit{Trabajos del grupo de investigación}} \\
\midrule
Picazo-Castillo (2024) \cite{picazo2024comparative} & Normalización + & Neumonía & -- & Representaciones \\
 et al. & Feature selection & (chest X-rays) & & comparativas \\
Ayala-Raggi (2023) \cite{ayala2023synergizing} & Normalización + & COVID-19 & +1.5\tnote{b} & Cropping + \\
 et al. & PCA features & (chest X-rays) & & PCA \\
\midrule
\multicolumn{5}{c}{\textit{Deformación afín por partes para clasificación: Brecha en literatura}} \\
\midrule
\textbf{Este trabajo (2026)} & \textbf{GPA + Deformación} & \textbf{COVID-19} & \textbf{98.10} & \textbf{Piecewise affine} \\
 & \textbf{afín por partes} & \textbf{Radiography} & \textbf{(norm.)\tnote{c}} & \textbf{(local)} \\
\cmidrule{4-4}
 & & & 98.68\tnote{d} (orig.) & 15 puntos ref. + Delaunay \\
\bottomrule
\end{tabular}
\begin{tablenotes}
\footnotesize
\item[a] Mejora de +2.1\% en AUC (\textit{Area Under Curve}), no exactitud directa
\item[b] Mejora estimada de +1.5\% aproximadamente (reportada sin números exactos en el paper)
\item[c] Exactitud en imágenes normalizadas geométricamente: 98.10\%
\item[d] Exactitud en imágenes originales (sin normalización): 98.68\%
\item norm. = Imágenes normalizadas geométricamente, orig. = Imágenes originales
\item Validación cruzada (5-fold): 98.60\% $\pm$ 0.26\%
\item STN = Red de Transformación Espacial (\textit{Spatial Transformer Network}), GPA = Análisis de Procrustes Generalizado (\textit{Generalized Procrustes Analysis})
\item[--] No reportado cuantitativamente en el trabajo original
\item \textbf{Brecha identificada:} Escasa literatura sobre deformación afín por partes aplicada a clasificación médica
\item \textbf{Compromiso observado:} Normalización reduce exactitud absoluta (-0.58\%) pero aprende características genuinas sin artefactos hospitalarios
\item \textit{Fill rate}: 47\% (conservador, preserva valores originales) vs 96-99\% con técnicas de relleno
\item \textit{Margin scale} óptimo: 1.05 (5\% expansión desde centroide de puntos de referencia)
\end{tablenotes}
\end{threeparttable}
\end{table}


\begin{table}[htbp]
\centering
\begin{threeparttable}
\caption{Comparación de métodos de normalización geométrica aplicados a clasificación de imágenes médicas}
\label{tab:geometric_normalization_comparison}
\small
\setlength{\tabcolsep}{2pt}
\begin{tabular}{@{}p{3.2cm}p{2.6cm}p{2.4cm}p{1.5cm}p{2.2cm}@{}}
\toprule
\textbf{Trabajo} & \textbf{Método} & \textbf{Conjunto de datos} & \textbf{Mejora Exact. (\%)} & \textbf{Transform. Tipo} \\
\midrule
Jaderberg et al. (2015) \cite{jaderberg2015spatial} & STN & MNIST, etc. & Variable & Affine global \\
Rocha et al. (2024) \cite{rocha2024stern} & STERN & ChestX-ray14 & +2.1\tnote{a} & STN + Attention \\
\midrule
\multicolumn{5}{c}{\textit{Trabajos del grupo de investigación}} \\
\midrule
Picazo-Castillo (2024) \cite{picazo2024comparative} & Normalización + & Neumonía & -- & Representaciones \\
 et al. & Feature selection & (chest X-rays) & & comparativas \\
Ayala-Raggi (2023) \cite{ayala2023synergizing} & Normalización + & COVID-19 & +1.5\tnote{b} & Cropping + \\
 et al. & PCA features & (chest X-rays) & & PCA \\
\midrule
\multicolumn{5}{c}{\textit{Deformación afín por partes para clasificación: Brecha en literatura}} \\
\midrule
\textbf{Este trabajo (2026)} & \textbf{GPA + Deformación} & \textbf{COVID-19} & \textbf{98.10} & \textbf{Piecewise affine} \\
 & \textbf{afín por partes} & \textbf{Radiography} & \textbf{(norm.)\tnote{c}} & \textbf{(local)} \\
\cmidrule{4-4}
 & & & 98.68\tnote{d} (orig.) & 15 puntos ref. + Delaunay \\
\bottomrule
\end{tabular}
\begin{tablenotes}
\footnotesize
\item[a] Mejora de +2.1\% en AUC (\textit{Area Under Curve}), no exactitud directa
\item[b] Mejora estimada de +1.5\% aproximadamente (reportada sin números exactos en el paper)
\item[c] Exactitud en imágenes normalizadas geométricamente: 98.10\%
\item[d] Exactitud en imágenes originales (sin normalización): 98.68\%
\item norm. = Imágenes normalizadas geométricamente, orig. = Imágenes originales
\item Validación cruzada (5-fold): 98.60\% $\pm$ 0.26\%
\item STN = Red de Transformación Espacial (\textit{Spatial Transformer Network}), GPA = Análisis de Procrustes Generalizado (\textit{Generalized Procrustes Analysis})
\item[--] No reportado cuantitativamente en el trabajo original
\item \textbf{Brecha identificada:} Escasa literatura sobre deformación afín por partes aplicada a clasificación médica
\item \textbf{Compromiso observado:} Normalización reduce exactitud absoluta (-0.58\%) pero aprende características genuinas sin artefactos hospitalarios
\item \textit{Fill rate}: 47\% (conservador, preserva valores originales) vs 96-99\% con técnicas de relleno
\item \textit{Margin scale} óptimo: 1.05 (5\% expansión desde centroide de puntos de referencia)
\end{tablenotes}
\end{threeparttable}
\end{table}
