% Tabla 3.2: Comparación de métodos de detección de puntos de referencia anatómicos
% Uso: % Tabla 3.2: Comparación de métodos de detección de puntos de referencia anatómicos
% Uso: % Tabla 3.2: Comparación de métodos de detección de puntos de referencia anatómicos
% Uso: % Tabla 3.2: Comparación de métodos de detección de puntos de referencia anatómicos
% Uso: \input{capitulo3/tabla_3_2_landmark_detection}

\begin{table}[htbp]
\centering
\begin{threeparttable}
\caption{Comparación de métodos de detección de puntos de referencia anatómicos en imágenes médicas}
\label{tab:landmark_detection_comparison}
\small
\begin{tabular}{@{}llccl@{}}
\toprule
\textbf{Trabajo} & \textbf{Tarea} & \textbf{Error} & \textbf{NME} & \textbf{Método} \\
 & & \textbf{(px/mm)} & \textbf{(\%)} & \\
\midrule
\multicolumn{5}{c}{\textit{Puntos de referencia faciales y de columna vertebral}} \\
\midrule
Feng et al. (2018) \cite{feng2018wing} & Facial (68 pts) & -- & 1.47 & Wing Loss + Heatmap \\
Wang et al. (2019) & Facial & -- & SOTA\tnote{a} & Adaptive Wing Loss \\
Yeh et al. (2021) \cite{yeh2021deep} & Columna (68 pts) & 2.3 mm & -- & Deep learning \\
\midrule
\multicolumn{5}{c}{\textit{Puntos de referencia en radiografías de tórax: Brecha identificada}} \\
\midrule
\textbf{Este trabajo (2026)} & \textbf{Pulmón} & \textbf{3.61 px} & \textbf{1.14} & \textbf{Wing Loss +} \\
 & \textbf{(15 pts)} & \textbf{(224$\times$224)} & & \textbf{Ensemble + TTA} \\
\cmidrule{3-4}
 & & 4.04 px\tnote{b} & 1.28 & Individual best \\
\bottomrule
\end{tabular}
\begin{tablenotes}
\footnotesize
\item[a] SOTA = Estado del arte (\textit{State-of-the-art}) reportado sin número exacto en el paper original
\item[b] Mejor modelo individual (seed 456), ensamble reportado en fila principal
\item[--] No reportado o no aplicable (escalas diferentes)
\item NME = Error Medio Normalizado (\textit{Normalized Mean Error}): $(error / diagonal\_imagen) \times 100$
\item px = píxeles en imagen redimensionada (la escala varía por conjunto de datos)
\item mm = milímetros en espacio físico (requiere calibración de imagen DICOM)
\item \textbf{Equivalencia aproximada:} 3.61 px $\approx$ 5.2 mm (suponiendo ancho de tórax $\sim$32 cm en 224 px)
\item \textbf{Limitación:} Conjunto de datos COVID-19 Radiography carece de metadata DICOM uniforme (mezcla RSNA, Kaggle, papers)
\item TTA = Aumento en Tiempo de Prueba (\textit{Test-Time Augmentation}) con corrección de simetría bilateral
\item Ensamble: promedio de 4 modelos (semillas 123, 321, 111, 666) con TTA + CLAHE
\item \textbf{Brecha identificada:} Escasa literatura sobre detección de contorno pulmonar completo en contexto COVID-19
\item Imágenes originales: 299$\times$299 px, redimensionadas a 224$\times$224 para el modelo
\item Error teórico mínimo (variabilidad inter-anotador): $\sim$1.3 px
\end{tablenotes}
\end{threeparttable}
\end{table}


\begin{table}[htbp]
\centering
\begin{threeparttable}
\caption{Comparación de métodos de detección de puntos de referencia anatómicos en imágenes médicas}
\label{tab:landmark_detection_comparison}
\small
\begin{tabular}{@{}llccl@{}}
\toprule
\textbf{Trabajo} & \textbf{Tarea} & \textbf{Error} & \textbf{NME} & \textbf{Método} \\
 & & \textbf{(px/mm)} & \textbf{(\%)} & \\
\midrule
\multicolumn{5}{c}{\textit{Puntos de referencia faciales y de columna vertebral}} \\
\midrule
Feng et al. (2018) \cite{feng2018wing} & Facial (68 pts) & -- & 1.47 & Wing Loss + Heatmap \\
Wang et al. (2019) & Facial & -- & SOTA\tnote{a} & Adaptive Wing Loss \\
Yeh et al. (2021) \cite{yeh2021deep} & Columna (68 pts) & 2.3 mm & -- & Deep learning \\
\midrule
\multicolumn{5}{c}{\textit{Puntos de referencia en radiografías de tórax: Brecha identificada}} \\
\midrule
\textbf{Este trabajo (2026)} & \textbf{Pulmón} & \textbf{3.61 px} & \textbf{1.14} & \textbf{Wing Loss +} \\
 & \textbf{(15 pts)} & \textbf{(224$\times$224)} & & \textbf{Ensemble + TTA} \\
\cmidrule{3-4}
 & & 4.04 px\tnote{b} & 1.28 & Individual best \\
\bottomrule
\end{tabular}
\begin{tablenotes}
\footnotesize
\item[a] SOTA = Estado del arte (\textit{State-of-the-art}) reportado sin número exacto en el paper original
\item[b] Mejor modelo individual (seed 456), ensamble reportado en fila principal
\item[--] No reportado o no aplicable (escalas diferentes)
\item NME = Error Medio Normalizado (\textit{Normalized Mean Error}): $(error / diagonal\_imagen) \times 100$
\item px = píxeles en imagen redimensionada (la escala varía por conjunto de datos)
\item mm = milímetros en espacio físico (requiere calibración de imagen DICOM)
\item \textbf{Equivalencia aproximada:} 3.61 px $\approx$ 5.2 mm (suponiendo ancho de tórax $\sim$32 cm en 224 px)
\item \textbf{Limitación:} Conjunto de datos COVID-19 Radiography carece de metadata DICOM uniforme (mezcla RSNA, Kaggle, papers)
\item TTA = Aumento en Tiempo de Prueba (\textit{Test-Time Augmentation}) con corrección de simetría bilateral
\item Ensamble: promedio de 4 modelos (semillas 123, 321, 111, 666) con TTA + CLAHE
\item \textbf{Brecha identificada:} Escasa literatura sobre detección de contorno pulmonar completo en contexto COVID-19
\item Imágenes originales: 299$\times$299 px, redimensionadas a 224$\times$224 para el modelo
\item Error teórico mínimo (variabilidad inter-anotador): $\sim$1.3 px
\end{tablenotes}
\end{threeparttable}
\end{table}


\begin{table}[htbp]
\centering
\begin{threeparttable}
\caption{Comparación de métodos de detección de puntos de referencia anatómicos en imágenes médicas}
\label{tab:landmark_detection_comparison}
\small
\begin{tabular}{@{}llccl@{}}
\toprule
\textbf{Trabajo} & \textbf{Tarea} & \textbf{Error} & \textbf{NME} & \textbf{Método} \\
 & & \textbf{(px/mm)} & \textbf{(\%)} & \\
\midrule
\multicolumn{5}{c}{\textit{Puntos de referencia faciales y de columna vertebral}} \\
\midrule
Feng et al. (2018) \cite{feng2018wing} & Facial (68 pts) & -- & 1.47 & Wing Loss + Heatmap \\
Wang et al. (2019) & Facial & -- & SOTA\tnote{a} & Adaptive Wing Loss \\
Yeh et al. (2021) \cite{yeh2021deep} & Columna (68 pts) & 2.3 mm & -- & Deep learning \\
\midrule
\multicolumn{5}{c}{\textit{Puntos de referencia en radiografías de tórax: Brecha identificada}} \\
\midrule
\textbf{Este trabajo (2026)} & \textbf{Pulmón} & \textbf{3.61 px} & \textbf{1.14} & \textbf{Wing Loss +} \\
 & \textbf{(15 pts)} & \textbf{(224$\times$224)} & & \textbf{Ensemble + TTA} \\
\cmidrule{3-4}
 & & 4.04 px\tnote{b} & 1.28 & Individual best \\
\bottomrule
\end{tabular}
\begin{tablenotes}
\footnotesize
\item[a] SOTA = Estado del arte (\textit{State-of-the-art}) reportado sin número exacto en el paper original
\item[b] Mejor modelo individual (seed 456), ensamble reportado en fila principal
\item[--] No reportado o no aplicable (escalas diferentes)
\item NME = Error Medio Normalizado (\textit{Normalized Mean Error}): $(error / diagonal\_imagen) \times 100$
\item px = píxeles en imagen redimensionada (la escala varía por conjunto de datos)
\item mm = milímetros en espacio físico (requiere calibración de imagen DICOM)
\item \textbf{Equivalencia aproximada:} 3.61 px $\approx$ 5.2 mm (suponiendo ancho de tórax $\sim$32 cm en 224 px)
\item \textbf{Limitación:} Conjunto de datos COVID-19 Radiography carece de metadata DICOM uniforme (mezcla RSNA, Kaggle, papers)
\item TTA = Aumento en Tiempo de Prueba (\textit{Test-Time Augmentation}) con corrección de simetría bilateral
\item Ensamble: promedio de 4 modelos (semillas 123, 321, 111, 666) con TTA + CLAHE
\item \textbf{Brecha identificada:} Escasa literatura sobre detección de contorno pulmonar completo en contexto COVID-19
\item Imágenes originales: 299$\times$299 px, redimensionadas a 224$\times$224 para el modelo
\item Error teórico mínimo (variabilidad inter-anotador): $\sim$1.3 px
\end{tablenotes}
\end{threeparttable}
\end{table}


\begin{table}[htbp]
\centering
\begin{threeparttable}
\caption{Comparación de métodos de detección de puntos de referencia anatómicos en imágenes médicas}
\label{tab:landmark_detection_comparison}
\small
\begin{tabular}{@{}llccl@{}}
\toprule
\textbf{Trabajo} & \textbf{Tarea} & \textbf{Error} & \textbf{NME} & \textbf{Método} \\
 & & \textbf{(px/mm)} & \textbf{(\%)} & \\
\midrule
\multicolumn{5}{c}{\textit{Puntos de referencia faciales y de columna vertebral}} \\
\midrule
Feng et al. (2018) \cite{feng2018wing} & Facial (68 pts) & -- & 1.47 & Wing Loss + Heatmap \\
Wang et al. (2019) & Facial & -- & SOTA\tnote{a} & Adaptive Wing Loss \\
Yeh et al. (2021) \cite{yeh2021deep} & Columna (68 pts) & 2.3 mm & -- & Deep learning \\
\midrule
\multicolumn{5}{c}{\textit{Puntos de referencia en radiografías de tórax: Brecha identificada}} \\
\midrule
\textbf{Este trabajo (2026)} & \textbf{Pulmón} & \textbf{3.61 px} & \textbf{1.14} & \textbf{Wing Loss +} \\
 & \textbf{(15 pts)} & \textbf{(224$\times$224)} & & \textbf{Ensemble + TTA} \\
\cmidrule{3-4}
 & & 4.04 px\tnote{b} & 1.28 & Individual best \\
\bottomrule
\end{tabular}
\begin{tablenotes}
\footnotesize
\item[a] SOTA = Estado del arte (\textit{State-of-the-art}) reportado sin número exacto en el paper original
\item[b] Mejor modelo individual (seed 456), ensamble reportado en fila principal
\item[--] No reportado o no aplicable (escalas diferentes)
\item NME = Error Medio Normalizado (\textit{Normalized Mean Error}): $(error / diagonal\_imagen) \times 100$
\item px = píxeles en imagen redimensionada (la escala varía por conjunto de datos)
\item mm = milímetros en espacio físico (requiere calibración de imagen DICOM)
\item \textbf{Equivalencia aproximada:} 3.61 px $\approx$ 5.2 mm (suponiendo ancho de tórax $\sim$32 cm en 224 px)
\item \textbf{Limitación:} Conjunto de datos COVID-19 Radiography carece de metadata DICOM uniforme (mezcla RSNA, Kaggle, papers)
\item TTA = Aumento en Tiempo de Prueba (\textit{Test-Time Augmentation}) con corrección de simetría bilateral
\item Ensamble: promedio de 4 modelos (semillas 123, 321, 111, 666) con TTA + CLAHE
\item \textbf{Brecha identificada:} Escasa literatura sobre detección de contorno pulmonar completo en contexto COVID-19
\item Imágenes originales: 299$\times$299 px, redimensionadas a 224$\times$224 para el modelo
\item Error teórico mínimo (variabilidad inter-anotador): $\sim$1.3 px
\end{tablenotes}
\end{threeparttable}
\end{table}
