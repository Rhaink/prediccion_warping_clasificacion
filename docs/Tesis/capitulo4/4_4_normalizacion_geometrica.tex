% =============================================================================
% CAPÍTULO 4: METODOLOGÍA
% Sección 4.4: Normalización Geométrica
% =============================================================================

\section{Normalización Geométrica}
\label{sec:normalizacion_geometrica}

La normalización geométrica constituye el componente central del sistema propuesto, transformando las radiografías de tórax a una forma canónica que elimina variaciones de pose, escala y orientación entre pacientes. Esta sección describe el proceso completo de normalización, desde el cálculo de la forma de referencia mediante Análisis Procrustes Generalizado hasta la aplicación de transformaciones afines por partes.

\subsection{Análisis Procrustes Generalizado}
\label{subsec:gpa}

El Análisis Procrustes Generalizado (GPA, por sus siglas en inglés) es una técnica estadística para alinear un conjunto de configuraciones de landmarks eliminando las diferencias debidas a traslación, escala y rotación \cite{gower1975generalized, dryden1998statistical}. El resultado es una \textit{forma canónica} o \textit{consenso de Procrustes} que representa la forma media del conjunto.

\subsubsection{Formulación Matemática}

Sea $\{\mathbf{X}_1, \mathbf{X}_2, \ldots, \mathbf{X}_n\}$ un conjunto de $n$ configuraciones de landmarks, donde cada $\mathbf{X}_i \in \mathbb{R}^{k \times 2}$ contiene las coordenadas $(x, y)$ de $k = 15$ landmarks. El objetivo de GPA es encontrar transformaciones que minimicen la distancia total entre las configuraciones alineadas.

\textbf{Paso 1: Eliminación de traslación.} Cada configuración se centra en el origen sustrayendo su centroide:

\begin{equation}
    \bar{\mathbf{X}}_i = \mathbf{X}_i - \mathbf{1}_k \bar{\mathbf{x}}_i^T
    \label{eq:centering}
\end{equation}

\noindent donde $\bar{\mathbf{x}}_i = \frac{1}{k}\sum_{j=1}^{k} \mathbf{x}_{ij}$ es el centroide de la configuración $i$, y $\mathbf{1}_k$ es un vector de unos de dimensión $k$.

\textbf{Paso 2: Eliminación de escala.} Las configuraciones centradas se normalizan a norma unitaria (norma de Frobenius igual a 1):

\begin{equation}
    \tilde{\mathbf{X}}_i = \frac{\bar{\mathbf{X}}_i}{\|\bar{\mathbf{X}}_i\|_F}
    \label{eq:scaling}
\end{equation}

\noindent donde $\|\mathbf{A}\|_F = \sqrt{\sum_{j,l} a_{jl}^2}$ es la norma de Frobenius.

\textbf{Paso 3: Eliminación de rotación.} Dada una forma de referencia $\mathbf{Y}$, se busca la matriz de rotación $\mathbf{R}_i$ que minimiza la distancia entre $\tilde{\mathbf{X}}_i$ y $\mathbf{Y}$:

\begin{equation}
    \mathbf{R}_i^* = \arg\min_{\mathbf{R} \in SO(2)} \|\tilde{\mathbf{X}}_i \mathbf{R} - \mathbf{Y}\|_F^2
    \label{eq:rotation_problem}
\end{equation}

\noindent donde $SO(2)$ denota el grupo de matrices de rotación ortogonales $2 \times 2$ con determinante $+1$.

\subsubsection{Solución mediante Descomposición en Valores Singulares}

La matriz de rotación óptima se obtiene mediante la descomposición en valores singulares (SVD) de la matriz de correlación cruzada \cite{schonemann1966generalized}:

\begin{equation}
    \mathbf{H}_i = \tilde{\mathbf{X}}_i^T \mathbf{Y} = \mathbf{U}_i \mathbf{\Sigma}_i \mathbf{V}_i^T
    \label{eq:svd}
\end{equation}

La rotación óptima es entonces:

\begin{equation}
    \mathbf{R}_i^* = \mathbf{V}_i \mathbf{U}_i^T
    \label{eq:optimal_rotation}
\end{equation}

Para garantizar una rotación propia (sin reflexión), se verifica que $\det(\mathbf{R}_i^*) = +1$. Si el determinante es $-1$, se corrige invirtiendo el signo de la última columna de $\mathbf{V}_i$ antes de calcular $\mathbf{R}_i^*$.

\subsubsection{Algoritmo Iterativo}

El GPA se implementa mediante un algoritmo iterativo que alterna entre alinear las formas con la referencia actual y actualizar la referencia como la media de las formas alineadas. El Algoritmo \ref{alg:gpa} presenta el pseudocódigo.

\begin{algorithm}[htbp]
\caption{Análisis Procrustes Generalizado Iterativo}
\label{alg:gpa}
\begin{algorithmic}[1]
\Require Conjunto de configuraciones $\{\mathbf{X}_1, \ldots, \mathbf{X}_n\}$, tolerancia $\tau$, máximo de iteraciones $T$
\Ensure Forma canónica $\mathbf{C}$, formas alineadas $\{\hat{\mathbf{X}}_1, \ldots, \hat{\mathbf{X}}_n\}$
\State \textbf{Inicialización:}
\For{$i = 1$ \textbf{a} $n$}
    \State Centrar: $\bar{\mathbf{X}}_i \leftarrow \mathbf{X}_i - \mathbf{1}_k \bar{\mathbf{x}}_i^T$
    \State Escalar: $\tilde{\mathbf{X}}_i \leftarrow \bar{\mathbf{X}}_i / \|\bar{\mathbf{X}}_i\|_F$
\EndFor
\State Referencia inicial: $\mathbf{Y} \leftarrow \frac{1}{n}\sum_{i=1}^{n} \tilde{\mathbf{X}}_i$
\State Normalizar referencia: $\mathbf{Y} \leftarrow \mathbf{Y} / \|\mathbf{Y}\|_F$
\State \textbf{Iteración:}
\For{$t = 1$ \textbf{a} $T$}
    \For{$i = 1$ \textbf{a} $n$}
        \State Calcular SVD: $\tilde{\mathbf{X}}_i^T \mathbf{Y} = \mathbf{U}_i \mathbf{\Sigma}_i \mathbf{V}_i^T$
        \State Rotación óptima: $\mathbf{R}_i \leftarrow \mathbf{V}_i \mathbf{U}_i^T$
        \State Alinear: $\hat{\mathbf{X}}_i \leftarrow \tilde{\mathbf{X}}_i \mathbf{R}_i$
    \EndFor
    \State Nueva referencia: $\mathbf{Y}_{new} \leftarrow \frac{1}{n}\sum_{i=1}^{n} \hat{\mathbf{X}}_i$
    \State Normalizar: $\mathbf{Y}_{new} \leftarrow \mathbf{Y}_{new} / \|\mathbf{Y}_{new}\|_F$
    \State Cambio: $\delta \leftarrow \|\mathbf{Y}_{new} - \mathbf{Y}\|_F$
    \If{$\delta < \tau$}
        \State \textbf{romper} \Comment{Convergencia alcanzada}
    \EndIf
    \State $\mathbf{Y} \leftarrow \mathbf{Y}_{new}$
\EndFor
\State Forma canónica: $\mathbf{C} \leftarrow \mathbf{Y}$
\State \Return $\mathbf{C}$, $\{\hat{\mathbf{X}}_1, \ldots, \hat{\mathbf{X}}_n\}$
\end{algorithmic}
\end{algorithm}

Los parámetros utilizados en la implementación son:
\begin{itemize}
    \item Tolerancia de convergencia: $\tau = 10^{-8}$
    \item Máximo de iteraciones: $T = 100$
\end{itemize}

En la práctica, el algoritmo converge típicamente en menos de 20 iteraciones para el conjunto de 957 configuraciones de landmarks anotadas.

\subsubsection{Transformación a Coordenadas de Imagen}

La forma canónica resultante del GPA está centrada en el origen con norma unitaria. Para utilizarla como destino del warping, se escala y traslada al sistema de coordenadas de la imagen:

\begin{equation}
    \mathbf{C}_{img} = s \cdot \mathbf{C} + \mathbf{t}
    \label{eq:canonical_to_image}
\end{equation}

\noindent donde el factor de escala $s$ y el vector de traslación $\mathbf{t}$ se calculan para que la forma canónica ocupe el área central de la imagen con un margen de padding del 10\%:

\begin{align}
    s &= \frac{(1 - 2p) \cdot W}{\max(\text{range}_x, \text{range}_y)} \\
    \mathbf{t} &= \begin{pmatrix} W/2 \\ H/2 \end{pmatrix} - s \cdot \bar{\mathbf{c}}
\end{align}

\noindent donde $W = H = 224$ es el tamaño de imagen, $p = 0.1$ es el padding relativo, $\bar{\mathbf{c}}$ es el centroide de $\mathbf{C}$, y el rango de la forma canónica se define como:
\begin{equation}
    \text{range}_x = \max_i(x_i) - \min_i(x_i), \quad
    \text{range}_y = \max_i(y_i) - \min_i(y_i)
\end{equation}
siendo $(x_i, y_i)$ las coordenadas del landmark $i$ en la forma canónica $\mathbf{C}$.

La Figura \ref{fig:gpa_proceso} ilustra el proceso de GPA aplicado al conjunto de landmarks.

\begin{figure}[htbp]
    \centering
    % [PENDIENTE: F4.6 - Proceso de GPA]
    \fbox{\parbox{0.9\textwidth}{\centering\vspace{4cm}
    [Figura F4.6: Proceso de GPA]\\
    (a) Configuraciones originales de landmarks superpuestas\\
    (b) Configuraciones después de centrado y escalado\\
    (c) Configuraciones después de alineación rotacional\\
    (d) Forma canónica resultante (consenso de Procrustes)
    \vspace{4cm}}}
    \caption{Proceso de Análisis Procrustes Generalizado. (a) Las 957 configuraciones de landmarks originales muestran variabilidad en posición, escala y orientación. (b) Después del centrado y escalado, las formas comparten origen y norma unitaria. (c) La alineación rotacional minimiza las diferencias residuales. (d) La forma canónica representa el consenso estadístico del conjunto.}
    \label{fig:gpa_proceso}
\end{figure}

\subsection{Triangulación de Delaunay}
\label{subsec:triangulacion_delaunay}

Para aplicar transformaciones afines locales, es necesario particionar el dominio de la imagen en regiones. La triangulación de Delaunay \cite{delaunay1934sphere} proporciona una partición óptima que maximiza el ángulo mínimo de los triángulos, evitando triángulos degenerados que causarían artefactos en el warping.

\subsubsection{Definición y Propiedades}

Dado un conjunto de puntos $P = \{p_1, p_2, \ldots, p_m\}$ en el plano, la triangulación de Delaunay $DT(P)$ es una triangulación tal que ningún punto de $P$ está dentro del circuncírculo de ningún triángulo. Esta propiedad, conocida como \textit{criterio del círculo vacío}, garantiza triángulos ``bien formados'' \cite{berg2008computational}.

Propiedades relevantes de la triangulación de Delaunay:
\begin{enumerate}
    \item \textbf{Maximización del ángulo mínimo:} Entre todas las triangulaciones posibles, la de Delaunay maximiza el ángulo más pequeño, produciendo triángulos lo más equiláteros posible.
    \item \textbf{Unicidad:} Para puntos en posición general (sin cuatro puntos cocirculares), la triangulación es única.
    \item \textbf{Eficiencia computacional:} Puede calcularse en tiempo $O(m \log m)$ mediante algoritmos como el de Fortune \cite{fortune1987sweepline}.
\end{enumerate}

\subsubsection{Aplicación a los Landmarks}

La triangulación se calcula sobre los landmarks de la forma canónica $\mathbf{C}_{img}$. Para 15 landmarks, la triangulación de Delaunay produce típicamente entre 20 y 25 triángulos, dependiendo de la disposición geométrica de los puntos.

La Figura \ref{fig:triangulacion_delaunay} muestra la triangulación resultante sobre la forma canónica.

\begin{figure}[htbp]
    \centering
    % [PENDIENTE: F4.7 - Triangulación Delaunay]
    \fbox{\parbox{0.85\textwidth}{\centering\vspace{4cm}
    [Figura F4.7: Triangulación Delaunay]\\
    Forma canónica con los 15 landmarks conectados por\\
    la triangulación de Delaunay, mostrando los triángulos\\
    que definen las regiones de transformación local
    \vspace{4cm}}}
    \caption{Triangulación de Delaunay sobre los 15 landmarks de la forma canónica. Los triángulos definen las regiones donde se aplicarán transformaciones afines independientes durante el proceso de warping.}
    \label{fig:triangulacion_delaunay}
\end{figure}

\subsection{Transformación Afín por Partes}
\label{subsec:warping_afin}

La transformación afín por partes (\textit{piecewise affine warping}) utiliza la correspondencia entre triángulos en la imagen fuente y destino para deformar localmente la imagen, preservando líneas rectas dentro de cada triángulo \cite{wolberg1990digital}.

\subsubsection{Transformación Afín de un Triángulo}

Dados un triángulo fuente con vértices $\{(x_1^s, y_1^s), (x_2^s, y_2^s), (x_3^s, y_3^s)\}$ y un triángulo destino con vértices $\{(x_1^d, y_1^d), (x_2^d, y_2^d), (x_3^d, y_3^d)\}$, la transformación afín que mapea el triángulo fuente al destino se representa mediante una matriz $\mathbf{M} \in \mathbb{R}^{2 \times 3}$:

\begin{equation}
    \begin{pmatrix} x^d \\ y^d \end{pmatrix} = \mathbf{M} \begin{pmatrix} x^s \\ y^s \\ 1 \end{pmatrix} = \begin{pmatrix} a & b & c \\ d & e & f \end{pmatrix} \begin{pmatrix} x^s \\ y^s \\ 1 \end{pmatrix}
    \label{eq:affine_transform}
\end{equation}

Los seis parámetros de la matriz se obtienen resolviendo el sistema de ecuaciones que mapea los tres vértices:

\begin{equation}
    \begin{pmatrix}
        x_1^s & y_1^s & 1 \\
        x_2^s & y_2^s & 1 \\
        x_3^s & y_3^s & 1
    \end{pmatrix}
    \begin{pmatrix} a & d \\ b & e \\ c & f \end{pmatrix}
    =
    \begin{pmatrix}
        x_1^d & y_1^d \\
        x_2^d & y_2^d \\
        x_3^d & y_3^d
    \end{pmatrix}
    \label{eq:affine_system}
\end{equation}

Este sistema tiene solución única siempre que los tres vértices fuente no sean colineales (triángulo no degenerado).

\subsubsection{Algoritmo de Warping}

El proceso de warping se realiza iterando sobre cada triángulo de la triangulación y aplicando la transformación afín correspondiente. El Algoritmo \ref{alg:warping} describe el procedimiento.

\begin{algorithm}[htbp]
\caption{Warping Afín por Partes}
\label{alg:warping}
\begin{algorithmic}[1]
\Require Imagen fuente $I_s$, landmarks fuente $\mathbf{L}_s$, landmarks destino $\mathbf{L}_d$, triangulación $\mathcal{T}$
\Ensure Imagen warpeada $I_d$
\State Inicializar imagen destino: $I_d \leftarrow \mathbf{0}$
\For{cada triángulo $T_k = (i, j, l) \in \mathcal{T}$}
    \State Obtener vértices fuente: $\Delta_s \leftarrow (\mathbf{L}_s[i], \mathbf{L}_s[j], \mathbf{L}_s[l])$
    \State Obtener vértices destino: $\Delta_d \leftarrow (\mathbf{L}_d[i], \mathbf{L}_d[j], \mathbf{L}_d[l])$
    \If{$\text{area}(\Delta_s) < \epsilon$ \textbf{o} $\text{area}(\Delta_d) < \epsilon$}
        \State \textbf{continuar} \Comment{Saltar triángulos degenerados}
    \EndIf
    \State Calcular bounding boxes: $B_s \leftarrow \text{bbox}(\Delta_s)$, $B_d \leftarrow \text{bbox}(\Delta_d)$
    \State Ajustar triángulos a coordenadas locales
    \State Calcular matriz afín: $\mathbf{M} \leftarrow \texttt{getAffineTransform}(\Delta_s^{local}, \Delta_d^{local})$
    \State Warpear región: $P_d \leftarrow \texttt{warpAffine}(I_s[B_s], \mathbf{M})$
    \State Crear máscara triangular: $M_k \leftarrow \texttt{fillConvexPoly}(\Delta_d^{local})$
    \State Copiar con máscara: $I_d[B_d] \leftarrow I_d[B_d] \cdot (1 - M_k) + P_d \cdot M_k$
\EndFor
\State \Return $I_d$
\end{algorithmic}
\end{algorithm}

La interpolación durante el warping se realiza mediante interpolación bilineal (\texttt{cv2.INTER\_LINEAR}), que proporciona un buen balance entre calidad visual y eficiencia computacional. Para el manejo de bordes, se utiliza reflexión (\texttt{cv2.BORDER\_REFLECT\_101}) para evitar artefactos en los límites de los triángulos.

\subsection{Estrategia de Cobertura Completa}
\label{subsec:full_coverage}

La triangulación de Delaunay sobre los 15 landmarks anatómicos cubre únicamente la región pulmonar central. Para garantizar que toda la imagen sea procesada sin dejar regiones negras (no warpeadas), se implementa una estrategia de \textit{cobertura completa} que extiende la triangulación hasta los bordes de la imagen.

\subsubsection{Puntos de Borde Adicionales}

Se agregan 8 puntos auxiliares a los 15 landmarks originales:
\begin{itemize}
    \item \textbf{4 esquinas de la imagen:} $(0, 0)$, $(W-1, 0)$, $(0, H-1)$, $(W-1, H-1)$
    \item \textbf{4 puntos medios de los bordes:} $(W/2, 0)$, $(0, H/2)$, $(W-1, H/2)$, $(W/2, H-1)$
\end{itemize}

Estos puntos se mantienen fijos tanto en la configuración fuente como en la destino, lo que garantiza que:
\begin{enumerate}
    \item Los bordes de la imagen permanecen en su posición original.
    \item La triangulación extendida cubre el 100\% del área de la imagen.
    \item Las regiones periféricas (fuera de los pulmones) sufren deformación mínima.
\end{enumerate}

Con los 8 puntos adicionales, el conjunto extendido tiene 23 puntos, y la triangulación de Delaunay produce aproximadamente 35-40 triángulos.

\subsubsection{Cálculo de la Tasa de Llenado}

La \textit{tasa de llenado} cuantifica la proporción de la imagen warpeada que contiene información útil (píxeles no negros):

\begin{equation}
    \text{fill\_rate} = 1 - \frac{\sum_{i,j} \mathbb{1}[I_d(i,j) = 0]}{W \times H}
    \label{eq:fill_rate}
\end{equation}

\noindent donde $\mathbb{1}[\cdot]$ es la función indicadora.

Sin la estrategia de cobertura completa, la tasa de llenado típica es del 40-50\%, dejando grandes regiones negras en las esquinas. Con la cobertura completa, la tasa de llenado alcanza valores superiores al 95\%, dependiendo del preprocesamiento aplicado a las imágenes.

La Figura \ref{fig:warping_comparison} muestra la diferencia visual entre el warping estándar y el de cobertura completa.

\begin{figure}[htbp]
    \centering
    % [PENDIENTE: F4.8 - Comparación Original vs Warped]
    \fbox{\parbox{0.9\textwidth}{\centering\vspace{4cm}
    [Figura F4.8: Comparación de imágenes]\\
    (a) Imagen original\\
    (b) Imagen warpeada sin cobertura completa (tasa de llenado $\approx$ 47\%)\\
    (c) Imagen warpeada con cobertura completa (tasa de llenado $\approx$ 96\%)
    \vspace{4cm}}}
    \caption{Comparación del proceso de warping. (a) Radiografía original con variabilidad de pose y escala. (b) Warping estándar que cubre solo la región delimitada por los landmarks, dejando esquinas negras. (c) Warping con cobertura completa que procesa toda la imagen, preservando el contexto periférico.}
    \label{fig:warping_comparison}
\end{figure}

\subsection{Parámetro de Escala de Margen}
\label{subsec:margin_scale}

El parámetro \textit{margin\_scale} controla la expansión de los landmarks predichos desde su centroide antes de aplicar el warping. Este parámetro permite ajustar el área efectiva de la región pulmonar en la imagen normalizada.

\subsubsection{Definición Matemática}

Dado un conjunto de landmarks predichos $\mathbf{L}$ con centroide $\bar{\mathbf{l}}$, los landmarks escalados se calculan como:

\begin{equation}
    \mathbf{L}_{scaled} = \bar{\mathbf{l}} + \alpha \cdot (\mathbf{L} - \bar{\mathbf{l}})
    \label{eq:margin_scale}
\end{equation}

\noindent donde $\alpha$ es el factor de escala de margen (\textit{margin\_scale}):
\begin{itemize}
    \item $\alpha = 1.0$: Sin expansión, los landmarks mantienen su posición original.
    \item $\alpha > 1.0$: Expansión desde el centroide, incluyendo más contexto alrededor de la región pulmonar.
    \item $\alpha < 1.0$: Contracción hacia el centroide (no utilizado en la práctica).
\end{itemize}

\subsubsection{Determinación del Valor Óptimo}

El valor óptimo de margin\_scale se determinó experimentalmente mediante una búsqueda en grid sobre los valores $\{1.00, 1.05, 1.10, 1.15, 1.20, 1.25, 1.30\}$. El criterio de selección fue minimizar el error de warping, definido como la distancia euclidiana promedio entre los landmarks de la forma canónica y los landmarks predichos después de aplicar la transformación inversa.

Los resultados indicaron que $\alpha = 1.05$ proporciona el mejor balance:
\begin{itemize}
    \item Expansión suficiente (5\%) para incluir la estructura pulmonar completa.
    \item Sin sobre-expansión que cause artefactos de warping o incluya regiones irrelevantes.
\end{itemize}

La Tabla \ref{tab:margin_scale_optimo} resume los parámetros de warping utilizados.

\begin{table}[htbp]
    \centering
    \caption{Parámetros óptimos del proceso de normalización geométrica.}
    \label{tab:margin_scale_optimo}
    \begin{tabular}{llc}
        \toprule
        \textbf{Parámetro} & \textbf{Descripción} & \textbf{Valor} \\
        \midrule
        margin\_scale & Factor de expansión desde centroide & 1.05 \\
        use\_full\_coverage & Agregar puntos de borde & Sí (8 puntos) \\
        Interpolación & Método de interpolación & Bilineal \\
        Manejo de bordes & Tratamiento de límites & Reflexión \\
        Tamaño de salida & Dimensiones de imagen warpeada & $224 \times 224$ \\
        \bottomrule
    \end{tabular}
\end{table}

La Figura \ref{fig:margin_scale_effect} ilustra el efecto de diferentes valores de margin\_scale.

\begin{figure}[htbp]
    \centering
    % [PENDIENTE: F4.9 - Efecto de margin_scale]
    \fbox{\parbox{0.9\textwidth}{\centering\vspace{4cm}
    [Figura F4.9: Efecto de margin\_scale]\\
    Comparación de imágenes warpeadas con diferentes valores:\\
    (a) $\alpha = 1.00$ - Sin margen\\
    (b) $\alpha = 1.05$ - Margen óptimo\\
    (c) $\alpha = 1.25$ - Margen excesivo
    \vspace{4cm}}}
    \caption{Efecto del parámetro margin\_scale en el resultado del warping. (a) Sin margen adicional, la región pulmonar puede quedar recortada. (b) Con el valor óptimo de 1.05, la estructura pulmonar se captura completamente. (c) Un margen excesivo incluye regiones periféricas irrelevantes.}
    \label{fig:margin_scale_effect}
\end{figure}

\subsection{Proceso Completo de Normalización}
\label{subsec:proceso_normalizacion}

La Figura \ref{fig:proceso_normalizacion} presenta el diagrama de flujo del proceso completo de normalización geométrica, integrando todos los componentes descritos.

\begin{figure}[htbp]
    \centering
    % [PENDIENTE: Diagrama de flujo del proceso de normalización]
    \fbox{\parbox{0.9\textwidth}{\centering\vspace{3cm}
    [Diagrama de flujo del proceso de normalización]\\
    1. Predicción de landmarks $\rightarrow$
    2. Escalado con margin\_scale $\rightarrow$\\
    3. Adición de puntos de borde $\rightarrow$
    4. Triangulación Delaunay $\rightarrow$\\
    5. Warping afín por partes $\rightarrow$
    6. Imagen normalizada
    \vspace{3cm}}}
    \caption{Proceso completo de normalización geométrica. El sistema transforma una radiografía de entrada en una imagen geométricamente normalizada mediante la secuencia de predicción de landmarks, escalado, triangulación y warping afín por partes.}
    \label{fig:proceso_normalizacion}
\end{figure}

El proceso completo para una imagen de entrada se resume en los siguientes pasos:

\begin{enumerate}
    \item \textbf{Predicción de landmarks:} El modelo ResNet-18 con Coordinate Attention predice las coordenadas de los 15 landmarks anatómicos.

    \item \textbf{Escalado de landmarks:} Los landmarks predichos se escalan desde su centroide con factor $\alpha = 1.05$.

    \item \textbf{Extensión con puntos de borde:} Se agregan 8 puntos auxiliares (4 esquinas + 4 puntos medios) para garantizar cobertura completa.

    \item \textbf{Triangulación:} Se calcula la triangulación de Delaunay sobre los 23 puntos de la forma canónica extendida.

    \item \textbf{Warping por triángulos:} Para cada triángulo, se calcula la transformación afín y se aplica a la región correspondiente de la imagen.

    \item \textbf{Imagen normalizada:} El resultado es una imagen de $224 \times 224$ píxeles donde la región pulmonar está alineada con la forma canónica.
\end{enumerate}

El tiempo de procesamiento para una imagen individual es de aproximadamente 15-20 milisegundos en CPU (Intel Core i7) o 3-5 milisegundos en GPU (NVIDIA RTX 3080), lo que permite su uso en aplicaciones de tiempo real.

% Referencias temporales para esta sección
% \cite{gower1975generalized} - Gower, J.C. (1975). Generalized Procrustes Analysis. Psychometrika.
% \cite{dryden1998statistical} - Dryden, I.L. & Mardia, K.V. (1998). Statistical Shape Analysis. Wiley.
% \cite{schonemann1966generalized} - Schönemann, P.H. (1966). Generalized solution of the orthogonal Procrustes problem.
% \cite{delaunay1934sphere} - Delaunay, B. (1934). Sur la sphère vide.
% \cite{berg2008computational} - de Berg et al. (2008). Computational Geometry. Springer.
% \cite{fortune1987sweepline} - Fortune, S. (1987). A sweepline algorithm for Voronoi diagrams.
% \cite{wolberg1990digital} - Wolberg, G. (1990). Digital Image Warping. IEEE Computer Society Press.
