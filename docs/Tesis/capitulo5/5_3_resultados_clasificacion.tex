% =============================================================================
% CAPÍTULO 5: RESULTADOS
% Sección 5.3: Resultados de Clasificación
% =============================================================================

\section{Clasificación de Enfermedades Pulmonares}
\label{sec:resultados_clasificacion}

Esta sección presenta los resultados del clasificador entrenado sobre las imágenes normalizadas geométricamente. El sistema clasifica cada radiografía en una de tres categorías: COVID-19, Normal o Neumonía Viral.

\subsection{Rendimiento General}
\label{subsec:metricas_globales}

El clasificador fue evaluado sobre un conjunto de prueba de 1,895 imágenes que no fueron utilizadas durante el entrenamiento. La Tabla \ref{tab:resultados_clasificacion_global} presenta las métricas principales de rendimiento.

\begin{table}[htbp]
    \centering
    \caption{Rendimiento del clasificador sobre el conjunto de prueba (1,895 imágenes).}
    \label{tab:resultados_clasificacion_global}
    \begin{tabular}{lc}
        \toprule
        \textbf{Métrica} & \textbf{Valor} \\
        \midrule
        Exactitud (Accuracy) & \textbf{98.10\%} \\
        F1-Score Macro & \textbf{97.17\%} \\
        F1-Score Ponderado & \textbf{98.09\%} \\
        \midrule
        Imágenes correctamente clasificadas & 1,859 de 1,895 \\
        Imágenes incorrectamente clasificadas & 36 de 1,895 \\
        \bottomrule
    \end{tabular}
\end{table}

El clasificador alcanza una exactitud de \textbf{98.10\%}, clasificando correctamente 1,859 de las 1,895 imágenes de prueba. El F1-Score Macro de 97.17\% indica un rendimiento equilibrado entre las tres categorías, lo cual es relevante dado que las categorías tienen diferente cantidad de muestras.

\subsection{Validación Cruzada}
\label{subsec:validacion_cruzada}

Para evaluar la estabilidad del clasificador, se realizó una validación cruzada
estratificada de $k=5$ usando el conjunto combinado de entrenamiento y validación (13,258 imágenes). El conjunto de prueba fijo (1,895 imágenes) se
mantuvo intacto y no se utilizó en los pliegues. La Tabla
\ref{tab:resultados_clasificacion_cv} resume el promedio y la desviación
estándar de las métricas en validación.

\begin{table}[htbp]
    \centering
    \caption{Resultados de validación cruzada (k=5) sobre train+val.}
    \label{tab:resultados_clasificacion_cv}
    \begin{tabular}{lc}
        \toprule
        \textbf{Métrica} & \textbf{Media $\pm$ DE} \\
        \midrule
        Exactitud (Accuracy) & \textbf{98.60\% $\pm$ 0.26} \\
        F1-Score Macro & \textbf{98.00\% $\pm$ 0.36} \\
        F1-Score Ponderado & \textbf{98.60\% $\pm$ 0.25} \\
        \bottomrule
    \end{tabular}
\end{table}

\subsection{Rendimiento por Categoría}
\label{subsec:rendimiento_por_clase}

La Tabla \ref{tab:metricas_por_clase} desglosa el rendimiento del clasificador para cada categoría diagnóstica.

\begin{table}[htbp]
    \centering
    \caption{Rendimiento del clasificador por categoría diagnóstica.}
    \label{tab:metricas_por_clase}
    \begin{tabular}{lcccc}
        \toprule
        \textbf{Categoría} & \textbf{Precisión} & \textbf{Sensibilidad} & \textbf{F1-Score} & \textbf{Muestras} \\
        \midrule
        COVID-19 & 99.09\% & 96.46\% & 97.76\% & 452 \\
        Normal & 97.84\% & 99.37\% & 98.60\% & 1,274 \\
        Neumonía Viral & 97.52\% & 92.90\% & 95.15\% & 169 \\
        \bottomrule
    \end{tabular}
\end{table}

\textbf{Normal (F1-Score: 98.60\%):} Es la categoría con mejor rendimiento. Esto es esperado porque representa la mayoría de las muestras (1,274 de 1,895) y los pulmones sanos tienen patrones visuales más consistentes.

\textbf{COVID-19 (F1-Score: 97.76\%):} El clasificador detecta correctamente el 96.46\% de los casos de COVID-19 (sensibilidad) y cuando predice COVID-19, acierta el 99.09\% de las veces (precisión).

\textbf{Neumonía Viral (F1-Score: 95.15\%):} Es la categoría con menor rendimiento, aunque aún superior al 95\%. Esto se debe a que es la categoría menos representada (169 muestras) y presenta mayor variabilidad en su presentación visual.

\subsection{Análisis de Errores}
\label{subsec:matriz_confusion}

La Tabla \ref{tab:matriz_confusion} muestra la matriz de confusión, que detalla cómo se distribuyen las predicciones correctas e incorrectas entre las categorías.

\begin{table}[htbp]
    \centering
    \caption{Matriz de confusión del clasificador. Las filas representan la categoría real y las columnas la categoría predicha por el sistema.}
    \label{tab:matriz_confusion}
    \begin{tabular}{lccc|c}
        \toprule
        & \multicolumn{3}{c}{\textbf{Predicción del Sistema}} & \\
        \textbf{Categoría Real} & \textbf{COVID-19} & \textbf{Normal} & \textbf{Neum. Viral} & \textbf{Total} \\
        \midrule
        COVID-19 & \textbf{436} & 16 & 0 & 452 \\
        Normal & 4 & \textbf{1,266} & 4 & 1,274 \\
        Neumonía Viral & 0 & 12 & \textbf{157} & 169 \\
        \midrule
        \textbf{Total} & 440 & 1,294 & 161 & 1,895 \\
        \bottomrule
    \end{tabular}
\end{table}

Los números en negrita (diagonal) representan las clasificaciones correctas. Los principales patrones de error observados son:

\begin{itemize}
    \item \textbf{COVID-19 confundido con Normal:} 16 casos (3.54\% de los casos COVID-19). Estos son casos donde las manifestaciones de COVID-19 son sutiles.

    \item \textbf{Neumonía Viral confundida con Normal:} 12 casos (7.10\% de los casos de Neumonía Viral). Representa el error más frecuente proporcionalmente.

    \item \textbf{Normal confundido con COVID-19:} 4 casos (0.31\% de los casos Normal). Falsos positivos de COVID-19.
\end{itemize}

Cabe destacar que ningún caso de COVID-19 o Neumonía Viral fue confundido entre sí (la celda COVID-19/Neum.Viral y Neum.Viral/COVID-19 son ambas 0), lo cual es clínicamente relevante ya que estas dos condiciones requieren tratamientos diferentes.

La Figura \ref{fig:matriz_confusion_visual} presenta la matriz de confusión de manera visual, facilitando la identificación de patrones de error mediante un mapa de calor.

\begin{figure}[htbp]
    \centering
    \includegraphics[width=0.85\textwidth]{Figures/F5.7_matriz_confusion_sahs.png}
    \caption{Matriz de confusión del sistema de clasificación presentada visualmente. Los valores en la diagonal representan clasificaciones correctas. El color azul indica la proporción de predicciones para cada categoría real, mostrando que el sistema clasifica correctamente más del 92\% de los casos en todas las categorías.}
    \label{fig:matriz_confusion_visual}
\end{figure}

La Figura \ref{fig:casos_mal_clasificados} presenta ejemplos reales de casos incorrectamente clasificados por el sistema. Del conjunto de prueba de 1895 imágenes normalizadas geométricamente y procesadas con SAHS, se identificaron 36 errores (1.9\%), distribuidos en cuatro tipos: COVID→Normal (16 casos), Neumonía Viral→Normal (12 casos), Normal→COVID (4 casos) y Normal→Neumonía Viral (4 casos). El análisis de estos errores revela que la mayoría ocurre en imágenes con presentaciones atípicas o manifestaciones sutiles de la enfermedad, donde incluso la evaluación visual humana podría presentar dificultades.

\begin{figure}[htbp]
    \centering
    \includegraphics[width=0.95\textwidth]{Figures/F5.9_casos_mal_clasificados.png}
    \caption{Ejemplos de casos mal clasificados del conjunto de prueba usando imágenes normalizadas geométricamente con SAHS. Se muestran 6 casos reales distribuidos entre los cuatro tipos de errores encontrados, con su confianza de predicción. Cada ejemplo indica la clase verdadera y la clase predicha (Verdadero → Predicho). Los errores representan únicamente el 1.9\% del total de casos evaluados (36/1895), siendo los más frecuentes: COVID-19 clasificado como Normal (16 casos, 3.5\% de los casos COVID), Neumonía Viral clasificada como Normal (12 casos, 7.1\% de los casos Viral), y en menor medida confusiones desde Normal hacia las otras clases (8 casos combinados, 0.6\%).}
    \label{fig:casos_mal_clasificados}
\end{figure}

\subsection{Efecto de la Normalización Geométrica}
\label{subsec:efecto_normalizacion}

Para evaluar el efecto de la normalización geométrica en la clasificación, se compararon tres configuraciones de preprocesamiento utilizando SAHS (\textit{Statistical Asymmetrical Histogram Stretching}) como método de mejora de contraste en todas ellas:

\begin{enumerate}
    \item \textbf{Original + SAHS:} Imágenes originales sin modificación geométrica.
    \item \textbf{Normalizado + SAHS:} Imágenes con normalización geométrica (sistema propuesto).
    \item \textbf{Recortado + SAHS:} Imágenes con recorte del 12\% en los bordes para eliminar artefactos hospitalarios, pero sin normalización geométrica.
\end{enumerate}

La Figura \ref{fig:comparacion_preprocesamiento_sahs} muestra ejemplos visuales de las tres configuraciones, todas con SAHS aplicado, para las tres categorías diagnósticas.

\begin{figure}[htbp]
    \centering
    \includegraphics[width=0.95\textwidth]{Figures/F5.11_comparacion_preprocesamiento_sahs.png}
    \caption{Comparación visual del preprocesamiento con SAHS. Filas: categorías diagnósticas (COVID-19, Normal, Neumonía Viral). Columnas: (a) Original + SAHS, (b) Normalizado + SAHS, (c) Recortado 12\% + SAHS.}
    \label{fig:comparacion_preprocesamiento_sahs}
\end{figure}

La Tabla \ref{tab:comparacion_configuraciones} presenta los resultados de esta comparación.

\begin{table}[htbp]
    \centering
    \caption{Comparación de configuraciones de preprocesamiento. Todas utilizan SAHS para mejora de contraste. La diferencia se calcula respecto a las imágenes originales.}
    \label{tab:comparacion_configuraciones}
    \begin{tabular}{lccc}
        \toprule
        \textbf{Configuración} & \textbf{Exactitud} & \textbf{F1-Macro} & \textbf{Diferencia} \\
        \midrule
        Original + SAHS & 98.68\% & 97.75\% & --- \\
        Normalizado + SAHS & 98.10\% & 97.17\% & -0.58\% \\
        Recortado (12\%) + SAHS & \textbf{95.36\%} & \textbf{94.28\%} & \textbf{-3.32\%} \\
        \bottomrule
    \end{tabular}
\end{table}

Los resultados revelan un patrón que permite establecer conclusiones sobre el origen de las características utilizadas por cada configuración:

\subsubsection{Observación}

\begin{itemize}
    \item \textbf{Original + SAHS:} Obtiene la mayor exactitud (98.68\%). Las imágenes sin procesar, tal como provienen del dataset, incluyen toda la información de la radiografía: región pulmonar, áreas periféricas y artefactos hospitalarios (etiquetas, marcadores de lateralidad) típicamente ubicados en las esquinas.

    \item \textbf{Normalizado + SAHS:} Alcanza 98.10\% de exactitud, apenas 0.58 puntos porcentuales menor. La normalización geométrica restringe el campo de visión del clasificador exclusivamente a la región pulmonar, eliminando acceso a información periférica.

    \item \textbf{Recortado (12\%) + SAHS:} Presenta la exactitud más baja (95.36\%), con una caída de 3.32 puntos porcentuales respecto a las imágenes originales. Este recorte mínimo elimina únicamente los bordes donde se ubican etiquetas hospitalarias, sin modificar la región central de la imagen.
\end{itemize}

\subsubsection{Evidencia Clave}

La diferencia de 3.32 puntos porcentuales entre imágenes originales (98.68\%) y recortadas (95.36\%) constituye evidencia directa de que el modelo entrenado con imágenes sin procesar utiliza características de los bordes, específicamente artefactos hospitalarios, para la clasificación. Al eliminar estas regiones mediante un recorte conservador del 12\%, la exactitud cae significativamente porque el modelo pierde acceso a estos ``atajos'' de clasificación.

\subsubsection{Interpretación}

\begin{enumerate}
    \item \textbf{Las imágenes originales aprenden características espurias:} La caída drástica de exactitud al recortar los bordes (3.32 puntos) demuestra que las etiquetas hospitalarias contribuyen significativamente a la clasificación en el modelo entrenado con imágenes sin procesar.

    \item \textbf{La normalización geométrica aprende características genuinas:} El sistema propuesto (98.10\%) mantiene rendimiento alto utilizando únicamente la región pulmonar, sin acceso a artefactos periféricos. Esto confirma que aprende características patológicas reales.

    \item \textbf{El recorte sin normalización es insuficiente:} Simplemente eliminar los bordes (95.36\%) no es una solución efectiva; el modelo entrenado de esta forma carece tanto de los artefactos como de la normalización que permite enfocarse en patrones pulmonares.
\end{enumerate}

\subsubsection{Validación mediante Validación Cruzada}

La estabilidad de estos resultados se confirma mediante validación cruzada estratificada ($k=5$) sobre el conjunto de imágenes normalizadas geométricamente, que arroja una exactitud de 98.60\% $\pm$ 0.26\% (Tabla \ref{tab:resultados_clasificacion_cv}). La baja desviación estándar indica que el rendimiento no depende de una partición particular de los datos, sino que refleja la capacidad real del modelo.

La Figura \ref{fig:comparacion_sahs_visual} presenta visualmente las matrices de confusión de las tres configuraciones evaluadas.

\begin{figure}[htbp]
    \centering
    \includegraphics[width=0.95\textwidth]{Figures/F5.8_comparacion_sahs.png}
    \caption{Comparación de matrices de confusión para las tres configuraciones de preprocesamiento. (a) Original + SAHS: 98.68\% de exactitud. (b) Normalizado + SAHS (sistema propuesto): 98.10\% de exactitud. (c) Recortado + SAHS: 95.36\% de exactitud. La caída de 3.32 puntos al recortar los bordes evidencia que las imágenes originales utilizan artefactos hospitalarios como características espurias.}
    \label{fig:comparacion_sahs_visual}
\end{figure}

\subsection{Resumen}
\label{subsec:resumen_clasificacion}

El sistema de clasificación basado en normalización geométrica alcanza los siguientes resultados:

\begin{itemize}
    \item \textbf{Exactitud de 98.10\%} sobre 1,895 imágenes de prueba.
    \item \textbf{F1-Score superior al 95\%} en las tres categorías diagnósticas.
    \item \textbf{Sensibilidad del 96.46\%} para detección de COVID-19.
    \item \textbf{Validación cruzada:} 98.60\% $\pm$ 0.26\% de exactitud, confirmando estabilidad.
    \item Solo \textbf{36 errores} de clasificación en 1,895 imágenes.
\end{itemize}

El experimento comparativo de tres configuraciones (Original, Normalizado, Recortado) proporciona evidencia directa de que:

\begin{enumerate}
    \item \textbf{Las imágenes originales utilizan características espurias:} La caída de 3.32 puntos porcentuales al recortar los bordes (de 98.68\% a 95.36\%) demuestra que las etiquetas hospitalarias contribuyen significativamente a la clasificación.

    \item \textbf{La normalización geométrica aprende características genuinas:} El sistema propuesto (98.10\%) mantiene alto rendimiento utilizando únicamente la región pulmonar, sin acceso a artefactos de los bordes.

    \item \textbf{La exactitud de 98.68\% en imágenes originales está inflada:} Este valor no refleja la capacidad del modelo para identificar patologías pulmonares, sino su capacidad para explotar correlaciones espurias con artefactos hospitalarios.
\end{enumerate}
