% =============================================================================
% CAPÍTULO 5: RESULTADOS
% Sección 5.5: Discusión de Resultados
% =============================================================================

\section{Discusión de Resultados}
\label{sec:discusion_resultados}

Esta sección interpreta los resultados experimentales presentados en las secciones anteriores, discute sus implicaciones y analiza las limitaciones del estudio.

\subsection{Interpretación de Resultados de Landmarks}
\label{subsec:interpretacion_landmarks}

El error medio de 3.61 píxeles del ensemble de landmarks representa el 1.6\% del tamaño de imagen ($224 \times 224$ píxeles). Este nivel de precisión es adecuado para la aplicación de warping, dado que:

\begin{enumerate}
    \item El error está por debajo del umbral perceptual: un error de 3.61 píxeles en una imagen de 224 píxeles equivale a un desplazamiento del 1.6\% de la dimensión de la imagen.

    \item La mediana de 3.07 píxeles indica que el 50\% de las predicciones tienen error inferior a 3.1 píxeles, aproximadamente 1.4\% del tamaño de imagen.

    \item Los landmarks del eje central (L9, L10, L11) presentan alta precisión (2.44--2.94 px), proporcionando puntos de referencia estables para la alineación.
\end{enumerate}

El error de landmarks es comparable o superior a trabajos relacionados en detección de puntos anatómicos en radiografías de tórax. Por ejemplo, Van Ginneken et al. \cite{ginneken2006active} reportan errores de 5--10 píxeles en imágenes de resolución similar para detección de campos pulmonares, mientras que nuestro sistema alcanza 3.61 píxeles con un método completamente automático.

\subsection{Validación de la Precisión Suficiente para Warping}
\label{subsec:precision_suficiente}

Un error medio de 3.61 píxeles en imágenes de $224 \times 224$ representa el 1.6\% del tamaño de imagen. Este nivel de precisión es adecuado para el propósito de normalización geométrica, donde el objetivo no es localización sub-píxel, sino alineación consistente de estructuras anatómicas. La Tabla \ref{tab:tolerancia_warping} contextualiza este error.

\begin{table}[htbp]
    \centering
    \caption{Interpretación del error de landmarks en el contexto de la aplicación de warping.}
    \label{tab:tolerancia_warping}
    \begin{tabular}{lcc}
        \toprule
        \textbf{Métrica} & \textbf{Valor} & \textbf{Interpretación} \\
        \midrule
        Error medio (px) & 3.61 & 1.6\% del tamaño de imagen (224 px) \\
        Error medio (mm) & $\sim$3--4 & Escala clínica (radiografías PA) \\
        Suficiencia para warping & Sí & Error < 5 px en 75\% de landmarks \\
        \bottomrule
    \end{tabular}
\end{table}

La precisión de 3.61 píxeles (1.6\% del tamaño de imagen) es suficiente para generar transformaciones de warping que preservan la estructura anatómica general sin introducir distorsiones significativas.

% Referencias temporales para esta sección
% [Se agregarán según sea necesario]
