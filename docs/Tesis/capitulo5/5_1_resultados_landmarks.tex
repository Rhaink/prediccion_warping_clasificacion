% =============================================================================
% CAPÍTULO 5: RESULTADOS
% Sección 5.1: Resultados de Predicción de Landmarks
% =============================================================================

\chapter{Resultados}
\label{cap:resultados}

Este capítulo presenta los resultados obtenidos en cada etapa del sistema: la detección de puntos de referencia anatómicos, la normalización geométrica de las imágenes y la clasificación de enfermedades pulmonares.

\section{Detección de Puntos de Referencia}
\label{sec:resultados_landmarks}

El sistema detecta automáticamente 15 puntos de referencia que definen el contorno de los pulmones en las radiografías. Esta sección presenta la precisión alcanzada por el modelo, evaluado sobre 96 imágenes del conjunto de prueba que cuentan con anotaciones de referencia realizadas por expertos.

\subsection{Precisión del Sistema}
\label{subsec:desempeno_ensemble}

La Tabla \ref{tab:resultados_ensemble_landmarks} presenta los resultados de detección de puntos de referencia. Se compara el mejor modelo individual contra el sistema final que combina cuatro modelos.

\begin{table}[htbp]
    \centering
    \caption{Precisión de detección de puntos de referencia. El error se mide en píxeles sobre imágenes de $224 \times 224$.}
    \label{tab:resultados_ensemble_landmarks}
    \begin{tabular}{lccc}
        \toprule
        \textbf{Configuración} & \textbf{Error promedio} & \textbf{Error mediano} & \textbf{Mejora} \\
        \midrule
        Mejor modelo individual & 4.04 px & --- & --- \\
        Sistema combinado (4 modelos) & 3.61 px & 3.07 px & 10.6\% \\
        \bottomrule
    \end{tabular}
\end{table}

El sistema combinado alcanza un error promedio de \textbf{3.61 píxeles}, lo que representa una mejora del 10.6\% respecto al mejor modelo individual. En una imagen de 224 píxeles, este error equivale al 1.6\% del tamaño de la imagen. El error mediano de 3.07 píxeles indica que la mitad de las predicciones tienen un error menor o igual a este valor.

\subsection{Precisión por Punto de Referencia}
\label{subsec:error_por_landmark}

No todos los puntos de referencia se detectan con la misma facilidad. Los puntos del eje central (línea de la columna vertebral) presentan los errores más bajos (2.44--2.94 píxeles) debido a que esta estructura está bien definida en las radiografías. En contraste, las esquinas superiores de los pulmones presentan mayor dificultad (5.35--5.43 píxeles) porque los límites del pulmón son menos nítidos en esa zona. Los puntos del contorno medio tienen errores intermedios (2.88--3.96 píxeles).

La Figura \ref{fig:error_por_landmark_visual} muestra visualmente la distribución de errores sobre la forma estándar, donde se aprecia que los puntos centrales y del contorno medio tienen menor error que las esquinas.

\begin{figure}[htbp]
    \centering
    \includegraphics[width=0.98\textwidth]{Figures/F5.1_error_por_landmark.png}
    \caption{Distribución del error de detección por punto de referencia. (a) Error medio en píxeles para cada punto L1--L15. (b) Visualización sobre la forma estándar pulmonar, donde el color indica el error medio (escala continua). Los puntos del eje central presentan la mayor precisión.}
    \label{fig:error_por_landmark_visual}
\end{figure}

La Figura \ref{fig:ejemplos_prediccion} presenta ejemplos de detección automática de puntos sobre imágenes reales de las tres categorías diagnósticas. Puede observarse cómo el sistema identifica correctamente el contorno pulmonar incluso en presencia de opacidades por COVID-19 o neumonía viral, manteniendo precisión consistente entre categorías (error promedio de 3.22 px en imágenes normales, 3.93 px en COVID-19, y 4.11 px en neumonía viral).

\begin{figure}[htbp]
    \centering
    \includegraphics[width=0.98\textwidth]{Figures/F5.2_ejemplos_prediccion.png}
    \caption{Ejemplos de predicción automática de puntos de referencia sobre imágenes reales. Cada fila muestra dos ejemplos de una categoría diagnóstica (Normal, COVID-19, Neumonía Viral). Los puntos indican las predicciones del sistema. El modelo mantiene precisión consistente independientemente del tipo de patología presente.}
    \label{fig:ejemplos_prediccion}
\end{figure}

\subsection{Resumen}
\label{subsec:resumen_landmarks}

El sistema de detección de puntos de referencia alcanza una precisión de 3.61 píxeles en promedio, equivalente al 1.6\% del tamaño de la imagen. Esta precisión es suficiente para el proceso de normalización geométrica, ya que permite alinear correctamente la región pulmonar sin introducir distorsiones significativas.
