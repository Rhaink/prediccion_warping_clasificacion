% =============================================================================
% CAPÍTULO 5: RESULTADOS
% Sección 5.4: Análisis Comparativo y Validación de la Hipótesis
% =============================================================================

\section{Análisis Comparativo y Validación de la Hipótesis}
\label{sec:analisis_comparativo}

Esta sección examina la evidencia que sustenta la hipótesis principal del trabajo: que la normalización geométrica mediante deformación mejora el rendimiento de clasificación de enfermedades pulmonares. Se analizan los mecanismos por los cuales la normalización contribuye al rendimiento observado y se compara con resultados reportados en la literatura.

\subsection{Efectividad del Sistema Completo}
\label{subsec:efectividad_sistema}

El sistema propuesto, que integra predicción de puntos de referencia, normalización geométrica y clasificación, alcanza los resultados presentados en la Tabla \ref{tab:resumen_sistema_completo}.

\begin{table}[htbp]
    \centering
    \caption{Resumen del rendimiento del sistema completo basado en normalización geométrica con preprocesamiento SAHS.}
    \label{tab:resumen_sistema_completo}
    \begin{tabular}{lcc}
        \toprule
        \textbf{Componente} & \textbf{Métrica} & \textbf{Valor} \\
        \midrule
        \multirow{2}{*}{Predicción de puntos de referencia}
            & Error medio (ensamble + TTA) & 3.61 px \\
            & Mejora vs. mejor individual & 10.6\% \\
        \midrule
        \multirow{2}{*}{Normalización geométrica}
            & Parámetro óptimo (\texttt{margin\_scale}) & 1.05 \\
            & Fill rate medio & 47\% \\
        \midrule
        \multirow{3}{*}{Clasificación}
            & Exactitud & 98.10\% \\
            & F1-Macro & 97.17\% \\
            & F1-Ponderado & 98.09\% \\
        \bottomrule
    \end{tabular}
\end{table}

La exactitud de 98.10\% y F1-Macro de 97.17\% demuestran que el sistema completo es efectivo para la clasificación de tres clases de patologías pulmonares, logrando una tasa de error de solo 1.90\% sobre el conjunto de prueba.

\subsection{Mecanismos de Mejora por Normalización Geométrica}
\label{subsec:mecanismos_mejora}

El proceso de normalización geométrica contribuye al rendimiento del clasificador mediante tres mecanismos principales:

\subsubsection{1. Eliminación de Variabilidad No Patológica}

La normalización geométrica elimina variaciones de posición, escala y orientación del paciente que no están relacionadas con la patología. La Tabla \ref{tab:variabilidad_eliminada} cuantifica las transformaciones aplicadas.

\begin{table}[htbp]
    \centering
    \caption{Tipos de variabilidad geométrica eliminados por el proceso de normalización.}
    \label{tab:variabilidad_eliminada}
    \begin{tabular}{lp{8cm}}
        \toprule
        \textbf{Tipo de variabilidad} & \textbf{Mecanismo de normalización} \\
        \midrule
        Traslación & Alineación de puntos de referencia predichos con forma estándar fija \\
        Escala & Normalización del tamaño de la silueta pulmonar a dimensiones consistentes \\
        Rotación & Corrección de inclinaciones del eje corporal del paciente \\
        Deformación local & Transformaciones afines por triángulos que corrigen variaciones anatómicas inter-paciente \\
        \bottomrule
    \end{tabular}
\end{table}

Al eliminar estas variaciones, el clasificador puede enfocarse en aprender patrones patológicos invariantes a la pose del paciente.

\subsubsection{2. Selección Implícita de Características}

La tasa de cobertura de 47\% indica que aproximadamente el 53\% de la imagen original (regiones periféricas, fondo, artefactos) se elimina durante la normalización. Este proceso actúa como un mecanismo de \textit{selección implícita de características a nivel de imagen}, análogo a mecanismos de atención \cite{jaderberg2015spatial}.

\begin{table}[htbp]
    \centering
    \caption{Efecto de selección de características por normalización geométrica.}
    \label{tab:feature_selection_effect}
    \begin{tabular}{lcc}
        \toprule
        \textbf{Región de imagen} & \textbf{Porcentaje} & \textbf{Información} \\
        \midrule
        Región pulmonar (normalizada) & 47\% & Patrones patológicos relevantes \\
        Región periférica (eliminada) & 53\% & Fondo, artefactos, marcas \\
        \bottomrule
    \end{tabular}
\end{table}

La eliminación de información no discriminante reduce el espacio de características que el clasificador debe aprender, potencialmente mejorando la capacidad de generalización.

\subsubsection{3. Regularización Implícita}

La transformación geométrica introduce una forma de regularización al forzar al modelo a operar sobre una representación estandarizada. Esta restricción puede actuar como un prior geométrico que guía el aprendizaje hacia soluciones más robustas.

\subsection{Evidencia de Características Espurias en Imágenes Originales}
\label{subsec:evidencia_espurias}

El experimento de recorte proporciona evidencia directa de que las imágenes originales (sin ningún procesamiento, tal como provienen del dataset COVID-19 Radiography Database) permiten al modelo aprender características no relacionadas con patología pulmonar.

\subsubsection{Diseño del Experimento}

Se entrenó el mismo clasificador ResNet-18 con tres variantes de las imágenes, todas con preprocesamiento SAHS:

\begin{enumerate}
    \item \textbf{Original:} Imágenes sin modificación geométrica, incluyendo toda la información de la radiografía.
    \item \textbf{Normalizado:} Imágenes alineadas geométricamente mediante el proceso de detección de puntos de referencia, restringidas a la región pulmonar.
    \item \textbf{Recortado (12\%):} Imágenes originales con un recorte conservador de los bordes para eliminar únicamente las etiquetas hospitalarias típicamente ubicadas en las esquinas.
\end{enumerate}

\subsubsection{Resultados y Análisis}

\begin{table}[htbp]
    \centering
    \caption{Comparación de exactitud según configuración de preprocesamiento. El recorte del 12\% elimina únicamente los bordes con etiquetas hospitalarias.}
    \label{tab:evidencia_espurias}
    \begin{tabular}{lcc}
        \toprule
        \textbf{Configuración} & \textbf{Exactitud} & \textbf{Interpretación} \\
        \midrule
        Original + SAHS & 98.68\% & Inflada por características espurias \\
        Normalizado + SAHS & 98.10\% & Basada en características pulmonares \\
        Recortado (12\%) + SAHS & 95.36\% & Exactitud ``real'' sin artefactos \\
        \bottomrule
    \end{tabular}
\end{table}

La caída de 3.32 puntos porcentuales entre Original (98.68\%) y Recortado (95.36\%) al eliminar \textbf{únicamente} las etiquetas de las esquinas demuestra que:

\begin{enumerate}
    \item \textbf{El modelo con imágenes originales utiliza las etiquetas como atajos:} La pérdida significativa de exactitud al recortar los bordes indica que el modelo dependía de estos artefactos para la clasificación.

    \item \textbf{El 95.36\% representa la exactitud ``real'':} Este valor refleja la capacidad del modelo para clasificar basándose en información de la imagen central, sin acceso a artefactos periféricos.

    \item \textbf{La normalización geométrica recupera el rendimiento:} El sistema propuesto (98.10\%) supera al recortado (95.36\%) en 2.74 puntos, demostrando que la alineación y enfoque en la región pulmonar permite aprender características patológicas genuinas que compensan la ausencia de artefactos.
\end{enumerate}

\subsubsection{Implicaciones}

Este hallazgo tiene implicaciones importantes para la evaluación de sistemas de clasificación de radiografías:

\begin{itemize}
    \item \textbf{La exactitud más alta no implica mejor modelo:} Un modelo con mayor exactitud en imágenes originales puede estar explotando correlaciones espurias que no generalizarán a datos de otros hospitales con diferentes sistemas de etiquetado.

    \item \textbf{Necesidad de validación rigurosa:} La comparación entre configuraciones (Original, Normalizado, Recortado) debería ser práctica estándar para evaluar si un clasificador aprende características genuinas.

    \item \textbf{Ventaja de la normalización geométrica:} Además de estandarizar la pose anatómica, la normalización actúa como filtro que elimina acceso a información no diagnóstica.
\end{itemize}

\subsection{Comparación con Trabajos Relacionados}
\label{subsec:comparacion_literatura}

La Tabla \ref{tab:comparacion_literatura} compara el rendimiento del sistema propuesto con trabajos relacionados en la literatura sobre clasificación de COVID-19 en radiografías de tórax.

\begin{table}[htbp]
    \centering
    \caption{Comparación del sistema propuesto con trabajos relacionados en clasificación de COVID-19 en radiografías de tórax. Los trabajos varían en número de clases, tamaño de dataset y arquitecturas utilizadas.}
    \label{tab:comparacion_literatura}
    \begin{tabular}{lcccc}
        \toprule
        \textbf{Trabajo} & \textbf{Clases} & \textbf{Dataset} & \textbf{Arquitectura} & \textbf{Exactitud} \\
        \midrule
        \textbf{Sistema propuesto} & 3 & 15,153 & ResNet-18 + Normalización & \textbf{98.10\%} \\
        \midrule
        Chowdhury et al. \cite{chowdhury2020can} & 4 & 423 & VGG19 & 96.58\% \\
        Rahman et al. \cite{rahman2021exploring} & 3 & 3,616 & VGG16 & 93.94\% \\
        Ozturk et al. \cite{ozturk2020automated} & 2 & 1,127 & DarkCovidNet & 98.08\% \\
        Apostolopoulos et al. \cite{apostolopoulos2020covid} & 3 & 1,428 & VGG19 & 93.48\% \\
        Narin et al. \cite{narin2021automatic} & 2 & 341 & ResNet-50 & 98.00\% \\
        \bottomrule
    \end{tabular}

    \vspace{0.5em}
    \footnotesize\textit{Nota: La comparación directa es limitada debido a diferencias en datasets, número de clases y protocolos de evaluación. Los trabajos citados son representativos del estado del arte en clasificación de COVID-19 en radiografías.}
\end{table}

El sistema propuesto alcanza rendimiento competitivo (98.10\%) con respecto a los trabajos reportados, destacando que:
\begin{itemize}
    \item Utiliza un dataset significativamente más grande (15,153 imágenes) que la mayoría de trabajos relacionados.
    \item Clasifica 3 clases (más difícil que clasificación binaria COVID/Normal).
    \item Emplea una arquitectura más ligera (ResNet-18, 11.2M parámetros) comparada con VGG16/19 o ResNet-50.
\end{itemize}

La contribución diferenciadora del presente trabajo reside en el \textbf{enfoque de normalización geométrica basado en puntos de referencia}, no ampliamente explorado en la literatura de clasificación de COVID-19.

\subsection{Limitaciones del Análisis Comparativo}
\label{subsec:limitaciones_comparativo}

Es importante reconocer las limitaciones del análisis comparativo presentado:

\begin{enumerate}
    \item \textbf{Conjunto de datos único:} La evaluación se realizó exclusivamente sobre el COVID-19 Radiography Database. La generalización a otros conjuntos de datos requiere validación adicional.

    \item \textbf{Variabilidad entre trabajos de la literatura:} La comparación con trabajos relacionados está limitada por diferencias en conjuntos de datos, protocolos de evaluación y definiciones de clases.
\end{enumerate}

A pesar de estas limitaciones, la \textbf{efectividad demostrada del sistema completo} (98.10\% de exactitud, 97.17\% de F1-Macro) valida que el enfoque de normalización geométrica es viable y produce resultados competitivos. La hipótesis de que la normalización geométrica mejora la clasificación se sustenta en:

\begin{itemize}
    \item Los mecanismos teóricos de eliminación de variabilidad no patológica y selección de características.
    \item El rendimiento competitivo alcanzado comparado con la literatura.
    \item La evidencia exploratoria de configuraciones alternativas con alto rendimiento.
\end{itemize}

\subsection{Interpretabilidad: Enfoque en Región Pulmonar}
\label{subsec:interpretabilidad}

Aunque el presente trabajo no incluye análisis de mapas de atención (Grad-CAM) o similar, el mecanismo de normalización geométrica proporciona una forma de interpretabilidad \textit{a priori}: al restringir la entrada del clasificador a la región pulmonar normalizada, se garantiza que las características aprendidas provienen exclusivamente de esta zona diagnóstica.

\subsection{Resumen del Análisis Comparativo}
\label{subsec:resumen_comparativo}

El análisis comparativo presenta la siguiente evidencia respecto a la hipótesis de que la normalización geométrica mejora la clasificación:

\begin{itemize}
    \item \textbf{Efectividad demostrada:} El sistema completo basado en normalización geométrica alcanza 98.10\% de exactitud y 97.17\% de F1-Macro.

    \item \textbf{Mecanismos fundamentados:} La normalización contribuye mediante eliminación de variabilidad no patológica, selección implícita de características (tasa de cobertura 47\%), y regularización geométrica.

    \item \textbf{Rendimiento competitivo:} Los resultados son comparables o superiores a trabajos relacionados en la literatura de clasificación de COVID-19.

    \item \textbf{Limitación reconocida:} La evaluación en un único conjunto de datos limita la generalización de los resultados a otros contextos hospitalarios.
\end{itemize}

La contribución principal del trabajo reside en la \textbf{demostración de viabilidad y efectividad} de un sistema completo basado en normalización geométrica automática mediante detección de puntos de referencia, más que en la cuantificación absoluta de mejora respecto a métodos sin normalización.
