% =============================================================================
% CAPÍTULO 5: RESULTADOS
% Sección 5.4: Análisis Comparativo y Validación de la Hipótesis
% =============================================================================

\section{Análisis Comparativo y Validación de la Hipótesis}
\label{sec:analisis_comparativo}

Esta sección examina la evidencia que sustenta la hipótesis principal del trabajo: que la normalización geométrica mediante warping mejora el rendimiento de clasificación de enfermedades pulmonares. Se analizan los mecanismos por los cuales la normalización contribuye al rendimiento observado y se compara con resultados reportados en la literatura.

\subsection{Efectividad del Sistema Completo}
\label{subsec:efectividad_sistema}

El sistema propuesto, que integra predicción de landmarks, normalización geométrica y clasificación, alcanza los resultados presentados en la Tabla \ref{tab:resumen_sistema_completo}.

\begin{table}[htbp]
    \centering
    \caption{Resumen del rendimiento del sistema completo basado en normalización geométrica.}
    \label{tab:resumen_sistema_completo}
    \begin{tabular}{lcc}
        \toprule
        \textbf{Componente} & \textbf{Métrica} & \textbf{Valor} \\
        \midrule
        \multirow{2}{*}{Predicción de landmarks}
            & Error medio (ensemble + TTA) & 3.61 px \\
            & Mejora vs. mejor individual & 10.6\% \\
        \midrule
        \multirow{2}{*}{Normalización geométrica}
            & Parámetro óptimo (\texttt{margin\_scale}) & 1.05 \\
            & Fill rate medio & 47\% \\
        \midrule
        \multirow{3}{*}{Clasificación}
            & Accuracy & 98.05\% \\
            & F1-Macro & 97.12\% \\
            & F1-Weighted & 98.04\% \\
        \bottomrule
    \end{tabular}
\end{table}

La accuracy de 98.05\% y F1-Macro de 97.12\% demuestran que el sistema completo es efectivo para la clasificación de tres clases de patologías pulmonares, logrando una tasa de error de solo 1.95\% sobre el conjunto de prueba.

\subsection{Mecanismos de Mejora por Normalización Geométrica}
\label{subsec:mecanismos_mejora}

El proceso de normalización geométrica contribuye al rendimiento del clasificador mediante tres mecanismos principales:

\subsubsection{1. Eliminación de Variabilidad No Patológica}

La normalización geométrica elimina variaciones de posición, escala y orientación del paciente que no están relacionadas con la patología. La Tabla \ref{tab:variabilidad_eliminada} cuantifica las transformaciones aplicadas.

\begin{table}[htbp]
    \centering
    \caption{Tipos de variabilidad geométrica eliminados por el proceso de normalización.}
    \label{tab:variabilidad_eliminada}
    \begin{tabular}{lp{8cm}}
        \toprule
        \textbf{Tipo de variabilidad} & \textbf{Mecanismo de normalización} \\
        \midrule
        Traslación & Alineación de landmarks predichos con forma canónica fija \\
        Escala & Normalización del tamaño de la silueta pulmonar a dimensiones consistentes \\
        Rotación & Corrección de inclinaciones del eje corporal del paciente \\
        Deformación local & Transformaciones afines por triángulos que corrigen variaciones anatómicas inter-paciente \\
        \bottomrule
    \end{tabular}
\end{table}

Al eliminar estas variaciones, el clasificador puede enfocarse en aprender patrones patológicos invariantes a la pose del paciente.

\subsubsection{2. Selección Implícita de Características (Feature Selection)}

El fill rate de 47\% indica que aproximadamente el 53\% de la imagen original (regiones periféricas, fondo, artefactos) se elimina durante el warping. Este proceso actúa como un mecanismo de \textit{selección implícita de características a nivel de imagen}, análogo a mecanismos de atención \cite{jaderberg2015spatial}.

\begin{table}[htbp]
    \centering
    \caption{Efecto de selección de características por normalización geométrica.}
    \label{tab:feature_selection_effect}
    \begin{tabular}{lcc}
        \toprule
        \textbf{Región de imagen} & \textbf{Porcentaje} & \textbf{Información} \\
        \midrule
        Región pulmonar (warped) & 47\% & Patrones patológicos relevantes \\
        Región periférica (eliminada) & 53\% & Fondo, artefactos, marcas \\
        \bottomrule
    \end{tabular}
\end{table}

La eliminación de información no discriminante reduce el espacio de características que el clasificador debe aprender, potencialmente mejorando la capacidad de generalización.

\subsubsection{3. Regularización Implícita}

La transformación geométrica introduce una forma de regularización al forzar al modelo a operar sobre una representación canónica estandarizada. Esta restricción puede actuar como un prior geométrico que guía el aprendizaje hacia soluciones más robustas.

\subsection{Comparación con Trabajos Relacionados}
\label{subsec:comparacion_literatura}

La Tabla \ref{tab:comparacion_literatura} compara el rendimiento del sistema propuesto con trabajos relacionados en la literatura sobre clasificación de COVID-19 en radiografías de tórax.

\begin{table}[htbp]
    \centering
    \caption{Comparación del sistema propuesto con trabajos relacionados en clasificación de COVID-19 en radiografías de tórax. Los trabajos varían en número de clases, tamaño de dataset y arquitecturas utilizadas.}
    \label{tab:comparacion_literatura}
    \begin{tabular}{lcccc}
        \toprule
        \textbf{Trabajo} & \textbf{Clases} & \textbf{Dataset} & \textbf{Arquitectura} & \textbf{Accuracy} \\
        \midrule
        \textbf{Sistema propuesto} & 3 & 15,153 & ResNet-18 + Warping & \textbf{98.05\%} \\
        \midrule
        Chowdhury et al. \cite{chowdhury2020can} & 4 & 423 & VGG19 & 96.58\% \\
        Rahman et al. \cite{rahman2021exploring} & 3 & 3,616 & VGG16 & 93.94\% \\
        Ozturk et al. \cite{ozturk2020automated} & 2 & 1,127 & DarkCovidNet & 98.08\% \\
        Apostolopoulos et al. \cite{apostolopoulos2020covid} & 3 & 1,428 & VGG19 & 93.48\% \\
        Narin et al. \cite{narin2021automatic} & 2 & 341 & ResNet-50 & 98.00\% \\
        \bottomrule
    \end{tabular}

    \vspace{0.5em}
    \footnotesize\textit{Nota: La comparación directa es limitada debido a diferencias en datasets, número de clases y protocolos de evaluación. Los trabajos citados son representativos del estado del arte en clasificación de COVID-19 en radiografías.}
\end{table}

El sistema propuesto alcanza rendimiento competitivo (98.05\%) con respecto a los trabajos reportados, destacando que:
\begin{itemize}
    \item Utiliza un dataset significativamente más grande (15,153 imágenes) que la mayoría de trabajos relacionados.
    \item Clasifica 3 clases (más difícil que clasificación binaria COVID/Normal).
    \item Emplea una arquitectura más ligera (ResNet-18, 11.2M parámetros) comparada con VGG16/19 o ResNet-50.
\end{itemize}

La contribución diferenciadora del presente trabajo reside en el \textbf{enfoque de normalización geométrica basado en landmarks}, no ampliamente explorado en la literatura de clasificación de COVID-19.

\subsection{Evidencia de Experimentos Exploratorios}
\label{subsec:evidencia_exploratoria}

Durante el desarrollo del sistema, se realizaron experimentos exploratorios con configuraciones alternativas de warping. Aunque estos resultados están marcados como obsoletos en favor del método actual (\texttt{warped\_lung\_best}), proporcionan evidencia adicional sobre el potencial de la normalización geométrica.

\begin{table}[htbp]
    \centering
    \caption{Resultados de configuraciones exploratorias de warping. Estos experimentos validaron diferentes parámetros de normalización antes de converger al método actual.}
    \label{tab:experimentos_exploratorios}
    \begin{tabular}{lccc}
        \toprule
        \textbf{Configuración} & \textbf{Fill Rate} & \textbf{Accuracy} & \textbf{Estado} \\
        \midrule
        \texttt{warped\_lung\_best} (actual) & 47\% & 98.05\% & Validado \\
        \texttt{warped\_96} (exploratorio) & 96\% & 99.10\% & Obsoleto \\
        \texttt{warped\_99} (exploratorio) & 99\% & 98.73\% & Obsoleto \\
        \texttt{warped\_47} (exploratorio) & 47\% & 98.02\% & Obsoleto \\
        \bottomrule
    \end{tabular}

    \vspace{0.5em}
    \footnotesize\textit{Nota: Las configuraciones exploratorias utilizaron diferentes estrategias de preprocesamiento (RGB+CLAHE en LAB space vs Grayscale+CLAHE) y parámetros de cobertura completa, resultando en diferentes fill rates. Estos experimentos no se consideran comparaciones directas válidas debido a diferencias metodológicas, pero ilustran la exploración del espacio de diseño.}
\end{table}

La configuración \texttt{warped\_96} alcanzó 99.10\% de accuracy, sugiriendo que con parámetros óptimos de cobertura completa, el método de normalización geométrica puede alcanzar rendimiento muy alto. Sin embargo, esta configuración fue descartada en favor del método actual (\texttt{warped\_lung\_best}) por razones de consistencia metodológica y reproducibilidad.

\subsection{Limitaciones del Análisis Comparativo}
\label{subsec:limitaciones_comparativo}

Es importante reconocer las limitaciones del análisis comparativo presentado:

\begin{enumerate}
    \item \textbf{Ausencia de comparación controlada:} El presente trabajo no incluye una evaluación sistemática del mismo clasificador ResNet-18 entrenado sobre imágenes originales (sin warping) versus imágenes warped en el conjunto \texttt{warped\_lung\_best}. Esta comparación directa sería necesaria para cuantificar de manera rigurosa la contribución específica del warping al rendimiento de clasificación.

    \item \textbf{Experimentos exploratorios no comparables:} Los resultados de configuraciones alternativas (\texttt{warped\_96}, \texttt{warped\_99}, etc.) no constituyen comparaciones válidas con el método actual debido a diferencias en preprocesamiento y parámetros.

    \item \textbf{Variabilidad entre trabajos de la literatura:} La comparación con trabajos relacionados está limitada por diferencias en datasets, protocolos de evaluación y definiciones de clases.
\end{enumerate}

A pesar de estas limitaciones, la \textbf{efectividad demostrada del sistema completo} (98.05\% accuracy, 97.12\% F1-Macro) valida que el enfoque de normalización geométrica es viable y produce resultados competitivos. La hipótesis de que la normalización geométrica mejora la clasificación se sustenta en:

\begin{itemize}
    \item Los mecanismos teóricos de eliminación de variabilidad no patológica y selección de características.
    \item El rendimiento competitivo alcanzado comparado con la literatura.
    \item La evidencia exploratoria de configuraciones alternativas con alto rendimiento.
\end{itemize}

\subsection{Interpretabilidad: Enfoque en Región Pulmonar}
\label{subsec:interpretabilidad}

Aunque el presente trabajo no incluye análisis de mapas de atención (Grad-CAM) o similar, el mecanismo de normalización geométrica proporciona una forma de interpretabilidad \textit{a priori}. La Figura \ref{fig:enfoque_region_pulmonar} ilustra cómo el warping enfoca el clasificador en la región pulmonar.

\begin{figure}[htbp]
    \centering
    % [PENDIENTE: F5.10 - Enfoque en región pulmonar]
    \fbox{\parbox{0.95\textwidth}{\centering\vspace{5cm}
    [Figura F5.10: Comparación de regiones informativas]\\
    Dos paneles lado a lado:\\
    (a) Imagen original con overlay indicando región eliminada (gris) vs región preservada (color)\\
    (b) Imagen warped mostrando únicamente la región pulmonar normalizada\\
    Para 3 ejemplos (Normal, COVID-19, Viral Pneumonia)
    \vspace{5cm}}}
    \caption{Efecto de selección de región por normalización geométrica. (a) Las imágenes originales contienen regiones periféricas sin información diagnóstica (gris). (b) El proceso de warping elimina estas regiones, enfocando el clasificador en la silueta pulmonar normalizada. Este mecanismo actúa como una forma de atención geométrica explícita.}
    \label{fig:enfoque_region_pulmonar}
\end{figure}

\subsection{Resumen del Análisis Comparativo}
\label{subsec:resumen_comparativo}

El análisis comparativo presenta la siguiente evidencia respecto a la hipótesis de que la normalización geométrica mejora la clasificación:

\begin{itemize}
    \item \textbf{Efectividad demostrada:} El sistema completo basado en normalización geométrica alcanza 98.05\% de accuracy y 97.12\% de F1-Macro.

    \item \textbf{Mecanismos fundamentados:} La normalización contribuye mediante eliminación de variabilidad no patológica, selección implícita de características (fill rate 47\%), y regularización geométrica.

    \item \textbf{Rendimiento competitivo:} Los resultados son comparables o superiores a trabajos relacionados en la literatura de clasificación de COVID-19.

    \item \textbf{Limitación reconocida:} La ausencia de una comparación controlada original vs. warped en el conjunto actual limita la cuantificación precisa de la mejora atribuible específicamente al warping.
\end{itemize}

La contribución principal del trabajo reside en la \textbf{demostración de viabilidad y efectividad} de un sistema completo basado en normalización geométrica automática mediante detección de landmarks, más que en la cuantificación absoluta de mejora respecto a métodos sin normalización.

% Referencias temporales para esta sección
% \cite{jaderberg2015spatial} - Jaderberg et al., Spatial Transformer Networks (NIPS 2015)
% \cite{chowdhury2020can} - Chowdhury et al., Can AI help in screening Viral and COVID-19 pneumonia? (2020)
% \cite{rahman2021exploring} - Rahman et al., Exploring the effect of image enhancement (2021)
% \cite{ozturk2020automated} - Ozturk et al., Automated detection of COVID-19 (2020)
% \cite{apostolopoulos2020covid} - Apostolopoulos & Mpesiana, Covid-19: automatic detection (2020)
% \cite{narin2021automatic} - Narin et al., Automatic detection (2021)
